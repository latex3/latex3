% \iffalse meta-comment
%
%% File: xparse.dtx (C) Copyright 1999 Frank Mittelbach, Chris Rowley,
%%                      David Carlisle
%%                  (C) Copyright 2004-2008 Frank Mittelbach,
%%                      The LaTeX3 Project
%%                  (C) Copyright 2009-2017 The LaTeX3 Project
%
% It may be distributed and/or modified under the conditions of the
% LaTeX Project Public License (LPPL), either version 1.3c of this
% license or (at your option) any later version.  The latest version
% of this license is in the file
%
%    http://www.latex-project.org/lppl.txt
%
% This file is part of the "l3packages bundle" (The Work in LPPL)
% and all files in that bundle must be distributed together.
%
% -----------------------------------------------------------------------
%
% The development version of the bundle can be found at
%
%    https://github.com/latex3/latex3
%
% for those people who are interested.
%
%<*driver|package>
% The version of expl3 required is tested as early as possible, as
% some really old versions do not define \ProvidesExplPackage.
\RequirePackage{expl3}[2017/05/29]
%<package>\@ifpackagelater{expl3}{2017/05/29}
%<package>  {}
%<package>  {%
%<package>    \PackageError{xparse}{Support package l3kernel too old}
%<package>      {%
%<package>        Please install an up to date version of l3kernel\MessageBreak
%<package>        using your TeX package manager or from CTAN.\MessageBreak
%<package>        \MessageBreak
%<package>        Loading xparse will abort!%
%<package>      }%
%<package>    \endinput
%<package>  }
%</driver|package>
%<*driver>
\documentclass[full]{l3doc}
\usepackage{amstext}
\begin{document}
  \DocInput{\jobname.dtx}
\end{document}
%</driver>
% \fi
%
% \providecommand\acro[1]{\textsc{\MakeLowercase{#1}}}
% \newenvironment{arg-description}{%
%   \begin{itemize}\def\makelabel##1{\hss\llap{\bfseries##1}}}{\end{itemize}}
%
% \title{^^A
%   The \textsf{xparse} package\\ Document command parser^^A
% }
%
% \author{^^A
%  The \LaTeX3 Project\thanks
%    {^^A
%      E-mail:
%        \href{mailto:latex-team@latex-project.org}
%          {latex-team@latex-project.org}^^A
%    }^^A
% }
%
% \date{Released 2017/05/29}
%
% \maketitle
%
% \begin{documentation}
%
% The \pkg{xparse} package provides a high-level interface for
% producing document-level commands. In that way, it is intended as
% a replacement for the \LaTeXe{} \cs{newcommand} macro. However,
% \pkg{xparse} works so that the interface to a function (optional
% arguments, stars and mandatory arguments, for example) is separate
% from the internal implementation. \pkg{xparse} provides a normalised
% input for the internal form of a function, independent of the
% document-level argument arrangement.
%
% At present, the functions in \pkg{xparse} which are regarded as
% \enquote{stable} are:
% \begin{itemize}
%   \item \cs{NewDocumentCommand}
%   \item \cs{RenewDocumentCommand}
%   \item \cs{ProvideDocumentCommand}
%   \item \cs{DeclareDocumentCommand}
%   \item \cs{NewDocumentEnvironment}
%   \item \cs{RenewDocumentEnvironment}
%   \item \cs{ProvideDocumentEnvironment}
%   \item \cs{DeclareDocumentEnvironment}
%   \item \cs{NewExpandableDocumentCommand}
%   \item \cs{RenewExpandableDocumentCommand}
%   \item \cs{ProvideExpandableDocumentCommand}
%   \item \cs{DeclareExpandableDocumentCommand}
%   \item \cs{IfNoValue(TF)}
%   \item \cs{IfValue(TF)}
%   \item \cs{IfBoolean(TF)}
% \end{itemize}
% with the other functions currently regarded as \enquote{experimental}. Please
% try all of the commands provided here, but be aware that the
% experimental ones may change or disappear.
%
% \subsection{Specifying arguments}
%
% Before introducing the functions used to create document commands,
% the method for specifying arguments with \pkg{xparse} will be
% illustrated. In order to allow each argument to be defined
% independently, \pkg{xparse} does not simply need to know the
% number of arguments for a function, but also the nature of each
% one. This is done by constructing an \emph{argument specification},
% which defines the number of arguments, the type of each argument
% and any additional information needed for \pkg{xparse} to read the
% user input and properly pass it through to internal functions.
%
% The basic form of the argument specifier is a list of letters, where
% each letter defines a type of argument. As will be described below,
% some of the types need additional information, such as default values.
% The argument types can be divided into two, those which define
% arguments that are mandatory (potentially raising an error if not
% found) and those which define optional arguments. The mandatory types
% are:
% \begin{itemize}[font=\ttfamily]
%   \item[m] A standard mandatory argument, which can either be a single
%     token alone or multiple tokens surrounded by curly braces.
%     Regardless of the input, the argument will be passed to the
%     internal code surrounded by a brace pair. This is the \pkg{xparse}
%     type specifier for a normal \TeX{} argument.
%   \item[l] An argument which reads everything up to the first
%     open group token: in standard \LaTeX{} this is a left brace.
%   \item[r] Reads a \enquote{required} delimited argument, where the
%     delimiters are given as \meta{token1} and \meta{token2}:
%     \texttt{r}\meta{token1}\meta{token2}. If the opening \meta{token}
%     is missing, the default marker |-NoValue-| will be inserted after
%     a suitable error.
%   \item[R] As for \texttt{r}, this is a \enquote{required} delimited
%     argument but has a user-definable recovery \meta{default}, given
%     as \texttt{R}\meta{token1}\meta{token2}\marg{default}.
%   \item[u] Reads an argument \enquote{until} \meta{tokens} are encountered,
%     where the desired \meta{tokens} are given as an argument to the
%     specifier: \texttt{u}\marg{tokens}.
%   \item[v] Reads an argument \enquote{verbatim}, between the following
%     character and its next occurrence, in a way similar to the argument
%     of the \LaTeXe{} command \cs{verb}. Thus a \texttt{v}-type argument
%     is read between two matching tokens, which cannot be any of |%|, |\|,
%     |#|, |{|, |}| or \verb*| |.
%     The verbatim argument can also be enclosed between braces, |{| and |}|.
%     A command with a verbatim
%     argument will not work when it appears within an argument of
%     another function.
% \end{itemize}
% The types which define optional arguments are:
% \begin{itemize}[font=\ttfamily]
%   \item[o] A standard \LaTeX{} optional argument, surrounded with square
%     brackets, which will supply
%     the special |-NoValue-| marker if not given (as described later).
%   \item[d] An optional argument which is delimited by \meta{token1}
%     and \meta{token2}, which are given as arguments:
%     \texttt{d}\meta{token1}\meta{token2}. As with \texttt{o}, if no
%     value is given the special marker |-NoValue-| is returned.
%   \item[O] As for \texttt{o}, but returns \meta{default} if no
%     value is given.  Should be given as \texttt{O}\marg{default}.
%   \item[D] As for \texttt{d}, but returns \meta{default} if no
%     value is given: \texttt{D}\meta{token1}\meta{token2}\marg{default}.
%     Internally, the \texttt{o}, \texttt{d} and \texttt{O} types are
%     short-cuts to an appropriated-constructed \texttt{D} type argument.
%   \item[s] An optional star, which will result in a value
%     \cs{BooleanTrue} if a star is present and \cs{BooleanFalse}
%     otherwise (as described later).
%   \item[t] An optional \meta{token}, which will result in a value
%     \cs{BooleanTrue} if \meta{token} is present and \cs{BooleanFalse}
%     otherwise. Given as \texttt{t}\meta{token}.
%   \item[g] An optional argument given inside a pair of \TeX{} group
%     tokens (in standard \LaTeX, |{| \ldots |}|), which returns
%     |-NoValue-| if not present.
%   \item[G] As for \texttt{g} but returns \meta{default} if no value
%     is given: \texttt{G}\marg{default}.
%   \item[e] A set of optional \emph{embellishments}, each of which
%     requires a \emph{value}: \texttt{e}\marg{tokens}.  If an
%     embellishment is not present, |-NoValue-| is returned.  Each
%     embellishment gives one argument, ordered as for the list of
%     \meta{tokens} in the argument specification.  All \meta{tokens}
%     must be distinct.  \emph{This is an experimental type}.
%   \item[E] As for \texttt{e} but returns one or more \meta{defaults}
%     if values are not given: \texttt{E}\marg{tokens}\marg{defaults}. See
%     Section~\ref{sec:embellishment} for more details.
% \end{itemize}
%
% Using these specifiers, it is possible to create complex input syntax
% very easily. For example, given the argument definition
% `|s o o m O{default}|', the input `|*[Foo]{Bar}|' would be parsed as:
% \begin{itemize}[nolistsep]
%   \item |#1| = |\BooleanTrue|
%   \item |#2| = |Foo|
%   \item |#3| = |-NoValue-|
%   \item |#4| = |Bar|
%   \item |#5| = |default|
% \end{itemize}
% whereas `|[One][Two]{}[Three]|' would be parsed as:
% \begin{itemize}[nolistsep]
%   \item |#1| = |\BooleanFalse|
%   \item |#2| = |One|
%   \item |#3| = |Two|
%   \item |#4| = ||
%   \item |#5| = |Three|
% \end{itemize}
%
% Delimited argument types (\texttt{d}, \texttt{o} and \texttt{r}) are
% defined such that they require matched pairs of delimiters when collecting
% an argument. For example
% \begin{verbatim}
%   \NewDocumentCommand{\foo}{o}{#1}
%   \foo[[content]] % #1 = "[content]"
%   \foo[[]         % Error: missing closing "]"
% \end{verbatim}
% Also note that |{| and |}| cannot be used as delimiters as they are used
% by \TeX{} as grouping tokens. Arguments to be grabbed inside these tokens
% must be created as either \texttt{m}- or \texttt{g}-type arguments.
%
% Within delimited arguments, non-balanced or otherwise awkward tokens may
% be included by protecting the entire argument with a brace pair
% \begin{verbatim}
%   \NewDocumentCommand{\foobar}{o}{#1}
%   \foobar[{[}]         % Allowed as the "[" is 'hidden'
% \end{verbatim}
% These braces will be stripped if they surround the \emph{entire} content
% of the optional argument
% \begin{verbatim}
%   \NewDocumentCommand{\foobaz}{o}{#1}
%   \foobaz[{abc}]         % => "abc"
%   \foobaz[ {abc}]         % => " {abc}"
% \end{verbatim}
%
% Two more tokens have a special meaning when creating an argument
% specifier. First, \texttt{+} is used to make an argument long (to
% accept paragraph tokens). In contrast to \LaTeXe's \cs{newcommand},
% this applies on an argument-by-argument basis. So modifying the
% example to `|s o o +m O{default}|' means that the mandatory argument
% is now \cs{long}, whereas the optional arguments are not.
%
% Secondly, the token \texttt{>} is used to declare so-called
% \enquote{argument processors}, which can be used to modify the contents of an
% argument before it is passed to the macro definition. The use of
% argument processors is a somewhat advanced topic, (or at least a less
% commonly used feature) and is covered in Section~\ref{sec:processors}.
%
% When an optional argument is followed by a mandatory argument with the
% same delimiter, \pkg{xparse} issues a warning because the optional
% argument could not be omitted by the user, thus becoming in effect
% mandatory.  This applies to \texttt{o}, \texttt{d}, \texttt{O},
% \texttt{D}, \texttt{s}, \texttt{t}, \texttt{e}, and \texttt{E} type
% arguments followed by \texttt{r} or \texttt{R}-type required
% arguments, but also to \texttt{g} or \texttt{G} type arguments
% followed by \texttt{m} type arguments.
%
% \subsection{Spacing and optional arguments}
%
% \TeX{} will find the first argument after a function name irrespective
% of any intervening spaces. This is true for both mandatory and
% optional arguments. So |\foo[arg]| and \verb*|\foo   [arg]| are
% equivalent. Spaces are also ignored when collecting arguments up
% to the last mandatory argument to be collected (as it must exist).
% So after
% \begin{verbatim}
%   \NewDocumentCommand \foo { m o m } { ... }
% \end{verbatim}
% the user input |\foo{arg1}[arg2]{arg3}| and
% \verb*|\foo{arg1}  [arg2]   {arg3}| will both be parsed in the same
% way. However, spaces are \emph{not} ignored when parsing optional
% arguments after the last mandatory argument. Thus with
% \begin{verbatim}
%   \NewDocumentCommand \foobar { m o } { ... }
% \end{verbatim}
% |\foobar{arg1}[arg2]| will find an optional argument but
% \verb*|\foobar{arg1} [arg2]| will not. This is so that trailing optional
% arguments are not picked up \enquote{by accident} in input.
%
% There is one major exception to the rules listed above: when
% \pkg{xparse} is used to define what \TeX\ defines as \enquote{control
% symbols} in which the function name is made up of a single character,
% such as \enquote{\cmd{\\}}, spaces are \emph{not} ignored directly
% after them even for mandatory arguments.
%
% \subsection{Required delimited arguments}
%
% The contrast between a delimited (\texttt{D}-type) and \enquote{required
% delimited} (\texttt{R}-type) argument is that an error will be raised if
% the latter is missing. Thus for example
% \begin{verbatim}
%   \NewDocumentCommand {\foobaz} {r()m} {}
%   \foobaz{oops}
% \end{verbatim}
% will lead to an error message being issued. The marker |-NoValue-|
% (\texttt{r}-type) or user-specified default (for \texttt{R}-type) will be
% inserted to allow error recovery.
%
% Users should note that support for required delimited arguments is somewhat
% experimental. Feedback is therefore very welcome on the \texttt{LaTeX-L}
% mailing list.
%
% \subsection{Verbatim arguments}
%
% Arguments of type~\texttt{v} are read in verbatim mode, which will
% result in the grabbed argument consisting of tokens of category codes
% $12$~(\enquote{other}) and $13$~(\enquote{active}), except spaces,
% which are given category code $10$~(\enquote{space}). The argument is
% delimited in a similar manner to the \LaTeXe{} \cs{verb} function, or
% by (correctly nested) pairs of braces.
%
% Functions containing verbatim arguments cannot appear in the arguments
% of other functions. The \texttt{v}~argument specifier includes code to check
% this, and will raise an error if the grabbed argument has already been
% tokenized by \TeX{} in an irreversible way.
%
% By default, an argument of type~\texttt{v} must be at most one line.
% Prefixing with \texttt{+} allows line breaks within the argument.
%
% Users should note that support for verbatim arguments is somewhat
% experimental. Feedback is therefore very welcome on the \texttt{LaTeX-L}
% mailing list.
%
% \subsection{Default values of arguments}
%
% Uppercase argument types (\texttt{O}, \texttt{D}, \ldots{}) allow to
% specify a default value to be used when the argument is missing; their
% lower-case counterparts use the special marker |-NoValue-|.  The
% default value can be expressed in terms of the value of any other
% arguments by using |#1|, |#2|, and so on.
% \begin{verbatim}
%   \NewDocumentCommand {\conjugate} { m O{#1ed} O{#2} } {(#1,#2,#3)}
%   \conjugate {walk}            % => (walk,walked,walked)
%   \conjugate {find} [found]    % => (find,found,found)
%   \conjugate {do} [did] [done] % => (do,did,done)
% \end{verbatim}
% The default values may refer to arguments that appear later in the
% argument specification.  For instance a command could accept two
% optional arguments, equal by default:
% \begin{verbatim}
%   \NewDocumentCommand {\margins} { O{#3} m O{#1} m } {(#1,#2,#3,#4)}
%   \margins {a} {b}              % => {(-NoValue-,a,-NoValue-,b)}
%   \margins [1cm] {a} {b}        % => {(1cm,a,1cm,b)}
%   \margins {a} [1cm] {b}        % => {(1cm,a,1cm,b)}
%   \margins [1cm] {a} [2cm] {b}  % => {(1cm,a,2cm,b)}
% \end{verbatim}
%
% Users should note that support for default arguments referring to
% other arguments is somewhat experimental. Feedback is therefore very
% welcome on the \texttt{LaTeX-L} mailing list.
%
% \subsection{Default values for \enquote{embellishments}}
% \label{sec:embellishment}
%
% The \texttt{E}-type argument allows one default value per test token.
% This is achieved by giving a list of defaults for each entry in the
% list, for example:
% \begin{verbatim}
%   E{^_}{{UP}{DOWN}}
% \end{verbatim}
% If the list of default values is \emph{shorter} than the list of test tokens,
% the special \cs{NoValue} marker will be returned (as for the \texttt{e}-type
% argument). Thus for example
% \begin{verbatim}
%   E{^_}{{UP}}
% \end{verbatim}
% has default \texttt{UP} for the |^| test token, but will return the
% \cs{NoValue} marker as a default for |_|. This allows mixing of explicit
% defaults with testing for missing values.
%
% \subsection{Declaring commands and environments}
%
% With the concept of an argument specifier defined, it is now
% possible to describe the methods available for creating both
% functions and environments using \pkg{xparse}.
%
% The interface-building commands are the preferred method for
% creating document-level functions in \LaTeX3. All of the functions
% generated in this way are naturally robust (using the \eTeX{}
% \cs{protected} mechanism).
%
% \begin{function}
%   {
%     \NewDocumentCommand     ,
%     \RenewDocumentCommand   ,
%     \ProvideDocumentCommand ,
%     \DeclareDocumentCommand
%   }
%   \begin{syntax}
%     \cs{NewDocumentCommand} \meta{Function} \Arg{arg spec} \Arg{code}
%   \end{syntax}
%   This family of commands are used to create a document-level
%   \meta{function}. The argument specification for the function is
%   given by \meta{arg spec},  and expanding
%   to be replaced by the \meta{code}.
% \end{function}
%
%   As an example:
%   \begin{verbatim}
%     \NewDocumentCommand \chapter { s o m }
%       {
%         \IfBooleanTF {#1}
%           { \typesetstarchapter {#3} }
%           { \typesetnormalchapter {#2} {#3} }
%       }
%   \end{verbatim}
%   would be a way to define a \cs{chapter} command which would
%   essentially behave like the current \LaTeXe{} command (except that it
%   would accept an optional argument even when a \texttt{*} was parsed).
%   The \cs{typesetnormalchapter} could test its first argument for being
%   |-NoValue-| to see if an optional argument was present.
%
%   The difference between the \cs{New\ldots} \cs{Renew\ldots},
%   \cs{Provide\ldots} and \cs{Declare\ldots} versions is the behaviour
%   if \meta{function} is already defined.
%   \begin{itemize}
%    \item \cs{NewDocumentCommand} will issue an error if \meta{function}
%      has already been defined.
%    \item \cs{RenewDocumentCommand} will issue an error if \meta{function}
%      has not previously been defined.
%    \item \cs{ProvideDocumentCommand} creates a new definition for
%      \meta{function} only if one has not already been given.
%     \item \cs{DeclareDocumentCommand} will always create the new
%       definition, irrespective of any existing \meta{function} with the
%       same name.  This should be used sparingly.
%   \end{itemize}
%
%   \begin{texnote}
%      Unlike \LaTeXe{}'s \cs{newcommand} and relatives, the
%      \cs{NewDocumentCommand} family of functions do not prevent creation of
%      functions with names starting \cs{end\ldots}.
%   \end{texnote}
%
% \begin{function}
%   {
%     \NewDocumentEnvironment     ,
%     \RenewDocumentEnvironment   ,
%     \ProvideDocumentEnvironment ,
%     \DeclareDocumentEnvironment
%   }
%   \begin{syntax}
%     \cs{NewDocumentEnvironment} \Arg{environment} \Arg{arg spec}
%     ~~\Arg{start code} \Arg{end code}
%   \end{syntax}
%   These commands work in the same way as \cs{NewDocumentCommand},
%   etc.\@, but create environments (\cs{begin}\Arg{environment} \ldots{}
%   \cs{end}\Arg{environment}). Both the \meta{start code} and
%   \meta{end code}
%   may access the arguments as defined by \meta{arg spec}.
%   The arguments will be given following \cs{begin}\Arg{environment}.
% \end{function}
%
% \subsection{Testing special values}
%
% Optional arguments created using \pkg{xparse} make use of dedicated
% variables to return information about the nature of the argument
% received.
%
% \begin{function}[EXP]{\IfNoValueT, \IfNoValueF, \IfNoValueTF}
%   \begin{syntax}
%     \cs{IfNoValueTF} \Arg{argument} \Arg{true code} \Arg{false code}
%     \cs{IfNoValueT} \Arg{argument} \Arg{true code}
%     \cs{IfNoValueF} \Arg{argument} \Arg{false code}
%   \end{syntax}
%   The \cs{IfNoValue(TF)} tests are used to check if \meta{argument} (|#1|,
%   |#2|, \emph{etc.}) is the special |-NoValue-| marker For example
%   \begin{verbatim}
%     \NewDocumentCommand \foo { o m }
%       {
%         \IfNoValueTF {#1}
%           { \DoSomethingJustWithMandatoryArgument {#2} }
%           {  \DoSomethingWithBothArguments {#1} {#2}   }
%       }
%   \end{verbatim}
%   will use a different internal function if the optional argument
%   is given than if it is not present.
%
%   Note that three tests are available, depending on which outcome
%   branches are required: \cs{IfNoValueTF}, \cs{IfNoValueT} and
%   \cs{IfNoValueF}.
%
%   As the \cs{IfNoValue(TF)} tests are expandable, it is possible to
%   test these values later, for example at the point of typesetting or
%   in an expansion context.
%
%   It is important to note that |-NoValue-| is constructed such that it
%   will \emph{not} match the simple text input |-NoValue-|, \emph{i.e.}
%   that
%   \begin{verbatim}
%     \IfNoValueTF{-NoValue-}
%   \end{verbatim}
%   will be logically \texttt{false}.
%
%   When two optional arguments follow each other (a syntax we typically
%   discourage), it can make sense to allow users of the command to
%   specify only the second argument by providing an empty first
%   argument.  Rather than testing separately for emptyness and for
%   |-NoValue-| it is then best to use the argument type~|O| with an
%   empty default value, and simply test for emptyness using the
%   \pkg{expl3} conditional \cs{tl_if_blank:nTF} or its \pkg{etoolbox}
%   analogue \tn{ifblank}.
% \end{function}
%
% \begin{function}[EXP]{\IfValueT, \IfValueF, \IfValueTF}
%   \begin{syntax}
%     \cs{IfValueTF} \Arg{argument} \Arg{true code} \Arg{false code}
%    \end{syntax}
%   The reverse form of the \cs{IfNoValue(TF)} tests are also available
%   as \cs{IfValue(TF)}. The context will determine which logical
%   form makes the most sense for a given code scenario.
% \end{function}
%
% \begin{variable}{\BooleanFalse, \BooleanTrue}
%   The \texttt{true} and \texttt{false} flags set when searching for
%   an optional token (using \texttt{s} or \texttt{t\meta{token}}) have
%   names which are accessible outside of code blocks.
% \end{variable}
%
% \begin{function}[EXP]{\IfBooleanT, \IfBooleanF, \IfBooleanTF}
%   \begin{syntax}
%     \cs{IfBooleanTF} \meta{argument} \Arg{true code} \Arg{false code}
%   \end{syntax}
%   Used to test if \meta{argument} (|#1|, |#2|, \emph{etc.}) is
%   \cs{BooleanTrue} or \cs{BooleanFalse}. For example
%   \begin{verbatim}
%     \NewDocumentCommand \foo { s m }
%       {
%         \IfBooleanTF #1
%           { \DoSomethingWithStar {#2} }
%           { \DoSomethingWithoutStar {#2} }
%       }
%   \end{verbatim}
%   checks for a star as the first argument, then chooses the action to
%   take based on this information.
% \end{function}
%
% \subsection{Argument processors}
% \label{sec:processors}
%
% \pkg{xparse} introduces the idea of an argument processor, which is
% applied to an argument \emph{after} it has been grabbed by the
% underlying system but before it is passed to \meta{code}. An argument
% processor can therefore be used to regularise input at an early stage,
% allowing the internal functions to be completely independent of input
% form. Processors are applied to user input and to default values for
% optional arguments, but \emph{not} to the special \cs{NoValue} marker.
%
% Each argument processor is specified by the syntax
% \texttt{>}\marg{processor} in the argument specification. Processors
% are applied from right to left, so that
% \begin{verbatim}
%   >{\ProcessorB} >{\ProcessorA} m
% \end{verbatim}
% would apply \cs{ProcessorA}
% followed by \cs{ProcessorB} to the tokens grabbed by the \texttt{m}
% argument.
%
% \begin{variable}{\ProcessedArgument}
%   \pkg{xparse} defines a very small set of processor functions. In the
%   main, it is anticipated that code writers will want to create their
%   own processors. These need to accept one argument, which is the
%   tokens as grabbed (or as returned by a previous processor function).
%   Processor functions should return the processed argument as the
%   variable \cs{ProcessedArgument}.
% \end{variable}
%
% \begin{function}{\ReverseBoolean}
%   \begin{syntax}
%     \cs{ReverseBoolean}
%   \end{syntax}
%   This processor reverses the logic of \cs{BooleanTrue} and
%   \cs{BooleanFalse}, so that the example from earlier would become
%   \begin{verbatim}
%     \NewDocumentCommand \foo { > { \ReverseBoolean } s m }
%       {
%         \IfBooleanTF #1
%           { \DoSomethingWithoutStar {#2} }
%           { \DoSomethingWithStar {#2} }
%       }
%   \end{verbatim}
% \end{function}
%
% \begin{function}[updated = 2012-02-12]{\SplitArgument}
%   \begin{syntax}
%     \cs{SplitArgument} \Arg{number} \Arg{token}
%   \end{syntax}
%   This processor splits the argument given at each occurrence of the
%   \meta{token} up to a maximum of \meta{number} tokens (thus
%   dividing the input into $\text{\meta{number}} + 1$ parts).
%   An error is given if too many \meta{tokens} are present in the
%   input. The processed input is placed inside
%   $\text{\meta{number}} + 1$ sets of braces for further use.
%   If there are fewer than \Arg{number} of \Arg{tokens} in the argument
%   then \cs{NoValue} markers are added at the end of the processed
%   argument.
%   \begin{verbatim}
%     \NewDocumentCommand \foo
%       { > { \SplitArgument { 2 } { ; } } m }
%       { \InternalFunctionOfThreeArguments #1 }
%   \end{verbatim}
%   Any category code $13$ (active) \meta{tokens} will be replaced
%   before the split takes place. Spaces are trimmed at each end of each
%   item parsed.
% \end{function}
%
% \begin{function}{\SplitList}
%   \begin{syntax}
%     \cs{SplitList} \Arg{token(s)}
%   \end{syntax}
%   This processor splits the argument given at each occurrence of the
%   \meta{token(s)} where the number of items is not fixed. Each item is
%   then wrapped in braces within |#1|. The result is that the
%   processed argument can be further processed using a mapping function.
%   \begin{verbatim}
%     \NewDocumentCommand \foo
%       { > { \SplitList { ; } } m }
%       { \MappingFunction #1 }
%   \end{verbatim}
%   If only a single \meta{token} is used for the split, any
%   category code $13$ (active) \meta{token} will be replaced
%   before the split takes place.
% \end{function}
%
% \begin{function}[EXP]{\ProcessList}
%   \begin{syntax}
%     \cs{ProcessList} \Arg{list} \Arg{function}
%   \end{syntax}
%   To support \cs{SplitList}, the function \cs{ProcessList} is available
%   to apply a \meta{function} to every entry in a \meta{list}. The
%   \meta{function} should absorb one argument: the list entry. For example
%   \begin{verbatim}
%     \NewDocumentCommand \foo
%       { > { \SplitList { ; } } m }
%       { \ProcessList {#1} { \SomeDocumentFunction } }
%   \end{verbatim}
%
%   \textbf{This function is experimental.}
% \end{function}
%
% \begin{function}{\TrimSpaces}
%   \begin{syntax}
%     \cs{TrimSpaces}
%   \end{syntax}
%   Removes any leading and trailing spaces (tokens with character code~$32$
%   and category code~$10$) for the ends of the argument. Thus for example
%   declaring a function
%   \begin{verbatim}
%     \NewDocumentCommand \foo
%       { > { \TrimSpaces } m }
%       { \showtokens {#1} }
%   \end{verbatim}
%   and using it in a document as
%   \begin{verbatim}
%     \foo{ hello world }
%   \end{verbatim}
%   will show \texttt{hello world} at the terminal, with the space at each
%   end removed. \cs{TrimSpaces} will remove multiple spaces from the ends of
%   the input in cases where these have been included such that the standard
%   \TeX{} conversion of multiple spaces to a single space does not apply.
%
%   \textbf{This function is experimental.}
% \end{function}
%
% \begin{function}[added = 2017-07-14]{\AssertInt}
%   \begin{syntax}
%     \cs{AssertInt}
%   \end{syntax}
%   This processor checks if the supplied argument is a valid integer
%   literal.
%   \begin{verbatim}
%     \NewDocumentCommand \foo { > { \AssertInt } m }
%       {
%         % ...
%       }
%   \end{verbatim}
% \end{function}
%
% \begin{function}[added = 2017-07-14]{\AssertFp}
%   \begin{syntax}
%     \cs{AssertFp}
%   \end{syntax}
%   This processor checks if the supplied argument is a valid floating
%   point literal.
%   \begin{verbatim}
%     \NewDocumentCommand \foo { > { \AssertFp } m }
%       {
%         % ...
%       }
%   \end{verbatim}
% \end{function}
%
% \begin{function}[added = 2017-07-14]{\AssertDim}
%   \begin{syntax}
%     \cs{AssertDim}
%   \end{syntax}
%   This processor checks if the supplied argument is a valid
%   dimension literal.
%   \begin{verbatim}
%     \NewDocumentCommand \foo { > { \AssertDim } m }
%       {
%         % ...
%       }
%   \end{verbatim}
% \end{function}
%
% \subsection{Fully-expandable document commands}
%
% There are \emph{very rare} occasion when it may be useful to create
% functions using a fully-expandable argument grabber. To support this,
% \pkg{xparse} can create expandable functions as well as the usual
% robust ones. This imposes a number of restrictions on the nature of
% the arguments accepted by a function, and the code it implements.
% This facility should only be used when \emph{absolutely necessary};
% if you do not understand when this might be, \emph{do not use these
% functions}!
%
% \begin{function}
%   {
%     \NewExpandableDocumentCommand     ,
%     \RenewExpandableDocumentCommand   ,
%     \ProvideExpandableDocumentCommand ,
%     \DeclareExpandableDocumentCommand
%   }
%   \begin{syntax}
%     \cs{NewExpandableDocumentCommand}
%     ~~~~\meta{function} \Arg{arg spec} \Arg{code}
%   \end{syntax}
%   This family of commands is used to create a document-level \meta{function},
%   which will grab its arguments in a fully-expandable manner. The
%   argument specification for the function is given by \meta{arg spec},
%   and the function will execute \meta{code}. In  general, \meta{code} will
%   also be fully expandable, although it is possible that this will
%   not be the case (for example, a function for use in a table might
%   expand so that \cs{omit} is the first non-expandable non-space token).
%
%   Parsing arguments expandably imposes a number of restrictions on
%   both the type of arguments that can be read and the error checking
%   available:
%   \begin{itemize}
%     \item The last argument (if any are present) must be one of the
%       mandatory types \texttt{m} or \texttt{r}.
%     \item All short arguments appear before long arguments.
%     \item The mandatory argument types \texttt{l} and \texttt{u} may
%       not be used after optional arguments.
%     \item The optional argument types \texttt{g}
%       and \texttt{G} are not available.
%     \item The \enquote{verbatim} argument type \texttt{v} is not available.
%     \item Argument processors (using \texttt{>}) are not available.
%     \item It is not possible to differentiate between, for example
%       |\foo[| and |\foo{[}|: in both cases the \texttt{[} will be
%       interpreted as the start of an optional argument. As a
%       result, checking for optional arguments is less robust than
%       in the standard version.
%   \end{itemize}
%   \pkg{xparse} will issue an error if an argument specifier is given
%   which does not conform to the first six requirements. The last
%   item is an issue when the function is used, and so is beyond the
%   scope of \pkg{xparse} itself.
% \end{function}
%
% \subsection{Access to the argument specification}
%
% The argument specifications for document commands and environments are
% available for examination and use.
%
% \begin{function}{\GetDocumentCommandArgSpec, \GetDocumentEnvironmentArgSpec}
%   \begin{syntax}
%     \cs{GetDocumentCommandArgSpec} \meta{function}
%     \cs{GetDocumentEnvironmentArgSpec} \meta{environment}
%   \end{syntax}
%   These functions transfer the current argument specification for the
%   requested \meta{function} or \meta{environment} into the token list
%   variable \cs{ArgumentSpecification}. If the \meta{function} or
%   \meta{environment} has no known argument specification then an error
%   is issued. The assignment to \cs{ArgumentSpecification} is local to
%   the current \TeX{} group.
% \end{function}
%
% \begin{function}
%   {\ShowDocumentCommandArgSpec,  \ShowDocumentEnvironmentArgSpec}
%   \begin{syntax}
%     \cs{ShowDocumentCommandArgSpec} \meta{function}
%     \cs{ShowDocumentEnvironmentArgSpec} \meta{environment}
%   \end{syntax}
%   These functions show the current argument specification for the
%   requested \meta{function} or \meta{environment} at the terminal. If
%   the \meta{function} or \meta{environment} has no known argument
%   specification then an error is issued.
% \end{function}
%
% \section{Load-time options}
%
% \DescribeOption{log-declarations}
% The package recognises the load-time option \texttt{log-declarations},
% which is a key--value option taking the value \texttt{true} and
% \texttt{false}. By default, the option is set to \texttt{true}, meaning
% that each command or environment declared is logged. By loading
% \pkg{xparse} using
% \begin{verbatim}
%   \usepackage[log-declarations=false]{xparse}
% \end{verbatim}
% this may be suppressed and no information messages are produced.
%
% \end{documentation}
%
% \begin{implementation}
%
% \section{\pkg{xparse} implementation}
%
%    \begin{macrocode}
%<*package>
%    \end{macrocode}
%
%    \begin{macrocode}
%<@@=xparse>
%    \end{macrocode}
%
%    \begin{macrocode}
\ProvidesExplPackage{xparse}{2017/05/29}{}
  {L3 Experimental document command parser}
%    \end{macrocode}
%
% \subsection{Variables and constants}
%
% \begin{variable}{\c_@@_no_value_tl}
%   A special \enquote{awkward} token list: it contains two |-|~tokens with
%   different category codes. This is used as the marker for nothing being
%   returned when no optional argument is given.
%    \begin{macrocode}
\tl_const:Nx \c_@@_no_value_tl
  { \char_generate:nn { `\- } { 11 } NoValue- }
%    \end{macrocode}
% \end{variable}
%
% \begin{variable}{\c_@@_special_chars_seq}
%   In \IniTeX{} mode, we store special characters in a sequence.
%   Maybe |$| or |&| will have to be added later.
%    \begin{macrocode}
%<*initex>
\seq_new:N \c_@@_special_chars_seq
\seq_set_split:Nnn \c_@@_special_chars_seq { }
  { \  \\ \{ \} \# \^ \_ \% \~ }
%</initex>
%    \end{macrocode}
% \end{variable}
%
% \begin{variable}{\l_@@_arg_spec_tl}
%   Holds the argument specification after normalization of shorthands.
%    \begin{macrocode}
\tl_new:N \l_@@_arg_spec_tl
%    \end{macrocode}
% \end{variable}
%
% \begin{variable}{\l_@@_args_tl}
%   Token list variable for grabbed arguments.
%    \begin{macrocode}
\tl_new:N \l_@@_args_tl
%    \end{macrocode}
% \end{variable}
%
% \begin{variable}{\l_@@_args_i_tl, \l_@@_args_ii_tl}
%   Hold the modified arguments when dealing with default values or
%   processors.
%    \begin{macrocode}
\tl_new:N \l_@@_args_i_tl
\tl_new:N \l_@@_args_ii_tl
%    \end{macrocode}
% \end{variable}
%
% \begin{variable}{\l_@@_current_arg_int}
%   The number of the current argument being set up: this is used to
%   make sure there are at most 9 arguments, then for creating the
%   expandable auxiliary functions and knowing how many arguments the
%   code function should take.
%    \begin{macrocode}
\int_new:N \l_@@_current_arg_int
%    \end{macrocode}
% \end{variable}
%
% \begin{variable}{\l_@@_defaults_bool, \l_@@_defaults_tl}
%   The boolean indicates whether there are any argument with default
%   value other than |-NoValue-|; the token list holds the code to
%   determine these default values in terms of other arguments.
%    \begin{macrocode}
\bool_new:N \l_@@_defaults_bool
\tl_new:N \l_@@_defaults_tl
%    \end{macrocode}
% \end{variable}
%
% \begin{variable}{\l_@@_environment_bool}
%   Generating environments uses the same mechanism as generating functions.
%   However, full processing of arguments is always needed for environments,
%   and so the function-generating code needs to know this.
%    \begin{macrocode}
\bool_new:N \l_@@_environment_bool
%    \end{macrocode}
% \end{variable}
%
% \begin{variable}{\l_@@_expandable_bool}
%   Used to indicate if an expandable command is begin generated, as this
%   affects both the acceptable argument types and how they are implemented.
%    \begin{macrocode}
\bool_new:N \l_@@_expandable_bool
%    \end{macrocode}
% \end{variable}
%
% \begin{variable}{\l_@@_expandable_aux_name_tl}
%   Used to create pretty-printing names for the auxiliaries: although the
%   immediate definition does not vary, the full expansion does and so it
%   does not count as a constant.
%    \begin{macrocode}
\tl_new:N \l_@@_expandable_aux_name_tl
\tl_set:Nn \l_@@_expandable_aux_name_tl
  {
    \l_@@_function_tl \c_space_tl
    ( arg~ \int_use:N \l_@@_current_arg_int )
  }
%    \end{macrocode}
% \end{variable}
%
% \begin{variable}{\g_@@_grabber_int}
%   Used (in exceptional cases) to get unique names for grabbers used by
%   expandable commands.
%    \begin{macrocode}
\int_new:N \g_@@_grabber_int
%    \end{macrocode}
% \end{variable}
%
% \begin{variable}{\l_@@_fn_tl}
%   For passing the pre-formed name of the auxiliary to be used as the
%   parsing function.
%    \begin{macrocode}
\tl_new:N \l_@@_fn_tl
%    \end{macrocode}
% \end{variable}
%
% \begin{variable}{\l_@@_fn_code_tl}
%   For passing the pre-formed name of the auxiliary that contains the
%   actual code.
%    \begin{macrocode}
\tl_new:N \l_@@_fn_code_tl
%    \end{macrocode}
% \end{variable}
%
% \begin{variable}{\l_@@_function_tl}
%   Holds the control sequence name of the function currently being
%   defined: used to avoid passing this as an argument and to avoid repeated
%   use of \cs{cs_to_str:N}.
%    \begin{macrocode}
\tl_new:N \l_@@_function_tl
%    \end{macrocode}
% \end{variable}
%
% \begin{variable}{\l_@@_grab_expandably_bool}
%   When defining a non-expandable command, indicates whether the
%   arguments can all safely be grabbed by expandable grabbers.  This is
%   to support abuses of \pkg{xparse} that use protected functions
%   inside csname constructions.
%    \begin{macrocode}
\bool_new:N \l_@@_grab_expandably_bool
%    \end{macrocode}
% \end{variable}
%
% \begin{variable}{\l_@@_last_delimiters_tl}
%   Holds the delimiters (first tokens) of all optional arguments since
%   the previous mandatory argument, to warn about cases where it would
%   be impossible to omit optional arguments completely because the
%   following mandatory argument has the same delimiter as one of the
%   optional arguments.
%    \begin{macrocode}
\tl_new:N \l_@@_last_delimiters_tl
%    \end{macrocode}
% \end{variable}
%
% \begin{variable}{\l_@@_long_bool}
%   Used to indicate that an argument is long, on a per-argument basis.
%    \begin{macrocode}
\bool_new:N \l_@@_long_bool
%    \end{macrocode}
% \end{variable}
%
% \begin{variable}{\l_@@_m_args_int}
%   The number of \texttt{m} arguments: if this is the same as the total
%   number of arguments, then a short-cut can be taken in the creation of
%   the grabber code.
%    \begin{macrocode}
\int_new:N \l_@@_m_args_int
%    \end{macrocode}
% \end{variable}
%
% \begin{variable}{\l_@@_mandatory_args_int}
%   Holds the total number of mandatory arguments for a function, which is
%   needed to tell whether further mandatory arguments follow an optional
%   one.
%    \begin{macrocode}
\int_new:N \l_@@_mandatory_args_int
%    \end{macrocode}
% \end{variable}
%
% \begin{variable}{\l_@@_prefixed_bool}
%   When preparing the signature of non-expandable commands, indicates
%   that the current argument is affected by a processor or by |+|
%   (namely is long).
%    \begin{macrocode}
\bool_new:N \l_@@_prefixed_bool
%    \end{macrocode}
% \end{variable}
%
% \begin{variable}{\l_@@_process_all_tl, \l_@@_process_one_tl, \l_@@_process_some_bool}
%   When preparing the signature, the processors that will be applied to
%   a given argument are collected in \cs{l_@@_process_one_tl}, while
%   \cs{l_@@_process_all_tl} contains processors for all arguments.  The
%   boolean indicates whether there are any processors (to bypass the
%   whole endeavour otherwise).
%    \begin{macrocode}
\tl_new:N \l_@@_process_all_tl
\tl_new:N \l_@@_process_one_tl
\bool_new:N \l_@@_process_some_bool
%    \end{macrocode}
% \end{variable}
%
% \begin{variable}{\l_@@_signature_tl}
%   Used when constructing the signature (code for argument grabbing) to
%   hold what will become the implementation of the main function.
%   When arguments are grabbed (at point of use of the command/environment),
%   it also stores the code for grabbing the remaining arguments.
%    \begin{macrocode}
\tl_new:N \l_@@_signature_tl
%    \end{macrocode}
% \end{variable}
%
% \begin{variable}{\l_@@_some_long_bool, \l_@@_some_short_bool}
%   Both of these flags are set while normalizing the argument
%   specification.  To grab arguments expandably, all short arguments
%   must appear before long arguments, so as soon as the first long
%   argument is seen (other than \texttt{t}-type, whose long status is
%   ignored) the \texttt{some_long} flag is set.  The
%   \texttt{some_short} flag is used for expandable commands, to know
%   whether to define a short auxiliary too.
%    \begin{macrocode}
\bool_new:N \l_@@_some_long_bool
\bool_new:N \l_@@_some_short_bool
%    \end{macrocode}
% \end{variable}
%
% \begin{variable}{\l_@@_tmp_prop, \l_@@_tmpa_tl, \l_@@_tmpb_tl}
% \begin{macro}{\@@_tmp:w}
%   Scratch space.
%    \begin{macrocode}
\prop_new:N \l_@@_tmp_prop
\tl_new:N \l_@@_tmpa_tl
\tl_new:N \l_@@_tmpb_tl
\cs_new_eq:NN \@@_tmp:w ?
%    \end{macrocode}
% \end{macro}
% \end{variable}
%
% \subsection{Declaring commands and environments}
%
% \begin{macro}{\@@_declare_cmd:Nnn, \@@_declare_expandable_cmd:Nnn}
% \begin{macro}[aux]{\@@_declare_cmd_aux:Nnn}
% \begin{macro}[int]
%   {\@@_declare_cmd_internal:Nnn, \@@_declare_cmd_internal:cnx}
%   The main functions for creating commands set the appropriate flag then
%   use the same internal code to do the definition.
%    \begin{macrocode}
\cs_new_protected:Npn \@@_declare_cmd:Nnn
  {
    \bool_set_false:N \l_@@_expandable_bool
    \@@_declare_cmd_aux:Nnn
  }
\cs_new_protected:Npn \@@_declare_expandable_cmd:Nnn
  {
    \bool_set_true:N \l_@@_expandable_bool
    \@@_declare_cmd_aux:Nnn
  }
%    \end{macrocode}
%  The first stage is to log information, both for the user in the log and
%  for programmatic use in a property list of all declared commands.
%    \begin{macrocode}
\cs_new_protected:Npn \@@_declare_cmd_aux:Nnn #1#2
  {
    \cs_if_exist:NTF #1
      {
        \__msg_kernel_info:nnxx { xparse } { redefine-command }
          { \token_to_str:N #1 } { \tl_to_str:n {#2} }
      }
      {
        \bool_lazy_or:nnT
          { \cs_if_exist_p:c { \cs_to_str:N #1 ~ code } }
          { \cs_if_exist_p:c { \cs_to_str:N #1 ~ defaults } }
          {
            \__msg_kernel_warning:nnx { xparse } { unsupported-let }
              { \token_to_str:N #1 }
          }
        \__msg_kernel_info:nnxx { xparse } { define-command }
          { \token_to_str:N #1 } { \tl_to_str:n {#2} }
      }
    \bool_set_false:N \l_@@_environment_bool
    \@@_declare_cmd_internal:Nnn #1 {#2}
  }
%    \end{macrocode}
%   The real business of defining a document command starts with setting up
%   the appropriate name, then normalizing the argument specification to get rid of
%   shorthands (this step also counts the number of mandatory arguments).
%    \begin{macrocode}
\cs_new_protected:Npn \@@_declare_cmd_internal:Nnn #1#2#3
  {
    \tl_set:Nx \l_@@_function_tl { \cs_to_str:N #1 }
    \@@_normalize_arg_spec:n {#2}
    \exp_args:No \@@_prepare_signature:n \l_@@_arg_spec_tl
    \@@_declare_cmd_code:Nnn #1 {#2} {#3}
    \@@_break_point:n {#2}
  }
\cs_generate_variant:Nn \@@_declare_cmd_internal:Nnn { cnx }
%    \end{macrocode}
% \end{macro}
% \end{macro}
% \end{macro}
%
% \begin{macro}{\@@_break_point:n}
%   A marker used to escape from creating a definition if necessary.
%    \begin{macrocode}
\cs_new_eq:NN \@@_break_point:n \use_none:n
%    \end{macrocode}
% \end{macro}
%
% \begin{macro}{\@@_declare_cmd_code:Nnn}
% \begin{macro}[aux]
%   {\@@_declare_cmd_code_aux:Nnn, \@@_declare_cmd_code_expandable:Nnn}
%   The appropriate auxiliary is called.
%    \begin{macrocode}
\cs_new_protected:Npn \@@_declare_cmd_code:Nnn
  {
    \bool_if:NTF \l_@@_grab_expandably_bool
      { \@@_declare_cmd_code_expandable:Nnn }
      { \@@_declare_cmd_code_aux:Nnn }
   }
%    \end{macrocode}
%   Standard functions call \cs{@@_start:nNNnnn}, which receives the
%   argument specification, an auxiliary used for
%   grabbing arguments, an auxiliary containing the code, and then the
%   signature, default arguments, and processors.
%    \begin{macrocode}
\cs_new_protected:Npn \@@_declare_cmd_code_aux:Nnn #1#2#3
  {
    \cs_generate_from_arg_count:cNnn
      { \l_@@_function_tl \c_space_tl code }
      \cs_set_protected:Npn \l_@@_current_arg_int {#3}
    \cs_set_protected_nopar:Npx #1
      {
        \@@_start:nNNnnn
          { \exp_not:n {#2} }
          \exp_not:c { \l_@@_function_tl \c_space_tl }
          \exp_not:c { \l_@@_function_tl \c_space_tl code }
          { \exp_not:o \l_@@_signature_tl }
          {
            \bool_if:NT \l_@@_defaults_bool
              { \exp_not:o \l_@@_defaults_tl }
          }
          {
            \bool_if:NT \l_@@_process_some_bool
              { \exp_not:o \l_@@_process_all_tl }
          }
      }
  }
%    \end{macrocode}
%   Expandable functions and functions whose arguments can be grabbed
%   expandably call \cs{@@_start_expandable:nNNNNn}, which receives the
%   argument specification, four auxiliaries (two for grabbing arguments, one for
%   the code, and one for default arguments), and finally the signature.
%   Non-expandable functions that take this branch should nevertheless
%   be protected, as well as their \texttt{code} function.  They will
%   only be expanded in contexts such as constructing a csname.
%   The two grabbers (named after the function with one or two spaces)
%   are needed when there are both short and long arguments; otherwise
%   the same grabber is included twice in the definition.  If all
%   arguments are long or all are short the (only) grabber is defined
%   correspondingly to be long/short.  Otherwise two grabbers are
%   defined, one long, one short.
%    \begin{macrocode}
\cs_new_protected:Npn \@@_declare_cmd_code_expandable:Nnn #1#2#3
  {
    \exp_args:Ncc \cs_generate_from_arg_count:NNnn
      { \l_@@_function_tl \c_space_tl code }
      { cs_set \bool_if:NF \l_@@_expandable_bool { _protected } :Npn }
      \l_@@_current_arg_int {#3}
    \bool_if:NT \l_@@_defaults_bool
      {
        \use:x
          {
            \cs_generate_from_arg_count:cNnn
              { \l_@@_function_tl \c_space_tl defaults }
              \cs_set:Npn \l_@@_current_arg_int
              { \exp_not:o \l_@@_defaults_tl }
          }
      }
    \bool_if:NTF \l_@@_expandable_bool
      { \cs_set_nopar:Npx } { \cs_set_protected_nopar:Npx } #1
      {
        \exp_not:N \@@_start_expandable:nNNNNn
          { \exp_not:n {#2} }
          \exp_not:c { \l_@@_function_tl \c_space_tl }
          \exp_not:c
            {
              \l_@@_function_tl \c_space_tl
              \bool_if:NT \l_@@_some_short_bool
                { \bool_if:NT \l_@@_some_long_bool { \c_space_tl } }
            }
          \exp_not:c { \l_@@_function_tl \c_space_tl code }
          \bool_if:NTF \l_@@_defaults_bool
            { \exp_not:c { \l_@@_function_tl \c_space_tl defaults } }
            { ? }
          { \exp_not:o \l_@@_signature_tl }
      }
    \bool_if:NTF \l_@@_some_long_bool
      {
        \bool_if:NT \l_@@_some_short_bool
          {
            \cs_set_nopar:cpx { \l_@@_function_tl \c_space_tl \c_space_tl }
              ##1##2 { ##1 {##2} }
          }
        \cs_set:cpx
      }
      { \cs_set_nopar:cpx }
          { \l_@@_function_tl \c_space_tl } ##1##2 { ##1 {##2} }
  }
%    \end{macrocode}
% \end{macro}
% \end{macro}
%
% \begin{macro}{\@@_declare_env:nnnn}
% \begin{macro}[int]{\@@_declare_env_internal:nnnn}
%   The lead-off to creating an environment is much the same as that for
%   creating a command: issue the appropriate message, store the argument
%   specification then hand off to an internal function.
%    \begin{macrocode}
\cs_new_protected:Npn \@@_declare_env:nnnn #1#2
  {
%<*initex>
    \cs_if_exist:cTF { environment~ #1 }
%</initex>
%<*package>
    \cs_if_exist:cTF {#1}
%</package>
      {
        \__msg_kernel_info:nnxx { xparse } { redefine-environment }
          {#1} { \tl_to_str:n {#2} }
      }
      {
        \__msg_kernel_info:nnxx { xparse } { define-environment }
          {#1} { \tl_to_str:n {#2} }
      }
    \bool_set_false:N \l_@@_expandable_bool
    \bool_set_true:N \l_@@_environment_bool
    \@@_declare_env_internal:nnnn {#1} {#2}
  }
%    \end{macrocode}
%   Creating a document environment requires a few more steps than creating
%   a single command. In order to pass the arguments of the command to the
%   end of the function, it is necessary to store the grabbed arguments.
%   To do that, the function used at the end of the environment has to be
%   redefined to contain the appropriate information. To minimize the amount
%   of expansion at point of use, the code here is expanded now as well as
%   when used.
%    \begin{macrocode}
\cs_new_protected:Npn \@@_declare_env_internal:nnnn #1#2#3#4
  {
    \@@_declare_cmd_internal:cnx { environment~ #1 } {#2}
      {
        \cs_set_nopar:Npx \exp_not:c { environment~ #1 ~end~aux }
          {
            \exp_not:N \exp_not:N \exp_not:c { environment~ #1~end~aux~ }
            \exp_not:n { \exp_not:o \l_@@_args_tl }
          }
        \exp_not:n {#3}
      }
    \cs_set_nopar:cpx { environment~ #1 ~end }
      { \exp_not:c { environment~ #1 ~end~aux } }
    \cs_generate_from_arg_count:cNnn
      { environment~ #1 ~end~aux~ } \cs_set:Npn
      \l_@@_current_arg_int {#4}
%<*package>
    \cs_set_eq:cc {#1}       { environment~ #1 }
    \cs_set_eq:cc { end #1 } { environment~ #1 ~end }
%</package>
  }
%    \end{macrocode}
% \end{macro}
% \end{macro}
%
% \subsection{Structure of \pkg{xparse} commands}
%
% \begin{macro}{\@@_start:nNNnnn}
% \begin{macro}[aux]{\@@_start_aux:nNNnnn}
%   This sets up a few variables to minimize the boilerplate code
%   included in all \pkg{xparse}-defined commands.  It then runs the
%   grabbers~|#4|.  Again, the argument specification |#1| is only for
%   diagnostics.  The control sequence equal to \cs{scan_stop:} protects
%   against \texttt{f}-expansion and ensures errors are more reasonable
%   when an \pkg{xparse} command is placed in a csname.
%    \begin{macrocode}
\cs_new_protected:Npx \@@_start:nNNnnn
  {
    \exp_not:c { xparse~function~is~not~expandable }
    \exp_not:N \@@_start_aux:nNNnnn
  }
\cs_new_protected:Npn \@@_start_aux:nNNnnn #1#2#3#4#5#6
  {
    \tl_clear:N \l_@@_args_tl
    \tl_set:Nn \l_@@_fn_tl {#2}
    \tl_set:Nn \l_@@_fn_code_tl {#3}
    \tl_set:Nn \l_@@_defaults_tl {#5}
    \tl_set:Nn \l_@@_process_all_tl {#6}
    #4 \@@_run_code:
  }
%    \end{macrocode}
% \end{macro}
% \end{macro}
%
% \begin{macro}{\@@_run_code:}
%   After arguments are grabbed, this function is responsible for
%   inserting default values, running processors, and finally running
%   the code.
%    \begin{macrocode}
\cs_new_protected:Npn \@@_run_code:
  {
    \tl_if_empty:NF \l_@@_defaults_tl { \@@_defaults: }
    \tl_if_empty:NF \l_@@_process_all_tl { \@@_args_process: }
    \exp_after:wN \l_@@_fn_code_tl \l_@@_args_tl
  }
%    \end{macrocode}
% \end{macro}
%
% \begin{macro}{\@@_defaults:}
% \begin{macro}[aux]{\@@_defaults_def:, \@@_defaults_def:nn, \@@_defaults_def:nnn}
% \begin{macro}[aux]{\@@_defaults_aux:,  \@@_defaults_error:w}
%   First construct \cs{@@_tmp:w} (see below) that will receive
%   the arguments found so far and determine default values for any
%   missing argument.  Then call it repeatedly until the set of
%   arguments stabilizes.  Since that could lead to an infinite loop we
%   only call it up to nine times, the maximal number needed for
%   stabilization if there is a chain of arguments that depend on each
%   other.  If that fails to stabilize raise an error.
%    \begin{macrocode}
\cs_new_protected:Npn \@@_defaults:
  {
    \@@_defaults_def:
    \tl_set_eq:NN \l_@@_args_i_tl \l_@@_args_tl
    \@@_defaults_aux: \@@_defaults_aux: \@@_defaults_aux:
    \@@_defaults_aux: \@@_defaults_aux: \@@_defaults_aux:
    \@@_defaults_aux: \@@_defaults_aux: \@@_defaults_aux:
    \@@_defaults_error:w
    \q_recursion_stop
    \tl_set_eq:NN \l_@@_args_tl \l_@@_args_i_tl
  }
\cs_new_protected:Npn \@@_defaults_aux:
  {
    \tl_set:Nx \l_@@_args_ii_tl
      { \exp_after:wN \@@_tmp:w \l_@@_args_i_tl }
    \tl_if_eq:NNT \l_@@_args_ii_tl \l_@@_args_i_tl
      { \use_none_delimit_by_q_recursion_stop:w }
    \tl_set_eq:NN \l_@@_args_i_tl \l_@@_args_ii_tl
  }
\cs_new_protected:Npn \@@_defaults_error:w \q_recursion_stop
  {
    \__msg_kernel_error:nnx { xparse } { loop-in-defaults }
      { \exp_after:wN \token_to_str:N \l_@@_fn_tl }
  }
%    \end{macrocode}
%   To construct \cs{@@_tmp:w}, first go through the arguments
%   found and the corresponding defaults, building a token list with
%   |{#|\meta{arg number}|}| for arguments found in the input (whose
%   default will not be used) and otherwise
%   |{|\cs{exp_not:n}\Arg{default}|}| for arguments whose default will
%   be used.
%    \begin{macrocode}
\cs_new_protected:Npn \@@_defaults_def:
  {
    \tl_clear:N \l_@@_tmpa_tl
    \int_zero:N \l_@@_current_arg_int
    \@@_tl_mapthread_function:NNN \l_@@_args_tl \l_@@_defaults_tl
      \@@_defaults_def:nn
    \cs_generate_from_arg_count:NNVo \@@_tmp:w \cs_set:Npn
      \l_@@_current_arg_int \l_@@_tmpa_tl
  }
\cs_generate_variant:Nn \cs_generate_from_arg_count:NNnn { NNVo }
\cs_new_protected:Npn \@@_defaults_def:nn
  {
    \int_incr:N \l_@@_current_arg_int
    \exp_args:NV \@@_defaults_def:nnn \l_@@_current_arg_int
  }
\cs_new_protected:Npn \@@_defaults_def:nnn #1#2#3
  {
    \tl_put_right:Nx \l_@@_tmpa_tl
      {
        {
          \exp_not:N \exp_not:n
            {
              \@@_if_no_value:nTF {#2}
                { \exp_not:o {#3} }
                { \exp_not:n { ## #1 } }
            }
        }
      }
  }
%    \end{macrocode}
% \end{macro}
% \end{macro}
% \end{macro}
%
% \begin{macro}{\@@_args_process:}
% \begin{macro}[aux]{\@@_args_process_loop:nn, \@@_args_process_aux:n, \@@_args_process_aux:nn}
%   Loop through arguments (stored in \cs{l_@@_args_tl}) and the
%   corresponding processors (in \cs{l_@@_process_all_tl})
%   simultaneously, apply all processors for each argument and store the
%   result back into \cs{l_@@_args_tl}.
%    \begin{macrocode}
\cs_new_protected:Npn \@@_args_process:
  {
    \tl_clear:N \l_@@_args_ii_tl
    \@@_tl_mapthread_function:NNN
      \l_@@_args_tl
      \l_@@_process_all_tl
      \@@_args_process_loop:nn
    \tl_set_eq:NN \l_@@_args_tl \l_@@_args_ii_tl
  }
\cs_new_protected:Npn \@@_args_process_loop:nn #1#2
  {
    \tl_set:Nn \ProcessedArgument {#1}
    \@@_if_no_value:nF {#1}
      { \tl_map_function:nN {#2} \@@_args_process_aux:n }
    \tl_put_right:No \l_@@_args_ii_tl
      { \exp_after:wN { \ProcessedArgument } }
  }
\cs_new_protected:Npn \@@_args_process_aux:n
  { \exp_args:No \@@_args_process_aux:nn { \ProcessedArgument } }
\cs_new_protected:Npn \@@_args_process_aux:nn #1#2 { #2 {#1} }
%    \end{macrocode}
% \end{macro}
% \end{macro}
%
% \begin{macro}[EXP]{\@@_start_expandable:nNNNNn}
%   This is called for all expandable commands.  |#6| is the signature,
%   responsible for grabbing arguments.  |#5| is used to determine
%   default values (or is |?| if there are none).  |#4| is the code to run.
%   |#2|~and~|#3| are functions (named after the command) that grab a single
%   argument in the input stream (|#3|~is~short).  The argument specification |#1| is
%   only used by diagnostic functions.
%    \begin{macrocode}
\cs_new:Npn \@@_start_expandable:nNNNNn #1#2#3#4#5#6
  { #6 \@@_end_expandable:NNw #5 #4 \q_@@ #2#3 }
%    \end{macrocode}
% \end{macro}
%
% \begin{macro}[EXP]{\@@_end_expandable:NNw}
% \begin{macro}[EXP,aux]{\@@_end_expandable_aux:w}
% \begin{macro}[EXP,aux]{\@@_end_expandable_aux:nNNNN}
% \begin{macro}[EXP,aux]{\@@_end_expandable_defaults:nnnNNn}
% \begin{macro}[EXP,aux]{\@@_end_expandable_defaults:nnw}
% \begin{macro}[EXP,aux]{\@@_end_expandable_defaults:nw}
%   Followed by a function |#1| to determine default values (or |?| if
%   there are no defaults), the code
%   |#2|, arguments that have been grabbed, then \cs{q_@@} and two generic
%   grabbers.  The idea to find default values is similar to the
%   non-expandable case but we cannot define an auxiliary function, so
%   at every step in the loop we need to go through all arguments
%   searching for which ones started out as |-NoValue-| and replacing
%   these by the newly computed values.  In fact we need to keep track
%   of three versions of all arguments: the original version, the
%   previous version with default values, and the currently built
%   version (first argument of \cs{@@_end_expandable_defaults:nnnNNn}).
%    \begin{macrocode}
\cs_new:Npn \@@_end_expandable:NNw #1#2
  { \@@_end_expandable_aux:w #1#2 \prg_do_nothing: }
\cs_new:Npn \@@_end_expandable_aux:w #1#2#3 \q_@@
  { \exp_args:No \@@_end_expandable_aux:nNNNN {#3} #1 #2 }
\cs_new:Npn \@@_end_expandable_aux:nNNNN #1#2#3#4#5
  {
    \token_if_eq_charcode:NNT ? #2 { \exp_after:wN \use_iv:nnnn }
    \@@_end_expandable_defaults:nnnNNn {#1} { } {#1} #2#3
      { } { } { } { } { } { } { } { } { } { }
      {
        \__msg_kernel_expandable_error:nnn
          { xparse } { loop-in-defaults } {#4}
        \use_iv:nnnn
      }
    \q_stop
  }
\cs_new:Npn \@@_end_expandable_defaults:nnnNNn #1#2#3#4#5#6
  {
    #6
    \str_if_eq:nnTF {#1} {#2}
      { \use_i_delimit_by_q_stop:nw { #5 #1 } }
      {
        \exp_args:No \@@_tl_mapthread_function:nnN
          { #4 #1 } {#3}
          \@@_end_expandable_defaults:nnw
        \@@_end_expandable_defaults:nnnNNn { } {#1} {#3} #4 #5
      }
  }
\cs_new:Npn \@@_end_expandable_defaults:nnw #1#2
  {
    \@@_if_no_value:nTF {#2}
      { \exp_args:No \@@_end_expandable_defaults:nw {#1} }
      { \@@_end_expandable_defaults:nw {#2} }
  }
\cs_new:Npn \@@_end_expandable_defaults:nw
    #1#2 \@@_end_expandable_defaults:nnnNNn #3
  { #2 \@@_end_expandable_defaults:nnnNNn { #3 {#1} } }
%    \end{macrocode}
% \end{macro}
% \end{macro}
% \end{macro}
% \end{macro}
% \end{macro}
% \end{macro}
%
% \subsection{Normalizing the argument specifications}
%
% The goal here is to expand aliases and check that the argument
% specification is valid before the main parsing run.  If it is not
% valid the entire set up is abandoned to avoid any strange internal
% errors.  A function is provided for each argument type that will grab
% any extra arguments and call the loop function.
%
% The first thing that is done in the loop is to check that the various
% argument types have the correct number of data items associated with
% them.  The opportunity is also taken to make sure that where a single
% token is required, one has actually been supplied.
%
% The second is that processors and the marker~|+| for long arguments
% must be followed by arguments.
%
% The third is to check for forbidden types for expandable commands,
% namely \texttt{G}/\texttt{v} always, and \texttt{l}/\texttt{u} after
% optional arguments (\pkg{xparse} may have inserted braces due to a
% failed search for an optional argument).
%
% The fourth is that an optional argument should not be followed by a
% mandatory argument with the same delimiter, as otherwise the optional
% argument could never be omitted.
%
% The fifth is to keep track in \cs{l_@@_some_long_bool} and
% \cs{l_@@_some_short_bool} of whether the command has some long/short
% arguments.
%
% The sixth is to keep track in \cs{l_@@_grab_expandably_bool} of
% whether all arguments are \texttt{m}/\texttt{l}/\texttt{u} type and
% short arguments appear before long ones, in which case they can be
% grabbed expandably just as safely as they could be grabbed expandably.
% Regardless of that, arguments of expandable commands will be grabbed
% expandably and arguments of environments will not (because the list of
% arguments built by non-expandable grabbing is used to pass them to the
% end-environment code).
%
% The last is to count mandatory arguments, used later to detect which
% optional arguments are trailing.
%
% \begin{macro}{\@@_normalize_arg_spec:n}
% \begin{macro}{\@@_normalize_arg_spec_loop:n}
%   Loop through the argument specification, calling an auxiliary
%   specific to each argument type.  If any argument is unknown stop the
%   definition.  After the loop, if there are more than
%   $9$ arguments, stop.  Additionally, expandable commands may not end
%   with an optional argument; this case is detected by using the fact
%   that \cs{l_@@_last_delimiters_tl} is cleared by every mandatory
%   argument and filled by every optional argument.
%    \begin{macrocode}
\cs_new_protected:Npn \@@_normalize_arg_spec:n #1
  {
    \int_zero:N \l_@@_mandatory_args_int
    \int_zero:N \l_@@_current_arg_int
    \tl_clear:N \l_@@_last_delimiters_tl
    \tl_clear:N \l_@@_arg_spec_tl
    \bool_set_true:N \l_@@_grab_expandably_bool
    \bool_set_false:N \l_@@_long_bool
    \bool_set_false:N \l_@@_some_long_bool
    \bool_set_false:N \l_@@_some_short_bool
    \@@_normalize_arg_spec_loop:n #1
      \q_recursion_tail \q_recursion_tail \q_recursion_tail \q_recursion_stop
    \int_compare:nNnT \l_@@_current_arg_int > 9
      {
        \__msg_kernel_error:nnxx { xparse } { too-many-arguments }
          { \iow_char:N \\ \l_@@_function_tl } { \tl_to_str:n {#1} }
        \@@_bad_def:wn
      }
    \bool_if:NT \l_@@_expandable_bool
      {
        \tl_if_empty:NF \l_@@_last_delimiters_tl
          {
            \__msg_kernel_error:nnxx { xparse } { expandable-ending-optional }
              { \iow_char:N \\ \l_@@_function_tl } { \tl_to_str:n {#1} }
            \@@_bad_def:wn
          }
      }
    \bool_if:NT \l_@@_expandable_bool
      { \bool_set_true:N \l_@@_grab_expandably_bool }
    \bool_if:NT \l_@@_environment_bool
      { \bool_set_false:N \l_@@_grab_expandably_bool }
  }
\cs_new_protected:Npn \@@_normalize_arg_spec_loop:n #1
  {
    \quark_if_recursion_tail_stop:n {#1}
    \int_incr:N \l_@@_current_arg_int
    \cs_if_exist_use:cF { @@_normalize_type_ \tl_to_str:n {#1} :w }
      {
        \__msg_kernel_error:nnxx { xparse } { unknown-argument-type }
          { \iow_char:N \\ \l_@@_function_tl } { \tl_to_str:n {#1} }
        \@@_bad_def:wn
      }
  }
%    \end{macrocode}
% \end{macro}
% \end{macro}
%
% \begin{macro}
%   {
%     \@@_normalize_type_d:w,
%     \@@_normalize_type_e:w,
%     \@@_normalize_type_g:w,
%     \@@_normalize_type_o:w,
%     \@@_normalize_type_O:w,
%     \@@_normalize_type_r:w,
%     \@@_normalize_type_s:w,
%   }
%   These argument types are aliases of more general ones, for example
%   with the default argument |-NoValue-|.  To easily insert that marker
%   expanded in the definitions we call \cs{@@_tmp:w} with the argument
%   |-NoValue-|.  For argument types that need additional data, check
%   that that data is present (not \cs{q_recursion_tail}) before
%   proceeding.
%    \begin{macrocode}
\cs_set_protected:Npn \@@_tmp:w #1
  {
    \cs_new_protected:Npn \@@_normalize_type_d:w ##1##2
      {
        \quark_if_recursion_tail_stop_do:nn {##2} { \@@_bad_arg_spec:wn }
        \@@_normalize_type_D:w {##1} {##2} {#1}
      }
    \cs_new_protected:Npn \@@_normalize_type_e:w ##1
      {
        \quark_if_recursion_tail_stop_do:nn {##1} { \@@_bad_arg_spec:wn }
        \@@_normalize_type_E:w {##1} { }
      }
    \cs_new_protected:Npn \@@_normalize_type_g:w
      { \@@_normalize_type_G:w {#1} }
    \cs_new_protected:Npn \@@_normalize_type_o:w
      { \@@_normalize_type_D:w [ ] {#1} }
    \cs_new_protected:Npn \@@_normalize_type_O:w
      { \@@_normalize_type_D:w [ ] }
    \cs_new_protected:Npn \@@_normalize_type_r:w ##1##2
      {
        \quark_if_recursion_tail_stop_do:nn {##2} { \@@_bad_arg_spec:wn }
        \@@_normalize_type_R:w {##1} {##2} {#1}
      }
    \cs_new_protected:Npn \@@_normalize_type_s:w
      { \@@_normalize_type_t:w * }
  }
\exp_args:No \@@_tmp:w { \c_@@_no_value_tl }
%    \end{macrocode}
% \end{macro}
%
% \begin{macro}
%   {
%     \@@_normalize_type_>:w,
%     \@@_normalize_type_+:w,
%   }
%   Check that these prefixes have arguments, namely that the next token
%   is not \cs{q_recursion_tail}, and remember to leave it after the
%   looping macro.  Processors are forbidden in expandable commands.
%   If all is good, store the prefix in the cleaned up
%   \cs{l_@@_arg_spec_tl}, and decrement the argument number as prefixes
%   do not correspond to arguments.
%    \begin{macrocode}
\cs_new_protected:cpn { @@_normalize_type_>:w } #1#2
  {
    \quark_if_recursion_tail_stop_do:nn {#2} { \@@_bad_arg_spec:wn }
    \bool_if:NT \l_@@_expandable_bool
      {
        \__msg_kernel_error:nnxx { xparse } { processor-in-expandable }
          { \iow_char:N \\ \l_@@_function_tl } { \tl_to_str:n {#1} }
        \@@_bad_def:wn
      }
    \tl_put_right:Nn \l_@@_arg_spec_tl { > {#1} }
    \int_decr:N \l_@@_current_arg_int
    \bool_set_false:N \l_@@_grab_expandably_bool
    \@@_normalize_arg_spec_loop:n {#2}
  }
\cs_new_protected:cpn { @@_normalize_type_+:w } #1
  {
    \quark_if_recursion_tail_stop_do:nn {#1} { \@@_bad_arg_spec:wn }
    \tl_put_right:Nn \l_@@_arg_spec_tl { + }
    \bool_set_true:N \l_@@_long_bool
    \int_decr:N \l_@@_current_arg_int
    \@@_normalize_arg_spec_loop:n {#1}
  }
%    \end{macrocode}
% \end{macro}
%
% \begin{macro}
%   {
%     \@@_normalize_type_D:w,
%     \@@_normalize_type_E:w,
%     \@@_normalize_type_G:w,
%     \@@_normalize_type_t:w,
%   }
% \begin{macro}[aux]{\@@_normalize_E_unique_check:w}
%   Optional argument types.  Check that all required data is present
%   (and consists of single tokens if applicable) and check for
%   forbidden types for expandable commands.  For \texttt{E}-type
%   require that there is at least one embellishment, that each one is a
%   single token, and that there aren't more optional arguments than
%   embellishments; also remember that each embellishment counts as one
%   argument for \cs{l_@@_current_arg_int}.  Then in each case
%   store the data in \cs{l_@@_arg_spec_tl}, and
%   for later checks store in \cs{l_@@_last_delimiters_tl} the tokens
%   whose presence determines whether there is an optional argument (for
%   braces store |{}|, seen later as an empty delimiter).
%    \begin{macrocode}
\cs_new_protected:Npn \@@_normalize_type_D:w #1#2#3
  {
    \@@_single_token_check:n {#1}
    \@@_single_token_check:n {#2}
    \quark_if_recursion_tail_stop_do:nn {#3} { \@@_bad_arg_spec:wn }
    \@@_add_arg_spec:n { D #1 #2 {#3} }
    \tl_put_right:Nn \l_@@_last_delimiters_tl {#1}
    \bool_set_false:N \l_@@_grab_expandably_bool
    \@@_normalize_arg_spec_loop:n
  }
\cs_new_protected:Npn \@@_normalize_type_E:w #1#2
  {
    \quark_if_recursion_tail_stop_do:nn {#2} { \@@_bad_arg_spec:wn }
    \tl_if_blank:nT {#1} { \@@_bad_arg_spec:wn }
    \tl_map_function:nN {#1} \@@_single_token_check:n
    \@@_normalize_E_unique_check:w #1 \q_nil \q_stop
    \int_compare:nNnT { \tl_count:n {#2} } > { \tl_count:n {#1} }
      { \@@_bad_arg_spec:wn }
    \@@_add_arg_spec:n { E {#1} {#2} }
    \tl_put_right:Nn \l_@@_last_delimiters_tl {#1}
    \bool_set_false:N \l_@@_grab_expandably_bool
    \int_add:Nn \l_@@_current_arg_int { \tl_count:n {#1} - 1 }
    \@@_normalize_arg_spec_loop:n
  }
\cs_new_protected:Npn \@@_normalize_E_unique_check:w #1#2 \q_stop
  {
    \quark_if_nil:NF #1
      {
        \tl_if_in:nnT {#2} {#1} { \@@_bad_arg_spec:wn }
        \@@_normalize_E_unique_check:w #2 \q_stop
      }
  }
\cs_new_protected:Npn \@@_normalize_type_G:w #1
  {
    \quark_if_recursion_tail_stop_do:nn {#1} { \@@_bad_arg_spec:wn }
    \@@_normalize_check_gv:N G
    \@@_add_arg_spec:n { G {#1} }
    \tl_put_right:Nn \l_@@_last_delimiters_tl { { } }
    \@@_normalize_arg_spec_loop:n
  }
\cs_new_protected:Npn \@@_normalize_type_t:w #1
  {
    \@@_single_token_check:n {#1}
    \quark_if_recursion_tail_stop_do:Nn #1 { \@@_bad_arg_spec:wn }
    \tl_put_right:Nn \l_@@_arg_spec_tl { t #1 }
    \tl_put_right:Nn \l_@@_last_delimiters_tl {#1}
    \bool_set_false:N \l_@@_grab_expandably_bool
    \bool_set_false:N \l_@@_long_bool
    \@@_normalize_arg_spec_loop:n
  }
%    \end{macrocode}
% \end{macro}
% \end{macro}
%
% \begin{macro}
%   {
%     \@@_normalize_type_l:w,
%     \@@_normalize_type_m:w,
%     \@@_normalize_type_R:w,
%     \@@_normalize_type_u:w
%     \@@_normalize_type_v:w
%   }
%   Mandatory arguments.  First check the required data is present,
%   consists of single tokens where applicable, and that the argument
%   type is allowed for expandable commands if applicable.  For the
%   \texttt{m} and \texttt{R} argument types check that they do not
%   follow some optional argument with that delimiter as otherwise the
%   optional argument could not be omitted.  Then save data in
%   \cs{l_@@_arg_spec_tl}, count the mandatory argument, and empty the
%   list of last delimiters.
%    \begin{macrocode}
\cs_new_protected:Npn \@@_normalize_type_l:w
  {
    \@@_normalize_check_lu:N l
    \@@_add_arg_spec:n { l }
    \int_incr:N \l_@@_mandatory_args_int
    \tl_clear:N \l_@@_last_delimiters_tl
    \@@_normalize_arg_spec_loop:n
  }
\cs_new_protected:Npn \@@_normalize_type_m:w
  {
    \@@_delimiter_check:nnn { } { m } { \iow_char:N \{ }
    \@@_add_arg_spec:n { m }
    \int_incr:N \l_@@_mandatory_args_int
    \tl_clear:N \l_@@_last_delimiters_tl
    \@@_normalize_arg_spec_loop:n
  }
\cs_new_protected:Npn \@@_normalize_type_R:w #1#2#3
  {
    \@@_single_token_check:n {#1}
    \@@_single_token_check:n {#2}
    \quark_if_recursion_tail_stop_do:nn {#3} { \@@_bad_arg_spec:wn }
    \@@_delimiter_check:nnn {#1} { R/r } { \tl_to_str:n {#1} }
    \@@_add_arg_spec:n { R #1 #2 {#3} }
    \int_incr:N \l_@@_mandatory_args_int
    \tl_clear:N \l_@@_last_delimiters_tl
    \bool_set_false:N \l_@@_grab_expandably_bool
    \@@_normalize_arg_spec_loop:n
  }
\cs_new_protected:Npn \@@_normalize_type_u:w #1
  {
    \quark_if_recursion_tail_stop_do:nn {#1} { \@@_bad_arg_spec:wn }
    \@@_normalize_check_lu:N u
    \@@_add_arg_spec:n { u {#1} }
    \int_incr:N \l_@@_mandatory_args_int
    \tl_clear:N \l_@@_last_delimiters_tl
    \@@_normalize_arg_spec_loop:n
  }
\cs_new_protected:Npn \@@_normalize_type_v:w
  {
    \@@_normalize_check_gv:N v
    \@@_add_arg_spec:n { v }
    \int_incr:N \l_@@_mandatory_args_int
    \tl_clear:N \l_@@_last_delimiters_tl
    \@@_normalize_arg_spec_loop:n
  }
%    \end{macrocode}
% \end{macro}
%
% \begin{macro}{\@@_single_token_check:n}
%   A spin-out function to check that what should be single tokens really
%   are single tokens.
%    \begin{macrocode}
\cs_new_protected:Npn \@@_single_token_check:n #1
  {
    \exp_args:Nx \tl_if_single_token:nF { \tl_trim_spaces:n {#1} }
      {
        \__msg_kernel_error:nnxx { xparse } { not-single-token }
          { \iow_char:N \\ \l_@@_function_tl } { \tl_to_str:n {#1} }
        \@@_bad_def:wn
      }
  }
%    \end{macrocode}
% \end{macro}
%
% \begin{macro}{\@@_normalize_check_gv:N, \@@_normalize_check_lu:N}
%   Called for arguments that are always forbidden, or forbidden after
%   an optional argument, for expandable commands.
%    \begin{macrocode}
\cs_new_protected:Npn \@@_normalize_check_gv:N #1
  {
    \bool_if:NT \l_@@_expandable_bool
      {
        \__msg_kernel_error:nnxx
          { xparse } { invalid-expandable-argument-type }
          { \iow_char:N \\ \l_@@_function_tl } { \tl_to_str:n {#1} }
        \@@_bad_def:wn
      }
    \bool_set_false:N \l_@@_grab_expandably_bool
  }
\cs_new_protected:Npn \@@_normalize_check_lu:N #1
  {
    \bool_if:NT \l_@@_expandable_bool
      {
        \tl_if_empty:NF \l_@@_last_delimiters_tl
          {
            \__msg_kernel_error:nnxx
              { xparse } { invalid-after-optional-expandably }
              { \iow_char:N \\ \l_@@_function_tl } { \tl_to_str:n {#1} }
            \@@_bad_def:wn
          }
      }
  }
%    \end{macrocode}
% \end{macro}
%
% \begin{macro}{\@@_delimiter_check:nnn}
%   Called for \texttt{m} and \texttt{R} arguments.  Checks that the
%   leading token does not coincide with the token denoting the presence
%   of a previous optional argument.  Instead of dealing with braces for
%   the \texttt{m}-type we use an empty delimiter to denote that case.
%    \begin{macrocode}
\cs_new_protected:Npn \@@_delimiter_check:nnn #1#2#3
  {
    \tl_map_inline:Nn \l_@@_last_delimiters_tl
      {
        \tl_if_eq:nnT {##1} {#1}
          {
            \__msg_kernel_warning:nnxx { xparse } { optional-mandatory }
              {#2} {#3}
          }
      }
  }
%    \end{macrocode}
% \end{macro}
%
% \begin{macro}{\@@_bad_arg_spec:wn, \@@_bad_def:wn}
%   If the argument specification is wrong, this provides an escape from
%   the entire definition process.
%    \begin{macrocode}
\cs_new_protected:Npn \@@_bad_arg_spec:wn #1 \@@_break_point:n #2
  {
    \__msg_kernel_error:nnxx { xparse } { bad-arg-spec }
      { \iow_char:N \\ \l_@@_function_tl } { \tl_to_str:n {#2} }
  }
\cs_new_protected:Npn \@@_bad_def:wn #1 \@@_break_point:n #2 { }
%    \end{macrocode}
% \end{macro}
%
% \begin{macro}{\@@_add_arg_spec:n}
%   When adding an argument to the argument specification, set the
%   \texttt{some_long} or \texttt{some_short} booleans as appropriate
%   and clear the boolean keeping track of whether the argument is long.
%   Before that, test for a short argument following some long
%   arguments: this is forbidden for expandable commands and prevents
%   grabbing arguments expandably.
%    \begin{macrocode}
\cs_new_protected:Npn \@@_add_arg_spec:n #1
  {
    \bool_if:NF \l_@@_long_bool
      {
        \bool_if:NT \l_@@_some_long_bool
          {
            \bool_if:NT \l_@@_expandable_bool
              {
                \__msg_kernel_error:nnx { xparse } { inconsistent-long }
                  { \iow_char:N \\ \l_@@_function_tl }
                \@@_bad_def:wn
              }
            \bool_set_false:N \l_@@_grab_expandably_bool
          }
      }
    \bool_if:NTF \l_@@_long_bool
      { \bool_set_true:N \l_@@_some_long_bool }
      { \bool_set_true:N \l_@@_some_short_bool }
    \bool_set_false:N \l_@@_long_bool
    \tl_put_right:Nn \l_@@_arg_spec_tl {#1}
  }
%    \end{macrocode}
% \end{macro}
%
% \subsection{Preparing the signature: general mechanism}
%
% \begin{macro}{\@@_prepare_signature:n}
% \begin{macro}{\@@_prepare_signature:N}
% \begin{macro}{\@@_prepare_signature_bypass:N}
%   Actually creating the signature uses the same loop approach as counting
%   up mandatory arguments. There are first a number of variables which need
%   to be set to track what is going on. Many of these variables are unused
%   when defining expandable commands.
%    \begin{macrocode}
\cs_new_protected:Npn \@@_prepare_signature:n #1
  {
    \int_zero:N \l_@@_current_arg_int
    \bool_set_false:N \l_@@_long_bool
    \int_zero:N \l_@@_m_args_int
    \bool_set_false:N \l_@@_defaults_bool
    \tl_clear:N \l_@@_defaults_tl
    \tl_clear:N \l_@@_process_all_tl
    \tl_clear:N \l_@@_process_one_tl
    \bool_set_false:N \l_@@_process_some_bool
    \tl_clear:N \l_@@_signature_tl
    \@@_prepare_signature:N #1 \q_recursion_tail \q_recursion_stop
    \bool_if:NF \l_@@_expandable_bool { \@@_flush_m_args: }
  }
%    \end{macrocode}
%  The main looping function does not take an argument, but carries out the
%  reset on the processor boolean. This is split off from the rest of the
%  process so that when actually setting up processors the flag-reset can
%  be bypassed.
%
%  For each known argument type there is an appropriate function to actually
%  do the addition to the signature. These are separate for expandable and
%  standard functions, as the approaches are different. Of course, if the type
%  is not known at all then a fall-back is needed.
%    \begin{macrocode}
\cs_new_protected:Npn \@@_prepare_signature:N
  {
    \bool_set_false:N \l_@@_prefixed_bool
    \@@_prepare_signature_bypass:N
  }
\cs_new_protected:Npn \@@_prepare_signature_bypass:N #1
  {
    \quark_if_recursion_tail_stop:N #1
    \use:c
      {
         @@_add
         \bool_if:NT \l_@@_grab_expandably_bool { _expandable }
         _type_  \token_to_str:N #1 :w
      }
  }
%    \end{macrocode}
% \end{macro}
% \end{macro}
% \end{macro}
%
% \subsection{Setting up a standard signature}
%
% Each argument-adding function appends to the signature a grabber (and
% for some types, the delimiters or default value), except the one for
% \texttt{m} arguments.  These are collected and added to the signature
% all at once by \cs{@@_flush_m_args:}, called for every other argument
% type.  All of the functions then call the loop function
% \cs{@@_prepare_signature:N}.  Default values of arguments are
% collected by \cs{@@_add_default:n} rather than being stored with the
% argument; this function and \cs{@@_add_default:} are also responsible
% for keeping track of \cs{l_@@_current_arg_int}.
%
% \begin{macro}{\@@_add_type_+:w}
%   Making the next argument long means setting the flag. The \texttt{m}
%   arguments are recorded here as
%   this has to be done for every case where there is then a long argument.
%    \begin{macrocode}
\cs_new_protected:cpn { @@_add_type_+:w }
  {
    \@@_flush_m_args:
    \bool_set_true:N \l_@@_long_bool
    \bool_set_true:N \l_@@_prefixed_bool
    \@@_prepare_signature_bypass:N
  }
%    \end{macrocode}
% \end{macro}
%
% \begin{macro}{\@@_add_type_>:w}
%   When a processor is found, the processor code is stored.  It will be
%   used by \cs{@@_args_process:} once arguments are all found. Here,
%   the loop calls \cs{@@_prepare_signature_bypass:N} rather than
%   \cs{@@_prepare_signature:N} so that the flag is not reset.
%    \begin{macrocode}
\cs_new_protected:cpn { @@_add_type_>:w } #1
  {
    \@@_flush_m_args:
    \bool_set_true:N \l_@@_prefixed_bool
    \bool_set_true:N \l_@@_process_some_bool
    \tl_put_left:Nn \l_@@_process_one_tl { {#1} }
    \@@_prepare_signature_bypass:N
  }
%    \end{macrocode}
% \end{macro}
%
% \begin{macro}{\@@_add_type_D:w}
%    \begin{macrocode}
\cs_new_protected:Npn \@@_add_type_D:w #1#2#3
  {
    \@@_flush_m_args:
    \@@_add_default:n {#3}
    \@@_add_grabber_optional:N D
    \tl_put_right:Nn \l_@@_signature_tl { #1 #2 }
    \@@_prepare_signature:N
  }
%    \end{macrocode}
% \end{macro}
%
% \begin{macro}{\@@_add_type_E:w}
%   The \texttt{E}-type argument needs a special handling of default
%   values.
%    \begin{macrocode}
\cs_new_protected:Npn \@@_add_type_E:w #1#2
  {
    \@@_flush_m_args:
    \@@_add_default_E:nn {#1} {#2}
    \@@_add_grabber_optional:N E
    \tl_put_right:Nn \l_@@_signature_tl { {#1} }
    \@@_prepare_signature:N
  }
%    \end{macrocode}
% \end{macro}
%
% \begin{macro}{\@@_add_type_G:w}
%   For the \texttt{G} type, the grabber and the default are added to the
%   signature.
%    \begin{macrocode}
\cs_new_protected:Npn \@@_add_type_G:w #1
  {
    \@@_flush_m_args:
    \@@_add_default:n {#1}
    \@@_add_grabber_optional:N G
    \@@_prepare_signature:N
  }
%    \end{macrocode}
% \end{macro}
%
% \begin{macro}{\@@_add_type_l:w}
%   Finding \texttt{l} arguments is very simple: there is nothing to do
%   other than add the grabber.
%    \begin{macrocode}
\cs_new_protected:Npn \@@_add_type_l:w
  {
    \@@_flush_m_args:
    \@@_add_default:
    \@@_add_grabber_mandatory:N l
    \@@_prepare_signature:N
  }
%    \end{macrocode}
% \end{macro}
%
% \begin{macro}{\@@_add_type_m:w}
%   The \texttt{m} type is special as short arguments which are not
%   post-processed are simply counted at this stage. Thus there is a check
%   to see if either of these cases apply. If so, a one-argument grabber
%   is added to the signature. On the other hand, if a standard short
%   argument is required it is simply counted at this stage, to be
%   added later using \cs{@@_flush_m_args:}.
%    \begin{macrocode}
\cs_new_protected:Npn \@@_add_type_m:w
  {
    \@@_add_default:
    \bool_if:NTF \l_@@_prefixed_bool
      { \@@_add_grabber_mandatory:N m }
      { \int_incr:N \l_@@_m_args_int }
    \@@_prepare_signature:N
  }
%    \end{macrocode}
% \end{macro}
%
% \begin{macro}{\@@_add_type_R:w}
%   The \texttt{R}-type argument is very similar to the \texttt{D}-type.
%    \begin{macrocode}
\cs_new_protected:Npn \@@_add_type_R:w #1#2#3
  {
    \@@_flush_m_args:
    \@@_add_default:n {#3}
    \@@_add_grabber_mandatory:N R
    \tl_put_right:Nn \l_@@_signature_tl { #1 #2 }
    \@@_prepare_signature:N
  }
%    \end{macrocode}
% \end{macro}
%
% \begin{macro}{\@@_add_type_t:w}
%   Setting up a \texttt{t} argument means collecting one token for the test,
%   and adding it along with the grabber to the signature.
%    \begin{macrocode}
\cs_new_protected:Npn \@@_add_type_t:w #1
  {
    \@@_flush_m_args:
    \@@_add_default:
    \@@_add_grabber_optional:N t
    \tl_put_right:Nn \l_@@_signature_tl {#1}
    \@@_prepare_signature:N
  }
%    \end{macrocode}
% \end{macro}
%
% \begin{macro}{\@@_add_type_u:w}
%   At the set up stage, the \texttt{u} type argument is identical to the
%   \texttt{G} type except for the name of the grabber function.
%    \begin{macrocode}
\cs_new_protected:Npn \@@_add_type_u:w #1
  {
    \@@_flush_m_args:
    \@@_add_default:
    \@@_add_grabber_mandatory:N u
    \tl_put_right:Nn \l_@@_signature_tl { {#1} }
    \@@_prepare_signature:N
  }
%    \end{macrocode}
% \end{macro}
%
% \begin{macro}{\@@_add_type_v:w}
%   At this stage, the \texttt{v} argument is identical to \texttt{l}
%   except that since the grabber may fail to read a verbatim argument
%   we need a default value.
%    \begin{macrocode}
\cs_new_protected:Npn \@@_add_type_v:w
  {
    \@@_flush_m_args:
    \exp_args:No \@@_add_default:n \c_@@_no_value_tl
    \@@_add_grabber_mandatory:N v
    \@@_prepare_signature:N
  }
%    \end{macrocode}
% \end{macro}
%
% \begin{macro}{\@@_flush_m_args:}
%   As \texttt{m} arguments are simply counted, there is a need to add
%   them to the token register in a block. As this function can only
%   be called if something other than \texttt{m} turns up, the flag can
%   be switched here. The total number of mandatory arguments which
%   remain to be added to
%   the signature is also decreased by the appropriate amount.
%    \begin{macrocode}
\cs_new_protected:Npn \@@_flush_m_args:
  {
    \int_compare:nNnT \l_@@_m_args_int > 0
      {
        \tl_put_right:Nx \l_@@_signature_tl
          { \exp_not:c { @@_grab_m_ \int_use:N \l_@@_m_args_int :w } }
        \int_sub:Nn \l_@@_mandatory_args_int { \l_@@_m_args_int }
        \tl_put_right:Nx \l_@@_process_all_tl
          { \prg_replicate:nn { \l_@@_m_args_int } { { } } }
      }
    \int_zero:N \l_@@_m_args_int
  }
%    \end{macrocode}
% \end{macro}
%
% \begin{macro}{\@@_add_grabber_mandatory:N}
% \begin{macro}{\@@_add_grabber_optional:N}
%   To keep the various checks needed in one place, adding the grabber to
%   the signature is done here. For mandatory arguments, the only question
%   is whether to add a long grabber. For optional arguments, there is
%   also a check to see if any mandatory arguments are still to be added.
%   This is used to determine whether to skip spaces or not when
%   searching for the argument.
%    \begin{macrocode}
\cs_new_protected:Npn \@@_add_grabber_mandatory:N #1
  {
    \tl_put_right:Nx \l_@@_signature_tl
      {
        \exp_not:c
          { @@_grab_ #1 \bool_if:NT \l_@@_long_bool { _long } :w }
      }
    \bool_set_false:N \l_@@_long_bool
    \tl_put_right:Nx \l_@@_process_all_tl
      { { \exp_not:o \l_@@_process_one_tl } }
    \tl_clear:N \l_@@_process_one_tl
    \int_decr:N \l_@@_mandatory_args_int
  }
\cs_new_protected:Npn \@@_add_grabber_optional:N #1
  {
    \tl_put_right:Nx \l_@@_signature_tl
      {
        \exp_not:c
          {
            @@_grab_ #1
            \bool_if:NT \l_@@_long_bool { _long }
            \int_compare:nNnF \l_@@_mandatory_args_int > 0
              { _trailing }
            :w
          }
      }
    \bool_set_false:N \l_@@_long_bool
    \tl_put_right:Nx \l_@@_process_all_tl
      { { \exp_not:o \l_@@_process_one_tl } }
    \tl_clear:N \l_@@_process_one_tl
  }
%    \end{macrocode}
% \end{macro}
% \end{macro}
%
% \begin{macro}{\@@_add_default:n, \@@_add_default:, \@@_add_default_E:nn}
%   Store the default value of an argument, or rather code that gives
%   that default value (it may involve other arguments).  This is
%   \cs{c_@@_no_value_tl} for arguments with no actual default or with
%   default |-NoValue-|; and (in a brace group) \cs{prg_do_nothing:}
%   followed by a default value for others.  For \texttt{E}-type
%   arguments, pad the defaults |#2| with some \cs{c_@@_no_value_tl}
%   until there are as many as embellishments~|#1|.  These functions are
%   also used when defining expandable commands.
%    \begin{macrocode}
\cs_new_protected:Npn \@@_add_default:n #1
  {
    \@@_if_no_value:nTF {#1}
      { \@@_add_default: }
      {
        \int_incr:N \l_@@_current_arg_int
        \bool_set_true:N \l_@@_defaults_bool
        \tl_put_right:Nn \l_@@_defaults_tl { { \prg_do_nothing: #1 } }
      }
  }
\cs_new_protected:Npn \@@_add_default:
  {
    \int_incr:N \l_@@_current_arg_int
    \tl_put_right:Nn \l_@@_defaults_tl { \c_@@_no_value_tl }
  }
\cs_new_protected:Npn \@@_add_default_E:nn #1#2
  {
    \tl_map_function:nN {#2} \@@_add_default:n
    \prg_replicate:nn
      { \tl_count:n {#1} - \tl_count:n {#2} }
      { \@@_add_default: }
  }
%    \end{macrocode}
% \end{macro}
%
% \subsection{Setting up expandable types}
%
% The approach here is not dissimilar to that for standard types, but fewer types
% are supported. There is
% also a need to define the per-function auxiliaries: this is done here, while
% the general grabbers are dealt with later.
%
% \begin{macro}{\@@_add_expandable_type_+:w}
%   We have already checked that short arguments are before long
%   arguments, so \cs{l_@@_long_bool} only changes from \texttt{false}
%   to \texttt{true} once (and there is no need to reset it after each
%   argument).  Also knock back the argument count because |+| is not an
%   argument.  Continue the loop.
%    \begin{macrocode}
\cs_new_protected:cpn { @@_add_expandable_type_+:w }
  {
    \bool_set_true:N \l_@@_long_bool
    \@@_prepare_signature:N
  }
%    \end{macrocode}
% \end{macro}
%
% \begin{macro}{\@@_add_expandable_type_D:w}
% \begin{macro}{\@@_add_expandable_type_D_aux:NNNn}
% \begin{macro}{\@@_add_expandable_type_D_aux:NNN}
% \begin{macro}{\@@_add_expandable_type_D_aux:NN}
%   The set up for \texttt{D}-type arguments involves constructing a
%   rather complex auxiliary which is used
%   repeatedly when grabbing. There is an auxiliary here so that the
%   \texttt{R}-type can share code readily: |#1| is |D| or~|R|.
%   The |_aux:NN| auxiliary is needed if the two delimiting tokens are
%   identical: in contrast to the non-expandable route, the grabber here
%   has to act differently for this case.
%    \begin{macrocode}
\cs_new_protected:Npn \@@_add_expandable_type_D:w
  { \@@_add_expandable_type_D_aux:NNNn D }
\cs_new_protected:Npn \@@_add_expandable_type_D_aux:NNNn #1#2#3#4
  {
    \@@_add_default:n {#4}
    \tl_if_eq:nnTF {#2} {#3}
      { \@@_add_expandable_type_D_aux:NN #1 #2 }
      { \@@_add_expandable_type_D_aux:NNN #1 #2 #3 }
    \@@_prepare_signature:N
  }
\cs_new_protected:Npn \@@_add_expandable_type_D_aux:NNN #1#2#3
  {
    \bool_if:NTF \l_@@_long_bool
      { \cs_set:cpx }
      { \cs_set_nopar:cpx }
      { \l_@@_expandable_aux_name_tl } ##1 ##2 #2 ##3 \q_@@ ##4 #3
      { ##1 {##2} {##3} {##4} }
    \@@_add_expandable_grabber:nn {#1}
      {
        \exp_not:c  { \l_@@_expandable_aux_name_tl }
        \exp_not:n { #2 #3 }
      }
  }
\cs_new_protected:Npn \@@_add_expandable_type_D_aux:NN #1#2
  {
    \bool_if:NTF \l_@@_long_bool
      { \cs_set:cpx }
      { \cs_set_nopar:cpx }
      { \l_@@_expandable_aux_name_tl } ##1 #2 ##2 #2
      { ##1 {##2} }
    \@@_add_expandable_grabber:nn { #1_alt }
      {
        \exp_not:c  { \l_@@_expandable_aux_name_tl }
        \exp_not:n {#2}
      }
  }
%    \end{macrocode}
% \end{macro}
% \end{macro}
% \end{macro}
% \end{macro}
%
% \begin{macro}{\@@_add_expandable_type_E:w}
% \begin{macro}[aux]{\@@_add_expandable_type_E_aux:n}
%   For each embellishment, use \cs{@@_get_grabber:NN} to obtain an
%   auxiliary delimited by that token and store a pair constituted of
%   the auxiliary and the token in \cs{l_@@_tmpb_tl}, before appending
%   the whole set of these pairs to the signature, and an equal number
%   of |-NoValue-| markers (regardless of the default values of
%   arguments).  Set the current argument appropriately.
%    \begin{macrocode}
\cs_new_protected:Npn \@@_add_expandable_type_E:w #1#2
  {
    \@@_add_default_E:nn {#1} {#2}
    \tl_clear:N \l_@@_tmpb_tl
    \tl_map_function:nN {#1} \@@_add_expandable_type_E_aux:n
    \@@_add_expandable_grabber:nn
      { E \bool_if:NT \l_@@_long_bool { _long } }
      {
        { \exp_not:o \l_@@_tmpb_tl }
        {
          \prg_replicate:nn { \tl_count:n {#1} }
            { { \c_@@_no_value_tl } }
        }
      }
    \@@_prepare_signature:N
  }
\cs_new_protected:Npn \@@_add_expandable_type_E_aux:n #1
  {
    \@@_get_grabber:NN #1 \l_@@_tmpa_tl
    \tl_put_right:Nx \l_@@_tmpb_tl
      { \exp_not:o \l_@@_tmpa_tl \exp_not:N #1 }
  }
%    \end{macrocode}
% \end{macro}
% \end{macro}
%
% \begin{macro}{\@@_add_expandable_type_l:w}
%   Reuse type \texttt{u}, thanks to the fact that \TeX{} macros whose
%   parameter text ends with |#| in fact end up being delimited by an
%   open brace.
%    \begin{macrocode}
\cs_new_protected:Npn \@@_add_expandable_type_l:w
  { \@@_add_expandable_type_u:w ## }
%    \end{macrocode}
% \end{macro}
%
% \begin{macro}{\@@_add_expandable_type_m:w}
%   Unlike the standard case, when working expandably each argument is always
%   grabbed separately.
%    \begin{macrocode}
\cs_new_protected:Npn \@@_add_expandable_type_m:w
  {
    \@@_add_default:
    \@@_add_expandable_grabber:nn
      { m \bool_if:NT \l_@@_long_bool { _long } } { }
    \@@_prepare_signature:N
  }
%    \end{macrocode}
% \end{macro}
%
% \begin{macro}{\@@_add_expandable_type_R:w}
%   The \texttt{R}-type is very similar to the \texttt{D}-type
%   argument, and so the same internals are used.
%    \begin{macrocode}
\cs_new_protected:Npn \@@_add_expandable_type_R:w
  { \@@_add_expandable_type_D_aux:NNNn R }
%    \end{macrocode}
% \end{macro}
%
% \begin{macro}{\@@_add_expandable_type_t:w}
%   An auxiliary delimited by |#1| is built now.  It will be used to
%   test for the presence of that token.
%    \begin{macrocode}
\cs_new_protected:Npn \@@_add_expandable_type_t:w #1
  {
    \@@_add_default:
    \@@_get_grabber:NN #1 \l_@@_tmpa_tl
    \@@_add_expandable_grabber:nn { t }
      {
        \exp_not:o \l_@@_tmpa_tl
        \exp_not:N #1
      }
    \@@_prepare_signature:N
  }
%    \end{macrocode}
% \end{macro}
%
% \begin{macro}{\@@_add_expandable_type_u:w}
%   Define an auxiliary that will be used directly in the signature.  It
%   grabs one argument delimited by |#1| and places it before \cs{q_@@}.
%    \begin{macrocode}
\cs_new_protected:Npn \@@_add_expandable_type_u:w #1
  {
    \@@_add_default:
    \bool_if:NTF \l_@@_long_bool
      { \cs_set:cpn }
      { \cs_set_nopar:cpn }
      { \l_@@_expandable_aux_name_tl } ##1 \q_@@ ##2 ##3 ##4 #1
      { ##1 {##4} \q_@@ ##2 ##3 }
    \@@_add_expandable_grabber:nn { u }
      { \exp_not:c  { \l_@@_expandable_aux_name_tl } }
    \@@_prepare_signature:N
  }
%    \end{macrocode}
% \end{macro}
%
% \begin{macro}{\@@_add_expandable_grabber:nn}
%   This is called for all arguments to place the right grabber in the
%   signature.
%    \begin{macrocode}
\cs_new_protected:Npn \@@_add_expandable_grabber:nn #1#2
  {
    \tl_put_right:Nx \l_@@_signature_tl
      { \exp_not:c { @@_expandable_grab_ #1 :w } #2 }
  }
%    \end{macrocode}
% \end{macro}
%
% \begin{macro}{\@@_get_grabber:NN}
% \begin{macro}[aux]{\@@_get_grabber_auxi:NN}
% \begin{macro}[aux]{\@@_get_grabber_auxii:NN}
%   Given a token |#1|, defines an expandable function delimited by that
%   token and stores it in the token list~|#2|.  The function is named
%   after the token, unless that function name is already taken by some
%   other grabber (this can happen in the rare case where delimiters
%   with different category codes are used in the same document): in
%   that case use a global counter to get a unique name.  Since the
%   grabbers are not named after \pkg{xparse} commands they should not
%   be used to get material from the input stream.
%    \begin{macrocode}
\cs_new_protected:Npn \@@_get_grabber:NN #1#2
  {
    \cs_set:Npn \@@_tmp:w ##1 #1 {##1}
    \exp_args:Nc \@@_get_grabber_auxi:NN
      { @@_grabber_ \token_to_str:N #1 :w } #2
  }
\cs_new_protected:Npn \@@_get_grabber_auxi:NN #1#2
  {
    \cs_if_eq:NNTF \@@_tmp:w #1
      { \tl_set:Nn #2 {#1} }
      {
        \cs_if_exist:NTF #1
          {
            \int_gincr:N \g_@@_grabber_int
            \exp_args:Nc \@@_get_grabber_auxi:NN
              {
                @@_grabber_
                - \int_use:N \g_@@_grabber_int :w
              }
              #2
          }
          { \@@_get_grabber_auxii:NN #1 #2 }
      }
  }
\cs_new_protected:Npn \@@_get_grabber_auxii:NN #1#2
  {
    \cs_set_eq:NN #1 \@@_tmp:w
    \tl_set:Nn #2 {#1}
  }
%    \end{macrocode}
% \end{macro}
% \end{macro}
% \end{macro}
%
% \subsection{Grabbing arguments}
%
% All of the grabbers follow the same basic pattern. The initial
% function stores in \cs{l_@@_signature_tl} the code to grab further
% arguments, defines (the function in) \cs{l_@@_fn_tl} that will grab
% the argument, and calls it.
%
% Defining \cs{l_@@_fn_tl} means determining whether to use
% \cs{cs_set:Npn} or \cs{cs_set_nopar:Npn}, and for optional arguments
% whether to skip spaces. Once the argument is found, \cs{l_@@_fn_tl}
% calls \cs{@@_add_arg:n}, responsible for calling processors and
% grabbing further arguments.
%
% \begin{macro}{\@@_grab_D:w}
% \begin{macro}{\@@_grab_D_long:w}
% \begin{macro}{\@@_grab_D_trailing:w}
% \begin{macro}{\@@_grab_D_long_trailing:w}
%   The generic delimited argument grabber. The auxiliary function does
%   a peek test before calling \cs{@@_grab_D_call:Nw}, so that the
%   optional nature of the argument works as expected.
%    \begin{macrocode}
\cs_new_protected:Npn \@@_grab_D:w #1#2#3 \@@_run_code:
  {
    \@@_grab_D_aux:NNnNn #1 #2 {#3} \cs_set_protected_nopar:Npn
      { _ignore_spaces }
  }
\cs_new_protected:Npn \@@_grab_D_long:w #1#2#3 \@@_run_code:
  {
    \@@_grab_D_aux:NNnNn #1 #2 {#3} \cs_set_protected:Npn
      { _ignore_spaces }
  }
\cs_new_protected:Npn \@@_grab_D_trailing:w #1#2#3 \@@_run_code:
  { \@@_grab_D_aux:NNnNn #1 #2 {#3} \cs_set_protected_nopar:Npn { } }
\cs_new_protected:Npn \@@_grab_D_long_trailing:w #1#2#3 \@@_run_code:
  { \@@_grab_D_aux:NNnNn #1 #2 {#3} \cs_set_protected:Npn { } }
%    \end{macrocode}
% \begin{macro}[aux]{\@@_grab_D_aux:NNnNn}
% \begin{macro}[aux]{\@@_grab_D_aux:NNnN}
%   This is a bit complicated. The idea is that, in order to check for
%   nested optional argument tokens (\texttt{[[...]]} and so on) the
%   argument needs to be grabbed without removing any braces at all. If
%   this is not done, then cases like |[{[}]| fail. So after testing for
%   an optional argument, it is collected piece-wise. Inserting a quark
%   prevents loss of braces, and there is then a test to see if there are
%   nested delimiters to handle.
%    \begin{macrocode}
\cs_new_protected:Npn \@@_grab_D_aux:NNnNn #1#2#3#4#5
  {
    \@@_grab_D_aux:NNnN #1#2 {#3} #4
    \use:c { peek_meaning_remove #5 :NTF } #1
      { \@@_grab_D_call:Nw #1 }
      { \@@_add_arg:o \c_@@_no_value_tl }
  }
%    \end{macrocode}
%   Inside the \enquote{standard} grabber, there is a test to see if the
%   grabbed argument is entirely enclosed by braces. There are a couple of
%   extra factors to allow for: the argument might be entirely empty, and
%   spaces at the start and end of the input must be retained around a brace
%   group. Also notice that a \emph{blank} argument might still contain
%   spaces.
%    \begin{macrocode}
\cs_new_protected:Npn \@@_grab_D_aux:NNnN #1#2#3#4
  {
    \tl_set:Nn \l_@@_signature_tl {#3}
    \exp_after:wN #4 \l_@@_fn_tl ##1 #2
      {
        \tl_if_in:nnTF {##1} {#1}
          { \@@_grab_D_nested:NNnN #1 #2 {##1} #4 }
          {
            \tl_if_blank:oTF { \use_none:n ##1 }
              { \@@_add_arg:o { \use_none:n ##1 } }
              {
                \str_if_eq_x:nnTF
                  { \exp_not:o { \use_none:n ##1 } }
                  { { \exp_not:o { \use_ii:nnn ##1 \q_nil } } }
                  { \@@_add_arg:o { \use_ii:nn ##1 } }
                  { \@@_add_arg:o { \use_none:n ##1 } }
              }
          }
      }
  }
%    \end{macrocode}
% \end{macro}
% \end{macro}
% \end{macro}
% \end{macro}
% \end{macro}
% \end{macro}
% \begin{macro}[aux]{\@@_grab_D_nested:NNnN}
% \begin{macro}[aux]{\@@_grab_D_nested:w}
% \begin{macro}{\l_@@_nesting_a_tl}
% \begin{macro}{\l_@@_nesting_b_tl}
% \begin{macro}{\q_@@}
%   Catching nested optional arguments means more work. The aim here is
%   to collect up each pair of optional tokens without \TeX{} helping out,
%   and without counting anything. The code above will already have
%   removed the leading opening token and a closing token, but the
%   wrong one. The aim is then to work through the material grabbed
%   so far and divide it up on each opening token, grabbing a closing
%   token to match (thus working in pairs). Once there are no opening
%   tokens, then there is a second check to see if there are any
%   opening tokens in the second part of the argument (for things
%   like |[][]|). Once everything has been found, the entire collected
%   material is added to the output as a single argument. The only tricky part
%   here is ensuring that any grabbing function that might run away is named
%   after the function currently being parsed and not after \pkg{xparse}. That
%   leads to some rather complex nesting! There is also a need to prevent the
%   loss of any braces, hence the insertion and removal of quarks along the
%   way.
%    \begin{macrocode}
\tl_new:N \l_@@_nesting_a_tl
\tl_new:N \l_@@_nesting_b_tl
\quark_new:N \q_@@
\cs_new_protected:Npn \@@_grab_D_nested:NNnN #1#2#3#4
  {
    \tl_clear:N \l_@@_nesting_a_tl
    \tl_clear:N \l_@@_nesting_b_tl
    \exp_after:wN #4 \l_@@_fn_tl ##1 #1 ##2 \q_@@ ##3 #2
      {
        \tl_put_right:No \l_@@_nesting_a_tl { \use_none:n ##1 #1 }
        \tl_put_right:No \l_@@_nesting_b_tl { \use_i:nn #2 ##3 }
        \tl_if_in:nnTF {##2} {#1}
          {
            \l_@@_fn_tl
              \q_nil ##2 \q_@@ \ERROR
          }
          {
            \tl_put_right:Nx \l_@@_nesting_a_tl
              { \@@_grab_D_nested:w \q_nil ##2 \q_stop }
            \tl_if_in:NnTF \l_@@_nesting_b_tl {#1}
              {
                \tl_set_eq:NN \l_@@_tmpa_tl \l_@@_nesting_b_tl
                \tl_clear:N \l_@@_nesting_b_tl
                \exp_after:wN \l_@@_fn_tl \exp_after:wN
                  \q_nil \l_@@_tmpa_tl \q_nil \q_@@ \ERROR
              }
              {
                \tl_put_right:No \l_@@_nesting_a_tl
                  \l_@@_nesting_b_tl
                \@@_add_arg:V \l_@@_nesting_a_tl
              }
          }
      }
    \l_@@_fn_tl #3 \q_nil \q_@@ \ERROR
  }
\cs_new:Npn \@@_grab_D_nested:w #1 \q_nil \q_stop
  { \exp_not:o { \use_none:n #1 } }
%    \end{macrocode}
% \end{macro}
% \end{macro}
% \end{macro}
% \end{macro}
% \end{macro}
% \begin{macro}{\@@_grab_D_call:Nw}
%   For \texttt{D} and \texttt{R}-type arguments, to avoid losing any
%   braces, a token needs to be inserted before the argument to be grabbed.
%   If the argument runs away because the closing token is missing then this
%   inserted token shows up in the terminal. Ideally, |#1| would therefore be
%   used directly, but that is no good as it will mess up the rest of the
%   grabber. Instead, a copy of |#1| with an altered category code is used,
%   as this will look right in the terminal but will not mess up the grabber.
%   The only issue then is that the category code of |#1| is unknown. So there
%   is a quick test to ensure that the inserted token can never be matched by
%   the grabber. (This assumes that the open and close delimiters are not the
%   same character with different category codes, but that really should not
%   happen in any sensible document-level syntax.)
%    \begin{macrocode}
\cs_set_protected_nopar:Npn \@@_grab_D_call:Nw #1
  {
    \token_if_eq_catcode:NNTF + #1
      {
        \exp_after:wN \exp_after:wN \exp_after:wN
          \l_@@_fn_tl \char_generate:nn { `#1 } { 11 }
      }
      {
        \exp_after:wN \l_@@_fn_tl
        \token_to_str:N #1
      }
  }
%    \end{macrocode}
% \end{macro}
%
% \begin{macro}
%   {
%     \@@_grab_E:w, \@@_grab_E_long:w,
%     \@@_grab_E_trailing:w, \@@_grab_E_long_trailing:w
%   }
% \begin{macro}[aux]{\@@_grab_E:nnNn}
% \begin{macro}[aux]{\@@_grab_E_loop:nnN}
% \begin{macro}[aux]{\@@_grab_E_finalise:}
%   Everything here needs to point to a loop.
%    \begin{macrocode}
\cs_new_protected:Npn \@@_grab_E:w #1#2 \@@_run_code:
  {
    \@@_grab_E:nnNn {#1} {#2}
      \cs_set_protected_nopar:Npn
      { _ignore_spaces }
  }
\cs_new_protected:Npn \@@_grab_E_long:w #1#2 \@@_run_code:
  {
    \@@_grab_E:nnNn {#1} {#2}
      \cs_set_protected:Npn
      { _ignore_spaces }
  }
\cs_new_protected:Npn \@@_grab_E_trailing:w #1#2 \@@_run_code:
  {
    \@@_grab_E:nnNn {#1} {#2}
      \cs_set_protected_nopar:Npn
      { }
  }
\cs_new_protected:Npn \@@_grab_E_long_trailing:w #1#2 \@@_run_code:
  {
    \@@_grab_E:nnNn {#1} {#2}
      \cs_set_protected:Npn
      { }
  }
%    \end{macrocode}
%   A loop is needed here to allow a random ordering of keys. These are
%   searched for one at a time, with any not found needing to be tracked:
%   they can appear later. The grabbed values are held in a property list
%   which is then turned into an ordered list to be passed back to the user.
%    \begin{macrocode}
\cs_new_protected:Npn \@@_grab_E:nnNn #1#2#3#4
  {
    \exp_after:wN #3 \l_@@_fn_tl ##1##2##3
      {
        \prop_put:Nnn \l_@@_tmp_prop {##1} {##3}
        \@@_grab_E_loop:nnN {#4} { } ##2 \q_recursion_stop
      }
    \prop_clear:N \l_@@_tmp_prop
    \tl_set:Nn \l_@@_signature_tl {#2}
    \cs_set_protected:Npn \@@_grab_E_finalise:
      {
        \tl_map_inline:nn {#1}
          {
            \prop_get:NnNF \l_@@_tmp_prop {####1} \l_@@_tmpb_tl
              { \tl_set_eq:NN \l_@@_tmpb_tl \c_@@_no_value_tl }
            \tl_put_right:Nx \l_@@_args_tl
              { { \exp_not:V \l_@@_tmpb_tl } }
          }
        \l_@@_signature_tl \@@_run_code:
      }
    \@@_grab_E_loop:nnN {#4} { } #1 \q_recursion_tail \q_recursion_stop
  }
\cs_new_protected:Npn \@@_grab_E_loop:nnN #1#2#3#4 \q_recursion_stop
  {
    \cs_if_eq:NNTF #3 \q_recursion_tail
      { \@@_grab_E_finalise: }
      {
        \use:c { peek_meaning_remove #1 :NTF } #3
          { \l_@@_fn_tl #3 {#2#4} }
          { \@@_grab_E_loop:nnN {#1} {#2#3} #4 \q_recursion_stop }
      }
  }
\cs_new_protected:Npn \@@_grab_E_finalise: { }
%    \end{macrocode}
% \end{macro}
% \end{macro}
% \end{macro}
% \end{macro}
%
% \begin{macro}{\@@_grab_G:w}
% \begin{macro}{\@@_grab_G_long:w}
% \begin{macro}{\@@_grab_G_trailing:w}
% \begin{macro}{\@@_grab_G_long_trailing:w}
% \begin{macro}[aux]{\@@_grab_G_aux:nNn}
%   Optional groups are checked by meaning, so that the same code will
%   work with, for example, Con\TeX{}t-like input.
%    \begin{macrocode}
\cs_new_protected:Npn \@@_grab_G:w #1 \@@_run_code:
  {
    \@@_grab_G_aux:nNn {#1} \cs_set_protected_nopar:Npn
      { _ignore_spaces }
  }
\cs_new_protected:Npn \@@_grab_G_long:w #1 \@@_run_code:
  {
    \@@_grab_G_aux:nNn {#1} \cs_set_protected:Npn { _ignore_spaces }
  }
\cs_new_protected:Npn \@@_grab_G_trailing:w #1 \@@_run_code:
  { \@@_grab_G_aux:nNn {#1} \cs_set_protected_nopar:Npn { } }
\cs_new_protected:Npn \@@_grab_G_long_trailing:w #1 \@@_run_code:
  { \@@_grab_G_aux:nNn {#1} \cs_set_protected:Npn { } }
\cs_new_protected:Npn \@@_grab_G_aux:nNn #1#2#3
  {
    \tl_set:Nn \l_@@_signature_tl {#1}
    \exp_after:wN #2 \l_@@_fn_tl ##1
      { \@@_add_arg:n {##1} }
    \use:c { peek_meaning #3 :NTF } \c_group_begin_token
      { \l_@@_fn_tl }
      { \@@_add_arg:o \c_@@_no_value_tl }
  }
%    \end{macrocode}
% \end{macro}
% \end{macro}
% \end{macro}
% \end{macro}
% \end{macro}
%
% \begin{macro}{\@@_grab_l:w}
% \begin{macro}{\@@_grab_l_long:w}
% \begin{macro}[aux]{\@@_grab_l_aux:nN}
%   Argument grabbers for mandatory \TeX{} arguments are pretty simple.
%    \begin{macrocode}
\cs_new_protected:Npn \@@_grab_l:w #1 \@@_run_code:
  { \@@_grab_l_aux:nN {#1} \cs_set_protected_nopar:Npn }
\cs_new_protected:Npn \@@_grab_l_long:w #1 \@@_run_code:
  { \@@_grab_l_aux:nN {#1} \cs_set_protected:Npn }
\cs_new_protected:Npn \@@_grab_l_aux:nN #1#2
  {
    \tl_set:Nn \l_@@_signature_tl {#1}
    \exp_after:wN #2 \l_@@_fn_tl ##1##
      { \@@_add_arg:n {##1} }
    \l_@@_fn_tl
  }
%    \end{macrocode}
% \end{macro}
% \end{macro}
% \end{macro}
%
% \begin{macro}{\@@_grab_m:w}
% \begin{macro}{\@@_grab_m_long:w}
%   Collecting a single mandatory argument is quite easy.
%    \begin{macrocode}
\cs_new_protected:Npn \@@_grab_m:w #1 \@@_run_code:
  {
    \tl_set:Nn \l_@@_signature_tl {#1}
    \exp_after:wN \cs_set_protected_nopar:Npn \l_@@_fn_tl ##1
      { \@@_add_arg:n {##1} }
    \l_@@_fn_tl
  }
\cs_new_protected:Npn \@@_grab_m_long:w #1 \@@_run_code:
  {
    \tl_set:Nn \l_@@_signature_tl {#1}
    \exp_after:wN \cs_set_protected:Npn \l_@@_fn_tl ##1
      { \@@_add_arg:n {##1} }
    \l_@@_fn_tl
  }
%    \end{macrocode}
% \end{macro}
% \end{macro}
%
% \begin{macro}{\@@_grab_m_1:w}
% \begin{macro}{\@@_grab_m_2:w}
% \begin{macro}{\@@_grab_m_3:w}
% \begin{macro}{\@@_grab_m_4:w}
% \begin{macro}{\@@_grab_m_5:w}
% \begin{macro}{\@@_grab_m_6:w}
% \begin{macro}{\@@_grab_m_7:w}
% \begin{macro}{\@@_grab_m_8:w}
% \begin{macro}{\@@_grab_m_9:w}
% \begin{macro}[aux]{\@@_grab_m_aux:Nnnnnnnnn}
%   Grabbing 1--8 mandatory arguments is done by giving 8--1 known
%   arguments to a 9-argument function that stores them in
%   \cs{l_@@_args_tl}.  For simplicity, grabbing 9 mandatory arguments
%   is done by grabbing 5 then 4 arguments.
%    \begin{macrocode}
\cs_new_protected_nopar:Npn \@@_grab_m_aux:Nnnnnnnnn #1#2#3#4#5#6#7#8#9
  {
    \tl_put_right:No \l_@@_args_tl
      { #1 {#2} {#3} {#4} {#5} {#6} {#7} {#8} {#9} }
    \l_@@_signature_tl \@@_run_code:
  }
\cs_new_protected:cpn { @@_grab_m_1:w } #1 \@@_run_code:
  {
    \tl_set:Nn \l_@@_signature_tl {#1}
    \exp_after:wN \cs_set_eq:NN \l_@@_fn_tl \@@_grab_m_aux:Nnnnnnnnn
    \l_@@_fn_tl \use_none:nnnnnnn { } { } { } { } { } { } { }
  }
\cs_new_protected:cpn { @@_grab_m_2:w } #1 \@@_run_code:
  {
    \tl_set:Nn \l_@@_signature_tl {#1}
    \exp_after:wN \cs_set_eq:NN \l_@@_fn_tl \@@_grab_m_aux:Nnnnnnnnn
    \l_@@_fn_tl \use_none:nnnnnn { } { } { } { } { } { }
  }
\cs_new_protected:cpn { @@_grab_m_3:w } #1 \@@_run_code:
  {
    \tl_set:Nn \l_@@_signature_tl {#1}
    \exp_after:wN \cs_set_eq:NN \l_@@_fn_tl \@@_grab_m_aux:Nnnnnnnnn
    \l_@@_fn_tl \use_none:nnnnn { } { } { } { } { }
  }
\cs_new_protected:cpn { @@_grab_m_4:w } #1 \@@_run_code:
  {
    \tl_set:Nn \l_@@_signature_tl {#1}
    \exp_after:wN \cs_set_eq:NN \l_@@_fn_tl \@@_grab_m_aux:Nnnnnnnnn
    \l_@@_fn_tl \use_none:nnnn { } { } { } { }
  }
\cs_new_protected:cpn { @@_grab_m_5:w } #1 \@@_run_code:
  {
    \tl_set:Nn \l_@@_signature_tl {#1}
    \exp_after:wN \cs_set_eq:NN \l_@@_fn_tl \@@_grab_m_aux:Nnnnnnnnn
    \l_@@_fn_tl \use_none:nnn { } { } { }
  }
\cs_new_protected:cpn { @@_grab_m_6:w } #1 \@@_run_code:
  {
    \tl_set:Nn \l_@@_signature_tl {#1}
    \exp_after:wN \cs_set_eq:NN \l_@@_fn_tl \@@_grab_m_aux:Nnnnnnnnn
    \l_@@_fn_tl \use_none:nn { } { }
  }
\cs_new_protected:cpn { @@_grab_m_7:w } #1 \@@_run_code:
  {
    \tl_set:Nn \l_@@_signature_tl {#1}
    \exp_after:wN \cs_set_eq:NN \l_@@_fn_tl \@@_grab_m_aux:Nnnnnnnnn
    \l_@@_fn_tl \use_none:n { }
  }
\cs_new_protected:cpn { @@_grab_m_8:w } #1 \@@_run_code:
  {
    \tl_set:Nn \l_@@_signature_tl {#1}
    \exp_after:wN \cs_set_eq:NN \l_@@_fn_tl \@@_grab_m_aux:Nnnnnnnnn
    \l_@@_fn_tl \prg_do_nothing:
  }
\cs_new_protected:cpx { @@_grab_m_9:w }
  {
    \exp_not:c { @@_grab_m_5:w }
    \exp_not:c { @@_grab_m_4:w }
  }
%    \end{macrocode}
% \end{macro}
% \end{macro}
% \end{macro}
% \end{macro}
% \end{macro}
% \end{macro}
% \end{macro}
% \end{macro}
% \end{macro}
% \end{macro}
%
% \begin{macro}{\@@_grab_R:w, \@@_grab_R_long:w}
% \begin{macro}[aux]{\@@_grab_R_aux:NNnN}
%  The grabber for \texttt{R}-type arguments is basically the same as
%  that for \texttt{D}-type ones, but always skips spaces (as it is mandatory)
%  and has a hard-coded error message.
%    \begin{macrocode}
\cs_new_protected:Npn \@@_grab_R:w #1#2#3 \@@_run_code:
  { \@@_grab_R_aux:NNnN #1 #2 {#3} \cs_set_protected_nopar:Npn }
\cs_new_protected:Npn \@@_grab_R_long:w #1#2#3 \@@_run_code:
  { \@@_grab_R_aux:NNnN #1 #2 {#3} \cs_set_protected:Npn }
\cs_new_protected:Npn \@@_grab_R_aux:NNnN #1#2#3#4
  {
    \@@_grab_D_aux:NNnN #1 #2 {#3} #4
    \peek_meaning_remove_ignore_spaces:NTF #1
      { \@@_grab_D_call:Nw #1 }
      {
        \__msg_kernel_error:nnxx { xparse } { missing-required }
          { \exp_after:wN \token_to_str:N \l_@@_fn_tl }
          { \token_to_str:N #1 }
        \@@_add_arg:o \c_@@_no_value_tl
      }
  }
%    \end{macrocode}
% \end{macro}
% \end{macro}
%
% \begin{macro}{\@@_grab_t:w}
% \begin{macro}{\@@_grab_t_long:w}
% \begin{macro}{\@@_grab_t_trailing:w}
% \begin{macro}{\@@_grab_t_long_trailing:w}
% \begin{macro}[aux]{\@@_grab_t_aux:NNw}
%   Dealing with a token is quite easy. Check the match, remove the
%   token if needed and add a flag to the output.
%    \begin{macrocode}
\cs_new_protected:Npn \@@_grab_t:w
  { \@@_grab_t_aux:NNw \peek_meaning_remove_ignore_spaces:NTF }
\cs_new_eq:NN \@@_grab_t_long:w \@@_grab_t:w
\cs_new_protected:Npn \@@_grab_t_trailing:w
  { \@@_grab_t_aux:NNw \peek_meaning_remove:NTF }
\cs_new_eq:NN \@@_grab_t_long_trailing:w \@@_grab_t_trailing:w
\cs_new_protected:Npn \@@_grab_t_aux:NNw #1#2#3 \@@_run_code:
  {
    \tl_set:Nn \l_@@_signature_tl {#3}
    \exp_after:wN \cs_set_protected:Npn \l_@@_fn_tl
      {
        #1 #2
          { \@@_add_arg:n { \BooleanTrue } }
          { \@@_add_arg:n { \BooleanFalse } }
      }
    \l_@@_fn_tl
  }
%    \end{macrocode}
% \end{macro}
% \end{macro}
% \end{macro}
% \end{macro}
% \end{macro}
%
% \begin{macro}{\@@_grab_u:w}
% \begin{macro}{\@@_grab_u_long:w}
% \begin{macro}[aux]{\@@_grab_u_aux:nnN}
%   Grabbing up to a list of tokens is quite easy: define the grabber,
%   and then collect.
%    \begin{macrocode}
\cs_new_protected:Npn \@@_grab_u:w #1#2 \@@_run_code:
  { \@@_grab_u_aux:nnN {#1} {#2} \cs_set_protected_nopar:Npn }
\cs_new_protected:Npn \@@_grab_u_long:w #1#2 \@@_run_code:
  { \@@_grab_u_aux:nnN {#1} {#2} \cs_set_protected:Npn }
\cs_new_protected:Npn \@@_grab_u_aux:nnN #1#2#3
  {
    \tl_set:Nn \l_@@_signature_tl {#2}
    \exp_after:wN #3 \l_@@_fn_tl ##1 #1
      { \@@_add_arg:n {##1} }
    \l_@@_fn_tl
  }
%    \end{macrocode}
% \end{macro}
% \end{macro}
% \end{macro}
%
% \begin{macro}{\@@_grab_v:w}
% \begin{macro}{\@@_grab_v_long:w}
% \begin{macro}{\@@_grab_v_aux:w}
% \begin{macro}{\@@_grab_v_group_end:}
% \begin{variable}{\l_@@_v_arg_tl}
%   The opening delimiter is the first non-space token, and is never
%   read verbatim.  This is required by consistency with the case where
%   the preceding argument was optional and absent: then \TeX{} has
%   already read and tokenized that token when looking for the optional
%   argument.  The first thing is thus to check is that this delimiter
%   is a character, and to distinguish the case of a left brace (in that
%   case, \cs{group_align_safe_end:} is needed to compensate for the
%   begin-group character that was just seen).  Then set verbatim
%   catcodes with \cs{@@_grab_v_aux_catcodes:}.
%
%   The group keep catcode changes local, and
%   \cs{group_align_safe_begin/end:} allow to use a character
%   with category code~$4$ (normally |&|) as the delimiter.
%   It is ended by \cs{@@_grab_v_group_end:}, which smuggles
%   the collected argument out of the group.
%    \begin{macrocode}
\tl_new:N \l_@@_v_arg_tl
\cs_new_protected:Npn \@@_grab_v:w
  {
    \bool_set_false:N \l_@@_long_bool
    \@@_grab_v_aux:w
  }
\cs_new_protected:Npn \@@_grab_v_long:w
  {
    \bool_set_true:N \l_@@_long_bool
    \@@_grab_v_aux:w
  }
\cs_new_protected:Npn \@@_grab_v_aux:w #1 \@@_run_code:
  {
    \tl_set:Nn \l_@@_signature_tl {#1}
    \group_begin:
      \group_align_safe_begin:
        \tex_escapechar:D = 92 \scan_stop:
        \tl_clear:N \l_@@_v_arg_tl
        \peek_meaning_remove_ignore_spaces:NTF \c_group_begin_token
          {
            \group_align_safe_end:
            \@@_grab_v_bgroup:
          }
          {
            \peek_N_type:TF
              { \@@_grab_v_aux_test:N }
              { \@@_grab_v_aux_abort:n { } }
          }
  }
\cs_new_protected:Npn \@@_grab_v_group_end:
  {
        \group_align_safe_end:
        \exp_args:NNNo
      \group_end:
    \tl_set:Nn \l_@@_v_arg_tl { \l_@@_v_arg_tl }
  }
%    \end{macrocode}
% \end{variable}
% \end{macro}
% \end{macro}
% \end{macro}
% \end{macro}
%
% \begin{macro}{\@@_grab_v_aux_test:N}
% \begin{macro}
%   {
%     \@@_grab_v_aux_loop:N,
%     \@@_grab_v_aux_loop:NN,
%     \@@_grab_v_aux_loop_end:
%   }
%   Check that the opening delimiter is a character, setup category codes,
%   then start reading tokens one by one, keeping the delimiter as an argument.
%   If the verbatim was not nested, we will be grabbing one character
%   at each step. Unfortunately, it can happen that what follows the
%   verbatim argument is already tokenized. Thus, we check at each step
%   that the next token is indeed a \enquote{nice}
%   character, \emph{i.e.}, is not a character with
%   category code $1$ (begin-group), $2$ (end-group)
%   or $6$ (macro parameter), nor the space character,
%   with category code~$10$ and character code~$32$,
%   nor a control sequence.
%   The partially built argument is stored in \cs{l_@@_v_arg_tl}.
%   If we ever meet a token which we cannot grab (non-N-type),
%   or which is not a character according to
%   \cs{@@_grab_v_token_if_char:NTF}, then we bail out with
%   \cs{@@_grab_v_aux_abort:n}. Otherwise, we stop at the first
%   character matching the delimiter.
%    \begin{macrocode}
\cs_new_protected:Npn \@@_grab_v_aux_test:N #1
  {
    \@@_grab_v_token_if_char:NTF #1
      {
        \@@_grab_v_aux_put:N #1
        \@@_grab_v_aux_catcodes:
        \@@_grab_v_aux_loop:N #1
      }
      { \@@_grab_v_aux_abort:n {#1} #1 }
  }
\cs_new_protected:Npn \@@_grab_v_aux_loop:N #1
  {
    \peek_N_type:TF
      { \@@_grab_v_aux_loop:NN #1 }
      { \@@_grab_v_aux_abort:n { } }
  }
\cs_new_protected:Npn \@@_grab_v_aux_loop:NN #1#2
  {
    \@@_grab_v_token_if_char:NTF #2
      {
        \token_if_eq_charcode:NNTF #1 #2
          { \@@_grab_v_aux_loop_end: }
          {
            \@@_grab_v_aux_put:N #2
            \@@_grab_v_aux_loop:N #1
          }
      }
      { \@@_grab_v_aux_abort:n {#2} #2 }
  }
\cs_new_protected:Npn \@@_grab_v_aux_loop_end:
  {
    \@@_grab_v_group_end:
    \@@_add_arg:x { \tl_tail:N \l_@@_v_arg_tl }
  }
%    \end{macrocode}
% \end{macro}
% \end{macro}
%
% \begin{macro}{\@@_grab_v_bgroup:}
% \begin{macro}{\@@_grab_v_bgroup_loop:}
% \begin{macro}{\@@_grab_v_bgroup_loop:N}
% \begin{variable}{\l_@@_v_nesting_int}
%   If the opening delimiter is a left brace, we keep track of
%   how many left and right braces were encountered so far in
%   \cs{l_@@_v_nesting_int} (the methods used for optional
%   arguments cannot apply here), and stop as soon as it reaches~$0$.
%
%   Some care was needed when removing the opening delimiter, which
%   has already been assigned category code~$1$: using
%   \cs{peek_meaning_remove:NTF} in the \cs{@@_grab_v_aux:w}
%   function would break within alignments. Instead, we first
%   convert that token to a string, and remove the result as a
%   normal undelimited argument.
%    \begin{macrocode}
\int_new:N \l_@@_v_nesting_int
\cs_new_protected:Npx \@@_grab_v_bgroup:
  {
    \exp_not:N \@@_grab_v_aux_catcodes:
    \exp_not:n { \int_set:Nn \l_@@_v_nesting_int { 1 } }
    \exp_not:N \@@_grab_v_aux_put:N \iow_char:N \{
    \exp_not:N \@@_grab_v_bgroup_loop:
  }
\cs_new_protected:Npn \@@_grab_v_bgroup_loop:
  {
    \peek_N_type:TF
      { \@@_grab_v_bgroup_loop:N }
      { \@@_grab_v_aux_abort:n { } }
  }
\cs_new_protected:Npn \@@_grab_v_bgroup_loop:N #1
  {
    \@@_grab_v_token_if_char:NTF #1
      {
        \token_if_eq_charcode:NNTF \c_group_end_token #1
          {
            \int_decr:N \l_@@_v_nesting_int
            \int_compare:nNnTF \l_@@_v_nesting_int > 0
              {
                \@@_grab_v_aux_put:N #1
                \@@_grab_v_bgroup_loop:
              }
              { \@@_grab_v_aux_loop_end: }
          }
          {
            \token_if_eq_charcode:NNT \c_group_begin_token #1
              { \int_incr:N \l_@@_v_nesting_int }
            \@@_grab_v_aux_put:N #1
            \@@_grab_v_bgroup_loop:
          }
      }
      { \@@_grab_v_aux_abort:n {#1} #1 }
  }
%    \end{macrocode}
% \end{variable}
% \end{macro}
% \end{macro}
% \end{macro}
%
% \begin{macro}{\@@_grab_v_aux_catcodes:}
% \begin{macro}{\@@_grab_v_aux_abort:n}
%   In a standalone format, the list of special characters is kept
%   as a sequence, \cs{c_@@_special_chars_seq}, and we use
%   \tn{dospecials} in package mode.
%   The approach for short verbatim arguments is to make the end-line
%   character a macro parameter character: this is forbidden by the
%   rest of the code. Then the error branch can check what caused the
%   bail out and give the appropriate error message.
%    \begin{macrocode}
\cs_new_protected:Npn \@@_grab_v_aux_catcodes:
  {
%<*initex>
    \seq_map_function:NN
      \c_@@_special_chars_seq
      \char_set_catcode_other:N
%</initex>
%<*package>
    \cs_set_eq:NN \do \char_set_catcode_other:N
    \dospecials
%</package>
    \tex_endlinechar:D = `\^^M \scan_stop:
    \bool_if:NTF \l_@@_long_bool
      { \char_set_catcode_other:n { \tex_endlinechar:D } }
      { \char_set_catcode_parameter:n { \tex_endlinechar:D } }
  }
\cs_new_protected:Npn \@@_grab_v_aux_abort:n #1
  {
    \@@_grab_v_group_end:
    \exp_after:wN \exp_after:wN \exp_after:wN
      \peek_meaning_remove:NTF \char_generate:nn { \tex_endlinechar:D } { 6 }
      {
        \__msg_kernel_error:nnxxx { xparse } { verbatim-newline }
          { \exp_after:wN \token_to_str:N \l_@@_fn_tl }
          { \tl_to_str:N \l_@@_v_arg_tl }
          { \tl_to_str:n {#1} }
        \@@_add_arg:o \c_@@_no_value_tl
      }
      {
        \__msg_kernel_error:nnxxx { xparse } { verbatim-tokenized }
          { \exp_after:wN \token_to_str:N \l_@@_fn_tl }
          { \tl_to_str:N \l_@@_v_arg_tl }
          { \tl_to_str:n {#1} }
        \@@_add_arg:o \c_@@_no_value_tl
      }
  }
%    \end{macrocode}
% \end{macro}
% \end{macro}
%
% \begin{macro}{\@@_grab_v_aux_put:N}
%   Storing one token in the collected argument. Most tokens are
%   converted to category code $12$, with the exception of active
%   characters, and spaces (not sure what should be done for those).
%    \begin{macrocode}
\cs_new_protected:Npn \@@_grab_v_aux_put:N #1
  {
    \tl_put_right:Nx \l_@@_v_arg_tl
      {
        \token_if_active:NTF #1
          { \exp_not:N #1 } { \token_to_str:N #1 }
      }
  }
%    \end{macrocode}
% \end{macro}
%
% \begin{macro}{\@@_grab_v_token_if_char:NTF}
%   This function assumes that the escape character is printable.
%   Then the string representation of control sequences is at least
%   two characters, and \cs{str_tail:n} only removes the escape
%   character. Macro parameter characters are doubled by
%   \cs{tl_to_str:n}, and will also yield a non-empty result,
%   hence are not considered as characters.
%    \begin{macrocode}
\cs_new_protected:Npn \@@_grab_v_token_if_char:NTF #1
  { \str_if_eq_x:nnTF { } { \str_tail:n {#1} } }
%    \end{macrocode}
% \end{macro}
%
% \begin{macro}{\@@_add_arg:n, \@@_add_arg:V, \@@_add_arg:o, \@@_add_arg:x}
%   When an argument is found it is stored, then further arguments are
%   grabbed by calling \cs{l_@@_signature_tl}.
%    \begin{macrocode}
\cs_new_protected:Npn \@@_add_arg:n #1
  {
    \tl_put_right:Nn \l_@@_args_tl { {#1} }
    \l_@@_signature_tl \@@_run_code:
  }
\cs_generate_variant:Nn \@@_add_arg:n { V , o , x }
%    \end{macrocode}
% \end{macro}
%
% \subsection{Grabbing arguments expandably}
%
% \begin{macro}[EXP]{\@@_expandable_grab_D:w}
% \begin{macro}[EXP, aux]{\@@_expandable_grab_D:NNNwNNn}
% \begin{macro}[EXP, aux]{\@@_expandable_grab_D:NNNwNNnnn}
% \begin{macro}[EXP, aux]{\@@_expandable_grab_D:Nw}
% \begin{macro}[EXP, aux]{\@@_expandable_grab_D:nnNNNwNN}
%   The first step is to grab the first token or group. The generic grabbers
%   \cs{\meta{function}}\verb*| | and \cs{\meta{function}}\verb*| | are just after \cs{q_@@}, we go and find
%   them (and use the long one).
%    \begin{macrocode}
\cs_new:Npn \@@_expandable_grab_D:w #1 \q_@@ #2#3
  { #2 { \@@_expandable_grab_D:NNNwNNn #1 \q_@@ #2 #3 } }
%    \end{macrocode}
%   We then wish to test whether |#7|, which we just grabbed, is exactly |#2|.
%   A preliminary test is whether their string representations coincide, then
%   expand the only grabber function we have, |#1|, once: the two strings below
%   are equal if and only if |#7| matches |#2| exactly.\footnote{It is obvious
%   that if \texttt{\#7} matches \texttt{\#2} then the strings are equal. We
%   must check the converse. The right-hand-side of \cs{str_if_eq:onTF} does
%   not end with \texttt{\#3}, implying that the grabber function took
%   everything as its arguments. The first brace group can only be empty if
%   \texttt{\#7} starts with \texttt{\#2}, otherwise the brace group preceding
%   \texttt{\#7} would not vanish. The third brace group is empty, thus the
%   \cs{q_@@} that was used by our grabber \texttt{\#1} must be the one
%   that we inserted (not some token in \texttt{\#7}), hence the second brace
%   group contains the end of \texttt{\#7} followed by \texttt{\#2}. Since this
%   is \texttt{\#2} on the right-hand-side, and no brace can be lost there,
%   \texttt{\#7} must contain nothing else than its leading \texttt{\#2}.}
%   The preliminary test is needed as |#7| could validly contain
%   \tn{par} (because a later mandatory argument could be long) and our
%   grabber may be short.  If
%   |#7| does not match |#2|, then the optional argument is missing, we use the
%   default |-NoValue-|, and put back the argument |#7| in the input stream.
%
%   If it does match, then interesting things need to be done. We will grab the
%   argument piece by piece, with the following pattern:
%   \begin{quote}
%     \meta{grabber} \Arg{tokens} \\
%     ~~\cs{q_nil} \Arg{piece 1} \meta{piece 2} \cs{ERROR} \cs{q_@@}\\
%     ~~\cs{q_nil} \meta{input stream}
%   \end{quote}
%   The \meta{grabber} will find an opening delimiter in \meta{piece 2}, take
%   the \cs{q_@@} as a second delimiter, and find more material delimited
%   by the closing delimiter in the \meta{input stream}. We then move the part
%   before the opening delimiter from \meta{piece 2} to \meta{piece 1}, and the
%   material taken from the \meta{input stream} to the \meta{piece 2}. Thus,
%   the argument moves gradually from the \meta{input stream} to the
%   \meta{piece 2}, then to the \meta{piece 1} when we have made sure to find
%   all opening and closing delimiters. This two-step process ensures that
%   nesting works: the number of opening delimiters minus closing delimiters in
%   \meta{piece 1} is always equal to the number of closing delimiters in
%   \meta{piece 2}. We stop grabbing arguments once the \meta{piece 2} contains
%   no opening delimiter any more, hence the balance is reached, and the final
%   argument is \meta{piece 1} \meta{piece 2}.
%   The indirection via \cs{@@_tmp:w} allows to insert |-NoValue-| expanded.
%    \begin{macrocode}
\cs_set_protected:Npn \@@_tmp:w #1
  {
    \cs_new:Npn \@@_expandable_grab_D:NNNwNNn ##1##2##3##4 \q_@@ ##5##6##7
      {
        \str_if_eq:nnTF {##2} {##7}
          {
            \str_if_eq:onTF
              { ##1 { } { } ##7 ##2 \q_@@ ##3 }
              { { } {##2} { } }
          }
          { \use_ii:nn }
          {
            ##1
              { \@@_expandable_grab_D:NNNwNNnnn ##1##2##3##4 \q_@@ ##5##6 }
              \q_nil { } ##2 \ERROR \q_@@ \ERROR
          }
          { ##4 {#1} \q_@@ ##5 ##6 {##7} }
      }
  }
\exp_args:No \@@_tmp:w { \c_@@_no_value_tl }
%    \end{macrocode}
%   At this stage, |#7| is \cs{q_nil} \Arg{piece 1} \meta{more for piece 1},
%   and we want to concatenate all that, removing \cs{q_nil}, and keeping the
%   opening delimiter |#2|. Simply use \cs{use_ii:nn}. Also, |#8| is
%   \meta{remainder of piece 2} \cs{ERROR}, and |#9| is \cs{ERROR} \meta{more
%   for piece 2}. We concatenate those, replacing the two \cs{ERROR} by the
%   closing delimiter |#3|.
%    \begin{macrocode}
\cs_new:Npn \@@_expandable_grab_D:NNNwNNnnn #1#2#3#4 \q_@@ #5#6#7#8#9
  {
    \exp_args:Nof \@@_expandable_grab_D:nnNNNwNN
      { \use_ii:nn #7 #2 }
      { \@@_expandable_grab_D:Nw #3 \exp_stop_f: #8 #9 }
    #1#2#3 #4 \q_@@ #5 #6
  }
\cs_new:Npn \@@_expandable_grab_D:Nw #1#2 \ERROR \ERROR { #2 #1 }
%    \end{macrocode}
%   Armed with our two new \meta{pieces}, we are ready to loop. However, we
%   must first see if \meta{piece 2} (here |#2|) contains any opening
%   delimiter |#4|. Again, we expand |#3|, this time removing its whole output
%   with \cs{use_none:nnn}. The test is similar to \cs{tl_if_in:nnTF}. The
%   token list is empty if and only if |#2| does not contain the opening
%   delimiter. In that case, we are done, and put the argument (from which we
%   remove a spurious pair of delimiters coming from how we started the loop).
%   Otherwise, we go back to looping with
%   \cs{@@_expandable_grab_D:NNNwNNnnn}. The code to deal with brace stripping
%   is much the same as for the non-expandable case.
%    \begin{macrocode}
\cs_new:Npn \@@_expandable_grab_D:nnNNNwNN #1#2#3#4#5#6 \q_@@ #7#8
  {
    \exp_args:No \tl_if_empty:oTF
      { #3 { \use_none:nnn } #2 \q_@@ #5 #4 \q_@@ #5 }
      {
        \tl_if_blank:oTF { \use_none:nn #1#2 }
          { \@@_put_arg_expandable:ow { \use_none:nn #1#2 } }
          {
            \str_if_eq_x:nnTF
              { \exp_not:o { \use_none:nn #1#2 } }
              { { \exp_not:o { \use_iii:nnnn #1#2 \q_nil } } }
              { \@@_put_arg_expandable:ow { \use_iii:nnn #1#2 } }
              { \@@_put_arg_expandable:ow { \use_none:nn #1#2 } }
          }
            #6 \q_@@ #7 #8
      }
      {
        #3
          { \@@_expandable_grab_D:NNNwNNnnn #3#4#5#6 \q_@@ #7 #8 }
          \q_nil {#1} #2 \ERROR \q_@@ \ERROR
      }
  }
%    \end{macrocode}
% \end{macro}
% \end{macro}
% \end{macro}
% \end{macro}
% \end{macro}
%
% \begin{macro}[EXP]{\@@_expandable_grab_D_alt:w}
% \begin{macro}[EXP]{\@@_expandable_grab_D_alt:NNwNNn}
% \begin{macro}[EXP]{\@@_expandable_grab_D_alt:Nwn}
%   When the delimiters are identical, nesting is not possible and a simplified
%   approach is used. The test concept here is the same as for the case where
%   the delimiters are different but there cannot be any nesting.
%    \begin{macrocode}
\cs_new:Npn \@@_expandable_grab_D_alt:w #1 \q_@@ #2#3
  { #2 { \@@_expandable_grab_D_alt:NNwNNn #1 \q_@@ #2 #3 } }
\cs_set_protected:Npn \@@_tmp:w #1
  {
    \cs_new:Npn \@@_expandable_grab_D_alt:NNwNNn ##1##2##3 \q_@@ ##4##5##6
      {
        \str_if_eq:nnTF {##6} {##2}
          {
            \str_if_eq:onTF
              { ##1 { } ##6 ##2 ##2 }
              { { } ##2 }
          }
          { \use_ii:nn }
          {
            ##1
              { \@@_expandable_grab_D_alt:NNwn ##4 ##5 ##3 \q_@@ }
              ##6 \ERROR
          }
          { ##3 {#1} \q_@@ ##4 ##5 {##6} }
      }
  }
\exp_args:No \@@_tmp:w { \c_@@_no_value_tl }
\cs_new:Npn \@@_expandable_grab_D_alt:NNwn #1#2#3 \q_@@ #4
  {
    \tl_if_blank:oTF { \use_none:n #4 }
      { \@@_put_arg_expandable:ow { \use_none:n #4 } }
      {
        \str_if_eq_x:nnTF
          { \exp_not:o { \use_none:n #4 } }
          { { \exp_not:o { \use_ii:nnn #4 \q_nil } } }
          { \@@_put_arg_expandable:ow { \use_ii:nn #4 } }
          { \@@_put_arg_expandable:ow { \use_none:n #4 } }
      }
        #3 \q_@@ #1 #2
  }
%    \end{macrocode}
% \end{macro}
% \end{macro}
% \end{macro}
%
% \begin{macro}[EXP]{\@@_expandable_grab_E:w, \@@_expandable_grab_E_long:w}
% \begin{macro}[EXP,aux]{\@@_expandable_grab_E_aux:w}
% \begin{macro}[EXP,aux]{\@@_expandable_grab_E_test:nnw}
% \begin{macro}[EXP,aux]{\@@_expandable_grab_E_loop:nnnNNw}
% \begin{macro}[EXP,aux]{\@@_expandable_grab_E_find:w}
% \begin{macro}[EXP,aux]{\@@_expandable_grab_E_find:nnw}
% \begin{macro}[EXP,aux]{\@@_expandable_grab_E_end:nnw}
%   We keep track of long/short by placing the appropriate grabber as
%   the third token after \cs{q_@@}; it is eventually removed by the
%   \texttt{end:nnw} auxiliary.  The \texttt{aux:w} auxiliary will be
%   called repeatedly with two arguments: the set
%   of pairs \meta{parser} \meta{token}, and the set of arguments found
%   so far (initially all |{-NoValue-}|).  At each step, grab what
%   follows in the input stream then call the \texttt{loop:nnnNNw}
%   auxiliary to compare it with each possible embellishment in turn.
%   This auxiliary's |#1| is what was found in the input, |#2| collects
%   \meta{parser} \meta{token} pairs that did not match, |#3| collects
%   the corresponding arguments found previously, |#4| and |#5| is the
%   current pair, |#6| is the remaining pairs, |#7| is empty or two
%   \cs{q_nil}, and |#8| is the current argument.  If none of the pairs
%   matched (determined by \cs{quark_if_nil:NTF}) then call the
%   \texttt{end} auxiliary to stop looking for embellishments,
%   remembering to put what was grabbed in the input back where it
%   belongs, and storing the arguments found just before \cs{q_@@}.  If
%   the current argument |#8| is not |-NoValue-| or if the input |#1|
%   does not match |#5| (see \texttt{t}-type arguments below for a
%   similar \cs{str_if_eq:onTF} test) then carry on the loop.
%   Otherwise, we found a new embellishment: grab the corresponding
%   argument in the input using the \texttt{find:w} auxiliary.  To avoid
%   losing braces around that auxiliary's argument we include a
%   space, which will be eliminated in the next loop through
%   embellishments.
%    \begin{macrocode}
\cs_new:Npn \@@_expandable_grab_E:w #1 \q_@@ #2#3
  { \@@_expandable_grab_E_aux:w #1 \q_@@ #2 #3 #3 }
\cs_new:Npn \@@_expandable_grab_E_long:w #1 \q_@@ #2#3
  { \@@_expandable_grab_E_aux:w #1 \q_@@ #2 #3 #2 }
\cs_new:Npn \@@_expandable_grab_E_aux:w #1 \q_@@ #2#3#4
  { #2 { \@@_expandable_grab_E_test:nnw #1 \q_@@ #2 #3 #4 } }
\cs_new:Npn \@@_expandable_grab_E_test:nnw #1#2#3 \q_@@ #4#5#6#7
  {
    \@@_expandable_grab_E_loop:nnnNNw {#7} { } { }
      #1 \q_nil \q_nil \q_nil \q_mark #2 \q_nil
    #3 \q_@@ #4 #5 #6
  }
\cs_new:Npn \@@_expandable_grab_E_loop:nnnNNw
    #1#2#3#4#5#6 \q_nil #7 \q_mark #8
  {
    \quark_if_nil:NTF #4
      { \@@_expandable_grab_E_end:nnw {#1} {#3} }
      {
        \@@_if_no_value:nTF {#8}
          { \str_if_eq:onTF { #4 { } #1 #5 } {#5} }
          { \use_ii:nn }
            { \@@_expandable_grab_E_find:w { #2 #4 #5 #6 } {#3} ~ }
            {
              \@@_expandable_grab_E_loop:nnnNNw
                {#1} { #2 #4 #5 } { #3 {#8} }
                #6 \q_nil #7 \q_mark
            }
      }
  }
\cs_new:Npn \@@_expandable_grab_E_find:w #1 \q_@@ #2#3#4
  { #4 { \@@_expandable_grab_E_find:nnw #1 \q_@@ #2 #3 #4 } }
\cs_new:Npn \@@_expandable_grab_E_find:nnw #1#2#3 \q_nil #4 \q_@@ #5#6#7#8
  { \@@_expandable_grab_E_aux:w {#1} { #2 {#8} #3 } #4 \q_@@ #5 #6 #7 }
\cs_new:Npn \@@_expandable_grab_E_end:nnw #1#2#3 \q_@@ #4#5#6
  { #3 #2 \q_@@ #4 #5 {#1} }
%    \end{macrocode}
% \end{macro}
% \end{macro}
% \end{macro}
% \end{macro}
% \end{macro}
% \end{macro}
% \end{macro}
%
% \begin{macro}[EXP]{\@@_expandable_grab_m:w, \@@_expandable_grab_m_long:w}
% \begin{macro}[EXP, aux]{\@@_expandable_grab_m_aux:wNn}
%   The mandatory case is easy: find the auxiliary after the \cs{q_@@},
%   and use it directly to grab the argument, then correctly position
%   the argument before \cs{q_@@}.
%    \begin{macrocode}
\cs_new:Npn \@@_expandable_grab_m:w #1 \q_@@ #2#3
  { #3 { \@@_expandable_grab_m_aux:wNn #1 \q_@@ #2 #3 } }
\cs_new:Npn \@@_expandable_grab_m_long:w #1 \q_@@ #2#3
  { #2 { \@@_expandable_grab_m_aux:wNn #1 \q_@@ #2 #3 } }
\cs_new:Npn \@@_expandable_grab_m_aux:wNn #1 \q_@@ #2#3#4
  { #1 {#4} \q_@@ #2 #3 }
%    \end{macrocode}
% \end{macro}
% \end{macro}
%
% \begin{macro}[EXP]{\@@_expandable_grab_R:w}
% \begin{macro}[EXP, aux]{\@@_expandable_grab_R_aux:NNNwNNn}
%   Much the same as for the \texttt{D}-type argument, with only the lead-off
%   function varying.
%    \begin{macrocode}
\cs_new:Npn \@@_expandable_grab_R:w #1 \q_@@ #2#3
  { #2 { \@@_expandable_grab_R_aux:NNNwNNn #1 \q_@@ #2#3 } }
\cs_set_protected:Npn \@@_tmp:w #1
  {
    \cs_new:Npn \@@_expandable_grab_R_aux:NNNwNNn ##1##2##3##4 \q_@@ ##5##6##7
      {
        \str_if_eq:nnTF {##7} {##2}
          {
            \str_if_eq:onTF
              { ##1 { } { } ##7 ##2 \q_@@ ##3 }
              { { } {##2} { } }
          }
          { \use_ii:nn }
          {
            ##1
              { \@@_expandable_grab_D:NNNwNNnnn ##1##2##3##4 \q_@@ ##5##6 }
              \q_nil { } ##2 \ERROR \q_@@ \ERROR
          }
          {
            \__msg_kernel_expandable_error:nnnn
              { xparse } { missing-required } {##5} {##2}
            ##4 {#1} \q_@@ ##5 ##6 {##7}
          }
      }
  }
\exp_args:No \@@_tmp:w { \c_@@_no_value_tl }
%    \end{macrocode}
% \end{macro}
% \end{macro}
%
% \begin{macro}[EXP]{\@@_expandable_grab_R_alt:w}
% \begin{macro}[EXP]{\@@_expandable_grab_R_alt_aux:NNwNNn}
%   When the delimiters are identical, nesting is not possible and a simplified
%   approach is used. The test concept here is the same as for the case where
%   the delimiters are different.
%    \begin{macrocode}
\cs_new:Npn \@@_expandable_grab_R_alt:w #1 \q_@@ #2#3
  { #2 { \@@_expandable_grab_R_alt_aux:NNnwNn #1 \q_@@ #2#3 } }
\cs_set_protected:Npn \@@_tmp:w #1
  {
    \cs_new:Npn \@@_expandable_grab_R_alt_aux:NNwNn ##1##2##3 \q_@@ ##4##5##6
      {
        \str_if_eq:nnTF {##6} {##2}
          {
            \str_if_eq:onTF
              { ##1 { } ##6 ##2 ##2 }
              { { } ##2 }
          }
          { \use_ii:nn }
          {
            ##1
              { \@@_expandable_grab_D_alt:NNwn ##4 ##5 ##3 \q_@@ }
              ##6 \ERROR
          }
          {
            \__msg_kernel_expandable_error:nnnn
              { xparse } { missing-required } {##4} {##2}
            ##3 {#1} \q_@@ ##4 ##5 {##6}
          }
      }
  }
\exp_args:No \@@_tmp:w { \c_@@_no_value_tl }
%    \end{macrocode}
% \end{macro}
% \end{macro}
%
% \begin{macro}[EXP]{\@@_expandable_grab_t:w}
% \begin{macro}[EXP, aux]{\@@_expandable_grab_t_aux:NNwn}
%   As for a \texttt{D}-type argument, here we compare the grabbed tokens using
%   the only parser we have in order to work out if |#2| is exactly equal to
%   the output of the grabber.
%    \begin{macrocode}
\cs_new:Npn \@@_expandable_grab_t:w #1 \q_@@ #2#3
  { #2 { \@@_expandable_grab_t_aux:NNwn #1 \q_@@ #2 #3 } }
\cs_new:Npn \@@_expandable_grab_t_aux:NNwn #1#2#3 \q_@@ #4#5#6
  {
    \str_if_eq:onTF { #1 { } #6 #2 } {#2}
      { #3 { \BooleanTrue } \q_@@ #4 #5 }
      { #3 { \BooleanFalse } \q_@@ #4 #5 {#6} }
  }
%    \end{macrocode}
% \end{macro}
% \end{macro}
%
% \begin{macro}[EXP]{\@@_expandable_grab_u:w}
%   It turns out there is nothing to do: this is followed by an
%   auxiliary named after the function, that does everything.
%    \begin{macrocode}
\cs_new_eq:NN \@@_expandable_grab_u:w \prg_do_nothing:
%    \end{macrocode}
% \end{macro}
%
% \begin{macro}[EXP]
%   {\@@_put_arg_expandable:nw, \@@_put_arg_expandable:ow}
%   A useful helper, to store arguments when they are ready.
%    \begin{macrocode}
\cs_new:Npn \@@_put_arg_expandable:nw #1#2 \q_@@ { #2 {#1} \q_@@ }
\cs_generate_variant:Nn \@@_put_arg_expandable:nw { o }
%    \end{macrocode}
% \end{macro}
%
% \subsection{Argument processors}
%
% \begin{macro}{\@@_bool_reverse:N}
%   A simple reversal.
%    \begin{macrocode}
\cs_new_protected:Npn \@@_bool_reverse:N #1
  {
    \bool_if:NTF #1
      { \tl_set:Nn \ProcessedArgument { \c_false_bool } }
      { \tl_set:Nn \ProcessedArgument { \c_true_bool } }
  }
%    \end{macrocode}
% \end{macro}
%
% \begin{variable}{\l_@@_split_list_seq, \l_@@_split_list_tl}
% \begin{macro}{\@@_split_list:nn}
% \begin{macro}[aux]{\@@_split_list_multi:nn, \@@_split_list_multi:nV}
% \begin{macro}[aux]{\@@_split_list_single:Nn}
%   Splitting can take place either at a single token or at a longer
%   identifier. To deal with single active tokens, a two-part procedure is
%   needed.
%    \begin{macrocode}
\seq_new:N \l_@@_split_list_seq
\tl_new:N \l_@@_split_list_tl
\cs_new_protected:Npn \@@_split_list:nn #1#2
  {
    \tl_if_single:nTF {#1}
      {
        \token_if_cs:NTF #1
          { \@@_split_list_multi:nn {#1} {#2} }
          { \@@_split_list_single:Nn #1 {#2} }
      }
      { \@@_split_list_multi:nn {#1} {#2} }
  }
\cs_new_protected:Npn \@@_split_list_multi:nn #1#2
  {
    \seq_set_split:Nnn \l_@@_split_list_seq {#1} {#2}
    \tl_clear:N \ProcessedArgument
    \seq_map_inline:Nn \l_@@_split_list_seq
      { \tl_put_right:Nn \ProcessedArgument { {##1} } }
  }
\cs_generate_variant:Nn \@@_split_list_multi:nn { nV }
\group_begin:
\char_set_catcode_active:N \^^@
\cs_new_protected:Npn \@@_split_list_single:Nn #1#2
  {
    \tl_set:Nn \l_@@_split_list_tl {#2}
    \group_begin:
    \char_set_lccode:nn { `\^^@ } { `#1 }
    \tex_lowercase:D
      {
        \group_end:
        \tl_replace_all:Nnn \l_@@_split_list_tl { ^^@ }
      }   {#1}
     \@@_split_list_multi:nV {#1} \l_@@_split_list_tl
   }
\group_end:
%    \end{macrocode}
% \end{macro}
% \end{macro}
% \end{macro}
% \end{variable}
%
% \begin{macro}{\@@_split_argument:nnn}
% \begin{macro}[aux]{\@@_split_argument_aux:nnnn}
% \begin{macro}[aux, EXP]{\@@_split_argument_aux:n}
% \begin{macro}[aux, rEXP]{\@@_split_argument_aux:wn}
%   Splitting to a known number of items is a special version of splitting
%   a list, in which the limit is hard-coded and where there will always be
%   exactly the correct number of output items. An auxiliary function is
%   used to save on working out the token list length several times.
%    \begin{macrocode}
\cs_new_protected:Npn \@@_split_argument:nnn #1#2#3
  {
    \@@_split_list:nn {#2} {#3}
    \exp_args:Nf \@@_split_argument_aux:nnnn
      { \tl_count:N \ProcessedArgument }
      {#1} {#2} {#3}
  }
\cs_new_protected:Npn \@@_split_argument_aux:nnnn #1#2#3#4
  {
    \int_compare:nNnF {#1} = { #2 + 1 }
      {
        \int_compare:nNnTF {#1} > { #2 + 1 }
          {
            \tl_set:Nx \ProcessedArgument
              {
                \exp_last_unbraced:NnNo
                  \@@_split_argument_aux:n
                  { #2 + 1 }
                  \use_none_delimit_by_q_stop:w
                  \ProcessedArgument
                  \q_stop
              }
            \__msg_kernel_error:nnxxx { xparse } { split-excess-tokens }
              { \tl_to_str:n {#3} } { \int_eval:n { #2 + 1 } }
              { \tl_to_str:n {#4} }
          }
          {
            \tl_put_right:Nx \ProcessedArgument
              {
                \prg_replicate:nn { #2 + 1 - (#1) }
                  { { \exp_not:V \c_@@_no_value_tl } }
              }
          }
      }
  }
%    \end{macrocode}
%   Auxiliaries to leave exactly the correct number of arguments in
%   \cs{ProcessedArgument}.
%    \begin{macrocode}
\cs_new:Npn \@@_split_argument_aux:n #1
  { \prg_replicate:nn {#1} { \@@_split_argument_aux:wn } }
\cs_new:Npn \@@_split_argument_aux:wn #1 \use_none_delimit_by_q_stop:w #2
  {
    \exp_not:n { {#2} }
    #1
    \use_none_delimit_by_q_stop:w
  }
%    \end{macrocode}
% \end{macro}
% \end{macro}
% \end{macro}
% \end{macro}
%
% \begin{macro}{\@@_trim_spaces:n}
%   This one is almost trivial.
%    \begin{macrocode}
\cs_new_protected:Npn \@@_trim_spaces:n #1
  { \tl_set:Nx \ProcessedArgument { \tl_trim_spaces:n {#1} } }
%    \end{macrocode}
% \end{macro}
%
% \begin{macro}{\@@_assert_int:n}
%   Check if the argument is a valid integer literal.
%    \begin{macrocode}
\cs_new_protected:Npn \@@_assert_int:n #1
  {
    \regex_match:nnF { ^[\+\-]?[\d]+$ } {#1} % $
      {
        \__msg_kernel_error:nnxxx { xparse } { type-mismatch }
          { \tl_to_str:n {#1} } { int } { \iow_char:N \\ \l_@@_function_tl }
      }
  }
%    \end{macrocode}
% \end{macro}
%
% \begin{macro}{\@@_assert_fp:n}
%   Check if the argument is a valid floating point literal.
%    \begin{macrocode}
\cs_new_protected:Npn \@@_assert_fp:n #1
  {
    \regex_match:nnF { ^[\+\-]?[\d]*\.?[\d]*([eE][\+\-]?[\d])?$ } {#1} % $
      {
        \__msg_kernel_error:nnxxx { xparse } { type-mismatch }
          { \tl_to_str:n {#1} } { fp } { \iow_char:N \\ \l_@@_function_tl }
      }
  }
%    \end{macrocode}
% \end{macro}
%
% \begin{macro}{\@@_assert_dim:n}
%   Check if the argument is a valid dimension literal.
%    \begin{macrocode}
\cs_new_protected:Npn \@@_assert_dim:n #1
  {
    \regex_match:nnF { ^[\+\-]?[\d]*\.?[\d]*\s*(pt|pc|in|bp|cm|mm|dd|cc|sp|em|ex)$ } {#1} % $
      {
        \__msg_kernel_error:nnxxx { xparse } { type-mismatch }
          { \tl_to_str:n {#1} } { dim } { \iow_char:N \\ \l_@@_function_tl }
      }
  }
%    \end{macrocode}
% \end{macro}
%
% \subsection{Access to the argument specification}
%
% \begin{macro}{\@@_get_arg_spec_error:N, \@@_get_arg_spec_error:n}
%   Provide an informative error when trying to get the argument
%   specification of a non-\pkg{xparse} command or environment.
%    \begin{macrocode}
\cs_new_protected:Npn \@@_get_arg_spec_error:N #1
  {
    \cs_if_exist:NTF #1
      {
        \__msg_kernel_error:nnx { xparse } { non-xparse-command }
          { \token_to_str:N #1 }
      }
      {
        \__msg_kernel_error:nnx { xparse } { unknown-command }
          { \token_to_str:N #1 }
      }
  }
\cs_new_protected:Npn \@@_get_arg_spec_error:n #1
  {
    \cs_if_exist:cTF {#1}
      {
        \__msg_kernel_error:nnx { xparse } { non-xparse-environment }
          { \tl_to_str:n {#1} }
      }
      {
        \__msg_kernel_error:nnx { xparse } { unknown-environment }
          { \tl_to_str:n {#1} }
      }
  }
%    \end{macrocode}
% \end{macro}
%
% \begin{macro}{\@@_get_arg_spec:NTF}
%   If the command is not an \pkg{xparse} command, complain.  If it is,
%   its second \enquote{item} is the argument specification.
%    \begin{macrocode}
\cs_new_protected:Npn \@@_get_arg_spec:NTF #1#2#3
  {
    \@@_cmd_if_xparse:NTF #1
      {
        \tl_set:Nx \ArgumentSpecification { \tl_item:Nn #1 { 2 } }
        #2
      }
      {#3}
  }
%    \end{macrocode}
% \end{macro}
%
% \begin{macro}{\@@_get_arg_spec:N}
% \begin{macro}{\@@_get_arg_spec:n}
% \begin{variable}{\ArgumentSpecification}
%   Recovering the argument specification is now trivial.
%    \begin{macrocode}
\cs_new_protected:Npn \@@_get_arg_spec:N #1
  {
    \@@_get_arg_spec:NTF #1 { }
      { \@@_get_arg_spec_error:N #1 }
  }
\cs_new_protected:Npn \@@_get_arg_spec:n #1
  {
    \exp_args:Nc \@@_get_arg_spec:NTF
      { environment~ \tl_to_str:n {#1} }
      { }
      { \@@_get_arg_spec_error:n {#1} }
  }
\tl_new:N \ArgumentSpecification
%    \end{macrocode}
% \end{variable}
% \end{macro}
% \end{macro}
%
% \begin{macro}{\@@_show_arg_spec:N}
% \begin{macro}{\@@_show_arg_spec:n}
%   Showing the argument specification simply means finding it and then
%   calling the \cs{tl_show:N} function.
%    \begin{macrocode}
\cs_new_protected:Npn \@@_show_arg_spec:N #1
  {
    \@@_get_arg_spec:NTF #1
      { \tl_show:N \ArgumentSpecification }
      { \@@_get_arg_spec_error:N #1 }
  }
\cs_new_protected:Npn \@@_show_arg_spec:n #1
  {
    \exp_args:Nc \@@_get_arg_spec:NTF
      { environment~ \tl_to_str:n {#1} }
      { \tl_show:N \ArgumentSpecification }
      { \@@_get_arg_spec_error:n {#1} }
  }
%    \end{macrocode}
% \end{macro}
% \end{macro}
%
% \subsection{Utilities}
%
% \begin{macro}[TF]{\@@_if_no_value:n}
%   Tests for |-NoValue-|: this is similar to \cs{tl_if_in:nn} but set
%   up to be expandable and to check the value exactly.  The question
%   mark prevents the auxiliary from losing braces.
%    \begin{macrocode}
\use:x
  {
    \prg_new_conditional:Npnn \exp_not:N \@@_if_no_value:n ##1
      { T ,  F , TF }
      {
        \exp_not:N \str_if_eq:onTF
          {
            \exp_not:N \@@_if_value_aux:w ? ##1 { }
              \c_@@_no_value_tl
          }
          { ? { } \c_@@_no_value_tl }
          { \exp_not:N \prg_return_true: }
          { \exp_not:N \prg_return_false: }
      }
    \cs_new:Npn \exp_not:N \@@_if_value_aux:w ##1 \c_@@_no_value_tl
      {##1}
  }
%    \end{macrocode}
% \end{macro}
%
% \begin{macro}{\@@_check_definable:nNT, \@@_check_definable_aux:nN}
%   Check that a token list is appropriate as a first argument of
%   \cs{NewDocumentCommand} and similar functions and otherwise
%   produce an error.  First trim whitespace to allow for spaces around
%   the actual command to be defined.  If the result has multiple
%   tokens, it is not a valid argument.  The single token is a control
%   sequence exactly if its string representation has more than one
%   character (using \cs{token_to_str:N} rather than \cs{tl_to_str:n}
%   to avoid problems with macro parameter characters, and setting
%   \cs{tex_escapechar:D} to prevent it from being non-printable).
%   Finally, check for an active character: this is done by lowercasing
%   the token to fix its character code (arbitrarily to that of~|?|)
%   and comparing the result to an active~|?|.  Both control sequences
%   and active characters are valid arguments, and non-active character
%   tokens are not.  In all cases, the group opened to keep assignments
%   local must be closed.
%    \begin{macrocode}
\cs_new_protected:Npn \@@_check_definable:nNT #1
  { \exp_args:Nx \@@_check_definable_aux:nN { \tl_trim_spaces:n {#1} } }
\group_begin:
  \char_set_catcode_active:n { `? }
  \cs_new_protected:Npn \@@_check_definable_aux:nN #1#2
    {
      \group_begin:
      \tl_if_single_token:nTF {#1}
        {
          \int_set:Nn \tex_escapechar:D { 92 }
          \exp_args:Nx \tl_if_empty:nTF
            { \exp_args:No \str_tail:n { \token_to_str:N #1 } }
            {
              \exp_args:Nx \char_set_lccode:nn
                { ` \str_head:n {#1} } { `? }
              \tex_lowercase:D { \tl_if_eq:nnTF {#1} } { ? }
                { \group_end: \use_iii:nnn }
                { \group_end: \use_i:nnn }
            }
            { \group_end: \use_iii:nnn }
        }
        { \group_end: \use_ii:nnn }
      {
        \__msg_kernel_error:nnxx { xparse } { not-definable }
          { \tl_to_str:n {#1} } { \token_to_str:N #2 }
      }
      {
        \__msg_kernel_error:nnxx { xparse } { not-one-token }
          { \tl_to_str:n {#1} } { \token_to_str:N #2 }
      }
    }
\group_end:
%    \end{macrocode}
% \end{macro}
%
% \begin{macro}{\@@_tl_mapthread_function:NNN, \@@_tl_mapthread_function:nnN}
% \begin{macro}[aux]{\@@_tl_mapthread_loop:w}
%   Analogue of \cs{seq_mapthread_function:NNN} for token lists.
%    \begin{macrocode}
\cs_new:Npn \@@_tl_mapthread_function:NNN #1#2#3
  {
    \exp_after:wN \exp_after:wN
    \exp_after:wN \@@_tl_mapthread_loop:w
    \exp_after:wN \exp_after:wN
    \exp_after:wN #3
    \exp_after:wN #1
    \exp_after:wN \q_recursion_tail
    \exp_after:wN \q_mark
    #2
    \q_recursion_tail
    \q_recursion_stop
  }
\cs_new:Npn \@@_tl_mapthread_function:nnN #1#2#3
  {
    \@@_tl_mapthread_loop:w #3
      #1 \q_recursion_tail \q_mark
      #2 \q_recursion_tail \q_recursion_stop
  }
\cs_new:Npn \@@_tl_mapthread_loop:w #1#2#3 \q_mark #4
  {
    \quark_if_recursion_tail_stop:n {#2}
    \quark_if_recursion_tail_stop:n {#4}
    #1 {#2} {#4}
    \@@_tl_mapthread_loop:w #1#3 \q_mark
  }
%    \end{macrocode}
% \end{macro}
% \end{macro}
%
% \begin{macro}{\@@_cmd_if_xparse:NTF}
% \begin{macro}[aux]{\@@_cmd_if_xparse_aux:N}
%   To determine whether the command is an \pkg{xparse} command check
%   that its |arg_spec| is empty (this also excludes non-macros) and
%   that its |replacement_spec| starts with either \cs{@@_start:nNNnnn}
%   (non-expandable command) or \cs{@@_start_expandable:nNNNNn}
%   (expandable command).
%    \begin{macrocode}
\cs_new_protected:Npn \@@_cmd_if_xparse:NTF #1
  {
    \exp_args:Nf \str_case_x:nnTF
      {
        \exp_args:Nf \tl_if_empty:nT { \token_get_arg_spec:N #1 }
          {
            \exp_last_unbraced:Nf \@@_cmd_if_xparse_aux:w
              { \token_get_replacement_spec:N #1 } ~ \q_stop
          }
      }
      {
        { \token_to_str:N \@@_start:nNNnnn } { }
        { \token_to_str:N \@@_start_expandable:nNNNNn } { }
      }
  }
\cs_new:Npn \@@_cmd_if_xparse_aux:w #1 ~ #2 \q_stop {#1}
%    \end{macrocode}
% \end{macro}
% \end{macro}
%
% \subsection{Messages}
%
% Some messages intended as errors.
%    \begin{macrocode}
\__msg_kernel_new:nnnn { xparse } { bad-arg-spec }
  { Bad~argument~specification~'#2'~for~command~'#1'. }
  {
    \c__msg_coding_error_text_tl
    The~argument~specification~provided~was~not~valid:~
    one~or~more~mandatory~pieces~of~information~were~missing. \\ \\
    LaTeX~will~ignore~this~entire~definition.
  }
\__msg_kernel_new:nnnn { xparse } { command-already-defined }
  { Command~'#1'~already~defined! }
  {
    You~have~used~#2~
    with~a~command~that~already~has~a~definition. \\
    The~existing~definition~of~'#1'~will~not~be~altered.
  }
\__msg_kernel_new:nnnn { xparse } { command-not-yet-defined }
  { Command ~'#1'~not~yet~defined! }
  {
    You~have~used~#2~
    with~a~command~that~was~never~defined. \\
    LaTeX~will~ignore~this~entire~definition.
  }
\__msg_kernel_new:nnnn { xparse } { environment-already-defined }
  { Environment~'#1'~already~defined! }
  {
    You~have~used~\NewDocumentEnvironment
    with~an~environment~that~already~has~a~definition. \\
    The~existing~definition~of~'#1'~will~be~overwritten.
  }
\__msg_kernel_new:nnnn { xparse } { environment-not-yet-defined }
  { Environment~'#1'~not~yet~defined! }
  {
    You~have~used~\RenewDocumentEnvironment
    with~an~environment~that~was~never~defined. \\
    LaTeX~will~ignore~this~entire~definition.
  }
\__msg_kernel_new:nnnn { xparse } { expandable-ending-optional }
  {
    Argument~specification~'#2'~for~expandable~command~'#1'~
    ends~with~optional~argument.
  }
  {
    \c__msg_coding_error_text_tl
    Expandable~commands~must~have~a~final~mandatory~argument~
    (or~no~arguments~at~all).~You~cannot~have~a~terminal~optional~
    argument~with~expandable~commands.
  }
\__msg_kernel_new:nnnn { xparse } { inconsistent-long }
  { Inconsistent~long~arguments~for~expandable~command~'#1'. }
  {
    \c__msg_coding_error_text_tl
    The~arguments~for~an~expandable~command~must~not~involve~short~
    arguments~after~long~arguments.~You~have~tried~to~mix~the~two~types.
  }
\__msg_kernel_new:nnnn { xparse } { invalid-expandable-argument-type }
  { Argument~type~'#2'~not~available~for~expandable~command~'#1'. }
  {
    \c__msg_coding_error_text_tl
    The~letter~'#2'~does~not~specify~an~argument~type~which~can~be~used~
    in~an~expandable~command.
    \\ \\
    LaTeX~will~ignore~this~entire~definition.
  }
\__msg_kernel_new:nnnn { xparse } { invalid-after-optional-expandably }
  {
    Argument~type~'#2'~not~available~after~optional~argument~
    for~expandable~command~'#1'.
  }
  {
    \c__msg_coding_error_text_tl
    The~letter~'#2'~does~not~specify~an~argument~type~which~can~be~used~
    in~an~expandable~command~after~an~optional~argument.
    \\ \\
    LaTeX~will~ignore~this~entire~definition.
  }
\__msg_kernel_new:nnnn { xparse } { loop-in-defaults }
  { Circular~dependency~in~defaults~of~'#1'. }
  {
    \c__msg_coding_error_text_tl
    The~default~values~of~two~or~more~arguments~of~'#1'~depend~on~each~
    other~in~a~way~that~cannot~be~resolved.
  }
\__msg_kernel_new:nnnn { xparse } { missing-required }
  { Failed~to~find~required~argument~starting~with~'#2'~for~command~'#1'. }
  {
    The~current~command~'#1'~expects~an~argument~starting~with~'#2'.~
    LaTeX~did~not~find~it,~and~will~insert~a~default~value~to~be~processed.
  }
\__msg_kernel_new:nnnn { xparse } { non-xparse-command }
  { Command~'#1'~not~defined~using~xparse. }
  {
    You~have~asked~for~the~argument~specification~for~a~command~'#1',~
    but~this~is~not~a~command~defined~using~xparse.
  }
\__msg_kernel_new:nnnn { xparse } { non-xparse-environment }
  { Environment~'#1'~not~defined~using~xparse. }
  {
    You~have~asked~for~the~argument~specification~for~an~environment~'#1',~
    but~this~is~not~an~environment~defined~using~xparse.
  }
\__msg_kernel_new:nnnn { xparse } { not-definable }
  { First~argument~of~'#2'~must~be~a~command. }
  {
    \c__msg_coding_error_text_tl
    The~first~argument~of~'#2'~should~be~the~document~command~that~will~
    be~defined.~The~provided~argument~'#1'~is~a~character.~Perhaps~a~
    backslash~is~missing?
    \\ \\
    LaTeX~will~ignore~this~entire~definition.
  }
\__msg_kernel_new:nnnn { xparse } { not-one-token }
  { First~argument~of~'#2'~must~be~a~command. }
  {
    \c__msg_coding_error_text_tl
    The~first~argument~of~'#2'~should~be~the~document~command~that~will~
    be~defined.~The~provided~argument~'#1'~contains~more~than~one~
    token.
    \\ \\
    LaTeX~will~ignore~this~entire~definition.
  }
\__msg_kernel_new:nnnn { xparse } { not-single-token }
  { Argument~delimiter~'#2'~for~the~command~'#1'~should~be~a~single~token. }
  {
    \c__msg_coding_error_text_tl
    The~argument~specification~provided~was~not~valid:~
    in~a~place~where~a~single~token~is~required,~LaTeX~found~'#2'. \\ \\
    LaTeX~will~ignore~this~entire~definition.
  }
\__msg_kernel_new:nnnn { xparse } { processor-in-expandable }
  { Argument~processor~'>{#2}'~cannot~be~used~for~the~expandable~command~'#1'. }
  {
    \c__msg_coding_error_text_tl
    The~argument~specification~for~#1~contains~a~processor~function:~
    this~is~only~supported~for~standard~robust~commands. \\ \\
    LaTeX~will~ignore~this~entire~definition.
  }
\__msg_kernel_new:nnnn { xparse } { split-excess-tokens }
  { Too~many~'#1'~tokens~when~trying~to~split~argument. }
  {
    LaTeX~was~asked~to~split~the~input~'#3'~
    at~each~occurrence~of~the~token~'#1',~up~to~a~maximum~of~#2~parts.~
    There~were~too~many~'#1'~tokens.
  }
\__msg_kernel_new:nnnn { xparse } { too-many-arguments }
  { Too~many~arguments~in~argument~specification~'#2'~of~command~'#1'. }
  {
    \c__msg_coding_error_text_tl
    The~argument~specification~provided~has~more~than~9~arguments.~
    This~cannot~be~implemented. \\ \\
    LaTeX~will~ignore~this~entire~definition.
  }
\__msg_kernel_new:nnnn { xparse } { unknown-argument-type }
  { Unknown~argument~type~'#2'~for~the~command~'#1'. }
  {
    \c__msg_coding_error_text_tl
    The~letter~'#2'~does~not~specify~a~known~argument~type.~
    LaTeX~will~ignore~this~entire~definition.
  }
\__msg_kernel_new:nnnn { xparse } { unknown-command }
  { Unknown~document~command~'#1'. }
  {
    You~have~asked~for~the~argument~specification~for~a~command~'#1',~
    but~it~is~not~defined.
  }
\__msg_kernel_new:nnnn { xparse } { unknown-environment }
  { Unknown~document~environment~'#1'. }
  {
    You~have~asked~for~the~argument~specification~for~an~environment~'#1',~
    but~it~is~not~defined.
  }
\__msg_kernel_new:nnnn { xparse } { verbatim-newline }
  { Verbatim~argument~of~'#1'~ended~by~end~of~line. }
  {
    The~verbatim~argument~of~'#1'~cannot~contain~more~than~one~line,~
    but~the~end~
    of~the~current~line~has~been~reached.~You~have~probably~forgotten~the~
    closing~delimiter.
    \\ \\
    LaTeX~will~ignore~'#2'.
  }
\__msg_kernel_new:nnnn { xparse } { verbatim-tokenized }
  {
    The~verbatim~command~'#1'~cannot~be~used~inside~an~argument.~
  }
  {
    The~command~'#1'~takes~a~verbatim~argument.~
    It~may~not~appear~within~the~argument~of~another~function.~
    It~received~an~illegal~token \tl_if_empty:nF {#3} { ~'#3' } .
    \\ \\
    LaTeX~will~ignore~'#2'.
  }
\__msg_kernel_new:nnnn { xparse } { type-mismatch }
  { The~type~of~'#1'~does~not~match~the~type~'#2'~expected~by~'#3'. }
  {
    \c__msg_coding_error_text_tl
    The~argument~is~required~to~be~of~type~'#2'.~This~
    requirement~was~not~met~by~the~input
    \\ \\
    '#1'
    \\ \\
    Please~make~sure~that~the argument~is~a~valid~'#2'~literal.
  }
%    \end{macrocode}
%
% Intended more for information.
%    \begin{macrocode}
\__msg_kernel_new:nnn { xparse } { define-command }
  {
    Defining~command~#1~
    with~sig.~'#2'~\msg_line_context:.
  }
\__msg_kernel_new:nnn { xparse } { define-environment }
  {
    Defining~environment~'#1'~
    with~sig.~'#2'~\msg_line_context:.
  }
\__msg_kernel_new:nnn { xparse } { redefine-command }
  {
    Redefining~command~#1~
    with~sig.~'#2'~\msg_line_context:.
  }
\__msg_kernel_new:nnn { xparse } { redefine-environment }
  {
    Redefining~environment~'#1'~
    with~sig.~'#2'~\msg_line_context:.
  }
\__msg_kernel_new:nnn { xparse } { optional-mandatory }
  {
    Since~the~mandatory~argument~'#1'~has~the~same~delimiter~'#2'~
    as~a~previous~optional~argument,~it~will~not~be~possible~to~
    omit~all~optional~arguments~when~calling~this~command.
  }
\__msg_kernel_new:nnn { xparse } { unsupported-let }
  {
    The~command~'#1'~was~undefined~but~not~the~associated~commands~
    '#1~code'~and/or~'#1~defaults'.~Maybe~you~tried~using~
    \iow_char:N\\let.~This~may~lead~to~an~infinite~loop.
  }
%    \end{macrocode}
%
% \subsection{User functions}
%
% The user functions are more or less just the internal functions
% renamed.
%
% \begin{macro}{\BooleanFalse}
% \begin{macro}{\BooleanTrue}
%   Design-space names for the Boolean values.
%    \begin{macrocode}
\cs_new_eq:NN \BooleanFalse \c_false_bool
\cs_new_eq:NN \BooleanTrue  \c_true_bool
%    \end{macrocode}
% \end{macro}
% \end{macro}
%
% \begin{macro}{\NewDocumentCommand}
% \begin{macro}{\RenewDocumentCommand}
% \begin{macro}{\ProvideDocumentCommand}
% \begin{macro}{\DeclareDocumentCommand}
%   The user macros are pretty simple wrappers around the internal ones.
%   There is however a check that the first argument is a single token,
%   possibly surrounded by spaces (hence the strange \cs{use:nnn}), and
%   is definable.
%    \begin{macrocode}
\cs_new_protected:Npn \NewDocumentCommand #1#2#3
  {
    \@@_check_definable:nNT {#1} \NewDocumentCommand
      {
        \cs_if_exist:NTF #1
          {
            \__msg_kernel_error:nnxx { xparse } { command-already-defined }
              { \use:nnn \token_to_str:N #1 { } }
              { \token_to_str:N \NewDocumentCommand }
          }
          { \@@_declare_cmd:Nnn #1 {#2} {#3} }
      }
  }
\cs_new_protected:Npn \RenewDocumentCommand #1#2#3
  {
    \@@_check_definable:nNT {#1} \RenewDocumentCommand
      {
        \cs_if_exist:NTF #1
          { \@@_declare_cmd:Nnn #1 {#2} {#3} }
          {
            \__msg_kernel_error:nnxx { xparse } { command-not-yet-defined }
              { \use:nnn \token_to_str:N #1 { } }
              { \token_to_str:N \RenewDocumentCommand }
          }
      }
  }
\cs_new_protected:Npn \ProvideDocumentCommand #1#2#3
  {
    \@@_check_definable:nNT {#1} \ProvideDocumentCommand
      { \cs_if_exist:NF #1 { \@@_declare_cmd:Nnn #1 {#2} {#3} } }
 }
\cs_new_protected:Npn \DeclareDocumentCommand #1#2#3
  {
    \@@_check_definable:nNT {#1} \DeclareDocumentCommand
      { \@@_declare_cmd:Nnn #1 {#2} {#3} }
  }
%    \end{macrocode}
% \end{macro}
% \end{macro}
% \end{macro}
% \end{macro}
%
% \begin{macro}{\NewDocumentEnvironment}
% \begin{macro}{\RenewDocumentEnvironment}
% \begin{macro}{\ProvideDocumentEnvironment}
% \begin{macro}{\DeclareDocumentEnvironment}
%   Very similar for environments.
%    \begin{macrocode}
\cs_new_protected:Npn \NewDocumentEnvironment #1#2#3#4
  {
    \cs_if_exist:cTF {#1}
      { \__msg_kernel_error:nnx { xparse } { environment-already-defined } {#1} }
      { \@@_declare_env:nnnn {#1} {#2} {#3} {#4} }
}
\cs_new_protected:Npn \RenewDocumentEnvironment #1#2#3#4
  {
    \cs_if_exist:cTF {#1}
      { \@@_declare_env:nnnn {#1} {#2} {#3} {#4} }
      { \__msg_kernel_error:nnx { xparse } { environment-not-yet-defined } {#1} }
  }
\cs_new_protected:Npn \ProvideDocumentEnvironment #1#2#3#4
  { \cs_if_exist:cF {#1} { \@@_declare_env:nnnn {#1} {#2} {#3} {#4} } }
\cs_new_protected:Npn \DeclareDocumentEnvironment #1#2#3#4
  { \@@_declare_env:nnnn {#1} {#2} {#3} {#4} }
%    \end{macrocode}
% \end{macro}
% \end{macro}
% \end{macro}
% \end{macro}
%
% \begin{macro}{\NewExpandableDocumentCommand}
% \begin{macro}{\RenewExpandableDocumentCommand}
% \begin{macro}{\ProvideExpandableDocumentCommand}
% \begin{macro}{\DeclareExpandableDocumentCommand}
%   The expandable versions are essentially the same as the basic functions.
%    \begin{macrocode}
\cs_new_protected:Npn \NewExpandableDocumentCommand #1#2#3
  {
    \@@_check_definable:nNT {#1} \NewExpandableDocumentCommand
      {
        \cs_if_exist:NTF #1
          {
            \__msg_kernel_error:nnxx { xparse } { command-already-defined }
              { \use:nnn \token_to_str:N #1 { } }
              { \token_to_str:N \NewExpandableDocumentCommand }
          }
          { \@@_declare_expandable_cmd:Nnn #1 {#2} {#3} }
      }
  }
\cs_new_protected:Npn \RenewExpandableDocumentCommand #1#2#3
  {
    \@@_check_definable:nNT {#1} \RenewExpandableDocumentCommand
      {
        \cs_if_exist:NTF #1
          { \@@_declare_expandable_cmd:Nnn #1 {#2} {#3} }
          {
            \__msg_kernel_error:nnxx { xparse } { command-not-yet-defined }
              { \use:nnn \token_to_str:N #1 { } }
              { \token_to_str:N \RenewExpandableDocumentCommand }
          }
      }
  }
\cs_new_protected:Npn \ProvideExpandableDocumentCommand #1#2#3
  {
    \@@_check_definable:nNT {#1} \ProvideExpandableDocumentCommand
      {
        \cs_if_exist:NF #1
          { \@@_declare_expandable_cmd:Nnn #1 {#2} {#3} }
      }
 }
\cs_new_protected:Npn \DeclareExpandableDocumentCommand #1#2#3
  {
    \@@_check_definable:nNT {#1} \DeclareExpandableDocumentCommand
      { \@@_declare_expandable_cmd:Nnn #1 {#2} {#3} }
  }
%    \end{macrocode}
% \end{macro}
% \end{macro}
% \end{macro}
% \end{macro}
%
% \begin{macro}{\IfBooleanT, \IfBooleanF, \IfBooleanTF}
%   The logical \meta{true} and \meta{false} statements are just the
%   normal \cs{c_true_bool} and \cs{c_false_bool}, so testing for them is
%   done with the \cs{bool_if:NTF} functions from \textsf{l3prg}.
%    \begin{macrocode}
\cs_new_eq:NN \IfBooleanTF \bool_if:NTF
\cs_new_eq:NN \IfBooleanT  \bool_if:NT
\cs_new_eq:NN \IfBooleanF  \bool_if:NF
%    \end{macrocode}
% \end{macro}
%
% \begin{macro}{\IfNoValueT, \IfNoValueF, \IfNoValueTF}
%   Simple re-naming.
%    \begin{macrocode}
\cs_new_eq:NN \IfNoValueF  \@@_if_no_value:nF
\cs_new_eq:NN \IfNoValueT  \@@_if_no_value:nT
\cs_new_eq:NN \IfNoValueTF \@@_if_no_value:nTF
%    \end{macrocode}
% \end{macro}
% \begin{macro}{\IfValueT, \IfValueF, \IfValueTF}
%   Inverted logic.
%    \begin{macrocode}
\cs_new:Npn \IfValueF { \@@_if_no_value:nT }
\cs_new:Npn \IfValueT { \@@_if_no_value:nF }
\cs_new:Npn \IfValueTF #1#2#3 { \@@_if_no_value:nTF {#1} {#3} {#2} }
%    \end{macrocode}
% \end{macro}
%
% \begin{macro}{\ProcessedArgument}
%   Processed arguments are returned using this name, which is reserved
%   here although the definition will change.
%    \begin{macrocode}
\tl_new:N \ProcessedArgument
%    \end{macrocode}
% \end{macro}
%
% \begin{macro}{\ReverseBoolean, \SplitArgument, \SplitList, \TrimSpaces, \AssertInt, \AssertFp, \AssertDim}
%   Simple copies.
%    \begin{macrocode}
\cs_new_eq:NN \ReverseBoolean \@@_bool_reverse:N
\cs_new_eq:NN \SplitArgument  \@@_split_argument:nnn
\cs_new_eq:NN \SplitList      \@@_split_list:nn
\cs_new_eq:NN \TrimSpaces     \@@_trim_spaces:n
\cs_new_eq:NN \AssertInt \__xparse_assert_int:n
\cs_new_eq:NN \AssertFp  \__xparse_assert_fp:n
\cs_new_eq:NN \AssertDim \__xparse_assert_dim:n
%    \end{macrocode}
% \end{macro}
%
% \begin{macro}{\ProcessList}
%   To support \cs{SplitList}.
%    \begin{macrocode}
\cs_new_eq:NN \ProcessList \tl_map_function:nN
%    \end{macrocode}
% \end{macro}
%
% \begin{macro}{\GetDocumentCommandArgSpec}
% \begin{macro}{\GetDocumentEnvironmentArgSpec}
% \begin{macro}{\ShowDocumentCommandArgSpec}
% \begin{macro}{\ShowDocumentEnvironmentArgSpec}
%   More simple mappings, with a check that the argument is a single
%   control sequence or active character.
%    \begin{macrocode}
\cs_new_protected:Npn \GetDocumentCommandArgSpec #1
  {
    \@@_check_definable:nNT {#1} \GetDocumentCommandArgSpec
      { \@@_get_arg_spec:N #1 }
  }
\cs_new_eq:NN \GetDocumentEnvironmentArgSpec \@@_get_arg_spec:n
\cs_new_protected:Npn \ShowDocumentCommandArgSpec #1
  {
    \@@_check_definable:nNT {#1} \ShowDocumentCommandArgSpec
      { \@@_show_arg_spec:N #1 }
  }
\cs_new_eq:NN \ShowDocumentEnvironmentArgSpec \@@_show_arg_spec:n
%    \end{macrocode}
% \end{macro}
% \end{macro}
% \end{macro}
% \end{macro}
%
% \subsection{Package options}
%
% \begin{variable}{\l_@@_options_clist}
% \begin{variable}{\l_@@_log_bool}
%   Key--value option to log information: done by hand to keep dependencies
%   down.
%    \begin{macrocode}
\clist_new:N \l_@@_options_clist
\DeclareOption* { \clist_put_right:NV \l_@@_options_clist \CurrentOption }
\ProcessOptions \relax
\keys_define:nn { xparse }
  {
    log-declarations .bool_set:N = \l_@@_log_bool ,
    log-declarations .initial:n  = true
  }
\keys_set:nV { xparse } \l_@@_options_clist
\bool_if:NF \l_@@_log_bool
  {
    \msg_redirect_module:nnn { LaTeX / xparse } { info }    { none }
    \msg_redirect_module:nnn { LaTeX / xparse } { warning } { none }
  }
%    \end{macrocode}
% \end{variable}
% \end{variable}
%
%    \begin{macrocode}
%</package>
%    \end{macrocode}
%
% \end{implementation}
%
% \PrintIndex
