% \iffalse meta-comment
%
%% File: xtemplate.dtx
%
% Copyright (C) 1999 Frank Mittelbach, Chris Rowley, David Carlisle
%           (C) 2004-2010 Frank Mittelbach, The LaTeX Project
%           (C) 2011-2024 The LaTeX Project
%
% It may be distributed and/or modified under the conditions of the
% LaTeX Project Public License (LPPL), either version 1.3c of this
% license or (at your option) any later version.  The latest version
% of this license is in the file
%
%    https://www.latex-project.org/lppl.txt
%
% This file is part of the "l3packages bundle" (The Work in LPPL)
% and all files in that bundle must be distributed together.
%
% -----------------------------------------------------------------------
%
% The development version of the bundle can be found at
%
%    https://github.com/latex3/latex3
%
% for those people who are interested.
%
%<*driver>
\documentclass[full]{l3doc}
\usepackage{amstext}
\begin{document}
  \DocInput{\jobname.dtx}
\end{document}
%</driver>
% \fi
%
% \title{^^A
%   The \pkg{xtemplate} package\\ Prototype document functions^^A
% }
%
% \author{^^A
%  The \LaTeX{} Project\thanks
%    {^^A
%      E-mail:
%        \href{mailto:latex-team@latex-project.org}
%          {latex-team@latex-project.org}^^A
%    }^^A
% }
%
% \date{Released 2024-08-16}
%
% \maketitle
%
% \begin{documentation}
%
% There are three broad \enquote{layers} between putting down ideas into
% a source file and ending up with a typeset document. These layers of
% document writing are
% \begin{enumerate}
%   \item authoring of the text with mark-up;
%   \item document layout design;
%   \item implementation (with \TeX{} programming) of the design.
% \end{enumerate}
% We write the text as an author, and we see the visual output of the design
% after the document is generated; the \TeX{} implementation in the middle is
% the glue between the two.
%
% \LaTeX{}'s greatest success has been to standardize a system of mark-up that
% balances the trade-off between ease of reading and ease of writing to suit
% almost all forms of technical writing. It's
% other original strength was a good background in typographical design; while
% the standard \LaTeXe{} classes look somewhat dated now in terms of their
% visual design, their typography is generally sound. (Barring the occasional
% minor faults.)
%
% However, \LaTeXe{} has always lacked a standard approach to customizing
% the visual design of a document. Changing the looks of the standard classes
% involved either:
% \begin{itemize}
%   \item Creating a new version of the implementation code of the class and
%     editing it.
%   \item Loading one of the many packages to customize certain elements of
%     the standard classes.
%   \item Loading a completely different document class, such as
%     \cls{KOMA-Script} or \cls{memoir}, that allows easy customization.
% \end{itemize}
% All three of these approaches have their drawbacks and learning curves.
%
% The idea behind \pkg{xtemplate} is to cleanly separate the three layers
% introduced at the beginning of this section, so that document authors who
% are not programmers can easily change the design of their documents.
% \pkg{xtemplate} also makes it easier for \LaTeX{} programmers to provide
% their own customizations on top of a pre-existing class.
%
% \section{What is a document?}
%
% Besides the textual content of the words themselves, the source file
% of a document contains mark-up elements that add structure to the
% document. These elements include sectional divisions, figure/table
% captions, lists of various sorts, theorems/proofs, and so on.
% The list will be different for every document that can be written.
%
% Each element can be represented logically without worrying about the
% formatting, with mark-up such as \cs{section}, \cs{caption},
% |\begin{enumerate}| and so on. The output of each one of these
% document elements will be a typeset representation of the information
% marked up, and the visual arrangement and design of these elements
% can vary widely in producing a variety of desired outcomes.
%
% For each type of document element, there may be design variations that
% contain the same sort of information but present it in slightly
% different ways. For example, the difference between a numbered and an
% unnumbered section, \cs{section} and |\section*|, or the difference
% between an itemized list or an enumerated list.
%
% There are three distinct layers in the definition of
% \enquote{a document} at this level
% \begin{enumerate}
%   \item semantic elements such as the ideas of sections and lists;
%   \item a set of design solutions for representing these elements
%     visually;
%   \item specific variations for these designs that represent the
%     elements in the document.
% \end{enumerate}
% In the parlance of the template system, these are called object types,
% templates, and instances, and they are discussed below in sections
% \ref{sec:objects}, \ref{sec:templates}, and~\ref{sec:instances},
% respectively.
%
% \section {Objects, templates, and instances}
%
% By formally declaring documents to be composed of mark-up elements
% grouped into objects, which are interpreted and typeset with a set of
% templates, each of which has one or more instances with which to
% compose each and every semantic unit of the text, we can cleanly
% separate the components of document construction.
%
% All of the structures provided by the template system are global,
% and do not respect \TeX{} grouping.
%
% \section{Object types}
% \label{sec:objects}
%
% An \emph{object type} (sometimes just \enquote{object}) is an
% abstract idea of a document element that takes a fixed number of
% arguments corresponding to the information from the document author
% that it is representing. A sectioning object, for example, might take
% three inputs: \enquote{title}, \enquote{short title}, and
% \enquote{label}.
%
% Any given document class will define which object types are to be
% used in the document, and any template of a given object type can be
% used to generate an instance for the object. (Of course, different
% templates will produce different typeset representations, but the
% underlying content will be the same.)
%
% \begin{function}{\DeclareObjectType}
%   \begin{syntax}
%     \cs{DeclareObjectType} \Arg{object type} \Arg{no. of args}
%   \end{syntax}
%  This function defines an \meta{object type} taking
%  \meta{number of arguments}, where the \meta{object type} is an
%  abstraction as discussed above. For example,
%   \begin{verbatim}
%     \DeclareObjectType{sectioning}{3}
%   \end{verbatim}
%  creates an object type \enquote{sectioning}, where each use of that
%  object type will need three arguments.
% \end{function}
%
% \section{Templates}
% \label{sec:templates}
%
% A \emph{template} is a generalized design solution for representing
% the information of a specified object type. Templates that do the same
% thing, but in different ways, are grouped together by their object type
% and given separate names. There are two important parts to a template:
% \begin{itemize}
%   \item the parameters it takes to vary the design it is producing;
%   \item the implementation of the design.
% \end{itemize}
% As a document author or designer does not care about the
% implementation but rather only the interface to the template, these two
% aspects of the template definition are split into two independent
% declarations, \cs{DeclareTemplateInterface} and
% \cs{DeclareTemplateCode}.
%
% \begin{function}{\DeclareTemplateInterface}
%   \begin{syntax}
%     \cs{DeclareTemplateInterface}
%     ~~\Arg{object type} \Arg{template} \Arg{no.~of args}
%     ~~\Arg{key list}
%   \end{syntax}
%   A \meta{template} interface is declared for a particular
%   \meta{object type}, where the \meta{number of arguments} must
%   agree with the object type declaration. The interface itself is
%   defined by the \meta{key list}, which is itself a key--value list
%   taking a specialized format:
%   \begin{quotation}
%     \obeylines
%     \noindent
%      \meta{key1}~":"~\meta{key type1}~","
%      \meta{key2}~":"~\meta{key type2}~","
%      \meta{key3}~":"~\meta{key type3}~"="~\meta{default3}~","
%      \meta{key4}~":"~\meta{key type4}~"="~\meta{default4}~","
%      \ldots
%    \end{quotation}
%   Each \meta{key} name should consist of \textsc{ascii} characters,
%   with the exception of |,|, |=| and \verb*| |. The recommended form
%   for key names is to use lower case letters, with dashes to separate
%   out different parts. Spaces are ignored in key names, so they can be
%   included or missed out at will. Each \meta{key} must have a
%   \meta{key type}, which defined the type of input that the \meta{key}
%   requires. A full list of key types is given in
%   Table~\ref{tab:key-types}.  Each key may have a \meta{default}
%   value, which will be used in by the template if the \meta{key} is
%   not set explicitly. The \meta{default} should be of the correct
%   form to be accepted by the \meta{key type} of the \meta{key}: this
%   is not checked by the code.
% \end{function}
%
%   \begin{table}
%     \centering
%     \begin{tabular}{>{\ttfamily}ll}
%       \toprule
%       \multicolumn{1}{l}{Key-type} & Description of input \\
%       \midrule
%       boolean    & \texttt{true} or \texttt{false}            \\
%       choice\Arg{choices}
%         & A list of pre-defined \meta{choices} \\
%       commalist  & A comma-separated list                        \\
%       function\Arg{$N$}
%         & A function definition with $N$ arguments
%          ($N$ from $0$ to $9$) \\
%       instance\Arg{name}
%         & An instance of type \meta{name} \\
%       integer    & An integer or integer expression            \\
%       length     & A fixed length                              \\
%       muskip     & A math length with shrink and stretch components \\
%       real       & A real (floating point) value               \\
%       skip       & A length with shrink and stretch components \\
%       tokenlist  & A token list: any text or commands          \\
%       \bottomrule
%     \end{tabular}
%     \caption{Key-types for defining template interfaces with
%       \cs{DeclareTemplateInterface}.}
%     \label{tab:key-types}
%   \end{table}
%
% \begin{function}{\KeyValue}
%   \begin{syntax}
%     \cs{KeyValue} \Arg{key name}
%   \end{syntax}
%   There are occasions where the default (or value) for one key
%   should be taken from another. The \cs{KeyValue} function can be
%   used to transfer this information without needing to know the
%   internal implementation of the key:
%   \begin{verbatim}
%     \DeclareTemplateInterface { object } { template } { no. of args }
%       {
%         key-name-1 : key-type = value ,
%         key-name-2 : key-type = \KeyValue { key-name-1 },
%         ...
%       }
%   \end{verbatim}
% \end{function}
%
% \begin{function}{\DeclareTemplateCode}
%   \begin{syntax}
%     \cs{DeclareTemplateCode}
%     ~~\Arg{object type} \Arg{template} \Arg{no.~of args}
%     ~~\Arg{key bindings} \Arg{code}
%   \end{syntax}
%   The relationship between a templates keys and the internal
%   implementation is created using the \cs{DeclareTemplateCode}
%   function. As with \cs{DeclareTemplateInterface}, the
%   \meta{template} name is given along with the \meta{object type}
%   and \meta{number of arguments} required. The \meta{key bindings}
%   argument is a key--value list which specifies the relationship
%   between each \meta{key} of the template interface with an
%   underlying\meta{variable}.
%
%   \begin{quotation}
%     \obeylines
%     \noindent
%     \meta{key1}~"="~\meta{variable1},
%     \meta{key2}~"="~\meta{variable2},
%     \meta{key3}~"="~global~\meta{variable3},
%     \meta{key4}~"="~global~\meta{variable4},
%     \ldots
%   \end{quotation}
%   With the exception of the choice, code and function key types,
%   the \meta{variable} here should be the name of an existing
%   \LaTeX3 register. As illustrated, the key word \enquote{global}
%   may be included in the listing to indicate that the \meta{variable}
%   should be assigned globally. A full list of variable bindings is
%   given in Table~\ref{tab:key-vars}.
%
%   The \meta{code} argument of \cs{DeclareTemplateCode} is used
%   as the replacement text for the template when it is used, either
%   directly or as an instance. This may therefore accept arguments
%   |#1|, |#2|, \emph{etc}.~as detailed by the \meta{number of arguments}
%   taken by the object type.
% \end{function}
%
%   \begin{table}
%     \centering
%     \begin{tabular}{>{\ttfamily}ll}
%       \toprule
%       \multicolumn{1}{l}{Key-type} & Description of binding \\
%       \midrule
%       boolean    & Boolean variable, \emph{e.g}.~\cs{l_tmpa_bool}      \\
%       choice
%         & List of choice implementations
%         (see Section~\ref{sec:choices-key}) \\
%       commalist  & Comma list, \emph{e.g}.~\cs{l_tmpa_clist}           \\
%       function
%         & Function taking $N$ arguments, \emph{e.g}.~\cs{use_i:nn}     \\
%       instance \\
%       integer    & Integer variable, \emph{e.g}.~\cs{l_tmpa_int}       \\
%       length     & Dimension variable, \emph{e.g}.~\cs{l_tmpa_dim}     \\
%       muskip     & Muskip variable, \emph{e.g}.~\cs{l_tmpa_muskip}     \\
%       real       & Floating-point variable, \emph{e.g}.~\cs{l_tmpa_fp} \\
%       skip       & Skip variable, \emph{e.g}.~\cs{l_tmpa_skip}         \\
%       tokenlist  & Token list variable, \emph{e.g}.~\cs{l_tmpa_tl}     \\
%       \bottomrule
%     \end{tabular}
%     \caption{Bindings required for different key types when defining
%       template implementations with \cs{DeclareTemplateCode}. Apart
%       from \texttt{code}, \texttt{choice} and \texttt{function}
%       all of these accept the key word \texttt{global} to carry
%       out a global assignment.}
%     \label{tab:key-vars}
%   \end{table}
%
% \begin{function}{\AssignTemplateKeys}
%   \begin{syntax}
%     \cs{AssignTemplateKeys}
%   \end{syntax}
%   In the final argument of \cs{DeclareTemplateCode} the assignment of
%   keys defined by the template may be delayed by including the command
%   \cs{AssignTemplateKeys}. If this is \emph{not} present, keys are assigned
%   immediately before the template code. If \cs{AssignTemplateKeys} is
%   present, assignment is delayed until this point. Note that the
%   command must be \emph{directly} present in the code, not placed
%   within a nested command/macro.
% \end{function}
%
% \section{Multiple choices}
% \label{sec:choices-key}
%
% The \texttt{choice} key type implements multiple choice input. At the
% interface level, only the list of valid choices is needed:
% \begin{verbatim}
%   \DeclareTemplateInterface { foo } { bar } { 0 }
%     { key-name : choice { A, B, C } }
% \end{verbatim}
% where the choices are given as a comma-list (which must therefore
% be wrapped in braces). A default value can also be given:
% \begin{verbatim}
%   \DeclareTemplateInterface { foo } { bar } { 0 }
%     { key-name : choice { A, B, C } = A }
% \end{verbatim}
%
% At the implementation level, each choice is associated with code,
% using a nested key--value list.
% \begin{verbatim}
%   \DeclareTemplateCode { foo } { bar } { 0 }
%     {
%       key-name =
%         {
%           A = Code-A ,
%           B = Code-B ,
%           C = Code-C
%         }
%      }
%      { ... }
% \end{verbatim}
% The two choice lists should match, but in the implementation a
% special \texttt{unknown} choice is also available. This can be used
% to ignore values and implement an \enquote{else} branch:
% \begin{verbatim}
%   \DeclareTemplateCode { foo } { bar } { 0 }
%     {
%       key-name =
%         {
%           A       = Code-A ,
%           B       = Code-B ,
%           C       = Code-C ,
%           unknown = Else-code
%         }
%      }
%      { ... }
% \end{verbatim}
% The \texttt{unknown} entry must be the last one given, and should
% \emph{not} be listed in the interface part of the template.
%
% For keys which accept the values \texttt{true} and \texttt{false}
% both the boolean and choice key types can be used. As template
% interfaces are intended to prompt clarity at the design level, the
% boolean key type should be favored, with the choice type reserved
% for keys which take arbitrary values.
%
% \section{Instances}
% \label{sec:instances}
%
% After a template is defined it still needs to be put to use. The
% parameters that it expects need to be defined before it can be used in
% a document. Every time a template has parameters given to it, an
% \emph{instance} is created, and this is the code that ends up in the
% document to perform the typesetting of whatever pieces of information
% are input into it.
%
% For example, a template might say \enquote{here is a section with or
% without a number that might be centered or left aligned and print its
% contents in a certain font of a certain size, with a bit of a gap
% before and after it} whereas an instance declares \enquote{this is a
% section with a number, which is centered and set in $12\,\text{pt}$
% italic with a $10\,\text{pt}$ skip before and a
% $12\,\text{pt}$ skip after it}. Therefore, an instance is just a
% frozen version of a template with specific settings as chosen by the
% designer.
%
% \begin{function}{\DeclareInstance}
%   \begin{syntax}
%     \cs{DeclareInstance}
%     ~~\Arg{object type} \Arg{instance} \Arg{template} \Arg{parameters}
%   \end{syntax}
%   This function uses a \meta{template} for an \meta{object type}
%   to create an \meta{instance}. The \meta{instance} will be set
%   up using the \meta{parameters}, which will set some of the
%   \meta{keys} in the \meta{template}.
%
%   As a practical example, consider an object type for document sections
%   (which might include chapters, parts, sections, \emph{etc}.), which
%   is called \texttt{sectioning}. One possible template for this
%   object type might be called \texttt{basic}, and one instance of this
%   template would be a numbered section. The instance declaration might
%   read:
%   \begin{verbatim}
%     \DeclareInstance { sectioning } { section-num } { basic }
%       {
%         numbered      = true ,
%         justification = center ,
%         font          =\normalsize\itshape ,
%         before-skip   = 10pt ,
%         after-skip    = 12pt ,
%       }
%   \end{verbatim}
%   Of course, the key names here are entirely imaginary, but illustrate
%   the general idea of fixing some settings.
% \end{function}
%
% \begin{function}{\IfInstanceExistT, \IfInstanceExistF, \IfInstanceExistTF}
%   \begin{syntax}
%     \cs{IfInstanceExistTF} \Arg{object type} \Arg{instance} \Arg{true code} \Arg{false code}
%   \end{syntax}
%   Tests if the named \meta{instance} of a \meta{object type} exists, and
%   then inserts the appropriate code into the input stream.
% \end{function}
%
% \begin{function}{\DeclareInstanceCopy}
%   \begin{syntax}
%     \cs{DeclareInstanceCopy}
%     ~~\Arg{object type} \Arg{instance2} \Arg{instance1}
%   \end{syntax}
%   Copies the \meta{values} for \meta{instance1} for an
%   \meta{object type} to \meta{instance2}.
% \end{function}
%
% \section{Document interface}
%
% After the instances have been chosen, document commands must be
% declared to use those instances in the document. \cs{UseInstance}
% calls instances directly, and this command should be used internally
% in document-level mark-up.
%
% \begin{function}{\UseInstance}
%   \begin{syntax}
%     \cs{UseInstance}
%     ~~\Arg{object type} \Arg{instance} \meta{arguments}
%   \end{syntax}
%   Uses an \meta{instance} of the \meta{object type}, which will require
%   \meta{arguments} as determined by the number specified for the
%   \meta{object type}. The \meta{instance} must have been declared
%   before it can be used, otherwise an error is raised.
% \end{function}
%
% \begin{function}{\UseTemplate}
%   \begin{syntax}
%     \cs{UseTemplate} \Arg{object type} \Arg{template}
%     ~~\Arg{settings} \meta{arguments}
%   \end{syntax}
%   Uses the \meta{template} of the specified \meta{object type},
%   applying the \meta{settings} and absorbing \meta{arguments} as
%   detailed by the \meta{object type} declaration. This in effect
%   is the same as creating an instance using \cs{DeclareInstance}
%   and immediately using it with \cs{UseInstance}, but without the
%   instance having any further existence. It is therefore useful where
%   a template needs to be used once.
%
%   This function can also be used as the argument to \texttt{instance}
%   key types:
%   \begin{verbatim}
%     \DeclareInstance { object } { template } { instance }
%       {
%         instance-key =
%           \UseTemplate { object2 } { template2 } { <settings> }
%       }
%   \end{verbatim}
% \end{function}
%
% \section{Changing existing definitions}
%
% Template parameters may be assigned specific defaults for instances
% to use if the instance declaration doesn't explicit set those
% parameters. In some cases, the document designer will wish to edit
% these defaults to allow them to \enquote{cascade} to the instances.
% The alternative would be to set each parameter identically for each
% instance declaration, a tedious and error-prone process.
%
% \begin{function}{\EditTemplateDefaults}
%   \begin{syntax}
%     \cs{EditTemplateDefaults}
%     ~~\Arg{object type} \Arg{template} \Arg{new defaults}
%   \end{syntax}
%   Edits the \meta{defaults} for a \meta{template} for an
%   \meta{object type}. The \meta{new defaults}, given as a key--value
%   list, replace the existing defaults for the \meta{template}. This
%   means that the change will apply to instances declared after the
%   editing, but that instances which have already been created are
%   unaffected.
% \end{function}
%
% \begin{function}{\EditInstance}
%   \begin{syntax}
%     \cs{EditInstance}
%     ~~\Arg{object type} \Arg{instance} \Arg{new values}
%   \end{syntax}
%   Edits the \meta{values} for an \meta{instance} for an
%   \meta{object type}. The \meta{new values}, given as a key--value
%   list, replace the existing values for the \meta{instance}. This
%   function is complementary to \cs{EditTemplateDefaults}:
%   \cs{EditInstance} changes a single instance while leaving the
%   template untouched.
% \end{function}
%
% \section{When template parameters should be frozen}
%
% A class designer may be inheriting templates declared by someone else,
% either third-party code or the \LaTeX{} kernel itself. Sometimes these
% templates will be overly general for the purposes of the document.
% The user should be able to customize parts of the template instances,
% but otherwise be restricted to only those parameters allowed by the
% designer.
%
% \begin{function}{\DeclareRestrictedTemplate}
%   \begin{syntax}
%     \cs{DeclareRestrictedTemplate}
%     ~~\Arg{object type} \Arg{parent template} \Arg{new template}
%     ~~\Arg{parameters}
%   \end{syntax}
%   Creates a copy of the \meta{parent template} for the
%   \meta{object type} called \meta{new template}. The key--value list
%   of \meta{parameters} applies in the \meta{new template} and cannot
%   be changed when creating an instance.
% \end{function}
%
% \section{\emph{Ad hoc} adjustment of templates}
%
% \begin{function}{\SetTemplateKeys}
%   \begin{syntax}
%    \cs{SetTemplateKeys} \Arg{object type} \Arg{template} \Arg{keyvals}
%   \end{syntax}
%   At point of use it may be useful to apply changed to individual instances.
%   This is supported as each template key is made available for adjustment
%   using \cs{SetTemplateKeys}.
% \end{function}
%
% For example, after
% \begin{verbatim}
%   \DeclareObjectType{MyObj}{0}
%   \DeclareTemplateInterface{MyObj}{TemplateA}{0}
%     {
%       akey: tokenlist  ,
%       bkey: function{2}
%     }
%   \DeclareTemplateCode{MyObj}{TemplateA}{0}
%     {
%       akey = SomeTokens ,
%       bkey = \func:nn ,
%     }
% \end{verbatim}
% the template keys could be adjusted in an \emph{ad hoc} fashion using
% \begin{verbatim}
%   \SetTemplateKeys{MyObj}{TemplateA}
%     {
%       akey = OtherTokens ,
%       bkey = \AltFunc:nn
%     }
% \end{verbatim}
%
% \section{Getting information about templates and instances}
%
% \begin{function}{\ShowInstanceValues}
%   \begin{syntax}
%    \cs{ShowInstanceValues} \Arg{object type} \Arg{instance}
%   \end{syntax}
%   Shows the \meta{values} for an \meta{instance} of the given
%   \meta{object type} at the terminal.
% \end{function}
%
% \begin{function}{\ShowTemplateCode}
%   \begin{syntax}
%     \cs{ShowTemplateCode} \Arg{object type} \Arg{template}
%   \end{syntax}
%   Shows the \meta{code} of a \meta{template} for an \meta{object type}
%   in the terminal.
% \end{function}
%
% \begin{function}{\ShowTemplateDefaults}
%   \begin{syntax}
%     \cs{ShowTemplateDefaults} \Arg{object type} \Arg{template}
%   \end{syntax}
%   Shows the \meta{default} values of a \meta{template} for an
%   \meta{object type} in the terminal.
% \end{function}
%
% \begin{function}{\ShowTemplateInterface}
%   \begin{syntax}
%     \cs{ShowTemplateInterface} \Arg{object type} \Arg{template}
%   \end{syntax}
%  Shows the \meta{keys} and associated \meta{key types} of a
%   \meta{template} for an \meta{object type} in the terminal.
% \end{function}
%
% \begin{function}{\ShowTemplateVariables}
%   \begin{syntax}
%     \cs{ShowTemplateVariables} \Arg{object type} \Arg{template}
%   \end{syntax}
%   Shows the \meta{variables} and associated \meta{keys} of a
%   \meta{template} for an \meta{object type} in the terminal. Note that
%   \texttt{code} and \texttt{choice} keys do not map directly to variables
%   but to arbitrary code. For \texttt{choice} keys, each valid choice
%   is shown as a separate entry in the list, with the key name and choice
%   separated by a space, for example
%   \begin{verbatim}
%     Template 'example' of object type 'example' has variable mapping:
%     >  demo unknown  =>  \def \demo {?}
%     >  demo c  =>  \def \demo {c}
%     >  demo b  =>  \def \demo {b}
%     >  demo a  =>  \def \demo {a}.
%   \end{verbatim}
%   would be shown for a choice key \texttt{demo} with valid choices
%   \texttt{a}, \texttt{b} and \texttt{c}, plus code for an \texttt{unknown}
%   branch.
% \end{function}
%
% \section{Collections}
%
% The implementation of templates includes a concept termed
% \enquote{collections}. The idea is that by activating a collection, a
% set of instances can rapidly be set up. An example use case would be
% collections for \texttt{frontmatter}, \texttt{mainmatter} and
% \texttt{backmatter} in a book. This mechanism is currently implemented
% by the commands \cs{DeclareCollectionInstance}, \cs{EditCollectionInstance}
% and \cs{UseCollection}. However, while the idea of switchable instances is
% a useful one, the team feel that collections are not the correct way to
% achieve this, at least with the current approach. As such, the collection
% functions should be regarded as deprecated: they remain available to support
% existing code, but will be removed when a better mechanism is developed.
%
%
% \begin{function}{\ShowCollectionInstanceValues}
%   \begin{syntax}
%    \cs{ShowInstanceValues} \Arg{collection} \Arg{object type} \Arg{instance}
%   \end{syntax}
%   Shows the \meta{values} for an \meta{instance} within
%   a \meta{collection} of the given \meta{object type} at the terminal.
%   As for other collection commands, this should be regarded as deprecated.
% \end{function}
%
% \end{documentation}
%
% \begin{implementation}
%
% \section{\pkg{xtemplate} implementation}
%
%    \begin{macrocode}
%<*package>
%    \end{macrocode}
%
%    \begin{macrocode}
%<@@=template>
%    \end{macrocode}
%
% \begin{macro}{\DeclareRestrictedTemplate}
% \begin{macro}{\DeclareObjectType}
%   If the new kernel code is not loaded, hand over to the frozen version.
%   Otherwise, make a couple of compatibility additions.
%    \begin{macrocode}
\@ifundefined{NewTemplateType}
  {%
    %%
%% This is file `xtemplate.sty',
%% generated with the docstrip utility.
%%
%% The original source files were:
%%
%% xtemplate.dtx  (with options: `package')
%% 
%% Copyright (C) 2011-2023 The LaTeX Project
%% 
%% It may be distributed and/or modified under the conditions of
%% the LaTeX Project Public License (LPPL), either version 1.3c of
%% this license or (at your option) any later version.  The latest
%% version of this license is in the file:
%% 
%%    http://www.latex-project.org/lppl.txt
%% 
%% This file is part of the "l3packages bundle" (The Work in LPPL)
%% and all files in that bundle must be distributed together.
%% 
%% File: xtemplate.dtx
\RequirePackage{expl3}[2018/02/21]
\@ifpackagelater{expl3}{2018/02/21}
  {}
  {%
    \PackageError{xtemplate}{Support package l3kernel too old}
      {%
        Please install an up to date version of l3kernel\MessageBreak
        using your TeX package manager or from CTAN.\MessageBreak
        \MessageBreak
        Loading xtemplate will abort!%
      }%
    \endinput
  }
\ProvidesExplPackage{xtemplate}{2023-10-10}{}
  {L3 Experimental prototype document functions}
\tl_const:Nn \c__xtemplate_code_root_tl      { template~code~>~ }
\tl_const:Nn \c__xtemplate_defaults_root_tl  { template~defaults~>~ }
\tl_const:Nn \c__xtemplate_instances_root_tl { template~instance~>~  }
\tl_const:Nn \c__xtemplate_keytypes_root_tl  { template~key~types~>~ }
\tl_const:Nn \c__xtemplate_key_order_root_tl { template~key~order~>~ }
\tl_const:Nn \c__xtemplate_restrict_root_tl  { template~restrictions~>~ }
\tl_const:Nn \c__xtemplate_values_root_tl    { template~values~>~ }
\tl_const:Nn \c__xtemplate_vars_root_tl      { template~vars~>~ }
\seq_const_from_clist:Nn \c__xtemplate_keytypes_arg_seq
  { choice , function , instance }
\prop_new:N \g__xtemplate_object_type_prop
\tl_new:N \l__xtemplate_assignments_tl
\tl_new:N \l__xtemplate_collection_tl
\prop_new:N \l__xtemplate_collections_prop
\tl_new:N \l__xtemplate_default_tl
\bool_new:N \l__xtemplate_error_bool
\bool_new:N \l__xtemplate_global_bool
\bool_new:N \l__xtemplate_restrict_bool
\clist_new:N \l__xtemplate_restrict_clist
\tl_new:N \l__xtemplate_key_name_tl
\tl_new:N \l__xtemplate_keytype_tl
\tl_new:N \l__xtemplate_keytype_arg_tl
\tl_new:N \l__xtemplate_value_tl
\tl_new:N \l__xtemplate_var_tl
\prop_new:N \l__xtemplate_keytypes_prop
\seq_new:N \l__xtemplate_key_order_seq
\prop_new:N \l__xtemplate_values_prop
\prop_new:N \l__xtemplate_vars_prop
\clist_new:N \l__xtemplate_tmp_clist
\dim_new:N \l__xtemplate_tmp_dim
\int_new:N \l__xtemplate_tmp_int
\muskip_new:N \l__xtemplate_tmp_muskip
\skip_new:N \l__xtemplate_tmp_skip
\tl_new:N \l__xtemplate_tmp_tl
\scan_new:N \s__xtemplate_mark
\scan_new:N \s__xtemplate_stop
\quark_new:N \q__xtemplate_nil
\__kernel_quark_new_conditional:Nn \__xtemplate_quark_if_nil:N { F }
\cs_generate_variant:Nn \prop_get:NnNTF { No }
\cs_generate_variant:Nn \prop_get:NnNT  { No }
\cs_generate_variant:Nn \prop_get:NnNF  { No }
\cs_new_protected:Npn \__xtemplate_execute_if_arg_agree:nnT #1#2#3
  {
    \prop_get:NnN \g__xtemplate_object_type_prop {#1} \l__xtemplate_tmp_tl
    \int_compare:nNnTF {#2} = \l__xtemplate_tmp_tl
       {#3}
       {
         \msg_error:nneee { xtemplate }
           { argument-number-mismatch } {#1} { \l__xtemplate_tmp_tl } {#2}
       }
  }
\cs_new_protected:Npn \__xtemplate_execute_if_code_exist:nnT #1#2#3
  {
    \cs_if_exist:cTF { \c__xtemplate_code_root_tl #1 / #2 }
      {#3}
      {
        \msg_error:nnee { xtemplate } { no-template-code }
          {#1} {#2}
      }
  }
\cs_new_protected:Npn \__xtemplate_execute_if_keytype_exist:nT #1#2
  {
    \cs_if_exist:cTF { __xtemplate_store_value_ #1 :n }
      {#2}
      { \msg_error:nne { xtemplate } { unknown-keytype } {#1} }
  }
\cs_generate_variant:Nn \__xtemplate_execute_if_keytype_exist:nT { o }
\cs_new_protected:Npn \__xtemplate_execute_if_type_exist:nT #1#2
  {
    \prop_if_in:NnTF \g__xtemplate_object_type_prop {#1}
      {#2}
      { \msg_error:nne { xtemplate } { unknown-object-type } {#1} }
  }
\cs_new_protected:Npn \__xtemplate_if_keys_exist:nnT #1#2#3
  {
    \cs_if_exist:cTF { \c__xtemplate_keytypes_root_tl #1 / #2 }
      {#3}
      {
        \msg_error:nnee { xtemplate } { unknown-template }
          {#1} {#2}
      }
   }
\prg_new_conditional:Npnn \__xtemplate_if_key_value:n #1 { T , F , TF }
  {
    \str_if_eq:noTF { \KeyValue } { \tl_head:w #1 \q_nil \q_stop }
      { \prg_return_true: }
      { \prg_return_false: }
  }
\cs_generate_variant:Nn \__xtemplate_if_key_value:nT  { o }
\cs_generate_variant:Nn \__xtemplate_if_key_value:nF  { o }
\cs_generate_variant:Nn \__xtemplate_if_key_value:nTF { o }
\prg_new_conditional:Npnn \__xtemplate_if_instance_exist:nnn #1#2#3
  { T, F, TF }
  {
    \cs_if_exist:cTF { \c__xtemplate_instances_root_tl #1 / #2 / #3 }
      { \prg_return_true: }
      { \prg_return_false: }
 }
\prg_new_conditional:Npnn \__xtemplate_if_use_template:n #1 { TF }
  {
    \str_if_eq:noTF { \UseTemplate } { \tl_head:w #1 \q_nil \q_stop }
      { \prg_return_true: }
      { \prg_return_false: }
}
\cs_new_protected:Npn \__xtemplate_store_defaults:n #1
  {
    \prop_gclear_new:c { \c__xtemplate_defaults_root_tl #1 }
    \prop_gset_eq:cN { \c__xtemplate_defaults_root_tl #1 }
      \l__xtemplate_values_prop
  }
\cs_new_protected:Npn \__xtemplate_store_keytypes:n #1
  {
    \prop_gclear_new:c { \c__xtemplate_keytypes_root_tl #1 }
    \prop_gset_eq:cN { \c__xtemplate_keytypes_root_tl #1 }
      \l__xtemplate_keytypes_prop
    \seq_gclear_new:c { \c__xtemplate_key_order_root_tl #1 }
    \seq_gset_eq:cN { \c__xtemplate_key_order_root_tl #1 }
      \l__xtemplate_key_order_seq
  }
\cs_new_protected:Npn \__xtemplate_store_values:n #1
  {
    \prop_clear_new:c { \c__xtemplate_values_root_tl #1 }
    \prop_set_eq:cN { \c__xtemplate_values_root_tl #1 }
      \l__xtemplate_values_prop
  }
\cs_new_protected:Npn \__xtemplate_store_restrictions:n #1
  {
    \clist_gclear_new:c { \c__xtemplate_restrict_root_tl #1 }
    \clist_gset_eq:cN { \c__xtemplate_restrict_root_tl #1 }
      \l__xtemplate_restrict_clist
  }
\cs_new_protected:Npn \__xtemplate_store_vars:n #1
  {
    \prop_gclear_new:c { \c__xtemplate_vars_root_tl #1 }
    \prop_gset_eq:cN { \c__xtemplate_vars_root_tl #1 }
      \l__xtemplate_vars_prop
  }
\cs_new_protected:Npn \__xtemplate_recover_defaults:n #1
  {
    \prop_if_exist:cTF
      { \c__xtemplate_defaults_root_tl #1 }
      {
        \prop_set_eq:Nc \l__xtemplate_values_prop
          { \c__xtemplate_defaults_root_tl #1 }
      }
      { \prop_clear:N \l__xtemplate_values_prop }
  }
\cs_new_protected:Npn \__xtemplate_recover_keytypes:n #1
  {
    \prop_if_exist:cTF
      { \c__xtemplate_keytypes_root_tl #1 }
      {
        \prop_set_eq:Nc \l__xtemplate_keytypes_prop
          { \c__xtemplate_keytypes_root_tl #1 }
      }
      { \prop_clear:N \l__xtemplate_keytypes_prop }
    \seq_if_exist:cTF { \c__xtemplate_key_order_root_tl #1 }
      {
        \seq_set_eq:Nc \l__xtemplate_key_order_seq
          { \c__xtemplate_key_order_root_tl #1 }
      }
      { \seq_clear:N \l__xtemplate_key_order_seq }
  }
\cs_new_protected:Npn \__xtemplate_recover_restrictions:n #1
  {
    \clist_if_exist:cTF
      { \c__xtemplate_restrict_root_tl #1 }
      {
        \clist_set_eq:Nc \l__xtemplate_restrict_clist
          { \c__xtemplate_restrict_root_tl #1 }
      }
      { \clist_clear:N \l__xtemplate_restrict_clist }
  }
\cs_new_protected:Npn \__xtemplate_recover_values:n #1
  {
    \prop_if_exist:cTF
      { \c__xtemplate_values_root_tl #1 }
      {
        \prop_set_eq:Nc \l__xtemplate_values_prop
          { \c__xtemplate_values_root_tl #1 }
      }
      { \prop_clear:N \l__xtemplate_values_prop }
  }
\cs_new_protected:Npn \__xtemplate_recover_vars:n #1
  {
    \prop_if_exist:cTF
      { \c__xtemplate_vars_root_tl #1 }
      {
        \prop_set_eq:Nc \l__xtemplate_vars_prop
          { \c__xtemplate_vars_root_tl #1 }
      }
      { \prop_clear:N \l__xtemplate_vars_prop }
  }
\cs_new_protected:Npn \__xtemplate_declare_object_type:nn #1#2
  {
    \int_set:Nn \l__xtemplate_tmp_int {#2}
    \int_compare:nTF { 0 <= \l__xtemplate_tmp_int <= 9 }
      {
        \msg_info:nnnV { xtemplate } { declare-object-type }
          {#1} \l__xtemplate_tmp_int
        \prop_gput:NnV \g__xtemplate_object_type_prop {#1}
          \l__xtemplate_tmp_int
      }
      {
        \msg_error:nnee { xtemplate } { bad-number-of-arguments }
          {#1} { \exp_not:V \l__xtemplate_tmp_int }
      }
  }
\cs_new_protected:Npn \__xtemplate_declare_template_keys:nnnn #1#2#3#4
  {
    \__xtemplate_execute_if_type_exist:nT {#1}
      {
        \__xtemplate_execute_if_arg_agree:nnT {#1} {#3}
          {
            \prop_clear:N \l__xtemplate_values_prop
            \prop_clear:N \l__xtemplate_keytypes_prop
            \seq_clear:N \l__xtemplate_key_order_seq
            \keyval_parse:NNn
              \__xtemplate_parse_keys_elt:n \__xtemplate_parse_keys_elt:nn {#4}
            \__xtemplate_store_defaults:n { #1 / #2 }
            \__xtemplate_store_keytypes:n { #1 / #2 }
          }
      }
  }
\cs_new_protected:Npn \__xtemplate_parse_keys_elt:n #1
  {
    \__xtemplate_split_keytype:n {#1}
    \bool_if:NF \l__xtemplate_error_bool
      {
        \__xtemplate_execute_if_keytype_exist:oT \l__xtemplate_keytype_tl
          {
            \seq_map_function:NN \c__xtemplate_keytypes_arg_seq
              \__xtemplate_parse_keys_elt_aux:n
            \bool_if:NF \l__xtemplate_error_bool
              {
                \seq_if_in:NoTF \l__xtemplate_key_order_seq
                  \l__xtemplate_key_name_tl
                  {
                    \msg_error:nne { xtemplate }
                      { duplicate-key-interface }
                      { \l__xtemplate_key_name_tl }
                  }
                  { \__xtemplate_parse_keys_elt_aux: }
              }
          }
      }
  }
\cs_new_protected:Npn \__xtemplate_parse_keys_elt_aux:n #1
  {
    \str_if_eq:onT \l__xtemplate_keytype_tl {#1}
      {
        \tl_if_empty:NT \l__xtemplate_keytype_arg_tl
          {
            \msg_error:nne { xtemplate }
              { keytype-requires-argument } {#1}
            \bool_set_true:N \l__xtemplate_error_bool
            \seq_map_break:
          }
      }
  }
\cs_new:Npn \__xtemplate_parse_keys_elt_aux:
  {
    \tl_set:Ne \l__xtemplate_tmp_tl
      {
        \l__xtemplate_keytype_tl
        \tl_if_empty:NF \l__xtemplate_keytype_arg_tl
          { { \l__xtemplate_keytype_arg_tl } }
      }
    \prop_put:Noo \l__xtemplate_keytypes_prop \l__xtemplate_key_name_tl
      \l__xtemplate_tmp_tl
    \seq_put_right:No \l__xtemplate_key_order_seq \l__xtemplate_key_name_tl
    \str_if_eq:onT \l__xtemplate_keytype_tl { choice }
      {
        \exp_args:No \clist_if_in:nnT \l__xtemplate_keytype_arg_tl { unknown }
          { \msg_error:nn { xtemplate } { choice-unknown-reserved } }
      }
  }
\cs_new_protected:Npn \__xtemplate_parse_keys_elt:nn #1#2
  {
    \__xtemplate_parse_keys_elt:n {#1}
    \use:c { __xtemplate_store_value_ \l__xtemplate_keytype_tl :n } {#2}
  }
\cs_new_protected:Npe \__xtemplate_split_keytype:n #1
  {
    \exp_not:N \bool_set_false:N \exp_not:N \l__xtemplate_error_bool
    \tl_set:Nn \exp_not:N \l__xtemplate_tmp_tl {#1}
    \tl_replace_all:Nnn \exp_not:N \l__xtemplate_tmp_tl { : } { \token_to_str:N : }
    \tl_if_in:onTF \exp_not:N \l__xtemplate_tmp_tl { \token_to_str:N : }
      {
        \exp_not:n
          {
            \tl_clear:N \l__xtemplate_key_name_tl
            \exp_after:wN \__xtemplate_split_keytype_aux:w
              \l__xtemplate_tmp_tl \s__xtemplate_stop
          }
      }
      {
        \exp_not:N \bool_set_true:N \exp_not:N \l__xtemplate_error_bool
        \msg_error:nne { xtemplate } { missing-keytype } {#1}
      }
  }
\use:e
  {
    \cs_new_protected:Npn \exp_not:N \__xtemplate_split_keytype_aux:w
      #1 \token_to_str:N : #2 \s__xtemplate_stop
      {
        \tl_put_right:Ne \exp_not:N \l__xtemplate_key_name_tl
          {
            \exp_not:N \tl_trim_spaces:e
              { \exp_not:N \tl_to_str:n {#1} }
          }
        \tl_if_in:nnTF {#2} { \token_to_str:N : }
          {
            \tl_put_right:Nn \exp_not:N \l__xtemplate_key_name_tl
              { \token_to_str:N : }
            \exp_not:N \__xtemplate_split_keytype_aux:w #2 \s__xtemplate_stop
          }
          {
            \exp_not:N \tl_if_empty:NTF \exp_not:N \l__xtemplate_key_name_tl
              {
                \msg_error:nne { xtemplate } { empty-key-name }
                  { \token_to_str:N : #2 }
              }
              { \exp_not:N \__xtemplate_split_keytype_arg:n {#2} }
          }
      }
  }
\cs_new_protected:Npn \__xtemplate_split_keytype_arg:n #1
  {
    \tl_set:Ne \l__xtemplate_keytype_tl { \tl_trim_spaces:n {#1} }
    \tl_clear:N \l__xtemplate_keytype_arg_tl
    \cs_set_protected:Npn \__xtemplate_split_keytype_arg_aux:n ##1
      {
        \tl_if_in:nnT {#1} {##1}
          {
            \cs_set:Npn \__xtemplate_split_keytype_arg_aux:w
              ####1 ##1 ####2 \s__xtemplate_stop
              {
                \tl_if_blank:nT {####1}
                  {
                    \tl_set:Ne \l__xtemplate_keytype_tl
                      { \tl_trim_spaces:n {##1} }
                    \tl_if_blank:nF {####2}
                      {
                        \tl_set:Ne \l__xtemplate_keytype_arg_tl
                          { \use:n ####2 }
                      }
                    \seq_map_break:
                  }
              }
            \__xtemplate_split_keytype_arg_aux:w #1 \s__xtemplate_stop
          }
      }
    \seq_map_function:NN \c__xtemplate_keytypes_arg_seq
      \__xtemplate_split_keytype_arg_aux:n
  }
\cs_generate_variant:Nn \__xtemplate_split_keytype_arg:n { o }
\cs_new:Npn \__xtemplate_split_keytype_arg_aux:n #1 { }
\cs_new:Npn \__xtemplate_split_keytype_arg_aux:w #1 \s__xtemplate_stop { }
\cs_new_protected:Npn \__xtemplate_store_value_boolean:n #1
  { \prop_put:Non \l__xtemplate_values_prop \l__xtemplate_key_name_tl {#1} }
\cs_new_protected:Npn \__xtemplate_store_value:n #1
  { \prop_put:Non \l__xtemplate_values_prop \l__xtemplate_key_name_tl {#1} }
\cs_new_eq:NN \__xtemplate_store_value_choice:n    \__xtemplate_store_value:n
\cs_new_eq:NN \__xtemplate_store_value_function:n  \__xtemplate_store_value:n
\cs_new_eq:NN \__xtemplate_store_value_instance:n  \__xtemplate_store_value:n
\cs_new_protected:Npn \__xtemplate_store_value_aux:Nn #1#2
  { \prop_put:Non \l__xtemplate_values_prop \l__xtemplate_key_name_tl {#2} }
\cs_new_protected:Npn \__xtemplate_store_value_integer:n
  { \__xtemplate_store_value_aux:Nn \int_eval:n }
\cs_new_protected:Npn \__xtemplate_store_value_length:n
  { \__xtemplate_store_value_aux:Nn \dim_eval:n }
\cs_new_protected:Npn \__xtemplate_store_value_muskip:n
  { \__xtemplate_store_value_aux:Nn \muskip_eval:n }
\cs_new_protected:Npn \__xtemplate_store_value_real:n
  { \__xtemplate_store_value_aux:Nn \fp_eval:n }
\cs_new_protected:Npn \__xtemplate_store_value_skip:n
  { \__xtemplate_store_value_aux:Nn \skip_eval:n }
\cs_new_protected:Npn \__xtemplate_store_value_tokenlist:n
  { \__xtemplate_store_value_aux:Nn \use:n }
\cs_new_eq:NN \__xtemplate_store_value_commalist:n \__xtemplate_store_value_tokenlist:n
\cs_new_protected:Npn \__xtemplate_declare_template_code:nnnnn #1#2#3#4#5
  {
    \__xtemplate_execute_if_type_exist:nT {#1}
      {
        \__xtemplate_execute_if_arg_agree:nnT {#1}{#3}
         {
          \__xtemplate_if_keys_exist:nnT {#1} {#2}
            {
              \__xtemplate_store_key_implementation:nnn {#1} {#2} {#4}
              \regex_match:nnTF { \c { AssignTemplateKeys } } {#5}
                { \__xtemplate_declare_template_code:nnn { #1 / #2 } {#3} {#5} }
                {
                  \__xtemplate_declare_template_code:nnn
                    { #1 / #2 } {#3} { \AssignTemplateKeys #5 }
                }
            }
         }
      }
  }
\cs_new_protected:Npn \__xtemplate_declare_template_code:nnn #1#2#3
  {
    \cs_generate_from_arg_count:cNnn
      { \c__xtemplate_code_root_tl #1 }
      \cs_gset_protected:Npn {#2} {#3}
  }
\cs_new_protected:Npn \__xtemplate_store_key_implementation:nnn #1#2#3
  {
    \__xtemplate_recover_defaults:n { #1 / #2 }
    \__xtemplate_recover_restrictions:n { #1 / #2 }
    \__xtemplate_recover_keytypes:n { #1 / #2 }
    \prop_clear:N \l__xtemplate_vars_prop
    \keyval_parse:nnn
      { \__xtemplate_parse_vars_elt:n } { \__xtemplate_parse_vars_elt:nnn { #1 / #2 } } {#3}
    \__xtemplate_store_vars:n { #1 / #2 }
    \__xtemplate_store_restrictions:n { #1 / #2 }
    \prop_map_inline:Nn \l__xtemplate_keytypes_prop
      {
        \msg_error:nneee { xtemplate } { key-not-implemented }
          {##1} {#2} {#1}
      }
  }
\cs_new_protected:Npn \__xtemplate_parse_vars_elt:n #1
  { \msg_error:nne { xtemplate } { key-no-variable } {#1} }
\cs_new_protected:Npn \__xtemplate_parse_vars_elt:nnn #1#2#3
 {
    \tl_set:Ne \l__xtemplate_key_name_tl
      { \tl_trim_spaces:e { \tl_to_str:n {#2} } }
    \prop_get:NoNTF
      \l__xtemplate_keytypes_prop
      \l__xtemplate_key_name_tl
      \l__xtemplate_keytype_tl
      {
        \__xtemplate_split_keytype_arg:o \l__xtemplate_keytype_tl
        \__xtemplate_parse_vars_elt_aux:nn {#1} {#3}
        \prop_remove:NV \l__xtemplate_keytypes_prop \l__xtemplate_key_name_tl
      }
      { \msg_error:nne { xtemplate } { unknown-key } {#2} }
  }
\cs_new_protected:Npn \__xtemplate_parse_vars_elt_aux:nn #1#2
  {
    \__xtemplate_parse_vars_elt_aux:nw {#1} #2 global global \s__xtemplate_stop
  }
\cs_new_protected:Npn \__xtemplate_parse_vars_elt_aux:nw
  #1#2 global #3 global #4 \s__xtemplate_stop
  {
    \tl_if_blank:nTF {#4}
      { \__xtemplate_parse_vars_elt_aux:nnn {#1} { } {#2} }
      {
        \tl_if_blank:nTF {#2}
          {
            \exp_args:Nnne \__xtemplate_parse_vars_elt_aux:nnn
              {#1} { global } { \tl_trim_spaces:n {#3} }
          }
          { \msg_error:nnn { xtemplate } { bad-variable } { #2 global #3 } }
      }
  }
\cs_new_protected:Npn \__xtemplate_parse_vars_elt_aux:nnn #1#2#3
  {
    \str_case:VnF \l__xtemplate_keytype_tl
      {
        { choice } { \__xtemplate_implement_choices:nn {#1} {#3} }
        { function }
          {
            \cs_if_exist:NF #3
              { \cs_new:Npn #3 { } }
            \__xtemplate_parse_vars_elt_key:nn {#1}
              {
                .code:n =
                  {
                    \cs_generate_from_arg_count:NNnn
                      \exp_not:N #3
                      \exp_not:c
                        { cs_ \str_if_eq:nnT {#1} { global } { g } set:Npn }
                      { \exp_not:o \l__xtemplate_keytype_arg_tl }
                      {##1}
                  }
              }
            \prop_put:Non \l__xtemplate_vars_prop
              \l__xtemplate_key_name_tl {#2#3}
          }
        { instance }
          {
            \__xtemplate_parse_vars_elt_key:nn {#1}
              {
                .code:n =
                  {
                    \exp_not:c
                      { cs_ \str_if_eq:nnT {#1} { global } { g } set:Npn }
                      \exp_not:N #3 { \UseInstance {##1} }
                  }
              }
            \prop_put:Non \l__xtemplate_vars_prop
              \l__xtemplate_key_name_tl {#2#3}
          }
      }
      {
        \tl_if_single:nTF {#3}
          {
            \cs_if_exist:NF #3
              { \use:c { \__xtemplate_map_var_type: _new:N } #3 }
            \__xtemplate_parse_vars_elt_key:nn {#1}
              {
                . \__xtemplate_map_var_type:
                  _ \str_if_eq:nnT {#1} { global } { g } set:N
                    = \exp_not:N #3
              }
            \prop_put:Non \l__xtemplate_vars_prop
              \l__xtemplate_key_name_tl {#2#3}
          }
          { \msg_error:nne { xtemplate } { bad-variable } { #2#3 } }
      }
  }
\cs_new_protected:Npn \__xtemplate_parse_vars_elt_key:nn #1#2
  {
    \keys_define:ne { template / #1 }
      { \l__xtemplate_key_name_tl #2 }
  }
\cs_new:Npn \__xtemplate_map_var_type:
  {
    \str_case:on \l__xtemplate_keytype_tl
      {
        { boolean }   { bool }
        { commalist } { clist }
        { integer }   { int }
        { length }    { dim }
        { muskip }    { muskip }
        { real }      { fp }
        { skip }      { skip }
        { tokenlist } { tl }
      }
  }
\cs_new_protected:Npn \__xtemplate_implement_choices:nn #1#2
  {
    \clist_set:No \l__xtemplate_tmp_clist { \l__xtemplate_keytype_arg_tl }
    \prop_put:Non \l__xtemplate_vars_prop \l__xtemplate_key_name_tl { }
    \keys_define:ne { template / #1 } { \l__xtemplate_key_name_tl .choice: }
    \keyval_parse:nnn
      { \__xtemplate_implement_choice_elt:n }
      { \__xtemplate_implement_choice_elt:nnn {#1} }
      {#2}
    \prop_get:NoNT \l__xtemplate_values_prop \l__xtemplate_key_name_tl
      \l__xtemplate_tmp_tl
      { \__xtemplate_implement_choices_default: }
    \clist_if_empty:NF \l__xtemplate_tmp_clist
      {
        \clist_map_inline:Nn \l__xtemplate_tmp_clist
          {
            \msg_error:nne { xtemplate } { choice-not-implemented }
              {##1}
          }
      }
  }
\cs_new_protected:Npn \__xtemplate_implement_choices_default:
  {
    \tl_set:Ne \l__xtemplate_tmp_tl
      { \l__xtemplate_key_name_tl \c_space_tl \l__xtemplate_tmp_tl }
    \prop_if_in:NoF \l__xtemplate_vars_prop \l__xtemplate_tmp_tl
      {
        \tl_set:Ne \l__xtemplate_tmp_tl
          { \l__xtemplate_key_name_tl \c_space_tl \l__xtemplate_tmp_tl }
        \prop_if_in:NoF \l__xtemplate_vars_prop \l__xtemplate_tmp_tl
          {
            \prop_get:NoN \l__xtemplate_keytypes_prop \l__xtemplate_key_name_tl
              \l__xtemplate_tmp_tl
            \__xtemplate_split_keytype_arg:o \l__xtemplate_tmp_tl
            \prop_get:NoN \l__xtemplate_values_prop \l__xtemplate_key_name_tl
              \l__xtemplate_tmp_tl
            \msg_error:nneee { xtemplate } { unknown-default-choice }
              { \l__xtemplate_key_name_tl } { \l__xtemplate_key_name_tl }
              { \l__xtemplate_keytype_arg_tl }
          }
      }
  }
\cs_new_protected:Npn \__xtemplate_implement_choice_elt:nnn #1#2#3
  {
    \clist_if_empty:NTF \l__xtemplate_tmp_clist
      {
        \str_if_eq:nnTF {#2} { unknown }
          { \__xtemplate_implement_choice_elt_aux:nnn {#1} {#2} {#3} }
          {
            \prop_get:NoN \l__xtemplate_keytypes_prop \l__xtemplate_key_name_tl
              \l__xtemplate_tmp_tl
            \__xtemplate_split_keytype_arg:o \l__xtemplate_tmp_tl
            \msg_error:nneee { xtemplate } { unknown-choice }
              { \l__xtemplate_key_name_tl } {#2}
              { \l__xtemplate_keytype_arg_tl }
          }
      }
      {
        \clist_if_in:NnTF \l__xtemplate_tmp_clist {#2}
          {
            \clist_remove_all:Nn \l__xtemplate_tmp_clist {#2}
            \__xtemplate_implement_choice_elt_aux:nnn {#1} {#2} {#3}
          }
          {
            \prop_get:NoN \l__xtemplate_keytypes_prop \l__xtemplate_key_name_tl
              \l__xtemplate_tmp_tl
            \__xtemplate_split_keytype_arg:o \l__xtemplate_tmp_tl
            \msg_error:nneee { xtemplate } { unknown-choice }
              { \l__xtemplate_key_name_tl } {#2}
              { \l__xtemplate_keytype_arg_tl }
          }
      }
  }
\cs_new_protected:Npn \__xtemplate_implement_choice_elt_aux:nnn #1#2#3
  {
    \keys_define:ne { template / #1 }
      { \l__xtemplate_key_name_tl / #2 .code:n = { \exp_not:n {#3} } }
    \tl_set:Ne \l__xtemplate_tmp_tl
      { \l__xtemplate_key_name_tl \c_space_tl #2 }
    \prop_put:Non \l__xtemplate_vars_prop \l__xtemplate_tmp_tl {#3}
  }
\cs_new_protected:Npn \__xtemplate_implement_choice_elt:n #1
  {
    \msg_error:nneee { xtemplate } { choice-requires-code }
      { \l__xtemplate_key_name_tl } {#1}
  }
\cs_new_protected:Npn \__xtemplate_declare_restricted:nnnn #1#2#3#4
  {
    \__xtemplate_if_keys_exist:nnT {#1} {#2}
      {
        \__xtemplate_set_template_eq:nn { #1 / #3 } { #1 / #2 }
        \bool_set_true:N \l__xtemplate_restrict_bool
        \__xtemplate_edit_defaults_aux:nnn {#1} {#3} {#4}
        \bool_set_false:N \l__xtemplate_restrict_bool
      }
  }
\cs_new_protected:Npn \__xtemplate_edit_defaults:nnn
  {
    \bool_set_false:N \l__xtemplate_restrict_bool
    \__xtemplate_edit_defaults_aux:nnn
  }
\cs_new_protected:Npn \__xtemplate_edit_defaults_aux:nnn #1#2#3
  {
    \__xtemplate_if_keys_exist:nnT {#1} {#2}
      {
        \__xtemplate_recover_defaults:n { #1 / #2 }
        \__xtemplate_recover_restrictions:n { #1 / #2 }
        \__xtemplate_parse_values:nn { #1 / #2 } {#3}
        \__xtemplate_store_defaults:n { #1 / #2 }
        \__xtemplate_store_restrictions:n { #1 / #2 }
      }
  }
\cs_new_protected:Npn \__xtemplate_parse_values:nn #1#2
  {
    \clist_clear:N \l__xtemplate_restrict_clist
    \__xtemplate_parse_values_aux:nn {#1} {#2}
  }
\cs_new_protected:Npn \__xtemplate_parse_values_aux:nn #1#2
  {
    \__xtemplate_recover_keytypes:n {#1}
    \keyval_parse:NNn
      \__xtemplate_parse_values_elt:n \__xtemplate_parse_values_elt:nn {#2}
  }
\cs_new_protected:Npn \__xtemplate_parse_values_elt:n #1
  {
    \bool_set_true:N \l__xtemplate_error_bool
    \msg_error:nne { xtemplate } { key-no-value } {#1}
  }
\cs_new_protected:Npn \__xtemplate_parse_values_elt:nn #1#2
  {
    \tl_set:Ne \l__xtemplate_key_name_tl
      { \tl_trim_spaces:e { \tl_to_str:n {#1} } }
    \prop_get:NoNTF \l__xtemplate_keytypes_prop \l__xtemplate_key_name_tl
      \l__xtemplate_tmp_tl
      {
        \bool_if:NTF \l__xtemplate_restrict_bool
          {
            \clist_if_in:NoF \l__xtemplate_restrict_clist
              \l__xtemplate_key_name_tl
                { \__xtemplate_parse_values_elt_aux:n {#2} }
          }
          { \__xtemplate_parse_values_elt_aux:n {#2} }
      }
      {
        \msg_error:nne { xtemplate } { unknown-key }
          { \l__xtemplate_key_name_tl }
      }
  }
\cs_new_protected:Npn \__xtemplate_parse_values_elt_aux:n #1
  {
    \clist_put_right:No \l__xtemplate_restrict_clist \l__xtemplate_key_name_tl
    \__xtemplate_split_keytype_arg:o \l__xtemplate_tmp_tl
    \use:c { __xtemplate_store_value_ \l__xtemplate_keytype_tl :n } {#1}
  }
\cs_new_protected:Npn \__xtemplate_set_template_eq:nn #1#2
  {
    \__xtemplate_recover_defaults:n {#2}
    \__xtemplate_store_defaults:n {#1}
    \__xtemplate_recover_keytypes:n {#2}
    \__xtemplate_store_keytypes:n {#1}
    \__xtemplate_recover_vars:n {#2}
    \__xtemplate_store_vars:n {#1}
    \cs_gset_eq:cc { \c__xtemplate_code_root_tl #1 }
      { \c__xtemplate_code_root_tl #2 }
  }
\cs_new_protected:Npn \__xtemplate_declare_instance:nnnnn #1#2#3#4#5
  {
    \__xtemplate_execute_if_code_exist:nnT {#1} {#2}
      {
        \__xtemplate_recover_defaults:n { #1 / #2 }
        \__xtemplate_recover_restrictions:n { #1 / #2 }
        \__xtemplate_recover_vars:n { #1 / #2 }
        \__xtemplate_declare_instance_aux:nnnnn {#1} {#2} {#3} {#4} {#5}
      }
  }
\cs_new_protected:Npn \__xtemplate_declare_instance_aux:nnnnn #1#2#3#4#5
  {
    \bool_set_false:N \l__xtemplate_error_bool
    \__xtemplate_parse_values_aux:nn { #1 / #2 } {#5}
    \bool_if:NF \l__xtemplate_error_bool
      {
        \prop_put:Nnn \l__xtemplate_values_prop { from~template } {#2}
        \__xtemplate_store_values:n { #1 / #3 / #4 }
        \__xtemplate_convert_to_assignments:
        \cs_set_protected:cpe { \c__xtemplate_instances_root_tl #1 / #3 / #4 }
          {
            \exp_not:N \__xtemplate_assignments_push:n
              { \exp_not:o \l__xtemplate_assignments_tl }
            \exp_not:c { \c__xtemplate_code_root_tl #1 / #2 }
          }
        \__xtemplate_if_instance_exist:nnnF {#1} { } {#4}
          {
            \cs_set_eq:cc
              { \c__xtemplate_instances_root_tl #1 /    / #4 }
              { \c__xtemplate_instances_root_tl #1 / #3 / #4 }
          }
      }
  }
\cs_new_protected:Npn \__xtemplate_copy_instance:nnnn #1#2#3#4
  {
    \__xtemplate_if_instance_exist:nnnTF {#1} {#2} {#3}
      {
        \__xtemplate_recover_values:n { #1 / #2 / #3 }
        \__xtemplate_store_values:n { #1 / #2 / #4 }
        \cs_set_eq:cc { \c__xtemplate_instances_root_tl #1 / #2 / #4 }
          { \c__xtemplate_instances_root_tl #1 / #2 / #3 }
      }
      {
        \msg_error:nnee { xtemplate } { unknown-instance }
          {#1} {#3}
      }
  }
\cs_new_protected:Npn \__xtemplate_edit_instance:nnnn #1#2#3#4
  {
    \__xtemplate_if_instance_exist:nnnTF {#1} {#2} {#3}
      {
        \__xtemplate_recover_values:n { #1 / #2 / #3 }
        \prop_get:NnN \l__xtemplate_values_prop { from~template }
          \l__xtemplate_tmp_tl
        \__xtemplate_edit_instance_aux:nonnn {#1} \l__xtemplate_tmp_tl
          {#2} {#3} {#4}
      }
      {
        \msg_error:nnee { xtemplate } { unknown-instance }
          {#1} {#3}
      }
  }
\cs_new_protected:Npn \__xtemplate_edit_instance_aux:nnnnn #1#2
  {
    \__xtemplate_recover_vars:n { #1 / #2 }
    \__xtemplate_declare_instance_aux:nnnnn {#1} {#2}
  }
\cs_generate_variant:Nn \__xtemplate_edit_instance_aux:nnnnn { no }
\cs_new_protected:Npn \__xtemplate_convert_to_assignments:
  {
    \tl_clear:N \l__xtemplate_assignments_tl
    \seq_map_function:NN \l__xtemplate_key_order_seq
      \__xtemplate_convert_to_assignments_aux:n
  }
\cs_new_protected:Npn \__xtemplate_convert_to_assignments_aux:n #1
  {
    \prop_get:NnN \l__xtemplate_keytypes_prop {#1} \l__xtemplate_tmp_tl
    \__xtemplate_convert_to_assignments_aux:no {#1} \l__xtemplate_tmp_tl
  }
\cs_new_protected:Npn \__xtemplate_convert_to_assignments_aux:nn #1#2
  {
    \prop_get:NnNT \l__xtemplate_values_prop {#1} \l__xtemplate_value_tl
      {
        \prop_get:NnNTF \l__xtemplate_vars_prop {#1} \l__xtemplate_var_tl
          {
            \__xtemplate_split_keytype_arg:n {#2}
            \str_if_eq:onF \l__xtemplate_keytype_tl { choice }
              {
                \str_if_eq:onF \l__xtemplate_keytype_tl { code }
                  { \__xtemplate_find_global: }
              }
            \tl_set:Nn \l__xtemplate_key_name_tl {#1}
            \use:c { __xtemplate_assign_ \l__xtemplate_keytype_tl : }
          }
          { \msg_error:nne { xtemplate } { unknown-attribute } {#1} }
      }
  }
\cs_generate_variant:Nn \__xtemplate_convert_to_assignments_aux:nn { no }
\cs_new_protected:Npn \__xtemplate_find_global:
  {
    \bool_set_false:N \l__xtemplate_global_bool
    \tl_if_in:onT \l__xtemplate_var_tl { global }
      {
        \exp_after:wN \__xtemplate_find_global_aux:w \l__xtemplate_var_tl \s__xtemplate_stop
      }
  }
\cs_new_protected:Npn \__xtemplate_find_global_aux:w  #1 global #2 \s__xtemplate_stop
  {
    \tl_set:Nn \l__xtemplate_var_tl {#2}
    \bool_set_true:N \l__xtemplate_global_bool
  }
\cs_new_protected:Npn \__xtemplate_use_template:nnn #1#2#3
  {
    \__xtemplate_execute_if_code_exist:nnT {#1} {#2}
      {
        \__xtemplate_recover_defaults:n { #1 / #2 }
        \__xtemplate_recover_vars:n { #1 / #2 }
        \__xtemplate_parse_values:nn { #1 / #2 } {#3}
        \__xtemplate_convert_to_assignments:
        \use:c { \c__xtemplate_code_root_tl #1 / #2  }
      }
  }
\cs_new_protected:Npn \__xtemplate_assign_boolean:
  {
    \bool_if:NTF \l__xtemplate_global_bool
      { \__xtemplate_assign_boolean_aux:n { bool_gset } }
      { \__xtemplate_assign_boolean_aux:n { bool_set } }
  }
\cs_new_protected:Npn \__xtemplate_assign_boolean_aux:n #1
  {
    \__xtemplate_if_key_value:oTF \l__xtemplate_value_tl
      {
        \__xtemplate_key_to_value:
        \tl_put_right:Ne \l__xtemplate_assignments_tl
          {
            \exp_not:c { #1 _eq:NN }
            \exp_not:o \l__xtemplate_var_tl
            \exp_not:o \l__xtemplate_value_tl
          }
      }
      {
        \tl_put_right:Ne \l__xtemplate_assignments_tl
          {
            \exp_not:c { #1 _ \l__xtemplate_value_tl :N }
            \exp_not:o \l__xtemplate_var_tl
          }
      }
  }
\cs_new_protected:Npn \__xtemplate_assign_choice:
  {
    \__xtemplate_assign_choice_aux:eF
      { \l__xtemplate_key_name_tl \c_space_tl \l__xtemplate_value_tl }
      {
        \__xtemplate_assign_choice_aux:eF
          { \l__xtemplate_key_name_tl \c_space_tl unknown }
          {
            \prop_get:NoN \l__xtemplate_keytypes_prop \l__xtemplate_key_name_tl
              \l__xtemplate_tmp_tl
            \__xtemplate_split_keytype_arg:o \l__xtemplate_tmp_tl
            \msg_error:nneee { xtemplate } { unknown-choice }
              { \l__xtemplate_key_name_tl } { \l__xtemplate_value_tl }
              { \l__xtemplate_keytype_arg_tl }
          }
      }
  }
\cs_new_protected:Npn \__xtemplate_assign_choice_aux:nF #1
  {
    \prop_get:NnNTF
      \l__xtemplate_vars_prop
      {#1}
      \l__xtemplate_tmp_tl
      { \tl_put_right:No \l__xtemplate_assignments_tl \l__xtemplate_tmp_tl }
  }
\cs_generate_variant:Nn \__xtemplate_assign_choice_aux:nF { e }
\cs_new_protected:Npn \__xtemplate_assign_function:
  {
    \bool_if:NTF \l__xtemplate_global_bool
      { \__xtemplate_assign_function_aux:N \cs_gset:Npn }
      { \__xtemplate_assign_function_aux:N \cs_set:Npn  }
  }
\cs_new_protected:Npn \__xtemplate_assign_function_aux:N #1
  {
    \tl_put_right:Ne \l__xtemplate_assignments_tl
      {
        \cs_generate_from_arg_count:NNnn
          \exp_not:o \l__xtemplate_var_tl
          \exp_not:N #1
          { \exp_not:o \l__xtemplate_keytype_arg_tl }
          { \exp_not:o \l__xtemplate_value_tl }
      }
  }
\cs_new_protected:Npn \__xtemplate_assign_instance:
  {
    \bool_if:NTF \l__xtemplate_global_bool
      { \__xtemplate_assign_instance_aux:N \cs_gset_protected:Npn }
      { \__xtemplate_assign_instance_aux:N \cs_set_protected:Npn  }
  }
\cs_new_protected:Npn \__xtemplate_assign_instance_aux:N #1
  {
    \tl_put_right:Ne \l__xtemplate_assignments_tl
      {
        \exp_not:N #1 \exp_not:o \l__xtemplate_var_tl
          {
            \__xtemplate_use_instance:nn
              { \exp_not:o \l__xtemplate_keytype_arg_tl }
              { \exp_not:o \l__xtemplate_value_tl }
          }
      }
  }
\cs_new_protected:Npn \__xtemplate_assign_integer:
  {
    \bool_if:NTF \l__xtemplate_global_bool
      { \__xtemplate_assign_variable:N \int_gset:Nn }
      { \__xtemplate_assign_variable:N \int_set:Nn  }
  }
\cs_new_protected:Npn \__xtemplate_assign_length:
  {
    \bool_if:NTF \l__xtemplate_global_bool
      { \__xtemplate_assign_variable:N \dim_gset:Nn }
      { \__xtemplate_assign_variable:N \dim_set:Nn  }
}
\cs_new_protected:Npn \__xtemplate_assign_muskip:
  {
    \bool_if:NTF \l__xtemplate_global_bool
      { \__xtemplate_assign_variable:N \muskip_gset:Nn }
      { \__xtemplate_assign_variable:N \muskip_set:Nn  }
  }
\cs_new_protected:Npn \__xtemplate_assign_real:
  {
    \bool_if:NTF \l__xtemplate_global_bool
      { \__xtemplate_assign_variable:N \fp_gset:Nn }
      { \__xtemplate_assign_variable:N \fp_set:Nn  }
  }
\cs_new_protected:Npn \__xtemplate_assign_skip:
  {
    \bool_if:NTF \l__xtemplate_global_bool
      { \__xtemplate_assign_variable:N \skip_gset:Nn }
      { \__xtemplate_assign_variable:N \skip_set:Nn  }
  }
\cs_new_protected:Npn \__xtemplate_assign_tokenlist:
  {
    \bool_if:NTF \l__xtemplate_global_bool
      { \__xtemplate_assign_tokenlist_aux:NN \tl_gset:NV \tl_gset:Nn }
      { \__xtemplate_assign_tokenlist_aux:NN \tl_set:NV  \tl_set:Nn }
  }
\cs_new_protected:Npn \__xtemplate_assign_tokenlist_aux:NN #1#2
  {
    \__xtemplate_if_key_value:oTF \l__xtemplate_value_tl
      {
        \__xtemplate_key_to_value:
        \tl_put_right:Ne \l__xtemplate_assignments_tl
          {
            #1 \exp_not:o \l__xtemplate_var_tl
              \exp_not:o \l__xtemplate_value_tl
          }
      }
      {
        \tl_put_right:Ne \l__xtemplate_assignments_tl
          {
            #2 \exp_not:o \l__xtemplate_var_tl
              { \exp_not:o \l__xtemplate_value_tl }
          }
      }
  }
\cs_new_protected:Npn \__xtemplate_assign_commalist:
  {
    \bool_if:NTF \l__xtemplate_global_bool
      { \__xtemplate_assign_tokenlist_aux:NN \clist_gset:NV \clist_gset:Nn }
      { \__xtemplate_assign_tokenlist_aux:NN \clist_set:NV  \clist_set:Nn }
  }
\cs_new_protected:Npn \__xtemplate_assign_variable:N #1
  {
    \__xtemplate_if_key_value:oT \l__xtemplate_value_tl
      { \__xtemplate_key_to_value: }
    \tl_put_right:Ne \l__xtemplate_assignments_tl
      {
        #1 \exp_not:o \l__xtemplate_var_tl
         { \exp_not:o \l__xtemplate_value_tl }
      }
  }
\cs_new_protected:Npn \__xtemplate_key_to_value:
  { \exp_after:wN \__xtemplate_key_to_value_auxi:w \l__xtemplate_value_tl }
\cs_new_protected:Npn \__xtemplate_key_to_value_auxi:w \KeyValue #1
  {
    \tl_set:Ne \l__xtemplate_tmp_tl { \tl_trim_spaces:e { \tl_to_str:n {#1} } }
    \prop_get:NoNTF
      \l__xtemplate_vars_prop
      \l__xtemplate_tmp_tl
      \l__xtemplate_value_tl
      {
        \exp_after:wN \__xtemplate_key_to_value_auxii:w \l__xtemplate_value_tl
          \s__xtemplate_mark global \q__xtemplate_nil \s__xtemplate_stop
      }
      {
        \msg_error:nne { xtemplate } { unknown-attribute }
          { \l__xtemplate_tmp_tl }
      }
  }
\cs_new_protected:Npn \__xtemplate_key_to_value_auxii:w #1 global #2#3 \s__xtemplate_stop
  {
    \__xtemplate_quark_if_nil:NF #2
      { \tl_set:Nn \l__xtemplate_value_tl {#2} }
  }
\cs_new_protected:Npn \__xtemplate_use_instance:nn #1#2
  {
    \__xtemplate_if_use_template:nTF {#2}
      { \__xtemplate_use_instance_aux:nNnnn {#1} #2 }
      { \__xtemplate_use_instance_aux:nn {#1} {#2} }
  }
\cs_new_protected:Npn \__xtemplate_use_instance_aux:nNnnn #1#2#3#4#5
  {
    \str_if_eq:nnTF {#1} {#3}
      { \__xtemplate_use_template:nnn {#3} {#4} {#5} }
      { \msg_error:nnee { xtemplate } { type-mismatch } {#1} {#3} }
}
\cs_new_protected:Npn \__xtemplate_use_instance_aux:nn #1#2
  {
    \__xtemplate_get_collection:n {#1}
    \__xtemplate_if_instance_exist:nnnTF
      {#1} { \l__xtemplate_collection_tl } {#2}
        {
          \use:c
            {
              \c__xtemplate_instances_root_tl #1 /
                \l__xtemplate_collection_tl / #2
            }
        }
        {
          \__xtemplate_if_instance_exist:nnnTF {#1} { } {#2}
            { \use:c { \c__xtemplate_instances_root_tl #1 / / #2 } }
            {
              \msg_error:nnee { xtemplate } { unknown-instance }
                {#1} {#2}
            }
        }
  }
\cs_new_protected:Npn \__xtemplate_use_collection:nn #1#2
  { \prop_put:Nnn \l__xtemplate_collections_prop {#1} {#2} }
\cs_new_protected:Npn \__xtemplate_get_collection:n #1
  {
    \prop_get:NnNF \l__xtemplate_collections_prop {#1}
      \l__xtemplate_collection_tl
      { \tl_clear:N \l__xtemplate_collection_tl }
  }
\cs_new:Npn \__xtemplate_assignments_pop: { \l__xtemplate_assignments_tl }
\cs_new_protected:Npn \__xtemplate_assignments_push:n #1
  { \tl_set:Nn \l__xtemplate_assignments_tl {#1} }
\cs_new_protected:Npn \__xtemplate_show_code:nn #1#2
  { \cs_show:c { \c__xtemplate_code_root_tl #1 / #2 } }
\cs_new_protected:Npn \__xtemplate_show_defaults:nn #1#2
  {
    \__xtemplate_if_keys_exist:nnT {#1} {#2}
      {
        \__xtemplate_recover_defaults:n { #1 / #2 }
        \__xtemplate_show:Nnnn \l__xtemplate_values_prop
          {#1} {#2} { default~values }
      }
  }
\cs_new_protected:Npn \__xtemplate_show_keytypes:nn #1#2
  {
    \__xtemplate_if_keys_exist:nnT {#1} {#2}
      {
        \__xtemplate_recover_keytypes:n { #1 / #2 }
        \__xtemplate_show:Nnnn \l__xtemplate_keytypes_prop
          {#1} {#2} { interface }
      }
  }
\cs_new_protected:Npn \__xtemplate_show_vars:nn #1#2
  {
     \__xtemplate_execute_if_code_exist:nnT {#1} {#2}
      {
        \__xtemplate_recover_vars:n { #1 / #2 }
        \__xtemplate_show:Nnnn \l__xtemplate_vars_prop
          {#1} {#2} { variable~mapping }
      }
  }
\cs_new_protected:Npn \__xtemplate_show:Nnnn #1#2#3#4
  {
    \msg_show:nneeee { xtemplate } { show-attribute }
      { \tl_to_str:n {#2} }
      { \tl_to_str:n {#3} }
      { \tl_to_str:n {#4} }
      { \prop_map_function:NN #1 \msg_show_item_unbraced:nn }
  }
\cs_new_protected:Npn \__xtemplate_show_values:nnn #1#2#3
  {
    \__xtemplate_if_instance_exist:nnnT {#1} {#2} {#3}
      {
        \__xtemplate_recover_values:n { #1 / #2 / #3 }
        \msg_show:nneeee { xtemplate } { show-values }
          { \tl_to_str:n {#1} }
          { \tl_to_str:n {#2} }
          { \tl_to_str:n {#3} }
          {
            \prop_map_function:NN \l__xtemplate_values_prop
              \msg_show_item_unbraced:nn
          }
      }
  }
\msg_new:nnnn { xtemplate } { argument-number-mismatch }
  { Object~type~'#1'~takes~#2~argument(s). }
  {
    Objects~of~type~'#1'~require~#2~argument(s).\\
    You~have~tried~to~make~a~template~for~'#1'~
    with~#3~argument(s),~which~is~not~possible:~
    the~number~of~arguments~must~agree.
  }
\msg_new:nnnn { xtemplate } { bad-number-of-arguments }
  { Bad~number~of~arguments~for~object~type~'#1'. }
  {
    An~object~may~accept~between~0~and~9~arguments.\\
    You~asked~to~use~#2~arguments:~this~is~not~supported.
  }
\msg_new:nnnn { xtemplate } { bad-variable }
  { Incorrect~variable~description~'#1'. }
  {
    The~argument~'#1'~is~not~of~the~form \\
    ~~'<variable>'\\
    ~or~\\
    ~~'global~<variable>'.\\
    It~must~be~given~in~one~of~these~formats~to~be~used~in~a~template.
  }
\msg_new:nnnn { xtemplate } { choice-not-implemented }
  { The~choice~'#1'~has~no~implementation. }
  {
    Each~choice~listed~in~the~interface~for~a~template~must~
    have~an~implementation.
  }
\msg_new:nnnn { xtemplate } { choice-no-code }
  { The~choice~'#1'~requires~implementation~details. }
  {
    When~creating~template~code~using~\DeclareTemplateCode,~
    each~choice~name~must~have~an~associated~implementation.\\
    This~should~be~given~after~a~'='~sign:~LaTeX~did~not~find~one.
  }
\msg_new:nnnn { xtemplate } { choice-requires-code }
  { The~choice~'#2'~for~key~'#1'~requires~an~implementation. }
  {
    You~should~have~put:\\
    \ \ #1~:~choice~{~#2 = <code> ~} \\
    but~LaTeX~did~not~find~any~<code>.
  }
\msg_new:nnnn { xtemplate } { duplicate-key-interface }
  { Key~'#1'~appears~twice~in~interface~definition~\msg_line_context:. }
  {
    Each~key~can~only~have~one~interface~declared~in~a~template.\\
    LaTeX~found~two~interfaces~for~'#1'.
  }
\msg_new:nnnn { xtemplate } { keytype-requires-argument }
  { The~key~type~'#1'~requires~an~argument~\msg_line_context:. }
  {
    You~should~have~put:\\
    \ \ <key-name>~:~#1~{~<argument>~} \\
    but~LaTeX~did~not~find~an~<argument>.
  }
\msg_new:nnnn { xtemplate } { invalid-keytype }
  { The~key~'#1'~is~missing~a~key-type~\msg_line_context:. }
  {
    Each~key~in~a~template~requires~a~key-type,~given~in~the~form:\\
    \ \ <key>~:~<key-type>\\
    LaTeX~could~not~find~a~<key-type>~in~your~input.
  }
\msg_new:nnnn { xtemplate } { key-no-value }
  { The~key~'#1'~has~no~value~\msg_line_context:. }
  {
    When~creating~an~instance~of~a~template~
    every~key~listed~must~include~a~value:\\
    \ \ <key>~=~<value>
  }
\msg_new:nnnn { xtemplate } { key-no-variable }
  { The~key~'#1'~requires~implementation~details~\msg_line_context:. }
  {
    When~creating~template~code~using~\DeclareTemplateCode,~
    each~key~name~must~have~an~associated~implementation.\\
    This~should~be~given~after~a~'='~sign:~LaTeX~did~not~find~one.
  }
\msg_new:nnnn { xtemplate } { key-not-implemented }
  { Key~'#1'~has~no~implementation~\msg_line_context:. }
  {
    The~definition~of~key~implementations~for~template~'#2'~
    of~object~type~'#3'~does~not~include~any~details~for~key~'#1'.\\
    The~key~was~declared~in~the~interface~definition,~
    and~so~an~implementation~is~required.
  }
\msg_new:nnnn { xtemplate } { missing-keytype }
  { The~key~'#1'~is~missing~a~key-type~\msg_line_context:. }
  {
    Key~interface~definitions~should~be~of~the~form\\
    \ \ #1~:~<key-type>\\
    but~LaTeX~could~not~find~a~<key-type>.
  }
\msg_new:nnnn { xtemplate } { no-template-code }
  {
    The~template~'#2'~of~type~'#1'~is~unknown~
    or~has~no~implementation.
  }
  {
    There~is~no~code~available~for~the~template~name~given.\\
    This~should~be~given~using~\DeclareTemplateCode.
  }
\msg_new:nnnn { xtemplate } { object-type-mismatch }
  { Object~types~'#1'~and~'#2'~do~not~agree. }
  {
    You~are~trying~to~use~a~template~directly~with~\UseInstance
    (or~a~similar~function),~but~the~object~types~do~not~match.
  }
\msg_new:nnnn { xtemplate } { unknown-attribute }
  { The~template~attribute~'#1'~is~unknown. }
  {
    There~is~a~definition~in~the~current~template~reading\\
    \ \ \token_to_str:N \KeyValue {~#1~} \\
    but~there~is~no~key~called~'#1'.
  }
\msg_new:nnnn { xtemplate } { unknown-choice }
  { The~choice~'#2'~was~not~declared~for~key~'#1'. }
  {
    The~key~'#1'~takes~a~fixed~list~of~choices~
    and~this~list~does~not~include~'#2'.
  }
\msg_new:nnnn { xtemplate } { unknown-default-choice }
  { The~default~choice~'#2'~was~not~declared~for~key~'#1'. }
  {
    The~key~'#1'~takes~a~fixed~list~of~choices~
    and~this~list~does~not~include~'#2'.
  }
\msg_new:nnnn { xtemplate } { unknown-instance }
  { The~instance~'#2'~of~type~'#1'~is~unknown. }
  {
    You~have~asked~to~use~an~instance~'#2',~
    but~this~has~not~been~created.
  }
\msg_new:nnnn { xtemplate } { unknown-key }
  { Unknown~template~key~'#1'. }
  {
    The~key~'#1'~was~not~declared~in~the~interface~
    for~the~current~template.
  }
\msg_new:nnnn { xtemplate } { unknown-keytype }
  { The~key-type~'#1'~is~unknown. }
  {
    Valid~key-types~are:\\
    -~boolean;\\
    -~choice;\\
    -~commalist;\\
    -~function;\\
    -~instance;\\
    -~integer;\\
    -~length;\\
    -~muskip;\\
    -~real;\\
    -~skip;\\
    -~tokenlist.
  }
\msg_new:nnnn { xtemplate } { unknown-object-type }
  { The~object~type~'#1'~is~unknown. }
  {
    An~object~type~needs~to~be~declared~with~\DeclareObjectType
    prior~to~using~it.
  }
\msg_new:nnnn { xtemplate } { unknown-template }
  { The~template~'#2'~of~type~'#1'~is~unknown. }
  {
    No~interface~has~been~declared~for~a~template~
    '#2'~of~object~type~'#1'.
  }
\msg_new:nnn { xtemplate } { declare-object-type }
  { Declaring~object~type~'#1'~taking~#2~argument(s)~\msg_line_context:. }
\msg_new:nnn { xtemplate } { declare-template-code }
  { Declaring~code~for~template~'#2'~of~object~type'#1'~\msg_line_context:. }
\msg_new:nnn { xtemplate } { declare-template-interface }
  {
    Declaring~interface~for~template~'#2'~of~object~type~'#1'~
    \msg_line_context:.
  }
\msg_new:nnn { xtemplate } { show-attribute }
  {
    The~template~'#2'~of~object~type~'#1'~has~
    \tl_if_empty:nTF {#4} { no~#3. } { #3 : #4 }
  }
\msg_new:nnn { xtemplate } { show-values }
  {
    \tl_if_empty:nTF {#2}
      { The~instance~'#3'~ }
      { The~collection~ instance~'#3'~ (from~collection~'#2')~ }
    of~object~type~'#1'~has~
    \tl_if_empty:nTF {#4} { no~values. } { values: #4 }
  }
\cs_new_protected:Npn \DeclareObjectType #1#2
  { \__xtemplate_declare_object_type:nn {#1} {#2} }
\cs_new_protected:Npn \DeclareTemplateInterface #1#2#3#4
  { \__xtemplate_declare_template_keys:nnnn {#1} {#2} {#3} {#4} }
\cs_new_protected:Npn \DeclareTemplateCode #1#2#3#4#5
  { \__xtemplate_declare_template_code:nnnnn {#1} {#2} {#3} {#4} {#5} }
\cs_new_protected:Npn \DeclareRestrictedTemplate #1#2#3#4
  { \__xtemplate_declare_restricted:nnnn {#1} {#2} {#3} {#4} }
\cs_new_protected:Npn \DeclareInstance #1#2#3#4
  { \__xtemplate_declare_instance:nnnnn {#1} {#3} { } {#2} {#4} }
\cs_new_protected:Npn \DeclareCollectionInstance #1#2#3#4#5
  { \__xtemplate_declare_instance:nnnnn {#2} {#4} {#1} {#3} {#5} }
\cs_new_protected:Npn \DeclareInstanceCopy #1#2#3
  { \__xtemplate_copy_instance:nnnn {#1} { } {#3} {#2} }
\cs_new_protected:Npn \EditTemplateDefaults #1#2#3
  { \__xtemplate_edit_defaults:nnn {#1} {#2} {#3} }
\cs_new_protected:Npn \EditInstance #1#2#3
  { \__xtemplate_edit_instance:nnnn {#1} { } {#2} {#3} }
\cs_new_protected:Npn \EditCollectionInstance #1#2#3#4
  { \__xtemplate_edit_instance:nnnn {#2} {#1} {#3} {#4} }
\cs_new_protected:Npn \UseTemplate #1#2#3
  { \__xtemplate_use_template:nnn {#1} {#2} {#3} }
\cs_new_protected:Npn \UseInstance #1#2
  { \__xtemplate_use_instance:nn {#1} {#2} }
\cs_new_protected:Npn \UseCollection #1#2
  { \__xtemplate_use_collection:nn {#1} {#2} }
\cs_new_protected:Npn \ShowTemplateCode #1#2
  { \__xtemplate_show_code:nn {#1} {#2} }
\cs_new_protected:Npn \ShowTemplateDefaults #1#2
  { \__xtemplate_show_defaults:nn {#1} {#2} }
\cs_new_protected:Npn \ShowTemplateInterface #1#2
  { \__xtemplate_show_keytypes:nn {#1} {#2} }
\cs_new_protected:Npn \ShowTemplateVariables #1#2
  { \__xtemplate_show_vars:nn {#1} {#2} }
\cs_new_protected:Npn \ShowInstanceValues #1#2
  { \__xtemplate_show_values:nnn {#1} { } {#2} }
\cs_new_protected:Npn \ShowCollectionInstanceValues #1#2#3
  { \__xtemplate_show_values:nnn {#1} {#2} {#3} }
\cs_new:Npn \IfInstanceExistTF #1#2
  { \__xtemplate_if_instance_exist:nnnTF {#1} { } {#2} }
\cs_new:Npn \IfInstanceExistT #1#2
  { \__xtemplate_if_instance_exist:nnnT {#1} { } {#2} }
\cs_new:Npn \IfInstanceExistF #1#2
  { \__xtemplate_if_instance_exist:nnnF {#1} { } {#2} }
\cs_new_protected:Npn \KeyValue #1 {#1}
\cs_new_protected:Npn \AssignTemplateKeys
  { \__xtemplate_assignments_pop: }
\cs_new_protected:Npn \SetTemplateKeys #1#2#3
  { \keys_set_known:nnN { template / #1 / #2 } {#3} \l__xtemplate_tmp_clist }
\cs_new_eq:NN \ShowTemplateKeytypes \ShowTemplateInterface
%% 
%%
%% End of file `xtemplate.sty'.
%
    \endinput
  }
  {%
    \ProvidesExplPackage{xtemplate}{2024-08-16}{}
      {L3 Experimental prototype document functions}
    \long\protected\def\DeclareRestrictedTemplate#1#2#3#4{%
      \DeclareTemplateCopy{#1}{#3}{#2}%
      \EditTemplateDefaults{#1}{#3}{#4}%
    }%
    \let\IfInstanceExistT\IfInstanceExistsT
    \let\IfInstanceExistF\IfInstanceExistsF
    \let\IfInstanceExistTF\IfInstanceExistsTF
  }
\ExplSyntaxOn
\cs_new_protected:Npn \DeclareObjectType #1#2
  { \@@_declare_type:nn {#1} {#2} }
\ExplSyntaxOff
%    \end{macrocode}
% \end{macro}
% \end{macro}
%
%    \begin{macrocode}
%</package>
%    \end{macrocode}
%
%\end{implementation}
%
%\PrintIndex
