%
% $Id$
%
% (C) Copyright 1997, 2004 Frank Mittelbach, Chris Rowley
%
%
\iffalse

Now updated by Chris: I wonder how well it matches the implementation?
Apart from typos and improved consistency, the main changes are to the
THD pointers (and a bit to SRs).

Note: the "earlier thoughts" are mostly now irrelevant.

as i'm sure you all are totally eager to see the real spec and are
only therefore waiting with your comments ... here it is

not at all ready but i think if one leaves out special regions and
puts in a lot of thoughts into the THD pointer model in fact already
implementable

as i said before any thoughts (or an implementation? :-)
are very welcome


frank

\fi


\documentclass{ltxguide}

\catcode`\<=\active
\catcode`\>=\active

\usepackage{shortvrb}  \MakeShortVerb{\|}

\begin{document}

\title{Language Information in
       Structured Documents:
       A spec for \LaTeX{}
}

\author{Frank Mittelbach \\ \texttt{fmitte01@de.eds.com}
   \and Chris Rowley \\ \texttt{C.A.Rowley@open.ac.uk}}

\date{21. April 1997}

\maketitle

\tableofcontents



\section{Notes}

As ever, the details of the command names have not had much
thought:-).

the SR and moving text bits are at best to be called vague (but i hope
that will change throughout the week)



\section{Compatibility problems}

\begin{itemize}

\item Processing in |\document|, unless it can be done via |\AtBeginDocument|.

\item SR specifications: these need, in principle, to be addable to any
  existing environment or command.
\end{itemize}



\section{Interfaces}


\subsection{Preamble Interfaces}

The following commands are intended to be used in class and/or package
files or, in some cases, in the preamble of the document to overwrite
decisions made in class or package files.

\begin{decl}
|\DeclareLanguageCommand| <cmd>
\end{decl}

declares <cmd> to be a language command used by the current package or
class. The purpose of this declaration is to announce to the \LaTeX{}
system that this class or package uses the <cmd> (or even commands, it
could be a comma separated list if we wish) as a language command and
that therefore any declarations made via |\SetLATvalue| or
|\SetPAMvalue| for this command are to be kept in the body of the
document.

-- add <cmd> to |\usedlanguagecmds| if necessary



\begin{decl}
|\SetLATvalue| <language> <cmd> <action>
\end{decl}

sets the action to <action> for <cmd> in <language>. To set the
default action to be anything other than a fixed error message, use
\texttt{default} as <language>. Any such declaration is kept only up
to the begin of the actual document unless <cmd> was declared within a
package or class with |\DeclareLanguageCommand|.

-- check that <language> is in the list |\knownlanguages|, if not, add
it to |\knownlanguages| and initiate |\language|<language>|defs|.

-- add or modify <cmd>/<action> pair in |\language|<language>|defs|.




\begin{decl}
|\SetPAMvalue| <cmd> <THD-type-or-level>
\end{decl}

sets the PAM value to <THD-type-or-level> for <cmd>. Any such
declaration is kept only up to the beginning of the actual document
unless <cmd> was declared within a package or class with
|\DeclareLanguageCommand|.  If no PAM value is declared for <cmd>, a
global default is used.

-- add or modify an entry in |\knownPAMvalues|.


\begin{decl}
|\DeclareDocumentLanguages| <language-list>
\end{decl}

lists all languages that are used in the document (any other language
will then generate an error message if it appears in a language change
command or environment).\footnote{Instead of enforcing the use of such a
command one could determine the used languages when parsing the
document for the first time but this will result in some time and
space penalty which is probably better avoided.} On a higher level
this could be specified via the option names to the language package.

-- produces, or adds to, the |\usedlanguages| list.



\begin{decl}
|\SetDocumentLanguage| <language>
\end{decl}

declares <language> to be the document language. It will also set the
base language to use at the start of the body of the document.  Again
this can also be set by making the last option to the language package
become the base language.


\subsection{Moving text}

What is needed:
\begin{itemize}
\item A data-structure and mutator for attaching language-information.
  to moving text.
\item An accessor for examining language-information attached to moved
  text, possibly combined with adding a language tag only when
  necessary, (ie when the language of the moved text is not the same
  as the current language).
\end{itemize}




\subsection{Special regions}

To be used within the code for an SR:

\begin{decl}
|\SetSRStartLanguage| <lang>       \\
|\SetSRInitLevel| <THD-type+level> \\
|\ExecuteSRStarttag|               \\
|\SetSRInnerLevel| <THD-type+level>
\end{decl}

these could be combined into one command with three parameters, if they
turn out to always be used together (the first two could be combined
anyway);

\begin{decl}
|\SRbegin| \\\
|\SRend|
\end{decl}

these are needed for use in SR commands where outer-values may need to
be stored since there may not be a group to restore their values.

Or:

for an existing environment or command, called <name>:

\begin{decl}
|\DeclareSREnvironment| <name> <start-lang-method>
    <init-node-THD-type+level> <inner-node-THD-type+level> \\
|\DeclareSRCommand| <name> <start-lang-method> <init-node-THD-type+level>
    <inner-node-THD-type+level>
\end{decl}



\subsection{Body Interfaces}

\begin{itemize}
\item
  A base-language change command with the language-label as
  argument.  This command is declarative to highlight the flat
  structure of base languages.

\item
  A language-environment with the language-label as argument and text
  as body.  The environment starts a new paragraph to enforce the block
  level nature.

\item
  A language-command with the language-label and text both as
  arguments. In contrast to the environment, this command applies
  language related actions to its second argument, which cannot
  directly contain full paragraphs.
\end{itemize}





\section{Internal structures}

None of the command names have sensible implementation names; but for
understanding what goes on this is probably better.

\subsection{Preamble structures}


\begin{decl}
|\usedlanguagecmds|
\end{decl}

list of commands declared by the |\DeclareLanguageCommand|
declarations found while parsing class and package files (and the
preamble although |\DeclareLanguageCommand| would normally not be
found here).  This list is used to determine, at |\begin{document}|,
which of the LAT and PAM declarations are actually needed for the
current document.


\begin{decl}
|\usedlanguages|
\end{decl}

list of languages that are declared to be used in the document via
|\DeclareDocumentLanguages|, etc.  An attempted change to a language
not found on this list will produce an error message.


\begin{decl}
|\knownlanguages|
\end{decl}

list of languages for which there was at least one |\SetLATvalue|
command.  These languages are known to the set-up mechanism but may
not be used in a particular document; this variable is used to keep
track of the internal structures like |\language|<language>|defs|
(since we need to know which of these to throw away later on).


\begin{decl}
|\knownPAMvalues|
\end{decl}

list of <cmd>/<PAM-value> pairs that have been explicitly set by
|\SetPAMvalue|.  (The name doesn't really fit with the other
|\known...| things, says Frank.)


\begin{decl}
|\language|<language>|defs|
\end{decl}

list of <cmd>/<action> pairs for language <language>. Such a variable
exists for each <language> in |\knownlanguages|.
It is updated by |\SetLATvalue|.



\subsection{Special regions}

For SRs (depending on the interface) we might have:

\begin{decl}
|\knownSRs|
\end{decl}

a list of SRs with, for each, its start-lang-method, init- and
inner-node-type+level.


\subsection{Body structures}


*** this probably still needs some cleanup ***

For each language in |\usedlanguage| and each <THD-node> in the THD:

\begin{decl}
|\actions|<language><THD-node>
\end{decl}

an `explicit' list of what happens, for that <THD-node>, when changing to
language <language> at a node of type or level <THD-node>.


For each <THD-node> in the THD:

\begin{decl}
|\THDactions|<THD-node>
\end{decl}

a list containing all the <action-node>s for which the command (with
the value for <language> set) |\actions|<language><action-node> must
be executed at a tag of that <THD-node-type>.  This could be made
dynamic (ie it would just contain the list for the current tag) and
combined with the first two pointers, |\THDpointer...|, below.

\begin{decl}
|\THDpointer...|
\end{decl}

a collection of pointers to nodes (either types or levels) in the THD.
At any time the expansions of these are:

  <THD-node-level of current language-tag>

  <THD-node-type of current language-tag>

  <THD-node-level of next nested language-tag>

  for each type <type> of language tag:\\
    <THD-node-type of next nested language-tag if the next
                                tag's type is \m{type}>

The first two of these could be combined with a dynamic version of
|\THDactions|.


\begin{decl}
|\THDpointersupdate|
\end{decl}

called at every language tag to update the above pointers.


\subsection{Special regions}

For SRs (depending on interface):
special language-tag for each SR.

\begin{decl}
|\savedTHDpointers|
\end{decl}

to save the outer values of all the |\THDpointer...| pointers when an SR
comes from a command.

A mechanism to set the correct values of all the |\THDpointer...|
pointers according to init-level.


\section{The preamble to body algorithm}


\begin{verbatim}
for each <cmd> in |\usedlanguagecmds|, do:

  "find PAM-value" and store in <PAMvalue>;

  for each <lang> in |\usedlanguages|, do:

    "find action for <cmd> in language <lang>";

    add `explicit' form of this action to |\actions|<lang><PAMvalue>.

for each <lang> in |\knownlanguages|, do:

  empty |\language|<lang>|defs|.

empty all other preamble structures.


"find PAM-value"

if PAM-value for <cmd> is in |\knownPAMvalues|,
  then
    retrieve it
  else
    use fixed default value.


"find action for <cmd> in language <lang>"

if <cmd> has entry in |\language|<lang>|defs|,
  then
    retrieve action
  else
    if <cmd> has entry in |\languagedefaultdefs|,
      then
        retrieve action
      else
        define action to be ERROR: <cmd> has no declaration for
                                     language <lang>
                            HELP: declare action using
                                     |\SetLATvalue| ...
\end{verbatim}


Add an internal command to each SR that:

-- evaluates the start-lang;

-- inserts the required type of language tag using start-lang;

-- resets the |\THDpointer...| pointers as for inner-node-type+level
(and stores the outer-level value if necessary).

Add an internal command to each SR that:

-- restores the outer-value of each |\THDpointer...| if necessary (ie
at the end of a command).


\section{Earlier thoughts by Chris}

Some thoughts on implementing the language model outlined in ***.

This may become a latex file sometime.

Extra stuff for spec:

-- actions must be order-independent since order of execution is not
specifiable.

-- need to separate completely \TeX's |\language| from \LaTeX's
language-label (the latter may correspond to babel's' ldf files?).

-- also, each \TeX{} |\language| needs to be associated with a
pair: (patterns-name, encoding); then the action |\hyph-fool|
needs to, inter alia)

-- note: each encoding also defines an lccode table; only changes to
patterns whose encodings use the same lccode table should be allowed
"within a paragraph", otherwise the "empty-patterns" should be used
(these work with any encoding).


Questions at the spec level:

-- Can language-settings of parameters (as specified in the PET) be
made local?

-- If they are local, will \LaTeX's grouping mechanism restore
everything correctly?  Or will some |\aftergroup| stuff be needed?

-- Maybe "model by parameter setting" is not quite right; better:
     model by action (ie a macro is expanded).

-- What is the default assigned to an action if no explicit value is
given?  That of the document language?  Or some more global default?

-- How do actions interact with conflicting settings outside language
tags?

-- Are the values of these actions global (literally): ie defined in
the lang.ldf file and not changeable within a class file or a document?

-- Is the PAT just set-up once or does it need to be incrementally
changeable?


Questions at the implementation level:

-- How to store method for each action/lang combination:
     one csname per pair?
     can something like short refs be reduced to one macro per lang
     (yes, as done now in babel?)?

-- Is `inherit via doc hierarchy' ever useful?  (It is easy to
implement as below.)

-- How to safely allow: use value from another lang?
   Maybe we can allow only a concept of "sub-lang": a sub-lang
   of fool inherits everything from fool (except some things that are
   explicitly set differently).
   Or "use default"?  Is "default" == "value for document lang".


Simple implementation (without interfaces and without any interaction
with other formatting settings):

-- define a csname |\action-lang| for every pair to expand to the
required value (or |\relax| if inherited via doc);

-- build an executable list for each node in the THD with
one csname per action with that stopping level, starting with those
done at all levels on that sub-branch, and with marker csnames at
level boundaries (all set to |\empty| by default);

-- at a lang-change to lang-label |fool| set the marker(s) for the
effective current level (on each branch) to |\GobbleRestOfList| then
execute the lists (in a compatible order) so that |\action| executes
|\action-fool| (if it is not gobbled).



\end{document}
