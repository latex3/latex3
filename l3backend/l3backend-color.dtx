% \iffalse meta-comment
%
%% File: l3backend-color.dtx
%
% Copyright (C) 2019-2025 The LaTeX Project
%
% It may be distributed and/or modified under the conditions of the
% LaTeX Project Public License (LPPL), either version 1.3c of this
% license or (at your option) any later version.  The latest version
% of this license is in the file
%
%    https://www.latex-project.org/lppl.txt
%
% This file is part of the "l3backend bundle" (The Work in LPPL)
% and all files in that bundle must be distributed together.
%
% -----------------------------------------------------------------------
%
% The development version of the bundle can be found at
%
%    https://github.com/latex3/latex3
%
% for those people who are interested.
%
%<*driver>
\documentclass[full,kernel]{l3doc}
\begin{document}
  \DocInput{\jobname.dtx}
\end{document}
%</driver>
% \fi
%
% \title{^^A
%   The \pkg{l3backend-color} module\\ Backend color support^^A
% }
%
% \author{^^A
%  The \LaTeX{} Project\thanks
%    {^^A
%      E-mail:
%        \href{mailto:latex-team@latex-project.org}
%          {latex-team@latex-project.org}^^A
%    }^^A
% }
%
% \date{Released 2025-04-14}
%
% \maketitle
%
% \begin{documentation}
%
% \end{documentation}
%
% \begin{implementation}
%
% \section{\pkg{l3backend-color} implementation}
%
%    \begin{macrocode}
%<*package>
%<@@=color>
%    \end{macrocode}
%
% Color support is split into parts: collecting data from \LaTeXe{}, the color
% stack, general color, separations, and color for drawings. We have different
% approaches in each backend, and have some choices to make about
% \texttt{dvipdfmx}/\XeTeX{} in particular. Whilst it is in some ways
% convenient to use the same approach in multiple backends, the fact that
% \texttt{dvipdfmx}/\XeTeX{} is PDF-based means it (largely) sticks closer to
% direct PDF output.
%
% \subsection{The color stack}
%
% For PDF-based engines, we have a color stack available inside the specials.
% This is used for concepts beyond color itself: it is needed to manage the
% graphics state generally. Although \texttt{dvipdfmx}/\XeTeX{} have multiple
% color stacks in recent releases, the way these interact with the original
% single stack and with other graphic state operations means that currently it
% is not feasible to use the multiple stacks.
%
% \subsubsection{Common code}
%
%    \begin{macrocode}
%<*luatex|pdftex>
%    \end{macrocode}
%
% \begin{variable}{\l_@@_backend_stack_int}
%   For tracking which stack is in use where multiple stacks are used:
%   currently just \pdfTeX{}/\LuaTeX{} but at some future stage may also cover
%   \texttt{dvipdfmx}/\XeTeX{}.
%    \begin{macrocode}
\int_new:N \l_@@_backend_stack_int
%    \end{macrocode}
% \end{variable}
%
%    \begin{macrocode}
%</luatex|pdftex>
%    \end{macrocode}
%
% \subsubsection{\LuaTeX and \pdfTeX{}}
%
%    \begin{macrocode}
%<*luatex|pdftex>
%    \end{macrocode}
%
% \begin{macro}{\__kernel_color_backend_stack_init:Nnn}
%    \begin{macrocode}
\cs_new_protected:Npn \__kernel_color_backend_stack_init:Nnn #1#2#3
  {
    \int_const:Nn #1
      {
%<*luatex>
        \tex_pdffeedback:D colorstackinit ~
%</luatex>
%<*pdftex>
        \tex_pdfcolorstackinit:D
%</pdftex>
        \tl_if_blank:nF {#2} { #2 ~ }
        {#3}
      }
  }
%    \end{macrocode}
% \end{macro}
%
% \begin{macro}{\__kernel_color_backend_stack_push:nn}
% \begin{macro}{\__kernel_color_backend_stack_pop:n}
%    \begin{macrocode}
\cs_new_protected:Npn \__kernel_color_backend_stack_push:nn #1#2
  {
%<*luatex>
    \tex_pdfextension:D colorstack ~
%</luatex>
%<*pdftex>
    \tex_pdfcolorstack:D
%</pdftex>
      \int_eval:n {#1} ~ push ~ {#2}
  }
\cs_new_protected:Npn \__kernel_color_backend_stack_pop:n #1
  {
%<*luatex>
    \tex_pdfextension:D colorstack ~
%</luatex>
%<*pdftex>
    \tex_pdfcolorstack:D
%</pdftex>
      \int_eval:n {#1} ~ pop \scan_stop:
  }
%    \end{macrocode}
% \end{macro}
% \end{macro}
%
%    \begin{macrocode}
%</luatex|pdftex>
%    \end{macrocode}
%
% \subsection{General color}
%
% \subsubsection{\texttt{dvips}-style}
%
%    \begin{macrocode}
%<*dvips|dvisvgm>
%    \end{macrocode}
%
% \begin{macro}
%   {
%     \@@_backend_select_cmyk:n  ,
%     \@@_backend_select_gray:n  ,
%     \@@_backend_select_named:n ,
%     \@@_backend_select_rgb:n   ,
%     \@@_backend_select:n
%   }
% \begin{macro}{\@@_backend_reset:, \@@_backend_reset_aux:}
%    Push the data to the stack. In the case of \texttt{dvips} also saves the
%    drawing color in raw PostScript. The |spot| model is for handling data
%    in classical format.
%    \begin{macrocode}
\cs_new_protected:Npn \@@_backend_select_cmyk:n #1
  { \@@_backend_select:n { cmyk ~ #1 } }
\cs_new_protected:Npn \@@_backend_select_gray:n #1
  { \@@_backend_select:n { gray ~ #1 } }
\cs_new_protected:Npn \@@_backend_select_named:n #1
  { \@@_backend_select:n { ~ #1 } }
\cs_new_protected:Npn \@@_backend_select_rgb:n #1
  { \@@_backend_select:n { rgb ~ #1 } }
\cs_new_protected:Npn \@@_backend_select:n #1
  {
    \__kernel_backend_literal:n { color~push~ #1 }
%<*dvips>
    \__kernel_backend_postscript:n { /color.sc ~ { } ~ def }
%</dvips>
  }
\cs_new_protected:Npn \@@_backend_reset:
  { \group_insert_after:N \@@_backend_reset_aux: }
\cs_new_protected:Npn \@@_backend_reset_aux:
  { \__kernel_backend_literal:n { color~pop } }
%    \end{macrocode}
% \end{macro}
% \end{macro}
%
%    \begin{macrocode}
%</dvips|dvisvgm>
%    \end{macrocode}
%
% \subsubsection{\LuaTeX{} and \pdfTeX{}}
%
%    \begin{macrocode}
%<*luatex|pdftex>
%    \end{macrocode}
%
% \begin{variable}{\l_@@_backend_fill_tl, \l_@@_backend_stroke_tl}
%    \begin{macrocode}
\tl_new:N \l_@@_backend_fill_tl
\tl_new:N \l_@@_backend_stroke_tl
\tl_set:Nn \l_@@_backend_fill_tl { 0 ~ g }
\tl_set:Nn \l_@@_backend_stroke_tl { 0 ~ G }
%    \end{macrocode}
% \end{variable}
%
% \begin{macro}
%   {
%     \@@_backend_select_cmyk:n ,
%     \@@_backend_select_gray:n ,
%     \@@_backend_select_rgb:n
%   }
% \begin{macro}{\@@_backend_select:nn}
% \begin{macro}{\@@_backend_reset:, \@@_backend_reset_aux:}
% \begin{macro}{\@@_backend_stack_push:nn}
% \begin{macro}
%   {
%     \@@_backend_stack_push:nnn ,
%     \@@_backend_stack_push:nVn ,
%     \@@_backend_stack_push:nnV
%   }
%   Store the values then pass to the stack. The abstraction here allows for
%   transparency needing the same fundamental approach but with switchable
%   code for \LuaTeX{}.
%    \begin{macrocode}
\cs_new_protected:Npn \@@_backend_select_cmyk:n #1
  { \@@_backend_select:nn { #1 ~ k } { #1 ~ K } }
\cs_new_protected:Npn \@@_backend_select_gray:n #1
  { \@@_backend_select:nn { #1 ~ g } { #1 ~ G } }
\cs_new_protected:Npn \@@_backend_select_rgb:n #1
  { \@@_backend_select:nn { #1 ~ rg } { #1 ~ RG } }
\cs_new_protected:Npn \@@_backend_select:nn #1#2
  {
    \tl_set:Nn \l_@@_backend_fill_tl {#1}
    \tl_set:Nn \l_@@_backend_stroke_tl {#2}
    \@@_backend_stack_push:nn \l_@@_backend_stack_int { #1 ~ #2 }
  }
\cs_new_protected:Npn \@@_backend_reset:
  { \group_insert_after:N \@@_backend_reset_aux: }
\cs_new_protected:Npn \@@_backend_reset_aux:
  { \__kernel_color_backend_stack_pop:n \l_@@_backend_stack_int }
\cs_new_eq:NN \@@_backend_stack_push:nn \__kernel_color_backend_stack_push:nn
\cs_new_protected:Npn \@@_backend_stack_push:nnn #1#2#3
  { \@@_backend_stack_push:nn {#1} { #2 ~ #3 } }
\cs_generate_variant:Nn \@@_backend_stack_push:nnn { nV , nnV }
%    \end{macrocode}
% \end{macro}
% \end{macro}
% \end{macro}
% \end{macro}
% \end{macro}
%
%    \begin{macrocode}
%</luatex|pdftex>
%    \end{macrocode}
%
% \subsubsection{\texttt{dvipmdfx}/\XeTeX{}}
%
% These backends have the most possible approaches: it recognizes both
% \texttt{dvips}-based color specials and its own format, plus one can
% include PDF statements directly. Recent releases also have a color stack
% approach similar to \pdfTeX{}. Of the stack methods, the dedicated
% the most versatile is the latter as it can cover all of the use cases
% we have. However, at present this interacts problematically with any color
% on the original stack. We therefore stick to a single-stack approach here.
%
%    \begin{macrocode}
%<*dvipdfmx|xetex>
%    \end{macrocode}
%
% \begin{macro}
%   {
%     \@@_backend_select:n      ,
%     \@@_backend_select_cmyk:n ,
%     \@@_backend_select_gray:n ,
%     \@@_backend_select_rgb:n
%   }
% \begin{macro}{\@@_backend_reset:, \@@_backend_reset_aux:}
%   Using the single stack is relatively easy as there is only one route.
%    \begin{macrocode}
\cs_new_protected:Npn \@@_backend_select:n #1
  { \__kernel_backend_literal:n { pdf : bc ~ [ #1 ] } }
\cs_new_eq:NN \@@_backend_select_cmyk:n \@@_backend_select:n
\cs_new_eq:NN \@@_backend_select_gray:n \@@_backend_select:n
\cs_new_eq:NN \@@_backend_select_rgb:n  \@@_backend_select:n
\cs_new_protected:Npn \@@_backend_reset:
  { \group_insert_after:N \@@_backend_reset_aux: }
\cs_new_protected:Npn \@@_backend_reset_aux:
  { \__kernel_backend_literal:n { pdf : ec } }
%    \end{macrocode}
% \end{macro}
% \end{macro}
%
% \begin{macro}{\@@_backend_select_named:n}
%   For classical named colors, the only value we should get is |Black|.
%    \begin{macrocode}
\cs_new_protected:Npn \@@_backend_select_named:n #1
  {
    \str_if_eq:nnTF {#1} { Black }
      { \@@_backend_select_gray:n { 0 } }
      { \msg_error:nnn { color } { unknown-named-color } {#1} }
  }
\msg_new:nnn { color } { unknown-named-color }
  { Named~color~'#1'~is~not~known. }
%    \end{macrocode}
% \end{macro}
%
%    \begin{macrocode}
%</dvipdfmx|xetex>
%    \end{macrocode}
%
% \subsection{Separations}
%
% Here, life gets interesting and we need essentially one approach per
% backend.
%
%    \begin{macrocode}
%<*dvipdfmx|luatex|pdftex|xetex|dvips>
%    \end{macrocode}
%
% But we start with some functionality needed for both PostScript and
% PDF based backends.
%
% \begin{variable}{\g_@@_backend_colorant_prop}
%    \begin{macrocode}
\prop_new:N \g_@@_backend_colorant_prop
%    \end{macrocode}
% \end{variable}
%
% \begin{macro}[EXP]{\@@_backend_devicen_colorants:n}
% \begin{macro}[EXP]{\@@_backend_devicen_colorants:w}
%    \begin{macrocode}
\cs_new:Npe \@@_backend_devicen_colorants:n #1
  {
    \exp_not:N \tl_if_blank:nF {#1}
      {
        \c_space_tl
        << ~
          /Colorants ~
            << ~
              \exp_not:N \@@_backend_devicen_colorants:w #1 ~
                \exp_not:N \q_recursion_tail \c_space_tl
                \exp_not:N \q_recursion_stop
            >> ~
        >>
      }
  }
\cs_new:Npn \@@_backend_devicen_colorants:w #1 ~
  {
    \quark_if_recursion_tail_stop:n {#1}
    \prop_if_in:NnT \g_@@_backend_colorant_prop {#1}
      {
        #1 ~
        \prop_item:Nn \g_@@_backend_colorant_prop {#1} ~
      }
    \@@_backend_devicen_colorants:w
  }
%    \end{macrocode}
% \end{macro}
% \end{macro}
%
%    \begin{macrocode}
%</dvipdfmx|luatex|pdftex|xetex|dvips>
%    \end{macrocode}
%
%    \begin{macrocode}
%<*dvips>
%    \end{macrocode}
%
% \begin{macro}{\@@_backend_select_separation:nn, \@@_backend_select_devicen:nn}
%    \begin{macrocode}
\cs_new_protected:Npn \@@_backend_select_separation:nn #1#2
  { \@@_backend_select:n { separation ~ #1 ~ #2 } }
\cs_new_eq:NN \@@_backend_select_devicen:nn \@@_backend_select_separation:nn
%    \end{macrocode}
% \end{macro}
%
% \begin{macro}{\@@_backend_select_iccbased:nn}
%   No support.
%    \begin{macrocode}
\cs_new_protected:Npn \@@_backend_select_iccbased:nn #1#2 { }
%    \end{macrocode}
% \end{macro}
%
% \begin{macro}
%   {
%     \@@_backend_separation_init:nnnnn,
%     \@@_backend_separation_init:neenn
%   }
% \begin{macro}
%   {\@@_backend_separation_init_aux:nnnnnn}
% \begin{macro}[EXP]
%   {
%     \@@_backend_separation_init_/DeviceCMYK:nnn ,
%     \@@_backend_separation_init_/DeviceGray:nnn ,
%     \@@_backend_separation_init_/DeviceRGB:nnn
%   }
% \begin{macro}[EXP]{\@@_backend_separation_init_Device:Nn}
% \begin{macro}[EXP]{\@@_backend_separation_init:nnn}
% \begin{macro}[EXP]{\@@_backend_separation_init_count:n}
% \begin{macro}[EXP]{\@@_backend_separation_init_count:w}
% \begin{macro}[EXP]{\@@_backend_separation_init:nnnn}
% \begin{macro}[EXP]{\@@_backend_separation_init:w}
% \begin{macro}[EXP]{\@@_backend_separation_init:n}
% \begin{macro}[EXP]{\@@_backend_separation_init:nw}
% \begin{macro}{\@@_backend_separation_init_CIELAB:nnn}
%   Initializing here means creating a small header set up plus massaging
%   some data. This comes about as we have to deal with PDF-focussed data,
%   which makes most sense \enquote{higher-up}. The approach is based on
%   ideas from \url{https://tex.stackexchange.com/q/560093} plus using
%   the PostScript manual for other aspects.
%    \begin{macrocode}
\cs_new_protected:Npe \@@_backend_separation_init:nnnnn #1#2#3#4#5
  {
    \bool_if:NT \g__kernel_backend_header_bool
      {
        \exp_not:N \exp_args:Ne \__kernel_backend_first_shipout:n
          {
            \exp_not:N \@@_backend_separation_init_aux:nnnnnn
              { \exp_not:N \int_use:N \g_@@_model_int }
              {#1} {#2} {#3} {#4} {#5}
          }
        \prop_gput:Nee \exp_not:N \g_@@_backend_colorant_prop
          { / \exp_not:N \str_convert_pdfname:n {#1} }
          {
            << ~
              /setcolorspace ~ {} ~
            >> ~ begin ~
              color \exp_not:N \int_use:N \g_@@_model_int \c_space_tl
            end
          }
      }
  }
\cs_generate_variant:Nn \@@_backend_separation_init:nnnnn { nee }
\cs_new_protected:Npn \@@_backend_separation_init_aux:nnnnnn #1#2#3#4#5#6
  {
    \__kernel_backend_literal:e
      {
        !
        TeXDict ~ begin ~
        /color #1
          {
            [ ~
              /Separation ~ ( \str_convert_pdfname:n {#2} ) ~
              [ ~ #3 ~ ] ~
                {
                  \cs_if_exist_use:cF { @@_backend_separation_init_ #3 :nnn }
                    { \@@_backend_separation_init:nnn }
                      {#4} {#5} {#6}
                }
            ] ~ setcolorspace
          } ~ def ~
        end
      }
  }
\cs_new:cpn { @@_backend_separation_init_ /DeviceCMYK :nnn } #1#2#3
  { \@@_backend_separation_init_Device:Nn 4 {#3} }
\cs_new:cpn { @@_backend_separation_init_ /DeviceGray :nnn } #1#2#3
  { \@@_backend_separation_init_Device:Nn 1 {#3} }
\cs_new:cpn { @@_backend_separation_init_ /DeviceRGB :nnn } #1#2#3
  { \@@_backend_separation_init_Device:Nn 2 {#3} }
\cs_new:Npn \@@_backend_separation_init_Device:Nn #1#2
  {
    #2 ~
    \prg_replicate:nn {#1}
      { #1 ~ index ~ mul ~ #1 ~ 1 ~ roll ~ }
    \int_eval:n { #1 + 1 } ~ -1 ~ roll ~ pop
  }
%    \end{macrocode}
%   For the generic case, we cannot use |/FunctionType 2| unfortunately, so
%   we have to code that idea up in PostScript. Here, we will therefore assume
%   that a range is \emph{always} given. First, we count values in each argument:
%   at the backend level, we can assume there are always well-behaved with
%   spaces present.
%    \begin{macrocode}
\cs_new:Npn \@@_backend_separation_init:nnn #1#2#3
  {
    \exp_args:Ne \@@_backend_separation_init:nnnn
      { \@@_backend_separation_init_count:n {#2} }
      {#1} {#2} {#3}
  }
\cs_new:Npn \@@_backend_separation_init_count:n #1
  { \int_eval:n { 0 \@@_backend_separation_init_count:w #1 ~ \s_@@_stop } }
\cs_new:Npn \@@_backend_separation_init_count:w #1 ~ #2 \s_@@_stop
  {
    +1
    \tl_if_blank:nF {#2}
      { \@@_backend_separation_init_count:w #2 \s_@@_stop }
  }
%    \end{macrocode}
%   Now we implement the algorithm. In the terms in the PostScript manual,
%   we have $\mathbf{N} = 1$ and $\mathbf{Domain} = [0~1]$, with
%   $\mathbf{Range}$ as |#2|, $\mathbf{C0}$ as |#3| and $\mathbf{C1}$
%   as |#4|, with the number of output components in |#1|. So all we have
%   to do is implement $y_{i} = \mathbf{C0}_{i} + x(\mathbf{C1}_{i} -
%   \mathbf{C0}_{i})$ with lots of stack manipulation, then check the
%   ranges. That's done by adding everything to the stack first, then using
%   the fact we know all of the offsets. As manipulating the stack is tricky,
%   we start by re-formatting the $\mathbf{C0}$ and $\mathbf{C1}$ arrays to
%   be interleaved, and add a \texttt{0} to each pair: this is used
%   to keep the stack of constant length while we are doing the first pass of
%   mathematics. We then working through that list, calculating from the
%   last to the first value before tidying up by removing all of the input
%   values. We do that by first copying all of the final $y$ values to the
%   end of the stack, then rolling everything so we can pop the now-unneeded
%   material.
%    \begin{macrocode}
\cs_new:Npn \@@_backend_separation_init:nnnn #1#2#3#4
  {
    \@@_backend_separation_init:w #3 ~ \s_@@_stop #4 ~ \s_@@_stop
    \prg_replicate:nn {#1}
      {
        pop ~ 1 ~ index ~ neg ~ 1 ~ index ~ add ~
        \int_eval:n { 3 * #1 } ~ index ~ mul ~
        2 ~ index ~ add ~
        \int_eval:n { 3 * #1 } ~ #1 ~ roll ~
      }
    \int_step_function:nnnN {#1} { -1 } { 1 }
      \@@_backend_separation_init:n
    \int_eval:n { 4 * #1 + 1 } ~ #1 ~ roll ~
    \prg_replicate:nn { 3 * #1 + 1 } { pop ~ }
    \tl_if_blank:nF {#2}
      { \@@_backend_separation_init:nw {#1} #2 ~ \s_@@_stop }
  }
\cs_new:Npn \@@_backend_separation_init:w
  #1 ~ #2 \s_@@_stop #3 ~ #4 \s_@@_stop
  {
    #1 ~ #3 ~ 0 ~
    \tl_if_blank:nF {#2}
      { \@@_backend_separation_init:w #2 \s_@@_stop #4 \s_@@_stop }
  }
\cs_new:Npn \@@_backend_separation_init:n #1
  { \int_eval:n { #1 * 2 } ~ index ~ }
%    \end{macrocode}
%   Finally, we deal with the range limit if required. This is handled
%   by splitting the range into pairs. It's then just a question of doing
%   the comparisons, this time dropping everything except the desired
%   result.
%    \begin{macrocode}
\cs_new:Npn \@@_backend_separation_init:nw #1#2 ~ #3 ~ #4 \s_@@_stop
  {
    #2 ~ #3 ~
    2 ~ index ~ 2 ~ index ~ lt ~
      { ~ pop ~ exch ~ pop ~ } ~
      { ~
        2 ~ index ~ 1 ~ index ~ gt ~
          { ~ exch ~ pop ~ exch ~ pop ~ } ~
          { ~ pop ~ pop ~ } ~
        ifelse ~
      }
    ifelse ~
    #1 ~ 1 ~ roll ~
    \tl_if_blank:nF {#4}
      { \@@_backend_separation_init:nw {#1} #4 \s_@@_stop }
  }
%    \end{macrocode}
%  CIELAB support uses the detail from the PostScript reference, page 227;
%  other than that block of PostScript, this is the same as for PDF-based
%  routes.
%    \begin{macrocode}
\cs_new_protected:Npn \@@_backend_separation_init_CIELAB:nnn #1#2#3
  {
    \@@_backend_separation_init:neenn
      {#2}
      {
        /CIEBasedABC ~
            << ~
              /RangeABC ~ [ ~ \c_@@_model_range_CIELAB_tl \c_space_tl ] ~
              /DecodeABC ~
                [ ~
                  { ~ 16 ~ add ~ 116 ~ div ~ } ~ bind ~
                  { ~ 500 ~ div ~ } ~ bind ~
                  { ~ 200 ~ div ~ } ~ bind ~
                ] ~
              /MatrixABC ~ [ ~ 1 ~ 1 ~ 1 ~ 1 ~ 0 ~ 0 ~ 0 ~ 0 ~ -1 ~ ] ~
              /DecodeLMN ~
                [ ~
                  { ~
                    dup ~ 6 ~ 29 ~ div ~ ge ~
                      { ~ dup ~ dup ~ mul ~ mul ~ ~ } ~
                      { ~ 4 ~ 29 ~ div ~ sub ~ 108 ~ 841 ~ div ~ mul ~ } ~
                    ifelse ~
                    0.9505 ~ mul ~
                  } ~ bind ~
                  { ~
                    dup ~ 6 ~ 29 ~ div ~ ge ~
                      { ~ dup ~ dup ~ mul ~ mul ~ } ~
                      { ~ 4 ~ 29 ~ div ~ sub ~ 108 ~ 841 ~ div ~ mul ~ } ~
                    ifelse ~
                  } ~ bind ~
                  { ~
                    dup ~ 6 ~ 29 ~ div ~ ge ~
                      { ~ dup ~ dup ~ mul ~ mul ~ } ~
                      { ~ 4 ~ 29 ~ div ~ sub ~ 108 ~ 841 ~ div ~ mul ~ } ~
                    ifelse ~
                    1.0890 ~ mul ~
                  } ~ bind
                ] ~
              /WhitePoint ~
                [ ~ \tl_use:c { c_@@_model_whitepoint_CIELAB_ #1 _tl } ~ ] ~
            >>
      }
      { \c_@@_model_range_CIELAB_tl }
      { 100 ~ 0 ~ 0 }
      {#3}
  }
%    \end{macrocode}
% \end{macro}
% \end{macro}
% \end{macro}
% \end{macro}
% \end{macro}
% \end{macro}
% \end{macro}
% \end{macro}
% \end{macro}
% \end{macro}
% \end{macro}
% \end{macro}
%
% \begin{macro}{\@@_backend_devicen_init:nnn}
%   Trivial as almost all of the work occurs in the shared code.
%    \begin{macrocode}
\cs_new_protected:Npn \@@_backend_devicen_init:nnn #1#2#3
  {
    \__kernel_backend_literal:e
      {
        !
        TeXDict ~ begin ~
        /color \int_use:N \g_@@_model_int
          {
            [ ~
              /DeviceN ~
              [ ~ #1 ~ ] ~
              #2 ~
              { ~ #3 ~ } ~
              \@@_backend_devicen_colorants:n {#1}
            ] ~ setcolorspace
          } ~ def ~
        end
      }
  }
%    \end{macrocode}
% \end{macro}
%
% \begin{macro}{\@@_backend_iccbased_init:nnn}
%   No support at present.
%    \begin{macrocode}
\cs_new_protected:Npn \@@_backend_iccbased_init:nnn #1#2#3 { }
%    \end{macrocode}
% \end{macro}
%
%    \begin{macrocode}
%</dvips>
%    \end{macrocode}
%
%    \begin{macrocode}
%<*dvisvgm>
%    \end{macrocode}
%
% \begin{macro}
%   {
%     \@@_backend_select_separation:nn,
%     \@@_backend_select_devicen:nn
%   }
%   No support at present.
%    \begin{macrocode}
\cs_new_protected:Npn \@@_backend_select_separation:nn #1#2 { }
\cs_new_eq:NN \@@_backend_select_devicen:nn \@@_backend_select_separation:nn
%    \end{macrocode}
% \end{macro}
%
% \begin{macro}
%   {\@@_backend_separation_init:nnnnn, \@@_backend_separation_init_CIELAB:nnn}
%   No support at present.
%    \begin{macrocode}
\cs_new_protected:Npn \@@_backend_separation_init:nnnnn #1#2#3#4#5 { }
\cs_new_protected:Npn \@@_backend_separation_init_CIELAB:nnnnnn #1#2#3 { }
%    \end{macrocode}
% \end{macro}
%
% \begin{macro}{\@@_backend_select_iccbased:nn}
%   As detailed in \url{https://www.w3.org/TR/css-color-4/#at-profile},
%   we can apply a color profile using CSS. As we have a local file, we use
%   a relative URL.
%    \begin{macrocode}
\cs_new_protected:Npn \@@_backend_select_iccbased:nn #1#2
  {
    \__kernel_backend_literal_svg:e
      {
        <style>
          @color-profile ~
            \str_if_eq:nnTF {#2} { cmyk }
              { device-cmyk }
              { --color \int_use:N \g_@@_model_int }
                \c_space_tl
            {
              src:("#1")
            }
        </style>
      }
  }
%    \end{macrocode}
% \end{macro}
%
%    \begin{macrocode}
%</dvisvgm>
%    \end{macrocode}
%
%    \begin{macrocode}
%<*dvipdfmx|luatex|pdftex|xetex>
%    \end{macrocode}
%
% \begin{macro}
%   {
%     \@@_backend_select_separation:nn,
%     \@@_backend_select_devicen:nn   ,
%     \@@_backend_select_iccbased:nn
%   }
%    \begin{macrocode}
%<*dvipdfmx|xetex>
\cs_new_protected:Npn \@@_backend_select_separation:nn #1#2
  { \__kernel_backend_literal:e { pdf : bc ~ \pdf_object_ref:n {#1} ~ [ #2 ] } }
%</dvipdfmx|xetex>
%<*luatex|pdftex>
\cs_new_protected:Npn \@@_backend_select_separation:nn #1#2
  { \@@_backend_select:nn { /#1 ~ cs ~ #2 ~ scn  } { /#1 ~ CS ~ #2 ~ SCN } }
%</luatex|pdftex>
\cs_new_eq:NN \@@_backend_select_devicen:nn \@@_backend_select_separation:nn
\cs_new_eq:NN \@@_backend_select_iccbased:nn \@@_backend_select_separation:nn
%    \end{macrocode}
% \end{macro}
%
% \begin{macro}{\@@_backend_init_resource:n}
%   Resource initiation comes up a few times. For \texttt{dvipdfmx}/\XeTeX{},
%   we skip this as at present it's handled by the backend.
%    \begin{macrocode}
\cs_new_protected:Npn \@@_backend_init_resource:n #1
  {
%<*luatex|pdftex>
    \bool_lazy_and:nnT
      { \cs_if_exist_p:N \pdfmanagement_if_active_p: }
      { \pdfmanagement_if_active_p: }
      {
        \use:e
          {
            \pdfmanagement_add:nnn
              { Page / Resources / ColorSpace }
              { #1 }
              { \pdf_object_ref_last: }
          }
      }
%</luatex|pdftex>
  }
%    \end{macrocode}
% \end{macro}
%
% \begin{macro}{\@@_backend_separation_init:nnnnn}
% \begin{macro}{\@@_backend_separation_init:nn}
% \begin{macro}{\@@_backend_separation_init_CIELAB:nnn}
%   Initializing the PDF structures needs two parts: creating an object
%   containing the \enquote{real} name of the Separation, then adding a reference
%   to that to each page. We use a separate object for the tint transformation
%   following the model in the PDF reference. The object here for the color
%   needs to be named as that way it's accessible to
%   \texttt{dvipdfmx}/\XeTeX{}.
%    \begin{macrocode}
\cs_new_protected:Npn \@@_backend_separation_init:nnnnn #1#2#3#4#5
  {
    \pdf_object_unnamed_write:ne { dict }
      {
        /FunctionType ~ 2
        /Domain ~ [0 ~ 1]
        \tl_if_blank:nF {#3} { /Range ~ [#3] }
        /C0 ~ [#4] ~
        /C1 ~ [#5] /N ~ 1
      }
    \exp_args:Ne \@@_backend_separation_init:nn
      { \str_convert_pdfname:n {#1} } {#2}
    \@@_backend_init_resource:n { color \int_use:N \g_@@_model_int }
  }
\cs_new_protected:Npn \@@_backend_separation_init:nn #1#2
  {
    \use:e
      {
        \pdf_object_new:n { color \int_use:N \g_@@_model_int }
        \pdf_object_write:nnn { color \int_use:N \g_@@_model_int } { array }
          { /Separation /#1 ~ #2 ~ \pdf_object_ref_last: }
      }
    \prop_gput:Nne \g_@@_backend_colorant_prop { /#1 }
      { \pdf_object_ref_last: }
  }
%    \end{macrocode}
%   For CIELAB colors, we need one object per document for the illuminant,
%   plus initialization of the color space referencing that object.
%    \begin{macrocode}
\cs_new_protected:Npn \@@_backend_separation_init_CIELAB:nnn #1#2#3
  {
    \pdf_object_if_exist:nF { @@_illuminant_CIELAB_ #1 }
      {
        \pdf_object_new:n { @@_illuminant_CIELAB_ #1 }
        \pdf_object_write:nne { @@_illuminant_CIELAB_ #1 } { array }
          {
            /Lab ~
            <<
              /WhitePoint ~
                [ \tl_use:c { c_@@_model_whitepoint_CIELAB_ #1 _tl } ]
              /Range ~ [ \c_@@_model_range_CIELAB_tl ]
            >>
          }
      }
    \@@_backend_separation_init:nnnnn
      {#2}
      { \pdf_object_ref:n { @@_illuminant_CIELAB_ #1 } }
      { \c_@@_model_range_CIELAB_tl }
      { 100 ~ 0 ~ 0 }
      {#3}
  }
%    \end{macrocode}
% \end{macro}
% \end{macro}
% \end{macro}
%
% \begin{macro}{\@@_backend_devicen_init:nnn}
% \begin{macro}[EXP]{\@@_backend_devicen_init:w}
%   Similar to the Separations case, but with an arbitrary function for
%   the alternative space work.
%    \begin{macrocode}
\cs_new_protected:Npn \@@_backend_devicen_init:nnn #1#2#3
  {
    \pdf_object_unnamed_write:ne { stream }
      {
        {
          /FunctionType ~ 4 ~
          /Domain ~
            [ ~
              \prg_replicate:nn
                { 0 \@@_backend_devicen_init:w #1 ~ \s_@@_stop }
                { 0 ~ 1 ~ }
            ] ~
          /Range ~
            [ ~
              \str_case:nn {#2}
                {
                  { /DeviceCMYK } { 0 ~ 1 ~ 0 ~ 1 ~ 0 ~ 1 ~ 0 ~ 1 }
                  { /DeviceGray } { 0 ~ 1 }
                  { /DeviceRGB }  { 0 ~ 1 ~ 0 ~ 1 ~ 0 ~ 1 }
                } ~
            ]
        }
        { {#3} }
      }
    \use:e
      {
        \pdf_object_new:n { color \int_use:N \g_@@_model_int }
        \pdf_object_write:nnn { color \int_use:N \g_@@_model_int } { array }
          {
            /DeviceN ~
            [ ~ #1 ~ ] ~
            #2 ~
            \pdf_object_ref_last:
            \@@_backend_devicen_colorants:n {#1}
          }
      }
    \@@_backend_init_resource:n { color \int_use:N \g_@@_model_int }
  }
\cs_new:Npn \@@_backend_devicen_init:w #1 ~ #2 \s_@@_stop
  {
    + 1
    \tl_if_blank:nF {#2}
      { \@@_backend_devicen_init:w #2 \s_@@_stop }
  }
%    \end{macrocode}
% \end{macro}
% \end{macro}
%
% \begin{macro}{\@@_backend_iccbased_init:nnn}
%  Lots of data to save here: we only want to do that once per file,
%  so track it by name.
%    \begin{macrocode}
\cs_new_protected:Npn \@@_backend_iccbased_init:nnn #1#2#3
  {
    \pdf_object_if_exist:nF { @@_icc_ #1 }
      {
        \pdf_object_new:n { @@_icc_ #1 }
        \pdf_object_write:nne { @@_icc_ #1 } { fstream }
          {
            {
              /N ~ \exp_not:n { #2 } ~
              \tl_if_empty:nF { #3 } { /Range~[ #3 ] }
            }
            {#1}
          }
      }
    \pdf_object_unnamed_write:ne { array }
      { /ICCBased ~ \pdf_object_ref:n { @@_icc_ #1 } }
    \@@_backend_init_resource:n { color \int_use:N \g_@@_model_int }
  }
%    \end{macrocode}
% \end{macro}
%
% \begin{macro}{\@@_backend_iccbased_device:nnn}
%   This is very similar to setting up a color space: the only part we
%   add to the page resources differently.
%    \begin{macrocode}
\cs_new_protected:Npn \@@_backend_iccbased_device:nnn #1#2#3
  {
    \pdf_object_if_exist:nF { @@_icc_ #1 }
      {
        \pdf_object_new:n { @@_icc_ #1 }
        \pdf_object_write:nnn { @@_icc_ #1 } { fstream }
          {
            { /N ~ #3 }
            {#1}
          }
      }
    \pdf_object_unnamed_write:ne { array }
      { /ICCBased ~ \pdf_object_ref:n { @@_icc_ #1 } }
    \@@_backend_init_resource:n { Default #2 }
  }
%    \end{macrocode}
% \end{macro} 
%
%    \begin{macrocode}
%</dvipdfmx|luatex|pdftex|xetex>
%    \end{macrocode}
%
% \subsection{Fill and stroke color}
%
% Here, \texttt{dvipdfmx}/\XeTeX{} we write direct PDF specials for the fill,
% and only use the stack for the stroke color (see above for comments on why
% we cannot use multiple stacks with these backends). \LuaTeX{} and \pdfTeX{}
% have multiple stacks that can deal with fill and stroke. For \texttt{dvips}
% we have to manage fill and stroke color ourselves. We also handle
% \texttt{dvisvgm} independently, as there we can create SVG directly.
%
%    \begin{macrocode}
%<*dvipdfmx|xetex>
%    \end{macrocode}
%
% \begin{macro}
%   {
%     \@@_backend_fill:n        ,
%     \@@_backend_fill_cmyk:n   ,
%     \@@_backend_fill_gray:n   ,
%     \@@_backend_fill_rgb:n    ,
%     \@@_backend_stroke:n      ,
%     \@@_backend_stroke_cmyk:n ,
%     \@@_backend_stroke_gray:n ,
%     \@@_backend_stroke_rgb:n
%   }
%    \begin{macrocode}
\cs_new_protected:Npn \@@_backend_fill:n #1
  { \__kernel_backend_literal:n { pdf : bc ~ fill ~ [ #1 ] } }
\cs_new_eq:NN \@@_backend_fill_cmyk:n \@@_backend_fill:n
\cs_new_eq:NN \@@_backend_fill_gray:n \@@_backend_fill:n
\cs_new_eq:NN \@@_backend_fill_rgb:n  \@@_backend_fill:n
\cs_new_protected:Npn \@@_backend_stroke:n #1
  { \__kernel_backend_literal:n { pdf : bc ~ stroke ~ [ #1 ] } }
\cs_new_eq:NN \@@_backend_stroke_cmyk:n \@@_backend_stroke:n
\cs_new_eq:NN \@@_backend_stroke_gray:n \@@_backend_stroke:n
\cs_new_eq:NN \@@_backend_stroke_rgb:n  \@@_backend_stroke:n
%    \end{macrocode}
% \end{macro}
%
% \begin{macro}
%   {
%     \@@_backend_fill_separation:nn,
%     \@@_backend_stroke_separation:nn,
%     \@@_backend_fill_devicen:nn,
%     \@@_backend_stroke_devicen:nn
%   }
%    \begin{macrocode}
\cs_new_protected:Npn \@@_backend_fill_separation:nn #1#2
  {
    \__kernel_backend_literal:e
      { pdf : bc ~ fill ~ \pdf_object_ref:n {#1} ~ [ #2 ] }
  }
\cs_new_protected:Npn \@@_backend_stroke_separation:nn #1#2
  {
    \__kernel_backend_literal:e
      { pdf : bc ~ stroke ~ \pdf_object_ref:n {#1} ~ [ #2 ] }
  }
\cs_new_eq:NN \@@_backend_fill_devicen:nn \@@_backend_fill_separation:nn
\cs_new_eq:NN \@@_backend_stroke_devicen:nn \@@_backend_stroke_separation:nn
%    \end{macrocode}
% \end{macro}
%
% \begin{macro}{\@@_backend_fill_reset:, \@@_backend_stroke_reset:}
%    \begin{macrocode}
\cs_new_eq:NN \@@_backend_fill_reset: \@@_backend_reset:
\cs_new_eq:NN \@@_backend_stroke_reset: \@@_backend_reset:
%    \end{macrocode}
% \end{macro}
%
%    \begin{macrocode}
%</dvipdfmx|xetex>
%    \end{macrocode}
%
%    \begin{macrocode}
%<*luatex|pdftex>
%    \end{macrocode}
%
% \begin{macro}
%   {
%     \@@_backend_fill_cmyk:n   ,
%     \@@_backend_fill_gray:n   ,
%     \@@_backend_fill_rgb:n    ,
%     \@@_backend_fill:n        ,
%     \@@_backend_stroke_cmyk:n ,
%     \@@_backend_stroke_gray:n ,
%     \@@_backend_stroke_rgb:n  ,
%     \@@_backend_stroke:n
%   }
%   Drawing (fill/stroke) color is handled in \texttt{dvipdfmx}/\XeTeX{} in the
%   same way as \LuaTeX{}/\pdfTeX{}. We use the same approach as earlier, except the
%   color stack is not involved so the generic direct PDF operation is used.
%   There is no worry about the nature of strokes: everything is handled
%   automatically.
%    \begin{macrocode}
\cs_new_protected:Npn \@@_backend_fill_cmyk:n #1
  { \@@_backend_fill:n { #1 ~ k } }
\cs_new_protected:Npn \@@_backend_fill_gray:n #1
  { \@@_backend_fill:n { #1 ~ g } }
\cs_new_protected:Npn \@@_backend_fill_rgb:n #1
  { \@@_backend_fill:n { #1 ~ rg } }
\cs_new_protected:Npn \@@_backend_fill:n #1
  {
    \tl_set:Nn \l_@@_backend_fill_tl {#1}
    \@@_backend_stack_push:nnV \l_@@_backend_stack_int
      {#1} \l_@@_backend_stroke_tl
  }
\cs_new_protected:Npn \@@_backend_stroke_cmyk:n #1
  { \@@_backend_stroke:n { #1 ~ K } }
\cs_new_protected:Npn \@@_backend_stroke_gray:n #1
  { \@@_backend_stroke:n { #1 ~ G } }
\cs_new_protected:Npn \@@_backend_stroke_rgb:n #1
  { \@@_backend_stroke:n { #1 ~ RG } }
\cs_new_protected:Npn \@@_backend_stroke:n #1
  {
    \tl_set:Nn \l_@@_backend_stroke_tl {#1}
    \@@_backend_stack_push:nVn \l_@@_backend_stack_int
      \l_@@_backend_fill_tl {#1}
  }
%    \end{macrocode}
% \end{macro}
%
% \begin{macro}
%   {
%     \@@_backend_fill_separation:nn,
%     \@@_backend_stroke_separation:nn,
%     \@@_backend_fill_devicen:nn,
%     \@@_backend_stroke_devicen:nn
%   }
%    \begin{macrocode}
\cs_new_protected:Npn \@@_backend_fill_separation:nn #1#2
  { \@@_backend_fill:n { /#1 ~ cs ~ #2 ~ scn } }
\cs_new_protected:Npn \@@_backend_stroke_separation:nn #1#2
  { \@@_backend_stroke:n { /#1 ~ CS ~ #2 ~ SCN } }
\cs_new_eq:NN \@@_backend_fill_devicen:nn \@@_backend_fill_separation:nn
\cs_new_eq:NN \@@_backend_stroke_devicen:nn \@@_backend_stroke_separation:nn
%    \end{macrocode}
% \end{macro}
%
% \begin{macro}{\@@_backend_fill_reset:, \@@_backend_stroke_reset:}
%    \begin{macrocode}
\cs_new_eq:NN \@@_backend_fill_reset: \@@_backend_reset:
\cs_new_eq:NN \@@_backend_stroke_reset: \@@_backend_reset:
%    \end{macrocode}
% \end{macro}
%
%    \begin{macrocode}
%</luatex|pdftex>
%    \end{macrocode}
%
%    \begin{macrocode}
%<*dvips>
%    \end{macrocode}
%
% \begin{macro}
%   {
%     \@@_backend_fill_cmyk:n   ,
%     \@@_backend_fill_gray:n   ,
%     \@@_backend_fill_rgb:n    ,
%     \@@_backend_fill:n    ,
%     \@@_backend_stroke_cmyk:n ,
%     \@@_backend_stroke_gray:n ,
%     \@@_backend_stroke_rgb:n
%   }
%   Fill color here is the same as general color \emph{except} we skip the
%   stroke part.
%    \begin{macrocode}
\cs_new_protected:Npn \@@_backend_fill_cmyk:n #1
  { \@@_backend_fill:n { cmyk ~ #1 } }
\cs_new_protected:Npn \@@_backend_fill_gray:n #1
  { \@@_backend_fill:n { gray ~ #1 } }
\cs_new_protected:Npn \@@_backend_fill_rgb:n #1
  { \@@_backend_fill:n { rgb ~ #1 } }
\cs_new_protected:Npn \@@_backend_fill:n #1
  {
    \__kernel_backend_literal:n { color~push~ #1 }
  }
\cs_new_protected:Npn \@@_backend_stroke_cmyk:n #1
  { \__kernel_backend_postscript:n { /color.sc { #1 ~ setcmykcolor } def } }
\cs_new_protected:Npn \@@_backend_stroke_gray:n #1
  { \__kernel_backend_postscript:n { /color.sc { #1 ~ setgray } def } }
\cs_new_protected:Npn \@@_backend_stroke_rgb:n #1
  { \__kernel_backend_postscript:n { /color.sc { #1 ~ setrgbcolor } def } }
%    \end{macrocode}
% \end{macro}
%
% \begin{macro}
%   {
%     \@@_backend_fill_separation:nn,
%     \@@_backend_stroke_separation:nn,
%     \@@_backend_fill_devicen:nn,
%     \@@_backend_stroke_devicen:nn
%   }
%    \begin{macrocode}
\cs_new_protected:Npn \@@_backend_fill_separation:nn #1#2
  { \@@_backend_fill:n { separation ~ #1 ~ #2 } }
\cs_new_protected:Npn \@@_backend_stroke_separation:nn #1#2
  { \__kernel_backend_postscript:n { /color.sc { separation ~ #1 ~ #2 } def } }
\cs_new_eq:NN \@@_backend_fill_devicen:nn \@@_backend_fill_separation:nn
\cs_new_eq:NN \@@_backend_stroke_devicen:nn \@@_backend_stroke_separation:nn
%    \end{macrocode}
% \end{macro}
%
% \begin{macro}{\@@_backend_fill_reset:, \@@_backend_stroke_reset:}
%    \begin{macrocode}
\cs_new_eq:NN \@@_backend_fill_reset: \@@_backend_reset:
\cs_new_protected:Npn \@@_backend_stroke_reset: { }
%    \end{macrocode}
% \end{macro}
%
%    \begin{macrocode}
%</dvips>
%    \end{macrocode}
%
%    \begin{macrocode}
%<*dvisvgm>
%    \end{macrocode}
%
% \begin{macro}
%   {
%     \@@_backend_fill_cmyk:n   ,
%     \@@_backend_fill_gray:n   ,
%     \@@_backend_fill_rgb:n    ,
%     \@@_backend_fill:n
%   }
%   Fill color here is the same as general color.
%    \begin{macrocode}
\cs_new_protected:Npn \@@_backend_fill_cmyk:n #1
  { \@@_backend_fill:n { cmyk ~ #1 } }
\cs_new_protected:Npn \@@_backend_fill_gray:n #1
  { \@@_backend_fill:n { gray ~ #1 } }
\cs_new_protected:Npn \@@_backend_fill_rgb:n #1
  { \@@_backend_fill:n { rgb ~ #1 } }
\cs_new_protected:Npn \@@_backend_fill:n #1
  {
    \__kernel_backend_literal:n { color~push~ #1 }
  }
%    \end{macrocode}
% \end{macro}
%
% \begin{macro}{\@@_backend_stroke_cmyk:n}
% \begin{macro}{\@@_backend_stroke_gray:n, \@@_backend_stroke_gray_aux:n}
% \begin{macro}{\@@_backend_stroke_rgb:n}
% \begin{macro}{\@@_backend_stroke_rgb:w}
% \begin{macro}{\@@_backend:nnn}
%   For drawings in SVG, we use scopes for all stroke colors. The backend
%   provides the necessary conversion for CMYK but only if that is set as
%   the main color: a little bit of gymnastics as a result.
%    \begin{macrocode}
\cs_new_protected:Npn \@@_backend_stroke_cmyk:n #1
  {
    \@@_backend_fill_cmyk:n {#1}
    \__kernel_backend_scope:n { stroke = "{?color}" }
    \@@_backend_reset:
  }
\cs_new_protected:Npn \@@_backend_stroke_gray:n #1
  {
    \use:e
      {
        \@@_backend_stroke_gray_aux:n
          { \fp_eval:n { 100 * (#1) } }
      }
  }
\cs_new_protected:Npn \@@_backend_stroke_gray_aux:n #1
  { \@@_backend:nnn {#1} {#1} {#1} }
\cs_new_protected:Npn \@@_backend_stroke_rgb:n #1
  { \@@_backend_rgb:w #1 \s_@@_stop }
\cs_new_protected:Npn \@@_backend_stroke_rgb:w
  #1 ~ #2 ~ #3 \s_@@_stop
  {
    \use:e
      {
        \@@_backend:nnn
          { \fp_eval:n { 100 * (#1) } }
          { \fp_eval:n { 100 * (#2) } }
          { \fp_eval:n { 100 * (#3) } }
      }
  }
\cs_new_protected:Npe \@@_backend:nnn #1#2#3
  {
    \__kernel_backend_scope:n
      {
        stroke =
          "
            rgb
              (
                #1 \c_percent_str ,
                #2 \c_percent_str ,
                #3 \c_percent_str
              )
          "
      }
  }
%    \end{macrocode}
% \end{macro}
% \end{macro}
% \end{macro}
% \end{macro}
% \end{macro}
%
% \begin{macro}
%   {
%     \@@_backend_fill_separation:nn,
%     \@@_backend_stroke_separation:nn,
%     \@@_backend_fill_devicen:nn,
%     \@@_backend_stroke_devicen:nn
%   }
%   At present, these are no-ops.
%    \begin{macrocode}
\cs_new_protected:Npn \@@_backend_fill_separation:nn #1#2 { }
\cs_new_protected:Npn \@@_backend_stroke_separation:nn #1#2 { }
\cs_new_eq:NN \@@_backend_fill_devicen:nn \@@_backend_fill_separation:nn
\cs_new_eq:NN \@@_backend_stroke_devicen:nn \@@_backend_stroke_separation:nn
%    \end{macrocode}
% \end{macro}
%
% \begin{macro}{\@@_backend_fill_reset:, \@@_backend_stroke_reset:}
%    \begin{macrocode}
\cs_new_eq:NN \@@_backend_fill_reset: \@@_backend_reset:
\cs_new_protected:Npn \@@_backend_stroke_reset: { }
%    \end{macrocode}
% \end{macro}
%
% \begin{macro}{\@@_backend_devicen_init:nnn, \@@_backend_iccbased_init:nnn}
%   No support at present.
%    \begin{macrocode}
\cs_new_protected:Npn \@@_backend_devicen_init:nnn #1#2#3 { }
\cs_new_protected:Npn \@@_backend_iccbased_init:nnn #1#2#3 { }
%    \end{macrocode}
% \end{macro}
%
%    \begin{macrocode}
%</dvisvgm>
%    \end{macrocode}
%
% \subsection{Math color}
%
% \begin{macro}{\@@_backend_math_reset:}
%   No grouping so no \cs{group_insert_after:N}: all the engines are the same
%   initially here but we need an abstraction for \LuaTeX{}.
%    \begin{macrocode}
\cs_new_eq:NN \@@_backend_math_reset: \@@_backend_reset_aux:
%    \end{macrocode}
% \end{macro}
%
%    \begin{macrocode}
%</package>
%    \end{macrocode}
%
% \subsection{Font handling integration}
%
% In \LuaTeX{} these colors should also be usable to color fonts, so
% \pkg{luaotfload} color handling is extended to include these.
%
%    \begin{macrocode}
%<*lua>
%    \end{macrocode}
%
% Load basic support and print some information.
%    \begin{macrocode}
require("ltluatex")
local luatexbase = luatexbase
luatexbase.provides_module({name = "l3backend-luatex", date = "2025-05-13"})
%    \end{macrocode}
%
%    \begin{macrocode}
local l = lpeg
local spaces = l.P' '^0
local digit16 = l.R('09', 'af', 'AF')

local octet = digit16 * digit16 / function(s)
  return string.format('%.3g ', tonumber(s, 16) / 255)
end

if luaotfload and luaotfload.set_transparent_colorstack then
  local htmlcolor = l.Cs(octet * octet * octet * -1 * l.Cc'rg')
  local color_export = {
    token.create'tex_endlocalcontrol:D',
    token.create'tex_hpack:D',
    token.new(0, 1),
    token.create'color_export:nnN',
    token.new(0, 1),
    '',
    token.new(0, 2),
    token.new(0, 1),
    'backend',
    token.new(0, 2),
    token.create'l_tmpa_tl',
    token.create'exp_after:wN',
    token.create'@@_select:nn',
    token.create'l_tmpa_tl',
    token.new(0, 2),
  }
  local group_end = token.create'group_end:'
  local value = (1 - l.P'}')^0
  luatexbase.add_to_callback('luaotfload.parse_color', function (value)
% Also allow HTML colors to preserve compatibility
    local html = htmlcolor:match(value)
    if html then return html end

% If no l3color named color with this name is known, check for defined xcolor colors
    local l3color_prop = token.get_macro(string.format('l_@@_named_%s_prop', value))
    if l3color_prop == nil or l3color_prop == '' then
      local legacy_color_macro = token.create(string.format('\\color@%s', value))
      if legacy_color_macro.cmdname ~= 'undefined_cs' then
        token.put_next(legacy_color_macro)
        return token.scan_argument()
      end
    end

    tex.runtoks(function()
      token.get_next()
      color_export[6] = value
      tex.sprint(-2, color_export)
    end)
    local list = token.scan_list()
    if not list.head or list.head.next
        or list.head.subtype ~= node.subtype'pdf_colorstack' then
      error'Unexpected backend behavior'
    end
    local cmd = list.head.data
    node.free(list)
    return cmd
  end, 'l3color')
end
%    \end{macrocode}
%
% \subsection{Color in \LuaTeX{}}
%
% In \LuaTeX{}, we can use an approach based on attributes to avoid having to
% insert whatsits. That means having the appropriate supporting Lua code. The
% underlying idea comes from Heiko Oberdiek's \pkg{luacolor} package.
%
% \begin{macro}{ltx.color}
% \begin{macro}{ltxcolor}
%   The basic table should be set up in \pkg{l3luatex}, but we want a separate
%   sub-table for color.
%    \begin{macrocode}
ltx = ltx or {color = {}}
ltx.color = ltx.color or {}
local ltxcolor = ltx.color
%    \end{macrocode}
% \end{macro}
% \end{macro}
%
% \begin{macro}{attr}
%   An attribute to track color. There is no need to access this from \TeX{} so
%   in contrast to \pkg{luacolor} we just allocate directly in Lua.
%    \begin{macrocode}
local color_attr = luatexbase.new_attribute("color")
%    \end{macrocode}
% \end{macro}
%
% \begin{macro}{map}
%   A map between color strings and their |color_index|, plus the tracking
%   number.
%    \begin{macrocode}
local map = {}
%    \end{macrocode}
% \end{macro}
%
% \begin{macro}{color_index}
%   Get the |color_index| of a color or create a new |color_index| if
%   necessary.
%    \begin{macrocode}
local function color_index(color)
  local n = map[color]
  if not n then
    n = #map + 1
    map[n] = color
    map[color] = n
  end
  return n
end
%    \end{macrocode}
% \end{macro}
%
%   Local copies of global tables.
%    \begin{macrocode}
local node   = node
local string = string
local tex    = tex
%    \end{macrocode}
%
%   Local copies of standard functions.
%    \begin{macrocode}
local getbox           = tex.getbox
local has_attribute    = node.has_attribute
local id               = node.id
local insert_before    = node.insert_before
local module_error     = luatexbase.module_error
local node_new         = node.new
local node_tail        = node.tail
local node_traverse    = node.traverse
local saveboxresources = tex.saveboxresources
local set_attribute    = tex.setattribute
local node_type        = node.type
local subtype          = node.subtype
%    \end{macrocode}
%
% \begin{macro}{ltx.color.select}
%   Get the |color_index| of a color or create a new |color_index| if necessary.
%    \begin{macrocode}
local function select(color_str)
  set_attribute(color_attr,color_index(color_str))
end
ltxcolor.select = select
%    \end{macrocode}
% \end{macro}
%
% \begin{macro}
%   {
%     id_list         ,
%     id_list_leaders ,
%     id_list_disc    ,
%     id_color        ,
%     id_no_color     ,
%     node_types      ,
%     literals
%   }
%   We abstract the idea of different types of node where it makes sense:
%   notice for example that in contrast to \pkg{luacolor} we extract some
%   one-off information in the function that needs it.
%    \begin{macrocode}
local id_list         = 1
local id_list_leaders = 2
local id_list_disc    = 3
local id_color        = 4
local id_no_color     = 5
local node_types = {
    [id("hlist")]   = id_list      ,
    [id("vlist")]   = id_list      ,
    [id("rule")]    = id_color     ,
    [id("glyph")]   = id_color     ,
    [id("disc")]    = id_list_disc ,
    [id("whatsit")] = {
        [subtype("special")]        = id_color ,
        [subtype("pdf_colorstack")] = id_color ,
        [subtype("pdf_literal")]    = id_color ,
        [subtype("pdf_save")]       = id_color ,
        [subtype("pdf_restore")]    = id_color ,
      } ,
    [id("glue")] =
      function(n)
        if n.subtype >= 100 then
          if n.leader.id == id("rule") then
            return id_color
          else
            return id_list_leaders
          end
        end
      end
  }
%    \end{macrocode}
% \end{macro}
%
% \begin{macro}{get_type}
%   Convert a node into the type, allowing for tables and functions.
%    \begin{macrocode}
local function get_type(n)
  local ret = node_types[n.id]
  if type(ret) == "table" then
    ret = ret[n.subtype]
  end
  if type(ret) == "function" then
    ret = ret(n)
  end
  return ret
end
%    \end{macrocode}
% \end{macro}
%
% \begin{macro}{set_color, clear_color}
%   Create the the node containing color data and insert it into the list.
%    \begin{macrocode}
local function set_color(head, n, new_color, old_color)
  if new_color == old_color then return head end
  local color_node = node_new("whatsit","pdf_colorstack")
  if new_color then
    color_node.command = old_color and 0 or 1 -- "set" or "push"
    color_node.data = new_color
  else
    color_node.command = 2 -- "pop"
  end
  color_node.attr = n.attr
  return insert_before(head, n, color_node)
end
local function clear_color(head, n, old_color)
  if not old_color then return head end
  local color_node = node_new("whatsit","pdf_colorstack")
  color_node.command = 2 -- "pop"
  if not n then n = node_tail(head) end
  color_node.attr = n.attr
  return insert_after(head, n, color_node)
end
%    \end{macrocode}
% \end{macro}
%
% \begin{macro}{traverse}
%   The business end of the process: transverse the node list.
%    \begin{macrocode}
local function traverse(list,color,dry_run,force_reset)
%    \end{macrocode}
%   We start with a sanity check: do we have a list at all.
%    \begin{macrocode}
  if not list then
    return color
  end
%    \end{macrocode}
%   Work out the type of head we have: should be a |list| or |list_disc|.
%    \begin{macrocode}
  local head
  if get_type(list) == id_list then
    head = list.head
  elseif get_type(list) == id_list_disc then
    head = list.replace
  else
    module_error("l3color","Wrong list type: " .. node_type(list.id))
    return color
  end
%    \end{macrocode}
%   The main loop. We have various bits of recursion to do to map over
%   the lists, before we get to the business end: color nodes.
%    \begin{macrocode}
  for n in node_traverse(head) do
    local t = get_type(n)
    if t == id_list or t == id_list_disc then
      color = traverse(n,color,dry_run)
    elseif t == id_list_leaders then
      local new_color = color == traverse(n.leader,color,true) and color or nil
      if not dry_run then
        head = set_color(head, n, new_color, color)
        traverse(n.leader,new_color,false,not new_color)
      end
      color = new_color
    elseif t == id_color then
      local new_color = map[has_attribute(n,color_attr)]
      if not dry_run then
        head = set_color(head, n, new_color, color)
      end
      color = new_color
    elseif t == id_no_color then
      if not dry_run then
        head = set_color(head, n, nil, color)
      end
      color = nil
    end
  end
  if force_reset and color then
    color = nil
    head = clear_color(head, nil, color)
  end
%    \end{macrocode}
%   Loop done, set up the list.
%    \begin{macrocode}
  if get_type(list) == id_list then
    list.head = head
  else
    list.replace = head
  end
  return color
end
%    \end{macrocode}
% \end{macro}
%
% Activate the code: we assume that the |pre_shipout_filter| callback is
% available as this is pretty burning-edge stuff.
%    \begin{macrocode}
if luatexbase.callbacktypes.pre_shipout_filter then
  luatexbase.add_to_callback(
    "pre_shipout_filter",
    function(list)
      traverse(list)
      return true
    end,
    "color")
end
%    \end{macrocode}
%
% For recent versions of \pkg{luaotfload}, we can register a callback to
% control how coloring glyph is handled for the color feature.
%    \begin{macrocode}
if luaotfload and luaotfload.set_colorhandler then
  local set_attribute = node.direct.set_attribute
  luaotfload.set_colorhandler(function(head,n,color)
    set_attribute(n,attribute,color_index(color))
    return head,n
  end)
end
%    \end{macrocode}
%
% \begin{macro}{luacmd}
%   Copied from \pkg{l3luatex}: we need to think about how to make this available
%   (perhaps in |ltx.utils|?).
%    \begin{macrocode}
local luacmd do
  local token_create = token.create
  local set_lua = token.set_lua
  local undefined_cs = token.command_id'undefined_cs'

  if not context and not luatexbase then require'ltluatex' end
  if luatexbase then
    local new_luafunction = luatexbase.new_luafunction
    local functions = lua.get_functions_table()
    function luacmd(name, func, ...)
      local id
      local tok = token_create(name)
      if tok.command == undefined_cs then
        id = new_luafunction(name)
        set_lua(name, id, ...)
      else
        id = tok.index or tok.mode
      end
      functions[id] = func
    end
  elseif context then
    local register = context.functions.register
    local functions = context.functions.known
    function luacmd(name, func, ...)
      local tok = token.create(name)
      if tok.command == undefined_cs then
        token.set_lua(name, register(func), ...)
      else
        functions[tok.index or tok.mode] = func
      end
    end
  end
end
%    \end{macrocode}
% \end{macro}
%
%    \begin{macrocode}
%<@@=pdfxform>
%    \end{macrocode}
%
% \begin{macro}{pdfxform}
% \begin{macro}{\@@_backend_now:w, \@@_backend_delayed:w}
%   Here we define an internal for the |pdfxform| module: the
%   same ideas work in \pdfTeX{} as macros, and other engines don't
%   use quite the same abstraction.
%    \begin{macrocode}
local function pdfxform(immediate)
  local attr = scan_keyword'attr' and scan_string() or nil
  local resources = scan_keyword'length' and scan_string() or nil
  local xtype = scan_keyword'type' and scan_string() or nil
  local margin = scan_keyword'margin' and scan_string() or nil
  local n = scan_int() or nil
  traverse(getbox(n))
  saveboxresources(n,attr,resources,immediate,xtype,margin)
end
luacmd('@@_backend_now:w', function()
  pdfxform(true)
end,'global','protected')
luacmd('@@_backend_delayed:w', function()
  pdfxform(false)
end,'global','protected')
%    \end{macrocode}
% \end{macro}
% \end{macro}
%
%    \begin{macrocode}
%</lua>
%    \end{macrocode}
%
%    \begin{macrocode}
%<@@=color>
%    \end{macrocode}
%
%    \begin{macrocode}
%<*package>
%    \end{macrocode}
%
% At present we only set up for \LuaTeX{} in PDF mode: later we will expand
% on that!
%
%    \begin{macrocode}
%<*luatex>
%    \end{macrocode}
%
% \begin{macro}{\@@_backend_stack_push:nn}
% \begin{macro}{\@@_backend_reset:}
%   A simple reworking: we may need to check about transparency here.
%   Math color has to be manually reset as there is no group to rely on:
%   we know that the current color tracked at the macro level will be the
%   one we need.
%    \begin{macrocode}
\cs_gset_protected:Npn \@@_backend_stack_push:nn #1#2
  {
    \lua_now:e { ltx.color.select(" \lua_escape:n {#2} ") }
  }
\cs_gset_protected:Npn \@@_backend_reset: { }
\cs_gset_protected:Npn \@@_backend_fill_reset: { }
\cs_gset_protected:Npn \@@_backend_stroke_reset: { }
\cs_gset_protected:Npn \@@_backend_math_reset:
  { \@@_select:N \l_@@_current_tl }
%    \end{macrocode}
% \end{macro}
% \end{macro}
%
%    \begin{macrocode}
\lua_load_module:n { l3backend-luatex }
%    \end{macrocode}
%
%    \begin{macrocode}
%</luatex>
%    \end{macrocode}
%
%    \begin{macrocode}
%</package>
%    \end{macrocode}
%
% \end{implementation}
%
% \PrintIndex
