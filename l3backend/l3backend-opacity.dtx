% \iffalse meta-comment
%
%% File: l3backend-opacity.dtx
%
% Copyright (C) 2021 The LaTeX Project
%
% It may be distributed and/or modified under the conditions of the
% LaTeX Project Public License (LPPL), either version 1.3c of this
% license or (at your option) any later version.  The latest version
% of this license is in the file
%
%    https://www.latex-project.org/lppl.txt
%
% This file is part of the "l3backend bundle" (The Work in LPPL)
% and all files in that bundle must be distributed together.
%
% -----------------------------------------------------------------------
%
% The development version of the bundle can be found at
%
%    https://github.com/latex3/latex3
%
% for those people who are interested.
%
%<*driver>
\documentclass[full,kernel]{l3doc}
\begin{document}
  \DocInput{\jobname.dtx}
\end{document}
%</driver>
% \fi
%
% \title{^^A
%   The \textsf{l3backend-opacity} package\\ Backend opacity support^^A
% }
%
% \author{^^A
%  The \LaTeX{} Project\thanks
%    {^^A
%      E-mail:
%        \href{mailto:latex-team@latex-project.org}
%          {latex-team@latex-project.org}^^A
%    }^^A
% }
%
% \date{Released 2021-01-09}
%
% \maketitle
%
% \begin{documentation}
%
% \end{documentation}
%
% \begin{implementation}
%
% \section{\pkg{l3backend-opacity} Implementation}
%
%    \begin{macrocode}
%<*package>
%<@@=opacity>
%    \end{macrocode}
%
% Although opacity is not color, it needs to be managed in a somewhat
% similar way: using a dedicated stack if possible. Depending on the backend,
% that may not be possible. There is also the need to cover fill/stroke setting
% as well as more general running opacity. It is easiest to describe the value
% used in terms of opacity, although commonly this is referred to as
% transparency.
%
%    \begin{macrocode}
%<*dvips>
%    \end{macrocode}
%
% \begin{macro}{\@@_backend_select:n}
%   No stack support.
%    \begin{macrocode}
\cs_new_protected:Npn \@@_backend_select:n #1 { }
%    \end{macrocode}
% \end{macro}
%
% \begin{macro}{\@@_backend_fill:n, \@@_backend_stroke:n}
% \begin{macro}{\@@_backend:nn, \@@_backend:xn}
%   Similar to the above but with no stack and only adding to one or other of
%   the entries.
%    \begin{macrocode}
\cs_new_protected:Npn \@@_backend_fill:n #1
  { \@@_backend:xn { \fp_eval:n { min(max(0,#1),1) } } { fill } }
\cs_new_protected:Npn \@@_backend_stroke:n #1
  { \@@_backend:xn { \fp_eval:n { min(max(0,#1),1) } } { stroke } }
\cs_new_protected:Npn \@@_backend:nn #1#2
  {
    \__kernel_backend_postscript:n { #1 ~ .set #2 constantalpha  }
  }
\cs_generate_variant:Nn \@@_backend:nn { x }
%    \end{macrocode}
% \end{macro}
% \end{macro}
%
%    \begin{macrocode}
%</dvips>
%    \end{macrocode}
%
%    \begin{macrocode}
%<*dvipdfmx|luatex|pdftex|xetex>
%    \end{macrocode}
%
% \begin{variable}{\c_@@_backend_stack_int}
%   Set up a stack.
%    \begin{macrocode}
\cs_if_exist:NT \pdfmanagement_add:nnn
  {
    \__kernel_color_backend_stack_init:Nnn \c_@@_backend_stack_int
      { page ~ direct } { /opacity 1 ~ gs }
    \pdfmanagement_add:nnn { Page / Resources / ExtGState }
      { opacity 1 } { << /ca ~ 1 /CA ~ 1 >> }
  }
%    \end{macrocode}
% \end{variable}
%
% \begin{macro}{\@@_backend_select:n, \@@_backend_select_aux:n}
% \begin{macro}{\@@_backend_reset:}
%   Other than the need to evaluate the opacity as an \texttt{fp}, much the
%   same as color.
%    \begin{macrocode}
\cs_new_protected:Npn \@@_backend_select:n #1
 {
   \exp_args:Nx \@@_backend_select_aux:n
     { \fp_eval:n { min(max(0,#1),1) } }
   \group_insert_after:N \@@_backend_reset:
 }
\cs_new_protected:Npn \@@_backend_select_aux:n #1
  {
    \pdfmanagement_add:nnn { Page / Resources / ExtGState }
      { opacity #1 }
      { << /ca ~ #1 /CA ~ #1 >> }
   \@@_backend_stack_push:nn \c_@@_backend_stack_int
     { /opacity #1 ~ gs }
  }
\cs_if_exist:NF \pdfmanagement_add:nnn
  {
    \cs_gset_protected:Npn \@@_backend_select_aux:n #1 { }
  }
\cs_new_protected:Npn \@@_backend_reset:
 { \@@_backend_stack_pop:n \c_@@_backend_stack_int }
%    \end{macrocode}
% \end{macro}
% \end{macro}
%
% \begin{macro}{\@@_backend_fill:n, \@@_backend_stroke:n}
% \begin{macro}{\@@_backend:nn, \@@_backend:xn}
%   Similar to the above but with no stack and only adding to one or other of
%   the entries.
%    \begin{macrocode}
\cs_new_protected:Npn \@@_backend_fill:n #1
  { \@@_backend:nn { \fp_eval:n { min(max(0,#1),1) } } { ca } }
\cs_new_protected:Npn \@@_backend_stroke:n #1
  { \@@_backend:nn { \fp_eval:n { min(max(0,#1),1) } } { CA } }
\cs_new_protected:Npn \@@_backend:nn #1#2
  {
    \pdfmanagement_add:nnn { Page / Resources / ExtGState }
      { opacity #1 }
      { << /#2 ~ #1 >> }
   \@@_backend_stack_push:nn \c_@@_backend_stack_int
     { /opacity #1 ~ gs }
  }
\cs_generate_variant:Nn \@@_backend:nn { x }
%    \end{macrocode}
% \end{macro}
% \end{macro}
%
%    \begin{macrocode}
%</dvipdfmx|luatex|pdftex|xetex>
%    \end{macrocode}
%
%    \begin{macrocode}
%<*dvipdfmx|xdvipdfmx>
%    \end{macrocode}
%
% \begin{macro}{\@@_backend_select:n}
%   Older backends have no stack support.
%    \begin{macrocode}
\int_compare:nNnT \c__kernel_sys_dvipdfmx_version_int < { 20201111 }
  {
    \cs_gset_protected:Npn \@@_backend_select:n #1 { }
  }
%    \end{macrocode}
% \end{macro}
%
%    \begin{macrocode}
%</dvipdfmx|xdvipdfmx>
%    \end{macrocode}
%
%    \begin{macrocode}
%<*dvisvgm>
%    \end{macrocode}
%
% \begin{macro}{\@@_backend_select:n}
%   No stack support.
%    \begin{macrocode}
\cs_new_protected:Npn \@@_backend_select:n #1 { }
%    \end{macrocode}
% \end{macro}
%
% \begin{macro}{\@@_backend_fill:n, \@@_backend_stroke:n}
% \begin{macro}{\@@_backend:nn, \@@_backend:xn}
%   Once again, we use a scope here. There is a general opacity function for
%   SVG, but that is of course not set up using the stack.
%    \begin{macrocode}
\cs_new_protected:Npn \@@_backend_fill:n #1
  { \@@_backend:nn { \fp_eval:n { min(max(0,#1),1) } } { fill } }
\cs_new_protected:Npn \@@_backend_stroke:n #1
  { \@@_backend:nn { \fp_eval:n { min(max(0,#1),1) } } { stroke } }
\cs_new_protected:Npn \@@_backend:nn #1#2
  { \__kernel_backend_scope:n { #2 -opacity = "#1" } }
\cs_generate_variant:Nn \@@_backend:nn { x }
%    \end{macrocode}
% \end{macro}
% \end{macro}
%
%    \begin{macrocode}
%</dvisvgm>
%    \end{macrocode}
%
%    \begin{macrocode}
%</package>
%    \end{macrocode}
%
% \end{implementation}
%
% \PrintIndex
