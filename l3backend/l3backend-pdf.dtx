% \iffalse meta-comment
%
%% File: l3backend-pdf.dtx
%
% Copyright (C) 2019-2025 The LaTeX Project
%
% It may be distributed and/or modified under the conditions of the
% LaTeX Project Public License (LPPL), either version 1.3c of this
% license or (at your option) any later version.  The latest version
% of this license is in the file
%
%    https://www.latex-project.org/lppl.txt
%
% This file is part of the "l3backend bundle" (The Work in LPPL)
% and all files in that bundle must be distributed together.
%
% -----------------------------------------------------------------------
%
% The development version of the bundle can be found at
%
%    https://github.com/latex3/latex3
%
% for those people who are interested.
%
%<*driver>
\documentclass[full,kernel]{l3doc}
\begin{document}
  \DocInput{\jobname.dtx}
\end{document}
%</driver>
% \fi
%
% \title{^^A
%   The \pkg{l3backend-pdf} module\\ Backend PDF features^^A
% }
%
% \author{^^A
%  The \LaTeX{} Project\thanks
%    {^^A
%      E-mail:
%        \href{mailto:latex-team@latex-project.org}
%          {latex-team@latex-project.org}^^A
%    }^^A
% }
%
% \date{Released 2025-04-14}
%
% \maketitle
%
% \begin{documentation}
%
% \end{documentation}
%
% \begin{implementation}
%
% \section{\pkg{l3backend-pdf} implementation}
%
%    \begin{macrocode}
%<*package>
%<@@=pdf>
%    \end{macrocode}
%
% Setting up PDF resources is a complex area with only limited documentation
% in the engine manuals. The following code builds heavily on existing ideas
% from \pkg{hyperref} work by Sebastian Rahtz and Heiko Oberdiek, and
% significant contributions by Alexander Grahn, in addition to the specific
% code referenced a various points.
%
% \subsection{\texttt{dvips} backend}
%
%    \begin{macrocode}
%<*dvips>
%    \end{macrocode}
%
% \begin{macro}{\@@_backend_pdfmark:n, \@@_backend_pdfmark:e}
%   Used often enough it should be a separate function.
%    \begin{macrocode}
\cs_new_protected:Npn \@@_backend_pdfmark:n #1
  { \__kernel_backend_postscript:n { mark #1 ~ pdfmark } }
\cs_generate_variant:Nn \@@_backend_pdfmark:n { e }
%    \end{macrocode}
% \end{macro}
%
% \subsubsection{Catalogue entries}
%
% \begin{macro}{\@@_backend_catalog_gput:nn, \@@_backend_info_gput:nn}
%    \begin{macrocode}
\cs_new_protected:Npn \@@_backend_catalog_gput:nn #1#2
  { \@@_backend_pdfmark:n { { Catalog } << /#1 ~ #2 >> /PUT } }
\cs_new_protected:Npn \@@_backend_info_gput:nn #1#2
  { \@@_backend_pdfmark:n { /#1 ~ #2 /DOCINFO } }
%    \end{macrocode}
% \end{macro}
%
% \subsubsection{Objects}
%
% \begin{macro}{\@@_backend_object_new:}
% \begin{macro}[EXP]{\@@_backend_object_ref:n, \@@_backend_object_id:n}
%    \begin{macrocode}
\cs_new_protected:Npn \@@_backend_object_new:
  { \int_gincr:N \g_@@_backend_object_int }
\cs_new:Npn \@@_backend_object_ref:n #1 { { pdf.obj #1 } }
\cs_new_eq:NN \@@_backend_object_id:n \@@_backend_object_ref:n
%    \end{macrocode}
% \end{macro}
% \end{macro}
%
% \begin{macro}
%   {
%     \@@_backend_object_write:nnn, \@@_backend_object_write:nne,
%     \@@_backend_object_write_aux:nnn
%   }
% \begin{macro}
%   {
%     \@@_backend_object_write_array:nn   ,
%     \@@_backend_object_write_dict:nn    ,
%     \@@_backend_object_write_fstream:nn ,
%     \@@_backend_object_write_stream:nn
%   }
% \begin{macro}{\@@_backend_object_write_stream:nnn}
%   This is where we choose the actual type: some work to get things
%   right. To allow code sharing with the anonymous version, we use an
%   auxiliary.
%    \begin{macrocode}
\cs_new_protected:Npn \@@_backend_object_write:nnn #1#2#3
  {
    \@@_backend_object_write_aux:nnn
      { \@@_backend_object_ref:n {#1} }
      {#2} {#3}
  }
\cs_generate_variant:Nn \@@_backend_object_write:nnn { nne }
\cs_new_protected:Npn \@@_backend_object_write_aux:nnn #1#2#3
  {
    \@@_backend_pdfmark:e
      {
        /_objdef ~ #1
        /type
        \str_case:nn {#2}
          {
            { array }   { /array }
            { dict }    { /dict }
            { fstream } { /stream }
            { stream }  { /stream }
          }
        /OBJ
      }
    \use:c { @@_backend_object_write_ #2 :nn } {#1} {#3}
  }
\cs_new_protected:Npn \@@_backend_object_write_array:nn #1#2
  {
    \@@_backend_pdfmark:e
      { #1 ~0~ [ ~ \exp_not:n {#2} ~ ] ~ /PUTINTERVAL }
  }
\cs_new_protected:Npn \@@_backend_object_write_dict:nn #1#2
  {
    \@@_backend_pdfmark:e
      { #1 << \exp_not:n {#2} >> /PUT }
  }
\cs_new_protected:Npn \@@_backend_object_write_fstream:nn #1#2
  {
    \exp_args:Ne
      \@@_backend_object_write_fstream:nnn {#1} #2
  }
\cs_new_protected:Npn \@@_backend_object_write_fstream:nnn #1#2#3
  {
    \__kernel_backend_postscript:n
      {
        SDict ~ begin ~
        mark ~ #1 ~ << #2 >> /PUT ~ pdfmark ~
        mark ~ #1 ~ ( #3 )~ ( r )~ file ~ /PUT ~ pdfmark ~
        end
      }
  }
\cs_new_protected:Npn \@@_backend_object_write_stream:nn #1#2
  {
    \exp_args:Ne
      \@@_backend_object_write_stream:nnn {#1} #2
  }
\cs_new_protected:Npn \@@_backend_object_write_stream:nnn #1#2#3
  {
    \__kernel_backend_postscript:n
      {
        mark ~ #1 ~ ( #3 ) /PUT ~ pdfmark ~
        mark ~ #1 ~ << #2 >> /PUT ~ pdfmark
      }
  }
%    \end{macrocode}
% \end{macro}
% \end{macro}
% \end{macro}
%
% \begin{macro}{\@@_backend_object_now:nn, \@@_backend_object_now:ne}
%   No anonymous objects, so things are done manually.
%    \begin{macrocode}
\cs_new_protected:Npn \@@_backend_object_now:nn #1#2
  {
    \int_gincr:N \g_@@_backend_object_int
    \@@_backend_object_write_aux:nnn
      { { pdf.obj \int_use:N \g_@@_backend_object_int } }
      {#1} {#2}
  }
\cs_generate_variant:Nn \@@_backend_object_now:nn { ne }
%    \end{macrocode}
% \end{macro}
%
% \begin{macro}[EXP]{\@@_backend_object_last:}
%   Much like the annotation version.
%    \begin{macrocode}
\cs_new:Npn \@@_backend_object_last:
  { { pdf.obj \int_use:N \g_@@_backend_object_int } }
%    \end{macrocode}
% \end{macro}
%
% \begin{macro}[EXP]{\@@_backend_pageobject_ref:n}
%   Page references are easy in \texttt{dvips}.
%    \begin{macrocode}
\cs_new:Npn \@@_backend_pageobject_ref:n #1
  { { Page #1 } }
%    \end{macrocode}
% \end{macro}
%
% \subsubsection{Destinations}
%
% \begin{macro}{\@@_backend_destination:nn}
% \begin{macro}{\@@_backend_destination:nnnn, \@@_backend_destination_aux:nnnn}
%   Here, we need to turn the zoom into a scale. We also need to know where
%   the current anchor point actually is: worked out in PostScript. For the
%   rectangle version, we have a bit more PostScript: we need two points.
%   fitr without rule spec doesn't work, so it falls back to \texttt{/Fit} here.
%    \begin{macrocode}
\cs_new_protected:Npn \@@_backend_destination:nn #1#2
  {
    \__kernel_backend_postscript:n { pdf.dest.anchor }
    \@@_backend_pdfmark:e
      {
        /View
        [
          \str_case:nnF {#2}
            {
              { xyz }   { /XYZ ~ pdf.dest.point ~ null }
              { fit }   { /Fit }
              { fitb }  { /FitB }
              { fitbh } { /FitBH ~ pdf.dest.y }
              { fitbv } { /FitBV ~ pdf.dest.x }
              { fith }  { /FitH ~ pdf.dest.y }
              { fitv }  { /FitV ~ pdf.dest.x }
              { fitr }  { /Fit }
            }
            {
              /XYZ ~ pdf.dest.point ~ \fp_eval:n { (#2) / 100 }
            }
        ]
        /Dest ( \exp_not:n {#1} ) cvn
        /DEST
      }
  }
\cs_new_protected:Npn \@@_backend_destination:nnnn #1#2#3#4
  {
    \exp_args:Ne \@@_backend_destination_aux:nnnn
      { \dim_eval:n {#2} } {#1} {#3} {#4}
  }
\cs_new_protected:Npn \@@_backend_destination_aux:nnnn #1#2#3#4
  {
    \vbox_to_zero:n
      {
        \__kernel_kern:n {#4}
        \hbox:n { \__kernel_backend_postscript:n { pdf.save.ll } }
        \tex_vss:D
      }
    \__kernel_kern:n {#1}
    \vbox_to_zero:n
      {
        \__kernel_kern:n { -#3 }
        \hbox:n { \__kernel_backend_postscript:n { pdf.save.ur } }
        \tex_vss:D
      }
    \__kernel_kern:n { -#1 }
    \@@_backend_pdfmark:n
      {
        /View
        [
          /FitR ~
            pdf.llx ~ pdf.lly ~ pdf.dest2device ~
            pdf.urx ~ pdf.ury ~ pdf.dest2device
        ]
        /Dest ( #2 ) cvn
        /DEST
      }
  }
%    \end{macrocode}
% \end{macro}
% \end{macro}
%
% \subsubsection{Structure}
%
% \begin{macro}{\@@_backend_compresslevel:n}
% \begin{macro}{\@@_backend_compress_objects:n}
%   Doable for the usual \texttt{ps2pdf} method.
%    \begin{macrocode}
\cs_new_protected:Npn \@@_backend_compresslevel:n #1
  {
    \int_compare:nNnT {#1} = 0
      {
        \__kernel_backend_literal_postscript:n
          {
            /setdistillerparams ~ where
              { pop << /CompressPages ~ false >> setdistillerparams }
            if
          }
      }
  }
\cs_new_protected:Npn \@@_backend_compress_objects:n #1
  {
    \bool_if:nF {#1}
      {
        \__kernel_backend_literal_postscript:n
          {
            /setdistillerparams ~ where
              { pop << /CompressStreams ~ false >> setdistillerparams }
            if
          }
      }
  }
%    \end{macrocode}
% \end{macro}
% \end{macro}
%
% \begin{macro}
%   {\@@_backend_version_major_gset:n, \@@_backend_version_minor_gset:n}
%    \begin{macrocode}
\cs_new_protected:Npn \@@_backend_version_major_gset:n #1
  {
    \cs_gset:Npe \@@_backend_version_major: { \int_eval:n {#1} }
  }
\cs_new_protected:Npn \@@_backend_version_minor_gset:n #1
  {
    \cs_gset:Npe \@@_backend_version_minor: { \int_eval:n {#1} }
  }
%    \end{macrocode}
% \end{macro}
%
% \begin{macro}[EXP]{\@@_backend_version_major:, \@@_backend_version_minor:}
%   Data not available!
%    \begin{macrocode}
\cs_new:Npn \@@_backend_version_major: { -1 }
\cs_new:Npn \@@_backend_version_minor: { -1 }
%    \end{macrocode}
% \end{macro}
%
% \subsubsection{Marked content}
%
% \begin{macro}{\@@_backend_bdc:nn}
% \begin{macro}{\@@_backend_emc:}
%   Simple wrappers.
%    \begin{macrocode}
\cs_new_protected:Npn \@@_backend_bdc:nn #1#2
  { \@@_backend_pdfmark:n { /#1 ~ #2 /BDC } }
\cs_new_protected:Npn \@@_backend_emc:
  { \@@_backend_pdfmark:n { /EMC } }
%    \end{macrocode}
% \end{macro}
% \end{macro}
%
%    \begin{macrocode}
%</dvips>
%    \end{macrocode}
%
% \subsection{\LuaTeX{} and \pdfTeX{} backend}
%
%    \begin{macrocode}
%<*luatex|pdftex>
%    \end{macrocode}
%
% \subsubsection{Destinations}
%
% \begin{macro}{\@@_backend_destination:nn}
% \begin{macro}{\@@_backend_destination:nnnn}
%   A simple task: pass the data to the primitive. The |\scan_stop:| deals
%   with the danger of an unterminated keyword. The zoom given here is a
%   percentage, but we need to pass it as \emph{per mille}. The rectangle
%   version is also easy as everything is build in.
%    \begin{macrocode}
\cs_new_protected:Npn \@@_backend_destination:nn #1#2
  {
%<*luatex>
    \tex_pdfextension:D dest ~
%</luatex>
%<*pdftex>
    \tex_pdfdest:D
%</pdftex>
        name {#1}
        \str_case:nnF {#2}
          {
            { xyz }   { xyz }
            { fit }   { fit }
            { fitb }  { fitb }
            { fitbh } { fitbh }
            { fitbv } { fitbv }
            { fith }  { fith }
            { fitv }  { fitv }
            { fitr }  { fitr }
          }
          { xyz ~ zoom \fp_eval:n { #2 * 10 } }
        \scan_stop:
  }
\cs_new_protected:Npn \@@_backend_destination:nnnn #1#2#3#4
  {
%<*luatex>
    \tex_pdfextension:D dest ~
%</luatex>
%<*pdftex>
    \tex_pdfdest:D
%</pdftex>
    name {#1}
    fitr ~
      width  \dim_eval:n {#2} ~
      height \dim_eval:n {#3} ~
      depth  \dim_eval:n {#4} \scan_stop:
  }
%    \end{macrocode}
% \end{macro}
% \end{macro}
%
% \subsubsection{Catalogue entries}
%
% \begin{macro}{\@@_backend_catalog_gput:nn, \@@_backend_info_gput:nn}
%    \begin{macrocode}
\cs_new_protected:Npn \@@_backend_catalog_gput:nn #1#2
  {
%<*luatex>
    \tex_pdfextension:D catalog
%</luatex>
%<*pdftex>
    \tex_pdfcatalog:D
%</pdftex>
      { / #1 ~ #2 }
  }
\cs_new_protected:Npn \@@_backend_info_gput:nn #1#2
  {
%<*luatex>
    \tex_pdfextension:D info
%</luatex>
%<*pdftex>
    \tex_pdfinfo:D
%</pdftex>
      { / #1 ~ #2 }
  }
%    \end{macrocode}
% \end{macro}
%
% \subsubsection{Objects}
%
% \begin{variable}{\g_@@_backend_object_prop}
%   For tracking objects to allow finalization.
%    \begin{macrocode}
\prop_new:N \g_@@_backend_object_prop
%    \end{macrocode}
% \end{variable}
%
% \begin{macro}{\@@_backend_object_new:}
% \begin{macro}[EXP]{\@@_backend_object_ref:n, \@@_backend_object_id:n}
%   Declaring objects means reserving at the PDF level plus starting
%   tracking.
%    \begin{macrocode}
\cs_new_protected:Npn \@@_backend_object_new:
  {
%<*luatex>
    \tex_pdfextension:D obj ~
%</luatex>
%<*pdftex>
    \tex_pdfobj:D
%</pdftex>
      reserveobjnum ~
    \int_gset:Nn \g_@@_backend_object_int
%<*luatex>
      { \tex_pdffeedback:D lastobj }
%</luatex>
%<*pdftex>
      { \tex_pdflastobj:D }
%</pdftex>
  }
\cs_new:Npn \@@_backend_object_ref:n #1 { #1 ~ 0 ~ R }
\cs_new:Npn \@@_backend_object_id:n #1 {#1}
%    \end{macrocode}
% \end{macro}
% \end{macro}
%
% \begin{macro}{\@@_backend_object_write:nnn, \@@_backend_object_write:nne}
% \begin{macro}[EXP]{\@@_backend_object_write:nn}
% \begin{macro}[EXP]{\@@_exp_not_i:nn, \@@_exp_not_ii:nn}
%   Writing the data needs a little information about the structure of the
%   object.
%    \begin{macrocode}
\cs_new_protected:Npn \@@_backend_object_write:nnn #1#2#3
  {
%<*luatex>
    \tex_immediate:D \tex_pdfextension:D obj ~
%</luatex>
%<*pdftex>
    \tex_immediate:D \tex_pdfobj:D
%</pdftex>
      useobjnum ~ #1
    \@@_backend_object_write:nn {#2} {#3}
  }
\cs_new:Npn \@@_backend_object_write:nn #1#2
  {
    \str_case:nn {#1}
      {
        { array } { { [ ~ \exp_not:n {#2} ~ ] } }
        { dict }  { { << ~ \exp_not:n {#2} ~ >> } }
        { fstream }
          {
            stream ~ attr ~ { \@@_exp_not_i:nn #2 } ~
              file ~ { \@@_exp_not_ii:nn #2 }
          }
        { stream }
          {
            stream ~ attr ~ { \@@_exp_not_i:nn #2 } ~
              { \@@_exp_not_ii:nn #2 }
          }
      }
  }
\cs_generate_variant:Nn \@@_backend_object_write:nnn { nne }
\cs_new:Npn \@@_exp_not_i:nn #1#2 { \exp_not:n {#1} }
\cs_new:Npn \@@_exp_not_ii:nn #1#2 { \exp_not:n {#2} }
%    \end{macrocode}
% \end{macro}
% \end{macro}
% \end{macro}
%
% \begin{macro}{\@@_backend_object_now:nn, \@@_backend_object_now:ne}
%   Much like writing, but direct creation.
%    \begin{macrocode}
\cs_new_protected:Npn \@@_backend_object_now:nn #1#2
  {
%<*luatex>
    \tex_immediate:D \tex_pdfextension:D obj ~
%</luatex>
%<*pdftex>
    \tex_immediate:D \tex_pdfobj:D
%</pdftex>
      \@@_backend_object_write:nn {#1} {#2}
  }
\cs_generate_variant:Nn \@@_backend_object_now:nn { ne }
%    \end{macrocode}
% \end{macro}
%
% \begin{macro}[EXP]{\@@_backend_object_last:}
%   Much like annotation.
%    \begin{macrocode}
\cs_new:Npe \@@_backend_object_last:
  {
    \exp_not:N \int_value:w
%<*luatex>
      \exp_not:N \tex_pdffeedback:D lastobj ~
%</luatex>
%<*pdftex>
      \exp_not:N \tex_pdflastobj:D
%</pdftex>
      \c_space_tl 0 ~ R
  }
%    \end{macrocode}
% \end{macro}
%
% \begin{macro}[EXP]{\@@_backend_pageobject_ref:n}
%   The usual wrapper situation; the three spaces here are essential.
%    \begin{macrocode}
\cs_new:Npe \@@_backend_pageobject_ref:n #1
  {
    \exp_not:N \int_value:w
%<*luatex>
      \exp_not:N \tex_pdffeedback:D pageref
%</luatex>
%<*pdftex>
      \exp_not:N \tex_pdfpageref:D
%</pdftex>
          \c_space_tl #1 \c_space_tl \c_space_tl \c_space_tl 0 ~ R
  }
%    \end{macrocode}
% \end{macro}
%
% \subsubsection{Structure}
%
% \begin{macro}{\@@_backend_compresslevel:n}
% \begin{macro}{\@@_backend_compress_objects:n}
% \begin{macro}{\@@_backend_objcompresslevel:n}
%   Simply pass data to the engine.
%    \begin{macrocode}
\cs_new_protected:Npn \@@_backend_compresslevel:n #1
  {
    \tex_global:D
%<*luatex>
      \tex_pdfvariable:D compresslevel
%</luatex>
%<*pdftex>
      \tex_pdfcompresslevel:D
%</pdftex>
        \int_value:w \int_eval:n {#1} \scan_stop:
  }
\cs_new_protected:Npn \@@_backend_compress_objects:n #1
  {
    \bool_if:nTF {#1}
      { \@@_backend_objcompresslevel:n { 2 } }
      { \@@_backend_objcompresslevel:n { 0 } }
  }
\cs_new_protected:Npn \@@_backend_objcompresslevel:n #1
  {
    \tex_global:D
%<*luatex>
      \tex_pdfvariable:D objcompresslevel
%</luatex>
%<*pdftex>
      \tex_pdfobjcompresslevel:D
%</pdftex>
        #1 \scan_stop:
  }
%    \end{macrocode}
% \end{macro}
% \end{macro}
% \end{macro}
%
% \begin{macro}
%   {\@@_backend_version_major_gset:n, \@@_backend_version_minor_gset:n}
%   The availability of the primitive is not universal, so we have to test
%   at load time.
%    \begin{macrocode}
\cs_new_protected:Npe \@@_backend_version_major_gset:n #1
  {
%<*luatex>
    \int_compare:nNnT \tex_luatexversion:D > { 106 }
      {
        \exp_not:N \tex_global:D \tex_pdfvariable:D majorversion
          \exp_not:N \int_eval:n {#1} \scan_stop:
      }
%</luatex>
%<*pdftex>
    \cs_if_exist:NT \tex_pdfmajorversion:D
      {
        \exp_not:N \tex_global:D \tex_pdfmajorversion:D
          \exp_not:N \int_eval:n {#1} \scan_stop:
      }
%</pdftex>
  }
\cs_new_protected:Npn \@@_backend_version_minor_gset:n #1
  {
    \tex_global:D
%<*luatex>
      \tex_pdfvariable:D minorversion
%</luatex>
%<*pdftex>
      \tex_pdfminorversion:D
%</pdftex>
        \int_eval:n {#1} \scan_stop:
  }
%    \end{macrocode}
% \end{macro}
%
% \begin{macro}[EXP]{\@@_backend_version_major:, \@@_backend_version_minor:}
%   As above.
%    \begin{macrocode}
\cs_new:Npe \@@_backend_version_major:
  {
%<*luatex>
    \int_compare:nNnTF \tex_luatexversion:D > { 106 }
      { \exp_not:N \tex_the:D \tex_pdfvariable:D majorversion }
      { 1 }
%</luatex>
%<*pdftex>
    \cs_if_exist:NTF \tex_pdfmajorversion:D
      { \exp_not:N \tex_the:D \tex_pdfmajorversion:D }
      { 1 }
%</pdftex>
  }
\cs_new:Npn \@@_backend_version_minor:
  {
    \tex_the:D
%<*luatex>
      \tex_pdfvariable:D minorversion
%</luatex>
%<*pdftex>
      \tex_pdfminorversion:D
%</pdftex>
  }
%    \end{macrocode}
% \end{macro}
%
% \subsubsection{Marked content}
%
% \begin{macro}{\@@_backend_bdc:nn}
% \begin{macro}{\@@_backend_emc:}
%   Simple wrappers. May need refinement: see
%   \url{https://chat.stackexchange.com/transcript/message/49970158#49970158}.
%    \begin{macrocode}
\cs_new_protected:Npn \@@_backend_bdc:nn #1#2
  { \__kernel_backend_literal_page:n { /#1 ~ #2 ~ BDC } }
\cs_new_protected:Npn \@@_backend_emc:
  { \__kernel_backend_literal_page:n { EMC } }
%    \end{macrocode}
% \end{macro}
% \end{macro}
%
%    \begin{macrocode}
%</luatex|pdftex>
%    \end{macrocode}
%
% \subsection{\texttt{dvipdfmx} backend}
%
%    \begin{macrocode}
%<*dvipdfmx|xetex>
%    \end{macrocode}
%
% \begin{macro}{\@@_backend:n, \@@_backend:e}
%   A generic function for the backend PDF specials: used where we can.
%    \begin{macrocode}
\cs_new_protected:Npe \@@_backend:n #1
  { \__kernel_backend_literal:n { pdf: #1 } }
\cs_generate_variant:Nn \@@_backend:n { e }
%    \end{macrocode}
% \end{macro}
%
% \subsubsection{Catalogue entries}
%
% \begin{macro}{\@@_backend_catalog_gput:nn, \@@_backend_info_gput:nn}
%    \begin{macrocode}
\cs_new_protected:Npn \@@_backend_catalog_gput:nn #1#2
  { \@@_backend:n { put ~ @catalog << /#1 ~ #2 >> } }
\cs_new_protected:Npn \@@_backend_info_gput:nn #1#2
  { \@@_backend:n { docinfo << /#1 ~ #2 >> } }
%    \end{macrocode}
% \end{macro}
%
% \subsubsection{Objects}
%
% \begin{variable}{\g_@@_backend_object_prop}
%   For tracking objects to allow finalization.
%    \begin{macrocode}
\prop_new:N \g_@@_backend_object_prop
%    \end{macrocode}
% \end{variable}
%
% \begin{macro}{\@@_backend_object_new:}
% \begin{macro}[EXP]{\@@_backend_object_ref:n, \@@_backend_object_id:n}
%   Objects are tracked at the macro level, but we don't have to do anything
%   at this stage.
%    \begin{macrocode}
\cs_new_protected:Npn \@@_backend_object_new:
  { \int_gincr:N \g_@@_backend_object_int }
\cs_new:Npn \@@_backend_object_ref:n #1 { @pdf.obj #1 }
\cs_new_eq:NN \@@_backend_object_id:n \@@_backend_object_ref:n
%    \end{macrocode}
% \end{macro}
% \end{macro}
%
% \begin{macro}{\@@_backend_object_write:nnn, \@@_backend_object_write:nne}
% \begin{macro}
%   {
%     \@@_backend_object_write_array:nn   ,
%     \@@_backend_object_write_dict:nn    ,
%     \@@_backend_object_write_fstream:nn ,
%     \@@_backend_object_write_stream:nn
%   }
% \begin{macro}{\@@_backend_object_write_stream:nnnn}
%   This is where we choose the actual type.
%    \begin{macrocode}
\cs_new_protected:Npn \@@_backend_object_write:nnn #1#2#3
  {
    \use:c { @@_backend_object_write_ #2 :nn }
      { \@@_backend_object_ref:n {#1} } {#3}
  }
\cs_generate_variant:Nn \@@_backend_object_write:nnn { nne }
\cs_new_protected:Npn \@@_backend_object_write_array:nn #1#2
  {
    \@@_backend:e
      { obj ~ #1 ~ [ ~ \exp_not:n {#2} ~ ] }
  }
\cs_new_protected:Npn \@@_backend_object_write_dict:nn #1#2
  {
    \@@_backend:e
      { obj ~ #1 ~ << ~ \exp_not:n {#2} ~ >> }
  }
\cs_new_protected:Npn \@@_backend_object_write_fstream:nn #1#2
  { \@@_backend_object_write_stream:nnnn { f } {#1} #2 }
\cs_new_protected:Npn \@@_backend_object_write_stream:nn #1#2
  { \@@_backend_object_write_stream:nnnn { } {#1} #2 }
\cs_new_protected:Npn \@@_backend_object_write_stream:nnnn #1#2#3#4
  {
    \@@_backend:e
      {
        #1 stream ~ #2 ~
          ( \exp_not:n {#4} ) ~ << \exp_not:n {#3} >>
      }
  }
%    \end{macrocode}
% \end{macro}
% \end{macro}
% \end{macro}
%
% \begin{macro}{\@@_backend_object_now:nn, \@@_backend_object_now:ne}
%   No anonymous objects with \texttt{dvipdfmx} so we have to give an
%   object name.
%    \begin{macrocode}
\cs_new_protected:Npn \@@_backend_object_now:nn #1#2
  {
    \int_gincr:N \g_@@_backend_object_int
    \exp_args:Nne \use:c { @@_backend_object_write_ #1 :nn }
      { @pdf.obj \int_use:N \g_@@_backend_object_int }
      {#2}
  }
\cs_generate_variant:Nn \@@_backend_object_now:nn { ne }
%    \end{macrocode}
% \end{macro}
%
% \begin{macro}{\@@_backend_object_last:}
%    \begin{macrocode}
\cs_new:Npn \@@_backend_object_last:
  { @pdf.obj \int_use:N \g_@@_backend_object_int }
%    \end{macrocode}
% \end{macro}
%
% \begin{macro}[EXP]{\@@_backend_pageobject_ref:n}
%   Page references are easy in \texttt{dvipdfmx}/\XeTeX{}.
%    \begin{macrocode}
\cs_new:Npn \@@_backend_pageobject_ref:n #1
  { @page #1 }
%    \end{macrocode}
% \end{macro}
%
% \subsubsection{Destinations}
%
% \begin{macro}{\@@_backend_destination:nn}
% \begin{macro}{\@@_backend_destination:nnnn,\@@_backend_destination_aux:nnnn}
%   Here, we need to turn the zoom into a scale. The method for \texttt{FitR}
%   is from Alexander Grahn: the idea is to avoid needing to do any calculations
%   in \TeX{} by using the backend data for \texttt{@xpos} and \texttt{@ypos}.
%   \texttt{/FitR} without rule spec doesn't work, so it falls back to
%   \texttt{/Fit} here.
%    \begin{macrocode}
\cs_new_protected:Npn \@@_backend_destination:nn #1#2
  {
    \@@_backend:e
      {
        dest ~ ( \exp_not:n {#1} )
        [
          @thispage
          \str_case:nnF {#2}
            {
              { xyz }   { /XYZ ~ @xpos ~ @ypos ~ null }
              { fit }   { /Fit }
              { fitb }  { /FitB }
              { fitbh } { /FitBH }
              { fitbv } { /FitBV ~ @xpos }
              { fith }  { /FitH ~ @ypos }
              { fitv }  { /FitV ~ @xpos }
              { fitr }  { /Fit }
            }
            { /XYZ ~ @xpos ~ @ypos ~ \fp_eval:n { (#2) / 100 } }
        ]
      }
  }
\cs_new_protected:Npn \@@_backend_destination:nnnn #1#2#3#4
  {
    \exp_args:Ne \@@_backend_destination_aux:nnnn
      { \dim_eval:n {#2} } {#1} {#3} {#4}
  }
\cs_new_protected:Npn \@@_backend_destination_aux:nnnn #1#2#3#4
  {
    \vbox_to_zero:n
      {
        \__kernel_kern:n {#4}
        \hbox:n
          {
            \@@_backend:n { obj ~ @pdf_ #2 _llx ~ @xpos }
            \@@_backend:n { obj ~ @pdf_ #2 _lly ~ @ypos }
          }
        \tex_vss:D
      }
    \__kernel_kern:n {#1}
    \vbox_to_zero:n
      {
        \__kernel_kern:n { -#3 }
        \hbox:n
          {
            \@@_backend:n
              {
                dest ~ (#2)
                [
                  @thispage
                  /FitR ~
                    @pdf_ #2 _llx ~ @pdf_ #2 _lly ~
                    @xpos ~ @ypos
                ]
              }
          }
        \tex_vss:D
      }
    \__kernel_kern:n { -#1 }
  }
%    \end{macrocode}
% \end{macro}
% \end{macro}
%
% \subsubsection{Structure}
%
% \begin{macro}{\@@_backend_compresslevel:n}
% \begin{macro}{\@@_backend_compress_objects:n}
%   Pass data to the backend: these are a one-shot.
%    \begin{macrocode}
\cs_new_protected:Npn \@@_backend_compresslevel:n #1
  { \__kernel_backend_literal:e { dvipdfmx:config~z~ \int_eval:n {#1} } }
\cs_new_protected:Npn \@@_backend_compress_objects:n #1
  {
    \bool_if:nF {#1}
      { \__kernel_backend_literal:n { dvipdfmx:config~C~0x40 } }
  }
%    \end{macrocode}
% \end{macro}
% \end{macro}
%
% \begin{macro}
%   {\@@_backend_version_major_gset:n, \@@_backend_version_minor_gset:n}
%   We start with the assumption that the default is active.
%    \begin{macrocode}
\cs_new_protected:Npn \@@_backend_version_major_gset:n #1
  {
    \cs_gset:Npe \@@_backend_version_major: { \int_eval:n {#1} }
    \__kernel_backend_literal:e { pdf:majorversion~ \@@_backend_version_major: }
  }
\cs_new_protected:Npn \@@_backend_version_minor_gset:n #1
  {
    \cs_gset:Npe \@@_backend_version_minor: { \int_eval:n {#1} }
    \__kernel_backend_literal:e { pdf:minorversion~ \@@_backend_version_minor: }
  }
%    \end{macrocode}
% \end{macro}
%
% \begin{macro}[EXP]{\@@_backend_version_major:, \@@_backend_version_minor:}
%   We start with the assumption that the default is active.
%    \begin{macrocode}
\cs_new:Npn \@@_backend_version_major: { 1 }
\cs_new:Npn \@@_backend_version_minor: { 5 }
%    \end{macrocode}
% \end{macro}
%
% \subsubsection{Marked content}
%
% \begin{macro}{\@@_backend_bdc:nn}
% \begin{macro}{\@@_backend_emc:}
%   Simple wrappers. May need refinement: see
%   \url{https://chat.stackexchange.com/transcript/message/49970158#49970158}.
%    \begin{macrocode}
\cs_new_protected:Npn \@@_backend_bdc:nn #1#2
  { \__kernel_backend_literal_page:n { /#1 ~ #2 ~ BDC } }
\cs_new_protected:Npn \@@_backend_emc:
  { \__kernel_backend_literal_page:n { EMC } }
%    \end{macrocode}
% \end{macro}
% \end{macro}
%
%    \begin{macrocode}
%</dvipdfmx|xetex>
%    \end{macrocode}
%
% \subsection{\texttt{dvisvgm} backend}
%
%    \begin{macrocode}
%<*dvisvgm>
%    \end{macrocode}
%
% \subsubsection{Destinations}
%
% \begin{macro}{\@@_backend_destination:nn}
% \begin{macro}{\@@_backend_destination:nnnn}
%    \begin{macrocode}
\cs_new_protected:Npn \@@_backend_destination:nn #1#2 { }
\cs_new_protected:Npn \@@_backend_destination:nnnn #1#2#3#4 { }
%    \end{macrocode}
% \end{macro}
% \end{macro}
%
% \subsubsection{Catalogue entries}
%
% \begin{macro}{\@@_backend_catalog_gput:nn, \@@_backend_info_gput:nn}
%   No-op.
%    \begin{macrocode}
\cs_new_protected:Npn \@@_backend_catalog_gput:nn #1#2 { }
\cs_new_protected:Npn \@@_backend_info_gput:nn #1#2 { }
%    \end{macrocode}
% \end{macro}
%
% \subsubsection{Objects}
%
% \begin{macro}{\@@_backend_object_new:}
% \begin{macro}[EXP]{\@@_backend_object_ref:n, \@@_backend_object_id:n}
% \begin{macro}{\@@_backend_object_write:nnn, \@@_backend_object_write:ne}
% \begin{macro}{\@@_backend_object_now:nn, , \@@_backend_object_now:ne}
% \begin{macro}{\@@_backend_object_last:}
% \begin{macro}[EXP]{\@@_backend_pageobject_ref:n}
%   All no-ops here.
%    \begin{macrocode}
\cs_new_protected:Npn \@@_backend_object_new: { }
\cs_new:Npn \@@_backend_object_ref:n #1 { }
\cs_new:Npn \@@_backend_object_id:n #1 { }
\cs_new_protected:Npn \@@_backend_object_write:nnn #1#2#3 { }
\cs_new_protected:Npn \@@_backend_object_write:nne #1#2#3 { }
\cs_new_protected:Npn \@@_backend_object_now:nn #1#2 { }
\cs_new_protected:Npn \@@_backend_object_now:ne #1#2 { }
\cs_new:Npn \@@_backend_object_last: { }
\cs_new:Npn \@@_backend_pageobject_ref:n #1 { }
%    \end{macrocode}
% \end{macro}
% \end{macro}
% \end{macro}
% \end{macro}
% \end{macro}
% \end{macro}
%
% \subsubsection{Structure}
%
% \begin{macro}{\@@_backend_compresslevel:n}
% \begin{macro}{\@@_backend_compress_objects:n}
%   These are all no-ops.
%    \begin{macrocode}
\cs_new_protected:Npn \@@_backend_compresslevel:n #1 { }
\cs_new_protected:Npn \@@_backend_compress_objects:n #1 { }
%    \end{macrocode}
% \end{macro}
% \end{macro}
%
% \begin{macro}
%   {\@@_backend_version_major_gset:n, \@@_backend_version_minor_gset:n}
%   Data not available!
%    \begin{macrocode}
\cs_new_protected:Npn \@@_backend_version_major_gset:n #1 { }
\cs_new_protected:Npn \@@_backend_version_minor_gset:n #1 { }
%    \end{macrocode}
% \end{macro}
%
% \begin{macro}[EXP]{\@@_backend_version_major:, \@@_backend_version_minor:}
%   Data not available!
%    \begin{macrocode}
\cs_new:Npn \@@_backend_version_major: { -1 }
\cs_new:Npn \@@_backend_version_minor: { -1 }
%    \end{macrocode}
% \end{macro}
%
% \begin{macro}{\@@_backend_bdc:nn}
% \begin{macro}{\@@_backend_emc:}
%   More no-ops.
%    \begin{macrocode}
\cs_new_protected:Npn \@@_backend_bdc:nn #1#2 { }
\cs_new_protected:Npn \@@_backend_emc: { }
%    \end{macrocode}
% \end{macro}
% \end{macro}
%
%    \begin{macrocode}
%</dvisvgm>
%    \end{macrocode}
%
% \subsection{PDF Page size (media box)}
%
% For setting the media box, the split between backends is somewhat different
% to other areas, thus we approach this separately. The code here assumes a
% recent \LaTeXe{}: that is ensured at the level above.
%
%    \begin{macrocode}
%<*dvipdfmx|dvips>
%    \end{macrocode}
%
% \begin{macro}{\@@_backend_pagesize_gset:nn}
%   This is done as a backend literal, so we deal with it using the shipout
%   hook.
%    \begin{macrocode}
\cs_new_protected:Npn \@@_backend_pagesize_gset:nn #1#2
  {
    \__kernel_backend_first_shipout:n
      {
        \__kernel_backend_literal:e
          {
%<*dvipdfmx>
            pdf:pagesize ~
              width  ~ \dim_eval:n {#1} ~
              height ~ \dim_eval:n {#2}
%</dvipdfmx>
%<*dvips>
            papersize = \dim_eval:n {#1} , \dim_eval:n {#2}
%</dvips>
          }
      }
  }
%    \end{macrocode}
% \end{macro}
%
%    \begin{macrocode}
%</dvipdfmx|dvips>
%    \end{macrocode}
%
%    \begin{macrocode}
%<*luatex|pdftex|xetex>
%    \end{macrocode}
%
% \begin{macro}{\@@_backend_pagesize_gset:nn}
%   Pass to the primitives.
%    \begin{macrocode}
\cs_new_protected:Npn \@@_backend_pagesize_gset:nn #1#2
  {
    \dim_gset:Nn \tex_pagewidth:D  {#1}
    \dim_gset:Nn \tex_pageheight:D {#2}
  }
%    \end{macrocode}
% \end{macro}
%
%    \begin{macrocode}
%</luatex|pdftex|xetex>
%    \end{macrocode}
%
%    \begin{macrocode}
%<*dvisvgm>
%    \end{macrocode}
%
% \begin{macro}{\@@_backend_pagesize_gset:nn}
%   A no-op.
%    \begin{macrocode}
\cs_new_protected:Npn \@@_backend_pagesize_gset:nn #1#2 { }
%    \end{macrocode}
% \end{macro}
%
%    \begin{macrocode}
%</dvisvgm>
%    \end{macrocode}
%
%    \begin{macrocode}
%</package>
%    \end{macrocode}
%
% \end{implementation}
%
% \PrintIndex
