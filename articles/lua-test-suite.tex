\documentclass[final]{ltugboat}

\makeatletter
\def\thepage{%
  \ifnum\value{page}>1000
    \textit{?\@arabic{\numexpr\the\c@page-1000\relax}}%
  \else
       \arabic{page}%
  \fi}
\makeatother


\usepackage{url}
\usepackage{fancyvrb}
\usepackage{ifmtarg}
\usepackage{microtype}

\begin{document}
\title{\texttt{l3build}\Dash A modern Lua test suite for~\TeX~programming}
  \date{2014-09-26}

\author{Frank Mittelbach}
\address{Mainz, Germany}

\author{Will Robertson}
\address{School of Mechanical Engineering, The University of Adelaide, Australia}

\author{The~\LaTeX3 team}
\netaddress{http://www.latex-project.org}



\newcommand\drivername{build.lua}
\newcommand\makename{l3build.lua}
\newcommand\execname{texlua \drivername}
\newcommand\compdirname{test/}

\newcommand\pdfTeX{pdf\TeX}
\newcommand\luaTeX{Lua\TeX}

\maketitle

\tableofcontents

\section{Introduction}

Regression tests are an important tool in any moderately complex
programming environment.  They allow the programmer to make extensive
changes to their code while providing confidence that something that
used to work still does.  Extensive regression test suites have been
an essential component of the maintenance and development of \LaTeXe{}
and \LaTeX3.

A regression test suite is typically composed of a number of
individual files that contain one or more testable units of the code
being tested. A testable unit might be either a certain computation
with an expected outcome, a series of logic tests, or\Dash in
particular for \TeX{}-based code\Dash material that is typeset and
intended to achieve some particular formatting.

During code development and before any new code is released to the
public, this test suite can be compiled to ensure that any changes to
the code have not introduced bugs or changed the behaviour compared to
previous versions.
As bugs in the code are reported, minimal examples demonstrating the
bug often form test files of their own, showing that the bug has been
fixed and won't re-occur.

As \TeX{}-based code operates in at least three different `modes'
(mouth, stomach, and output), regression testing is more complex than
simply asserting the outcome of certain programming logic.  As part of
the work of the \LaTeX3 project, a new Lua-based testing environment
has been written to support ongoing development.  This testing
environment, presented at the 2014 \tug\ conference in
Portland~\cite{Mittelbach:l3buildtalk}, is suitable for use by the
general \TeX{} community.


\section{History}

The ideas for a regression test suite for \LaTeX{} date back to the
early nineties\footnote{As with many ideas in the \TeX{} world, this one too
  can be partly traced back to Don Knuth, who already provided his own
  regression test for \TeX{} a decade
  earlier~\cite{Knuth:1984:TTT}.} when \LaTeX\,2.09 existed in various
incompatible flavours around the world due to its limitations in
properly supporting font selection, complex mathematics, and languages
other than English.  Because of that situation \LaTeXe{} was designed
and implemented to reunite the different format and to provide a
stable platform for future \LaTeX{} development.

However, to successfully introduce \LaTeXe{} as an accepted successor
of \LaTeX\,2.09 it was essential to win over the huge \LaTeX{} user
base and provide them with a system that was as stable and upward
compatible as possible. Thus existing user interfaces should be
preserved and typesetting should provide identical output except in
those cases where bug fixes or deliberate design decisions resulted in
changes.

To achieve this we devised a validation mechanism that could be used
to ensure that interfaces behave as expected and typesetting results
do not change even though the underlying code gets modified. With
this in place the \LaTeX3 Project Team together with additional
volunteers set out to create a large number of test files and
verify them against the current \LaTeX\,2.09
implementation. Figure~\ref{fig:volunteers} shows the original request
for volunteers (exhibiting a severe underestimation of the amount of
work involved); see also \cite{Mittelbach:TB18-4-309} for a more extensive
description of this endeavor. 

This effort resulted in something like 200 test files
that were then used to assure ourselves that the new \LaTeXe{}
implementation was faithfully supporting all interfaces\Dash it was one
of the key factors that ensured the new system became an accepted replacement
for \LaTeX\,2.09 within a reasonably short period.

\begin{figure}
\centering
\fbox{\begin{minipage}{.9\linewidth}
\raggedright\hbadness=10000
\textbf{Validating \LaTeX\,2.09}\\[\smallskipamount]
%
Writing test files for regression testing: checking bug fixes and
improvements to verify that they don't have undesirable side
effects; making sure that bug fixes really correct the problem
they were intended to correct; testing interaction with various
document styles, style options, and environments. We~would
like three kinds of validation files:
\begin{enumerate}
\item General documents.
\item Exhaustive tests of special environments/modules such as
tables, displayed equations, theorems, floating figures,
pictures, etc.
\item Bug files containing tests of all bugs that are supposed to
be fixed (as well as those that are not fixed, with comments
about their status).
\end{enumerate}
A procedure for processing validation files has been devised;
details will be furnished to anyone interested in this task.
Estimated time required: 2 to 3 weeks, could be divided~up.
\end{minipage}}
\caption{Original request for volunteers}\label{fig:volunteers}
\end{figure}

Once in place this regression test suite was augmented over time and
now contains roughly 350 test files altogether. Whenever a bug was
found and fixed we added a new test file that would exhibit the undesired behavior
if that bug would somehow resurface through later changes. 

Though not
perfect (after all we introduced a number of bugs that initially were
not caught by the regression test suite), the approach served us very
well and prevented a number of horrible mistakes that would otherwise
have made it into public releases of~\LaTeX.

\subsection{The needs in the '90s}

With the initial regression test suite we solved a number of burning
problems. First of all we wanted to be confident that the code and the
documented user interfaces worked as expected. Whenever we recoded
an internal function the test suite would automatically alert us if that
resulted in any noticeable changes at the user level or in downright
bugs. 

Furthermore \LaTeXe{} came with much more documentation and the
tests included compiling and checking the documentation files for
errors and
missing
references.% just seemed to flow better this way

In addition the Makefiles that ran the tests also included goals to
build the distribution automatically.  Compared to \LaTeX\,2.09,
which consisted of very few files, the format for \LaTeXe{} was
generated from many source \texttt{.dtx} files, so the housekeeping
complexity was greatly increased.

Another issue we had to tackle was that the code was no longer
maintained by a single person but by developers living in different
places around the world and using different operating systems and
installations. So the regression suite had to function with different
installations without creating spurious differences.

Finally all tasks had to work without user intervention or manual work because
only in that case will such a system be used on a regular basis
and thus benefits be realized.


\subsection{The general approach}

Designing a test system for verifying \TeX's typesetting behavior is
not easy\Dash how do you test for correctness and how do you ensure
that the tests are repeatable over time and in different places?

The approach we came up with was to build test files that generate
suitable data in their \texttt{.log} files. Suitable data would be, for
example,
the state of counters or dimensions produced with \cs{showthe}, data
written with \cs{typeout}, and box content shown with
\cs{showbox}. Some of the tracing parameters of \TeX{} could be used to
verify paragraph building or page breaking decisions, but something
like \cs{tracingall} would be inadvisable, as that would show the
internal coding and not the expected functionality.

The result of running such a test file would then be manually verified
and stored away as a certified result. However, as many readers will
already be aware,
\LaTeX's \texttt{.log} files contain a lot of irrelevant data, some of
which differs from run to run and some of which differs when running
on different installations. So to make this approach workable we
introduced a cleanup step in which we modified the result files
removing irrelevant material and normalized some of the remaining
parts. Of course one has to be careful not to sanitize too far, but we
found a number of things necessary or at least advisable, including
\begin{itemize}
\item  shortening file path info to avoid differences between
installations
\item drop empty lines (different \TeX{} implementations put
in different numbers of these)
\item drop line numbers in `\texttt{on line <num>}' to avoid differences
 just because extra lines got introduced in a test file.
\end{itemize}
Putting it all together we ended up with a system consisting of
test files (with the extension \texttt{.lvt}), certified result files
to compare against (extension \texttt{.tlg}) and a fairly complex
Makefile and a number of Perl scripts used to run the different
tasks. These tasks included running the test suite, producing the
documentation and generating the distribution (ready to be shipped to
\CTAN). It also contained a number of special functions such as
unpacking and locally installing the code, cleaning up the source
directories, checking individual test files, and % Oxford comma necessary for comprehension
producing a new
\texttt{.tlg} file for a given test file.


\subsection{The new needs (in the new century)}

As mentioned above, the initial system served us well, when moving
from \LaTeX\,2.09 to \LaTeXe{} and then throughout the '90s, which
had very active \LaTeXe{} development with releases produced at
half-year intervals.% bad break but not sure what to do

In this century, development of the core of the \LaTeXe{} kernel has slowed
to a minimum (releases are now only every couple of years and the
changes are small) % avoid repeating "minim*"
while it has intensified in other areas such as
actively progressing the development of the \LaTeX3 programming
language \texttt{expl3}. With this new focus, newly important requirements for a
regression test system became apparent.

Instead of a single distribution we now had to deal with a growing
number of distributions: core \LaTeXe{} and its packages, Babel (with
a different release cycle), \texttt{expl3} and possibly smaller and
larger distributions of third party code that also wanted to benefit
from a functional regression test system.

Windows and \MacOSX\ became the operating systems of choice for several
developers and the Makefile approach of the original test suite did not
work on Windows and only with modifications on \MacOSX.

Last but not least, a number of new \TeX-based engines matured and people
now wanted to use \LaTeX{} and friends not only on \pdfTeX{} but also
on these new engines all of which provided additional capabilities.
These new engines showed a number of subtle differences when adding
data to the \texttt{.log} file, or due to extended capabilities showed
additional data (such as extra nodes in listings). Furthermore the new
engines still have bugs and a number of them showed up when we
initially ran test files and compared their output with the certified
\texttt{.tlg} data.

Thus testing became a multi-dimensional problem: one had to verify
test results with several engines and it had to work on multiple
operating systems. Furthermore new code sources posed new or different
requirements for building a distribution or doing the testing and
we soon found that the original approach made a number of hardwired
decisions that were no longer applicable if the system was used with a
distribution different from \LaTeXe{}.

For a short while we tried to accommodate the need for Windows support
by using a set of \texttt{.bat} files in parallel with the Makefile
approach but obviously that was doomed to failure, being impractical to
maintain. Another avenue we explored was switching to a fully
Perl-based approach (using Cons) but that again didn't work well with
Windows and furthermore it would have been a solution not available
out of the box on any \TeX{} installation.

Eventually, we decided to apply the same principle used
long ago with \texttt{docstrip.tex}: use the scripting language with
some operating system capabilities that is available out of the box on
all \TeX{} installations. Back then the answer was that only \TeX{} itself
fit that bill and so \TeX{} became the tool to build style files,
etc., from \texttt{.dtx} sources. However, while \TeX{} as such is too
limited to be used for scripting a regression test system, we now had
\luaTeX{} as an engine that offers a full-fledged Lua interpreter\Dash
and these days \luaTeX{} is part of all modern \TeX{} installations.

Moving to Lua (or \texttt{texlua} to be precise) means that the test
and distribution system is now not tied to either the operating system
(as the script runs on Windows and Unix variants) or to third-party
tools (as Lua is available as part of a modern \TeX{} system).

\[
\centerline{$--*--$}
\]

\noindent In the remaining sections of this article we describe the new system
and how it can be applied to support arbitrary code within the \TeX{}
world.

\section{Overview of the new system}
\label{sec:overview}

To illustrate, a hypothetical package will be described that
uses the new system:
consider a package \textsf{abc} with a collection of source
files in the following layout.
\begin{Verbatim}
  abc/
      abc.dtx
      abc.ins
      build.lua
      README
      testfiles/
                test1.lvt
                test1.tlg
                ...
                support/
                        abc-test.cls
\end{Verbatim}
What is added in addition to the normal source files is a short Lua
script, normally called \texttt{\drivername}.  Test files and their
certified results are located in the folder `\texttt{testfiles/}'
with extensions \texttt{.lvt} and \texttt{.tlg}, respectively.  The files in
\texttt{support/} (if any) are used when running the test files.

Upon
running the test suite, a new folder `\texttt{build}' is created in
which the package is unpacked, support files are copied across, and
each test file is run in turn and compared to its original
\texttt{.tlg} file. Directories and file names are adjustable and
other setups are possible; the above structure is simply the default.


\subsection{Modes of testing}

The best way to perform regression tests for \TeX{} programming is to
use the \texttt{.log} file; only here
can box content be tested, not just logical and
programmatic constructs.  Box
content is essential for checking from the very highest level that
code changes do not result in different typeset output.

\TeX{} programming can be either \emph{expandable} or not.
Code that is expected to be
expandable should be tested as such.  This can be done by evaluating it within something like \cs{typeout} (in the case of \LaTeX). For non-expandable tests one should output
their results to the \texttt{.log} once they have been evaluated.
%
As mentioned earlier there are also a number of \TeX{} tracing parameters
and commands like \cs{showbox}, \cs{showlists}, or \cs{showthe} that
can be used to generate relevant test data in the \texttt{.log}~file.

To aid in producing a structured test suite we provide a number of
commands for use in the test files.  The \cs{TYPE} command is used to write
material to the \texttt{.log} file; it works like \cs{typeout}, but it
allows `long' input.
%
A variety of commands, following, then use \cs{TYPE} to output strings
to the \texttt{.log} file.
\begin{itemize}
\item
\cs{SEPARATOR} inserts a long line of \texttt{=} symbols to break up
the output.
\item
\cs{TRUE}, \cs{FALSE}, \cs{YES}, \cs{NO} insert text strings for
standardized comparison.
\item
\cs{ERROR} is not defined but is commonly used to indicate a code path
that should never be reached.
\end{itemize}
To produce individual tests we offer the commands \cs{TEST} and
\cs{TESTEXP}. These commands take two arguments: a title
and the actual test body.  \cs{TESTEXP} executes the body within a
\cs{TYPE} command to test expandability but with \cs{TEST} you are
responsible for generating test output using \cs{TYPE}, \cs{TRUE},
etc.\ as it is intended to be used for non-expandable tests.
Both  commands  surround the generated output with \cs{SEPARATOR}s and 
display the title and a test number. Here is an example:
\begin{Verbatim}
\begin{TEST}{stepping counters}
 {
  \setcounter{chapter}{2}
  \setcounter{section}{5}
  \setcounter{subsection}{4}
  \stepcounter{chapter}%
  \TYPE{\arabic{chapter}-%
    \arabic{section}-\arabic{subsection}}
  \SEPARATOR
  \refstepcounter{section}
  \TYPE{\arabic{chapter}-%
    \arabic{section}-\arabic{subsection}}
 }
\end{Verbatim}
This test will then produce the following output, as in standard \LaTeX{} only
a counter directly ``within'' is reset to zero (e.g., the
\texttt{subsection} counter is not touched when \texttt{chapter} is stepped):
\begin{Verbatim}
==========================================
TEST 8: stepping counters
==========================================
3-0-4
==========================================
3-1-0
==========================================
\end{Verbatim}
(Assuming it's the eighth test in the file.)



%\section{Use of the regression test system}
\section{Setting up the regression test system}

Consider the case that a \LaTeX\ package consists of one or more
\texttt{.dtx} files in a flat directory structure.  By default,
to set up a regression test suite, you would create a driver file
named `\texttt{\drivername}' and sub-folder named
`\texttt{testfiles/}' to contain the test files.  An example driver
file is shown in Section~\ref{sec:example}.

The test files can be called basically anything (but should be logical
in some way), and by default have the extension \texttt{.lvt}.  These are
accompanied by a pre-saved \texttt{.tlg} file which contains the
`results' of the test file to be checked against subsequent
compilation of that test. If a test file has different results for
different engines it is possible to ``certify'' \texttt{.tlg} files for
each engine; those then have extensions such as \texttt{.luatex.tlg}.


\subsection{Creating and checking test output}

The first time a \texttt{.lvt} test file is written, it will need to
be compiled to obtain the necessary \texttt{.tlg} output for future
tests.  This is performed with:
\begin{quote}\ttfamily
\execname~save~\meta{test name}
\end{quote}
(To produce an engine-specific \texttt{.tlg} file an additional
\meta{engine} argument can be given.)  This task can be re-run as
many times as necessary until the test file demonstrates the necessary
behaviour being tested.
%
At this point, 
\begin{quote}\ttfamily
\execname~check~\meta{test name}
\end{quote}
will then re-run the \texttt{.lvt} file and compare the result to the
original \texttt{.tlg} output.  
If no \meta{test name} is specified all tests in the test directory are run. Presuming no code has changed to
affect the output of the tests, the console output of this task will show the name of the test files being processed followed by the line:
\begin{quote}\ttfamily
~~All checks passed
\end{quote}
If only one test file is run the usual console output from the \TeX{} compilation is also shown otherwise it is suppressed.

These compilations take place in the subdirectory
`\texttt{build/test}', and if a test fails, a diff file is deposited
there with the information about what has changed in the output of the
test file. Also deposited there are the full \texttt{.log} files  for each \meta{engine} (i.e., without modifications from the cleanup step) which can be helpful to debug complex issues.

\divide\abovecaptionskip by2
\subsection{An example driver file}
\label{sec:example}

For a simple setup such as shown in the overview in
Section~\ref{sec:overview}, the driver file (\texttt{\drivername}) is
quite simple.  An example of such a driver file is shown in
Figure~\ref{fig:driver}; it need do little more than inform the build
system of the name of the package and perhaps set some flags or change
some defaults if they are not adequate.

The main script is \texttt{l3build.lua}, which is automatically
found in the \texttt{texmf} tree (via \texttt{kpsewhich}) and then
loaded.  Thus, there is no need to hard-wire locations in the driver
file and it will work on different installations.

\begin{figure}
\begin{Verbatim}[frame=single,fontsize=\small]
#!/ usr/bin/env texlua

 -- Build script for abc package

 module = "abc"

 -- variable overwrites (if needed)

 -- call standard script

 kpse.set_program_name ("kpsewhich")
 dofile (kpse.lookup ("l3build.lua")) 
\end{Verbatim}
\caption{Driver file for a hypothetical \textsf{abc} package}
\label{fig:driver}
\end{figure}


\begin{figure}
\begin{Verbatim}[frame=single,fontsize=\small]
\documentclass{breqn-test}
%%
%% This is file `regression-test.tex',
%% generated with the docstrip utility.
%%
%% The original source files were:
%%
%% l3build.dtx  (with options: `package')
%% 
%% Copyright (C) 2014-2016 The LaTeX3 Project
%% 
%% It may be distributed and/or modified under the conditions of
%% the LaTeX Project Public License (LPPL), either version 1.3c of
%% this license or (at your option) any later version.  The latest
%% version of this license is in the file:
%% 
%%    http://www.latex-project.org/lppl.txt
%% 
%% This file is part of the "l3build bundle" (The Work in LPPL)
%% and all files in that bundle must be distributed together.
%% 
\begingroup\expandafter\expandafter\expandafter\endgroup
\expandafter\ifx\csname eTeXversion\endcsname\relax
  \errmessage{e-TeX is required to use regression-test.tex}%
  \expandafter\endinput
\fi
\ifx\unprotect\undefined
  \expandafter\edef\csname reset\string @catcodes\endcsname{%
    \catcode`\noexpand\@=\the\catcode`\@\relax
  }%
  \catcode`\@=11 %
\else
  \unprotect
  \def\reset@catcodes{\protect}%
\fi
\ifnum\interactionmode>1 \scrollmode\fi
\errorcontextlines=-1 %
\showboxbreadth=\maxdimen
\showboxdepth=\maxdimen
\def\loggingoutput{%
  \tracingoutput=1 %
  \showboxbreadth=\maxdimen
  \showboxdepth=\maxdimen
}
\newlinechar=`\^^J
\long\def\LONGTYPEOUT#1{%
  \begingroup
    \long\def\TYPE##1{##1}%
    \immediate\write128{#1}%
  \endgroup
}
\let\TYPE\LONGTYPEOUT
\def\STARTMESSAGE{This is a generated file for the l3build validation system.}
\def\START{%
  \LONGTYPEOUT{^^JSTART-TEST-LOG^^J}%
  \LONGTYPEOUT{^^J%
     \STARTMESSAGE%
     ^^J^^JDon't change this file in any respect.%
     ^^J^^J%
  }%
}
\ifx\@@end\@undefined
  \let\@@@end\end
\else
  \let\@@@end\@@end
\fi
\def\END{%
  \ifnum\currentgrouplevel>0 %
    \LONGTYPEOUT{Bad grouping: \the\currentgrouplevel!}%
  \fi
  \ifnum\currentiflevel>2 %
    \LONGTYPEOUT{Bad conditionals: \the\numexpr\currentiflevel-2!}%
  \fi
  \LONGTYPEOUT{^^JEND-TEST-LOG^^J}%
  \@@@end
}
\ifx\@@end\@undefined
  \let\end\END
\else
  \let\@@end\END
\fi
\def\OMIT{\LONGTYPEOUT{OMIT}}
\def\TIMO{\LONGTYPEOUT{TIMO}}
\ifx\InputIfFileExists\@undefined
  \newread\@inputcheck
  \long\def\InputIfFileExists#1#2#3{%
    \openin\@inputcheck#1\relax
    \ifeof\@inputcheck
      \def\reserved@a{#3}%
    \else
      \def\reserved@a{#2\input #1\relax}%
    \fi
    \closein\@inputcheck
    \reserved@a
  }%
\fi
\InputIfFileExists{regression-test.cfg}
  {\LONGTYPEOUT{^^J***^^Jregression-test.cfg in operation^^J***^^J}}{}
\newcount\gTESTint
\def\SEPARATOR{%
  \TYPE{%
    ============================================================%
  }%
}
\long\def\TEST#1#2{%
  \global\advance\gTESTint by 1 %
  \SEPARATOR
  \LONGTYPEOUT{TEST \the\gTESTint: \detokenize{#1}}%
  \SEPARATOR
  \begingroup
    \let\TYPE\LONGTYPEOUT
    #2%
  \endgroup
  \SEPARATOR
  \LONGTYPEOUT{}%
}
\long\def\TESTEXP#1#2{%
  \global\advance\gTESTint by 1 %
  \SEPARATOR
  \LONGTYPEOUT{%
    TEST \the\gTESTint: \detokenize{#1}}%
  \SEPARATOR
  \begingroup
    \long\def\TYPE##1{##1}%
    \LONGTYPEOUT{#2}%
  \endgroup
  \SEPARATOR
  \LONGTYPEOUT{}%
}
\def \TRUE  {\TYPE{TRUE}}
\def \FALSE {\TYPE{FALSE}}
\def \YES   {\TYPE{YES}}
\def \NO    {\TYPE{NO}}
\def \NEWLINE {\TYPE{^^J}}
\ifx\pdfoutput\@undefined
  \ifx\outputmode\@undefined
  \else
    \ifnum\outputmode>0 %
      \pdfextension mapfile{pdftex.map}%
    \fi
  \fi
\else
  \ifnum\pdfoutput>0 %
    \pdfmapfile{pdftex.map}%
  \fi
\fi
\ifnum 0%
  \ifx\pdfoutput\@undefined\else\ifnum\pdfoutput>0 1\fi\fi
  \ifx\outputmode\@undefined\else\ifnum\outputmode>0 1\fi\fi
  >0 %
  \ifx\pdfvariable\@undefined
    \pdfinfo{/Producer (\ifx\directlua\@undefined pdf\else Lua\fi TeX)}
    \ifx\pdfinfoomitdate\@undefined\else
      \pdfinfoomitdate     = 1 %
      \pdfsuppressptexinfo = 1 %
      \pdftrailerid{}
    \fi
  \else
    \pdfextension info{/Producer (LuaTeX)}
    \pdfvariable suppressoptionalinfo \numexpr
          0
        +   1 % PTEX.Fullbanner
        +  32 % CreationDate
        +  64 % ModDate
      \relax
  \fi
\else
  \special{%
    pdf: docinfo
      <<
        /Creator        (TeX)
        /CreationDate   ()
        /ModDate        ()
        /Producer       (\ifx\XeTeXversion\@undefined\else x\fi dvipdfmx)
      >>
  }
\fi
\reset@catcodes
%% 
%%
%% End of file `regression-test.tex'.

\usepackage{breqn}
\begin{document}
\START
\AUTHOR{Will Robertson}
\begin{dmath}
a+b+c+d+e+f+g+h+i+j+k+l+m+
  n+o+p+q+r+s+t+u+v+w+x+y+z
\end{dmath}
\showoutput
\end{document}
\end{Verbatim}
\caption{Example test from \textsf{breqn}}
\label{fig:breqn}
\end{figure}

\subsection{The structure of test files}

As mentioned previously, the method of using the \texttt{.log} file
allows various types of tests to be conducted.  The most simple test
might load a package and execute some commands to produce a small
amount of typeset output.  A complete example of such a test is shown
in Figure~\ref{fig:breqn}.  Some points to note:
\begin{enumerate}
\item The first line, \verb|%%
%% This is file `regression-test.tex',
%% generated with the docstrip utility.
%%
%% The original source files were:
%%
%% l3build.dtx  (with options: `package')
%% 
%% Copyright (C) 2014-2016 The LaTeX3 Project
%% 
%% It may be distributed and/or modified under the conditions of
%% the LaTeX Project Public License (LPPL), either version 1.3c of
%% this license or (at your option) any later version.  The latest
%% version of this license is in the file:
%% 
%%    http://www.latex-project.org/lppl.txt
%% 
%% This file is part of the "l3build bundle" (The Work in LPPL)
%% and all files in that bundle must be distributed together.
%% 
\begingroup\expandafter\expandafter\expandafter\endgroup
\expandafter\ifx\csname eTeXversion\endcsname\relax
  \errmessage{e-TeX is required to use regression-test.tex}%
  \expandafter\endinput
\fi
\ifx\unprotect\undefined
  \expandafter\edef\csname reset\string @catcodes\endcsname{%
    \catcode`\noexpand\@=\the\catcode`\@\relax
  }%
  \catcode`\@=11 %
\else
  \unprotect
  \def\reset@catcodes{\protect}%
\fi
\ifnum\interactionmode>1 \scrollmode\fi
\errorcontextlines=-1 %
\showboxbreadth=\maxdimen
\showboxdepth=\maxdimen
\def\loggingoutput{%
  \tracingoutput=1 %
  \showboxbreadth=\maxdimen
  \showboxdepth=\maxdimen
}
\newlinechar=`\^^J
\long\def\LONGTYPEOUT#1{%
  \begingroup
    \long\def\TYPE##1{##1}%
    \immediate\write128{#1}%
  \endgroup
}
\let\TYPE\LONGTYPEOUT
\def\STARTMESSAGE{This is a generated file for the l3build validation system.}
\def\START{%
  \LONGTYPEOUT{^^JSTART-TEST-LOG^^J}%
  \LONGTYPEOUT{^^J%
     \STARTMESSAGE%
     ^^J^^JDon't change this file in any respect.%
     ^^J^^J%
  }%
}
\ifx\@@end\@undefined
  \let\@@@end\end
\else
  \let\@@@end\@@end
\fi
\def\END{%
  \ifnum\currentgrouplevel>0 %
    \LONGTYPEOUT{Bad grouping: \the\currentgrouplevel!}%
  \fi
  \ifnum\currentiflevel>2 %
    \LONGTYPEOUT{Bad conditionals: \the\numexpr\currentiflevel-2!}%
  \fi
  \LONGTYPEOUT{^^JEND-TEST-LOG^^J}%
  \@@@end
}
\ifx\@@end\@undefined
  \let\end\END
\else
  \let\@@end\END
\fi
\def\OMIT{\LONGTYPEOUT{OMIT}}
\def\TIMO{\LONGTYPEOUT{TIMO}}
\ifx\InputIfFileExists\@undefined
  \newread\@inputcheck
  \long\def\InputIfFileExists#1#2#3{%
    \openin\@inputcheck#1\relax
    \ifeof\@inputcheck
      \def\reserved@a{#3}%
    \else
      \def\reserved@a{#2\input #1\relax}%
    \fi
    \closein\@inputcheck
    \reserved@a
  }%
\fi
\InputIfFileExists{regression-test.cfg}
  {\LONGTYPEOUT{^^J***^^Jregression-test.cfg in operation^^J***^^J}}{}
\newcount\gTESTint
\def\SEPARATOR{%
  \TYPE{%
    ============================================================%
  }%
}
\long\def\TEST#1#2{%
  \global\advance\gTESTint by 1 %
  \SEPARATOR
  \LONGTYPEOUT{TEST \the\gTESTint: \detokenize{#1}}%
  \SEPARATOR
  \begingroup
    \let\TYPE\LONGTYPEOUT
    #2%
  \endgroup
  \SEPARATOR
  \LONGTYPEOUT{}%
}
\long\def\TESTEXP#1#2{%
  \global\advance\gTESTint by 1 %
  \SEPARATOR
  \LONGTYPEOUT{%
    TEST \the\gTESTint: \detokenize{#1}}%
  \SEPARATOR
  \begingroup
    \long\def\TYPE##1{##1}%
    \LONGTYPEOUT{#2}%
  \endgroup
  \SEPARATOR
  \LONGTYPEOUT{}%
}
\def \TRUE  {\TYPE{TRUE}}
\def \FALSE {\TYPE{FALSE}}
\def \YES   {\TYPE{YES}}
\def \NO    {\TYPE{NO}}
\def \NEWLINE {\TYPE{^^J}}
\ifx\pdfoutput\@undefined
  \ifx\outputmode\@undefined
  \else
    \ifnum\outputmode>0 %
      \pdfextension mapfile{pdftex.map}%
    \fi
  \fi
\else
  \ifnum\pdfoutput>0 %
    \pdfmapfile{pdftex.map}%
  \fi
\fi
\ifnum 0%
  \ifx\pdfoutput\@undefined\else\ifnum\pdfoutput>0 1\fi\fi
  \ifx\outputmode\@undefined\else\ifnum\outputmode>0 1\fi\fi
  >0 %
  \ifx\pdfvariable\@undefined
    \pdfinfo{/Producer (\ifx\directlua\@undefined pdf\else Lua\fi TeX)}
    \ifx\pdfinfoomitdate\@undefined\else
      \pdfinfoomitdate     = 1 %
      \pdfsuppressptexinfo = 1 %
      \pdftrailerid{}
    \fi
  \else
    \pdfextension info{/Producer (LuaTeX)}
    \pdfvariable suppressoptionalinfo \numexpr
          0
        +   1 % PTEX.Fullbanner
        +  32 % CreationDate
        +  64 % ModDate
      \relax
  \fi
\else
  \special{%
    pdf: docinfo
      <<
        /Creator        (TeX)
        /CreationDate   ()
        /ModDate        ()
        /Producer       (\ifx\XeTeXversion\@undefined\else x\fi dvipdfmx)
      >>
  }
\fi
\reset@catcodes
%% 
%%
%% End of file `regression-test.tex'.
| loads the
  necessary settings and commands to format the \texttt{.log} file
  properly for testing.

\item It is not necessary to load a special document class (most
  tests use \texttt{article} or \texttt{minimal}), but a package
  author may wish to adjust page margins, etc., without repeating the
  commands for each test. Such a special test class or package could
  then be kept in the \texttt{support/} directory.

\item The test begins proper at \cs{START}\Dash everything before that point
  in the \texttt{.log} file will be ignored. This prevents, for example,
  package version numbers displayed while the preamble is
  processed from becoming part of the test.
%
  The \cs{AUTHOR} declaration is an
  optional way of indicating who might know how to fix the problem
  should the test begin failing.

\item In this example \cs{showoutput} generates the actual test data
  by generating a symbolic representation of the page content in the
  \texttt{.log} file.

\item A slightly modified version of \verb|\end{document}| finishes
  the test document. Alternatively, one can end the test file with
  \cs{END} which avoids the final processing done by
  \verb|\end{document}| and thus prevents unwanted material from
  becoming part of the test data. In this example \cs{END} cannot be
  used as that would stop the run immediately without producing a page\Dash
  which is our goal here.

\end{enumerate}
Not shown is the \cs{OMIT} \dots \cs{TIMO} construction, which puts
flags into the \texttt{.log} file between which no test comparisons
will be made.  This can be used around code that generates variable
log information that is known to be irrelevant for the test. For
example, statements like \cs{newlength} or \cs{newcounter} write some
tracing information into the \texttt{.log} that shows the allocated
register number. If the code gets revised these numbers might change
and thereby unnecessarily invalidate the test result.

\cs{OMIT} can also be used before \verb|\end{document}| if you
need the final processing to happen, but want to ensure that nothing
written at that time becomes part of the test.


An example of a more structured test from the \LaTeX3 test suite is
shown in Figure~\ref{fig:nonexp}.  Here, a number of different tests
are contained within a single file, and a few of these are included in
the example.  The content of the test is not really important here (it is
testing aspects of the integer module from \texttt{expl3}) but it does
show a few best practices.

\cs{OMIT}/\cs{TIMO} is used to hide the register allocation numbers
from \verb|\int_new:N|. The first test then exercises integer addition
and subtraction which is not expandable (therefore \cs{TEST} together
with \cs{TYPE} is used) and it consists in fact of several small
tests. The expected results are written as comments into the test file
which is helpful in case it ever fails.

Converting integers is supposed to be expandable so \cs{TESTEXP} is
used for the second test. The same is true for the case selection
commands. Here the test output is generated by \cs{YES} or \cs{NO}.

Can you guess the test results, even if you are not familiar with the
\texttt{expl3} language? They are shown in Figure~\ref{fig:results}.


\begin{figure}
\begin{Verbatim}[frame=single,fontsize=\small]
\documentclass{minimal}
%%
%% This is file `regression-test.tex',
%% generated with the docstrip utility.
%%
%% The original source files were:
%%
%% l3build.dtx  (with options: `package')
%% 
%% Copyright (C) 2014-2016 The LaTeX3 Project
%% 
%% It may be distributed and/or modified under the conditions of
%% the LaTeX Project Public License (LPPL), either version 1.3c of
%% this license or (at your option) any later version.  The latest
%% version of this license is in the file:
%% 
%%    http://www.latex-project.org/lppl.txt
%% 
%% This file is part of the "l3build bundle" (The Work in LPPL)
%% and all files in that bundle must be distributed together.
%% 
\begingroup\expandafter\expandafter\expandafter\endgroup
\expandafter\ifx\csname eTeXversion\endcsname\relax
  \errmessage{e-TeX is required to use regression-test.tex}%
  \expandafter\endinput
\fi
\ifx\unprotect\undefined
  \expandafter\edef\csname reset\string @catcodes\endcsname{%
    \catcode`\noexpand\@=\the\catcode`\@\relax
  }%
  \catcode`\@=11 %
\else
  \unprotect
  \def\reset@catcodes{\protect}%
\fi
\ifnum\interactionmode>1 \scrollmode\fi
\errorcontextlines=-1 %
\showboxbreadth=\maxdimen
\showboxdepth=\maxdimen
\def\loggingoutput{%
  \tracingoutput=1 %
  \showboxbreadth=\maxdimen
  \showboxdepth=\maxdimen
}
\newlinechar=`\^^J
\long\def\LONGTYPEOUT#1{%
  \begingroup
    \long\def\TYPE##1{##1}%
    \immediate\write128{#1}%
  \endgroup
}
\let\TYPE\LONGTYPEOUT
\def\STARTMESSAGE{This is a generated file for the l3build validation system.}
\def\START{%
  \LONGTYPEOUT{^^JSTART-TEST-LOG^^J}%
  \LONGTYPEOUT{^^J%
     \STARTMESSAGE%
     ^^J^^JDon't change this file in any respect.%
     ^^J^^J%
  }%
}
\ifx\@@end\@undefined
  \let\@@@end\end
\else
  \let\@@@end\@@end
\fi
\def\END{%
  \ifnum\currentgrouplevel>0 %
    \LONGTYPEOUT{Bad grouping: \the\currentgrouplevel!}%
  \fi
  \ifnum\currentiflevel>2 %
    \LONGTYPEOUT{Bad conditionals: \the\numexpr\currentiflevel-2!}%
  \fi
  \LONGTYPEOUT{^^JEND-TEST-LOG^^J}%
  \@@@end
}
\ifx\@@end\@undefined
  \let\end\END
\else
  \let\@@end\END
\fi
\def\OMIT{\LONGTYPEOUT{OMIT}}
\def\TIMO{\LONGTYPEOUT{TIMO}}
\ifx\InputIfFileExists\@undefined
  \newread\@inputcheck
  \long\def\InputIfFileExists#1#2#3{%
    \openin\@inputcheck#1\relax
    \ifeof\@inputcheck
      \def\reserved@a{#3}%
    \else
      \def\reserved@a{#2\input #1\relax}%
    \fi
    \closein\@inputcheck
    \reserved@a
  }%
\fi
\InputIfFileExists{regression-test.cfg}
  {\LONGTYPEOUT{^^J***^^Jregression-test.cfg in operation^^J***^^J}}{}
\newcount\gTESTint
\def\SEPARATOR{%
  \TYPE{%
    ============================================================%
  }%
}
\long\def\TEST#1#2{%
  \global\advance\gTESTint by 1 %
  \SEPARATOR
  \LONGTYPEOUT{TEST \the\gTESTint: \detokenize{#1}}%
  \SEPARATOR
  \begingroup
    \let\TYPE\LONGTYPEOUT
    #2%
  \endgroup
  \SEPARATOR
  \LONGTYPEOUT{}%
}
\long\def\TESTEXP#1#2{%
  \global\advance\gTESTint by 1 %
  \SEPARATOR
  \LONGTYPEOUT{%
    TEST \the\gTESTint: \detokenize{#1}}%
  \SEPARATOR
  \begingroup
    \long\def\TYPE##1{##1}%
    \LONGTYPEOUT{#2}%
  \endgroup
  \SEPARATOR
  \LONGTYPEOUT{}%
}
\def \TRUE  {\TYPE{TRUE}}
\def \FALSE {\TYPE{FALSE}}
\def \YES   {\TYPE{YES}}
\def \NO    {\TYPE{NO}}
\def \NEWLINE {\TYPE{^^J}}
\ifx\pdfoutput\@undefined
  \ifx\outputmode\@undefined
  \else
    \ifnum\outputmode>0 %
      \pdfextension mapfile{pdftex.map}%
    \fi
  \fi
\else
  \ifnum\pdfoutput>0 %
    \pdfmapfile{pdftex.map}%
  \fi
\fi
\ifnum 0%
  \ifx\pdfoutput\@undefined\else\ifnum\pdfoutput>0 1\fi\fi
  \ifx\outputmode\@undefined\else\ifnum\outputmode>0 1\fi\fi
  >0 %
  \ifx\pdfvariable\@undefined
    \pdfinfo{/Producer (\ifx\directlua\@undefined pdf\else Lua\fi TeX)}
    \ifx\pdfinfoomitdate\@undefined\else
      \pdfinfoomitdate     = 1 %
      \pdfsuppressptexinfo = 1 %
      \pdftrailerid{}
    \fi
  \else
    \pdfextension info{/Producer (LuaTeX)}
    \pdfvariable suppressoptionalinfo \numexpr
          0
        +   1 % PTEX.Fullbanner
        +  32 % CreationDate
        +  64 % ModDate
      \relax
  \fi
\else
  \special{%
    pdf: docinfo
      <<
        /Creator        (TeX)
        /CreationDate   ()
        /ModDate        ()
        /Producer       (\ifx\XeTeXversion\@undefined\else x\fi dvipdfmx)
      >>
  }
\fi
\reset@catcodes
%% 
%%
%% End of file `regression-test.tex'.

\RequirePackage{expl3}
\begin{document}
\START
\AUTHOR{Frank Mittelbach, LaTeX3 Project}
\ExplSyntaxOn

\OMIT
\int_new:N \l_testa_int
\int_new:N \g_testa_int
\TIMO

\TEST { adding~and~subtracting }
  {
    \int_zero:N \l_testa_int
    \int_add:Nn \l_testa_int { 5 * 7 }
    \int_add:Nn \l_testa_int { 15 }
 % we hope for a value of 50
    \TYPE { \int_use:N \l_testa_int }
    \int_sub:Nn \l_testa_int { 3 * 5 }
 % we hope for a value of 35
    \TYPE { \int_use:N \l_testa_int }
    \int_gzero:N \g_testa_int
    {
      \int_gadd:Nn \g_testa_int
                  { (2 + 13) / (2 * 3) }
      \int_gadd:Nn \g_testa_int { 3 }
 % we hope for a value of 6
      \TYPE { \int_use:N \g_testa_int }
      \int_gsub:Nn \g_testa_int { 5 * 5 }
    }
 % we hope for a value of -19
    \TYPE { \int_use:N \g_testa_int }
  }

\TESTEXP { converting~from~and~to~base }
  {
    \int_to_base:nn { 17 } { 8 }  ~
    \int_from_base:nn { 21 } { 8 }
  }
\TESTEXP{ Case~statements }
  {
    \int_case:nnn
      { -1 + 1 }
      { {    -1 } {  \NO }
        { 3 - 3 } { \YES } }
    { \NO }
  \NEWLINE
  \int_case:nnn
    { 7 - 2 }
    { { -1 + 3 } { \NO } }
    { \YES }
  }
% more tests here omitted
\END
\end{Verbatim}
\caption{Expandable and non-expandable tests}
\label{fig:nonexp}
\end{figure}


\begin{figure}
\begin{Verbatim}[frame=single,fontsize=\small]
=============================================
TEST 1: adding and subtracting
=============================================
50
35
6
-19
=============================================

=============================================
TEST 2: converting from and to base
=============================================
21 17
=============================================

=============================================
TEST 3: Case statements
=============================================
YES
YES
=============================================
\end{Verbatim}
\caption{Test results}
\label{fig:results}
\end{figure}

\subsection{Options}

While the examples shown previously demonstrate the behaviour in the standard
setup, the new build system provides significantly greater
flexibility. This is achieved by providing a large number of
variables that can be (re)set as necessary in the driver file.
%
For example, the new system supports building complex distributions
consisting of
several modules in different directories with dependencies between
them. You can also control if the processing should happen in a
sandbox or if it is allowed to draw any support files needed for the
tests (e.g., extra packages or classes) from the \acro{TDS} tree.
The latter is the default as this is better for most distributions.
%
For details consult the documentation in~\cite{L3Project:l3build}.

There is one option that one may have to modify even for simple setups:
\texttt{checkruns}. This controls the number of times each test file
is run; to speed up processing it defaults to~\texttt{1}. If,
however, the codes require multiple runs to function (e.g., if you
test material that is passed through the \texttt{.aux} file) you have
to set this variable to \texttt{2} or higher to ensure that your tests
actually work correctly.




\section{Operating the system}

As indicated earlier the system does a bit more than managing a set
of test files, so here is a short description of the main tasks that
can be executed once the setup is in place. Each task takes zero or
more arguments as described below and is executed by running the
driver file (default \texttt{build.lua}) through a Lua interpreter
(\texttt{texlua}) and passing it the task name and any further
argument as necessary, e.g.,
\begin{quote}\ttfamily
\execname~check~\meta{test name}
\end{quote}
would run the check on \meta{test name} using all engines.
%\pagebreak[3]
So here is the list of available tasks:
\begin{description}
\item[\texttt{check} \meta{name} \meta{engine}] \hfil \parskip=0pt Without
  arguments, runs all test files found in the directory that contains
  the \texttt{.lvt} files. It reports progress by displaying each test
  file name currently processed (but otherwise hides any \TeX{} output
  to avoid cluttering the screen) and at the end displays a summary
  indicating success or failure.

  If \meta{name} is specified, it will run only the tests for that
  \texttt{.lvt} file, and if additionally given an \meta{engine} name will
  run only the test for that specific engine. In either case it will
  show everything on the screen, which is helpful if the run shows
  abnormal behaviour (especially if it ends up in an endless loop and
  never returns for some reason).

\item[\texttt{clean} ] Cleans up the source tree, removing
  temporary files and directories.

\item[\texttt{ctan} ] Runs all tests, typesets all
  documentation and if there are no errors, generates a
  \texttt{.zip} file suitable for uploading to \CTAN.

\item[\texttt{doc} ] Typesets all documentation (by default
  \texttt{.dtx} files), thus checking them for trivial processing errors.

\item[\texttt{install} ] This installs the distribution in the local
  tree of the user.

\item[\texttt{save} \meta{name} \meta{engine}] This generates (or
  regenerates) the \texttt{.tlg} file for \meta{name}. If additionally
  supplied an \meta{engine} argument it generates a specific
  \texttt{.tlg} as discussed above. 

  It is the responsibility of the developer to verify that the data
  placed into the \texttt{.tlg} produces the desired result, i.e., is
  actually correct. Once produced or updated with \texttt{save}, the
  output is considered certified and will be used to verify future check
  runs!

%\item[\texttt{unpack} ]
\end{description}



\section{Acknowledgements}

The original test suite system was a joint effort by the whole
\LaTeX{} project team at that time, i.e.,
%
Frank Mittelbach,
Rainer Sch\"opf,
David Carlisle,
Michael Downes,
Alan Jeffrey, and
Chris Rowley.
%
We also had significant help when writing the initial set of test files from a
number of volunteers, in particular Daniel Flipo and Chris Martin.

Around 2008 Rainer replaced the Makefile approach used for \LaTeXe{}
by Cons (a Perl-based solution) as the Makefile got so complex over
time that it was difficult to manage.

For the \LaTeX3 development we stayed with Make as the requirements
of the \texttt{expl3} distribution were initially much simpler.

Joseph Wright wrote a first set of \texttt{.bat} files for
\texttt{expl3}, as by then many developers worked on Windows. Modelled
after this, Frank replaced the Cons solution for \LaTeXe{} in 2013.

Finally in 2014 Joseph then implemented most of the new Lua-based
system and it is now successfully used to manage the \LaTeX3
(\texttt{expl3}) distribution as well as several smaller package
distributions. The \LaTeXe{} distribution will follow shortly.

\nocite{*}

\iffalse  % easier to include the bbl for TUB
\bibliographystyle{plain}
\bibliography{\jobname}
\else
%\SetBibJustification{\small \advance\itemsep by .9pt } %\raggedright}
\SetBibJustification{\advance\itemsep by 2pt \raggedright}
\smallskip
\begin{thebibliography}{1}

\bibitem{Knuth:1984:TTT}
Donald~E. Knuth.
\newblock A torture test for {\TeX}.
\newblock Report STAN-CS-84-1027, 1984.

\bibitem{Mittelbach:TB18-4-309}
Frank Mittelbach.
\newblock {{A} regression test suite for {\LaTeXe}}.
\newblock {\em TUGboat}, 18(4):309--311, December 1997.
\newblock \url{http://tug.org/TUGboat/tb18-4/tb57mitt.pdf}

\bibitem{Mittelbach:l3buildtalk}
Frank Mittelbach.
\newblock A modern regression test suite for \TeX\ programming, July 2014.
\newblock Talk given at \tug\ conference in Portland. Video and slide material
  available at \url{http://www.latex-project.org/papers}.

\bibitem{L3Project:l3build}
{\LaTeX3}~Project.
\newblock The \textsf{l3build} package: Checking and building packages,
September 2014.\\
\newblock \url{http://ctan.org/pkg/l3build}

\end{thebibliography}
\fi

%\bigskip
\makesignature

\end{document}

