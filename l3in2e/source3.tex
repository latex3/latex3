% \iffalse
%% File: source3.dtx Copyright (C) 1990-2009 LaTeX3 project
%%
%% It may be distributed and/or modified under the conditions of the
%% LaTeX Project Public License (LPPL), either version 1.3c of this
%% license or (at your option) any later version.  The latest version
%% of this license is in the file
%%
%%    http://www.latex-project.org/lppl.txt
%%
%% This file is part of the ``expl3 bundle'' (The Work in LPPL)
%% and all files in that bundle must be distributed together.
%%
%% The released version of this bundle is available from CTAN.
%%
%% -----------------------------------------------------------------------
%%
%% The development version of the bundle can be found at
%%
%%    http://www.latex-project.org/cgi-bin/cvsweb.cgi/
%%
%% for those people who are interested.
%%
%%%%%%%%%%%
%% NOTE: %%
%%%%%%%%%%%
%%
%%   Snapshots taken from the repository represent work in progress and may
%%   not work or may contain conflicting material!  We therefore ask
%%   people _not_ to put them into distributions, archives, etc. without
%%   prior consultation with the LaTeX Project Team.
%%
%% -----------------------------------------------------------------------
%% \fi

% This document will typeset the LaTeX3 sources as a single document.
% This will produce quite a large file (roughly ??? pages) and may
% take a long time on a slow machine.

\documentclass{l3doc}
\listfiles

\begin{document}

\title{The \LaTeX3 Sources}
\author{\Team}


\pagenumbering{roman}
\maketitle

\begin{abstract}
\noindent This is the reference documentation for the \pkg{expl3} programming
environment. The \pkg{expl3} modules set up an experimental
naming scheme for \LaTeX\ commands, which allow the \LaTeX\ programmer to
systematically name functions and variables, and specify the argument types of
functions.

\parindent=0pt
\parskip=\baselineskip

The \TeX\ and \eTeX\ primitives are all given a new name according to these
conventions.

At present, the \pkg{expl3} modules are designed to be loaded on top of \LaTeXe. In time, a \LaTeX3 format will be produced based on this code.

\begin{bfseries}
Warning: This package, and all packages using it should be regarded as
\emph{experimental}!

The names of these packages, and the names and syntax of any commands
defined in them might change at any time.

These conventions are being distributed in this form to encourage
discussion and experimentation. It is \emph{not} intended that
these packages be used in `real' documents at this stage.
\end{bfseries}
\end{abstract}

\clearpage

{\def\\{:}% fix "newlines" in the ToC
\tableofcontents}

\clearpage
\pagenumbering{arabic}

%%%%%%%%%%%%%%%%%%%%%%%%%%%%%%%%%%%%%%%%%%%%%%%%%%%%%%%%%%%%

% Each of the following \DocInput lines includes a file with extension
% .dtx. Each of these files may be typeset separately. For instance
%   pdflatex l3box.dtx
% will typeset the source of the LaTeX3 box commands. If you use the
% Makefile, the index will be generated automatically; e.g.,
%   make doc F=l3box
%
% If this file is processed, each of these separate dtx files will be
% contained as a part of a single document.

\makeatletter
\def\partname{Part}
\def\maketitle{\part{\@title}}
\let\thanks\@gobble
\let\DelayPrintIndex\PrintIndex
\let\PrintIndex\@empty
\makeatother

\DisableImplementation

\DocInput{l3names.dtx}

\DocInput{l3basics.dtx}
\DocInput{l3expan.dtx}
\DocInput{l3prg.dtx}
\DocInput{l3quark.dtx}
\DocInput{l3token.dtx}

\DocInput{l3int.dtx}
\DocInput{l3num.dtx}
\DocInput{l3intexpr.dtx}
\DocInput{l3skip.dtx}

\DocInput{l3tl.dtx}
\DocInput{l3toks.dtx}
\DocInput{l3seq.dtx}
\DocInput{l3clist.dtx}
\DocInput{l3prop.dtx}

\DocInput{l3io.dtx}
\DocInput{l3msg.dtx}
\DocInput{l3box.dtx}
\DocInput{l3xref.dtx}
\DocInput{l3keyval.dtx}
\DocInput{l3calc.dtx}
\DocInput{l3file.dtx}

% \DocInput{l3precom.dtx}
% \DocInput{l3alloc.dtx}
% \DocInput{l3chk.dtx}

\part{Implementation}
\def\maketitle{}
\EnableImplementation
\DisableDocumentation
\DocInputAgain

%% \DocInput{l3vers.dtx}   % Current version date

\includeltpatch       % Corrections distributed after the full release

\clearpage
\pagestyle{headings}

% Make TeX shut up.
\hbadness=10000
\newcount\hbadness
\hfuzz=\maxdimen

\PrintChanges
\clearpage

\begingroup
\def\endash{--}
\catcode`\-\active
\def-{\futurelet\temp\indexdash}
\def\indexdash{\ifx\temp-\endash\fi}

\DelayPrintIndex
\endgroup

\end{document}


