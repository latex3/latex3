% \iffalse meta-comment
%
%% File: l3tl.dtx Copyright (C) 1990-2012 The LaTeX3 Project
%%
%% It may be distributed and/or modified under the conditions of the
%% LaTeX Project Public License (LPPL), either version 1.3c of this
%% license or (at your option) any later version.  The latest version
%% of this license is in the file
%%
%%    http://www.latex-project.org/lppl.txt
%%
%% This file is part of the "l3kernel bundle" (The Work in LPPL)
%% and all files in that bundle must be distributed together.
%%
%% The released version of this bundle is available from CTAN.
%%
%% -----------------------------------------------------------------------
%%
%% The development version of the bundle can be found at
%%
%%    http://www.latex-project.org/svnroot/experimental/trunk/
%%
%% for those people who are interested.
%%
%%%%%%%%%%%
%% NOTE: %%
%%%%%%%%%%%
%%
%%   Snapshots taken from the repository represent work in progress and may
%%   not work or may contain conflicting material!  We therefore ask
%%   people _not_ to put them into distributions, archives, etc. without
%%   prior consultation with the LaTeX3 Project.
%%
%% -----------------------------------------------------------------------
%
%<*driver|package>
\RequirePackage{l3names}
\GetIdInfo$Id$
  {L3 Experimental token lists}
%</driver|package>
%<*driver>
\documentclass[full]{l3doc}
\begin{document}
  \DocInput{\jobname.dtx}
\end{document}
%</driver>
% \fi
%
% \title{^^A
%   The \pkg{l3tl} package\\ Token lists^^A
%   \thanks{This file describes v\ExplFileVersion,
%      last revised \ExplFileDate.}^^A
% }
%
% \author{^^A
%  The \LaTeX3 Project\thanks
%    {^^A
%      E-mail:
%        \href{mailto:latex-team@latex-project.org}
%          {latex-team@latex-project.org}^^A
%    }^^A
% }
%
% \date{Released \ExplFileDate}
%
% \maketitle
%
% \begin{documentation}
%
% \TeX{} works with tokens, and \LaTeX3 therefore provides a number of
% functions to deal with token lists. Token lists may be present directly in
% the argument to a function:
% \begin{verbatim}
%   \foo:n { a collection of \tokens }
% \end{verbatim}
% or may be stored for processing in a so-called \enquote{token list
% variable}, which have the suffix \texttt{tl}:
% the argument to a function:
% \begin{verbatim}
%   \foo:N \l_some_tl
% \end{verbatim}
% In both cases, functions are available to test an manipulate the lists
% of tokens, and these have the module prefix \texttt{tl}.
% In many cases, function which can be applied to token list variables
% are paired with similar functions for application to explicit lists
% of tokens: the two \enquote{views} of a token list are therefore collected
% together here.
%
% A token list can be seen either as a list of \enquote{items},
% or a list of \enquote{tokens}. An item is whatever \cs{use_none:n}
% grabs as its argument: either a single token or a brace group,
% with optional leading explicit space characters (each item is thus
% itself a token list). A token is either a normal \texttt{N} argument,
% or | |, |{|, or |}| (assuming normal \TeX{} category codes).
% Thus for example
% \begin{verbatim}
%   { Hello } ~ world
% \end{verbatim}
% contains six items (\texttt{Hello}, \texttt{w}, \texttt{o}, \texttt{r},
% \texttt{l} and \texttt{d}), but thirteen tokens (|{|, \texttt{H}, \texttt{e},
% \texttt{l}, \texttt{l}, \texttt{o}, |}|, \verb*| |, \texttt{w}, \texttt{o},
% \texttt{r}, \texttt{l} and \texttt{d}).
% Functions which act on items are often faster than their analogue acting
% directly on tokens.
%
% \section{Creating and initialising token list variables}
%
% \begin{function}{\tl_new:N, \tl_new:c}
%   \begin{syntax}
%     \cs{tl_new:N} \meta{tl~var}
%   \end{syntax}
%   Creates a new \meta{tl~var} or raises an error if the
%   name is already taken. The declaration is global. The
%   \meta{tl~var} will initially be empty.
% \end{function}
%
% \begin{function}{\tl_const:Nn, \tl_const:Nx, \tl_const:cn, \tl_const:cx}
%   \begin{syntax}
%     \cs{tl_const:Nn} \meta{tl~var} \Arg{token list}
%   \end{syntax}
%   Creates a new constant \meta{tl~var} or raises an error
%   if the name is already taken. The value of the
%   \meta{tl~var} will be set globally to the
%   \meta{token list}.
% \end{function}
%
% \begin{function}{\tl_clear:N, \tl_clear:c, \tl_gclear:N, \tl_gclear:c}
%   \begin{syntax}
%     \cs{tl_clear:N} \meta{tl~var}
%   \end{syntax}
%   Clears all entries from the \meta{tl~var} within the scope of
%   the current \TeX{} group.
% \end{function}
%
% \begin{function}
%   {\tl_clear_new:N, \tl_clear_new:c, \tl_gclear_new:N, \tl_gclear_new:c}
%   \begin{syntax}
%     \cs{tl_clear_new:N} \meta{tl~var}
%   \end{syntax}
%   Ensures that the \meta{tl~var} exists globally by applying
%   \cs{tl_new:N} if necessary, then applies \cs{tl_(g)clear:N} to leave
%   the \meta{tl~var} empty.
% \end{function}
%
% \begin{function}
%   {
%     \tl_set_eq:NN,  \tl_set_eq:cN,  \tl_set_eq:Nc,  \tl_set_eq:cc,
%     \tl_gset_eq:NN, \tl_gset_eq:cN, \tl_gset_eq:Nc, \tl_gset_eq:cc
%   }
%   \begin{syntax}
%     \cs{tl_set_eq:NN} \meta{tl~var1} \meta{tl~var2}
%   \end{syntax}
%   Sets the content of \meta{tl~var1} equal to that of
%   \meta{tl~var2}.
% \end{function}
%
% \begin{function}[added = 2012-05-18]
%   {
%     \tl_concat:NNN,  \tl_concat:ccc,
%     \tl_gconcat:NNN, \tl_gconcat:ccc
%   }
%   \begin{syntax}
%     \cs{tl_concat:NNN} \meta{tl~var1} \meta{tl~var2} \meta{tl~var3}
%   \end{syntax}
%   Concatenates the content of \meta{tl~var2} and \meta{tl~var3}
%   together and saves the result in \meta{tl~var1}. The \meta{tl~var2}
%   will be placed at the left side of the new token list.
% \end{function}
%
% \begin{function}[EXP, pTF, added=2012-03-03]{\tl_if_exist:N, \tl_if_exist:c}
%   \begin{syntax}
%     \cs{tl_if_exist_p:N} \meta{tl~var}
%     \cs{tl_if_exist:NTF} \meta{tl~var} \Arg{true code} \Arg{false code}
%   \end{syntax}
%   Tests whether the \meta{tl~var} is currently defined.  This does not
%   check that the \meta{tl~var} really is a token list variable.
% \end{function}
%
% \section{Adding data to token list variables}
%
% \begin{function}
%   {
%     \tl_set:Nn, \tl_set:NV, \tl_set:Nv, \tl_set:No, \tl_set:Nf, \tl_set:Nx,
%     \tl_set:cn, \tl_set:NV, \tl_set:Nv, \tl_set:co, \tl_set:cf, \tl_set:cx,
%     \tl_gset:Nn, \tl_gset:NV, \tl_gset:Nv,
%     \tl_gset:No, \tl_gset:Nf, \tl_gset:Nx,
%     \tl_gset:cn, \tl_gset:cV, \tl_gset:cv,
%     \tl_gset:co, \tl_gset:cf, \tl_gset:cx
%   }
%   \begin{syntax}
%     \cs{tl_set:Nn} \meta{tl~var} \Arg{tokens}
%   \end{syntax}
%   Sets \meta{tl~var} to contain \meta{tokens},
%   removing any previous content from the variable.
% \end{function}
%
% \begin{function}
%   {
%     \tl_put_left:Nn,  \tl_put_left:NV,  \tl_put_left:No,  \tl_put_left:Nx,
%     \tl_put_left:cn,  \tl_put_left:cV,  \tl_put_left:co,  \tl_put_left:cx,
%     \tl_gput_left:Nn, \tl_gput_left:NV, \tl_gput_left:No, \tl_gput_left:Nx,
%     \tl_gput_left:cn, \tl_gput_left:cV, \tl_gput_left:co, \tl_gput_left:cx
%   }
%   \begin{syntax}
%     \cs{tl_put_left:Nn} \meta{tl~var} \Arg{tokens}
%   \end{syntax}
%   Appends \meta{tokens} to the left side of the current content of
%   \meta{tl~var}.
% \end{function}
%
% \begin{function}
%   {
%     \tl_put_right:Nn, \tl_put_right:NV, \tl_put_right:No, \tl_put_right:Nx,
%     \tl_put_right:cn, \tl_put_right:cV, \tl_put_right:co, \tl_put_right:cx,
%     \tl_gput_right:Nn, \tl_gput_right:NV, \tl_gput_right:No,
%     \tl_gput_right:Nx,
%     \tl_gput_right:cn, \tl_gput_right:cV, \tl_gput_right:co,
%     \tl_gput_right:cx
%   }
%   \begin{syntax}
%     \cs{tl_put_right:Nn} \meta{tl~var} \Arg{tokens}
%   \end{syntax}
%   Appends \meta{tokens} to the right side of the current content of
%   \meta{tl~var}.
% \end{function}
%
% \section{Modifying token list variables}
%
% \begin{function}[updated = 2011-08-11]
%   {
%     \tl_replace_once:Nnn,  \tl_replace_once:cnn,
%     \tl_greplace_once:Nnn, \tl_greplace_once:cnn
%   }
%   \begin{syntax}
%     \cs{tl_replace_once:Nnn} \meta{tl~var} \Arg{old tokens} \Arg{new tokens}
%   \end{syntax}
%   Replaces the first (leftmost) occurrence of \meta{old tokens} in the
%   \meta{tl~var} with \meta{new tokens}. \meta{Old tokens}
%   cannot contain |{|, |}| or |#|
%   (assuming normal \TeX{} category codes).
% \end{function}
%
% \begin{function}[updated = 2011-08-11]
%   {
%     \tl_replace_all:Nnn, \tl_replace_all:cnn,
%     \tl_greplace_all:Nnn, \tl_greplace_all:cnn
%   }
%   \begin{syntax}
%     \cs{tl_replace_all:Nnn} \meta{tl~var} \Arg{old tokens} \Arg{new tokens}
%   \end{syntax}
%   Replaces all occurrences of \meta{old tokens} in the
%   \meta{tl~var} with \meta{new tokens}. \meta{Old tokens}
%   cannot contain |{|, |}| or |#|
%   (assuming normal \TeX{} category codes). As this function
%   operates from left to right, the pattern \meta{old tokens}
%   may remain after the replacement (see \cs{tl_remove_all:Nn}
%   for an example). The assignment is
%   restricted to the current \TeX{} group.
% \end{function}
%
% \begin{function}[updated = 2011-08-11]
%   {
%     \tl_remove_once:Nn,  \tl_remove_once:cn,
%     \tl_gremove_once:Nn, \tl_gremove_once:cn
%   }
%   \begin{syntax}
%     \cs{tl_remove_once:Nn} \meta{tl~var} \Arg{tokens}
%   \end{syntax}
%   Removes the first (leftmost) occurrence of \meta{tokens} from the
%   \meta{tl~var}. \meta{Tokens} cannot contain |{|, |}| or
%   |#| (assuming normal \TeX{} category codes).
% \end{function}
%
% \begin{function}[updated = 2011-08-11]
%   {
%     \tl_remove_all:Nn,  \tl_remove_all:cn,
%     \tl_gremove_all:Nn, \tl_gremove_all:cn
%   }
%   \begin{syntax}
%     \cs{tl_remove_all:Nn} \meta{tl~var} \Arg{tokens}
%   \end{syntax}
%   Removes all occurrences of \meta{tokens} from the
%   \meta{tl~var}. \meta{Tokens} cannot contain |{|, |}| or
%   |#| (assuming normal \TeX{} category codes). As this function
%   operates from left to right, the pattern \meta{tokens}
%   may remain after the removal, for instance,
%   \begin{quote}
%     \cs{tl_set:Nn} \cs{l_tmpa_tl} |{abbccd}|
%     \cs{tl_remove_all:Nn} \cs{l_tmpa_tl} |{bc}|
%   \end{quote}
%   will result in \cs{l_tmpa_tl} containing \texttt{abcd}.
% \end{function}
%
% \section{Reassigning token list category codes}
%
% \begin{function}[updated = 2011-12-18]
%   {
%     \tl_set_rescan:Nnn,  \tl_set_rescan:Nno,  \tl_set_rescan:Nnx,
%     \tl_set_rescan:cnn,  \tl_set_rescan:cno,  \tl_set_rescan:cnx,
%     \tl_gset_rescan:Nnn, \tl_gset_rescan:Nno, \tl_gset_rescan:Nnx,
%     \tl_gset_rescan:cnn, \tl_gset_rescan:cno, \tl_gset_rescan:cnx
%   }
%   \begin{syntax}
%     \cs{tl_set_rescan:Nnn} \meta{tl~var} \Arg{setup} \Arg{tokens}
%   \end{syntax}
%   Sets \meta{tl~var} to contain \meta{tokens},
%   applying the category code r{\'e}gime specified in the
%   \meta{setup} before carrying out the assignment. This allows the
%   \meta{tl~var} to contain material with category codes
%   other than those that apply when \meta{tokens} are absorbed. See also
%   \cs{tl_rescan:nn}.
% \end{function}
%
% \begin{function}[updated = 2011-12-18]{\tl_rescan:nn}
%   \begin{syntax}
%     \cs{tl_rescan:nn} \Arg{setup} \Arg{tokens}
%   \end{syntax}
%   Rescans \meta{tokens} applying the category code r\'egime specified
%   in the \meta{setup}, and leaves the resulting tokens in the input
%   stream. See also \cs{tl_set_rescan:Nnn}.
% \end{function}
%
% \section{Reassigning token list character codes}
%
% \begin{function}{\tl_to_lowercase:n}
%   \begin{syntax}
%     \cs{tl_to_lowercase:n} \Arg{tokens}
%   \end{syntax}
%   Works through all of the \meta{tokens}, replacing each character
%   with the lower case equivalent as defined by \cs{char_set_lccode:nn}.
%   Characters with no defined lower case character code are left
%   unchanged. This process does not alter the category code assigned
%   to the \meta{tokens}.
%   \begin{texnote}
%     This is the \TeX{} primitive \tn{lowercase} renamed.
%     As a result, this function takes place on execution and
%     not on expansion.
%   \end{texnote}
% \end{function}
%
% \begin{function}{\tl_to_uppercase:n}
%   \begin{syntax}
%     \cs{tl_to_uppercase:n} \Arg{tokens}
%   \end{syntax}
%   Works through all of the \meta{tokens}, replacing each character
%   with the upper case equivalent as defined by \cs{char_set_uccode:nn}.
%   Characters with no defined lower case character code are left
%   unchanged. This process does not alter the category code assigned
%   to the \meta{tokens}.
%   \begin{texnote}
%     This is the \TeX{} primitive \tn{uppercase} renamed.
%     As a result, this function takes place on execution and
%     not on expansion.
%   \end{texnote}
% \end{function}
%
% \section{Token list conditionals}
%
% \begin{function}[EXP,pTF]{\tl_if_blank:n, \tl_if_blank:V, \tl_if_blank:o}
%   \begin{syntax}
%     \cs{tl_if_blank_p:n} \Arg{token list}
%     \cs{tl_if_blank:nTF} \Arg{token list} \Arg{true code} \Arg{false code}
%   \end{syntax}
%   Tests if the \meta{token list} consists only of blank spaces
%   (\emph{i.e.}~contains no item). The test is \texttt{true} if
%   \meta{token list} is zero or more explicit tokens of character code~$32$
%   and category code~$10$, and is \texttt{false} otherwise.
% \end{function}
%
% \begin{function}[EXP,pTF]{\tl_if_empty:N, \tl_if_empty:c}
%   \begin{syntax}
%     \cs{tl_if_empty_p:N} \meta{tl var}
%     \cs{tl_if_empty:NTF} \meta{tl var} \Arg{true code} \Arg{false code}
%   \end{syntax}
%   Tests if the \meta{token list variable} is entirely empty
%   (\emph{i.e.}~contains no tokens at all).
% \end{function}
%
% \begin{function}[EXP,pTF]
%   {\tl_if_empty:n, \tl_if_empty:V, \tl_if_empty:o, \tl_if_empty:x}
%   \begin{syntax}
%     \cs{tl_if_empty_p:n} \Arg{token list}
%     \cs{tl_if_empty:nTF} \Arg{token list} \Arg{true code} \Arg{false code}
%   \end{syntax}
%   Tests if the \meta{token list} is entirely empty
%   (\emph{i.e.}~contains no tokens at all).
%   All versions of these functions are fully expandable
%   (including those involving an \texttt{x}-type expansion).
% \end{function}
%
% \begin{function}[EXP,pTF]
%   {\tl_if_eq:NN, \tl_if_eq:Nc, \tl_if_eq:cN, \tl_if_eq:cc}
%   \begin{syntax}
%     \cs{tl_if_eq_p:NN} \Arg{tl var1} \Arg{tl var2}
%     \cs{tl_if_eq:NNTF} \Arg{tl var1} \Arg{tl var2} \Arg{true code} \Arg{false code}
%   \end{syntax}
%   Compares the content of two \meta{token list variables} and
%   is logically \texttt{true} if the two contain the same list of
%   tokens (\emph{i.e.}~identical in both the list of characters they
%   contain and the category codes of those characters). Thus for example
%   \begin{verbatim}
%     \tl_set:Nn \l_tmpa_tl { abc }
%     \tl_set:Nx \l_tmpb_tl { \tl_to_str:n { abc } }
%     \tl_if_eq_p:NN \l_tmpa_tl \l_tmpb_tl
%   \end{verbatim}
%   is logically \texttt{false}.
% \end{function}
%
% \begin{function}[TF]{\tl_if_eq:nn}
%   \begin{syntax}
%     \cs{tl_if_eq:nnTF} \meta{token list1} \Arg{token list2} \Arg{true code} \Arg{false code}
%   \end{syntax}
%   Tests if \meta{token list1} and \meta{token list2} are equal, both in
%   respect of character codes and category codes.
% \end{function}
%
% \begin{function}[TF]{\tl_if_in:Nn, \tl_if_in:cn}
%   \begin{syntax}
%     \cs{tl_if_in:NnTF} \meta{tl~var} \Arg{token list} \Arg{true code} \Arg{false code}
%   \end{syntax}
%   Tests if the \meta{token list} is found in the content of the
%   \meta{token list variable}. The \meta{token list} cannot contain
%   the tokens |{|, |}| or |#| (assuming the usual \TeX{} category
%   codes apply).
% \end{function}
%
% \begin{function}[TF]
%   {\tl_if_in:nn, \tl_if_in:Vn, \tl_if_in:on, \tl_if_in:no}
%   \begin{syntax}
%     \cs{tl_if_in:nnTF} \Arg{token list1} \Arg{token list2} \Arg{true code} \Arg{false code}
%   \end{syntax}
%   Tests if <token list2> is found inside <token list1>.
%   The \meta{token list} cannot contain the tokens |{|, |}| or |#|
%   (assuming the usual \TeX{} category codes apply).
% \end{function}
%
% \begin{function}[updated = 2011-08-13, EXP,pTF]
%   {\tl_if_single:N, \tl_if_single:c}
%   \begin{syntax}
%     \cs{tl_if_single_p:N} \Arg{tl~var}
%     \cs{tl_if_single:NTF} \Arg{tl~var} \Arg{true code} \Arg{false code}
%   \end{syntax}
%   Tests if the content of the \meta{tl~var} consists of a single item,
%   \emph{i.e.}~is either a single normal token (excluding spaces,
%   and brace tokens) or a single brace group, surrounded by optional
%   spaces on both sides. In other words, such a token list has token count
%   $1$ according to \cs{tl_count:N}.
% \end{function}
%
% \begin{function}[updated = 2011-08-13, EXP,pTF]{\tl_if_single:n}
%   \begin{syntax}
%     \cs{tl_if_single_p:n} \Arg{token list}
%     \cs{tl_if_single:nTF} \Arg{token list} \Arg{true code} \Arg{false code}
%   \end{syntax}
%   Tests if the token list has exactly one item, \emph{i.e.}~is either
%   a single normal token or a single brace group, surrounded by
%   optional spaces on both sides. In other words, such a token list
%   has token count $1$ according to \cs{tl_count:n}.
% \end{function}
%
% \begin{function}[added = 2011-08-11,EXP,pTF]{\tl_if_single_token:n}
%   \begin{syntax}
%   \cs{tl_if_single_token_p:n} \Arg{token list}
%   \cs{tl_if_single_token:nTF} \Arg{token list} \Arg{true code} \Arg{false code}
%   \end{syntax}
%   Tests if the token list consists of exactly one token, \emph{i.e.}~is
%   either a single space character or a single \enquote{normal} token.
%   Token groups (|{|\ldots|}|) are not single tokens.
% \end{function}
%
% \section{Mapping to token lists}
%
% \begin{function}[rEXP]{\tl_map_function:NN, \tl_map_function:cN}
%   \begin{syntax}
%     \cs{tl_map_function:NN} \meta{tl~var} \meta{function}
%   \end{syntax}
%   Applies \meta{function} to every \meta{item} in the \meta{tl~var}.
%   The \meta{function} will receive one argument for each iteration.
%   This may be a number of tokens if the \meta{item} was stored within
%   braces. Hence the \meta{function} should anticipate receiving
%   \texttt{n}-type arguments. See also \cs{tl_map_function:nN}.
% \end{function}
%
% \begin{function}[rEXP]{\tl_map_function:nN}
%   \begin{syntax}
%     \cs{tl_map_function:nN} \meta{token list} \meta{function}
%   \end{syntax}
%   Applies \meta{function} to every \meta{item} in the \meta{token list},
%   The \meta{function} will receive one argument for each iteration.
%   This may be a number of tokens if the \meta{item} was stored within
%   braces. Hence the \meta{function} should anticipate receiving
%   \texttt{n}-type arguments. See also \cs{tl_map_function:NN}.
% \end{function}
%
% \begin{function}{\tl_map_inline:Nn, \tl_map_inline:cn}
%   \begin{syntax}
%     \cs{tl_map_inline:Nn} \meta{tl~var} \Arg{inline function}
%   \end{syntax}
%   Applies the \meta{inline function} to every \meta{item} stored within the
%   \meta{tl~var}. The \meta{inline function} should consist of code which
%   will receive the \meta{item} as |#1|. One in line mapping can be nested
%   inside another. See also \cs{tl_map_function:Nn}.
% \end{function}
%
% \begin{function}{\tl_map_inline:nn}
%   \begin{syntax}
%     \cs{tl_map_inline:nn} \meta{token list} \Arg{inline function}
%   \end{syntax}
%   Applies the \meta{inline function} to every \meta{item} stored within the
%   \meta{token list}. The \meta{inline function}  should consist of code which
%   will receive the \meta{item} as |#1|. One in line mapping can be nested
%   inside another. See also \cs{tl_map_function:nn}.
% \end{function}
%
% \begin{function}{\tl_map_variable:NNn, \tl_map_variable:cNn}
%   \begin{syntax}
%     \cs{tl_map_variable:NNn} \meta{tl~var} \meta{variable} \Arg{function}
%   \end{syntax}
%   Applies the \meta{function} to every \meta{item} stored
%   within the \meta{tl~var}. The \meta{function} should consist of code
%   which will receive the \meta{item} stored in the \meta{variable}.
%   One variable mapping can be nested inside another. See also
%   \cs{tl_map_inline:Nn}.
% \end{function}
%
% \begin{function}{\tl_map_variable:nNn}
%   \begin{syntax}
%     \cs{tl_map_variable:nNn} \meta{token list} \meta{variable} \Arg{function}
%   \end{syntax}
%   Applies the \meta{function} to every \meta{item} stored
%   within the \meta{token list}. The \meta{function} should consist of code
%   which will receive the \meta{item} stored in the \meta{variable}.
%   One variable mapping can be nested inside another. See also
%   \cs{tl_map_inline:nn}.
% \end{function}
%
% \begin{function}[rEXP]{\tl_map_break:}
%   \begin{syntax}
%     \cs{tl_map_break:}
%   \end{syntax}
%   Used to terminate a \cs{tl_map_\ldots} function before all
%   entries in the \meta{token list variable} have been processed. This
%   will normally take place within a conditional statement, for example
%   \begin{verbatim}
%     \tl_map_inline:Nn \l_my_tl
%       {
%         \str_if_eq:nnTF { #1 } { bingo }
%           { \tl_map_break: }
%           {
%             % Do something useful
%           }
%       }
%   \end{verbatim}
%   Use outside of a \cs{tl_map_\ldots} scenario will lead low
%   level \TeX{} errors.
% \end{function}
%
% \section{Using token lists}
%
% \begin{function}[EXP]{\tl_to_str:N, \tl_to_str:c}
%   \begin{syntax}
%     \cs{tl_to_str:N} \meta{tl~var}
%   \end{syntax}
%   Converts the content of the  \meta{tl~var} into a series of characters
%   with category code $12$ (other) with the exception of spaces, which
%   retain category code $10$ (space). This \meta{string} is then left
%   in the input stream.
% \end{function}
%
% \begin{function}[EXP]{\tl_to_str:n}
%   \begin{syntax}
%     \cs{tl_to_str:n} \Arg{tokens}
%   \end{syntax}
%   Converts the given \meta{tokens} into a series of characters with
%   category code $12$ (other) with the exception of spaces, which
%   retain category code $10$ (space). This \meta{string} is then left
%   in the input stream. Note that this function requires only a single
%   expansion.
%   \begin{texnote}
%     This is the \eTeX{} primitive \tn{detokenize}.
%   \end{texnote}
% \end{function}
%
% \begin{function}[EXP]{\tl_use:N, \tl_use:c}
%   \begin{syntax}
%     \cs{tl_use:N} \meta{tl~var}
%   \end{syntax}
%   Recovers the content of a \meta{tl~var} and places it
%   directly in the input stream. An error will be raised if the variable
%   does not exist or if it is invalid. Note that it is possible to use
%   a \meta{tl~var} directly without an accessor function.
% \end{function}
%
% \section{Working with the content of token lists}
%
% \begin{function}[added = 2012-05-13, EXP]
%   {\tl_count:n, \tl_count:V, \tl_count:o}
%   \begin{syntax}
%     \cs{tl_count:n} \Arg{tokens}
%   \end{syntax}
%   Counts the number of \meta{items} in \meta{tokens} and leaves this
%   information in the input stream. Unbraced tokens count as one
%   element as do each token group (|{|\ldots|}|). This process will
%   ignore any unprotected spaces within \meta{tokens}. See also
%   \cs{tl_count:N}. This function requires three expansions,
%   giving an \meta{integer denotation}.
% \end{function}
%
% \begin{function}[added = 2012-05-13, EXP]{\tl_count:N, \tl_count:c}
%   \begin{syntax}
%     \cs{tl_count:N} \Arg{tl~var}
%   \end{syntax}
%   Counts the number of token groups in the \meta{tl~var}
%   and leaves this information in the input stream. Unbraced tokens
%   count as one element as do each token group (|{|\ldots|}|). This
%   process will ignore any unprotected spaces within \meta{tokens}.
%   See also \cs{tl_count:n}. This function requires three expansions,
%   giving an \meta{integer denotation}.
% \end{function}
%
% \begin{function}[updated = 2012-01-08, EXP]
%   {\tl_reverse:n, \tl_reverse:V, \tl_reverse:o}
%   \begin{syntax}
%     \cs{tl_reverse:n} \Arg{token list}
%   \end{syntax}
%   Reverses the order of the \meta{items} in the \meta{token list},
%   so that \meta{item1}\meta{item2}\meta{item3} \ldots \meta{item$_n$}
%   becomes \meta{item$_n$}\ldots \meta{item3}\meta{item2}\meta{item1}.
%   This process will preserve unprotected space within the
%   \meta{token list}. Tokens are not reversed within braced token
%   groups, which keep their outer set of braces.
%   In situations where performance is important,
%   consider \cs{tl_reverse_items:n}.
%   See also \cs{tl_reverse:N}.
%   \begin{texnote}
%     The result is returned within the \tn{unexpanded}
%     primitive (\cs{exp_not:n}), which means that the token
%     list will not expand further when appearing in an x-type
%     argument expansion.
%   \end{texnote}
% \end{function}
%
% \begin{function}[updated = 2012-01-08]
%   {\tl_reverse:N, \tl_reverse:c, \tl_greverse:N, \tl_greverse:c}
%   \begin{syntax}
%     \cs{tl_reverse:N} \Arg{tl~var}
%   \end{syntax}
%   Reverses the order of the \meta{items} stored in \meta{tl~var}, so
%   that \meta{item1}\meta{item2}\meta{item3} \ldots \meta{item$_n$}
%   becomes \meta{item$_n$}\ldots \meta{item3}\meta{item2}\meta{item1}.
%   This process will preserve unprotected spaces within the
%   \meta{token list variable}. Braced token groups are copied without
%   reversing the order of tokens, but keep the outer set of braces.
%   See also \cs{tl_reverse:n}.
% \end{function}
%
% \begin{function}[added = 2012-01-08, EXP]{\tl_reverse_items:n}
%   \begin{syntax}
%     \cs{tl_reverse_items:n} \Arg{token list}
%   \end{syntax}
%   Reverses the order of the \meta{items} stored in \meta{tl~var},
%   so that \Arg{item_1}\Arg{item_2}\Arg{item_3} \ldots \Arg{item$_n$}
%   becomes \Arg{item$_n$} \ldots{} \Arg{item_3}\Arg{item_2}\Arg{item_1}.
%   This process will remove any unprotected space within the
%   \meta{token list}. Braced token groups are copied without
%   reversing the order of tokens, and keep the outer set of braces.
%   Items which are initially not braced are copied with braces in
%   the result. In cases where preserving spaces is important,
%   consider \cs{tl_reverse:n} or \cs{tl_reverse_tokens:n}.
%   \begin{texnote}
%     The result is returned within the \tn{unexpanded}
%     primitive (\cs{exp_not:n}), which means that the token
%     list will not expand further when appearing in an x-type
%     argument expansion.
%   \end{texnote}
% \end{function}
%
% \begin{function}[added = 2011-07-09, updated = 2011-08-13, EXP]
%   {\tl_trim_spaces:n}
%   \begin{syntax}
%     \cs{tl_trim_spaces:n} \meta{token list}
%   \end{syntax}
%   Removes any leading and trailing explicit space characters
%   from the \meta{token list} and leaves the result in the input
%   stream. This process requires two expansions.
%   \begin{texnote}
%     The result is returned within the \tn{unexpanded}
%     primitive (\cs{exp_not:n}), which means that the token
%     list will not expand further when appearing in an x-type
%     argument expansion.
%   \end{texnote}
% \end{function}
%
% \begin{function}[added = 2011-07-09]
%   {
%     \tl_trim_spaces:N,  \tl_trim_spaces:c,
%     \tl_gtrim_spaces:N, \tl_gtrim_spaces:c
%   }
%   \begin{syntax}
%     \cs{tl_trim_spaces:N} \meta{tl~var}
%   \end{syntax}
%   Removes any leading and trailing explicit space characters
%   from the content of the \meta{tl~var}.
% \end{function}
%
% \section{The first token from a token list}
%
% Functions which deal with either only the very first token of a
% token list or everything except the first token.
%
% \begin{function}[updated = 2012-02-08, EXP]
%   {\tl_head:N, \tl_head:n, \tl_head:V, \tl_head:v, \tl_head:f}
%   \begin{syntax}
%     \cs{tl_head:n} \Arg{tokens}
%   \end{syntax}
%   Leaves in the input stream the first non-space token from the
%   \meta{tokens}. Any leading space tokens will be discarded, and thus for
%   example
%   \begin{verbatim}
%     \tl_head:n { abc }
%   \end{verbatim}
%   and
%   \begin{verbatim}
%     \tl_head:n { ~ abc }
%   \end{verbatim}
%   will both leave |a| in the input stream.
%   An empty list of \meta{tokens} or one which consists
%   only of space (category code $10$) tokens will result in \cs{tl_head:n}
%   leaving nothing in the input stream.
%   \begin{texnote}
%     The result is returned within the \tn{unexpanded}
%     primitive (\cs{exp_not:n}), which means that the token
%     list will not expand further when appearing in an x-type
%     argument expansion.
%   \end{texnote}
% \end{function}
%
% \begin{function}[EXP]{\tl_head:w}
%   \begin{syntax}
%     \cs{tl_head:w} \meta{tokens} \cs{q_stop}
%   \end{syntax}
%   Leaves in the input stream the first non-space token from the
%   \meta{tokens}. An empty list of \meta{tokens} or one which consists
%   only of space (category code $10$) tokens will result in an error, and
%   thus \meta{tokens} must \emph{not} be \enquote{blank} as determined by
%   \cs{tl_if_blank:n(TF)}. This function requires only a single expansion,
%   and thus is suitable for use within an \texttt{o}-type expansion. In
%   general, \cs{tl_head:n} should be preferred if the number of expansions
%   is not critical.
% \end{function}
%
% \begin{function}[updated = 2012-02-08, EXP]
%   {\tl_tail:N, \tl_tail:n, \tl_tail:V, \tl_tail:v, \tl_tail:f}
%   \begin{syntax}
%     \cs{tl_tail:n} \Arg{tokens}
%   \end{syntax}
%   Discards the all leading space tokens and the first non-space token
%   in the \meta{tokens}, and leaves the remaining tokens in the
%   input stream. Thus for example
%   \begin{verbatim}
%     \tl_tail:n { abc }
%   \end{verbatim}
%   and
%   \begin{verbatim}
%     \tl_tail:n { ~ abc }
%   \end{verbatim}
%   will both leave |bc| in the input stream.
%   An empty list of \meta{tokens} or one which consists
%   only of space (category code $10$) tokens will result in \cs{tl_tail:n}
%   leaving nothing in the input stream.
%   \begin{texnote}
%     The result is returned within the \tn{unexpanded}
%     primitive (\cs{exp_not:n}), which means that the token
%     list will not expand further when appearing in an x-type
%     argument expansion.
%   \end{texnote}
% \end{function}
%
% \begin{function}[EXP]{\tl_tail:w}
%   \begin{syntax}
%     \cs{tl_tail:w} \Arg{tokens} \cs{q_stop}
%   \end{syntax}
%   Discards the all leading space tokens and the first non-space token
%   in the \meta{tokens}, and leaves the remaining tokens in the
%   input stream.
%   An empty list of \meta{tokens} or one which consists
%   only of space (category code $10$) tokens will result in an error, and
%   thus \meta{tokens} must \emph{not} be \enquote{blank} as determined by
%   \cs{tl_if_blank:n(TF)}. This function requires only a single expansion,
%   and thus is suitable for use within an \texttt{o}-type expansion. In
%   general, \cs{tl_tail:n} should be preferred if the number of expansions
%   is not critical.
% \end{function}
%
% \begin{function}[added = 2011-08-10, EXP]{\str_head:n,\str_tail:n}
%   \begin{syntax}
%     \cs{str_head:n} \Arg{tokens}
%     \cs{str_tail:n} \Arg{tokens}
%   \end{syntax}
%   Converts the \meta{tokens} into a string, as described for
%   \cs{tl_to_str:n}. The \cs{str_head:n} function then leaves
%   the first character of this string in the input stream.
%   The \cs{str_tail:n} function leaves all characters except
%   the first in the input stream. The first character may be
%   a space. If the \meta{tokens} argument is entirely empty,
%   nothing is left in the input stream.
% \end{function}
%
% \begin{function}[updated = 2011-08-10, EXP,pTF]{\tl_if_head_eq_catcode:nN}
%   \begin{syntax}
%     \cs{tl_if_head_eq_catcode_p:nN} \Arg{token list} \meta{test token}
%     \cs{tl_if_head_eq_catcode:nNTF} \Arg{token list} \meta{test token}
%     ~~\Arg{true code} \Arg{false code}
%   \end{syntax}
%   Tests if the first \meta{token} in the \meta{token list} has the same
%   category code as the \meta{test token}. In the case where
%   \meta{token list} is empty, its head is considered to be \cs{q_nil},
%   and the test will be true if \meta{test token} is a control sequence.
% \end{function}
%
% \begin{function}[updated = 2011-08-10, EXP,pTF]
%   {\tl_if_head_eq_charcode:nN, \tl_if_head_eq_charcode:fN}
%   \begin{syntax}
%     \cs{tl_if_head_eq_charcode_p:nN} \Arg{token list} \meta{test token}
%     \cs{tl_if_head_eq_charcode:nNTF} \Arg{token list} \meta{test token}
%     ~~\Arg{true code} \Arg{false code}
%   \end{syntax}
%   Tests if the first \meta{token} in the \meta{token list} has the same
%   character code as the \meta{test token}. In the case where
%   \meta{token list} is empty, its head is considered to be \cs{q_nil},
%   and the test will be true if \meta{test token} is a control sequence.
% \end{function}
%
% \begin{function}[updated = 2011-08-10, EXP,pTF]{\tl_if_head_eq_meaning:nN}
%   \begin{syntax}
%     \cs{tl_if_head_eq_meaning_p:nN} \Arg{token list} \meta{test token}
%     \cs{tl_if_head_eq_meaning:nNTF} \Arg{token list} \meta{test token}
%     ~~\Arg{true code} \Arg{false code}
%   \end{syntax}
%   Tests if the first \meta{token} in the \meta{token list} has the same
%   meaning as the \meta{test token}. In the case where \meta{token list}
%   is empty, its head is considered to be \cs{q_nil}, and the test will
%   be true if \meta{test token} has the same meaning as \cs{q_nil}.
% \end{function}
%
% \begin{function}[updated = 2011-08-11, EXP,pTF]{\tl_if_head_group:n}
%   \begin{syntax}
%     \cs{tl_if_head_group_p:n} \Arg{token list}
%     \cs{tl_if_head_group:nTF} \Arg{token list} \Arg{true code} \Arg{false code}
%   \end{syntax}
%   Tests if the first \meta{token} in the \meta{token list}
%   is an explicit begin-group character (with category code~1
%   and any character code), in other words, if the \meta{token list}
%   starts with a brace group. In particular, the test is false
%   if the \meta{token list} starts with an implicit token such as
%   \cs{c_group_begin_token}, or if it empty.
%   This function is useful to implement actions on token lists on
%   a token by token basis.
% \end{function}
%
% \begin{function}[added = 2011-08-11,EXP,pTF]{\tl_if_head_N_type:n}
%   \begin{syntax}
%     \cs{tl_if_head_N_type_p:n} \Arg{token list}
%     \cs{tl_if_head_N_type:nTF} \Arg{token list} \Arg{true code} \Arg{false code}
%   \end{syntax}
%   Tests if the first \meta{token} in the \meta{token list}
%   is a normal \texttt{N}-type argument. In other words,
%   it is neither an explicit space character (with category code~10
%   and character code~32) nor an explicit begin-group character
%   (with category code~1 and any character code). An empty
%   argument yields false, as it does not have a \enquote{normal}
%   first token.
%   This function is useful to implement actions on token lists on
%   a token by token basis.
% \end{function}
%
% \begin{function}[updated = 2011-08-11,EXP,pTF]{\tl_if_head_space:n}
%   \begin{syntax}
%     \cs{tl_if_head_space_p:n} \Arg{token list}
%     \cs{tl_if_head_space:nTF} \Arg{token list} \Arg{true code} \Arg{false code}
%   \end{syntax}
%^^A We need to add a discussion of
%^^A explicit vs implicit tokens somewhere in the doc.
%   Tests if the first \meta{token} in the \meta{token list}
%   is an explicit space character (with category code~$10$
%   and character code~$32$). If \meta{token list} starts with
%   an implicit token such as \cs{c_space_token}, the test
%   will yield false, as well as if the argument is empty.
%   This function is useful to implement actions on token lists on
%   a token by token basis.
%   \begin{texnote}
%     When \TeX{} reads a character of category code $10$ for the
%     first time, it is converted to an explicit space token, with
%     character code $32$, regardless of the initial character code.
%     \enquote{Funny} spaces with a different category code, can be produced
%     using \tn{lowercase}. Explicit spaces are also produced
%     as a result of \cs{token_to_str:N}, \cs{tl_to_str:n}, etc.
%   \end{texnote}
% \end{function}
%
% \section{Viewing token lists}
%
% \begin{function}{\tl_show:N, \tl_show:c}
%   \begin{syntax}
%     \cs{tl_show:N} \meta{tl~var}
%   \end{syntax}
%   Displays the content of the \meta{tl~var} on the terminal.
%   \begin{texnote}
%     \cs{tl_show:N} is the \TeX{} primitive \tn{show}.
%   \end{texnote}
% \end{function}
%
% \begin{function}{\tl_show:n}
%   \begin{syntax}
%     \cs{tl_show:n} \meta{token list}
%   \end{syntax}
%   Displays the \meta{token list} on the terminal.
%   \begin{texnote}
%     \cs{tl_show:n} is the \eTeX{} primitive \tn{showtokens}.
%   \end{texnote}
% \end{function}
%
% \section{Constant token lists}
%
% \begin{variable}[updated = 2011-08-18]{\c_job_name_tl}
%   Constant that gets the \enquote{job name} assigned when \TeX{} starts.
%   \begin{texnote}
%     This is the new name for the primitive \tn{jobname}. It is a constant
%     that is set by \TeX{} and should not be overwritten by the package.
%   \end{texnote}
% \end{variable}
%
% \begin{variable}{\c_empty_tl}
%   Constant that is always empty.
% \end{variable}
%
% \begin{variable}{\c_space_tl}
%   A space token contained in a token list (compare this with
%   \cs{c_space_token}). For use where an explicit space is required.
% \end{variable}
%
% \section{Scratch token lists}
%
% \begin{variable}{\l_tmpa_tl, \l_tmpb_tl}
%   Scratch token lists for local assignment. These are never used by
%   the kernel code, and so are safe for use with any \LaTeX3-defined
%   function. However, they may be overwritten by other non-kernel
%   code and so should only be used for short-term storage.
% \end{variable}
%
% \begin{variable}{\g_tmpa_tl, \g_tmpb_tl}
%   Scratch token lists for global assignment. These are never used by
%   the kernel code, and so are safe for use with any \LaTeX3-defined
%   function. However, they may be overwritten by other non-kernel
%   code and so should only be used for short-term storage.
% \end{variable}
%
% \section{Experimental token list functions}
%
% \begin{function}[added = 2012-01-08, EXP]{\tl_reverse_tokens:n}
%   \begin{syntax}
%     \cs{tl_reverse_tokens:n} \Arg{tokens}
%   \end{syntax}
%   This function, which works directly on \TeX{} tokens, reverses
%   the order of the \meta{tokens}: the first will be the last and
%   the last will become first. Spaces are preserved. The reversal
%   also operates within brace groups, but the braces themselves
%   are not exchanged, as this would lead to an unbalanced token
%   list. For instance, \cs{tl_reverse_tokens:n} |{a~{b()}}|
%   leaves |{)(b}~a| in the input stream. This function requires
%   two steps of expansion.
%   \begin{texnote}
%     The result is returned within the \tn{unexpanded}
%     primitive (\cs{exp_not:n}), which means that the token
%     list will not expand further when appearing in an x-type
%     argument expansion.
%   \end{texnote}
% \end{function}
%
% \begin{function}[added = 2012-05-13, EXP]{\tl_count_tokens:n}
%   \begin{syntax}
%     \cs{tl_count_tokens:n} \Arg{tokens}
%   \end{syntax}
%   Counts the number of \TeX{} tokens in the \meta{tokens} and leaves
%   this information in the input stream. Every token, including spaces and
%   braces, contributes one to the total; thus for instance, the token count of
%   |a~{bc}| is $6$.
%   This function requires three expansions,
%   giving an \meta{integer denotation}.
% \end{function}
%
% \begin{function}[added = 2012-01-08, EXP]
%   {\tl_expandable_uppercase:n,\tl_expandable_lowercase:n}
%   \begin{syntax}
%     \cs{tl_expandable_uppercase:n} \Arg{tokens}
%     \cs{tl_expandable_lowercase:n} \Arg{tokens}
%   \end{syntax}
%   The \cs{tl_expandable_uppercase:n} function works through all of
%   the \meta{tokens}, replacing characters in the range |a|--|z|
%   (with arbitrary category code) by the corresponding letter
%   in the range |A|--|Z|, with category code $11$ (letter). Similarly,
%   \cs{tl_expandable_lowercase:n} replaces characters in the range
%   |A|--|Z| by letters in the range |a|--|z|, and leaves other tokens
%   unchanged. This function requires two steps of expansion.
%   \begin{texnote}
%     Begin-group and end-group characters are normalized and become
%     |{| and |}|, respectively.
%     The result is returned within the \tn{unexpanded}
%     primitive (\cs{exp_not:n}), which means that the token
%     list will not expand further when appearing in an x-type
%     argument expansion.
%   \end{texnote}
% \end{function}
%
% \begin{function}[added = 2011-11-21, updated = 2012-01-08, EXP]
%   {\tl_item:nn, \tl_item:Nn, \tl_item:cn}
%   \begin{syntax}
%     \cs{tl_item:nn} \Arg{token list} \Arg{integer expression}
%   \end{syntax}
%   Indexing items in the \meta{token list} from $0$ on the left, this
%   function will evaluate the \meta{integer expression} and leave the
%   appropriate item from the \meta{token list} in the input stream.
%   If the \meta{integer expression} is negative, indexing occurs from
%   the right of the token list, starting at $-1$ for the right-most item.
%   If the index is out of bounds, then thr function expands to nothing.
%   \begin{texnote}
%     The result is returned within the \tn{unexpanded}
%     primitive (\cs{exp_not:n}), which means that the \meta{item}
%     will not expand further when appearing in an x-type
%     argument expansion.
%   \end{texnote}
% \end{function}
%
% \section{Internal functions}
%
% \begin{variable}{\q_tl_act_mark,\q_tl_act_stop}
%   Quarks which are only used for the particular purposes of
%   \cs{tl_act_...} functions.
% \end{variable}
%
% \end{documentation}
%
% \begin{implementation}
%
% \section{\pkg{l3tl} implementation}
%
%    \begin{macrocode}
%<*initex|package>
%    \end{macrocode}
%
%    \begin{macrocode}
%<*package>
\ProvidesExplPackage
  {\ExplFileName}{\ExplFileDate}{\ExplFileVersion}{\ExplFileDescription}
\package_check_loaded_expl:
%</package>
%    \end{macrocode}
%
% A token list variable is a \TeX{} macro that holds tokens. By using the
% \eTeX{} primitive \tn{unexpanded} inside a \TeX{} \tn{edef} it is
% possible to store any tokens, including |#|, in this way.
%
% \subsection{Functions}
%
% \begin{macro}{\tl_new:N, \tl_new:c}
%   Creating new token list variables is a case of checking for an
%   existing definition and if free doing the definition.
%    \begin{macrocode}
\cs_new_protected:Npn \tl_new:N #1
  {
    \chk_if_free_cs:N #1
    \cs_gset_eq:NN #1 \c_empty_tl
  }
\cs_generate_variant:Nn \tl_new:N { c }
%    \end{macrocode}
% \end{macro}
%
% \begin{macro}{\tl_const:Nn, \tl_const:Nx, \tl_const:cn, \tl_const:cx}
%   Constants are also easy to generate.
%    \begin{macrocode}
\cs_new_protected:Npn \tl_const:Nn #1#2
  {
    \chk_if_free_cs:N #1
    \cs_gset_nopar:Npx #1 { \exp_not:n {#2} }
  }
\cs_new_protected:Npn \tl_const:Nx #1#2
  {
    \chk_if_free_cs:N #1
    \cs_gset_nopar:Npx #1 {#2}
  }
\cs_generate_variant:Nn \tl_const:Nn { c }
\cs_generate_variant:Nn \tl_const:Nx { c }
%    \end{macrocode}
% \end{macro}
%
% \begin{variable}{\c_empty_tl}
%   Never full. We need to define that constant early
%   for \cs{tl_new:N} to work properly.
%    \begin{macrocode}
\tl_const:Nn \c_empty_tl { }
%    \end{macrocode}
% \end{variable}
%
% \begin{macro}{\tl_clear:N, \tl_clear:c}
% \begin{macro}{\tl_gclear:N, \tl_gclear:c}
%   Clearing a token list variable means setting it to an empty value.
%   Error checking will be sorted out by the parent function.
%    \begin{macrocode}
\cs_new_protected:Npn \tl_clear:N  #1
  { \tl_set_eq:NN #1 \c_empty_tl }
\cs_new_protected:Npn \tl_gclear:N #1
  { \tl_gset_eq:NN #1 \c_empty_tl }
\cs_generate_variant:Nn \tl_clear:N  { c }
\cs_generate_variant:Nn \tl_gclear:N { c }
%    \end{macrocode}
% \end{macro}
% \end{macro}
%
% \begin{macro}{\tl_clear_new:N, \tl_clear_new:c}
% \begin{macro}{\tl_gclear_new:N, \tl_gclear_new:c}
%   Clearing a token list variable means setting it to an empty value.
%   Error checking will be sorted out by the parent function.
%    \begin{macrocode}
\cs_new_protected:Npn \tl_clear_new:N  #1
  { \tl_if_exist:NTF #1 { \tl_clear:N #1 } { \tl_new:N #1 } }
\cs_new_protected:Npn \tl_gclear_new:N #1
  { \tl_if_exist:NTF #1 { \tl_gclear:N #1 } { \tl_new:N #1 } }
\cs_generate_variant:Nn \tl_clear_new:N  { c }
\cs_generate_variant:Nn \tl_gclear_new:N { c }
%    \end{macrocode}
% \end{macro}
% \end{macro}
%
% \begin{macro}{\tl_set_eq:NN, \tl_set_eq:Nc, \tl_set_eq:cN, \tl_set_eq:cc}
% \begin{macro}{\tl_gset_eq:NN, \tl_gset_eq:Nc, \tl_gset_eq:cN, \tl_gset_eq:cc}
%   For setting token list variables equal to each other.
%    \begin{macrocode}
\cs_new_eq:NN \tl_set_eq:NN  \cs_set_eq:NN
\cs_new_eq:NN \tl_set_eq:cN  \cs_set_eq:cN
\cs_new_eq:NN \tl_set_eq:Nc  \cs_set_eq:Nc
\cs_new_eq:NN \tl_set_eq:cc  \cs_set_eq:cc
\cs_new_eq:NN \tl_gset_eq:NN \cs_gset_eq:NN
\cs_new_eq:NN \tl_gset_eq:cN \cs_gset_eq:cN
\cs_new_eq:NN \tl_gset_eq:Nc \cs_gset_eq:Nc
\cs_new_eq:NN \tl_gset_eq:cc \cs_gset_eq:cc
%    \end{macrocode}
% \end{macro}
% \end{macro}
%
% \begin{macro}{\tl_concat:NNN, \tl_concat:ccc}
% \begin{macro}{\tl_gconcat:NNN, \tl_gconcat:ccc}
%   Concatenating token lists is easy.
%    \begin{macrocode}
\cs_new_protected:Npn \tl_concat:NNN #1#2#3
  { \tl_set:Nx #1 { \exp_not:o {#2} \exp_not:o {#3} } }
\cs_new_protected:Npn \tl_gconcat:NNN #1#2#3
  { \tl_gset:Nx #1 { \exp_not:o {#2} \exp_not:o {#3} } }
\cs_generate_variant:Nn \tl_concat:NNN  { ccc }
\cs_generate_variant:Nn \tl_gconcat:NNN { ccc }
%    \end{macrocode}
% \end{macro}
% \end{macro}
%
% \begin{macro}[pTF]{\tl_if_exist:N, \tl_if_exist:c}
%   Copies of the \texttt{cs} functions defined in \pkg{l3basics}.
%    \begin{macrocode}
\cs_new_eq:NN \tl_if_exist:NTF \cs_if_exist:NTF
\cs_new_eq:NN \tl_if_exist:NT  \cs_if_exist:NT
\cs_new_eq:NN \tl_if_exist:NF  \cs_if_exist:NF
\cs_new_eq:NN \tl_if_exist_p:N \cs_if_exist_p:N
\cs_new_eq:NN \tl_if_exist:cTF \cs_if_exist:cTF
\cs_new_eq:NN \tl_if_exist:cT  \cs_if_exist:cT
\cs_new_eq:NN \tl_if_exist:cF  \cs_if_exist:cF
\cs_new_eq:NN \tl_if_exist_p:c \cs_if_exist_p:c
%    \end{macrocode}
% \end{macro}
% 
% \subsection{Constant token lists}
%
% \begin{variable}{\c_job_name_tl}
%   Inherited from the \LaTeX3 name for the primitive: this needs to
%   actually contain the text of the job name rather than the name of
%   the primitive, of course. \LuaTeX{} does not quote file names containing
%   spaces, whereas \pdfTeX{} and \XeTeX{} do. So there may be a correction to
%   make in the \LuaTeX{} case.
%    \begin{macrocode}
%<*initex>
\tex_everyjob:D \exp_after:wN
  {
    \tex_the:D \tex_everyjob:D
    \luatex_if_engine:T
      {
        \lua_now:x
          { dofile ( assert ( kpse.find_file ("lualatexquotejobname.lua" ) ) ) }
      }
  }
%</initex>
\tl_const:Nx \c_job_name_tl { \tex_jobname:D }
%    \end{macrocode}
% \end{variable}
%
% \begin{variable}{\c_space_tl}
%   A space as a token list (as opposed to as a character).
%    \begin{macrocode}
\tl_const:Nn \c_space_tl { ~ }
%    \end{macrocode}
% \end{variable}
%
% \subsection{Adding to token list variables}
%
% \begin{macro}
%   {
%     \tl_set:Nn, \tl_set:NV, \tl_set:Nv, \tl_set:No, \tl_set:Nf, \tl_set:Nx,
%     \tl_set:cn, \tl_set:NV, \tl_set:Nv, \tl_set:co, \tl_set:cf, \tl_set:cx
%   }
% \begin{macro}
%   {
%     \tl_gset:Nn, \tl_gset:NV, \tl_gset:Nv,
%     \tl_gset:No, \tl_gset:Nf, \tl_gset:Nx,
%     \tl_gset:cn, \tl_gset:NV, \tl_gset:Nv,
%     \tl_gset:co, \tl_gset:cf, \tl_gset:cx
%   }
%   By using \cs{exp_not:n} token list variables can contain |#| tokens,
%   which makes the token list registers provided by \TeX{}
%   more or less redundant. The \cs{tl_set:No} version is done
%   \enquote{by hand} as it is used quite a lot.
%    \begin{macrocode}
\cs_new_protected:Npn \tl_set:Nn #1#2
  { \cs_set_nopar:Npx #1 { \exp_not:n {#2} } }
\cs_new_protected:Npn \tl_set:No #1#2
  { \cs_set_nopar:Npx #1 { \exp_not:o {#2} } }
\cs_new_protected:Npn \tl_set:Nx #1#2
  { \cs_set_nopar:Npx #1 {#2} }
\cs_new_protected:Npn \tl_gset:Nn #1#2
  { \cs_gset_nopar:Npx #1 { \exp_not:n {#2} } }
\cs_new_protected:Npn \tl_gset:No #1#2
  { \cs_gset_nopar:Npx #1 { \exp_not:o {#2} } }
\cs_new_protected:Npn \tl_gset:Nx #1#2
  { \cs_gset_nopar:Npx #1 {#2} }
\cs_generate_variant:Nn \tl_set:Nn  {         NV , Nv , Nf }
\cs_generate_variant:Nn \tl_set:Nx  { c }
\cs_generate_variant:Nn \tl_set:Nn  { c, co , cV , cv , cf }
\cs_generate_variant:Nn \tl_gset:Nn {         NV , Nv , Nf }
\cs_generate_variant:Nn \tl_gset:Nx { c }
\cs_generate_variant:Nn \tl_gset:Nn { c, co , cV , cv , cf }
%    \end{macrocode}
% \end{macro}
% \end{macro}
%
% \begin{macro}
%   {
%     \tl_put_left:Nn, \tl_put_left:NV, \tl_put_left:No, \tl_put_left:Nx,
%     \tl_put_left:cn, \tl_put_left:cV, \tl_put_left:co, \tl_put_left:cx
%   }
% \begin{macro}
%   {
%     \tl_gput_left:Nn, \tl_gput_left:NV, \tl_gput_left:No, \tl_gput_left:Nx,
%     \tl_gput_left:cn, \tl_gput_left:cV, \tl_gput_left:co, \tl_gput_left:cx
%   }
% Adding to the left is done directly to gain a little performance.
%    \begin{macrocode}
\cs_new_protected:Npn \tl_put_left:Nn #1#2
  { \cs_set_nopar:Npx #1 { \exp_not:n {#2} \exp_not:o #1 } }
\cs_new_protected:Npn \tl_put_left:NV #1#2
  { \cs_set_nopar:Npx #1 { \exp_not:V #2 \exp_not:o #1 } }
\cs_new_protected:Npn \tl_put_left:No #1#2
  { \cs_set_nopar:Npx #1 { \exp_not:o {#2} \exp_not:o #1 } }
\cs_new_protected:Npn \tl_put_left:Nx #1#2
  { \cs_set_nopar:Npx #1 { #2 \exp_not:o #1 } }
\cs_new_protected:Npn \tl_gput_left:Nn #1#2
  { \cs_gset_nopar:Npx #1 { \exp_not:n {#2} \exp_not:o #1 } }
\cs_new_protected:Npn \tl_gput_left:NV #1#2
  { \cs_gset_nopar:Npx #1 { \exp_not:V #2 \exp_not:o #1 } }
\cs_new_protected:Npn \tl_gput_left:No #1#2
  { \cs_gset_nopar:Npx #1 { \exp_not:o {#2} \exp_not:o #1 } }
\cs_new_protected:Npn \tl_gput_left:Nx #1#2
  { \cs_gset_nopar:Npx #1 { #2 \exp_not:o {#1} } }
\cs_generate_variant:Nn \tl_put_left:Nn  { c }
\cs_generate_variant:Nn \tl_put_left:NV  { c }
\cs_generate_variant:Nn \tl_put_left:No  { c }
\cs_generate_variant:Nn \tl_put_left:Nx  { c }
\cs_generate_variant:Nn \tl_gput_left:Nn { c }
\cs_generate_variant:Nn \tl_gput_left:NV { c }
\cs_generate_variant:Nn \tl_gput_left:No { c }
\cs_generate_variant:Nn \tl_gput_left:Nx { c }
%    \end{macrocode}
% \end{macro}
% \end{macro}
%
% \begin{macro}
%   {
%     \tl_put_right:Nn, \tl_put_right:NV, \tl_put_right:No, \tl_put_right:Nx,
%     \tl_put_right:cn, \tl_put_right:cV, \tl_put_right:co, \tl_put_right:cx
%   }
% \begin{macro}
%   {
%     \tl_gput_right:Nn, \tl_gput_right:NV, \tl_gput_right:No,
%     \tl_gput_right:Nx,
%     \tl_gput_right:cn, \tl_gput_right:cV, \tl_gput_right:co,
%     \tl_gput_right:cx
%   }
% The same on the right.
%    \begin{macrocode}
\cs_new_protected:Npn \tl_put_right:Nn #1#2
  { \cs_set_nopar:Npx #1 { \exp_not:o #1 \exp_not:n {#2} } }
\cs_new_protected:Npn \tl_put_right:NV #1#2
  { \cs_set_nopar:Npx #1 { \exp_not:o #1 \exp_not:V #2 } }
\cs_new_protected:Npn \tl_put_right:No #1#2
  { \cs_set_nopar:Npx #1 { \exp_not:o #1 \exp_not:o {#2} } }
\cs_new_protected:Npn \tl_put_right:Nx #1#2
  { \cs_set_nopar:Npx #1 { \exp_not:o #1 #2 } }
\cs_new_protected:Npn \tl_gput_right:Nn #1#2
  { \cs_gset_nopar:Npx #1 { \exp_not:o #1 \exp_not:n {#2} } }
\cs_new_protected:Npn \tl_gput_right:NV #1#2
  { \cs_gset_nopar:Npx #1 { \exp_not:o #1 \exp_not:V #2 } }
\cs_new_protected:Npn \tl_gput_right:No #1#2
  { \cs_gset_nopar:Npx #1 { \exp_not:o #1 \exp_not:o {#2} } }
\cs_new_protected:Npn \tl_gput_right:Nx #1#2
  { \cs_gset_nopar:Npx #1 { \exp_not:o {#1} #2 } }
\cs_generate_variant:Nn \tl_put_right:Nn  { c }
\cs_generate_variant:Nn \tl_put_right:NV  { c }
\cs_generate_variant:Nn \tl_put_right:No  { c }
\cs_generate_variant:Nn \tl_put_right:Nx  { c }
\cs_generate_variant:Nn \tl_gput_right:Nn { c }
\cs_generate_variant:Nn \tl_gput_right:NV { c }
\cs_generate_variant:Nn \tl_gput_right:No { c }
\cs_generate_variant:Nn \tl_gput_right:Nx { c }
%    \end{macrocode}
% \end{macro}
% \end{macro}
%
% \subsection{Reassigning token list category codes}
%
% \begin{variable}{\c_tl_rescan_marker_tl}
%   The rescanning code needs a special token list containing the same
%   character with two different category codes. This is set up here,
%   while the detail is described below.
%    \begin{macrocode}
\group_begin:
  \tex_lccode:D `\A = `\@ \scan_stop:
  \tex_lccode:D `\B = `\@ \scan_stop:
  \tex_catcode:D `\A = 8 \scan_stop:
  \tex_catcode:D `\B = 3 \scan_stop:
\tex_lowercase:D
  {
    \group_end:
    \tl_const:Nn \c_tl_rescan_marker_tl { A B }
  }
%    \end{macrocode}
% \end{variable}
%
% \begin{macro}
%   {
%     \tl_set_rescan:Nnn, \tl_set_rescan:Nno, \tl_set_rescan:Nnx,
%     \tl_set_rescan:cnn, \tl_set_rescan:cno, \tl_set_rescan:cnx
%   }
% \begin{macro}
%   {
%     \tl_gset_rescan:Nnn, \tl_gset_rescan:Nno, \tl_gset_rescan:Nnx,
%     \tl_gset_rescan:cnn, \tl_gset_rescan:cno, \tl_gset_rescan:cnx
%   }
% \begin{macro}{\tl_rescan:nn}
% \begin{macro}[aux]{\tl_set_rescan_aux:NNnn}
% \begin{macro}[aux]{\tl_rescan_aux:w}
%   The idea here is to deal cleanly with the problem that
%   \tn{scantokens} treats the argument as a file, and without
%   the correct settings a \TeX{} error occurs:
%   \begin{verbatim}
%    ! File ended while scanning definition of ...
%   \end{verbatim}
%   When expanding a token list this can be handled using \cs{exp_not:N}
%   but this fails if the token list is not being expanded. So instead
%   a delimited argument is used with an end marker which cannot appear
%   within the token list which is scanned: two |@| symbols with different
%   category codes. The rescanned token list cannot contain the end marker,
%   because all |@| present in the token list are read with the same category
%   code. As every character with charcode \tn{newlinechar} is replaced
%   by the \tn{endlinechar}, and an extra \tn{endlinechar} is
%   added at the end, we need to set both of those to $-1$,
%   \enquote{unprintable}.
%    \begin{macrocode}
\cs_new_protected_nopar:Npn \tl_set_rescan:Nnn
  { \tl_set_rescan_aux:NNnn \tl_set:Nn }
\cs_new_protected_nopar:Npn \tl_gset_rescan:Nnn
  { \tl_set_rescan_aux:NNnn \tl_gset:Nn }
\cs_new_protected_nopar:Npn \tl_rescan:nn
  { \tl_set_rescan_aux:NNnn \prg_do_nothing: \use:n }
\cs_new_protected:Npn \tl_set_rescan_aux:NNnn #1#2#3#4
  {
    \group_begin:
      \exp_args:No \etex_everyeof:D { \c_tl_rescan_marker_tl \exp_not:N }
      \tex_endlinechar:D \c_minus_one
      \tex_newlinechar:D \c_minus_one
      #3
      \use:x
        {
          \group_end:
          #1 \exp_not:N #2
            {
              \exp_after:wN \tl_rescan_aux:w
              \exp_after:wN \prg_do_nothing:
              \etex_scantokens:D {#4}
            }
        }
  }
\use:x
  {
    \cs_new:Npn \exp_not:N \tl_rescan_aux:w ##1
      \c_tl_rescan_marker_tl
      { \exp_not:N \exp_not:o { ##1 } }
  }
\cs_generate_variant:Nn \tl_set_rescan:Nnn  {     Nno , Nnx }
\cs_generate_variant:Nn \tl_set_rescan:Nnn  { c , cno , cnx }
\cs_generate_variant:Nn \tl_gset_rescan:Nnn {     Nno , Nnx }
\cs_generate_variant:Nn \tl_gset_rescan:Nnn { c , cno }
%    \end{macrocode}
% \end{macro}
% \end{macro}
% \end{macro}
% \end{macro}
% \end{macro}
%
% \subsection{Reassigning token list character codes}
%
% \begin{macro}{\tl_to_lowercase:n}
% \begin{macro}{\tl_to_uppercase:n}
%   Just some names for a few primitives.
%    \begin{macrocode}
\cs_new_eq:NN \tl_to_lowercase:n \tex_lowercase:D
\cs_new_eq:NN \tl_to_uppercase:n \tex_uppercase:D
%    \end{macrocode}
% \end{macro}
% \end{macro}
%
% \subsection{Modifying token list variables}
%
% \begin{macro}{\tl_replace_all:Nnn, \tl_replace_all:cnn}
% \begin{macro}{\tl_greplace_all:Nnn, \tl_greplace_all:cnn}
% \begin{macro}{\tl_replace_once:Nnn, \tl_replace_once:cnn}
% \begin{macro}{\tl_greplace_once:Nnn, \tl_greplace_once:cnn}
% \begin{macro}[aux]{\tl_replace_aux:NNNnn, \tl_replace_aux_ii:w}
% \begin{macro}[aux]{\tl_replace_all_aux:, \tl_replace_once_aux:}
% \begin{macro}[aux]{\tl_replace_once_aux_end:w}
%   All of the replace functions are based on \cs{tl_replace_aux:NNNnn},
%   whose arguments are: \meta{function}, \cs{tl_(g)set:Nx}, \meta{tl~var},
%   \meta{search tokens}, \meta{replacement tokens}.
%    \begin{macrocode}
\cs_new_protected_nopar:Npn \tl_replace_once:Nnn
  { \tl_replace_aux:NNNnn \tl_replace_once_aux: \tl_set:Nx }
\cs_new_protected_nopar:Npn \tl_greplace_once:Nnn
  { \tl_replace_aux:NNNnn \tl_replace_once_aux: \tl_gset:Nx }
\cs_new_protected_nopar:Npn \tl_replace_all:Nnn
  { \tl_replace_aux:NNNnn \tl_replace_all_aux: \tl_set:Nx }
\cs_new_protected_nopar:Npn \tl_greplace_all:Nnn
  { \tl_replace_aux:NNNnn \tl_replace_all_aux: \tl_gset:Nx }
\cs_generate_variant:Nn \tl_replace_once:Nnn  { c }
\cs_generate_variant:Nn \tl_greplace_once:Nnn { c }
\cs_generate_variant:Nn \tl_replace_all:Nnn   { c }
\cs_generate_variant:Nn \tl_greplace_all:Nnn  { c }
%    \end{macrocode}
%   The idea is easier to understand by considering the case of
%   \cs{tl_replace_all:Nnn}. The replacement happens within an
%   \texttt{x}-type expansion. We use an auxiliary function \cs{tl_tmp:w},
%   which essentially replaces the next \meta{search tokens} by
%   \meta{replacement tokens}. To avoid runaway arguments,
%   we expand something like \cs{tl_tmp:w} \meta{token list} \cs{q_mark}
%   \meta{search tokens} \cs{q_stop}, repeating until the end. How do we
%   detect that we have reached the last occurrence of \meta{search tokens}?
%   The last replacement is characterized by the fact that the argument of
%   \cs{tl_tmp:w} contains \cs{q_mark}. In the code below,
%   \cs{tl_replace_aux_ii:w} takes an argument delimited by \cs{q_mark},
%   and removes the following token. Before we reach the end, this gobbles
%   \cs{q_mark} \cs{use_none_delimit_by_q_stop:w} which appear in the
%   definition of \cs{tl_tmp:w}, and leaves the \meta{replacement tokens},
%   passed to \cs{exp_not:n}, to be included in the \texttt{x}-expanding
%   definition. At the end, the first \cs{q_mark} is within the argument
%   of \cs{tl_tmp:w}, and \cs{tl_replace_aux_ii:w} gobbles the second
%   \cs{q_mark} as well, leaving \cs{use_none_delimit_by_q_stop:w},
%   which ends the recursion cleanly.
%    \begin{macrocode}
\cs_new_protected:Npn \tl_replace_aux:NNNnn #1#2#3#4#5
  {
    \tl_if_empty:nTF {#4}
      {
         \msg_kernel_error:nnx { tl } { empty-search-pattern }
           { \tl_to_str:n {#5} }
      }
      {
        \group_align_safe_begin:
        \cs_set:Npx \tl_tmp:w ##1##2 #4
          {
            ##2
            \exp_not:N \q_mark
            \exp_not:N \use_none_delimit_by_q_stop:w
            \exp_not:n { \exp_not:n {#5} }
            ##1
          }
        \group_align_safe_end:
        #2 #3
          {
            \exp_after:wN #1
            #3 \q_mark #4 \q_stop
          }
      }
  }
\cs_new:Npn \tl_replace_aux_ii:w #1 \q_mark #2 { \exp_not:o {#1} }
%    \end{macrocode}
%   The first argument of \cs{tl_tmp:w} is responsible for repeating
%   the replacement in the case of \texttt{replace_all}, and stopping
%   it early for \texttt{replace_once}. Note also that we build
%   \cs{tl_tmp:w} within an \texttt{x}-expansion so that the
%   \meta{replacement tokens} can contain |#|. The second
%   \cs{exp_not:n} ensures that the \meta{replacement tokens}
%   are not expanded by \cs{tl_(g)set:Nx}.
%
%   Now on to the difference between \enquote{once} and \enquote{all}.
%   The \cs{prg_do_nothing:} and accompanying \texttt{o}-expansion
%   ensure that we don't lose braces in case the tokens between two
%   occurrences of the \meta{search tokens} form a brace group.
%    \begin{macrocode}
\cs_new:Npn \tl_replace_all_aux:
  {
    \exp_after:wN \tl_replace_aux_ii:w
    \tl_tmp:w \tl_replace_all_aux: \prg_do_nothing:
  }
\cs_new_nopar:Npn \tl_replace_once_aux:
  {
    \exp_after:wN \tl_replace_aux_ii:w
    \tl_tmp:w { \tl_replace_once_aux_end:w \prg_do_nothing: } \prg_do_nothing:
  }
\cs_new:Npn \tl_replace_once_aux_end:w #1 \q_mark #2 \q_stop
  { \exp_not:o {#1} }
%    \end{macrocode}
% \end{macro}
% \end{macro}
% \end{macro}
% \end{macro}
% \end{macro}
% \end{macro}
% \end{macro}
%
% \begin{macro}{\tl_remove_once:Nn, \tl_remove_once:cn}
% \begin{macro}{\tl_gremove_once:Nn, \tl_gremove_once:cn}
%   Removal is just a special case of replacement.
%    \begin{macrocode}
\cs_new_protected:Npn \tl_remove_once:Nn #1#2
  { \tl_replace_once:Nnn #1 {#2} { } }
\cs_new_protected:Npn \tl_gremove_once:Nn #1#2
  { \tl_greplace_once:Nnn #1 {#2} { } }
\cs_generate_variant:Nn \tl_remove_once:Nn  { c }
\cs_generate_variant:Nn \tl_gremove_once:Nn { c }
%    \end{macrocode}
% \end{macro}
% \end{macro}
%
% \begin{macro}{\tl_remove_all:Nn, \tl_remove_all:cn}
% \begin{macro}{\tl_gremove_all:Nn, \tl_gremove_all:cn}
%   Removal is just a special case of replacement.
%    \begin{macrocode}
\cs_new_protected:Npn \tl_remove_all:Nn #1#2
  { \tl_replace_all:Nnn #1 {#2} { } }
\cs_new_protected:Npn \tl_gremove_all:Nn #1#2
  { \tl_greplace_all:Nnn #1 {#2} { } }
\cs_generate_variant:Nn \tl_remove_all:Nn  { c }
\cs_generate_variant:Nn \tl_gremove_all:Nn { c }
%    \end{macrocode}
% \end{macro}
%
% \subsection{Token list conditionals}
%
% \begin{macro}[pTF]{\tl_if_blank:n,\tl_if_blank:V,\tl_if_blank:o}
% \begin{macro}[aux]{\tl_if_blank_p_aux:NNw}
%   \TeX{} skips spaces when reading a non-delimited arguments. Thus,
%   a \meta{token list} is blank if and only if \cs{use_none:n}
%   \meta{token list} |?| is empty. For performance reasons, we hard-code
%   the emptyness test done in \cs{tl_if_empty:n(TF)}: convert to harmless
%   characters with \cs{tl_to_str:n}, and then use
%   \cs{if_meaning:w} \cs{q_nil} |...| \cs{q_nil}.
%   Note that converting to a string is done after reading the delimited
%   argument for \cs{use_none:n}. The similar construction
%   \cs{exp_after:wN} \cs{use_none:n} \cs{tl_to_str:n} \Arg{token list} |?|
%   would fail if the token list contains the control sequence \cs{ },
%   while \tn{escapechar} is a space or is unprintable.
%    \begin{macrocode}
\prg_new_conditional:Npnn \tl_if_blank:n #1 { p , T , F , TF }
  { \tl_if_empty_return:o { \use_none:n #1 ? } }
\cs_generate_variant:Nn \tl_if_blank_p:n { V }
\cs_generate_variant:Nn \tl_if_blank:nT  { V }
\cs_generate_variant:Nn \tl_if_blank:nF  { V }
\cs_generate_variant:Nn \tl_if_blank:nTF { V }
\cs_generate_variant:Nn \tl_if_blank_p:n { o }
\cs_generate_variant:Nn \tl_if_blank:nT  { o }
\cs_generate_variant:Nn \tl_if_blank:nF  { o }
\cs_generate_variant:Nn \tl_if_blank:nTF { o }
%    \end{macrocode}
% \end{macro}
% \end{macro}
% \end{macro}
%
% \begin{macro}[pTF]{\tl_if_empty:N,\tl_if_empty:c}
%    These functions check whether the token list in the argument is
%    empty and execute the proper code from their argument(s).
%    \begin{macrocode}
\prg_new_conditional:Npnn \tl_if_empty:N #1 { p , T , F , TF }
  {
    \if_meaning:w #1 \c_empty_tl
      \prg_return_true:
    \else:
      \prg_return_false:
    \fi:
  }
\cs_generate_variant:Nn \tl_if_empty_p:N { c }
\cs_generate_variant:Nn \tl_if_empty:NT  { c }
\cs_generate_variant:Nn \tl_if_empty:NF  { c }
\cs_generate_variant:Nn \tl_if_empty:NTF { c }
%    \end{macrocode}
% \end{macro}
%
% \begin{macro}[pTF]{\tl_if_empty:n,\tl_if_empty:V}
%   It would be tempting to just use |\if_meaning:w \q_nil #1 \q_nil| as
%   a test since this works really well. However, it fails on a token
%   list starting with |\q_nil| of course but more troubling is the
%   case where argument is a complete conditional such as |\if_true:|
%   a |\else:| b |\fi:| because then |\if_true:| is used by
%   |\if_meaning:w|, the test turns out false, the |\else:| executes
%   the false branch, the |\fi:| ends it and the |\q_nil| at the end
%   starts executing\dots{} A safer route is to convert the entire
%   token list into harmless characters first and then compare
%   that. This way the test will even accept |\q_nil| as the first
%   token.
%    \begin{macrocode}
\prg_new_conditional:Npnn \tl_if_empty:n #1 { p , TF , T , F }
  {
    \exp_after:wN \if_meaning:w \exp_after:wN \q_nil \tl_to_str:n {#1} \q_nil
      \prg_return_true:
    \else:
      \prg_return_false:
    \fi:
  }
\cs_generate_variant:Nn \tl_if_empty_p:n { V }
\cs_generate_variant:Nn \tl_if_empty:nTF { V }
\cs_generate_variant:Nn \tl_if_empty:nT  { V }
\cs_generate_variant:Nn \tl_if_empty:nF  { V }
%    \end{macrocode}
% \end{macro}
%
% \begin{macro}[pTF]{\tl_if_empty:o}
% \begin{macro}[EXP,aux]{\tl_if_empty_return:o}
%   The auxiliary function \cs{tl_if_empty_return:o} is for use
%   in conditionals on token lists, which mostly reduce to testing
%   if a given token list is empty after applying a simple function
%   to it.
%   The test for emptiness is based on \cs{tl_if_empty:n(TF)}, but
%   the expansion is hard-coded for efficiency, as this auxiliary
%   function is used in many places.
%   Note that this works because \cs{tl_to_str:n} expands tokens
%   that follow until reading a catcode $1$ (begin-group) token.
%    \begin{macrocode}
\cs_new:Npn \tl_if_empty_return:o #1
  {
    \exp_after:wN \if_meaning:w \exp_after:wN \q_nil
      \tl_to_str:n \exp_after:wN {#1} \q_nil
      \prg_return_true:
    \else:
      \prg_return_false:
    \fi:
  }
\prg_new_conditional:Npnn \tl_if_empty:o #1 { p , TF , T , F }
  { \tl_if_empty_return:o {#1} }
%    \end{macrocode}
% \end{macro}
% \end{macro}
%
% \begin{macro}[pTF]{\tl_if_eq:NN, \tl_if_eq:Nc, \tl_if_eq:cN, \tl_if_eq:cc}
%   Returns \cs{c_true_bool} if and only if the two token list variables are
%   equal.
%    \begin{macrocode}
\prg_new_conditional:Npnn \tl_if_eq:NN #1#2 { p , T , F , TF }
  {
    \if_meaning:w #1 #2
      \prg_return_true:
    \else:
      \prg_return_false:
    \fi:
  }
\cs_generate_variant:Nn \tl_if_eq_p:NN { Nc , c , cc }
\cs_generate_variant:Nn \tl_if_eq:NNTF { Nc , c , cc }
\cs_generate_variant:Nn \tl_if_eq:NNT  { Nc , c , cc }
\cs_generate_variant:Nn \tl_if_eq:NNF  { Nc , c , cc }
%    \end{macrocode}
% \end{macro}
%
% \begin{macro}[TF]{\tl_if_eq:nn}
% \begin{variable}{\l_tl_internal_a_tl, \l_tl_internal_b_tl}
%   A simple store and compare routine.
%    \begin{macrocode}
\prg_new_protected_conditional:Npnn \tl_if_eq:nn #1#2 { T , F ,  TF }
  {
    \group_begin:
      \tl_set:Nn \l_tl_internal_a_tl {#1}
      \tl_set:Nn \l_tl_internal_b_tl {#2}
      \if_meaning:w \l_tl_internal_a_tl \l_tl_internal_b_tl
        \group_end:
        \prg_return_true:
      \else:
        \group_end:
        \prg_return_false:
      \fi:
  }
\tl_new:N \l_tl_internal_a_tl
\tl_new:N \l_tl_internal_b_tl
%    \end{macrocode}
% \end{variable}
% \end{macro}
%
% \begin{macro}[TF]{\tl_if_in:Nn, \tl_if_in:cn}
%   See \cs{tl_if_in:nn(TF)} for further comments. Here we simply
%   expand the token list variable and pass it to \cs{tl_if_in:nn(TF)}.
%    \begin{macrocode}
\cs_new_protected_nopar:Npn \tl_if_in:NnT  { \exp_args:No \tl_if_in:nnT  }
\cs_new_protected_nopar:Npn \tl_if_in:NnF  { \exp_args:No \tl_if_in:nnF  }
\cs_new_protected_nopar:Npn \tl_if_in:NnTF { \exp_args:No \tl_if_in:nnTF }
\cs_generate_variant:Nn \tl_if_in:NnT { c }
\cs_generate_variant:Nn \tl_if_in:NnF  { c }
\cs_generate_variant:Nn \tl_if_in:NnTF { c }
%    \end{macrocode}
% \end{macro}
%
% \begin{macro}[TF]{\tl_if_in:nn, \tl_if_in:Vn, \tl_if_in:on, \tl_if_in:no}
%   Once more, the test relies on \cs{tl_to_str:n} for robustness.
%   The function \cs{tl_tmp:w} removes tokens until the first occurrence
%   of |#2|. If this does not appear in |#1|, then the final |#2| is removed,
%   leaving an empty token list. Otherwise some tokens remain, and the
%   test is false. See \cs{tl_if_empty:n(TF)} for details on
%   the emptyness test.
%
%   Special care is needed to treat correctly cases like
%   |\tl_if_in:nnTF {a state}{states}|, where |#1#2| contains |#2| before
%   the end. To cater for this case, we insert |{}{}| between the two token
%   lists. This marker may not appear in |#2| because of \TeX{} limitations
%   on what can delimit a parameter, hence we are safe. Using two brace
%   groups makes the test work also for empty arguments.
%    \begin{macrocode}
\prg_new_protected_conditional:Npnn \tl_if_in:nn #1#2 { T  , F , TF }
  {
    \cs_set:Npn \tl_tmp:w ##1 #2 { }
    \tl_if_empty:oTF { \tl_tmp:w #1 {} {} #2 }
      { \prg_return_false: } { \prg_return_true: }
  }
\cs_generate_variant:Nn \tl_if_in:nnT  { V , o , no }
\cs_generate_variant:Nn \tl_if_in:nnF  { V , o , no }
\cs_generate_variant:Nn \tl_if_in:nnTF { V , o , no }
%    \end{macrocode}
% \end{macro}
%
% \subsection{Mapping to token lists}
%
% \begin{macro}{\tl_map_function:nN}
% \begin{macro}{\tl_map_function:NN, \tl_map_function:cN}
% \begin{macro}[aux]{\tl_map_function_aux:Nn}
%   Expandable loop macro for token lists. These have the advantage of not
%   needing to test if the argument is empty, because if it is, the stop
%   marker will be read immediately and the loop terminated.
%    \begin{macrocode}
\cs_new:Npn \tl_map_function:nN #1#2
  {
    \tl_map_function_aux:Nn #2 #1
      \q_recursion_tail
    \prg_break_point:n { }
  }
\cs_new_nopar:Npn \tl_map_function:NN
  { \exp_args:No \tl_map_function:nN }
\cs_new:Npn \tl_map_function_aux:Nn #1#2
  {
    \quark_if_recursion_tail_break:n {#2}
    #1 {#2} \tl_map_function_aux:Nn #1
  }
\cs_generate_variant:Nn \tl_map_function:NN { c }
%    \end{macrocode}
% \end{macro}
% \end{macro}
% \end{macro}
%
% \begin{macro}{\tl_map_inline:nn}
% \begin{macro}{\tl_map_inline:Nn, \tl_map_inline:cn}
%   The inline functions are straight forward by now. We use a little
%   trick with the counter \cs{g_prg_map_int} to make
%   them nestable. We can also make use of \cs{tl_map_function_aux:Nn}
%   from before.
%    \begin{macrocode}
\cs_new_protected:Npn \tl_map_inline:nn #1#2
  {
    \int_gincr:N \g_prg_map_int
    \cs_gset:cpn { tl_map_inline_ \int_use:N \g_prg_map_int :n }
      ##1 {#2}
    \exp_args:Nc \tl_map_function_aux:Nn
      { tl_map_inline_ \int_use:N \g_prg_map_int :n }
      #1 \q_recursion_tail
    \prg_break_point:n { \int_gdecr:N \g_prg_map_int }
  }
\cs_new_protected:Npn \tl_map_inline:Nn
  { \exp_args:No \tl_map_inline:nn }
\cs_generate_variant:Nn \tl_map_inline:Nn { c }
%    \end{macrocode}
% \end{macro}
% \end{macro}
%
% \begin{macro}{\tl_map_variable:nNn}
% \begin{macro}{\tl_map_variable:NNn, \tl_map_variable:cNn}
% \begin{macro}[aux]{\tl_map_variable_aux:Nnn}
%   \cs{tl_map_variable:nNn} \meta{token list} \meta{temp} \meta{action}
%   assigns
%   \meta{temp} to each element and executes \meta{action}.
%    \begin{macrocode}
\cs_new_protected:Npn \tl_map_variable:nNn #1#2#3
  {
    \tl_map_variable_aux:Nnn #2 {#3} #1
      \q_recursion_tail
    \prg_break_point:n { }
  }
\cs_new_protected_nopar:Npn \tl_map_variable:NNn
  { \exp_args:No \tl_map_variable:nNn }
\cs_new_protected:Npn \tl_map_variable_aux:Nnn #1#2#3
  {
    \tl_set:Nn #1 {#3}
    \quark_if_recursion_tail_break:N #1
    \use:n {#2}
    \tl_map_variable_aux:Nnn #1 {#2}
  }
\cs_generate_variant:Nn \tl_map_variable:NNn { c }
%    \end{macrocode}
% \end{macro}
% \end{macro}
% \end{macro}
%
% \begin{macro}{\tl_map_break:}
% \begin{macro}{\tl_map_break:n}
%   The break statements are simply copies.
%    \begin{macrocode}
\cs_new_eq:NN \tl_map_break:  \prg_map_break:
\cs_new_eq:NN \tl_map_break:n \prg_map_break:n
%    \end{macrocode}
% \end{macro}
% \end{macro}
%
% \subsection{Using token lists}
%
% \begin{macro}{\tl_to_str:n}
%   Another name for a primitive.
%    \begin{macrocode}
\cs_new_eq:NN \tl_to_str:n \etex_detokenize:D
%    \end{macrocode}
% \end{macro}
%
% \begin{macro}{\tl_to_str:N, \tl_to_str:c}
%    These functions return the replacement text of a token list as a
%    string.
%    \begin{macrocode}
\cs_new:Npn \tl_to_str:N #1 { \etex_detokenize:D \exp_after:wN {#1} }
\cs_generate_variant:Nn \tl_to_str:N { c }
%    \end{macrocode}
% \end{macro}
%
% \begin{macro}{\tl_use:N, \tl_use:c}
% Token lists which are simply not defined will give a clear \TeX{}
% error here. No such luck for ones equal to \cs{scan_stop:} so
% instead a test is made and if there is an issue an error is forced.
%    \begin{macrocode}
\cs_new:Npn \tl_use:N #1
  {
    \tl_if_exist:NTF #1 {#1}
      { \msg_expandable_kernel_error:nnn { kernel } { bad-var } {#1} }
  }
\cs_generate_variant:Nn \tl_use:N { c }
%    \end{macrocode}
% \end{macro}
%
% \subsection{Working with the contents of token lists}
%
% \begin{macro}{\tl_count:n, \tl_count:V, \tl_count:o}
% \begin{macro}{\tl_count:N, \tl_count:c}
% \begin{macro}[aux]{\tl_count_aux:n}
%   Count number of elements within a token list or token list
%   variable. Brace groups within the list are read as a single
%   element. Spaces are ignored.
%   \cs{tl_count_aux:n} grabs the element and replaces it by |+1|.
%   The |0| to ensure it works on an empty list.
%    \begin{macrocode}
\cs_new:Npn \tl_count:n #1
  {
    \int_eval:n
      { 0 \tl_map_function:nN {#1} \tl_count_aux:n }
  }
\cs_new:Npn \tl_count:N #1
  {
    \int_eval:n
      { 0 \tl_map_function:NN #1 \tl_count_aux:n }
  }
\cs_new:Npn \tl_count_aux:n #1 { + \c_one }
\cs_generate_variant:Nn \tl_count:n { V , o }
\cs_generate_variant:Nn \tl_count:N { c }
%    \end{macrocode}
% \end{macro}
% \end{macro}
% \end{macro}
%
% \begin{macro}{\tl_reverse_items:n}
% \begin{macro}[aux]{\tl_reverse_items_aux:nwNwn}
% \begin{macro}[aux]{\tl_reverse_items_aux:wn}
%    Reversal of a token list is done by taking one item at a time
%    and putting it after \cs{q_stop}.
%    \begin{macrocode}
\cs_new:Npn \tl_reverse_items:n #1
  {
    \tl_reverse_items_aux:nwNwn #1 ?
      \q_mark \tl_reverse_items_aux:nwNwn
      \q_mark \tl_reverse_items_aux:wn
      \q_stop { }
  }
\cs_new:Npn \tl_reverse_items_aux:nwNwn #1 #2 \q_mark #3 #4 \q_stop #5
  {
    #3 #2
      \q_mark \tl_reverse_items_aux:nwNwn
      \q_mark \tl_reverse_items_aux:wn
      \q_stop { {#1} #5 }
  }
\cs_new:Npn \tl_reverse_items_aux:wn #1 \q_stop #2
  { \exp_not:o { \use_none:nn #2 } }
%    \end{macrocode}
% \end{macro}
% \end{macro}
% \end{macro}
%
% \begin{macro}{\tl_trim_spaces:n}
% \begin{macro}
%   {
%     \tl_trim_spaces:N, \tl_trim_spaces:c,
%     \tl_gtrim_spaces:N, \tl_gtrim_spaces:c
%   }
% \begin{macro}[aux]
%   {
%     \tl_trim_spaces_aux_i:w, \tl_trim_spaces_aux_ii:w
%     \tl_trim_spaces_aux_iii:w, \tl_trim_spaces_aux_iv:w
%   }
%   Trimming spaces from around the input is done using delimited
%   arguments and quarks, and to get spaces at odd places in the
%   definitions, we nest those in \cs{tl_tmp:w}, which then receives
%   a single space as its argument: |#1| is \verb*+ +.
%   Removing leading spaces is done with \cs{tl_trim_spaces_aux_i:w},
%   which loops until \cs{q_mark}\verb*+ + matches the end of the token
%   list: then |##1| is the token list and |##3| is
%   \cs{tl_trim_spaces_aux_ii:w}. This hands the relevant tokens to the
%   loop \cs{tl_trim_spaces_aux_iii:w}, responsible for trimming
%   trailing spaces. The end is reached when \verb*+ + \cs{q_nil}
%   matches the one present in the definition of \cs{tl_trim_spacs:n}.
%   Then \cs{tl_trim_spaces_aux_iv:w} puts the token list into a group,
%   as the argument of the initial \tn{unexpanded}.
%   The \tn{unexpanded} here is used so that space trimming will
%   behave correctly within an \texttt{x}-type expansion.
%
%   Some of the auxiliaries used in this code are also used
%   in the \pkg{l3clist} module. Change with care.
%    \begin{macrocode}
\cs_set:Npn \tl_tmp:w #1
  {
    \cs_new:Npn \tl_trim_spaces:n ##1
      {
        \etex_unexpanded:D
          \tl_trim_spaces_aux_i:w
          \q_mark
          ##1
          \q_nil
          \q_mark #1 { }
          \q_mark \tl_trim_spaces_aux_ii:w
          \tl_trim_spaces_aux_iii:w
          #1 \q_nil
          \tl_trim_spaces_aux_iv:w
          \q_stop
      }
    \cs_new:Npn \tl_trim_spaces_aux_i:w ##1 \q_mark #1 ##2 \q_mark ##3
      {
        ##3
        \tl_trim_spaces_aux_i:w
        \q_mark
        ##2
        \q_mark #1 {##1}
      }
    \cs_new:Npn \tl_trim_spaces_aux_ii:w ##1 \q_mark \q_mark ##2
      {
        \tl_trim_spaces_aux_iii:w
        ##2
      }
    \cs_new:Npn \tl_trim_spaces_aux_iii:w ##1 #1 \q_nil ##2
      {
        ##2
        ##1 \q_nil
        \tl_trim_spaces_aux_iii:w
      }
    \cs_new:Npn \tl_trim_spaces_aux_iv:w ##1 \q_nil ##2 \q_stop
      { \exp_after:wN { \use_none:n ##1 } }
  }
\tl_tmp:w { ~ }
\cs_new_protected:Npn \tl_trim_spaces:N #1
  { \tl_set:Nx #1 { \exp_after:wN \tl_trim_spaces:n \exp_after:wN {#1} } }
\cs_new_protected:Npn \tl_gtrim_spaces:N #1
  { \tl_gset:Nx #1 { \exp_after:wN \tl_trim_spaces:n \exp_after:wN {#1} } }
\cs_generate_variant:Nn \tl_trim_spaces:N  { c }
\cs_generate_variant:Nn \tl_gtrim_spaces:N { c }
%    \end{macrocode}
% \end{macro}
% \end{macro}
% \end{macro}
% 
% \subsection{Token by token changes}
% 
% \begin{variable}{\q_tl_act_mark,\q_tl_act_stop}
%   The \cs{tl_act} functions may be applied to any token list.
%   Hence, we use two private quarks, to allow any token, even quarks,
%   in the token list.^^A in particular critical for future \::e.
%   Only \cs{q_tl_act_mark} and \cs{q_tl_act_stop} may not appear
%   in the token lists manipulated by \cs{tl_act} functions. The quarks
%   are effectively defined in \pkg{l3quark}.
% \end{variable}
% 
% \begin{macro}[EXP,int]{\tl_act:NNNnn}
% \begin{macro}[EXP,aux]{\tl_act_output:n, \tl_act_reverse_output:n}
% \begin{macro}[EXP,aux]{\tl_act_loop:w}
% \begin{macro}[EXP,aux]{\tl_act_normal:NwnNNN}
% \begin{macro}[EXP,aux]{\tl_act_group:nwnNNN}
% \begin{macro}[EXP,aux]{\tl_act_space:wwnNNN}
% \begin{macro}[EXP,aux]{\tl_act_end:w}
%   To help control the expansion, \cs{tl_act:NNNnn} should always
%   be proceeded by \tn{romannumeral} and ends by producing \cs{c_zero}
%   once the result has been obtained. Then loop over tokens,
%   groups, and spaces in |#5|. The marker \cs{q_tl_act_mark}
%   is used both to avoid losing outer braces and to detect the
%   end of the token list more easily. The result is stored
%   as an argument for the dummy function \cs{tl_act_result:n}.
%    \begin{macrocode}
\cs_new:Npn \tl_act:NNNnn #1#2#3#4#5
  {
    \group_align_safe_begin:
    \tl_act_loop:w #5 \q_tl_act_mark \q_tl_act_stop
    {#4} #1 #2 #3
    \tl_act_result:n { }
  }
%    \end{macrocode}
%   In the loop, we check how the token list begins and act
%   accordingly. In the \enquote{normal} case, we may have
%   reached \cs{q_tl_act_mark}, the end of the list. Then
%   leave \cs{c_zero} and the result in the input stream,
%   to terminate the expansion of \tn{romannumeral}.
%   Otherwise, apply the relevant function to the
%   \enquote{arguments}, |#3|
%   and to the head of the token list. Then repeat the loop.
%   The scheme is the same if the token list starts with a
%   group or with a space. Some extra work is needed to
%   make \cs{tl_act_space:wwnNNN} gobble the space.
%    \begin{macrocode}
\cs_new:Npn \tl_act_loop:w #1 \q_tl_act_stop
  {
    \tl_if_head_N_type:nTF {#1}
      { \tl_act_normal:NwnNNN }
      {
        \tl_if_head_group:nTF {#1}
          { \tl_act_group:nwnNNN }
          { \tl_act_space:wwnNNN }
      }
    #1 \q_tl_act_stop
  }
\cs_new:Npn \tl_act_normal:NwnNNN #1 #2 \q_tl_act_stop #3#4
  {
    \if_meaning:w \q_tl_act_mark #1
      \exp_after:wN \tl_act_end:wn
    \fi:
    #4 {#3} #1
    \tl_act_loop:w #2 \q_tl_act_stop
    {#3} #4
  }
\cs_new:Npn \tl_act_end:wn #1 \tl_act_result:n #2
  { \group_align_safe_end: \c_zero #2 }
\cs_new:Npn \tl_act_group:nwnNNN #1 #2 \q_tl_act_stop #3#4#5
  {
    #5 {#3} {#1}
    \tl_act_loop:w #2 \q_tl_act_stop
    {#3} #4 #5
  }
\exp_last_unbraced:NNo
  \cs_new:Npn \tl_act_space:wwnNNN \c_space_tl #1 \q_tl_act_stop #2#3#4#5
  {
    #5 {#2}
    \tl_act_loop:w #1 \q_tl_act_stop
    {#2} #3 #4 #5
  }
%    \end{macrocode}
%   Typically, the output is done to the right of what was already output,
%   using \cs{tl_act_output:n}, but for the \cs{tl_act_reverse} functions,
%   it should be done to the left.
%    \begin{macrocode}
\cs_new:Npn \tl_act_output:n #1 #2 \tl_act_result:n #3
  { #2 \tl_act_result:n { #3 #1 } }
\cs_new:Npn \tl_act_reverse_output:n #1 #2 \tl_act_result:n #3
  { #2 \tl_act_result:n { #1 #3 } }
%    \end{macrocode}
% \end{macro}
% \end{macro}
% \end{macro}
% \end{macro}
% \end{macro}
% \end{macro}
% \end{macro}
%
% \begin{macro}[EXP]{\tl_reverse:n, \tl_reverse:o, \tl_reverse:V}
% \begin{macro}[EXP,aux]{\tl_reverse_normal:nN}
% \begin{macro}[EXP,aux]{\tl_reverse_group_preserve:nn}
% \begin{macro}[EXP,aux]{\tl_reverse_space:n}
%   The goal here is to reverse without losing spaces nor braces.
%   This is done using the general internal function \cs{tl_act:NNNnn}.
%   Spaces and \enquote{normal} tokens are output on the left of the current
%   output. Grouped tokens are output to the left but without any reversal
%   within the group. All of the internal functions here drop one argument:
%   this is needed by \cs{tl_act:NNNnn} when changing case (to record
%   which direction the change is in), but not when reversing the tokens.
%    \begin{macrocode}
\cs_new:Npn \tl_reverse:n #1
  {
    \etex_unexpanded:D \exp_after:wN
      {
        \tex_romannumeral:D
        \tl_act:NNNnn
          \tl_reverse_normal:nN
          \tl_reverse_group_preserve:nn
          \tl_reverse_space:n
          { }
          {#1}
      }
  }
\cs_generate_variant:Nn \tl_reverse:n { o , V }
\cs_new:Npn \tl_reverse_normal:nN #1#2
  { \tl_act_reverse_output:n {#2} }
\cs_new:Npn \tl_reverse_group_preserve:nn #1#2
  { \tl_act_reverse_output:n { {#2} } }
\cs_new:Npn \tl_reverse_space:n #1
  { \tl_act_reverse_output:n { ~ } }
%    \end{macrocode}
% \end{macro}
% \end{macro}
% \end{macro}
% \end{macro}
%
% \begin{macro}{\tl_reverse:N, \tl_reverse:c, \tl_greverse:N, \tl_greverse:c}
%   This reverses the list, leaving \cs{exp_stop_f:} in front,
%   which stops the \texttt{f}-expansion.
%    \begin{macrocode}
\cs_new_protected:Npn \tl_reverse:N #1
  { \tl_set:Nx #1 { \exp_args:No \tl_reverse:n { #1 } } }
\cs_new_protected:Npn \tl_greverse:N #1
  { \tl_gset:Nx #1 { \exp_args:No \tl_reverse:n { #1 } } }
\cs_generate_variant:Nn \tl_reverse:N  { c }
\cs_generate_variant:Nn \tl_greverse:N { c }
%    \end{macrocode}
% \end{macro}
%
% \subsection{The first token from a token list}
%
% \begin{macro}{\tl_head:N, \tl_head:n, \tl_head:V, \tl_head:v, \tl_head:f}
% \begin{macro}{\tl_head:w}
% \begin{macro}{\tl_tail:N, \tl_tail:n, \tl_tail:V, \tl_tail:v, \tl_tail:f}
% \begin{macro}{\tl_tail:w}
%  These functions pick up either the head or the tail of a list. The
%  empty brace groups in \cs{tl_head:n} and \cs{tl_tail:n} ensure that
%  a blank argument gives an empty result. The result is returned within
%  the \tn{unexpanded} primitive.
%    \begin{macrocode}
\cs_new:Npn \tl_head:w #1#2 \q_stop {#1}
\cs_new:Npn \tl_tail:w #1#2 \q_stop {#2}
\cs_new:Npn \tl_head:n #1
  { \etex_unexpanded:D \exp_after:wN { \tl_head:w #1 { } \q_stop } }
\cs_new:Npn \tl_tail:n #1
  { \etex_unexpanded:D \tl_tail_aux:w #1 \q_mark { } \q_mark \q_stop }
\cs_new:Npn \tl_tail_aux:w #1 #2 \q_mark #3 \q_stop { {#2} }
\cs_new_nopar:Npn \tl_head:N { \exp_args:No \tl_head:n }
\cs_generate_variant:Nn \tl_head:n { V , v , f }
\cs_new_nopar:Npn \tl_tail:N { \exp_args:No \tl_tail:n }
\cs_generate_variant:Nn \tl_tail:n { V , v , f }
%    \end{macrocode}
% \end{macro}
% \end{macro}
% \end{macro}
% \end{macro}
%
% \begin{macro}{\str_head:n, \str_tail:n}
% \begin{macro}[aux]{\str_head_aux:w}
% \begin{macro}[aux]{\str_tail_aux:w}
%   After \cs{tl_to_str:n}, we have a list of character tokens,
%   all with category code 12, except the space, which has category
%   code 10. Directly using \cs{tl_head:w} would thus lose leading spaces.
%   Instead, we take an argument delimited by an explicit space, and
%   then only use \cs{tl_head:w}. If the string started with a
%   space, then the argument of \cs{str_head_aux:w} is empty, and
%   the function correctly returns a space character. Otherwise,
%   it returns the first token of |#1|, which is the first token
%   of the string. If the string is empty, we return an empty result.
%
%   To remove the first character of \cs{tl_to_str:n} |{#1}|,
%   we test it using \cs{if_charcode:w} \cs{scan_stop:},
%   always false for characters. If the argument was non-empty,
%   then \cs{str_tail_aux:w} returns everything until the first
%   \texttt{X} (with category code letter, no risk of confusing
%   with the user input). If the argument was empty, the first
%   \texttt{X} is taken by \cs{if_charcode:w}, and nothing
%   is returned. We use \texttt{X} as a \meta{marker}, rather than
%   a quark because the test \cs{if_charcode:w} \cs{scan_stop:}
%   \meta{marker} has to be false.
%    \begin{macrocode}
\cs_new:Npn \str_head:n #1
  {
    \exp_after:wN \str_head_aux:w
    \tl_to_str:n {#1}
    { { } } ~ \q_stop
  }
\cs_new:Npn \str_head_aux:w #1 ~ %
  { \tl_head:w #1 { ~ } }
\cs_new:Npn \str_tail:n #1
  {
    \exp_after:wN \str_tail_aux:w
    \reverse_if:N \if_charcode:w
        \scan_stop: \tl_to_str:n {#1} X X \q_stop
  }
\cs_new:Npn \str_tail_aux:w #1 X #2 \q_stop { \fi: #1 }
%    \end{macrocode}
% \end{macro}
% \end{macro}
% \end{macro}
%
% \begin{macro}[pTF]{\tl_if_head_eq_meaning:nN}
% \begin{macro}[pTF]{\tl_if_head_eq_charcode:nN}
% \begin{macro}[pTF]{\tl_if_head_eq_charcode:fN}
% \begin{macro}[pTF]{\tl_if_head_eq_catcode:nN}
%   Accessing the first token of a token list is tricky in two cases:
%   when it has category code $1$ (begin-group token), or when it is
%   an explicit space, with category code $10$ and character code $32$.
%
%   Forgetting temporarily about this issue we would use the
%   following test in \cs{tl_if_head_eq_charcode:nN}. Here,
%   an empty |#1| argument yields \cs{q_nil}, otherwise the
%   first token of the token list.
% \begin{verbatim}
% \if_charcode:w
%     \exp_after:wN \exp_not:N \tl_head:w #1 \q_nil \q_stop
%     \exp_not:N #2
% \end{verbatim}
%   The special cases are detected using \cs{tl_if_head_N_type:n}
%   (the extra |?| takes care of empty arguments).
%   In those cases, the first token is a character, and
%   since we only care about its character code, we can
%   use \cs{str_head:n} to access it (this works even if
%   it is a space character).
%    \begin{macrocode}
\prg_new_conditional:Npnn \tl_if_head_eq_charcode:nN #1#2 { p , T , F , TF }
  {
    \if_charcode:w
        \exp_not:N #2
        \tl_if_head_N_type:nTF { #1 ? }
          { \exp_after:wN \exp_not:N \tl_head:w #1 \q_nil \q_stop }
          { \str_head:n {#1} }
      \prg_return_true:
    \else:
      \prg_return_false:
    \fi:
  }
\cs_generate_variant:Nn \tl_if_head_eq_charcode_p:nN { f }
\cs_generate_variant:Nn \tl_if_head_eq_charcode:nNTF { f }
\cs_generate_variant:Nn \tl_if_head_eq_charcode:nNT  { f }
\cs_generate_variant:Nn \tl_if_head_eq_charcode:nNF  { f }
%    \end{macrocode}
%   For \cs{tl_if_head_eq_catcode:nN}, again we detect special
%   cases with a \cs{tl_if_head_N_type}. Then we need to test
%   if the first token is a begin-group token or an explicit
%   space token, and produce the relevant token, either
%   \cs{c_group_begin_token} or \cs{c_space_token}.
%    \begin{macrocode}
\prg_new_conditional:Npnn \tl_if_head_eq_catcode:nN #1 #2 { p , T , F , TF }
  {
    \if_catcode:w
        \exp_not:N #2
        \tl_if_head_N_type:nTF { #1 ? }
          { \exp_after:wN \exp_not:N \tl_head:w #1 \q_nil \q_stop }
          {
            \tl_if_head_group:nTF {#1}
              { \c_group_begin_token }
              { \c_space_token }
          }
      \prg_return_true:
    \else:
      \prg_return_false:
    \fi:
  }
%    \end{macrocode}
%   For \cs{tl_if_head_eq_meaning:nN}, again, detect special cases.
%   In the normal case, use \cs{tl_head:w}, with no \cs{exp_not:N}
%   this time, since \cs{if_meaning:w} causes no expansion.
%   In the special cases, we know that the first token is a character,
%   hence \cs{if_charcode:w} and \cs{if_catcode:w} together are enough.
%   We combine them in some order, hopefully faster than the reverse.
%    \begin{macrocode}
\prg_new_conditional:Npnn \tl_if_head_eq_meaning:nN #1#2 { p , T , F , TF }
  {
    \tl_if_head_N_type:nTF { #1 ? }
      { \tl_if_head_eq_meaning_aux_normal:nN }
      { \tl_if_head_eq_meaning_aux_special:nN }
    {#1} #2
  }
\cs_new:Npn \tl_if_head_eq_meaning_aux_normal:nN #1 #2
  {
    \exp_after:wN \if_meaning:w \tl_head:w #1 \q_nil \q_stop #2
      \prg_return_true:
    \else:
      \prg_return_false:
    \fi:
  }
\cs_new:Npn \tl_if_head_eq_meaning_aux_special:nN #1 #2
  {
    \if_charcode:w \str_head:n {#1} \exp_not:N #2
      \exp_after:wN \use:n
    \else:
      \prg_return_false:
      \exp_after:wN \use_none:n
    \fi:
    {
      \if_catcode:w \exp_not:N #2
                    \tl_if_head_group:nTF {#1}
                      { \c_group_begin_token }
                      { \c_space_token }
        \prg_return_true:
      \else:
        \prg_return_false:
      \fi:
    }
  }
%    \end{macrocode}
% \end{macro}
% \end{macro}
% \end{macro}
% \end{macro}
%
% \begin{macro}[pTF]{\tl_if_head_N_type:n}
%   The first token of a token list can be either an N-type argument,
%   a begin-group token (catcode 1), or an explicit space token
%   (catcode 10 and charcode 32). These two cases are characterized
%   by the fact that \cs{use:n} removes some tokens from |#1|, hence
%   changing its string representation (no token can have an empty
%   string representation). The extra brace group covers the case of
%   an empty argument, whose head is not \enquote{normal}.
%    \begin{macrocode}
\prg_new_conditional:Npnn \tl_if_head_N_type:n #1 { p , T , F , TF }
  {
    \str_if_eq_return:xx
      { \exp_not:o { \use:n #1 { } } }
      { \exp_not:n { #1 { } } }
  }
%    \end{macrocode}
% \end{macro}
%
% \begin{macro}[EXP,pTF]{\tl_if_head_group:n}
%   Pass the first token of |#1| through \cs{token_to_str:N},
%   then check for the brace balance. The extra \texttt{?}
%   caters for an empty argument.\footnote{Bruno: this could
%     be made faster, but we don't: if we hope to ever have
%     an e-type argument, we need all brace \enquote{tricks}
%     to happen in one step of expansion, keeping the token
%     list brace balanced at all times.}
%    \begin{macrocode}
\prg_new_conditional:Npnn \tl_if_head_group:n #1 { p , T , F , TF }
  {
    \if_catcode:w *
        \exp_after:wN \use_none:n
          \exp_after:wN {
            \exp_after:wN {
              \token_to_str:N #1 ?
            }
          }
          *
      \prg_return_false:
    \else:
      \prg_return_true:
    \fi:
  }
%    \end{macrocode}
% \end{macro}
%
% \begin{macro}[EXP,pTF]{\tl_if_head_space:n}
% \begin{macro}[EXP,aux]{\tl_if_head_space_aux:w}
%   If the first token of the token list is an explicit space, i.e.,
%   a character token with character code $32$ and category code $10$,
%   then this test will be \meta{true}. It is \meta{false} if the token
%   list is empty, if the first token is an implicit space token,
%   such as \cs{c_space_token}, or any token other than an explicit space.
%   The slightly convoluted approach with \tn{romannumeral} ensures that
%   each expansion step gives a balanced token list.
%    \begin{macrocode}
\prg_new_conditional:Npnn \tl_if_head_space:n #1 { p , T , F , TF }
  {
    \tex_romannumeral:D \if_false: { \fi:
      \tl_if_head_space_aux:w ? #1 ? ~ }
  }
\cs_new:Npn \tl_if_head_space_aux:w #1 ~
  {
    \tl_if_empty:oTF { \use_none:n #1 }
      { \exp_after:wN \c_zero \exp_after:wN \prg_return_true: }
      { \exp_after:wN \c_zero \exp_after:wN \prg_return_false: }
    \exp_after:wN \use_none:n \exp_after:wN { \if_false: } \fi:
  }
%    \end{macrocode}
% \end{macro}
% \end{macro}
%
% \subsection{Viewing token lists}
%
% \begin{macro}{\tl_show:N, \tl_show:c}
%   Showing token list variables is done directly: at the moment do not
%   worry if they are defined.
%    \begin{macrocode}
\cs_new_protected:Npn \tl_show:N #1 { \cs_show:N #1 }
\cs_generate_variant:Nn \tl_show:N { c }
%    \end{macrocode}
%\end{macro}
%
% \begin{macro}{\tl_show:n}
%   For literal token lists, life is easy.
%    \begin{macrocode}
\cs_new_eq:NN \tl_show:n \etex_showtokens:D
%    \end{macrocode}
%\end{macro}
%
% \subsection{Scratch token lists}
%
% \begin{variable}{\g_tmpa_tl, \g_tmpb_tl}
%    Global temporary token list variables.
%    They are supposed to be set and used immediately,
%    with no delay between the definition and the use because you
%    can't count on other macros not to redefine them from under you.
%    \begin{macrocode}
\tl_new:N \g_tmpa_tl
\tl_new:N \g_tmpb_tl
%    \end{macrocode}
% \end{variable}
%
% \begin{variable}{\l_tmpa_tl, \l_tmpb_tl}
%    These are local temporary token list variables. Be sure not to assume
%    that the value you put into them will survive for
%    long---see discussion above.
%    \begin{macrocode}
\tl_new:N \l_tmpa_tl
\tl_new:N \l_tmpb_tl
%    \end{macrocode}
% \end{variable}
%
% \subsection{Experimental functions}
%
% \begin{macro}[EXP]{\str_if_eq_return:xx}
%   It turns out that we often need to compare a token list
%   with the result of applying some function to it, and
%   return with \cs{prg_return_true/false:}. This test is
%   similar to \cs{str_if_eq:nnTF}, but hard-coded for speed.
%    \begin{macrocode}
\cs_new:Npn \str_if_eq_return:xx #1 #2
  {
    \if_int_compare:w \pdftex_strcmp:D {#1} {#2} = \c_zero
      \prg_return_true:
    \else:
      \prg_return_false:
    \fi:
  }
%    \end{macrocode}
% \end{macro}
%
% \begin{macro}[EXP,pTF]{\tl_if_single:N}
%   Expand the token list and feed it to \cs{tl_if_single:n}.
%    \begin{macrocode}
\cs_new:Npn \tl_if_single_p:N { \exp_args:No \tl_if_single_p:n }
\cs_new:Npn \tl_if_single:NT  { \exp_args:No \tl_if_single:nT  }
\cs_new:Npn \tl_if_single:NF  { \exp_args:No \tl_if_single:nF  }
\cs_new:Npn \tl_if_single:NTF { \exp_args:No \tl_if_single:nTF }
%    \end{macrocode}
% \end{macro}
%
% \begin{macro}[EXP,pTF]{\tl_if_single:n}
%   A token list has exactly one item if it is either a single
%   token surrounded by optional explicit spaces, or a single brace
%   group surrounded by optional explicit spaces. The naive
%   version of this test would do \cs{use_none:n} |#1|, and
%   test if the result is empty. However, this will fail when
%   the token list is empty. Furthermore, it does not allow optional
%   trailing spaces.
%    \begin{macrocode}
\prg_new_conditional:Npnn \tl_if_single:n #1 { p , T , F , TF }
  { \str_if_eq_return:xx { \exp_not:o { \use_none:nn #1 ?? } } {?} }
%    \end{macrocode}
% \end{macro}
%
% \begin{macro}[EXP,pTF]{\tl_if_single_token:n}
%   There are four cases: empty token list, token list starting with
%   a normal token, with a brace group, or with a space token.
%   If the token list starts with a normal token, remove it
%   and check for emptyness. Otherwise, compare with a single
%   space, only case where we have a single token.
%    \begin{macrocode}
\prg_new_conditional:Npnn \tl_if_single_token:n #1 { p , T , F , TF }
  {
    \tl_if_head_N_type:nTF {#1}
      { \str_if_eq_return:xx { \exp_not:o { \use_none:n #1 } } { } }
      { \str_if_eq_return:xx { \exp_not:n {#1} } { ~ } }
  }
%    \end{macrocode}
% \end{macro}
%
% \begin{macro}[EXP]{\tl_reverse_tokens:n}
% \begin{macro}[EXP,aux]{\tl_reverse_group:nn}
%   The same as \cs{tl_reverse:n} but with recursion within brace groups.
%    \begin{macrocode}
\cs_new:Npn \tl_reverse_tokens:n #1
  {
    \etex_unexpanded:D \exp_after:wN
      {
        \tex_romannumeral:D
        \tl_act:NNNnn
          \tl_reverse_normal:nN
          \tl_reverse_group:nn
          \tl_reverse_space:n
          { }
          {#1}
      }
  }
\cs_new:Npn \tl_reverse_group:nn #1
  {
    \tl_act_group_recurse:Nnn
      \tl_act_reverse_output:n
      { \tl_reverse_tokens:n }
  }
%    \end{macrocode}
% \end{macro}
% \begin{macro}[EXP,aux]{\tl_act_group_recurse:Nnn}
%   In many applications of \cs{tl_act:NNNnn}, we need to recursively
%   apply some transformation within brace groups, then output. In this
%   code, |#1| is the output function, |#2| is the transformation,
%   which should expand in two steps, and |#3| is the group.
%    \begin{macrocode}
\cs_new:Npn \tl_act_group_recurse:Nnn #1#2#3
  {
    \exp_args:Nf #1
      { \exp_after:wN \exp_after:wN \exp_after:wN { #2 {#3} } }
  }
%    \end{macrocode}
% \end{macro}
% \end{macro}
%
% \begin{macro}[EXP]{\tl_count_tokens:n}
% \begin{macro}[EXP,aux]{\tl_act_count_normal:nN,
%     \tl_act_count_group:nn,\tl_act_count_space:n}
%   The token coung is computed through an \cs{int_eval:n} construction.
%   Each \texttt{1+} is output to the \emph{left}, into the integer
%   expression, and the sum is ended by the \cs{c_zero} inserted by
%   \cs{tl_act_end:wn}. Somewhat a hack.
%    \begin{macrocode}
\cs_new:Npn \tl_count_tokens:n #1
  {
    \int_eval:n
      {
        \tl_act:NNNnn
          \tl_act_count_normal:nN
          \tl_act_count_group:nn
          \tl_act_count_space:n
          { }
          {#1}
      }
  }
\cs_new:Npn \tl_act_count_normal:nN #1 #2 { 1 + }
\cs_new:Npn \tl_act_count_space:n #1 { 1 + }
\cs_new:Npn \tl_act_count_group:nn #1 #2
  { 2 + \tl_count_tokens:n {#2} + }
%    \end{macrocode}
% \end{macro}
% \end{macro}
%
% \begin{variable}{\c_tl_act_uppercase_tl, \c_tl_act_lowercase_tl}
%   These constants contain the correspondance between lowercase
%   and uppercase letters, in the form |aAbBcC...| and |AaBbCc...|
%   respectively.
%    \begin{macrocode}
\tl_const:Nn \c_tl_act_uppercase_tl
  {
    aA bB cC dD eE fF gG hH iI jJ kK lL mM
    nN oO pP qQ rR sS tT uU vV wW xX yY zZ
  }
\tl_const:Nn \c_tl_act_lowercase_tl
  {
    Aa Bb Cc Dd Ee Ff Gg Hh Ii Jj Kk Ll Mm
    Nn Oo Pp Qq Rr Ss Tt Uu Vv Ww Xx Yy Zz
  }
%    \end{macrocode}
% \end{variable}
%
% \begin{macro}[EXP]{\tl_expandable_uppercase:n,\tl_expandable_lowercase:n}
% \begin{macro}[EXP,aux]{\tl_act_case_normal:nN,
%     \tl_act_case_group:nn,\tl_act_case_space:n}
%   The only difference between uppercasing and lowercasing is
%   the table of correspondance that is used. As for other
%   token list actions, we feed \cs{tl_act:NNNnn} three
%   functions, and this time, we use the \meta{parameters}
%   argument to carry which case-changing we are applying.
%   A space is simply output. A normal token is compared
%   to each letter in the alphabet using \cs{str_if_eq:nn}
%   tests, and converted if necessary to upper/lowercase,
%   before being output. For a group, we must perform the
%   conversion within the group (the \cs{exp_after:wN} trigger
%   \tn{romannumeral}, which expands fully to give the
%   converted group), then output.
%    \begin{macrocode}
\cs_new:Npn \tl_expandable_uppercase:n #1
  {
    \etex_unexpanded:D \exp_after:wN
      {
        \tex_romannumeral:D
          \tl_act_case_aux:nn { \c_tl_act_uppercase_tl } {#1}
      }
  }
\cs_new:Npn \tl_expandable_lowercase:n #1
  {
    \etex_unexpanded:D \exp_after:wN
      {
        \tex_romannumeral:D
          \tl_act_case_aux:nn { \c_tl_act_lowercase_tl } {#1}
      }
  }
\cs_new:Npn \tl_act_case_aux:nn
  {
    \tl_act:NNNnn
      \tl_act_case_normal:nN
      \tl_act_case_group:nn
      \tl_act_case_space:n
  }
\cs_new:Npn \tl_act_case_space:n #1 { \tl_act_output:n {~} }
\cs_new:Npn \tl_act_case_normal:nN #1 #2
  {
    \exp_args:Nf \tl_act_output:n
      {
        \exp_args:NNo \prg_case_str:nnn #2 {#1}
          { \exp_stop_f: #2 }
      }
  }
\cs_new:Npn \tl_act_case_group:nn #1 #2
  {
    \exp_after:wN \tl_act_output:n \exp_after:wN
      { \exp_after:wN { \tex_romannumeral:D \tl_act_case_aux:nn {#1} {#2} } }
  }
%    \end{macrocode}
% \end{macro}
% \end{macro}
%
% \begin{macro}{\tl_item:nn, \tl_item:Nn, \tl_item:cn}
% \begin{macro}[aux]{\tl_item_aux:nn}
%   The idea here is to find the offset of the item from the left, then use
%   a loop to grab the correct item. If the resulting offset is too large,
%   then \cs{quark_if_recursion_tail_stop:n} terminates the loop, and returns
%   nothing at all.
%    \begin{macrocode}
\cs_new:Npn \tl_item:nn #1#2
  {
    \exp_args:Nf \tl_item_aux:nn
      {
        \int_eval:n
          {
            \int_compare:nNnT {#2} < \c_zero
              { \tl_count:n {#1} + }
            #2
          }
      }
    #1
    \q_recursion_tail
    \prg_break_point:n { }
  }
\cs_new:Npn \tl_item_aux:nn #1#2
  {
    \quark_if_recursion_tail_break:n {#2}
    \int_compare:nNnTF {#1} = \c_zero
      { \tl_map_break:n { \exp_not:n {#2} } }
      { \exp_args:Nf \tl_item_aux:nn { \int_eval:n { #1 - 1 } } }
  }
\cs_new_nopar:Npn \tl_item:Nn { \exp_args:No \tl_item:nn }
\cs_generate_variant:Nn \tl_item:Nn { c }
%    \end{macrocode}
% \end{macro}
% \end{macro}
%
% \begin{macro}[pTF]{\tl_if_empty:x}
%   We can test expandably the emptyness of an expanded token list
%   thanks to the primitive \tn{pdfstrcmp} which expands its argument:
%   a token list is empty if and only if its string representation is
%   empty.
%    \begin{macrocode}
\prg_new_conditional:Npnn \tl_if_empty:x #1 { p , T , F , TF }
  { \str_if_eq_return:xx { } {#1} }
%    \end{macrocode}
% \end{macro}
%
% \subsection{Deprecated functions}
%
% \begin{macro}{\tl_new:Nn, \tl_new:cn, \tl_new:Nx}
%   Use either \cs{tl_const:Nn} or \cs{tl_new:N}.
%    \begin{macrocode}
%<*deprecated>
\cs_new_protected:Npn \tl_new:Nn #1#2
  {
    \tl_new:N #1
    \tl_gset:Nn #1 {#2}
  }
\cs_generate_variant:Nn \tl_new:Nn { c }
\cs_generate_variant:Nn \tl_new:Nn { Nx }
%</deprecated>
%    \end{macrocode}
% \end{macro}
%
% \begin{macro}{\tl_gset:Nc}
% \begin{macro}{\tl_set:Nc}
%   This was useful once, but nowadays does not make much sense.
%    \begin{macrocode}
%<*deprecated>
\cs_new_protected_nopar:Npn \tl_gset:Nc
  { \tex_global:D \tl_set:Nc }
\cs_new_protected:Npn \tl_set:Nc #1#2
  { \tl_set:No #1 { \cs:w #2 \cs_end: } }
%</deprecated>
%    \end{macrocode}
% \end{macro}
% \end{macro}
%
% \begin{macro}{\tl_replace_in:Nnn, \tl_replace_in:cnn}
% \begin{macro}{\tl_greplace_in:Nnn, \tl_greplace_in:cnn}
% \begin{macro}{\tl_replace_all_in:Nnn, \tl_replace_all_in:cnn}
% \begin{macro}{\tl_greplace_all_in:Nnn, \tl_greplace_all_in:cnn}
%   These are renamed.
%    \begin{macrocode}
%<*deprecated>
\cs_new_eq:NN \tl_replace_in:Nnn  \tl_replace_once:Nnn
\cs_new_eq:NN \tl_replace_in:cnn  \tl_replace_once:cnn
\cs_new_eq:NN \tl_greplace_in:Nnn \tl_greplace_once:Nnn
\cs_new_eq:NN \tl_greplace_in:cnn \tl_greplace_once:cnn
\cs_new_eq:NN \tl_replace_all_in:Nnn  \tl_replace_all:Nnn
\cs_new_eq:NN \tl_replace_all_in:cnn  \tl_replace_all:cnn
\cs_new_eq:NN \tl_greplace_all_in:Nnn \tl_greplace_all:Nnn
\cs_new_eq:NN \tl_greplace_all_in:cnn \tl_greplace_all:cnn
%</deprecated>
%    \end{macrocode}
% \end{macro}
% \end{macro}
% \end{macro}
% \end{macro}
%
% \begin{macro}{\tl_remove_in:Nn, \tl_remove_in:cn}
% \begin{macro}{\tl_gremove_in:Nn, \tl_gremove_in:cn}
% \begin{macro}{\tl_remove_all_in:Nn, \tl_remove_all_in:cn}
% \begin{macro}{\tl_gremove_all_in:Nn, \tl_gremove_all_in:cn}
%   Also renamed.
%    \begin{macrocode}
%<*deprecated>
\cs_new_eq:NN \tl_remove_in:Nn  \tl_remove_once:Nn
\cs_new_eq:NN \tl_remove_in:cn  \tl_remove_once:cn
\cs_new_eq:NN \tl_gremove_in:Nn \tl_gremove_once:Nn
\cs_new_eq:NN \tl_gremove_in:cn \tl_gremove_once:cn
\cs_new_eq:NN \tl_remove_all_in:Nn  \tl_remove_all:Nn
\cs_new_eq:NN \tl_remove_all_in:cn  \tl_remove_all:cn
\cs_new_eq:NN \tl_gremove_all_in:Nn \tl_gremove_all:Nn
\cs_new_eq:NN \tl_gremove_all_in:cn \tl_gremove_all:cn
%</deprecated>
%    \end{macrocode}
% \end{macro}
% \end{macro}
% \end{macro}
% \end{macro}
%
% \begin{macro}{\tl_elt_count:n, \tl_elt_count:V, \tl_elt_count:o}
% \begin{macro}{\tl_elt_count:N, \tl_elt_count:c}
% Another renaming job.
%    \begin{macrocode}
%<*deprecated>
\cs_new_eq:NN \tl_elt_count:n \tl_count:n
\cs_new_eq:NN \tl_elt_count:V \tl_count:V
\cs_new_eq:NN \tl_elt_count:o \tl_count:o
\cs_new_eq:NN \tl_elt_count:N \tl_count:N
\cs_new_eq:NN \tl_elt_count:c \tl_count:c
%</deprecated>
%    \end{macrocode}
% \end{macro}
% \end{macro}
%
% \begin{macro}{\tl_head_i:n}
% \begin{macro}{\tl_head_i:w}
% \begin{macro}{\tl_head_iii:n}
% \begin{macro}{\tl_head_iii:f}
% \begin{macro}{\tl_head_iii:w}
%  Two renames, and a few that are rather too specialised.
%    \begin{macrocode}
%<*deprecated>
\cs_new_eq:NN \tl_head_i:n \tl_head:n
\cs_new_eq:NN \tl_head_i:w \tl_head:w
\cs_new:Npn \tl_head_iii:n #1 { \tl_head_iii:w #1 \q_stop }
\cs_generate_variant:Nn \tl_head_iii:n { f }
\cs_new:Npn \tl_head_iii:w #1#2#3#4 \q_stop {#1#2#3}
%</deprecated>
%    \end{macrocode}
% \end{macro}
% \end{macro}
% \end{macro}
% \end{macro}
% \end{macro}
% 
% Deprecated on 2012-05-13 for removal by 2012-08-31.
% 
% \begin{macro}{\tl_length_tokens:n}
%    \begin{macrocode}
\cs_new_eq:NN \tl_length_tokens:n \tl_count_tokens:n
%    \end{macrocode}
% \end{macro}
% 
% Deprecated 2012-05-13 for removal by 2012-11-31.
% 
% \begin{macro}
%   {\tl_length:N, \tl_length:c, \tl_length:n, \tl_length:V, \tl_length:o}
%   Renames.
%    \begin{macrocode}
\cs_new_eq:NN \tl_length:N \tl_count:N
\cs_new_eq:NN \tl_length:c \tl_count:c
\cs_new_eq:NN \tl_length:n \tl_count:n
\cs_new_eq:NN \tl_length:V \tl_count:V
\cs_new_eq:NN \tl_length:o \tl_count:o
%    \end{macrocode}
% \end{macro}
%
%    \begin{macrocode}
%</initex|package>
%    \end{macrocode}
%
% \end{implementation}
%
% \PrintIndex
