% \iffalse meta-comment
%
%% File: l3skip.dtx
%
% Copyright (C) 2004-2011 Frank Mittelbach, The LaTeX Project
%           (C) 2012-2025 The LaTeX Project
%
% It may be distributed and/or modified under the conditions of the
% LaTeX Project Public License (LPPL), either version 1.3c of this
% license or (at your option) any later version.  The latest version
% of this license is in the file
%
%    https://www.latex-project.org/lppl.txt
%
% This file is part of the "l3kernel bundle" (The Work in LPPL)
% and all files in that bundle must be distributed together.
%
% -----------------------------------------------------------------------
%
% The development version of the bundle can be found at
%
%    https://github.com/latex3/latex3
%
% for those people who are interested.
%
%<*driver>
\documentclass[full,kernel]{l3doc}
\begin{document}
  \DocInput{\jobname.dtx}
\end{document}
%</driver>
% \fi
%
% \title{^^A
%   The \pkg{l3skip} module\\ Dimensions and skips^^A
% }
%
% \author{^^A
%  The \LaTeX{} Project\thanks
%    {^^A
%      E-mail:
%        \href{mailto:latex-team@latex-project.org}
%          {latex-team@latex-project.org}^^A
%    }^^A
% }
%
% \date{Released 2025-09-02}
%
% \maketitle
%
% \begin{documentation}
%
% \LaTeX3 provides two general length variables: \texttt{dim} and
% \texttt{skip}. Lengths stored as \texttt{dim} variables have a fixed
% length, whereas \texttt{skip} lengths have a rubber (stretch/shrink)
% component. In addition, the \texttt{muskip} type is available for
% use in math mode: this is a special form of \texttt{skip} where the
% lengths involved are determined by the current math font (in
% \texttt{mu)}. There are common features in the creation and setting of
% length variables, but for clarity the functions are grouped by variable
% type.
%
% Many functions take
% \emph{dimension expressions} (\enquote{\meta{dim expr}}) or
% \emph{skip expressions} (\enquote{\meta{skip expr}}) as arguments.
%
%
% \section{Creating and initializing \texttt{dim} variables}
%
% \begin{function}{\dim_new:N, \dim_new:c}
%   \begin{syntax}
%     \cs{dim_new:N} \meta{dimension}
%   \end{syntax}
%   Creates a new \meta{dimension} or raises an error if the name is
%   already taken. The declaration is global. The \meta{dimension}
%   is initially equal to $0$\,pt.
% \end{function}
%
% \begin{function}[added = 2012-03-05]{\dim_const:Nn, \dim_const:cn}
%   \begin{syntax}
%     \cs{dim_const:Nn} \meta{dimension} \Arg{dim expr}
%   \end{syntax}
%   Creates a new constant \meta{dimension} or raises an error if the
%   name is already taken. The value of the \meta{dimension} is set
%   globally to the \meta{dim expr}.
% \end{function}
%
% \begin{function}{\dim_zero:N, \dim_zero:c, \dim_gzero:N, \dim_gzero:c}
%   \begin{syntax}
%     \cs{dim_zero:N} \meta{dimension}
%   \end{syntax}
%   Sets \meta{dimension} to $0$\,pt.
% \end{function}
%
% \begin{function}[added = 2012-01-07]
%   {\dim_zero_new:N, \dim_zero_new:c, \dim_gzero_new:N, \dim_gzero_new:c}
%   \begin{syntax}
%     \cs{dim_zero_new:N} \meta{dimension}
%   \end{syntax}
%   Ensures that the \meta{dimension} exists globally by applying
%   \cs{dim_new:N} if necessary, then applies
%   \cs[index=dim_zero:N]{dim_(g)zero:N} to leave
%   the \meta{dimension} set to zero.
% \end{function}
%
% \begin{function}[EXP, pTF, added=2012-03-03]{\dim_if_exist:N, \dim_if_exist:c}
%   \begin{syntax}
%     \cs{dim_if_exist_p:N} \meta{dimension}
%     \cs{dim_if_exist:NTF} \meta{dimension} \Arg{true code} \Arg{false code}
%   \end{syntax}
%   Tests whether the \meta{dimension} is currently defined.  This does
%   not check that the \meta{dimension} really is a dimension variable.
% \end{function}
%
% \section{Setting \texttt{dim} variables}
%
% \begin{function}[updated = 2011-10-22]
%   {\dim_add:Nn, \dim_add:cn, \dim_gadd:Nn, \dim_gadd:cn}
%   \begin{syntax}
%     \cs{dim_add:Nn} \meta{dimension} \Arg{dim expr}
%   \end{syntax}
%   Adds the result of the \meta{dim expr} to the current
%   content of the \meta{dimension}.
% \end{function}
%
% \begin{function}[updated = 2011-10-22]
%   {
%     \dim_set:Nn, \dim_set:cn, \dim_set:NV, \dim_set:cV,
%     \dim_gset:Nn, \dim_gset:cn, \dim_gset:NV, \dim_gset:cV
%   }
%   \begin{syntax}
%     \cs{dim_set:Nn} \meta{dimension} \Arg{dim expr}
%   \end{syntax}
%   Sets \meta{dimension} to the value of \meta{dim expr}, which
%   must evaluate to a length with units.
% \end{function}
%
% \begin{function}
%   {
%     \dim_set_eq:NN,  \dim_set_eq:cN,  \dim_set_eq:Nc,  \dim_set_eq:cc,
%     \dim_gset_eq:NN, \dim_gset_eq:cN, \dim_gset_eq:Nc, \dim_gset_eq:cc
%   }
%   \begin{syntax}
%     \cs{dim_set_eq:NN} \meta{dimension_1} \meta{dimension_2}
%   \end{syntax}
%   Sets the content of \meta{dimension_1} equal to that of
%   \meta{dimension_2}.
% \end{function}
%
% \begin{function}[updated = 2011-10-22]
%   {\dim_sub:Nn, \dim_sub:cn, \dim_gsub:Nn, \dim_gsub:cn}
%   \begin{syntax}
%     \cs{dim_sub:Nn} \meta{dimension} \Arg{dim expr}
%   \end{syntax}
%   Subtracts the result of the \meta{dim expr} from the
%   current content of the \meta{dimension}.
% \end{function}
%
% \section{Utilities for dimension calculations}
%
% \begin{function}[updated = 2012-09-26, EXP]{\dim_abs:n}
%   \begin{syntax}
%     \cs{dim_abs:n} \Arg{dim expr}
%   \end{syntax}
%   Converts the \meta{dim expr} to its absolute value, leaving the result
%   in the input stream as a \meta{dimension denotation}.
% \end{function}
%
% \begin{function}[added = 2012-09-09, updated = 2012-09-26, EXP]
%   {\dim_max:nn, \dim_min:nn}
%   \begin{syntax}
%     \cs{dim_max:nn} \Arg{dim expr_1} \Arg{dim expr_2}
%     \cs{dim_min:nn} \Arg{dim expr_1} \Arg{dim expr_2}
%   \end{syntax}
%   Evaluates the two \meta{dim exprs} and leaves either the
%   maximum or minimum value in the input stream as appropriate, as a
%   \meta{dimension denotation}.
% \end{function}
%
% \begin{function}[updated = 2011-10-22, rEXP]{\dim_ratio:nn}
%   \begin{syntax}
%     \cs{dim_ratio:nn} \Arg{dim expr_1} \Arg{dim expr_2}
%   \end{syntax}
%   Parses the two \meta{dim exprs} and converts the ratio of
%   the two to a form suitable for use inside a \meta{dim expr}.
%   This ratio is then left in the input stream, allowing syntax such as
%   \begin{verbatim}
%     \dim_set:Nn \l_my_dim
%       { 10 pt * \dim_ratio:nn { 5 pt } { 10 pt } }
%   \end{verbatim}
%   The output of \cs{dim_ratio:nn} on full expansion is a ratio expression
%   between two integers, with all distances converted to scaled points.
%   Thus
%   \begin{verbatim}
%     \tl_set:Ne \l_my_tl { \dim_ratio:nn { 5 pt } { 10 pt } }
%     \tl_show:N \l_my_tl
%   \end{verbatim}
%   displays |327680/655360| on the terminal.
% \end{function}
%
% \section{Dimension expression conditionals}
%
% \begin{function}[EXP,pTF]{\dim_compare:nNn}
%   \begin{syntax}
%     \cs{dim_compare_p:nNn} \Arg{dim expr_1} \meta{relation} \Arg{dim expr_2} \\
%     \cs{dim_compare:nNnTF}
%     ~~\Arg{dim expr_1} \meta{relation} \Arg{dim expr_2}
%     ~~\Arg{true code} \Arg{false code}
%   \end{syntax}
%   This function first evaluates each of the \meta{dim exprs}
%   as described for \cs{dim_eval:n}. The two results are then
%   compared using the \meta{relation}:
%   \begin{center}
%     \begin{tabular}{ll}
%       Equal                 & |=| \\
%       Greater than          & |>| \\
%       Less than             & |<| \\
%     \end{tabular}
%   \end{center}
%   This function is less flexible than \cs{dim_compare:nTF} but around
%   $5$~times faster.
% \end{function}
%
% \begin{function}[updated = 2013-01-13, EXP, pTF]{\dim_compare:n}
%   \begin{syntax}
%     \cs{dim_compare_p:n} \\
%     ~~\{ \\
%     ~~~~\meta{dim expr_1} \meta{relation_1} \\
%     ~~~~\ldots{} \\
%     ~~~~\meta{dim expr_N} \meta{relation_N} \\
%     ~~~~\meta{dim expr_{N+1}} \\
%     ~~\} \\
%     \cs{dim_compare:nTF}
%     ~~\{ \\
%     ~~~~\meta{dim expr_1} \meta{relation_1} \\
%     ~~~~\ldots{} \\
%     ~~~~\meta{dim expr_N} \meta{relation_N} \\
%     ~~~~\meta{dim expr_{N+1}} \\
%     ~~\} \\
%     ~~\Arg{true code} \Arg{false code}
%   \end{syntax}
%   This function evaluates the \meta{dim exprs} as
%   described for \cs{dim_eval:n} and compares consecutive result using
%   the corresponding \meta{relation}, namely it compares
%   \meta{dim expr_1} and \meta{dim expr_2} using the \meta{relation_1},
%   then \meta{dim expr_2} and \meta{dim expr_3} using the
%   \meta{relation_2}, until finally comparing \meta{dim expr_N} and
%   \meta{dim expr_{N+1}} using the \meta{relation_N}.  The test yields
%   \texttt{true} if all comparisons are \texttt{true}.  Each
%   \meta{dim expr} is evaluated only once, and the
%   evaluation is lazy, in the sense that if one comparison is
%   \texttt{false}, then no other \meta{dim expr} is
%   evaluated and no other comparison is performed.  The
%   \meta{relations} can be any of the following:
%   \begin{center}
%     \begin{tabular}{ll}
%       Equal                    & |=| or |==| \\
%       Greater than or equal to & |>=|        \\
%       Greater than             & |>|         \\
%       Less than or equal to    & |<=|        \\
%       Less than                & |<|         \\
%       Not equal                & |!=|        \\
%     \end{tabular}
%   \end{center}
%   This function is more flexible than \cs{dim_compare:nNnTF} but
%   around $5$~times slower.
% \end{function}
%
% \begin{function}[added = 2013-07-24, EXP, noTF]{\dim_case:nn}
%   \begin{syntax}
%     \cs{dim_case:nnTF} \Arg{test dim expr} \\
%     ~~|{| \\
%     ~~~~\Arg{dim expr case_1} \Arg{code case_1} \\
%     ~~~~\Arg{dim expr case_2} \Arg{code case_2} \\
%     ~~~~\ldots \\
%     ~~~~\Arg{dim expr case_n} \Arg{code case_n} \\
%     ~~|}| \\
%     ~~\Arg{true code}
%     ~~\Arg{false code}
%   \end{syntax}
%   This function evaluates the \meta{test dim expr} and
%   compares this in turn to each of the
%   \meta{dim expr case}s until a match is found.
%   If the two are equal then the
%   associated \meta{code} is left in the input stream
%   and other cases are discarded. If any of the
%   cases are matched, the \meta{true code} is also inserted into the
%   input stream (after the code for the appropriate case), while if none
%   match then the \meta{false code} is inserted. The function
%   \cs{dim_case:nn}, which does nothing if there is no match, is also
%   available. For example
%   \begin{verbatim}
%     \dim_set:Nn \l_tmpa_dim { 5 pt }
%     \dim_case:nnF
%       { 2 \l_tmpa_dim }
%       {
%         { 5 pt }        { Small }
%         { 4 pt + 6 pt } { Medium }
%         { - 10 pt }     { Negative }
%       }
%       { No idea! }
%    \end{verbatim}
%    leaves \enquote{\texttt{Medium}} in the input stream.
%   Since evaluation of the test expressions stops at the first
%   successful case, the order of possible matches should normally
%   be that the most likely are earlier: this will reduce the average
%   steps required to complete expansion.
% \end{function}
%
% \section{Dimension expression loops}
%
% \begin{function}[rEXP]{\dim_do_until:nNnn}
%   \begin{syntax}
%      \cs{dim_do_until:nNnn} \Arg{dim expr_1} \meta{relation} \Arg{dim expr_2} \Arg{code}
%   \end{syntax}
%   Places the \meta{code} in the input stream for \TeX{} to process, and
%   then evaluates the relationship between the two
%   \meta{dim exprs} as described for \cs{dim_compare:nNnTF}.
%   If the test is \texttt{false} then the \meta{code} is inserted
%   into the input stream again and a loop occurs until the
%   \meta{relation} is \texttt{true}.
% \end{function}
%
% \begin{function}[rEXP]{\dim_do_while:nNnn}
%   \begin{syntax}
%      \cs{dim_do_while:nNnn} \Arg{dim expr_1} \meta{relation} \Arg{dim expr_2} \Arg{code}
%   \end{syntax}
%   Places the \meta{code} in the input stream for \TeX{} to process, and
%   then evaluates the relationship between the two
%   \meta{dim exprs} as described for \cs{dim_compare:nNnTF}.
%   If the test is \texttt{true} then the \meta{code} is inserted
%   into the input stream again and a loop occurs until the
%   \meta{relation} is \texttt{false}.
% \end{function}
%
% \begin{function}[rEXP]{\dim_until_do:nNnn}
%   \begin{syntax}
%      \cs{dim_until_do:nNnn} \Arg{dim expr_1} \meta{relation} \Arg{dim expr_2} \Arg{code}
%   \end{syntax}
%   Evaluates the relationship between the two \meta{dim exprs}
%   as described for \cs{dim_compare:nNnTF}, and then places the
%   \meta{code} in the input stream if the \meta{relation} is
%   \texttt{false}. After the \meta{code} has been processed by \TeX{} the
%   test is repeated, and a loop occurs until the test is
%   \texttt{true}.
% \end{function}
%
% \begin{function}[rEXP]{\dim_while_do:nNnn}
%   \begin{syntax}
%      \cs{dim_while_do:nNnn} \Arg{dim expr_1} \meta{relation} \Arg{dim expr_2} \Arg{code}
%   \end{syntax}
%   Evaluates the relationship between the two \meta{dim exprs}
%   as described for \cs{dim_compare:nNnTF}, and then places the
%   \meta{code} in the input stream if the \meta{relation} is
%   \texttt{true}. After the \meta{code} has been processed by \TeX{} the
%   test is repeated, and a loop occurs until the test is
%   \texttt{false}.
% \end{function}
%
% \begin{function}[updated = 2013-01-13, rEXP]{\dim_do_until:nn}
%   \begin{syntax}
%      \cs{dim_do_until:nn} \Arg{dimension relation} \Arg{code}
%   \end{syntax}
%   Places the \meta{code} in the input stream for \TeX{} to process, and
%   then evaluates the \meta{dimension relation}
%   as described for \cs{dim_compare:nTF}.
%   If the test is \texttt{false} then the \meta{code} is inserted
%   into the input stream again and a loop occurs until the
%   \meta{relation} is \texttt{true}.
% \end{function}
%
% \begin{function}[updated = 2013-01-13, rEXP]{\dim_do_while:nn}
%   \begin{syntax}
%      \cs{dim_do_while:nn} \Arg{dimension relation} \Arg{code}
%   \end{syntax}
%   Places the \meta{code} in the input stream for \TeX{} to process, and
%   then evaluates the \meta{dimension relation}
%   as described for \cs{dim_compare:nTF}.
%   If the test is \texttt{true} then the \meta{code} is inserted
%   into the input stream again and a loop occurs until the
%   \meta{relation} is \texttt{false}.
% \end{function}
%
% \begin{function}[updated = 2013-01-13, rEXP]{\dim_until_do:nn}
%   \begin{syntax}
%      \cs{dim_until_do:nn} \Arg{dimension relation} \Arg{code}
%   \end{syntax}
%   Evaluates the \meta{dimension relation}
%   as described for \cs{dim_compare:nTF}, and then places the
%   \meta{code} in the input stream if the \meta{relation} is
%   \texttt{false}. After the \meta{code} has been processed by \TeX{} the
%   test is repeated, and a loop occurs until the test is
%   \texttt{true}.
% \end{function}
%
% \begin{function}[updated = 2013-01-13, rEXP]{\dim_while_do:nn}
%   \begin{syntax}
%      \cs{dim_while_do:nn} \Arg{dimension relation} \Arg{code}
%   \end{syntax}
%   Evaluates the \meta{dimension relation}
%   as described for \cs{dim_compare:nTF}, and then places the
%   \meta{code} in the input stream if the \meta{relation} is
%   \texttt{true}. After the \meta{code} has been processed by \TeX{} the
%   test is repeated, and a loop occurs until the test is
%   \texttt{false}.
% \end{function}
%
% \section{Dimension step functions}
%
% \begin{function}[added = 2018-02-18, rEXP]
%   {\dim_step_function:nnnN}
%   \begin{syntax}
%     \cs{dim_step_function:nnnN} \Arg{initial value} \Arg{step} \Arg{final value} \meta{function}
%   \end{syntax}
%   This function first evaluates the \meta{initial value}, \meta{step}
%   and \meta{final value}, all of which should be dimension expressions.
%   The \meta{function} is then placed in front of each \meta{value}
%   from the \meta{initial value} to the \meta{final value} in turn
%   (using \meta{step} between each \meta{value}).  The \meta{step} must
%   be non-zero.  If the \meta{step} is positive, the loop stops when
%   the \meta{value} becomes larger than the \meta{final value}.  If the
%   \meta{step} is negative, the loop stops when the \meta{value}
%   becomes smaller than the \meta{final value}.  The \meta{function}
%   should absorb one argument.
% \end{function}
%
% \begin{function}[added = 2018-02-18]
%   {\dim_step_inline:nnnn}
%   \begin{syntax}
%     \cs{dim_step_inline:nnnn} \Arg{initial value} \Arg{step} \Arg{final value} \Arg{code}
%   \end{syntax}
%   This function first evaluates the \meta{initial value}, \meta{step}
%   and \meta{final value}, all of which should be dimension expressions.
%   Then for each \meta{value} from the \meta{initial value} to the
%   \meta{final value} in turn (using \meta{step} between each
%   \meta{value}), the \meta{code} is inserted into the input stream
%   with |#1| replaced by the current \meta{value}.  Thus the
%   \meta{code} should define a function of one argument~(|#1|).
% \end{function}
%
% \begin{function}[added = 2018-02-18]
%   {\dim_step_variable:nnnNn}
%   \begin{syntax}
%     \cs{dim_step_variable:nnnNn} \\
%     ~~\Arg{initial value} \Arg{step} \Arg{final value} \meta{tl~var} \Arg{code}
%   \end{syntax}
%   This function first evaluates the \meta{initial value}, \meta{step}
%   and \meta{final value}, all of which should be dimension expressions.
%   Then for each \meta{value} from the \meta{initial value} to the
%   \meta{final value} in turn (using \meta{step} between each
%   \meta{value}), the \meta{code} is inserted into the input stream,
%   with the \meta{tl~var} defined as the current \meta{value}.  Thus
%   the \meta{code} should make use of the \meta{tl~var}.
% \end{function}
%
% \section{Using \texttt{dim} expressions and variables}
%
% \begin{function}[updated = 2011-10-22, EXP]{\dim_eval:n}
%   \begin{syntax}
%     \cs{dim_eval:n} \Arg{dim expr}
%   \end{syntax}
%   Evaluates the \meta{dim expr}, expanding any
%   dimensions and token list variables within the \meta{expression}
%   to their content (without requiring \cs{dim_use:N}/\cs{tl_use:N})
%   and applying the standard mathematical rules. The result of the
%   calculation is left in the input stream as a
%   \meta{dimension denotation} after two expansions. This is
%   expressed in points (\texttt{pt}), and requires suitable
%   termination if used in a \TeX{}-style assignment as it is \emph{not}
%   an \meta{internal dimension}.
% \end{function}
%
% \begin{function}[EXP, added = 2018-11-03]{\dim_sign:n}
%   \begin{syntax}
%     \cs{dim_sign:n} \Arg{dim expr}
%   \end{syntax}
%   Evaluates the \meta{dim expr} then leaves $1$ or $0$ or $-1$ in the
%   input stream according to the sign of the result.
% \end{function}
%
% \begin{function}[EXP]{\dim_use:N, \dim_use:c}
%   \begin{syntax}
%     \cs{dim_use:N} \meta{dimension}
%   \end{syntax}
%   Recovers the content of a \meta{dimension} and places it directly
%   in the input stream. An error is raised if the variable does
%   not exist or if it is invalid. Can be omitted in places where a
%   \meta{dimension} is required (such as in the argument of
%   \cs{dim_eval:n}).
%   \begin{texnote}
%     \cs{dim_use:N} is the \TeX{} primitive \tn{the}: this is one of
%     several \LaTeX3 names for this primitive.
%   \end{texnote}
% \end{function}
%
% \begin{function}[added = 2014-07-15, EXP]{\dim_to_decimal:n}
%   \begin{syntax}
%     \cs{dim_to_decimal:n} \Arg{dim expr}
%   \end{syntax}
%   Evaluates the \meta{dim expr}, and leaves the result,
%   expressed in points (\texttt{pt}) in the input stream, with \emph{no
%   units}.  The result is rounded by \TeX{} to at most five decimal
%   places.  If the decimal part of the result is zero, it is omitted,
%   together with the decimal marker.
%
%   For example
%   \begin{verbatim}
%     \dim_to_decimal:n { 1bp }
%   \end{verbatim}
%   leaves |1.00374| in the input stream, i.e., the magnitude of
%   one \enquote{big point} when converted to (\TeX{}) points.
% \end{function}
%
% \begin{function}[added = 2014-07-15, updated = 2023-05-20, EXP]
%   {\dim_to_decimal_in_bp:n}
%   \begin{syntax}
%     \cs{dim_to_decimal_in_bp:n} \Arg{dim expr}
%   \end{syntax}
%   Evaluates the \meta{dim expr}, and leaves the result,
%   expressed in big points (\texttt{bp}) in the input stream, with \emph{no
%   units}.  The result is rounded by \TeX{} to at most five decimal
%   places.  If the decimal part of the result is zero, it is omitted,
%   together with the decimal marker.
%
%   For example
%   \begin{verbatim}
%     \dim_to_decimal_in_bp:n { 1pt }
%   \end{verbatim}
%   leaves |0.99628| in the input stream, i.e., the magnitude of
%   one (\TeX{}) point when converted to big points.
%   \begin{texnote}
%      The implementation of this function is re-entrant: the result of
%      \begin{verbatim}
%        \dim_compare:nNnTF
%          { <x>bp } =
%            { \dim_to_decimal_in_bp:n { <x>bp } bp }
%      \end{verbatim}
%      will be logically \texttt{true}. The decimal representations may
%      differ provided they produce the same \TeX{} dimension.
%   \end{texnote}
% \end{function}
%
% \begin{function}[added = 2023-05-20, EXP]
%   {
%     \dim_to_decimal_in_cc:n ,
%     \dim_to_decimal_in_cm:n ,
%     \dim_to_decimal_in_dd:n ,
%     \dim_to_decimal_in_in:n ,
%     \dim_to_decimal_in_mm:n ,
%     \dim_to_decimal_in_pc:n
%   }
%   \begin{syntax}
%     \cs{dim_to_decimal_in_cm:n} \Arg{dim expr}
%   \end{syntax}
%   Evaluates the \meta{dim expr}, and leaves the result,
%   expressed with the appropriate scaling in the input stream, with
%   \emph{no units}. If the decimal part of the result is zero, it is omitted,
%   together with the decimal marker. The precisions of the result is limited
%   to a maximum of five decimal places with trailing zeros omitted.
%
%   The maximum \TeX{} allowable dimension value (available as
%   \tn{maxdimen} in plain \TeX{} and \LaTeX{} and \cs{c_max_dim} in
%   \pkg{expl3}) can only be expressed exactly in the units
%   \texttt{pt}, \texttt{bp} and \texttt{sp}. The maximum allowable
%   input values to five decimal places are\\
%   \begin{center}
%     \begin{tabular}{@{}>{$}r<{$}@{\,}l@{}}
%      1276.00215 & cc \\
%       575.83174 & cm \\
%     15312.02584 & dd \\
%       226.70540 & in \\
%      5758.31742 & mm \\
%      1365.33333 & pc \\
%     \end{tabular}
%   \end{center}
%   (Note that these are not all equal, but rather any larger value will overflow
%   due to the way \TeX{} converts to \texttt{sp}.)
%   Values given to five decimal places larger that these will result in \TeX{}
%   errors; the behavior if additional decimal places are given depends on the
%   \TeX{} internals and thus larger values are \emph{not} supported by
%   \pkg{expl3}.
%   \begin{texnote}
%      The implementation of these functions is re-entrant: the result of
%      \begin{verbatim}
%        \dim_compare:nNnTF
%          { <x><unit> } =
%            { \dim_to_decimal_in_<unit>:n { <x><unit> } <unit> }
%      \end{verbatim}
%      will be logically \texttt{true}. The decimal representations may
%      differ provided they produce the same \TeX{} dimension.
%   \end{texnote}
% \end{function}
%
% \begin{function}[added = 2015-05-18, EXP]{\dim_to_decimal_in_sp:n}
%   \begin{syntax}
%     \cs{dim_to_decimal_in_sp:n} \Arg{dim expr}
%   \end{syntax}
%   Evaluates the \meta{dim expr}, and leaves the result,
%   expressed in scaled points (\texttt{sp}) in the input stream, with \emph{no
%   units}.  The result is necessarily an integer.
% \end{function}
%
% \begin{function}[added = 2014-07-15, updated = 2023-05-20, EXP]
%   {\dim_to_decimal_in_unit:nn}
%   \begin{syntax}
%     \cs{dim_to_decimal_in_unit:nn} \Arg{dim expr_1} \Arg{dim expr_2}
%   \end{syntax}
%   Evaluates the \meta{dim exprs}, and leaves the value of
%   \meta{dim expr_1}, expressed in a unit given by \meta{dim expr_2}, in
%   the input stream. If the decimal part of the result
%   is zero, it is omitted, together with the decimal marker.
%   The precisions of the result is limited
%   to a maximum of five decimal places with trailing zeros omitted.
%
%   For example
%   \begin{verbatim}
%     \dim_to_decimal_in_unit:nn { 1bp } { 1mm }
%   \end{verbatim}
%   leaves |0.35278| in the input stream, i.e., the magnitude of
%   one big point when expressed in millimetres. The conversions do
%   \emph{not} guarantee that \TeX{} would yield identical results
%   for the direct input in an equality test, thus for instance
%   \begin{verbatim}
%     \dim_compare:nNnTF
%       { 1bp } =
%         { \dim_to_decimal_in_unit:nn { 1bp } { 1mm } mm }
%   \end{verbatim}
%   will take the \texttt{false} branch.
% \end{function}
%
% \begin{function}[EXP, added = 2012-05-08, tested = m3fp-convert002]
%   {\dim_to_fp:n}
%   \begin{syntax}
%     \cs{dim_to_fp:n} \Arg{dim expr}
%   \end{syntax}
%   Expands to an internal floating point number equal to the value of
%   the \meta{dim expr} in \texttt{pt}.  Since dimension expressions are
%   evaluated much faster than their floating point equivalent,
%   \cs{dim_to_fp:n} can be used to speed up parts of a computation
%   where a low precision and a smaller range are acceptable.
% \end{function}
%
% \section{Viewing \texttt{dim} variables}
%
% \begin{function}{\dim_show:N, \dim_show:c}
%   \begin{syntax}
%     \cs{dim_show:N} \meta{dimension}
%    \end{syntax}
%   Displays the value of the \meta{dimension} on the terminal.
% \end{function}
%
% \begin{function}[added = 2011-11-22, updated = 2015-08-07]{\dim_show:n}
%   \begin{syntax}
%     \cs{dim_show:n} \Arg{dim expr}
%    \end{syntax}
%   Displays the result of evaluating the \meta{dim expr}
%   on the terminal.
% \end{function}
%
% \begin{function}[added = 2014-08-22, updated = 2015-08-03]{\dim_log:N, \dim_log:c}
%   \begin{syntax}
%     \cs{dim_log:N} \meta{dimension}
%    \end{syntax}
%   Writes the value of the \meta{dimension} in the log file.
% \end{function}
%
% \begin{function}[added = 2014-08-22, updated = 2015-08-07]{\dim_log:n}
%   \begin{syntax}
%     \cs{dim_log:n} \Arg{dim expr}
%    \end{syntax}
%   Writes the result of evaluating the \meta{dim expr}
%   in the log file.
% \end{function}
%
% \section{Constant dimensions}
%
% \begin{variable}{\c_max_dim}
%   The maximum value that can be stored as a dimension.  This can also
%   be used as a component of a skip.
% \end{variable}
%
% \begin{variable}{\c_zero_dim}
%   A zero length as a dimension.  This can also be used as a component
%   of a skip.
% \end{variable}
%
% \section{Scratch dimensions}
%
% \begin{variable}{\l_tmpa_dim, \l_tmpb_dim}
%   Scratch dimension for local assignment. These are never used by
%   the kernel code, and so are safe for use with any \LaTeX3-defined
%   function. However, they may be overwritten by other non-kernel
%   code and so should only be used for short-term storage.
% \end{variable}
%
% \begin{variable}{\g_tmpa_dim, \g_tmpb_dim}
%   Scratch dimension for global assignment. These are never used by
%   the kernel code, and so are safe for use with any \LaTeX3-defined
%   function. However, they may be overwritten by other non-kernel
%   code and so should only be used for short-term storage.
% \end{variable}
%
% \section{Creating and initializing \texttt{skip} variables}
%
% \begin{function}{\skip_new:N, \skip_new:c}
%   \begin{syntax}
%     \cs{skip_new:N} \meta{skip}
%   \end{syntax}
%   Creates a new \meta{skip} or raises an error if the name is
%   already taken. The declaration is global. The \meta{skip}
%   is initially equal to $0$\,pt.
% \end{function}
%
% \begin{function}[added = 2012-03-05]{\skip_const:Nn, \skip_const:cn}
%   \begin{syntax}
%     \cs{skip_const:Nn} \meta{skip} \Arg{skip expr}
%   \end{syntax}
%   Creates a new constant \meta{skip} or raises an error if the
%   name is already taken. The value of the \meta{skip} is set
%   globally to the \meta{skip expr}.
% \end{function}
%
% \begin{function}{\skip_zero:N, \skip_zero:c, \skip_gzero:N, \skip_gzero:c}
%   \begin{syntax}
%     \cs{skip_zero:N} \meta{skip}
%   \end{syntax}
%   Sets \meta{skip} to $0$\,pt.
% \end{function}
%
% \begin{function}[added = 2012-01-07]
%   {\skip_zero_new:N, \skip_zero_new:c, \skip_gzero_new:N, \skip_gzero_new:c}
%   \begin{syntax}
%     \cs{skip_zero_new:N} \meta{skip}
%   \end{syntax}
%   Ensures that the \meta{skip} exists globally by applying
%   \cs{skip_new:N} if necessary, then applies
%   \cs[index=skip_zero:N]{skip_(g)zero:N} to leave
%   the \meta{skip} set to zero.
% \end{function}
%
% \begin{function}[EXP, pTF, added=2012-03-03]
%   {\skip_if_exist:N, \skip_if_exist:c}
%   \begin{syntax}
%     \cs{skip_if_exist_p:N} \meta{skip}
%     \cs{skip_if_exist:NTF} \meta{skip} \Arg{true code} \Arg{false code}
%   \end{syntax}
%   Tests whether the \meta{skip} is currently defined.  This does not
%   check that the \meta{skip} really is a skip variable.
% \end{function}
%
% \section{Setting \texttt{skip} variables}
%
% \begin{function}[updated = 2011-10-22]
%   {\skip_add:Nn, \skip_add:cn, \skip_gadd:Nn, \skip_gadd:cn}
%   \begin{syntax}
%     \cs{skip_add:Nn} \meta{skip} \Arg{skip expr}
%   \end{syntax}
%   Adds the result of the \meta{skip expr} to the current
%   content of the \meta{skip}.
% \end{function}
%
% \begin{function}[updated = 2011-10-22]
%   {
%     \skip_set:Nn, \skip_set:cn, \skip_set:NV, \skip_set:cV,
%     \skip_gset:Nn, \skip_gset:cn, \skip_gset:NV, \skip_gset:cV
%   }
%   \begin{syntax}
%     \cs{skip_set:Nn} \meta{skip} \Arg{skip expr}
%   \end{syntax}
%   Sets \meta{skip} to the value of \meta{skip expr}, which
%   must evaluate to a length with units and may include a rubber
%   component (for example |1 cm plus 0.5 cm|).
% \end{function}
%
% \begin{function}
%   {
%     \skip_set_eq:NN,  \skip_set_eq:cN,  \skip_set_eq:Nc,  \skip_set_eq:cc,
%     \skip_gset_eq:NN, \skip_gset_eq:cN, \skip_gset_eq:Nc, \skip_gset_eq:cc
%   }
%   \begin{syntax}
%     \cs{skip_set_eq:NN} \meta{skip_1} \meta{skip_2}
%   \end{syntax}
%   Sets the content of \meta{skip_1} equal to that of \meta{skip_2}.
% \end{function}
%
% \begin{function}[updated = 2011-10-22]
%   {\skip_sub:Nn, \skip_sub:cn, \skip_gsub:Nn, \skip_gsub:cn}
%   \begin{syntax}
%     \cs{skip_sub:Nn} \meta{skip} \Arg{skip expr}
%   \end{syntax}
%   Subtracts the result of the \meta{skip expr} from the
%   current content of the \meta{skip}.
% \end{function}
%
% \section{Skip expression conditionals}
%
% \begin{function}[EXP,pTF]{\skip_if_eq:nn}
%   \begin{syntax}
%     \cs{skip_if_eq_p:nn} \Arg{skip expr_1} \Arg{skip expr_2}
%     \cs{skip_if_eq:nnTF}
%     ~~\Arg{skip expr_1} \Arg{skip expr_2}
%     ~~\Arg{true code} \Arg{false code}
%   \end{syntax}
%   This function first evaluates each of the
%   \meta{skip exprs} as described for \cs{skip_eval:n}.
%   The two results are then compared for exact equality,
%   i.e., both the fixed and rubber components must be the same
%   for the test to be true.
% \end{function}
%
% \begin{function}[EXP, pTF, added = 2012-03-05]{\skip_if_finite:n}
%   \begin{syntax}
%     \cs{skip_if_finite_p:n} \Arg{skip expr}
%     \cs{skip_if_finite:nTF} \Arg{skip expr} \Arg{true code} \Arg{false code}
%   \end{syntax}
%   Evaluates the \meta{skip expr} as described for \cs{skip_eval:n},
%   and then tests if all of its components are finite.
% \end{function}
%
% \section{Using \texttt{skip} expressions and variables}
%
% \begin{function}[updated = 2011-10-22, EXP]{\skip_eval:n}
%   \begin{syntax}
%     \cs{skip_eval:n} \Arg{skip expr}
%   \end{syntax}
%   Evaluates the \meta{skip expr}, expanding any skips
%   and token list variables within the \meta{expression}
%   to their content (without requiring \cs{skip_use:N}/\cs{tl_use:N})
%   and applying the standard mathematical rules. The result of the
%   calculation is left in the input stream as a \meta{glue denotation}
%   after two expansions. This is expressed in points (\texttt{pt}),
%   and requires suitable termination if used in a \TeX{}-style
%   assignment as it is \emph{not} an \meta{internal glue}.
% \end{function}
%
% \begin{function}[EXP]{\skip_use:N, \skip_use:c}
%   \begin{syntax}
%     \cs{skip_use:N} \meta{skip}
%   \end{syntax}
%   Recovers the content of a \meta{skip} and places it directly
%   in the input stream. An error is raised if the variable does
%   not exist or if it is invalid. Can be omitted in places where a
%   \meta{dimension} or \meta{skip} is required (such as in the argument of
%   \cs{skip_eval:n}).
%   \begin{texnote}
%     \cs{skip_use:N} is the \TeX{} primitive \tn{the}: this is one of
%     several \LaTeX3 names for this primitive.
%   \end{texnote}
% \end{function}
%
% \section{Viewing \texttt{skip} variables}
%
% \begin{function}[updated = 2015-08-03]{\skip_show:N, \skip_show:c}
%   \begin{syntax}
%     \cs{skip_show:N} \meta{skip}
%    \end{syntax}
%   Displays the value of the \meta{skip} on the terminal.
% \end{function}
%
% \begin{function}[added = 2011-11-22, updated = 2015-08-07]{\skip_show:n}
%   \begin{syntax}
%     \cs{skip_show:n} \Arg{skip expr}
%    \end{syntax}
%   Displays the result of evaluating the \meta{skip expr}
%   on the terminal.
% \end{function}
%
% \begin{function}[added = 2014-08-22, updated = 2015-08-03]{\skip_log:N, \skip_log:c}
%   \begin{syntax}
%     \cs{skip_log:N} \meta{skip}
%    \end{syntax}
%   Writes the value of the \meta{skip} in the log file.
% \end{function}
%
% \begin{function}[added = 2014-08-22, updated = 2015-08-07]{\skip_log:n}
%   \begin{syntax}
%     \cs{skip_log:n} \Arg{skip expr}
%    \end{syntax}
%   Writes the result of evaluating the \meta{skip expr}
%   in the log file.
% \end{function}
%
% \section{Constant skips}
%
% \begin{variable}[updated = 2012-11-02]{\c_max_skip}
%   The maximum value that can be stored as a skip (equal to
%   \cs{c_max_dim} in length), with no stretch nor shrink component.
% \end{variable}
%
% \begin{variable}[updated = 2012-11-01]{\c_zero_skip}
%   A zero length as a skip, with no stretch nor shrink component.
% \end{variable}
%
% \section{Scratch skips}
%
% \begin{variable}{\l_tmpa_skip, \l_tmpb_skip}
%   Scratch skip for local assignment. These are never used by
%   the kernel code, and so are safe for use with any \LaTeX3-defined
%   function. However, they may be overwritten by other non-kernel
%   code and so should only be used for short-term storage.
% \end{variable}
%
% \begin{variable}{\g_tmpa_skip, \g_tmpb_skip}
%   Scratch skip for global assignment. These are never used by
%   the kernel code, and so are safe for use with any \LaTeX3-defined
%   function. However, they may be overwritten by other non-kernel
%   code and so should only be used for short-term storage.
% \end{variable}
%
% \section{Inserting skips into the output}
%
% \begin{function}[updated = 2011-10-22]
%   {\skip_horizontal:N, \skip_horizontal:c, \skip_horizontal:n}
%   \begin{syntax}
%     \cs{skip_horizontal:N} \meta{skip}
%     \cs{skip_horizontal:n} \Arg{skip expr}
%   \end{syntax}
%   Inserts a horizontal \meta{skip} into the current list.
%   The argument can also be a \meta{dim}.
%   \begin{texnote}
%     \cs{skip_horizontal:N} is the \TeX{} primitive \tn{hskip}.
%   \end{texnote}
% \end{function}
%
% \begin{function}[updated = 2011-10-22]
%   {\skip_vertical:N, \skip_vertical:c, \skip_vertical:n}
%   \begin{syntax}
%     \cs{skip_vertical:N} \meta{skip}
%     \cs{skip_vertical:n} \Arg{skip expr}
%   \end{syntax}
%   Inserts a vertical \meta{skip} into the current list.
%   The argument can also be a \meta{dim}.
%   \begin{texnote}
%     \cs{skip_vertical:N} is the \TeX{} primitive \tn{vskip}.
%   \end{texnote}
% \end{function}
%
% \section{Creating and initializing \texttt{muskip} variables}
%
% \begin{function}{\muskip_new:N, \muskip_new:c}
%   \begin{syntax}
%     \cs{muskip_new:N} \meta{muskip}
%   \end{syntax}
%   Creates a new \meta{muskip} or raises an error if the name is
%   already taken. The declaration is global. The \meta{muskip}
%   is initially equal to $0$\,mu.
% \end{function}
%
% \begin{function}[added = 2012-03-05]{\muskip_const:Nn, \muskip_const:cn}
%   \begin{syntax}
%     \cs{muskip_const:Nn} \meta{muskip} \Arg{muskip expr}
%   \end{syntax}
%   Creates a new constant \meta{muskip} or raises an error if the
%   name is already taken. The value of the \meta{muskip} is set
%   globally to the \meta{muskip expr}.
% \end{function}
%
% \begin{function}
%   {\muskip_zero:N, \muskip_zero:c, \muskip_gzero:N, \muskip_gzero:c}
%   \begin{syntax}
%     \cs{skip_zero:N} \meta{muskip}
%   \end{syntax}
%   Sets \meta{muskip} to $0$\,mu.
% \end{function}
%
% \begin{function}[added = 2012-01-07]
%   {
%     \muskip_zero_new:N,  \muskip_zero_new:c,
%     \muskip_gzero_new:N, \muskip_gzero_new:c
%   }
%   \begin{syntax}
%     \cs{muskip_zero_new:N} \meta{muskip}
%   \end{syntax}
%   Ensures that the \meta{muskip} exists globally by applying
%   \cs{muskip_new:N} if necessary, then applies
%   \cs[index=muskip_zero:N]{muskip_(g)zero:N}
%   to leave the \meta{muskip} set to zero.
% \end{function}
%
% \begin{function}[EXP, pTF, added=2012-03-03]
%   {\muskip_if_exist:N, \muskip_if_exist:c}
%   \begin{syntax}
%     \cs{muskip_if_exist_p:N} \meta{muskip}
%     \cs{muskip_if_exist:NTF} \meta{muskip} \Arg{true code} \Arg{false code}
%   \end{syntax}
%   Tests whether the \meta{muskip} is currently defined.  This does not
%   check that the \meta{muskip} really is a muskip variable.
% \end{function}
%
% \section{Setting \texttt{muskip} variables}
%
% \begin{function}[updated = 2011-10-22]
%   {\muskip_add:Nn, \muskip_add:cn, \muskip_gadd:Nn, \muskip_gadd:cn}
%   \begin{syntax}
%     \cs{muskip_add:Nn} \meta{muskip} \Arg{muskip expr}
%   \end{syntax}
%   Adds the result of the \meta{muskip expr} to the current
%   content of the \meta{muskip}.
% \end{function}
%
% \begin{function}[updated = 2011-10-22]
%   {
%     \muskip_set:Nn, \muskip_set:cn, \muskip_set:NV, \muskip_set:cV,
%     \muskip_gset:Nn, \muskip_gset:cn, \muskip_gset:NV, \muskip_gset:cV
%   }
%   \begin{syntax}
%     \cs{muskip_set:Nn} \meta{muskip} \Arg{muskip expr}
%   \end{syntax}
%   Sets \meta{muskip} to the value of \meta{muskip expr}, which
%   must evaluate to a math length with units and may include a rubber
%   component (for example |1 mu plus 0.5 mu|.
% \end{function}
%
% \begin{function}
%   {
%     \muskip_set_eq:NN,  \muskip_set_eq:cN,
%     \muskip_set_eq:Nc,  \muskip_set_eq:cc,
%     \muskip_gset_eq:NN, \muskip_gset_eq:cN,
%     \muskip_gset_eq:Nc, \muskip_gset_eq:cc
%   }
%   \begin{syntax}
%     \cs{muskip_set_eq:NN} \meta{muskip_1} \meta{muskip_2}
%   \end{syntax}
%   Sets the content of \meta{muskip_1} equal to that of
%   \meta{muskip_2}.
% \end{function}
%
% \begin{function}[updated = 2011-10-22]
%   {\muskip_sub:Nn, \muskip_sub:cn, \muskip_gsub:Nn, \muskip_gsub:cn}
%   \begin{syntax}
%     \cs{muskip_sub:Nn} \meta{muskip} \Arg{muskip expr}
%   \end{syntax}
%   Subtracts the result of the \meta{muskip expr} from the
%   current content of the \meta{muskip}.
% \end{function}
%
% \section{Using \texttt{muskip} expressions and variables}
%
% \begin{function}[updated = 2011-10-22, EXP]{\muskip_eval:n}
%   \begin{syntax}
%     \cs{muskip_eval:n} \Arg{muskip expr}
%   \end{syntax}
%   Evaluates the \meta{muskip expr}, expanding any skips
%   and token list variables within the \meta{expression}
%   to their content (without requiring \cs{muskip_use:N}/\cs{tl_use:N})
%   and applying the standard mathematical rules. The result of the
%   calculation is left in the input stream as a \meta{muglue denotation}
%   after two expansions. This is expressed in \texttt{mu},
%   and requires suitable termination if used in a \TeX{}-style
%   assignment as it is \emph{not} an \meta{internal muglue}.
% \end{function}
%
% \begin{function}[EXP]{\muskip_use:N, \muskip_use:c}
%   \begin{syntax}
%     \cs{muskip_use:N} \meta{muskip}
%   \end{syntax}
%   Recovers the content of a \meta{skip} and places it directly
%   in the input stream. An error is raised if the variable does
%   not exist or if it is invalid. Can be omitted in places where a
%   \meta{dimension} is required (such as in the argument of
%   \cs{muskip_eval:n}).
%   \begin{texnote}
%     \cs{muskip_use:N} is the \TeX{} primitive \tn{the}: this is one of
%     several \LaTeX3 names for this primitive.
%   \end{texnote}
% \end{function}
%
% \section{Viewing \texttt{muskip} variables}
%
% \begin{function}[updated = 2015-08-03]{\muskip_show:N, \muskip_show:c}
%   \begin{syntax}
%     \cs{muskip_show:N} \meta{muskip}
%    \end{syntax}
%   Displays the value of the \meta{muskip} on the terminal.
% \end{function}
%
% \begin{function}[added = 2011-11-22, updated = 2015-08-07]{\muskip_show:n}
%   \begin{syntax}
%     \cs{muskip_show:n} \Arg{muskip expr}
%    \end{syntax}
%   Displays the result of evaluating the \meta{muskip expr}
%   on the terminal.
% \end{function}
%
% \begin{function}[added = 2014-08-22, updated = 2015-08-03]{\muskip_log:N, \muskip_log:c}
%   \begin{syntax}
%     \cs{muskip_log:N} \meta{muskip}
%    \end{syntax}
%   Writes the value of the \meta{muskip} in the log file.
% \end{function}
%
% \begin{function}[added = 2014-08-22, updated = 2015-08-07]{\muskip_log:n}
%   \begin{syntax}
%     \cs{muskip_log:n} \Arg{muskip expr}
%    \end{syntax}
%   Writes the result of evaluating the \meta{muskip expr}
%   in the log file.
% \end{function}
%
% \section{Constant muskips}
%
% \begin{variable}{\c_max_muskip}
%   The maximum value that can be stored as a muskip, with no stretch
%   nor shrink component.
% \end{variable}
%
% \begin{variable}{\c_zero_muskip}
%   A zero length as a muskip, with no stretch nor shrink component.
% \end{variable}
%
% \section{Scratch muskips}
%
% \begin{variable}{\l_tmpa_muskip, \l_tmpb_muskip}
%   Scratch muskip for local assignment. These are never used by
%   the kernel code, and so are safe for use with any \LaTeX3-defined
%   function. However, they may be overwritten by other non-kernel
%   code and so should only be used for short-term storage.
% \end{variable}
%
% \begin{variable}{\g_tmpa_muskip, \g_tmpb_muskip}
%   Scratch muskip for global assignment. These are never used by
%   the kernel code, and so are safe for use with any \LaTeX3-defined
%   function. However, they may be overwritten by other non-kernel
%   code and so should only be used for short-term storage.
% \end{variable}
%
% \section{Primitive conditional}
%
% \begin{function}[EXP]{\if_dim:w}
%   \begin{syntax}
%     \cs{if_dim:w} \meta{dimen_1} \meta{relation} \meta{dimen_2}
%     ~~\meta{true code}
%     \cs{else:}
%     ~~\meta{false}
%     \cs{fi:}
%   \end{syntax}
%   Compare two dimensions. The \meta{relation} is one of
%   |<|, |=| or |>| with category code $12$.
%   \begin{texnote}
%     This is the \TeX{} primitive \tn{ifdim}.
%   \end{texnote}
% \end{function}
%
% \end{documentation}
%
% \begin{implementation}
%
% \section{\pkg{l3skip} implementation}
%
%    \begin{macrocode}
%<*code>
%    \end{macrocode}
%
%    \begin{macrocode}
%<@@=dim>
%    \end{macrocode}
%
% \subsection{Length primitives renamed}
%
% \begin{macro}{\if_dim:w}
% \begin{macro}{\@@_eval:w}
% \begin{macro}{\@@_eval_end:}
%   Primitives renamed.
%    \begin{macrocode}
\cs_new_eq:NN \if_dim:w      \tex_ifdim:D
\cs_new_eq:NN \@@_eval:w      \tex_dimexpr:D
\cs_new_eq:NN \@@_eval_end:   \tex_relax:D
%    \end{macrocode}
% \end{macro}
% \end{macro}
% \end{macro}
%
% \subsection{Internal auxiliaries}
%
% \begin{variable}{\s_@@_mark,\s_@@_stop}
%   Internal scan marks.
%    \begin{macrocode}
\scan_new:N \s_@@_mark
\scan_new:N \s_@@_stop
%    \end{macrocode}
% \end{variable}
%
% \begin{macro}[EXP]{\@@_use_none_delimit_by_s_stop:w}
%   Functions to gobble up to a scan mark.
%    \begin{macrocode}
\cs_new:Npn \@@_use_none_delimit_by_s_stop:w #1 \s_@@_stop { }
%    \end{macrocode}
% \end{macro}
%
% \subsection{Creating and initializing \texttt{dim} variables}
%
% \begin{macro}{\dim_new:N, \dim_new:c}
%    Allocating \meta{dim} registers \ldots
%    \begin{macrocode}
\cs_new_protected:Npn \dim_new:N #1
  {
    \__kernel_chk_if_free_cs:N #1
    \cs:w newdimen \cs_end: #1
  }
\cs_generate_variant:Nn \dim_new:N { c }
%    \end{macrocode}
% \end{macro}
%
% \begin{macro}{\dim_const:Nn, \dim_const:cn}
%   Contrarily to integer constants, we cannot avoid using a register,
%   even for constants.  We cannot use \cs{dim_gset:Nn} because
%   debugging code would complain that the constant is not a global
%   variable.  Since \cs{dim_const:Nn} does not need to be fast, use
%   \cs{dim_eval:n} to avoid needing a debugging patch that wraps the
%   expression in checking code.
%    \begin{macrocode}
\cs_new_protected:Npn \dim_const:Nn #1#2
  {
    \dim_new:N #1
    \tex_global:D #1 = \dim_eval:n {#2} \scan_stop:
  }
\cs_generate_variant:Nn \dim_const:Nn { c }
%    \end{macrocode}
% \end{macro}
%
% \begin{macro}{\dim_zero:N, \dim_zero:c}
% \begin{macro}{\dim_gzero:N, \dim_gzero:c}
%   Reset the register to zero.  Using \cs{c_zero_skip} deals with the
%   case where the variable passed is incorrectly a skip (for example a
%   \LaTeXe{} length).  Besides, these functions are then simply copied
%   for \cs{skip_zero:N} and related functions.
%    \begin{macrocode}
\cs_new_protected:Npn \dim_zero:N #1 { #1 = \c_zero_skip }
\cs_new_protected:Npn \dim_gzero:N #1
  { \tex_global:D #1 = \c_zero_skip }
\cs_generate_variant:Nn \dim_zero:N  { c }
\cs_generate_variant:Nn \dim_gzero:N { c }
%    \end{macrocode}
% \end{macro}
% \end{macro}
%
% \begin{macro}
%   {\dim_zero_new:N, \dim_zero_new:c, \dim_gzero_new:N, \dim_gzero_new:c}
%   Create a register if needed, otherwise clear it.
%    \begin{macrocode}
\cs_new_protected:Npn \dim_zero_new:N  #1
  { \dim_if_exist:NTF #1 { \dim_zero:N #1 } { \dim_new:N #1 } }
\cs_new_protected:Npn \dim_gzero_new:N #1
  { \dim_if_exist:NTF #1 { \dim_gzero:N #1 } { \dim_new:N #1 } }
\cs_generate_variant:Nn \dim_zero_new:N  { c }
\cs_generate_variant:Nn \dim_gzero_new:N { c }
%    \end{macrocode}
% \end{macro}
%
% \begin{macro}[pTF]{\dim_if_exist:N, \dim_if_exist:c}
%   Copies of the \texttt{cs} functions defined in \pkg{l3basics}.
%    \begin{macrocode}
\prg_new_eq_conditional:NNn \dim_if_exist:N \cs_if_exist:N
  { TF , T , F , p }
\prg_new_eq_conditional:NNn \dim_if_exist:c \cs_if_exist:c
  { TF , T , F , p }
%    \end{macrocode}
% \end{macro}
%
% \subsection{Setting \texttt{dim} variables}
%
% \begin{macro}{\dim_set:Nn, \dim_set:cn, \dim_set:NV, \dim_set:cV}
% \begin{macro}{\dim_gset:Nn, \dim_gset:cn, \dim_gset:NV, \dim_gset:cV}
%   Setting dimensions is easy enough but when debugging we want both to
%   check that the variable is correctly local/global and to wrap the
%   expression in some code.  The \cs{scan_stop:} deals with the case
%   where the variable passed is a skip (for example a \LaTeXe{}
%   length).
%    \begin{macrocode}
\cs_new_protected:Npn \dim_set:Nn #1#2
  { #1 = \@@_eval:w #2 \@@_eval_end: \scan_stop: }
\cs_new_protected:Npn \dim_gset:Nn #1#2
  { \tex_global:D #1 = \@@_eval:w #2 \@@_eval_end: \scan_stop: }
\cs_generate_variant:Nn \dim_set:Nn  { NV , c , cV }
\cs_generate_variant:Nn \dim_gset:Nn { NV , c , cV }
%    \end{macrocode}
% \end{macro}
% \end{macro}
%
% \begin{macro}{\dim_set_eq:NN, \dim_set_eq:cN, \dim_set_eq:Nc, \dim_set_eq:cc}
% \begin{macro}
%   {\dim_gset_eq:NN, \dim_gset_eq:cN, \dim_gset_eq:Nc, \dim_gset_eq:cc}
%   All straightforward, with a \cs{scan_stop:} to deal with the case
%   where |#1| is (incorrectly) a skip.
%    \begin{macrocode}
\cs_new_protected:Npn \dim_set_eq:NN #1#2
  { #1 = #2 \scan_stop: }
\cs_generate_variant:Nn \dim_set_eq:NN { c , Nc , cc }
\cs_new_protected:Npn \dim_gset_eq:NN #1#2
  { \tex_global:D #1 = #2 \scan_stop: }
\cs_generate_variant:Nn \dim_gset_eq:NN { c , Nc , cc }
%    \end{macrocode}
% \end{macro}
% \end{macro}
%
% \begin{macro}{\dim_add:Nn, \dim_add:cn}
% \begin{macro}{\dim_gadd:Nn, \dim_gadd:cn}
% \begin{macro}{\dim_sub:Nn, \dim_sub:cn}
% \begin{macro}{\dim_gsub:Nn, \dim_gsub:cn}
%   Using |by| here would slow things down just to detect nonsensical
%   cases such as passing |\dimen 123| as the first argument.
%   Using \cs{scan_stop:} deals with skip variables.  Since
%   debugging checks that the variable is correctly local/global, the
%   global versions cannot be defined as \cs{tex_global:D} followed by
%   the local versions.
%    \begin{macrocode}
\cs_new_protected:Npn \dim_add:Nn #1#2
  { \tex_advance:D #1 \@@_eval:w #2 \@@_eval_end: \scan_stop: }
\cs_new_protected:Npn \dim_gadd:Nn #1#2
  {
    \tex_global:D \tex_advance:D #1
      \@@_eval:w #2 \@@_eval_end: \scan_stop:
  }
\cs_generate_variant:Nn \dim_add:Nn  { c }
\cs_generate_variant:Nn \dim_gadd:Nn { c }
\cs_new_protected:Npn \dim_sub:Nn #1#2
  { \tex_advance:D #1 - \@@_eval:w #2 \@@_eval_end: \scan_stop: }
\cs_new_protected:Npn \dim_gsub:Nn #1#2
  {
    \tex_global:D \tex_advance:D #1
      -\@@_eval:w #2 \@@_eval_end: \scan_stop:
  }
\cs_generate_variant:Nn \dim_sub:Nn  { c }
\cs_generate_variant:Nn \dim_gsub:Nn { c }
%    \end{macrocode}
% \end{macro}
% \end{macro}
% \end{macro}
% \end{macro}
%
% \subsection{Utilities for dimension calculations}
%
% \begin{macro}{\@@_sep:}
%    \begin{macrocode}
\cs_new_eq:NN \@@_sep: \__kernel_int_sep:
%    \end{macrocode}
% \end{macro}
%
% \begin{macro}[EXP]{\dim_abs:n}
% \begin{macro}[EXP]{\@@_abs:N}
% \UnitTested
% \begin{macro}[EXP]{\dim_max:nn}
% \begin{macro}[EXP]{\dim_min:nn}
% \begin{macro}[EXP]{\@@_maxmin:wwN}
% \UnitTested
% \UnitTested
%   Functions for $\min$, $\max$, and absolute value with only one evaluation.
%   The absolute value is evaluated by removing a leading~|-| if present.
%    \begin{macrocode}
\cs_new:Npn \dim_abs:n #1
  {
    \exp_after:wN \@@_abs:N
    \dim_use:N \@@_eval:w #1 \@@_eval_end:
  }
\cs_new:Npn \@@_abs:N #1
  { \if_meaning:w - #1 \else: \exp_after:wN #1 \fi: }
\cs_new:Npn \dim_max:nn #1#2
  {
    \dim_use:N \@@_eval:w \exp_after:wN \@@_maxmin:wwN
      \dim_use:N \@@_eval:w #1 \exp_after:wN \@@_sep:
      \dim_use:N \@@_eval:w #2 \@@_sep:
      >
    \@@_eval_end:
  }
\cs_new:Npn \dim_min:nn #1#2
  {
    \dim_use:N \@@_eval:w \exp_after:wN \@@_maxmin:wwN
      \dim_use:N \@@_eval:w #1 \exp_after:wN \@@_sep:
      \dim_use:N \@@_eval:w #2 \@@_sep:
      <
    \@@_eval_end:
  }
\cs_new:Npn \@@_maxmin:wwN #1 \@@_sep: #2 \@@_sep: #3
  {
    \if_dim:w #1 #3 #2 ~
      #1
    \else:
      #2
    \fi:
  }
%    \end{macrocode}
% \end{macro}
% \end{macro}
% \end{macro}
% \end{macro}
% \end{macro}
%
% \begin{macro}{\dim_ratio:nn}
% \begin{macro}{\@@_ratio:n}
%   With dimension expressions, something like |10 pt * ( 5 pt / 10 pt )| does
%   not work. Instead, the ratio part needs to be converted to an integer
%   expression. Using \cs{int_value:w} forces everything into |sp|, avoiding
%   any decimal parts.
%    \begin{macrocode}
\cs_new:Npn \dim_ratio:nn #1#2
  { \@@_ratio:n {#1} / \@@_ratio:n {#2} }
\cs_new:Npn \@@_ratio:n #1
  { \int_value:w \@@_eval:w (#1) \@@_eval_end: }
%    \end{macrocode}
% \end{macro}
% \end{macro}
%
% \subsection{Dimension expression conditionals}
%
% \begin{macro}[pTF, EXP]{\dim_compare:nNn}
%   Simple comparison.
%    \begin{macrocode}
\prg_new_conditional:Npnn \dim_compare:nNn #1#2#3 { p , T , F , TF }
  {
    \if_dim:w \@@_eval:w #1 #2 \@@_eval:w #3 \@@_eval_end:
      \prg_return_true: \else: \prg_return_false: \fi:
  }
%    \end{macrocode}
% \end{macro}
%
% \begin{macro}[pTF, EXP]{\dim_compare:n}
% \begin{macro}[EXP]{\@@_compare:w, \@@_compare:wNN}
% \begin{macro}[EXP]
%   {
%     \@@_compare_=:w,
%     \@@_compare_!:w,
%     \@@_compare_<:w,
%     \@@_compare_>:w
%   }
% \begin{macro}{\@@_compare_error:}
%   This code is adapted from the \cs{int_compare:nTF} function.  First
%   make sure that there is at least one relation operator, by
%   evaluating a dimension expression with a trailing
%   \cs{@@_compare_error:}.  Just like for integers, the looping
%   auxiliary \cs{@@_compare:wNN} closes a primitive conditional and
%   opens a new one.  It is actually easier to grab a dimension operand
%   than an integer one, because once evaluated, dimensions all end with
%   \texttt{pt} (with category other).  Thus we do not need specific
%   auxiliaries for the three \enquote{simple} relations |<|, |=|,
%   and~|>|.
%    \begin{macrocode}
\prg_new_conditional:Npnn \dim_compare:n #1 { p , T , F , TF }
  {
    \exp_after:wN \@@_compare:w
    \dim_use:N \@@_eval:w #1 \@@_compare_error:
  }
\cs_new:Npn \@@_compare:w #1 \@@_compare_error:
  {
    \exp_after:wN \if_false: \exp:w \exp_end_continue_f:w
      \@@_compare:wNN #1 ? { = \@@_compare_end:w \else: } \s_@@_stop
  }
\exp_args:Nno \use:nn
  { \cs_new:Npn \@@_compare:wNN #1 } { \tl_to_str:n {pt} #2#3 }
  {
      \if_meaning:w = #3
        \use:c { @@_compare_#2:w }
      \fi:
        #1 pt \exp_stop_f:
      \prg_return_false:
      \exp_after:wN \@@_use_none_delimit_by_s_stop:w
    \fi:
    \reverse_if:N \if_dim:w #1 pt #2
      \exp_after:wN \@@_compare:wNN
      \dim_use:N \@@_eval:w #3
  }
\cs_new:cpn { @@_compare_ ! :w }
    #1 \reverse_if:N #2 ! #3 = { #1 #2 = #3 }
\cs_new:cpn { @@_compare_ = :w }
    #1 \@@_eval:w = { #1 \@@_eval:w }
\cs_new:cpn { @@_compare_ < :w }
    #1 \reverse_if:N #2 < #3 = { #1 #2 > #3 }
\cs_new:cpn { @@_compare_ > :w }
    #1 \reverse_if:N #2 > #3 = { #1 #2 < #3 }
\cs_new:Npn \@@_compare_end:w #1 \prg_return_false: #2 \s_@@_stop
  { #1 \prg_return_false: \else: \prg_return_true: \fi: }
\cs_new_protected:Npn \@@_compare_error:
  {
    \if_int_compare:w \c_zero_int \c_zero_int \fi:
    =
    \@@_compare_error:
  }
%    \end{macrocode}
% \end{macro}
% \end{macro}
% \end{macro}
% \end{macro}
%
% \begin{macro}[EXP, noTF]{\dim_case:nn}
% \begin{macro}{\@@_case:nnTF}
% \begin{macro}{\@@_case:nw, \@@_case_end:nw}
%   For dimension cases, the first task to fully expand the check
%   condition. The over all idea is then much the same as for
%   \cs{str_case:nnTF} as described in \pkg{l3basics}.
%    \begin{macrocode}
\cs_new:Npn \dim_case:nnTF #1
  {
    \exp:w
    \exp_args:Nf \@@_case:nnTF { \dim_eval:n {#1} }
  }
\cs_new:Npn \dim_case:nnT #1#2#3
  {
    \exp:w
    \exp_args:Nf \@@_case:nnTF { \dim_eval:n {#1} } {#2} {#3} { }
  }
\cs_new:Npn \dim_case:nnF #1#2
  {
    \exp:w
    \exp_args:Nf \@@_case:nnTF { \dim_eval:n {#1} } {#2} { }
  }
\cs_new:Npn \dim_case:nn #1#2
  {
    \exp:w
    \exp_args:Nf \@@_case:nnTF { \dim_eval:n {#1} } {#2} { } { }
  }
\cs_new:Npn \@@_case:nnTF #1#2#3#4
  { \@@_case:nw {#1} #2 {#1} { } \s_@@_mark {#3} \s_@@_mark {#4} \s_@@_stop }
\cs_new:Npn \@@_case:nw #1#2#3
  {
    \dim_compare:nNnTF {#1} = {#2}
      { \@@_case_end:nw {#3} }
      { \@@_case:nw {#1} }
  }
\cs_new:Npn \@@_case_end:nw #1#2#3 \s_@@_mark #4#5 \s_@@_stop
  { \exp_end: #1 #4 }
%    \end{macrocode}
% \end{macro}
% \end{macro}
% \end{macro}
%
% \subsection{Dimension expression loops}
%
% \begin{macro}{\dim_while_do:nn}
% \begin{macro}{\dim_until_do:nn}
% \begin{macro}{\dim_do_while:nn}
% \begin{macro}{\dim_do_until:nn}
%   |while_do| and |do_while| functions for dimensions. Same as for the
%   |int| type only the names have changed.
%    \begin{macrocode}
\cs_new:Npn \dim_while_do:nn #1#2
  {
    \dim_compare:nT {#1}
      {
        #2
        \dim_while_do:nn {#1} {#2}
      }
  }
\cs_new:Npn \dim_until_do:nn #1#2
  {
    \dim_compare:nF {#1}
      {
        #2
        \dim_until_do:nn {#1} {#2}
      }
  }
\cs_new:Npn \dim_do_while:nn #1#2
  {
    #2
    \dim_compare:nT {#1}
      { \dim_do_while:nn {#1} {#2} }
  }
\cs_new:Npn \dim_do_until:nn #1#2
  {
    #2
    \dim_compare:nF {#1}
      { \dim_do_until:nn {#1} {#2} }
  }
%    \end{macrocode}
% \end{macro}
% \end{macro}
% \end{macro}
% \end{macro}
%
% \begin{macro}{\dim_while_do:nNnn}
% \begin{macro}{\dim_until_do:nNnn}
% \begin{macro}{\dim_do_while:nNnn}
% \begin{macro}{\dim_do_until:nNnn}
%   |while_do| and |do_while| functions for dimensions. Same as for the
%   |int| type only the names have changed.
%    \begin{macrocode}
\cs_new:Npn \dim_while_do:nNnn #1#2#3#4
  {
    \dim_compare:nNnT {#1} #2 {#3}
      {
        #4
        \dim_while_do:nNnn {#1} #2 {#3} {#4}
      }
  }
\cs_new:Npn \dim_until_do:nNnn #1#2#3#4
  {
  \dim_compare:nNnF {#1} #2 {#3}
    {
      #4
      \dim_until_do:nNnn {#1} #2 {#3} {#4}
    }
  }
\cs_new:Npn \dim_do_while:nNnn #1#2#3#4
  {
    #4
    \dim_compare:nNnT {#1} #2 {#3}
      { \dim_do_while:nNnn {#1} #2 {#3} {#4} }
  }
\cs_new:Npn \dim_do_until:nNnn #1#2#3#4
  {
    #4
    \dim_compare:nNnF {#1} #2 {#3}
      { \dim_do_until:nNnn {#1} #2 {#3} {#4} }
  }
%    \end{macrocode}
% \end{macro}
% \end{macro}
% \end{macro}
% \end{macro}
%
% \subsection{Dimension step functions}
%
% \begin{macro}{\dim_step_function:nnnN}
% \begin{macro}{\@@_step:wwwN, \@@_step:NnnnN}
%   Before all else, evaluate the initial value, step, and final value.
%   Repeating a function by steps first needs a check on the direction
%   of the steps. After that, do the function for the start value then
%   step and loop around. It would be more symmetrical to test for a
%   step size of zero before checking the sign, but we optimize for the
%   most frequent case (positive step).
%    \begin{macrocode}
\cs_new:Npn \dim_step_function:nnnN #1#2#3
  {
    \exp_after:wN \@@_step:wwwN
    \tex_the:D \@@_eval:w #1 \exp_after:wN \@@_sep:
    \tex_the:D \@@_eval:w #2 \exp_after:wN \@@_sep:
    \tex_the:D \@@_eval:w #3 \@@_sep:
  }
\cs_new:Npn \@@_step:wwwN #1 \@@_sep: #2 \@@_sep: #3 \@@_sep: #4
  {
    \dim_compare:nNnTF {#2} > \c_zero_dim
      { \@@_step:NnnnN > }
      {
        \dim_compare:nNnTF {#2} = \c_zero_dim
          {
            \msg_expandable_error:nnn { kernel } { zero-step } {#4}
            \use_none:nnnn
          }
          { \@@_step:NnnnN < }
      }
      {#1} {#2} {#3} #4
  }
\cs_new:Npn \@@_step:NnnnN #1#2#3#4#5
  {
    \dim_compare:nNnF {#2} #1 {#4}
      {
        #5 {#2}
        \exp_args:NNf \@@_step:NnnnN
          #1 { \dim_eval:n { #2 + #3 } } {#3} {#4} #5
      }
  }
%    \end{macrocode}
% \end{macro}
% \end{macro}
%
% \begin{macro}{\dim_step_inline:nnnn}
% \begin{macro}{\dim_step_variable:nnnNn}
% \begin{macro}{\@@_step:NNnnnn}
%   The approach here is to build a function, with a global integer
%   required to make the nesting safe (as seen in other in line
%   functions), and map that function using \cs{dim_step_function:nnnN}.
%   We put a \cs{prg_break_point:Nn} so that \texttt{map_break}
%   functions from other modules correctly decrement \cs{g__kernel_prg_map_int}
%   before looking for their own break point.  The first argument is
%   \cs{scan_stop:}, so that no breaking function recognizes this break
%   point as its own.
%    \begin{macrocode}
\cs_new_protected:Npn \dim_step_inline:nnnn
  {
    \int_gincr:N \g__kernel_prg_map_int
    \exp_args:NNc \@@_step:NNnnnn
      \cs_gset_protected:Npn
      { @@_map_ \int_use:N \g__kernel_prg_map_int :w }
  }
\cs_new_protected:Npn \dim_step_variable:nnnNn #1#2#3#4#5
  {
    \int_gincr:N \g__kernel_prg_map_int
    \exp_args:NNc \@@_step:NNnnnn
      \cs_gset_protected:Npe
      { @@_map_ \int_use:N \g__kernel_prg_map_int :w }
      {#1}{#2}{#3}
      {
        \tl_set:Nn \exp_not:N #4 {##1}
        \exp_not:n {#5}
      }
  }
\cs_new_protected:Npn \@@_step:NNnnnn #1#2#3#4#5#6
  {
    #1 #2 ##1 {#6}
    \dim_step_function:nnnN {#3} {#4} {#5} #2
    \prg_break_point:Nn \scan_stop: { \int_gdecr:N \g__kernel_prg_map_int }
  }
%    \end{macrocode}
% \end{macro}
% \end{macro}
% \end{macro}
%
% \subsection{Using \texttt{dim} expressions and variables}
%
% \begin{macro}{\dim_eval:n}
%   Evaluating a dimension expression expandably.
%    \begin{macrocode}
\cs_new:Npn \dim_eval:n #1
  { \dim_use:N \@@_eval:w #1 \@@_eval_end: }
%    \end{macrocode}
% \end{macro}
%
% \begin{macro}[EXP]{\dim_sign:n, \@@_sign:Nw}
%   See \cs{dim_abs:n}.  Contrarily to \cs{int_sign:n} the case of a
%   zero dimension cannot be distinguished from a positive dimension by
%   looking only at the first character, since |0.2pt| and |0pt| start
%   the same way.  We need explicit comparisons.  We start by
%   distinguishing the most common case of a positive dimension.
%    \begin{macrocode}
\cs_new:Npn \dim_sign:n #1
  {
    \int_value:w \exp_after:wN \@@_sign:Nw
      \dim_use:N \@@_eval:w #1 \@@_eval_end: \@@_sep:
    \exp_stop_f:
  }
\cs_new:Npn \@@_sign:Nw #1#2 \@@_sep:
  {
    \if_dim:w #1#2 > \c_zero_dim
      1
    \else:
      \if_meaning:w - #1
        -1
      \else:
        0
      \fi:
    \fi:
  }
%    \end{macrocode}
% \end{macro}
%
% \begin{macro}{\dim_use:N, \dim_use:c}
%   Accessing a \meta{dim}.  We hand-code the |c| variant for some speed gain.
%    \begin{macrocode}
\cs_new_eq:NN \dim_use:N \tex_the:D
\cs_new:Npn \dim_use:c #1 { \tex_the:D \cs:w #1 \cs_end: }
%    \end{macrocode}
% \end{macro}
%
% \begin{macro}[EXP]{\dim_to_decimal:n}
% \begin{macro}[EXP]{\@@_to_decimal:w}
%   A function which comes up often enough to deserve a place in the
%   kernel.  Evaluate the dimension expression~|#1| then remove the
%   trailing \texttt{pt}.  When debugging is enabled, the argument is
%   put in parentheses as this prevents the dimension expression from
%   terminating early and leaving extra tokens lying around.  This is
%   used a lot by low-level manipulations.
%    \begin{macrocode}
\cs_new:Npn \dim_to_decimal:n #1
  {
    \exp_after:wN
      \@@_to_decimal:w \dim_use:N \@@_eval:w #1 \@@_eval_end:
  }
\use:e
  {
    \cs_new:Npn \exp_not:N \@@_to_decimal:w
      #1 . #2 \tl_to_str:n { pt }
  }
      {
        \int_compare:nNnTF {#2} > \c_zero_int
          { #1 . #2 }
          { #1 }
      }
%    \end{macrocode}
% \end{macro}
% \end{macro}
%
% \begin{macro}[EXP]{\dim_to_fp:n}
%   Defined in \pkg{l3fp-convert}, documented here.
% \end{macro}
%
% \subsection{Conversion of \texttt{dim} to other units}
%
% The conversion from \texttt{pt} or \texttt{sp} to other units is complicated
% by the fact that \TeX{}'s conversion to \texttt{sp} involves rounding and
% hard-coded ratios. In order to give re-entrant outcomes, we therefore need
% to do quite a bit of work: see
% \url{https://github.com/latex3/latex3/issues/954} for detailed discussion.
% After dealing with the trivial case, we therefore have some work to do.
% The code to do this is contributed by Ruixi Zhang.
%
% \begin{macro}[EXP]{\dim_to_decimal_in_sp:n}
%   The one easy case: the only requirement here is that we avoid an
%   overflow.
%    \begin{macrocode}
\cs_new:Npn \dim_to_decimal_in_sp:n #1
  { \int_value:w \@@_eval:w #1 \@@_eval_end: }
%    \end{macrocode}
% \end{macro}
%
% \begin{macro}[EXP]
%   {
%     \dim_to_decimal_in_bp:n ,
%     \dim_to_decimal_in_cc:n ,
%     \dim_to_decimal_in_cm:n ,
%     \dim_to_decimal_in_dd:n ,
%     \dim_to_decimal_in_in:n ,
%     \dim_to_decimal_in_mm:n ,
%     \dim_to_decimal_in_pc:n
%   }
% \begin{macro}[EXP]{\@@_to_decimal_aux:w}
%   We first set up a helper macro \cs[no-index]{@@_tmp:w} which takes two
%   arguments. The first argument is one of the following engine-defined
%   units: |in|,~|pc|, |cm|, |mm|, |bp|, |dd|, |cc|, |nd|, and~|nc|.
%   The second argument is $\frac{1}{2}\delta^{-1}$ in reduced fraction,
%   where $\delta>1$~is the engine-defined conversion factor for each unit.
%   Note that $\delta$~must be strictly larger than~$1$ for the following
%   algorithm to work.
%
%   Here is how the algorithm works: Suppose that a user inputs a
%   non-negative dimension in a unit that has conversion factor~$\delta>1$.
%   Then this dimension is internally represented as $X$\,sp, where
%   $X=\lfloor N\delta\rfloor$ for some integer $N\ge0$. We then seek a
%   formula to express this $N$ using~$X$.
%   The \cs[no-index]{dim_to_decimal_in_<unit>:n} functions shall return
%   the number $N/2^{16}$ in decimal. This way, we guarantee the returned
%   decimal followed by the original unit will parse to exactly~$X$\,sp.
%
%   So how do we get $N$ from~$X$? Well, since $X=\lfloor N\delta\rfloor$,
%   we have $X\le N\delta<X+1$ and $X\delta^{-1}\le N<(X+1)\delta^{-1}$.
%   Let's focus on the midpoint of this bounding interval for~$N$. The
%   midpoint is $(X+\frac{1}{2})\delta^{-1}$. The fact $\delta>1$ implies
%   that the bounding interval is shorter than~$1$ in length. Thus,
%   (1)~$\hbox{midpoint}+\frac{1}{2}>N$ and
%   (2)~$\hbox{midpoint}+\frac{1}{2}<N+1$. In other words,
%   $N=\lfloor\hbox{midpoint}+\frac{1}{2}\rfloor$. As long as we can
%   rewrite the midpoint as the result of a \enquote{scaling operation} of \eTeX,
%   the $\lfloor\ldots+\frac{1}{2}\rfloor$ part will follow naturally.
%   Indeed we can: $\hbox{midpoint}=(2X+1)\times(\frac{1}{2}\delta^{-1})$.
%
%   Addendum: If $\delta\ge2$, then the bounding interval for~$N$ is at
%   most~$\frac{1}{2}$ wide in length. In this case, the leftpoint
%   $X\delta^{-1}$ suffices as $N=\lfloor X\delta^{-1}+\frac{1}{2}\rfloor$.
%   Six out of the nine units listed above can be handled in this way,
%   which is much simpler than using midpoint. But three remaining units
%   have $1<\delta<2$; they are |bp|~($\delta=7227/7200$),
%   |nd|~($\delta=685/642$), and |dd|~($\delta=1238/1157$),
%   and these three must be handled using midpoint.
%   For consistency, we shall use the midpoint approach for all nine units.
%    \begin{macrocode}
\group_begin:
  \cs_set_protected:Npn \@@_tmp:w #1#2
    {
      \cs_new:cpn { dim_to_decimal_in_ #1 :n } ##1
        {
          \exp_after:wN \@@_to_decimal_aux:w
            \int_value:w \@@_eval:w ##1 \@@_eval_end: \@@_sep: #2 \@@_sep:
        }
    }
%    \end{macrocode}
%   Conversions to other units are now coded.
%   Consult the pdf\/\TeX{} source for each conversion factor~$\delta$.
%   Each factor $\frac{1}{2}\delta^{-1}$ is hand-coded
%   for accuracy (and speed). As the units \texttt{nc} and \texttt{nd}
%   are not supported by \XeTeX{} or (u)p\TeX{}, they are not included
%   here.
%    \begin{macrocode}
  \@@_tmp:w { in } {   50 /  7227 } % delta = 7227/100
  \@@_tmp:w { pc } {    1 /    24 } % delta = 12/1
  \@@_tmp:w { cm } {  127 /  7227 } % delta = 7227/254
  \@@_tmp:w { mm } { 1270 /  7227 } % delta = 7227/2540
  \@@_tmp:w { bp } {  400 /   803 } % delta = 7227/7200
  \@@_tmp:w { dd } { 1157 /  2476 } % delta = 1238/1157
  \@@_tmp:w { cc } { 1157 / 29712 } % delta = 14856/1157
\group_end:
%    \end{macrocode}
%   The tokens after \cs{@@_to_decimal_aux:w} shall have the following form:
%   |<number>|\cs{@@_sep:}|<half of delta inverse>|\cs{@@_sep:}, where
%   |<number>| represents the input dimension in |sp| unit.
%   If |<number>| is positive, then |#1| is its leading digit and |#2|
%   (possibly empty) is all the remaining digits;
%   If |<number>| is zero, then |#1| is~|0|$_{12}$ and |#2| is empty;
%   If |<number>| is negative, then |#1| is its sign~|-|$_{12}$ and |#2|
%   is all its digits.
%   In all three cases, |#1#2| is the original |<number>|. We can use |#1|
%   to decide whether to use the |-1| formula or the |+1| formula.
%    \begin{macrocode}
\cs_new:Npn \@@_to_decimal_aux:w #1#2 \@@_sep: #3 \@@_sep:
  {
    \dim_to_decimal:n
      {
%    \end{macrocode}
%   We need different formulae depending on whether the user input dimension
%   is negative or not.
%   For negative dimension (internally represented as $X$\,sp), the formula
%   is $(2X-1)\times(\frac{1}{2}\delta^{-1})$.
%   For non-negative dimension, the formula
%   is $(2X+1)\times(\frac{1}{2}\delta^{-1})$.
%   The intermediate step doubles the dimension~$X$.
%   To avoid overflow, we must invoke \cs[no-index]{int_eval:n}.
%    \begin{macrocode}
        \int_eval:n
          { ( 2 * #1#2 \if:w #1 - - \else: + \fi: 1 ) * #3 }
%    \end{macrocode}
%   Now we append~|sp| to finish the dimension specification.
%    \begin{macrocode}
        sp
      }
  }
%    \end{macrocode}
% \end{macro}
% \end{macro}
%
% \begin{macro}[EXP]{\dim_to_decimal_in_unit:nn}
%    \begin{macrocode}
\cs_new:Npn \dim_to_decimal_in_unit:nn #1#2
  {
    \exp_after:wN \@@_chk_unit:w
      \int_value:w \@@_eval:w #2 \@@_eval_end: \@@_sep: {#1}
  }
%    \end{macrocode}
% \end{macro}
% \begin{macro}[EXP]{\@@_chk_unit:w}
%   The tokens after \cs{@@_chk_unit:w} shall have the following form:
%   |<number2>|\cs{@@_sep:}|{<dimexpr1>}|, where |<number2>| represents
%   |<dimexpr2>| in |sp| unit.
%   If |#1| is~|0|$_{12}$, the \enquote{unit} |<dimexpr2>| must also be zero.
%   So we throw out a \enquote{division by zero} error message at this point.
%   Otherwise, if |#1| is~|-|$_{12}$, we shall negate both |<dimexpr1>| and
%   |<dimexpr2>| for later procedures.
%    \begin{macrocode}
\cs_new:Npn \@@_chk_unit:w #1#2 \@@_sep: #3
  {
    \token_if_eq_charcode:NNTF #1 0
      { \msg_expandable_error:nn { dim } { zero-unit } }
      {
        \exp_after:wN \@@_branch_unit:w
          \int_value:w \if:w #1 - - \fi: \@@_eval:w #3 \exp_after:wN \@@_sep:
          \int_value:w \if:w #1 - - \fi: #1#2 \@@_sep:
      }
  }
%    \end{macrocode}
% \end{macro}
% \begin{macro}[EXP]{\@@_branch_unit:w}
%   The tokens after \cs{@@_branch_unit:w} shall have the following form:
%   |<number1>|\cs{@@_sep:}|<number2>|\cs{@@_sep:}, where |<number1>|
%   represents |<dimexpr1>| in
%   |sp| unit (whose sign is taken care of) and |<number2>| represents the
%   absolute value of |<dimexpr2>| in |sp| unit (which is strictly positive).
%
%   As explained, the formulae $(2X\pm1)\times(\frac{1}{2}\delta^{-1})$ work
%   if and only if $\delta=|<number2>|/65536>1$. This corresponds to
%   |<dimexpr2>| strictly larger than 1\,pt in absolute value.
%   In this case, we simply call \cs{@@_to_decimal_aux:w} and supply
%   $\frac{1}{2}\delta^{-1}=32768/|<number2>|$ as |<half of delta inverse>|.
%
%   Otherwise if $|<number2>|=65536$, then |<dimexpr2>| is 1\,pt in absolute
%   value and we call \cs{dim_to_decimal:n} directly.
%
%   Otherwise $0<|<number2>|<65536$ and we shall proceed differently.
%
%   For unit less than 1\,pt, write $n=|<number2>|$, then $\delta=n/65536<1$.
%   The midpoint formulae are not optimal. Let's go back to the inequalities
%   $X\delta^{-1}\le N<(X+1)\delta^{-1}$. Since now $\delta<1$, the bounding
%   interval is wider than~$1$ in length. Consider the ceiling integer
%   $M=\lceil X\delta^{-1}\rceil$, then $X\delta^{-1}\le M<(X+1)\delta^{-1}$,
%   or equivalently $X\le M\delta<X+1$, and thus $\lfloor M\delta\rfloor=X$.
%   The key point here is that we \emph{don't} need to solve for~$N$;
%   in fact, any integer that can reproduce~$X$ (such as~$M$) is good enough.
%   So the algorithm goes like this: (1)~Compute rounding of $X\delta^{-1}$,
%   i.e., $M'=\lfloor X\delta^{-1}+\frac{1}{2}\rfloor$; this $M'$ could be
%   either $M$ or $M-1$. (2)~Check if $\lfloor M'\delta\rfloor=X$, i.e.,
%   whether our candidate $M'$ can reproduce~$X$. If so, then this $M'$ is
%   good enough; if not, then we add one to~$M'$.
%
%   But when $0<n<65536$, we cannot delay the problem of overflow any more.
%   For $X\delta^{-1}=X\times65536/n$, where $X$ can go up to $2^{30}-1$ and
%   $n$ can be as small as~$1$, the result is well over $2^{31}-1$ (largest
%   integer allowed within |\numexpr|).
%   For example, |\dim_to_decimal_in_unit:nn { \maxdimen } { 1sp }|.
%   Here, all inputs are legal, so we should be able to output |1073741823|
%   \emph{without} causing arithmetic overflow.
%
%   As a workaround, let's write $X=qn+r$ with some $q\ge0$ and $0\le r<n$.
%   Then $X\delta^{-1}=65536q+65536r/n$, and so
%   $M'=65536q+\lfloor65536r/n+\frac{1}{2}\rfloor=65536q+R'$.
%   Computing $R'$ will never overflow. If this $R'$ can reproduce~$r$, then
%   it is good enough; otherwise we add one to~$R'$. In the end, we shall
%   output $q+R'/65536$ in decimal.
%
%   Note: $q=\lfloor X/n\rfloor=\lfloor\frac{2X-n}{2n}+\frac{1}{2}\rfloor$
%   represents the \enquote{integer} part, while $0\le R'\le65536$ represents the
%   \enquote{fractional} part. (Can $R'=65536$ really happen? Didn't investigate.)
%    \begin{macrocode}
\cs_new:Npn \@@_branch_unit:w #1 \@@_sep: #2 \@@_sep:
  {
    \int_compare:nNnTF {#2} > { 65536 }
      { \@@_to_decimal_aux:w #1 \@@_sep: 32768 / #2 \@@_sep: }
      {
        \int_compare:nNnTF {#2} = { 65536 }
          { \dim_to_decimal:n { #1sp } }
          { \@@_get_quotient:w #1 \@@_sep: #2 \@@_sep: }
      }
  }
%    \end{macrocode}
% \end{macro}
% \begin{macro}[EXP]{\@@_get_quotient:w}
%   We wish to get the quotient $q$ via rounding of $\frac{2X-n}{2n}$.
%   When $0\le X<n/2$, we have $\frac{2X-n}{2n}<0$. So, strictly speaking,
%   |\numexpr| performs its rounding as
%   $\lceil\frac{2X-n}{2n}-\frac{1}{2}\rceil$, not exactly what we want.
%   However, lucky for us, only $X=0$ makes
%   $\lceil\frac{2X-n}{2n}-\frac{1}{2}\rceil=-1\neq0$ (we want~$0$);
%   all other $0<X<n/2$ make $\lceil\frac{2X-n}{2n}-\frac{1}{2}\rceil=0=q$.
%   Thus, let's filter out $X=0$ early.
%   If $X\neq0$, we extract its sign and leave the sign to the back.
%   The sign does not participate in any calculations (also the code works
%   with positive integers only). The sign is used at the last stages when
%   we parse the decimal output.
%
%   After \cs{@@_get_quotient:w} has done its job, either we have the
%   decimal~|0|, or we have \cs{@@_get_remainder:w} followed by
%   $q$\cs{@@_sep:}$\lvert X\rvert$\cs{@@_sep:}$n$\cs{@@_sep:}|<sign of X>|\cs{@@_sep:}.
%    \begin{macrocode}
\cs_new:Npn \@@_get_quotient:w #1#2 \@@_sep: #3 \@@_sep:
  {
    \token_if_eq_charcode:NNTF #1 0
      { 0 }
      {
        \token_if_eq_charcode:NNTF #1 -
          {
            \exp_after:wN \exp_after:wN \exp_after:wN \@@_get_remainder:w
              \int_eval:n {  ( 2 * #2 - #3 ) / ( 2 * #3 ) } \@@_sep:
              #2 \@@_sep: #3 \@@_sep: - \@@_sep:
          }
          {
            \exp_after:wN \exp_after:wN \exp_after:wN \@@_get_remainder:w
              \int_eval:n { ( 2 * #1#2 - #3 ) / ( 2 * #3 ) } \@@_sep:
              #1#2 \@@_sep: #3 \@@_sep: \@@_sep:
          }
      }
  }
%    \end{macrocode}
% \end{macro}
% \begin{macro}[EXP]{\@@_get_remainder:w}
%   \cs{@@_get_remainder:w} does not need to read the sign.
%   After finding the remainder~$r$, the number~$\lvert X\rvert$ is no longer
%   needed. We should then have \cs{@@_convert_remainder:w} followed by
%   $r$\cs{@@_sep:}$n$\cs{@@_sep:}$q$\cs{@@_sep:}|<sign of X>|\cs{@@_sep:}.
%    \begin{macrocode}
\cs_new:Npn \@@_get_remainder:w #1 \@@_sep: #2 \@@_sep: #3 \@@_sep:
  {
    \exp_after:wN \exp_after:wN \exp_after:wN \@@_convert_remainder:w
      \int_eval:n { #2 - #1 * #3 } \@@_sep:
      #3 \@@_sep: #1 \@@_sep:
  }
%    \end{macrocode}
% \end{macro}
% \begin{macro}[EXP]{\@@_convert_remainder:w}
%   This is trivial. We compute $R'=\lfloor65536r/n+\frac{1}{2}\rfloor$,
%   then leave \cs{@@_test_candidate:w} followed by
%   $R'$\cs{@@_sep:}$r$\cs{@@_sep:}$n$\cs{@@_sep:}$q$\cs{@@_sep:}|<sign of X>|\cs{@@_sep:}.
%    \begin{macrocode}
\cs_new:Npn \@@_convert_remainder:w #1 \@@_sep: #2 \@@_sep:
  {
    \exp_after:wN \exp_after:wN \exp_after:wN \@@_test_candidate:w
      \int_eval:n { #1 * 65536 / #2 } \@@_sep:
      #1 \@@_sep: #2 \@@_sep:
  }
%    \end{macrocode}
% \end{macro}
% \begin{macro}[EXP]{\@@_test_candidate:w}
%   Now the fun part: We take $R'$, $r$ and~$n$ to test whether
%   $r=\lfloor R'\delta\rfloor$. This is done as a dimension comparison.
%   The left-hand side, $r$, is simply |r sp|. The right-hand side,
%   $\lfloor R'\delta\rfloor$, is exactly |<R' as decimal><dimen = n sp>|.
%   If the result is true, then we've found~$R'$;
%   otherwise we add one to~$R'$.
%   After this step, $r$ and~$n$ are no longer needed. We should then have
%   \cs{@@_parse_decimal:w} followed by $R'$\cs{@@_sep:}$q$\cs{@@_sep:}|<sign of X>|\cs{@@_sep:}.
%    \begin{macrocode}
\cs_new:Npn \@@_test_candidate:w #1 \@@_sep: #2 \@@_sep: #3 \@@_sep:
  {
    \dim_compare:nNnTF { #2sp } =
      { \dim_to_decimal:n { #1sp } \@@_eval:w #3sp \@@_eval_end: }
      { \@@_parse_decimal:w #1 \@@_sep: }
      {
        \@@_parse_decimal:w \int_eval:n { #1 + 1 } \@@_sep:
      }
  }
%    \end{macrocode}
% \end{macro}
% \begin{macro}[EXP]{\@@_parse_decimal:w, \@@_parse_decimal_aux:w}
%   The Grand Finale: We sum $q$ and $R'/65536$ together, and negate the
%   result if necessary. These are all done expandably.
%   If $0<R'/65536<1$, the integer summation is naturally terminated at the
%   decimal point. If $R'/65536=0$ (or~$1$?), the summation is terminated
%   at the \cs{@@_sep:}. The auxiliary function \cs{@@_parse_decimal_aux:w}
%   takes care of both cases.
%    \begin{macrocode}
\cs_new:Npn \@@_parse_decimal:w #1 \@@_sep: #2 \@@_sep: #3 \@@_sep:
  {
    \exp_after:wN \@@_parse_decimal_aux:w
      \int_value:w #3 \int_eval:w #2 + \dim_to_decimal:n { #1sp } \@@_sep:
  }
\cs_new:Npn \@@_parse_decimal_aux:w #1 \@@_sep: {#1}
%    \end{macrocode}
% \end{macro}
%
% \subsection{Viewing \texttt{dim} variables}
%
% \begin{macro}{\dim_show:N, \dim_show:c}
%   Diagnostics.
%    \begin{macrocode}
\cs_new_eq:NN  \dim_show:N \__kernel_register_show:N
\cs_generate_variant:Nn \dim_show:N { c }
%    \end{macrocode}
%  \end{macro}
%
% \begin{macro}{\dim_show:n}
%   Diagnostics.  We don't use the \TeX{} primitive \tn{showthe} to show
%   dimension expressions: this gives a more unified output.
%    \begin{macrocode}
\cs_new_protected:Npn \dim_show:n
  { \__kernel_msg_show_eval:Nn \dim_eval:n }
%    \end{macrocode}
%  \end{macro}
%
% \begin{macro}{\dim_log:N, \dim_log:c, \dim_log:n}
%   Diagnostics.  Redirect output of \cs{dim_show:n} to the log.
%    \begin{macrocode}
\cs_new_eq:NN \dim_log:N \__kernel_register_log:N
\cs_new_eq:NN \dim_log:c \__kernel_register_log:c
\cs_new_protected:Npn \dim_log:n
  { \__kernel_msg_log_eval:Nn \dim_eval:n }
%    \end{macrocode}
% \end{macro}
%
% \subsection{Constant dimensions}
%
% \begin{variable}{\c_zero_dim, \c_max_dim}
%   Constant dimensions.
%    \begin{macrocode}
\dim_const:Nn \c_zero_dim { 0 pt }
\dim_const:Nn \c_max_dim { 16383.99999 pt }
%    \end{macrocode}
% \end{variable}
%
% \subsection{Scratch dimensions}
%
% \begin{variable}{\l_tmpa_dim, \l_tmpb_dim}
% \begin{variable}{\g_tmpa_dim, \g_tmpb_dim}
%    We provide two local and two global scratch registers, maybe we
%    need more or less.
%    \begin{macrocode}
\dim_new:N \l_tmpa_dim
\dim_new:N \l_tmpb_dim
\dim_new:N \g_tmpa_dim
\dim_new:N \g_tmpb_dim
%    \end{macrocode}
% \end{variable}
% \end{variable}
%
% \subsection{Creating and initializing \texttt{skip} variables}
%
%    \begin{macrocode}
%<@@=skip>
%    \end{macrocode}
%
% \begin{variable}{\s_@@_stop}
%   Internal scan marks.
%    \begin{macrocode}
\scan_new:N \s_@@_stop
%    \end{macrocode}
% \end{variable}
%
% \begin{macro}{\skip_new:N, \skip_new:c}
%    Allocation of a new internal registers.
%    \begin{macrocode}
\cs_new_protected:Npn \skip_new:N #1
  {
    \__kernel_chk_if_free_cs:N #1
    \cs:w newskip \cs_end: #1
  }
\cs_generate_variant:Nn \skip_new:N { c }
%    \end{macrocode}
% \end{macro}
%
% \begin{macro}{\skip_const:Nn, \skip_const:cn}
%   Contrarily to integer constants, we cannot avoid using a register,
%   even for constants.  See \cs{dim_const:Nn} for why we cannot use
%   \cs{skip_gset:Nn}.
%    \begin{macrocode}
\cs_new_protected:Npn \skip_const:Nn #1#2
  {
    \skip_new:N #1
    \tex_global:D #1 = \skip_eval:n {#2} \scan_stop:
  }
\cs_generate_variant:Nn \skip_const:Nn { c }
%    \end{macrocode}
% \end{macro}
%
% \begin{macro}{\skip_zero:N, \skip_zero:c}
% \begin{macro}{\skip_gzero:N, \skip_gzero:c}
%   Reset the register to zero.
%    \begin{macrocode}
\cs_new_eq:NN \skip_zero:N  \dim_zero:N
\cs_new_eq:NN \skip_gzero:N \dim_gzero:N
\cs_generate_variant:Nn \skip_zero:N  { c }
\cs_generate_variant:Nn \skip_gzero:N { c }
%    \end{macrocode}
% \end{macro}
% \end{macro}
%
% \begin{macro}
%   {\skip_zero_new:N, \skip_zero_new:c, \skip_gzero_new:N, \skip_gzero_new:c}
%   Create a register if needed, otherwise clear it.
%    \begin{macrocode}
\cs_new_protected:Npn \skip_zero_new:N  #1
  { \skip_if_exist:NTF #1 { \skip_zero:N #1 } { \skip_new:N #1 } }
\cs_new_protected:Npn \skip_gzero_new:N #1
  { \skip_if_exist:NTF #1 { \skip_gzero:N #1 } { \skip_new:N #1 } }
\cs_generate_variant:Nn \skip_zero_new:N  { c }
\cs_generate_variant:Nn \skip_gzero_new:N { c }
%    \end{macrocode}
% \end{macro}
%
% \begin{macro}[pTF]{\skip_if_exist:N, \skip_if_exist:c}
%   Copies of the \texttt{cs} functions defined in \pkg{l3basics}.
%    \begin{macrocode}
\prg_new_eq_conditional:NNn \skip_if_exist:N \cs_if_exist:N
  { TF , T , F , p }
\prg_new_eq_conditional:NNn \skip_if_exist:c \cs_if_exist:c
  { TF , T , F , p }
%    \end{macrocode}
% \end{macro}
%
% \subsection{Setting \texttt{skip} variables}
%
% \begin{macro}{\skip_set:Nn, \skip_set:cn, \skip_set:NV, \skip_set:cV}
% \begin{macro}{\skip_gset:Nn, \skip_gset:cn, \skip_gset:NV, \skip_gset:cV}
%   Much the same as for dimensions.
%    \begin{macrocode}
\cs_new_protected:Npn \skip_set:Nn #1#2
  { #1 = \tex_glueexpr:D #2 \scan_stop: }
\cs_new_protected:Npn \skip_gset:Nn #1#2
  { \tex_global:D #1 = \tex_glueexpr:D #2 \scan_stop: }
\cs_generate_variant:Nn \skip_set:Nn  { NV , c , cV }
\cs_generate_variant:Nn \skip_gset:Nn { NV , c , cV }
%    \end{macrocode}
% \end{macro}
% \end{macro}
%
% \begin{macro}
%   {\skip_set_eq:NN, \skip_set_eq:cN, \skip_set_eq:Nc, \skip_set_eq:cc}
% \begin{macro}
%   {\skip_gset_eq:NN, \skip_gset_eq:cN, \skip_gset_eq:Nc, \skip_gset_eq:cc}
%   All straightforward.
%    \begin{macrocode}
\cs_new_protected:Npn \skip_set_eq:NN #1#2 { #1 = #2 }
\cs_generate_variant:Nn \skip_set_eq:NN { c , Nc , cc }
\cs_new_protected:Npn \skip_gset_eq:NN #1#2 { \tex_global:D #1 = #2 }
\cs_generate_variant:Nn \skip_gset_eq:NN { c , Nc , cc }
%    \end{macrocode}
% \end{macro}
% \end{macro}
%
% \begin{macro}{\skip_add:Nn, \skip_add:cn}
% \begin{macro}{\skip_gadd:Nn, \skip_gadd:cn}
% \begin{macro}{\skip_sub:Nn, \skip_sub:cn}
% \begin{macro}{\skip_gsub:Nn, \skip_gsub:cn}
%   Using |by| here deals with the (incorrect) case |\skip123|.
%    \begin{macrocode}
\cs_new_protected:Npn \skip_add:Nn #1#2
  { \tex_advance:D #1 \tex_glueexpr:D #2 \scan_stop: }
\cs_new_protected:Npn \skip_gadd:Nn #1#2
  { \tex_global:D \tex_advance:D #1 \tex_glueexpr:D #2 \scan_stop: }
\cs_generate_variant:Nn \skip_add:Nn  { c }
\cs_generate_variant:Nn \skip_gadd:Nn { c }
\cs_new_protected:Npn \skip_sub:Nn #1#2
  { \tex_advance:D #1 - \tex_glueexpr:D #2 \scan_stop: }
\cs_new_protected:Npn \skip_gsub:Nn #1#2
  { \tex_global:D \tex_advance:D #1 - \tex_glueexpr:D #2 \scan_stop: }
\cs_generate_variant:Nn \skip_sub:Nn  { c }
\cs_generate_variant:Nn \skip_gsub:Nn { c }
%    \end{macrocode}
% \end{macro}
% \end{macro}
% \end{macro}
% \end{macro}
%
% \subsection{Skip expression conditionals}
%
% \begin{macro}[pTF]{\skip_if_eq:nn}
%   Comparing skips means doing two expansions to make strings, and then
%   testing them. As a result, only equality is tested.
%    \begin{macrocode}
\prg_new_conditional:Npnn \skip_if_eq:nn #1#2 { p , T , F , TF }
  {
    \str_if_eq:eeTF { \skip_eval:n {#1} } { \skip_eval:n {#2} }
      { \prg_return_true: }
      { \prg_return_false: }
  }
%    \end{macrocode}
% \end{macro}
%
% \begin{macro}{\@@_sep:}
%    \begin{macrocode}
\cs_new_eq:NN \@@_sep: \__kernel_int_sep:
%    \end{macrocode}
% \end{macro}
%
% \begin{macro}[EXP,pTF]{\skip_if_finite:n}
% \begin{macro}[EXP]{\@@_if_finite:wwNw}
%   With \eTeX{}, we have an easy access to the order of infinities of
%   the stretch and shrink components of a skip.  However, to access
%   both, we either need to evaluate the expression twice, or evaluate
%   it, then call an auxiliary to extract both pieces of information
%   from the result.  Since we are going to need an auxiliary anyways,
%   it is quicker to make it search for the string \texttt{fil} which
%   characterizes infinite glue.
%    \begin{macrocode}
\cs_set_protected:Npn \@@_tmp:w #1
  {
    \prg_new_conditional:Npnn \skip_if_finite:n ##1 { p , T , F , TF }
      {
        \exp_after:wN \@@_if_finite:wwNw
        \skip_use:N \tex_glueexpr:D ##1 \@@_sep: \prg_return_false:
        #1 \@@_sep: \prg_return_true: \s_@@_stop
      }
    \cs_new:Npn \@@_if_finite:wwNw ##1 #1 ##2 \@@_sep: ##3 ##4 \s_@@_stop {##3}
  }
\exp_args:No \@@_tmp:w { \tl_to_str:n { fil } }
%    \end{macrocode}
% \end{macro}
% \end{macro}
%
% \subsection{Using \texttt{skip} expressions and variables}
%
% \begin{macro}{\skip_eval:n}
%   Evaluating a skip expression expandably.
%    \begin{macrocode}
\cs_new:Npn \skip_eval:n #1
  { \skip_use:N \tex_glueexpr:D #1 \scan_stop: }
%    \end{macrocode}
% \end{macro}
%
% \begin{macro}{\skip_use:N, \skip_use:c}
%   Accessing a \meta{skip}.
%    \begin{macrocode}
\cs_new_eq:NN \skip_use:N \dim_use:N
\cs_new_eq:NN \skip_use:c \dim_use:c
%    \end{macrocode}
% \end{macro}
%
% \subsection{Inserting skips into the output}
%
% \begin{macro}{\skip_horizontal:N, \skip_horizontal:c, \skip_horizontal:n}
% \begin{macro}{\skip_vertical:N, \skip_vertical:c, \skip_vertical:n}
%    Inserting skips.
%    \begin{macrocode}
\cs_new_eq:NN  \skip_horizontal:N \tex_hskip:D
\cs_new:Npn \skip_horizontal:n #1
  { \skip_horizontal:N \tex_glueexpr:D #1 \scan_stop: }
\cs_new_eq:NN  \skip_vertical:N \tex_vskip:D
\cs_new:Npn \skip_vertical:n #1
  { \skip_vertical:N \tex_glueexpr:D #1 \scan_stop: }
\cs_generate_variant:Nn \skip_horizontal:N { c }
\cs_generate_variant:Nn \skip_vertical:N { c }
%    \end{macrocode}
% \end{macro}
% \end{macro}
%
% \subsection{Viewing \texttt{skip} variables}
%
% \begin{macro}{\skip_show:N, \skip_show:c}
%   Diagnostics.
%    \begin{macrocode}
\cs_new_eq:NN  \skip_show:N \__kernel_register_show:N
\cs_generate_variant:Nn \skip_show:N { c }
%    \end{macrocode}
%  \end{macro}
%
% \begin{macro}{\skip_show:n}
%   Diagnostics.  We don't use the \TeX{} primitive \tn{showthe} to show
%   skip expressions: this gives a more unified output.
%    \begin{macrocode}
\cs_new_protected:Npn \skip_show:n
  { \__kernel_msg_show_eval:Nn \skip_eval:n }
%    \end{macrocode}
% \end{macro}
%
% \begin{macro}{\skip_log:N, \skip_log:c, \skip_log:n}
%   Diagnostics.  Redirect output of \cs{skip_show:n} to the log.
%    \begin{macrocode}
\cs_new_eq:NN \skip_log:N \__kernel_register_log:N
\cs_new_eq:NN \skip_log:c \__kernel_register_log:c
\cs_new_protected:Npn \skip_log:n
  { \__kernel_msg_log_eval:Nn \skip_eval:n }
%    \end{macrocode}
% \end{macro}
%
% \subsection{Constant skips}
%
% \begin{macro}{\c_zero_skip, \c_max_skip}
%   Skips with no rubber component are just dimensions but need to terminate
%   correctly.
%    \begin{macrocode}
\skip_const:Nn \c_zero_skip { \c_zero_dim }
\skip_const:Nn \c_max_skip { \c_max_dim }
%    \end{macrocode}
% \end{macro}
%
% \subsection{Scratch skips}
%
% \begin{variable}{\l_tmpa_skip, \l_tmpb_skip}
% \begin{variable}{\g_tmpa_skip, \g_tmpb_skip}
%    We provide two local and two global scratch registers, maybe we
%    need more or less.
%    \begin{macrocode}
\skip_new:N \l_tmpa_skip
\skip_new:N \l_tmpb_skip
\skip_new:N \g_tmpa_skip
\skip_new:N \g_tmpb_skip
%    \end{macrocode}
% \end{variable}
% \end{variable}
%
% \subsection{Creating and initializing \texttt{muskip} variables}
%
% \begin{macro}{\muskip_new:N, \muskip_new:c}
% And then we add muskips.
%    \begin{macrocode}
\cs_new_protected:Npn \muskip_new:N #1
  {
    \__kernel_chk_if_free_cs:N #1
    \cs:w newmuskip \cs_end: #1
  }
\cs_generate_variant:Nn \muskip_new:N { c }
%    \end{macrocode}
% \end{macro}
%
% \begin{macro}{\muskip_const:Nn, \muskip_const:cn}
%   See \cs{skip_const:Nn}.
%    \begin{macrocode}
\cs_new_protected:Npn \muskip_const:Nn #1#2
  {
    \muskip_new:N #1
    \tex_global:D #1 = \muskip_eval:n {#2} \scan_stop:
  }
\cs_generate_variant:Nn \muskip_const:Nn { c }
%    \end{macrocode}
% \end{macro}
%
% \begin{macro}{\muskip_zero:N, \muskip_zero:c}
% \begin{macro}{\muskip_gzero:N, \muskip_gzero:c}
%   Reset the register to zero.
%    \begin{macrocode}
\cs_new_protected:Npn \muskip_zero:N #1
  { #1 = \c_zero_muskip }
\cs_new_protected:Npn \muskip_gzero:N #1
  { \tex_global:D #1 = \c_zero_muskip }
\cs_generate_variant:Nn \muskip_zero:N  { c }
\cs_generate_variant:Nn \muskip_gzero:N { c }
%    \end{macrocode}
% \end{macro}
% \end{macro}
%
% \begin{macro}
%   {
%     \muskip_zero_new:N, \muskip_zero_new:c,
%     \muskip_gzero_new:N, \muskip_gzero_new:c
%   }
%   Create a register if needed, otherwise clear it.
%    \begin{macrocode}
\cs_new_protected:Npn \muskip_zero_new:N  #1
  { \muskip_if_exist:NTF #1 { \muskip_zero:N #1 } { \muskip_new:N #1 } }
\cs_new_protected:Npn \muskip_gzero_new:N #1
  { \muskip_if_exist:NTF #1 { \muskip_gzero:N #1 } { \muskip_new:N #1 } }
\cs_generate_variant:Nn \muskip_zero_new:N  { c }
\cs_generate_variant:Nn \muskip_gzero_new:N { c }
%    \end{macrocode}
% \end{macro}
%
% \begin{macro}[pTF]{\muskip_if_exist:N, \muskip_if_exist:c}
%   Copies of the \texttt{cs} functions defined in \pkg{l3basics}.
%    \begin{macrocode}
\prg_new_eq_conditional:NNn \muskip_if_exist:N \cs_if_exist:N
  { TF , T , F , p }
\prg_new_eq_conditional:NNn \muskip_if_exist:c \cs_if_exist:c
  { TF , T , F , p }
%    \end{macrocode}
% \end{macro}
%
% \subsection{Setting \texttt{muskip} variables}
%
% \begin{macro}{\muskip_set:Nn, \muskip_set:cn, \muskip_set:NV, \muskip_set:cV}
% \begin{macro}{\muskip_gset:Nn, \muskip_gset:cn, \muskip_gset:NV, \muskip_gset:cV}
%   This should be pretty familiar.
%    \begin{macrocode}
\cs_new_protected:Npn \muskip_set:Nn #1#2
  { #1 = \tex_muexpr:D #2 \scan_stop: }
\cs_new_protected:Npn \muskip_gset:Nn #1#2
  { \tex_global:D #1 = \tex_muexpr:D #2 \scan_stop: }
\cs_generate_variant:Nn \muskip_set:Nn  { NV , c , cV }
\cs_generate_variant:Nn \muskip_gset:Nn { NV , c , cV }
%    \end{macrocode}
% \end{macro}
% \end{macro}
%
% \begin{macro}
%   {
%     \muskip_set_eq:NN, \muskip_set_eq:cN,
%     \muskip_set_eq:Nc, \muskip_set_eq:cc
%   }
% \begin{macro}
%   {
%     \muskip_gset_eq:NN, \muskip_gset_eq:cN,
%     \muskip_gset_eq:Nc, \muskip_gset_eq:cc
%   }
%   All straightforward.
%    \begin{macrocode}
\cs_new_protected:Npn \muskip_set_eq:NN #1#2 { #1 = #2 }
\cs_generate_variant:Nn \muskip_set_eq:NN { c , Nc , cc }
\cs_new_protected:Npn \muskip_gset_eq:NN #1#2 { \tex_global:D #1 = #2 }
\cs_generate_variant:Nn \muskip_gset_eq:NN { c , Nc , cc }
%    \end{macrocode}
% \end{macro}
% \end{macro}
%
% \begin{macro}{\muskip_add:Nn, \muskip_add:cn}
% \begin{macro}{\muskip_gadd:Nn, \muskip_gadd:cn}
% \begin{macro}{\muskip_sub:Nn, \muskip_sub:cn}
% \begin{macro}{\muskip_gsub:Nn, \muskip_gsub:cn}
%   Using |by| here deals with the (incorrect) case |\muskip123|.
%    \begin{macrocode}
\cs_new_protected:Npn \muskip_add:Nn #1#2
  { \tex_advance:D #1 \tex_muexpr:D #2 \scan_stop: }
\cs_new_protected:Npn \muskip_gadd:Nn #1#2
  { \tex_global:D \tex_advance:D #1 \tex_muexpr:D #2 \scan_stop: }
\cs_generate_variant:Nn \muskip_add:Nn  { c }
\cs_generate_variant:Nn \muskip_gadd:Nn { c }
\cs_new_protected:Npn \muskip_sub:Nn #1#2
  { \tex_advance:D #1 - \tex_muexpr:D #2 \scan_stop: }
\cs_new_protected:Npn \muskip_gsub:Nn #1#2
  { \tex_global:D \tex_advance:D #1 - \tex_muexpr:D #2 \scan_stop: }
\cs_generate_variant:Nn \muskip_sub:Nn  { c }
\cs_generate_variant:Nn \muskip_gsub:Nn { c }
%    \end{macrocode}
% \end{macro}
% \end{macro}
% \end{macro}
% \end{macro}
%
% \subsection{Using \texttt{muskip} expressions and variables}
%
% \begin{macro}{\muskip_eval:n}
%   Evaluating a muskip expression expandably.
%    \begin{macrocode}
\cs_new:Npn \muskip_eval:n #1
  { \muskip_use:N \tex_muexpr:D #1 \scan_stop: }
%    \end{macrocode}
% \end{macro}
%
% \begin{macro}{\muskip_use:N, \muskip_use:c}
%   Accessing a \meta{muskip}.
%    \begin{macrocode}
\cs_new_eq:NN \muskip_use:N \dim_use:N
\cs_new_eq:NN \muskip_use:c \dim_use:c
%    \end{macrocode}
% \end{macro}
%
% \subsection{Viewing \texttt{muskip} variables}
%
% \begin{macro}{\muskip_show:N, \muskip_show:c}
%   Diagnostics.
%    \begin{macrocode}
\cs_new_eq:NN  \muskip_show:N \__kernel_register_show:N
\cs_generate_variant:Nn \muskip_show:N { c }
%    \end{macrocode}
% \end{macro}
%
% \begin{macro}{\muskip_show:n}
%   Diagnostics.  We don't use the \TeX{} primitive \tn{showthe} to show
%   muskip expressions: this gives a more unified output.
%    \begin{macrocode}
\cs_new_protected:Npn \muskip_show:n
  { \__kernel_msg_show_eval:Nn \muskip_eval:n }
%    \end{macrocode}
% \end{macro}
%
% \begin{macro}{\muskip_log:N, \muskip_log:c, \muskip_log:n}
%   Diagnostics.  Redirect output of \cs{muskip_show:n} to the log.
%    \begin{macrocode}
\cs_new_eq:NN \muskip_log:N \__kernel_register_log:N
\cs_new_eq:NN \muskip_log:c \__kernel_register_log:c
\cs_new_protected:Npn \muskip_log:n
  { \__kernel_msg_log_eval:Nn \muskip_eval:n }
%    \end{macrocode}
% \end{macro}
%
% \subsection{Constant muskips}
%
% \begin{macro}{\c_zero_muskip}
% \begin{macro}{\c_max_muskip}
%   Constant muskips given by their value.
%    \begin{macrocode}
\muskip_const:Nn \c_zero_muskip { 0 mu }
\muskip_const:Nn \c_max_muskip  { 16383.99999 mu }
%    \end{macrocode}
% \end{macro}
% \end{macro}
%
% \subsection{Scratch muskips}
%
% \begin{variable}{\l_tmpa_muskip, \l_tmpb_muskip}
% \begin{variable}{\g_tmpa_muskip, \g_tmpb_muskip}
%    We provide two local and two global scratch registers, maybe we
%    need more or less.
%    \begin{macrocode}
\muskip_new:N \l_tmpa_muskip
\muskip_new:N \l_tmpb_muskip
\muskip_new:N \g_tmpa_muskip
\muskip_new:N \g_tmpb_muskip
%    \end{macrocode}
% \end{variable}
% \end{variable}
%
%    \begin{macrocode}
%</code>
%    \end{macrocode}
%
% \end{implementation}
%
% \PrintIndex
