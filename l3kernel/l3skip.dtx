% \iffalse meta-comment
%
%% File: l3skip.dtx Copyright (C) 2004-2011 Frank Mittelbach, The LaTeX3 Project
%%                            (C) 2012 The LaTeX3 Project
%%
%% It may be distributed and/or modified under the conditions of the
%% LaTeX Project Public License (LPPL), either version 1.3c of this
%% license or (at your option) any later version.  The latest version
%% of this license is in the file
%%
%%    http://www.latex-project.org/lppl.txt
%%
%% This file is part of the "l3kernel bundle" (The Work in LPPL)
%% and all files in that bundle must be distributed together.
%%
%% The released version of this bundle is available from CTAN.
%%
%% -----------------------------------------------------------------------
%%
%% The development version of the bundle can be found at
%%
%%    http://www.latex-project.org/svnroot/experimental/trunk/
%%
%% for those people who are interested.
%%
%%%%%%%%%%%
%% NOTE: %%
%%%%%%%%%%%
%%
%%   Snapshots taken from the repository represent work in progress and may
%%   not work or may contain conflicting material!  We therefore ask
%%   people _not_ to put them into distributions, archives, etc. without
%%   prior consultation with the LaTeX3 Project.
%%
%% -----------------------------------------------------------------------
%
%<*driver|package>
\RequirePackage{l3bootstrap}
\GetIdInfo$Id$
  {L3 Dimensions and skips}
%</driver|package>
%<*driver>
\documentclass[full]{l3doc}
\begin{document}
  \DocInput{\jobname.dtx}
\end{document}
%</driver>
% \fi
%
% \title{^^A
%   The \pkg{l3skip} package\\ Dimensions and skips^^A
%   \thanks{This file describes v\ExplFileVersion,
%      last revised \ExplFileDate.}^^A
% }
%
% \author{^^A
%  The \LaTeX3 Project\thanks
%    {^^A
%      E-mail:
%        \href{mailto:latex-team@latex-project.org}
%          {latex-team@latex-project.org}^^A
%    }^^A
% }
%
% \date{Released \ExplFileDate}
%
% \maketitle
%
% \begin{documentation}
%
% \LaTeX3 provides two general length variables: \texttt{dim} and
% \texttt{skip}. Lengths stored as \texttt{dim} variables have a fixed
% length, whereas \texttt{skip} lengths have a rubber (stretch/shrink)
% component. In addition, the \texttt{muskip} type is available for
% use in math mode: this is a special form of \texttt{skip} where the
% lengths involved are determined by the current math font (in
% \texttt{mu)}. There are common features in the creation and setting of
% length variables, but for clarity the functions are grouped by variable
% type.
%
% \section{Creating and initialising \texttt{dim} variables}
%
% \begin{function}{\dim_new:N, \dim_new:c}
%   \begin{syntax}
%     \cs{dim_new:N} \meta{dimension}
%   \end{syntax}
%   Creates a new \meta{dimension} or raises an error if the name is
%   already taken. The declaration is global. The \meta{dimension}
%   will initially be equal to $0$\,pt.
% \end{function}
%
% \begin{function}[added = 2012-03-05]{\dim_const:Nn, \dim_const:cn}
%   \begin{syntax}
%     \cs{dim_const:Nn} \meta{dimension} \Arg{dimension expression}
%   \end{syntax}
%   Creates a new constant \meta{dimension} or raises an error if the
%   name is already taken. The value of the \meta{dimension} will be set
%   globally to the \meta{dimension expression}.
% \end{function}
%
% \begin{function}{\dim_zero:N, \dim_zero:c, \dim_gzero:N, \dim_gzero:c}
%   \begin{syntax}
%     \cs{dim_zero:N} \meta{dimension}
%   \end{syntax}
%   Sets \meta{dimension} to $0$\,pt.
% \end{function}
%
% \begin{function}[added = 2012-01-07]
%   {\dim_zero_new:N, \dim_zero_new:c, \dim_gzero_new:N, \dim_gzero_new:c}
%   \begin{syntax}
%     \cs{dim_zero_new:N} \meta{dimension}
%   \end{syntax}
%   Ensures that the \meta{dimension} exists globally by applying
%   \cs{dim_new:N} if necessary, then applies \cs{dim_(g)zero:N} to leave
%   the \meta{dimension} set to zero.
% \end{function}
%
% \begin{function}[EXP, pTF, added=2012-03-03]{\dim_if_exist:N, \dim_if_exist:c}
%   \begin{syntax}
%     \cs{dim_if_exist_p:N} \meta{dimension}
%     \cs{dim_if_exist:NTF} \meta{dimension} \Arg{true code} \Arg{false code}
%   \end{syntax}
%   Tests whether the \meta{dimension} is currently defined.  This does
%   not check that the \meta{dimension} really is a dimension variable.
% \end{function}
%
% \section{Setting \texttt{dim} variables}
%
% \begin{function}[updated = 2011-10-22]
%   {\dim_add:Nn, \dim_add:cn, \dim_gadd:Nn, \dim_gadd:cn}
%   \begin{syntax}
%     \cs{dim_add:Nn} \meta{dimension} \Arg{dimension expression}
%   \end{syntax}
%   Adds the result of the \meta{dimension expression} to the current
%   content of the \meta{dimension}.
% \end{function}
%
% \begin{function}[updated = 2011-10-22]
%   {\dim_set:Nn, \dim_set:cn, \dim_gset:Nn, \dim_gset:cn}
%   \begin{syntax}
%     \cs{dim_set:Nn} \meta{dimension} \Arg{dimension expression}
%   \end{syntax}
%   Sets \meta{dimension} to the value of \meta{dimension expression}, which
%   must evaluate to a length with units.
% \end{function}
%
% \begin{function}
%   {
%     \dim_set_eq:NN,  \dim_set_eq:cN,  \dim_set_eq:Nc,  \dim_set_eq:cc,
%     \dim_gset_eq:NN, \dim_gset_eq:cN, \dim_gset_eq:Nc, \dim_gset_eq:cc
%   }
%   \begin{syntax}
%     \cs{dim_set_eq:NN} \meta{dimension_1} \meta{dimension_2}
%   \end{syntax}
%   Sets the content of \meta{dimension_1} equal to that of
%   \meta{dimension_2}.
% \end{function}
%
% \begin{function}[updated = 2011-10-22]
%   {\dim_sub:Nn, \dim_sub:cn, \dim_gsub:Nn, \dim_gsub:cn}
%   \begin{syntax}
%     \cs{dim_sub:Nn} \meta{dimension} \Arg{dimension expression}
%   \end{syntax}
%   Subtracts the result of the \meta{dimension expression} from the
%   current content of the \meta{dimension}.
% \end{function}
%
% \section{Utilities for dimension calculations}
%
% \begin{function}[updated = 2012-09-26, EXP]{\dim_abs:n}
%   \begin{syntax}
%     \cs{dim_abs:n} \Arg{dimexpr}
%   \end{syntax}
%   Converts the \meta{dimexpr} to its absolute value, leaving the result
%   in the input stream as a \meta{dimension denotation}.
% \end{function}
%
% \begin{function}[added = 2012-09-09, updated = 2012-09-26, EXP]
%   {\dim_max:nn, \dim_min:nn}
%   \begin{syntax}
%     \cs{dim_max:nn} \Arg{dimexpr_1} \Arg{dimexpr_2}
%     \cs{dim_min:nn} \Arg{dimexpr_1} \Arg{dimexpr_2}
%   \end{syntax}
%   Evaluates the two \meta{dimension expressions} and leaves either the
%   maximum or minimum value in the input stream as appropriate, as a
%   \meta{dimension denotation}.
% \end{function}
%
% \begin{function}[updated = 2011-10-22, rEXP]{\dim_ratio:nn}
%   \begin{syntax}
%     \cs{dim_ratio:nn} \Arg{dimexpr_1} \Arg{dimexpr_2}
%   \end{syntax}
%   Parses the two \meta{dimension expressions} and converts the ratio of
%   the two to a form suitable for use inside a \meta{dimension expression}.
%   This ratio is then left in the input stream, allowing syntax such as
%   \begin{verbatim}
%     \dim_set:Nn \l_my_dim
%       { 10 pt * \dim_ratio:nn { 5 pt } { 10 pt } }
%   \end{verbatim}
%   The output of \cs{dim_ratio:nn} on full expansion is a ration expression
%   between two integers, with all distances converted to scaled points.
%   Thus
%   \begin{verbatim}
%     \tl_set:Nx \l_my_tl { \dim_ratio:nn { 5 pt } { 10 pt } }
%     \tl_show:N \l_my_tl
%   \end{verbatim}
%   will display |327680/655360| on the terminal.
% \end{function}
%
% \section{Dimension expression conditionals}
%
% \begin{function}[EXP,pTF]{\dim_compare:nNn}
%   \begin{syntax}
%     \cs{@@_compare_p:nNn} \Arg{dimexpr_1} \meta{relation} \Arg{dimexpr_2} \\
%     \cs{dim_compare:nNnTF}
%     ~~\Arg{dimexpr_1} \meta{relation} \Arg{dimexpr_2}
%     ~~\Arg{true code} \Arg{false code}
%   \end{syntax}
%   This function first evaluates each of the \meta{dimension expressions}
%   as described for \cs{dim_eval:n}. The two results are then
%   compared using the \meta{relation}:
%   \begin{center}
%     \begin{tabular}{ll}
%       Equal                 & |=| \\
%       Greater than          & |>| \\
%       Less than             & |<| \\
%     \end{tabular}
%   \end{center}
% \end{function}
%
% \begin{function}[EXP,pTF]{\dim_compare:n}
%   \begin{syntax}
%     \cs{@@_compare_p:n} \{ \meta{dimexpr_1} \meta{relation} \meta{dimexpr_2} \} \\
%     \cs{dim_compare:nTF}
%     ~~\{ \meta{dimexpr_1} \meta{relation} \meta{dimexpr_2} \}
%     ~~\Arg{true code} \Arg{false code}
%   \end{syntax}
%   This function first evaluates each of the \meta{dimension expressions}
%   as described for \cs{dim_eval:n}. The two results are then
%   compared using the \meta{relation}:
%   \begin{center}
%     \begin{tabular}{ll}
%       Equal                    & |=| or |==| \\
%       Greater than or equal to & |>=|        \\
%       Greater than             & |>|         \\
%       Less than or equal to    & |<=|        \\
%       Less than                & |<|         \\
%       Not equal                & |!=|        \\
%     \end{tabular}
%   \end{center}
% \end{function}
%
% \begin{function}[added = 2012-06-03, EXP]{\dim_case:nnn}
%   \begin{syntax}
%     \cs{dim_case:nnn} \Arg{test dimension expression} \\
%     ~~|{| \\
%     ~~~~\Arg{dimexpr case_1} \Arg{code case_1} \\
%     ~~~~\Arg{dimexpr case_2} \Arg{code case_2} \\
%     ~~~~\ldots \\
%     ~~~~\Arg{dimexpr case_n} \Arg{code case_n} \\
%     ~~|}| \\
%     ~~\Arg{else code}
%   \end{syntax}
%   This function evaluates the \meta{test dimension expression} and
%   compares this in turn to each of the
%   \meta{dimension expression cases}. If the two are equal then the
%   associated \meta{code} is left in the input stream. If none of
%   the tests are \texttt{true} then the \texttt{else code} will be
%   left in the input stream. For example
%   \begin{verbatim}
%     \dim_set:Nn \l_tmpa_dim { 5 pt }
%     \dim_case:nnn
%       { 2 \l_tmpa_dim }
%       {
%         { 5 pt }        { Small }
%         { 4 pt + 6 pt } { Medium }
%         { - 10 pt }     { Negative }
%       }
%       { No idea! }
%    \end{verbatim}
%    will leave \enquote{\texttt{Medium}} in the input stream.
% \end{function}
%
% \section{Dimension expression loops}
%
% \begin{function}[rEXP]{\dim_do_until:nNnn}
%   \begin{syntax}
%      \cs{dim_do_until:nNnn} \Arg{dimexpr_1} \meta{relation} \Arg{dimexpr_2} \Arg{code}
%   \end{syntax}
%   Places the \meta{code} in the input stream for \TeX{} to process, and
%   then evaluates the relationship between the two
%   \meta{dimension expressions} as described for \cs{dim_compare:nNnTF}.
%   If the test is \texttt{false} then the \meta{code} will be inserted
%   into the input stream again and a loop will occur until the
%   \meta{relation} is \texttt{true}.
% \end{function}
%
% \begin{function}[rEXP]{\dim_do_while:nNnn}
%   \begin{syntax}
%      \cs{dim_do_while:nNnn} \Arg{dimexpr_1} \meta{relation} \Arg{dimexpr_2} \Arg{code}
%   \end{syntax}
%   Places the \meta{code} in the input stream for \TeX{} to process, and
%   then evaluates the relationship between the two
%   \meta{dimension expressions} as described for \cs{dim_compare:nNnTF}.
%   If the test is \texttt{true} then the \meta{code} will be inserted
%   into the input stream again and a loop will occur until the
%   \meta{relation} is \texttt{false}.
% \end{function}
%
% \begin{function}[rEXP]{\dim_until_do:nNnn}
%   \begin{syntax}
%      \cs{dim_until_do:nNnn} \Arg{dimexpr_1} \meta{relation} \Arg{dimexpr_2} \Arg{code}
%   \end{syntax}
%   Evaluates the relationship between the two \meta{dimension expressions}
%   as described for \cs{dim_compare:nNnTF}, and then places the
%   \meta{code} in the input stream if the \meta{relation} is
%   \texttt{false}. After the \meta{code} has been processed by \TeX{} the
%   test will be repeated, and a loop will occur until the test is
%   \texttt{true}.
% \end{function}
%
% \begin{function}[rEXP]{\dim_while_do:nNnn}
%   \begin{syntax}
%      \cs{dim_while_do:nNnn} \Arg{dimexpr_1} \meta{relation} \Arg{dimexpr_2} \Arg{code}
%   \end{syntax}
%   Evaluates the relationship between the two \meta{dimension expressions}
%   as described for \cs{dim_compare:nNnTF}, and then places the
%   \meta{code} in the input stream if the \meta{relation} is
%   \texttt{true}. After the \meta{code} has been processed by \TeX{} the
%   test will be repeated, and a loop will occur until the test is
%   \texttt{false}.
% \end{function}
%
% \begin{function}[rEXP]{\dim_do_until:nn}
%   \begin{syntax}
%      \cs{dim_do_until:nn} \{ \meta{dimexpr_1} \meta{relation} \meta{dimexpr_2} \} \Arg{code}
%   \end{syntax}
%   Places the \meta{code} in the input stream for \TeX{} to process, and
%   then evaluates the relationship between the two
%   \meta{dimension expressions} as described for \cs{dim_compare:nTF}.
%   If the test is \texttt{false} then the \meta{code} will be inserted
%   into the input stream again and a loop will occur until the
%   \meta{relation} is \texttt{true}.
% \end{function}
%
% \begin{function}[rEXP]{\dim_do_while:nn}
%   \begin{syntax}
%      \cs{dim_do_while:nn} \{ \meta{dimexpr_1} \meta{relation} \meta{dimexpr_2} \} \Arg{code}
%   \end{syntax}
%   Places the \meta{code} in the input stream for \TeX{} to process, and
%   then evaluates the relationship between the two
%   \meta{dimension expressions} as described for \cs{dim_compare:nTF}.
%   If the test is \texttt{true} then the \meta{code} will be inserted
%   into the input stream again and a loop will occur until the
%   \meta{relation} is \texttt{false}.
% \end{function}
%
% \begin{function}[rEXP]{\dim_until_do:nn}
%   \begin{syntax}
%      \cs{dim_until_do:nn} \{ \meta{dimexpr_1} \meta{relation} \meta{dimexpr_2} \} \Arg{code}
%   \end{syntax}
%   Evaluates the relationship between the two \meta{dimension expressions}
%   as described for \cs{dim_compare:nTF}, and then places the
%   \meta{code} in the input stream if the \meta{relation} is
%   \texttt{false}. After the \meta{code} has been processed by \TeX{} the
%   test will be repeated, and a loop will occur until the test is
%   \texttt{true}.
% \end{function}
%
% \begin{function}[rEXP]{\dim_while_do:nn}
%   \begin{syntax}
%      \cs{dim_while_do:nn} \{ \meta{dimexpr_1} \meta{relation} \meta{dimexpr_2} \} \Arg{code}
%   \end{syntax}
%   Evaluates the relationship between the two \meta{dimension expressions}
%   as described for \cs{dim_compare:nTF}, and then places the
%   \meta{code} in the input stream if the \meta{relation} is
%   \texttt{true}. After the \meta{code} has been processed by \TeX{} the
%   test will be repeated, and a loop will occur until the test is
%   \texttt{false}.
% \end{function}
%
% \section{Using \texttt{dim} expressions and variables}
%
% \begin{function}[updated = 2011-10-22, EXP]{\dim_eval:n}
%   \begin{syntax}
%     \cs{dim_eval:n} \Arg{dimension expression}
%   \end{syntax}
%   Evaluates the \meta{dimension expression}, expanding any
%   dimensions and token list variables within the \meta{expression}
%   to their content (without requiring \cs{dim_use:N}/\cs{tl_use:N})
%   and applying the standard mathematical rules. The result of the
%   calculation is left in the input stream as a
%   \meta{dimension denotation} after two expansions. This will be
%   expressed in points (\texttt{pt}), and will require suitable
%   termination if used in a \TeX{}-style assignment as it is \emph{not}
%   an \meta{internal dimension}.
% \end{function}
%
% \begin{function}[EXP]{\dim_use:N, \dim_use:c}
%   \begin{syntax}
%     \cs{dim_use:N} \meta{dimension}
%   \end{syntax}
%   Recovers the content of a \meta{dimension} and places it directly
%   in the input stream. An error will be raised if the variable does
%   not exist or if it is invalid. Can be omitted in places where a
%   \meta{dimension} is required (such as in the argument of
%   \cs{dim_eval:n}).
%   \begin{texnote}
%     \cs{dim_use:N} is the \TeX{} primitive \tn{the}: this is one of
%     several \LaTeX3 names for this primitive.
%   \end{texnote}
% \end{function}
%
% \section{Viewing \texttt{dim} variables}
%
% \begin{function}{\dim_show:N, \dim_show:c}
%   \begin{syntax}
%     \cs{dim_show:N} \meta{dimension}
%    \end{syntax}
%   Displays the value of the \meta{dimension} on the terminal.
% \end{function}
%
% \begin{function}[added = 2011-11-22, updated = 2012-05-27]{\dim_show:n}
%   \begin{syntax}
%     \cs{dim_show:n} \meta{dimension expression}
%    \end{syntax}
%   Displays the result of evaluating the \meta{dimension expression}
%   on the terminal.
% \end{function}
%
% \section{Constant dimensions}
%
% \begin{variable}{\c_max_dim}
%   The maximum value that can be stored as a dimension or skip (these
%   are equivalent).
% \end{variable}
%
% \begin{variable}{\c_zero_dim}
%   A zero length as a dimension or a skip (these are equivalent).
% \end{variable}
%
% \section{Scratch dimensions}
%
% \begin{variable}{\l_tmpa_dim, \l_tmpb_dim}
%   Scratch dimension for local assignment. These are never used by
%   the kernel code, and so are safe for use with any \LaTeX3-defined
%   function. However, they may be overwritten by other non-kernel
%   code and so should only be used for short-term storage.
% \end{variable}
%
% \begin{variable}{\g_tmpa_dim, \g_tmpb_dim}
%   Scratch dimension for global assignment. These are never used by
%   the kernel code, and so are safe for use with any \LaTeX3-defined
%   function. However, they may be overwritten by other non-kernel
%   code and so should only be used for short-term storage.
% \end{variable}
%
% \section{Creating and initialising \texttt{skip} variables}
%
% \begin{function}{\skip_new:N, \skip_new:c}
%   \begin{syntax}
%     \cs{skip_new:N} \meta{skip}
%   \end{syntax}
%   Creates a new \meta{skip} or raises an error if the name is
%   already taken. The declaration is global. The \meta{skip}
%   will initially be equal to $0$\,pt.
% \end{function}
%
% \begin{function}[added = 2012-03-05]{\skip_const:Nn, \skip_const:cn}
%   \begin{syntax}
%     \cs{skip_const:Nn} \meta{skip} \Arg{skip expression}
%   \end{syntax}
%   Creates a new constant \meta{skip} or raises an error if the
%   name is already taken. The value of the \meta{skip} will be set
%   globally to the \meta{skip expression}.
% \end{function}
%
% \begin{function}{\skip_zero:N, \skip_zero:c, \skip_gzero:N, \skip_gzero:c}
%   \begin{syntax}
%     \cs{skip_zero:N} \meta{skip}
%   \end{syntax}
%   Sets \meta{skip} to $0$\,pt.
% \end{function}
%
% \begin{function}[added = 2012-01-07]
%   {\skip_zero_new:N, \skip_zero_new:c, \skip_gzero_new:N, \skip_gzero_new:c}
%   \begin{syntax}
%     \cs{skip_zero_new:N} \meta{skip}
%   \end{syntax}
%   Ensures that the \meta{skip} exists globally by applying
%   \cs{skip_new:N} if necessary, then applies \cs{skip_(g)zero:N} to leave
%   the \meta{skip} set to zero.
% \end{function}
%
% \begin{function}[EXP, pTF, added=2012-03-03]
%   {\skip_if_exist:N, \skip_if_exist:c}
%   \begin{syntax}
%     \cs{skip_if_exist_p:N} \meta{skip}
%     \cs{skip_if_exist:NTF} \meta{skip} \Arg{true code} \Arg{false code}
%   \end{syntax}
%   Tests whether the \meta{skip} is currently defined.  This does not
%   check that the \meta{skip} really is a skip variable.
% \end{function}
%
% \section{Setting \texttt{skip} variables}
%
% \begin{function}[updated = 2011-10-22]
%   {\skip_add:Nn, \skip_add:cn, \skip_gadd:Nn, \skip_gadd:cn}
%   \begin{syntax}
%     \cs{skip_add:Nn} \meta{skip} \Arg{skip expression}
%   \end{syntax}
%   Adds the result of the \meta{skip expression} to the current
%   content of the \meta{skip}.
% \end{function}
%
% \begin{function}[updated = 2011-10-22]
%   {\skip_set:Nn, \skip_set:cn, \skip_gset:Nn, \skip_gset:cn}
%   \begin{syntax}
%     \cs{skip_set:Nn} \meta{skip} \Arg{skip expression}
%   \end{syntax}
%   Sets \meta{skip} to the value of \meta{skip expression}, which
%   must evaluate to a length with units and may include a rubber
%   component (for example |1 cm plus 0.5 cm|.
% \end{function}
%
% \begin{function}
%   {
%     \skip_set_eq:NN,  \skip_set_eq:cN,  \skip_set_eq:Nc,  \skip_set_eq:cc,
%     \skip_gset_eq:NN, \skip_gset_eq:cN, \skip_gset_eq:Nc, \skip_gset_eq:cc
%   }
%   \begin{syntax}
%     \cs{skip_set_eq:NN} \meta{skip_1} \meta{skip_2}
%   \end{syntax}
%   Sets the content of \meta{skip_1} equal to that of \meta{skip_2}.
% \end{function}
%
% \begin{function}[updated = 2011-10-22]
%   {\skip_sub:Nn, \skip_sub:cn, \skip_gsub:Nn, \skip_gsub:cn}
%   \begin{syntax}
%     \cs{skip_sub:Nn} \meta{skip} \Arg{skip expression}
%   \end{syntax}
%   Subtracts the result of the \meta{skip expression} from the
%   current content of the \meta{skip}.
% \end{function}
%
% \section{Skip expression conditionals}
%
% \begin{function}[EXP,pTF]{\skip_if_eq:nn}
%   \begin{syntax}
%     \cs{skip_if_eq_p:nn} \Arg{skipexpr_1} \Arg{skipexpr_2}
%     \cs{dim_compare:nTF}
%     ~~\Arg{skipexpr_1} \Arg{skipexpr_2}
%     ~~\Arg{true code} \Arg{false code}
%   \end{syntax}
%   This function first evaluates each of the
%   \meta{skip expressions} as described for \cs{skip_eval:n}.
%   The two results are then compared for exact equality,
%   \emph{i.e.}~both the fixed and rubber components must be the same
%   for the test to be true.
% \end{function}
%
% \begin{function}[EXP, pTF, added = 2012-03-05]{\skip_if_finite:n}
%   \begin{syntax}
%     \cs{skip_if_finite_p:n} \Arg{skipexpr}
%     \cs{skip_if_finite:nTF} \Arg{skipexpr} \Arg{true code} \Arg{false code}
%   \end{syntax}
%   Evaluates the \meta{skip expression} as described for \cs{skip_eval:n},
%   and then tests if all of its components are finite.
% \end{function}
%
% \section{Using \texttt{skip} expressions and variables}
%
% \begin{function}[updated = 2011-10-22, EXP]{\skip_eval:n}
%   \begin{syntax}
%     \cs{skip_eval:n} \Arg{skip expression}
%   \end{syntax}
%   Evaluates the \meta{skip expression}, expanding any skips
%   and token list variables within the \meta{expression}
%   to their content (without requiring \cs{skip_use:N}/\cs{tl_use:N})
%   and applying the standard mathematical rules. The result of the
%   calculation is left in the input stream as a \meta{glue denotation}
%   after two expansions. This will be expressed in points (\texttt{pt}),
%   and will require suitable termination if used in a \TeX{}-style
%   assignment as it is \emph{not} an \meta{internal glue}.
% \end{function}
%
% \begin{function}[EXP]{\skip_use:N, \skip_use:c}
%   \begin{syntax}
%     \cs{skip_use:N} \meta{skip}
%   \end{syntax}
%   Recovers the content of a \meta{skip} and places it directly
%   in the input stream. An error will be raised if the variable does
%   not exist or if it is invalid. Can be omitted in places where a
%   \meta{dimension} is required (such as in the argument of
%   \cs{skip_eval:n}).
%   \begin{texnote}
%     \cs{skip_use:N} is the \TeX{} primitive \tn{the}: this is one of
%     several \LaTeX3 names for this primitive.
%   \end{texnote}
% \end{function}
%
% \section{Viewing \texttt{skip} variables}
%
% \begin{function}{\skip_show:N, \skip_show:c}
%   \begin{syntax}
%     \cs{skip_show:N} \meta{skip}
%    \end{syntax}
%   Displays the value of the \meta{skip} on the terminal.
% \end{function}
%
% \begin{function}[added = 2011-11-22, updated = 2012-05-27]{\skip_show:n}
%   \begin{syntax}
%     \cs{skip_show:n} \meta{skip expression}
%    \end{syntax}
%   Displays the result of evaluating the \meta{skip expression}
%   on the terminal.
% \end{function}
%
% \section{Constant skips}
%
% \begin{variable}{\c_max_skip}
%   The maximum value that can be stored as a dimension or skip (these
%   are equivalent).
% \end{variable}
%
% \begin{variable}[updated = 2012-11-01]{\c_zero_skip}
%   A zero length as a dimension or a skip (these are equivalent).
% \end{variable}
%
% \section{Scratch skips}
%
% \begin{variable}{\l_tmpa_skip, \l_tmpb_skip}
%   Scratch skip for local assignment. These are never used by
%   the kernel code, and so are safe for use with any \LaTeX3-defined
%   function. However, they may be overwritten by other non-kernel
%   code and so should only be used for short-term storage.
% \end{variable}
%
% \begin{variable}{\g_tmpa_skip, \g_tmpb_skip}
%   Scratch skip for global assignment. These are never used by
%   the kernel code, and so are safe for use with any \LaTeX3-defined
%   function. However, they may be overwritten by other non-kernel
%   code and so should only be used for short-term storage.
% \end{variable}
%
% \section{Inserting skips into the output}
%
% \begin{function}[updated = 2011-10-22]
%   {\skip_horizontal:N, \skip_horizontal:c, \skip_horizontal:n}
%   \begin{syntax}
%     \cs{skip_horizontal:N} \meta{skip}
%     \cs{skip_horizontal:n} \Arg{skipexpr}
%   \end{syntax}
%   Inserts a horizontal \meta{skip} into the current list.
%   \begin{texnote}
%     \cs{skip_horizontal:N} is the \TeX{} primitive \tn{hskip} renamed.
%   \end{texnote}
% \end{function}
%
% \begin{function}[updated = 2011-10-22]
%   {\skip_vertical:N, \skip_vertical:c, \skip_vertical:n}
%   \begin{syntax}
%     \cs{skip_vertical:N} \meta{skip}
%     \cs{skip_vertical:n} \Arg{skipexpr}
%   \end{syntax}
%   Inserts a vertical \meta{skip} into the current list.
%   \begin{texnote}
%     \cs{skip_vertical:N} is the \TeX{} primitive \tn{vskip} renamed.
%   \end{texnote}
% \end{function}
%
% \section{Creating and initialising \texttt{muskip} variables}
%
% \begin{function}{\muskip_new:N, \muskip_new:c}
%   \begin{syntax}
%     \cs{muskip_new:N} \meta{muskip}
%   \end{syntax}
%   Creates a new \meta{muskip} or raises an error if the name is
%   already taken. The declaration is global. The \meta{muskip}
%   will initially be equal to $0$\,mu.
% \end{function}
%
% \begin{function}[added = 2012-03-05]{\muskip_const:Nn, \muskip_const:cn}
%   \begin{syntax}
%     \cs{muskip_const:Nn} \meta{muskip} \Arg{muskip expression}
%   \end{syntax}
%   Creates a new constant \meta{muskip} or raises an error if the
%   name is already taken. The value of the \meta{muskip} will be set
%   globally to the \meta{muskip expression}.
% \end{function}
%
% \begin{function}
%   {\muskip_zero:N, \muskip_zero:c, \muskip_gzero:N, \muskip_gzero:c}
%   \begin{syntax}
%     \cs{skip_zero:N} \meta{muskip}
%   \end{syntax}
%   Sets \meta{muskip} to $0$\,mu.
% \end{function}
%
% \begin{function}[added = 2012-01-07]
%   {
%     \muskip_zero_new:N,  \muskip_zero_new:c,
%     \muskip_gzero_new:N, \muskip_gzero_new:c
%   }
%   \begin{syntax}
%     \cs{muskip_zero_new:N} \meta{muskip}
%   \end{syntax}
%   Ensures that the \meta{muskip} exists globally by applying
%   \cs{muskip_new:N} if necessary, then applies \cs{muskip_(g)zero:N}
%   to leave the \meta{muskip} set to zero.
% \end{function}
%
% \begin{function}[EXP, pTF, added=2012-03-03]
%   {\muskip_if_exist:N, \muskip_if_exist:c}
%   \begin{syntax}
%     \cs{muskip_if_exist_p:N} \meta{muskip}
%     \cs{muskip_if_exist:NTF} \meta{muskip} \Arg{true code} \Arg{false code}
%   \end{syntax}
%   Tests whether the \meta{muskip} is currently defined.  This does not
%   check that the \meta{muskip} really is a muskip variable.
% \end{function}
%
% \section{Setting \texttt{muskip} variables}
%
% \begin{function}[updated = 2011-10-22]
%   {\muskip_add:Nn, \muskip_add:cn, \muskip_gadd:Nn, \muskip_gadd:cn}
%   \begin{syntax}
%     \cs{muskip_add:Nn} \meta{muskip} \Arg{muskip expression}
%   \end{syntax}
%   Adds the result of the \meta{muskip expression} to the current
%   content of the \meta{muskip}.
% \end{function}
%
% \begin{function}[updated = 2011-10-22]
%   {\muskip_set:Nn, \muskip_set:cn, \muskip_gset:Nn, \muskip_gset:cn}
%   \begin{syntax}
%     \cs{muskip_set:Nn} \meta{muskip} \Arg{muskip expression}
%   \end{syntax}
%   Sets \meta{muskip} to the value of \meta{muskip expression}, which
%   must evaluate to a math length with units and may include a rubber
%   component (for example |1 mu plus 0.5 mu|.
% \end{function}
%
% \begin{function}
%   {
%     \muskip_set_eq:NN,  \muskip_set_eq:cN,
%     \muskip_set_eq:Nc,  \muskip_set_eq:cc,
%     \muskip_gset_eq:NN, \muskip_gset_eq:cN,
%     \muskip_gset_eq:Nc, \muskip_gset_eq:cc
%   }
%   \begin{syntax}
%     \cs{muskip_set_eq:NN} \meta{muskip_1} \meta{muskip_2}
%   \end{syntax}
%   Sets the content of \meta{muskip_1} equal to that of
%   \meta{muskip_2}.
% \end{function}
%
% \begin{function}[updated = 2011-10-22]
%   {\muskip_sub:Nn, \muskip_sub:cn, \muskip_gsub:Nn, \muskip_gsub:cn}
%   \begin{syntax}
%     \cs{muskip_sub:Nn} \meta{muskip} \Arg{muskip expression}
%   \end{syntax}
%   Subtracts the result of the \meta{muskip expression} from the
%   current content of the \meta{skip}.
% \end{function}
%
% \section{Using \texttt{muskip} expressions and variables}
%
% \begin{function}[updated = 2011-10-22, EXP]{\muskip_eval:n}
%   \begin{syntax}
%     \cs{muskip_eval:n} \Arg{muskip expression}
%   \end{syntax}
%   Evaluates the \meta{muskip expression}, expanding any skips
%   and token list variables within the \meta{expression}
%   to their content (without requiring \cs{muskip_use:N}/\cs{tl_use:N})
%   and applying the standard mathematical rules. The result of the
%   calculation is left in the input stream as a \meta{muglue denotation}
%   after two expansions. This will be expressed in \texttt{mu},
%   and will require suitable termination if used in a \TeX{}-style
%   assignment as it is \emph{not} an \meta{internal muglue}.
% \end{function}
%
% \begin{function}[EXP]{\muskip_use:N, \muskip_use:c}
%   \begin{syntax}
%     \cs{muskip_use:N} \meta{muskip}
%   \end{syntax}
%   Recovers the content of a \meta{skip} and places it directly
%   in the input stream. An error will be raised if the variable does
%   not exist or if it is invalid. Can be omitted in places where a
%   \meta{dimension} is required (such as in the argument of
%   \cs{muskip_eval:n}).
%   \begin{texnote}
%     \cs{muskip_use:N} is the \TeX{} primitive \tn{the}: this is one of
%     several \LaTeX3 names for this primitive.
%   \end{texnote}
% \end{function}
%
% \section{Viewing \texttt{muskip} variables}
%
% \begin{function}{\muskip_show:N, \muskip_show:c}
%   \begin{syntax}
%     \cs{muskip_show:N} \meta{muskip}
%    \end{syntax}
%   Displays the value of the \meta{muskip} on the terminal.
% \end{function}
%
% \begin{function}[added = 2011-11-22, updated = 2012-05-27]{\muskip_show:n}
%   \begin{syntax}
%     \cs{muskip_show:n} \meta{muskip expression}
%    \end{syntax}
%   Displays the result of evaluating the \meta{muskip expression}
%   on the terminal.
% \end{function}
%
% \section{Constant muskips}
%
% \begin{variable}{\c_max_muskip}
%   The maximum value that can be stored as a muskip.
% \end{variable}
%
% \begin{variable}{\c_zero_muskip}
%   A zero length as a muskip.
% \end{variable}
%
% \section{Scratch muskips}
%
% \begin{variable}{\l_tmpa_muskip, \l_tmpb_muskip}
%   Scratch muskip for local assignment. These are never used by
%   the kernel code, and so are safe for use with any \LaTeX3-defined
%   function. However, they may be overwritten by other non-kernel
%   code and so should only be used for short-term storage.
% \end{variable}
%
% \begin{variable}{\g_tmpa_muskip, \g_tmpb_muskip}
%   Scratch muskip for global assignment. These are never used by
%   the kernel code, and so are safe for use with any \LaTeX3-defined
%   function. However, they may be overwritten by other non-kernel
%   code and so should only be used for short-term storage.
% \end{variable}
%
% \section{Primitive conditional}
%
% \begin{function}{\if_dim:w}
%   \begin{syntax}
%     \cs{if_dim:w} \meta{dimen_1} \meta{relation} \meta{dimen_2}
%     ~~\meta{true code}
%     \cs{else:}
%     ~~\meta{false}
%     \cs{fi:}
%   \end{syntax}
%   Compare two dimensions. The \meta{relation} is one of
%   |<|, |=| or |>| with category code $12$.
%   \begin{texnote}
%     This is the \TeX{} primitive \tn{ifdim}.
%   \end{texnote}
% \end{function}
%
% \section{Internal functions}
%
% \begin{function}[EXP]{\__dim_eval:w, \__dim_eval_end:}
%   \begin{syntax}
%     \cs{__dim_eval:w} \meta{dimexpr} \cs{__dim_eval_end:}
%   \end{syntax}
%   Evaluates \meta{dimension expression} as described for \cs{dim_eval:n}.
%   The evaluation stops when an unexpandable token which is not a valid
%   part of a dimension is read or when \cs{__dim_eval_end:} is
%   reached. The latter is gobbled by the scanner mechanism:
%   \cs{__dim_eval_end:} itself is unexpandable but used correctly
%   the entire construct is expandable.
%   \begin{texnote}
%     This is the \eTeX{} primitive \tn{dimexpr}.
%   \end{texnote}
% \end{function}
%
% \begin{function}[added = 2011-11-11, EXP]
%   {\__dim_strip_bp:n, \__dim_strip_pt:n}
%   \begin{syntax}
%     \cs{__dim_strip_bp:n} \Arg{dimension expression}
%     \cs{__dim_strip_pt:n} \Arg{dimension expression}
%   \end{syntax}
%   Evaluates the \meta{dimension expression}, expanding any
%   dimensions and token list variables within the \meta{expression}
%   to their content (without requiring \cs{dim_use:N}/\cs{tl_use:N})
%   and applying the standard mathematical rules. The magnitude of the
%   result, expressed in big points (\texttt{bp}) or  points (\texttt{pt}),
%   will be left in the input stream with \emph{no units}. If the decimal
%   part of the magnitude is zero, this will be omitted.
%
%   If the \Arg{dimension expression} contains additional units, these will
%   be ignored, so for example
%   \begin{verbatim}
%     \__dim_strip_pt:n { 1 bp pt }
%   \end{verbatim}
%   will leave |1.00374| in the input stream (\emph{i.e.}~the magnitude of
%   one \enquote{big point} when converted to points).
% \end{function}
%
% \end{documentation}
%
% \begin{implementation}
%
% \section{\pkg{l3skip} implementation}
%
%    \begin{macrocode}
%<*initex|package>
%    \end{macrocode}
%
%    \begin{macrocode}
%<@@=dim>
%    \end{macrocode}
%
%    \begin{macrocode}
%<*package>
\ProvidesExplPackage
  {\ExplFileName}{\ExplFileDate}{\ExplFileVersion}{\ExplFileDescription}
\__expl_package_check:
%</package>
%    \end{macrocode}
%
% \subsection{Length primitives renamed}
%
% \begin{macro}{\if_dim:w}
% \begin{macro}{\@@_eval:w}
% \begin{macro}{\@@_eval_end:}
%   Primitives renamed.
%    \begin{macrocode}
\cs_new_eq:NN \if_dim:w      \tex_ifdim:D
\cs_new_eq:NN \@@_eval:w      \etex_dimexpr:D
\cs_new_eq:NN \@@_eval_end:   \tex_relax:D
%    \end{macrocode}
% \end{macro}
% \end{macro}
% \end{macro}
%
% \subsection{Creating and initialising \texttt{dim} variables}
%
% \begin{macro}{\dim_new:N,\dim_new:c}
%    Allocating \meta{dim} registers \ldots
%    \begin{macrocode}
%<*package>
\cs_new_protected:Npn \dim_new:N #1
  {
    \__chk_if_free_cs:N #1
    \newdimen #1
  }
%</package>
\cs_generate_variant:Nn \dim_new:N { c }
%    \end{macrocode}
% \end{macro}
%
% \begin{macro}{\dim_const:Nn, \dim_const:cn}
%   Contrarily to integer constants, we cannot avoid using a register,
%   even for constants.
%    \begin{macrocode}
\cs_new_protected:Npn \dim_const:Nn #1
  {
    \dim_new:N #1
    \dim_gset:Nn #1
  }
\cs_generate_variant:Nn \dim_const:Nn { c }
%    \end{macrocode}
% \end{macro}
%
% \begin{macro}{\dim_zero:N, \dim_zero:c}
% \begin{macro}{\dim_gzero:N, \dim_gzero:c}
%   Reset the register to zero.
%    \begin{macrocode}
\cs_new_protected:Npn \dim_zero:N #1 { #1 \c_zero_dim }
\cs_new_protected:Npn \dim_gzero:N { \tex_global:D \dim_zero:N }
\cs_generate_variant:Nn \dim_zero:N  { c }
\cs_generate_variant:Nn \dim_gzero:N { c }
%    \end{macrocode}
% \end{macro}
% \end{macro}
%
% \begin{macro}
%   {\dim_zero_new:N, \dim_zero_new:c, \dim_gzero_new:N, \dim_gzero_new:c}
%   Create a register if needed, otherwise clear it.
%    \begin{macrocode}
\cs_new_protected:Npn \dim_zero_new:N  #1
  { \dim_if_exist:NTF #1 { \dim_zero:N #1 } { \dim_new:N #1 } }
\cs_new_protected:Npn \dim_gzero_new:N #1
  { \dim_if_exist:NTF #1 { \dim_gzero:N #1 } { \dim_new:N #1 } }
\cs_generate_variant:Nn \dim_zero_new:N  { c }
\cs_generate_variant:Nn \dim_gzero_new:N { c }
%    \end{macrocode}
% \end{macro}
%
% \begin{macro}[pTF]{\dim_if_exist:N, \dim_if_exist:c}
%   Copies of the \texttt{cs} functions defined in \pkg{l3basics}.
%    \begin{macrocode}
\cs_new_eq:NN \dim_if_exist:NTF \cs_if_exist:NTF
\cs_new_eq:NN \dim_if_exist:NT  \cs_if_exist:NT
\cs_new_eq:NN \dim_if_exist:NF  \cs_if_exist:NF
\cs_new_eq:NN \dim_if_exist_p:N \cs_if_exist_p:N
\cs_new_eq:NN \dim_if_exist:cTF \cs_if_exist:cTF
\cs_new_eq:NN \dim_if_exist:cT  \cs_if_exist:cT
\cs_new_eq:NN \dim_if_exist:cF  \cs_if_exist:cF
\cs_new_eq:NN \dim_if_exist_p:c \cs_if_exist_p:c
%    \end{macrocode}
% \end{macro}
%
% \subsection{Setting \texttt{dim} variables}
%
% \begin{macro}{\dim_set:Nn, \dim_set:cn}
% \begin{macro}{\dim_gset:Nn, \dim_gset:cn}
%   Setting dimensions is easy enough.
%    \begin{macrocode}
\cs_new_protected:Npn \dim_set:Nn #1#2
  { #1 ~ \@@_eval:w #2 \@@_eval_end: }
\cs_new_protected:Npn \dim_gset:Nn { \tex_global:D \dim_set:Nn }
\cs_generate_variant:Nn \dim_set:Nn  { c }
\cs_generate_variant:Nn \dim_gset:Nn { c }
%    \end{macrocode}
% \end{macro}
% \end{macro}
%
% \begin{macro}{\dim_set_eq:NN,\dim_set_eq:cN, \dim_set_eq:Nc,\dim_set_eq:cc}
% \begin{macro}
%   {\dim_gset_eq:NN,\dim_gset_eq:cN, \dim_gset_eq:Nc,\dim_gset_eq:cc}
%   All straightforward.
%    \begin{macrocode}
\cs_new_protected:Npn \dim_set_eq:NN #1#2 { #1 = #2 }
\cs_generate_variant:Nn \dim_set_eq:NN {       c }
\cs_generate_variant:Nn \dim_set_eq:NN { Nc , cc }
\cs_new_protected:Npn \dim_gset_eq:NN #1#2 { \tex_global:D #1 = #2 }
\cs_generate_variant:Nn \dim_gset_eq:NN {       c }
\cs_generate_variant:Nn \dim_gset_eq:NN { Nc , cc }
%    \end{macrocode}
% \end{macro}
% \end{macro}
%
% \begin{macro}{\dim_add:Nn, \dim_add:cn}
% \begin{macro}{\dim_gadd:Nn, \dim_gadd:cn}
% \begin{macro}{\dim_sub:Nn, \dim_sub:cn}
% \begin{macro}{\dim_gsub:Nn, \dim_gsub:cn}
%   Using |by| here deals with the (incorrect) case |\dimen123|.
%    \begin{macrocode}
\cs_new_protected:Npn \dim_add:Nn #1#2
  { \tex_advance:D #1 by \@@_eval:w #2 \@@_eval_end: }
\cs_new_protected:Npn \dim_gadd:Nn { \tex_global:D \dim_add:Nn }
\cs_generate_variant:Nn \dim_add:Nn  { c }
\cs_generate_variant:Nn \dim_gadd:Nn { c }
\cs_new_protected:Npn \dim_sub:Nn #1#2
  { \tex_advance:D #1 by - \@@_eval:w #2 \@@_eval_end: }
\cs_new_protected:Npn \dim_gsub:Nn { \tex_global:D \dim_sub:Nn }
\cs_generate_variant:Nn \dim_sub:Nn  { c }
\cs_generate_variant:Nn \dim_gsub:Nn { c }
%    \end{macrocode}
% \end{macro}
% \end{macro}
% \end{macro}
% \end{macro}
%
% \subsection{Utilities for dimension calculations}
%
% \begin{macro}[EXP]{\dim_abs:n}
% \begin{macro}[aux, EXP]{\@@_abs:N}
% \UnitTested
% \begin{macro}[EXP]{\dim_max:nn}
% \begin{macro}[EXP]{\dim_min:nn}
% \begin{macro}[aux, EXP]{\@@_maxmin:wwN}
% \UnitTested
% \UnitTested
%   Functions for $\min$, $\max$, and absolute value with only one evaluation.
%   The absolute value is evaluated by removing a leading~|-| if present.
%    \begin{macrocode}
\cs_new:Npn \dim_abs:n #1
  {
    \exp_after:wN \@@_abs:N
    \dim_use:N \@@_eval:w #1 \@@_eval_end:
  }
\cs_new:Npn \@@_abs:N #1
  { \if_meaning:w - #1 \else: \exp_after:wN #1 \fi: }
\cs_set:Npn \dim_max:nn #1#2
  {
    \dim_use:N \@@_eval:w \exp_after:wN \@@_maxmin:wwN
      \dim_use:N \@@_eval:w #1 \exp_after:wN ;
      \dim_use:N \@@_eval:w #2 ;
      >
    \@@_eval_end:
  }
\cs_set:Npn \dim_min:nn #1#2
  {
    \dim_use:N \@@_eval:w \exp_after:wN \@@_maxmin:wwN
      \dim_use:N \@@_eval:w #1 \exp_after:wN ;
      \dim_use:N \@@_eval:w #2 ;
      <
    \@@_eval_end:
  }
\cs_new:Npn \@@_maxmin:wwN #1 ; #2 ; #3
  {
    \if_dim:w #1 #3 #2 ~
      #1
    \else:
      #2
    \fi:
  }
%    \end{macrocode}
% \end{macro}
% \end{macro}
% \end{macro}
% \end{macro}
% \end{macro}
%
% \begin{macro}{\dim_ratio:nn}
% \begin{macro}[aux]{\@@_ratio:n}
%   With dimension expressions, something like |10 pt * ( 5 pt / 10 pt )| will
%   not work. Instead, the ratio part needs to be converted to an integer
%   expression. Using \cs{__int_value:w} forces everything into |sp|, avoiding
%   any decimal parts.
%    \begin{macrocode}
\cs_new:Npn \dim_ratio:nn #1#2
  { \@@_ratio:n {#1} / \@@_ratio:n {#2} }
\cs_new:Npn \@@_ratio:n #1
  { \__int_value:w \@@_eval:w #1 \@@_eval_end: }
%    \end{macrocode}
% \end{macro}
% \end{macro}
%
% \subsection{Dimension expression conditionals}
%
% \begin{macro}[pTF, EXP]{\dim_compare:nNn}
%   Simple comparison.
%    \begin{macrocode}
\prg_new_conditional:Npnn \dim_compare:nNn #1#2#3 { p , T , F , TF }
  {
    \if_dim:w \@@_eval:w #1 #2 \@@_eval:w #3 \@@_eval_end:
      \prg_return_true: \else: \prg_return_false: \fi:
  }
%    \end{macrocode}
% \end{macro}
%
% \begin{macro}[pTF, EXP]{\dim_compare:n}
% \begin{macro}[aux, EXP]{\@@_compare_aux:w, \@@_compare_aux:NNw}
% \begin{macro}[aux, EXP]
%   {
%     \@@_compare_=:NNw,
%     \@@_compare_<:NNw,
%     \@@_compare_>:NNw,
%     \@@_compare_==:NNw,
%     \@@_compare_!=:NNw,
%     \@@_compare_<=:NNw,
%     \@@_compare_>=:NNw
%   }
%   This code is adapted from the \cs{int_compare:nTF} function.  First
%   evaluate the left-hand side.  Then access the relation symbol by
%   grabbing until \texttt{pt} (with category other), and pursue
%   otherwise just as for \cs{int_compare:nTF}.
%    \begin{macrocode}
\prg_new_conditional:Npnn \dim_compare:n #1 { p , T , F , TF }
  {
    \exp_after:wN \@@_compare_aux:w \dim_use:N \@@_eval:w #1
        \__prg_compare_error: \@@_eval_end:
      \prg_return_true:
    \else:
      \prg_return_false:
    \fi:
  }
\exp_args:Nno \use:nn
  { \cs_new:Npn \@@_compare_aux:w #1 }
  { \tl_to_str:n { pt } }
  #2 \__prg_compare_error:
  {
    \exp_after:wN \@@_compare_aux:NNw #2 ?? \q_mark
    #1 pt #2
  }
\cs_new:Npn \@@_compare_aux:NNw #1#2#3 \q_mark
  {
    \use:c { @@_compare_  #1  \if_meaning:w = #2 =  \fi: :NNw }
      \__prg_compare_error:Nw #1
  }
\cs_new:cpn { @@_compare_=:NNw } #1#2#3 =
  { \if_dim:w #3 = \@@_eval:w }
\cs_new:cpn { @@_compare_<:NNw } #1#2#3 <
  { \if_dim:w #3 < \@@_eval:w }
\cs_new:cpn { @@_compare_>:NNw } #1#2#3 >
  { \if_dim:w #3 > \@@_eval:w }
\cs_new:cpn { @@_compare_==:NNw } #1#2#3 ==
  { \if_dim:w #3 = \@@_eval:w }
\cs_new:cpn { @@_compare_!=:NNw } #1#2#3 !=
  { \reverse_if:N \if_dim:w #3 = \@@_eval:w }
\cs_new:cpn { @@_compare_<=:NNw } #1#2#3 <=
  { \reverse_if:N \if_dim:w #3 > \@@_eval:w }
\cs_new:cpn { @@_compare_>=:NNw } #1#2#3 >=
  { \reverse_if:N \if_dim:w #3 < \@@_eval:w }
%    \end{macrocode}
% \end{macro}
% \end{macro}
% \end{macro}
%
% \begin{macro}[EXP]{\dim_case:nnn}
% \begin{macro}[aux, EXP]{\@@_case_aux:nnn}
% \begin{macro}[aux, EXP]{\@@_case_aux:nw,\@@_case_end:nw}
%   The dimension function is the same as the \texttt{int} version, so there
%   is not much to say here.
%    \begin{macrocode}
\cs_new:Npn \dim_case:nnn #1
  {
    \tex_romannumeral:D
    \exp_args:Nf \@@_case_aux:nnn { \dim_eval:n {#1} }
  }
\cs_new:Npn \@@_case_aux:nnn #1#2#3
  { \@@_case_aux:nw {#1} #2 {#1} {#3} \q_recursion_stop }
\cs_new:Npn \@@_case_aux:nw #1#2#3
  {
    \dim_compare:nNnTF {#1} = {#2}
      { \@@_case_end:nw {#3} }
      { \@@_case_aux:nw {#1} }
  }
\cs_new_eq:NN \@@_case_end:nw \__prg_case_end:nw
%    \end{macrocode}
% \end{macro}
% \end{macro}
% \end{macro}
%
% \subsection{Dimension expression loops}
%
% \begin{macro}{\dim_while_do:nn}
% \begin{macro}{\dim_until_do:nn}
% \begin{macro}{\dim_do_while:nn}
% \begin{macro}{\dim_do_until:nn}
%   |while_do| and |do_while| functions for dimensions. Same as for the
%   |int| type only the names have changed.
%    \begin{macrocode}
\cs_set:Npn \dim_while_do:nn #1#2
  {
    \dim_compare:nT {#1}
      {
        #2
        \dim_while_do:nn {#1} {#2}
      }
  }
\cs_set:Npn \dim_until_do:nn #1#2
  {
    \dim_compare:nF {#1}
      {
        #2
        \dim_until_do:nn {#1} {#2}
      }
  }
\cs_set:Npn \dim_do_while:nn #1#2
  {
    #2
    \dim_compare:nT {#1}
      { \dim_do_while:nn {#1} {#2} }
  }
\cs_set:Npn \dim_do_until:nn #1#2
  {
    #2
    \dim_compare:nF {#1}
      { \dim_do_until:nn {#1} {#2} }
  }
%    \end{macrocode}
% \end{macro}
% \end{macro}
% \end{macro}
% \end{macro}
%
% \begin{macro}{\dim_while_do:nNnn}
% \begin{macro}{\dim_until_do:nNnn}
% \begin{macro}{\dim_do_while:nNnn}
% \begin{macro}{\dim_do_until:nNnn}
%   |while_do| and |do_while| functions for dimensions. Same as for the
%   |int| type only the names have changed.
%    \begin{macrocode}
\cs_set:Npn \dim_while_do:nNnn #1#2#3#4
  {
    \dim_compare:nNnT {#1} #2 {#3}
      {
        #4
        \dim_while_do:nNnn {#1} #2 {#3} {#4}
      }
  }
\cs_set:Npn \dim_until_do:nNnn #1#2#3#4
  {
  \dim_compare:nNnF {#1} #2 {#3}
    {
      #4
      \dim_until_do:nNnn {#1} #2 {#3} {#4}
    }
  }
\cs_set:Npn \dim_do_while:nNnn #1#2#3#4
  {
    #4
    \dim_compare:nNnT {#1} #2 {#3}
      { \dim_do_while:nNnn {#1} #2 {#3} {#4} }
  }
\cs_set:Npn \dim_do_until:nNnn #1#2#3#4
  {
    #4
    \dim_compare:nNnF {#1} #2 {#3}
      { \dim_do_until:nNnn {#1} #2 {#3} {#4} }
  }
%    \end{macrocode}
% \end{macro}
% \end{macro}
% \end{macro}
% \end{macro}
%
% \subsection{Using \texttt{dim} expressions and variables}
%
% \begin{macro}{\dim_eval:n}
%   Evaluating a dimension expression expandably.
%    \begin{macrocode}
\cs_new:Npn \dim_eval:n #1
  { \dim_use:N \@@_eval:w #1 \@@_eval_end: }
%    \end{macrocode}
% \end{macro}
%
% \begin{macro}[EXP, int]{\@@_strip_bp:n}
%    \begin{macrocode}
\cs_new:Npn \@@_strip_bp:n #1
  { \@@_strip_pt:n { 0.996 26 \@@_eval:w #1 \@@_eval_end: } }
%    \end{macrocode}
% \end{macro}
%
% \begin{macro}[EXP]{\@@_strip_pt:n}
% \begin{macro}[EXP, aux]{\@@_strip_pt:w}
%   A function which comes up often enough to deserve a place in the kernel.
%   The idea here is that the input is assumed to be in \texttt{pt}, but can
%   be given in other units, while the output is the value of the dimension
%   in \texttt{pt} but with no units given. This is used a lot by low-level
%   manipulations.
%    \begin{macrocode}
\cs_new:Npn \@@_strip_pt:n #1
  {
    \exp_after:wN
      \@@_strip_pt:w \dim_use:N \@@_eval:w #1 \@@_eval_end: \q_stop
  }
\use:x
  {
    \cs_new:Npn \exp_not:N \@@_strip_pt:w
      ##1 . ##2 \tl_to_str:n { pt } ##3 \exp_not:N \q_stop
      {
        ##1
        \exp_not:N \int_compare:nNnT {##2} > \c_zero
          { . ##2 }
      }
  }
%    \end{macrocode}
% \end{macro}
% \end{macro}
%
% \begin{macro}{\dim_use:N, \dim_use:c}
%   Accessing a \meta{dim}.
%    \begin{macrocode}
\cs_new_eq:NN \dim_use:N \tex_the:D
\cs_generate_variant:Nn \dim_use:N { c }
%    \end{macrocode}
% \end{macro}
%
% \subsection{Viewing \texttt{dim} variables}
%
% \begin{macro}{\dim_show:N, \dim_show:c}
%   Diagnostics.
%    \begin{macrocode}
\cs_new_eq:NN  \dim_show:N \__kernel_register_show:N
\cs_generate_variant:Nn \dim_show:N { c }
%    \end{macrocode}
%  \end{macro}
%
% \begin{macro}{\dim_show:n}
%   Diagnostics.  We don't use the \TeX{} primitive \tn{showthe} to show
%   dimension expressions: this gives a more unified output, since the
%   closing brace is read by the dimension expression in all cases.
%    \begin{macrocode}
\cs_new_protected:Npn \dim_show:n #1
  { \etex_showtokens:D \exp_after:wN { \dim_use:N \@@_eval:w #1 } }
%    \end{macrocode}
%  \end{macro}
%
% \subsection{Constant dimensions}
%
% \begin{variable}{\c_zero_dim, \c_max_dim}
%   Constant dimensions: in package mode, a couple of registers can be saved.
%    \begin{macrocode}
%<*initex>
\dim_const:Nn \c_zero_dim { 0 pt }
\dim_const:Nn \c_max_dim { 16383.99999 pt }
%</initex>
%<*package>
\cs_new_eq:NN \c_zero_dim \z@
\cs_new_eq:NN \c_max_dim  \maxdimen
%</package>
%    \end{macrocode}
% \end{variable}
%
% \subsection{Scratch dimensions}
%
% \begin{variable}{\l_tmpa_dim, \l_tmpb_dim}
% \begin{variable}{\g_tmpa_dim, \g_tmpb_dim}
%    We provide two local and two global scratch registers, maybe we
%    need more or less.
%    \begin{macrocode}
\dim_new:N \l_tmpa_dim
\dim_new:N \l_tmpb_dim
\dim_new:N \g_tmpa_dim
\dim_new:N \g_tmpb_dim
%    \end{macrocode}
% \end{variable}
% \end{variable}
%
% \subsection{Creating and initialising \texttt{skip} variables}
%
% \begin{macro}{\skip_new:N,\skip_new:c}
%    Allocation of a new internal registers.
%    \begin{macrocode}
%<*package>
\cs_new_protected:Npn \skip_new:N #1
  {
    \__chk_if_free_cs:N #1
    \newskip #1
  }
%</package>
\cs_generate_variant:Nn \skip_new:N { c }
%    \end{macrocode}
% \end{macro}
%
% \begin{macro}{\skip_const:Nn, \skip_const:cn}
%   Contrarily to integer constants, we cannot avoid using a register,
%   even for constants.
%    \begin{macrocode}
\cs_new_protected:Npn \skip_const:Nn #1
  {
    \skip_new:N #1
    \skip_gset:Nn #1
  }
\cs_generate_variant:Nn \skip_const:Nn { c }
%    \end{macrocode}
% \end{macro}
%
% \begin{macro}{\skip_zero:N, \skip_zero:c}
% \begin{macro}{\skip_gzero:N, \skip_gzero:c}
%   Reset the register to zero.
%    \begin{macrocode}
\cs_new_protected:Npn \skip_zero:N #1 { #1 \c_zero_skip }
\cs_new_protected:Npn \skip_gzero:N { \tex_global:D \skip_zero:N }
\cs_generate_variant:Nn \skip_zero:N  { c }
\cs_generate_variant:Nn \skip_gzero:N { c }
%    \end{macrocode}
% \end{macro}
% \end{macro}
%
% \begin{macro}
%   {\skip_zero_new:N, \skip_zero_new:c, \skip_gzero_new:N, \skip_gzero_new:c}
%   Create a register if needed, otherwise clear it.
%    \begin{macrocode}
\cs_new_protected:Npn \skip_zero_new:N  #1
  { \skip_if_exist:NTF #1 { \skip_zero:N #1 } { \skip_new:N #1 } }
\cs_new_protected:Npn \skip_gzero_new:N #1
  { \skip_if_exist:NTF #1 { \skip_gzero:N #1 } { \skip_new:N #1 } }
\cs_generate_variant:Nn \skip_zero_new:N  { c }
\cs_generate_variant:Nn \skip_gzero_new:N { c }
%    \end{macrocode}
% \end{macro}
%
% \begin{macro}[pTF]{\skip_if_exist:N, \skip_if_exist:c}
%   Copies of the \texttt{cs} functions defined in \pkg{l3basics}.
%    \begin{macrocode}
\cs_new_eq:NN \skip_if_exist:NTF \cs_if_exist:NTF
\cs_new_eq:NN \skip_if_exist:NT  \cs_if_exist:NT
\cs_new_eq:NN \skip_if_exist:NF  \cs_if_exist:NF
\cs_new_eq:NN \skip_if_exist_p:N \cs_if_exist_p:N
\cs_new_eq:NN \skip_if_exist:cTF \cs_if_exist:cTF
\cs_new_eq:NN \skip_if_exist:cT  \cs_if_exist:cT
\cs_new_eq:NN \skip_if_exist:cF  \cs_if_exist:cF
\cs_new_eq:NN \skip_if_exist_p:c \cs_if_exist_p:c
%    \end{macrocode}
% \end{macro}
%
% \subsection{Setting \texttt{skip} variables}
%
% \begin{macro}{\skip_set:Nn, \skip_set:cn}
% \begin{macro}{\skip_gset:Nn, \skip_gset:cn}
%   Much the same as for dimensions.
%    \begin{macrocode}
\cs_new_protected:Npn \skip_set:Nn #1#2
  { #1 ~ \etex_glueexpr:D #2 \scan_stop: }
\cs_new_protected:Npn \skip_gset:Nn { \tex_global:D \skip_set:Nn }
\cs_generate_variant:Nn \skip_set:Nn  { c }
\cs_generate_variant:Nn \skip_gset:Nn { c }
%    \end{macrocode}
% \end{macro}
% \end{macro}
%
% \begin{macro}
%   {\skip_set_eq:NN,\skip_set_eq:cN, \skip_set_eq:Nc,\skip_set_eq:cc}
% \begin{macro}
%   {\skip_gset_eq:NN,\skip_gset_eq:cN, \skip_gset_eq:Nc,\skip_gset_eq:cc}
%   All straightforward.
%    \begin{macrocode}
\cs_new_protected:Npn \skip_set_eq:NN #1#2 { #1 = #2 }
\cs_generate_variant:Nn \skip_set_eq:NN {       c }
\cs_generate_variant:Nn \skip_set_eq:NN { Nc , cc }
\cs_new_protected:Npn \skip_gset_eq:NN #1#2 { \tex_global:D #1 = #2 }
\cs_generate_variant:Nn \skip_gset_eq:NN {       c }
\cs_generate_variant:Nn \skip_gset_eq:NN { Nc , cc }
%    \end{macrocode}
% \end{macro}
% \end{macro}
%
% \begin{macro}{\skip_add:Nn, \skip_add:cn}
% \begin{macro}{\skip_gadd:Nn, \skip_gadd:cn}
% \begin{macro}{\skip_sub:Nn, \skip_sub:cn}
% \begin{macro}{\skip_gsub:Nn, \skip_gsub:cn}
%   Using |by| here deals with the (incorrect) case |\skip123|.
%    \begin{macrocode}
\cs_new_protected:Npn \skip_add:Nn #1#2
  { \tex_advance:D #1 by \etex_glueexpr:D #2 \scan_stop: }
\cs_new_protected:Npn \skip_gadd:Nn { \tex_global:D \skip_add:Nn }
\cs_generate_variant:Nn \skip_add:Nn  { c }
\cs_generate_variant:Nn \skip_gadd:Nn { c }
\cs_new_protected:Npn \skip_sub:Nn #1#2
  { \tex_advance:D #1 by - \etex_glueexpr:D #2 \scan_stop: }
\cs_new_protected:Npn \skip_gsub:Nn { \tex_global:D \skip_sub:Nn }
\cs_generate_variant:Nn \skip_sub:Nn  { c }
\cs_generate_variant:Nn \skip_gsub:Nn { c }
%    \end{macrocode}
% \end{macro}
% \end{macro}
% \end{macro}
% \end{macro}
%
% \subsection{Skip expression conditionals}
%
% \begin{macro}[pTF]{\skip_if_eq:nn}
%   Comparing skips means doing two expansions to make strings, and then
%   testing them. As a result, only equality is tested.
%    \begin{macrocode}
\prg_new_conditional:Npnn \skip_if_eq:nn #1#2 { p , T , F , TF }
  {
    \if_int_compare:w
      \pdftex_strcmp:D { \skip_eval:n { #1 } } { \skip_eval:n { #2 } }
      = \c_zero
        \prg_return_true:
    \else:
        \prg_return_false:
    \fi:
  }
%    \end{macrocode}
% \end{macro}
%
% \begin{macro}[EXP,pTF]{\skip_if_finite:n}
% \begin{macro}[EXP,aux]{\__skip_if_finite:wwNw}
%   With \eTeX{}, we have an easy access to the order of infinities of
%   the stretch and shrink components of a skip.  However, to access
%   both, we either need to evaluate the expression twice, or evaluate
%   it, then call an auxiliary to extract both pieces of information
%   from the result.  Since we are going to need an auxiliary anyways,
%   it is quicker to make it search for the string \texttt{fil} which
%   characterizes infinite glue.
%    \begin{macrocode}
\cs_set_protected:Npn \__cs_tmp:w #1
  {
    \prg_new_conditional:Npnn \skip_if_finite:n ##1 { p , T , F , TF }
      {
        \exp_after:wN \__skip_if_finite:wwNw
        \skip_use:N \etex_glueexpr:D ##1 ; \prg_return_false:
        #1 ; \prg_return_true: \q_stop
      }
    \cs_new:Npn \__skip_if_finite:wwNw ##1 #1 ##2 ; ##3 ##4 \q_stop {##3}
  }
\exp_args:No \__cs_tmp:w { \tl_to_str:n { fil } }
%    \end{macrocode}
% \end{macro}
% \end{macro}
%
% \subsection{Using \texttt{skip} expressions and variables}
%
% \begin{macro}{\skip_eval:n}
%   Evaluating a skip expression expandably.
%    \begin{macrocode}
\cs_new:Npn \skip_eval:n #1
  { \skip_use:N \etex_glueexpr:D #1 \scan_stop: }
%    \end{macrocode}
% \end{macro}
%
% \begin{macro}{\skip_use:N, \skip_use:c}
%   Accessing a \meta{skip}.
%    \begin{macrocode}
\cs_new_eq:NN \skip_use:N \tex_the:D
\cs_generate_variant:Nn \skip_use:N { c }
%    \end{macrocode}
% \end{macro}
%
% \subsection{Inserting skips into the output}
%
% \begin{macro}{\skip_horizontal:N, \skip_horizontal:c, \skip_horizontal:n}
% \begin{macro}{\skip_vertical:N, \skip_vertical:c, \skip_vertical:n}
%    Inserting skips.
%    \begin{macrocode}
\cs_new_eq:NN  \skip_horizontal:N \tex_hskip:D
\cs_new:Npn \skip_horizontal:n #1
  { \skip_horizontal:N \etex_glueexpr:D #1 \scan_stop: }
\cs_new_eq:NN  \skip_vertical:N \tex_vskip:D
\cs_new:Npn \skip_vertical:n #1
  { \skip_vertical:N \etex_glueexpr:D #1 \scan_stop: }
\cs_generate_variant:Nn \skip_horizontal:N { c }
\cs_generate_variant:Nn \skip_vertical:N { c }
%    \end{macrocode}
% \end{macro}
% \end{macro}
%
% \subsection{Viewing \texttt{skip} variables}
%
% \begin{macro}{\skip_show:N, \skip_show:c}
%   Diagnostics.
%    \begin{macrocode}
\cs_new_eq:NN  \skip_show:N \__kernel_register_show:N
\cs_generate_variant:Nn \skip_show:N { c }
%    \end{macrocode}
%  \end{macro}
%
% \begin{macro}{\skip_show:n}
%   Diagnostics.  We don't use the \TeX{} primitive \tn{showthe} to show
%   skip expressions: this gives a more unified output, since the
%   closing brace is read by the skip expression in all cases.
%    \begin{macrocode}
\cs_new_protected:Npn \skip_show:n #1
  { \etex_showtokens:D \exp_after:wN { \tex_the:D \etex_glueexpr:D #1 } }
%    \end{macrocode}
% \end{macro}
%
% \subsection{Constant skips}
%
% \begin{macro}{\c_zero_skip}
% \begin{macro}{\c_max_skip}
%   Skips with no rubber component are just dimensions.
%    \begin{macrocode}
\skip_new:N \c_zero_skip
\cs_new_eq:NN \c_max_skip  \c_max_dim
%    \end{macrocode}
% \end{macro}
% \end{macro}
%
% \subsection{Scratch skips}
%
% \begin{variable}{\l_tmpa_skip, \l_tmpb_skip}
% \begin{variable}{\g_tmpa_skip, \g_tmpb_skip}
%    We provide two local and two global scratch registers, maybe we
%    need more or less.
%    \begin{macrocode}
\skip_new:N \l_tmpa_skip
\skip_new:N \l_tmpb_skip
\skip_new:N \g_tmpa_skip
\skip_new:N \g_tmpb_skip
%    \end{macrocode}
% \end{variable}
% \end{variable}
%
% \subsection{Creating and initialising \texttt{muskip} variables}
%
% \begin{macro}{\muskip_new:N, \muskip_new:c}
% And then we add muskips.
%    \begin{macrocode}
%<*package>
\cs_new_protected:Npn \muskip_new:N #1
  {
    \__chk_if_free_cs:N #1
    \newmuskip #1
  }
%</package>
\cs_generate_variant:Nn \muskip_new:N { c }
%    \end{macrocode}
% \end{macro}
%
% \begin{macro}{\muskip_const:Nn, \muskip_const:cn}
%   Contrarily to integer constants, we cannot avoid using a register,
%   even for constants.
%    \begin{macrocode}
\cs_new_protected:Npn \muskip_const:Nn #1
  {
    \muskip_new:N #1
    \muskip_gset:Nn #1
  }
\cs_generate_variant:Nn \muskip_const:Nn { c }
%    \end{macrocode}
% \end{macro}
%
% \begin{macro}{\muskip_zero:N, \muskip_zero:c}
% \begin{macro}{\muskip_gzero:N, \muskip_gzero:c}
%   Reset the register to zero.
%    \begin{macrocode}
\cs_new_protected:Npn \muskip_zero:N #1
  { #1 \c_zero_muskip }
\cs_new_protected:Npn \muskip_gzero:N { \tex_global:D \muskip_zero:N }
\cs_generate_variant:Nn \muskip_zero:N  { c }
\cs_generate_variant:Nn \muskip_gzero:N { c }
%    \end{macrocode}
% \end{macro}
% \end{macro}
%
% \begin{macro}
%   {
%     \muskip_zero_new:N, \muskip_zero_new:c,
%     \muskip_gzero_new:N, \muskip_gzero_new:c
%   }
%   Create a register if needed, otherwise clear it.
%    \begin{macrocode}
\cs_new_protected:Npn \muskip_zero_new:N  #1
  { \muskip_if_exist:NTF #1 { \muskip_zero:N #1 } { \muskip_new:N #1 } }
\cs_new_protected:Npn \muskip_gzero_new:N #1
  { \muskip_if_exist:NTF #1 { \muskip_gzero:N #1 } { \muskip_new:N #1 } }
\cs_generate_variant:Nn \muskip_zero_new:N  { c }
\cs_generate_variant:Nn \muskip_gzero_new:N { c }
%    \end{macrocode}
% \end{macro}
%
% \begin{macro}[pTF]{\muskip_if_exist:N, \muskip_if_exist:c}
%   Copies of the \texttt{cs} functions defined in \pkg{l3basics}.
%    \begin{macrocode}
\cs_new_eq:NN \muskip_if_exist:NTF \cs_if_exist:NTF
\cs_new_eq:NN \muskip_if_exist:NT  \cs_if_exist:NT
\cs_new_eq:NN \muskip_if_exist:NF  \cs_if_exist:NF
\cs_new_eq:NN \muskip_if_exist_p:N \cs_if_exist_p:N
\cs_new_eq:NN \muskip_if_exist:cTF \cs_if_exist:cTF
\cs_new_eq:NN \muskip_if_exist:cT  \cs_if_exist:cT
\cs_new_eq:NN \muskip_if_exist:cF  \cs_if_exist:cF
\cs_new_eq:NN \muskip_if_exist_p:c \cs_if_exist_p:c
%    \end{macrocode}
% \end{macro}
%
% \subsection{Setting \texttt{muskip} variables}
%
% \begin{macro}{\muskip_set:Nn, \muskip_set:cn}
% \begin{macro}{\muskip_gset:Nn, \muskip_gset:cn}
%   This should be pretty familiar.
%    \begin{macrocode}
\cs_new_protected:Npn \muskip_set:Nn #1#2
  { #1 ~ \etex_muexpr:D #2 \scan_stop: }
\cs_new_protected:Npn \muskip_gset:Nn { \tex_global:D \muskip_set:Nn }
\cs_generate_variant:Nn \muskip_set:Nn  { c }
\cs_generate_variant:Nn \muskip_gset:Nn { c }
%    \end{macrocode}
% \end{macro}
% \end{macro}
%
% \begin{macro}
%   {
%     \muskip_set_eq:NN, \muskip_set_eq:cN,
%     \muskip_set_eq:Nc, \muskip_set_eq:cc
%   }
% \begin{macro}
%   {
%     \muskip_gset_eq:NN,\muskip_gset_eq:cN,
%     \muskip_gset_eq:Nc,\muskip_gset_eq:cc
%   }
%   All straightforward.
%    \begin{macrocode}
\cs_new_protected:Npn \muskip_set_eq:NN #1#2 { #1 = #2 }
\cs_generate_variant:Nn \muskip_set_eq:NN {       c }
\cs_generate_variant:Nn \muskip_set_eq:NN { Nc , cc }
\cs_new_protected:Npn \muskip_gset_eq:NN #1#2 { \tex_global:D #1 = #2 }
\cs_generate_variant:Nn \muskip_gset_eq:NN {       c }
\cs_generate_variant:Nn \muskip_gset_eq:NN { Nc , cc }
%    \end{macrocode}
% \end{macro}
% \end{macro}
%
% \begin{macro}{\muskip_add:Nn, \muskip_add:cn}
% \begin{macro}{\muskip_gadd:Nn, \muskip_gadd:cn}
% \begin{macro}{\muskip_sub:Nn, \muskip_sub:cn}
% \begin{macro}{\muskip_gsub:Nn, \muskip_gsub:cn}
%   Using |by| here deals with the (incorrect) case |\muskip123|.
%    \begin{macrocode}
\cs_new_protected:Npn \muskip_add:Nn #1#2
  { \tex_advance:D #1 by \etex_muexpr:D #2 \scan_stop: }
\cs_new_protected:Npn \muskip_gadd:Nn { \tex_global:D \muskip_add:Nn }
\cs_generate_variant:Nn \muskip_add:Nn  { c }
\cs_generate_variant:Nn \muskip_gadd:Nn { c }
\cs_new_protected:Npn \muskip_sub:Nn #1#2
  { \tex_advance:D #1 by - \etex_muexpr:D #2 \scan_stop: }
\cs_new_protected:Npn \muskip_gsub:Nn { \tex_global:D \muskip_sub:Nn }
\cs_generate_variant:Nn \muskip_sub:Nn  { c }
\cs_generate_variant:Nn \muskip_gsub:Nn { c }
%    \end{macrocode}
% \end{macro}
% \end{macro}
% \end{macro}
% \end{macro}
%
% \subsection{Using \texttt{muskip} expressions and variables}
%
% \begin{macro}{\muskip_eval:n}
%   Evaluating a muskip expression expandably.
%    \begin{macrocode}
\cs_new:Npn \muskip_eval:n #1
  { \muskip_use:N \etex_muexpr:D #1 \scan_stop: }
%    \end{macrocode}
% \end{macro}
%
% \begin{macro}{\muskip_use:N, \muskip_use:c}
%   Accessing a \meta{muskip}.
%    \begin{macrocode}
\cs_new_eq:NN \muskip_use:N \tex_the:D
\cs_generate_variant:Nn \muskip_use:N { c }
%    \end{macrocode}
% \end{macro}
%
% \subsection{Viewing \texttt{muskip} variables}
%
% \begin{macro}{\muskip_show:N, \muskip_show:c}
%   Diagnostics.
%    \begin{macrocode}
\cs_new_eq:NN  \muskip_show:N \__kernel_register_show:N
\cs_generate_variant:Nn \muskip_show:N { c }
%    \end{macrocode}
% \end{macro}
%
% \begin{macro}{\muskip_show:n}
%   Diagnostics.  We don't use the \TeX{} primitive \tn{showthe} to show
%   muskip expressions: this gives a more unified output, since the
%   closing brace is read by the muskip expression in all cases.
%    \begin{macrocode}
\cs_new_protected:Npn \muskip_show:n #1
  { \etex_showtokens:D \exp_after:wN { \tex_the:D \etex_muexpr:D #1 } }
%    \end{macrocode}
% \end{macro}
%
% \subsection{Constant muskips}
%
% \begin{macro}{\c_zero_muskip}
% \begin{macro}{\c_max_muskip}
%   Constant muskips given by their value.
%    \begin{macrocode}
\muskip_const:Nn \c_zero_muskip { 0 mu }
\muskip_const:Nn \c_max_muskip  { 16383.99999 mu }
%    \end{macrocode}
% \end{macro}
% \end{macro}
%
% \subsection{Scratch muskips}
%
% \begin{variable}{\l_tmpa_muskip, \l_tmpb_muskip}
% \begin{variable}{\g_tmpa_muskip, \g_tmpb_muskip}
%    We provide two local and two global scratch registers, maybe we
%    need more or less.
%    \begin{macrocode}
\muskip_new:N \l_tmpa_muskip
\muskip_new:N \l_tmpb_muskip
\muskip_new:N \g_tmpa_muskip
\muskip_new:N \g_tmpb_muskip
%    \end{macrocode}
% \end{variable}
% \end{variable}
%
% \subsection{Deprecated functions}
%
% Deprecated on 2012-05-10, for removal by 2012-08-31.
%
% \begin{macro}[EXP, pTF]{\skip_if_infinite_glue:n}
%   Reverse of \cs{skip_if_finite:nTF}.
%    \begin{macrocode}
\prg_new_conditional:Npnn \skip_if_infinite_glue:n #1 { p , T , F , TF }
  { \skip_if_finite:nTF {#1} \prg_return_false: \prg_return_true: }
%    \end{macrocode}
% \end{macro}
%
% Deprecated 2012-06-03 for removal after 2012-12-31.
%
% \begin{macro}[EXP]{\prg_case_dim:nnn}
%   Moved here, was in \pkg{l3prg} but load order means we define it here
%   now.
%    \begin{macrocode}
\cs_new_eq:NN \prg_case_dim:nnn \dim_case:nnn
%    \end{macrocode}
% \end{macro}
% 
% \begin{macro}{\dim_eval:w, \dim_eval_end:}
%    \begin{macrocode}
\cs_new_eq:NN \dim_eval:w    \__dim_eval:w
\cs_new_eq:NN \dim_eval_end: \__dim_eval_end:
%    \end{macrocode}
% \end{macro}
%
% \begin{macro}{\dim_set_max:Nn, \dim_set_max:cn}
% \begin{macro}{\dim_set_min:Nn, \dim_set_min:cn}
% \begin{macro}{\dim_gset_max:Nn, \dim_gset_max:cn}
% \begin{macro}{\dim_gset_min:Nn, \dim_gset_min:cn}
% \begin{macro}[aux]{\@@_set_max:NNNn}
%   Deprecated on 2012-09-09 for removal after 2012-12-31.
%    \begin{macrocode}
\cs_new_protected_nopar:Npn \dim_set_max:Nn
  { \@@_set_max:NNNn < \dim_set:Nn }
\cs_new_protected_nopar:Npn \dim_gset_max:Nn
  { \@@_set_max:NNNn < \dim_gset:Nn }
\cs_new_protected_nopar:Npn \dim_set_min:Nn
  { \@@_set_max:NNNn > \dim_set:Nn }
\cs_new_protected_nopar:Npn \dim_gset_min:Nn
  { \@@_set_max:NNNn > \dim_gset:Nn }
\cs_new_protected:Npn \@@_set_max:NNNn #1#2#3#4
  { \dim_compare:nNnT {#3} #1 {#4} { #2 #3 {#4} } }
\cs_generate_variant:Nn \dim_set_max:Nn  { c }
\cs_generate_variant:Nn \dim_gset_max:Nn { c }
\cs_generate_variant:Nn \dim_set_min:Nn  { c }
\cs_generate_variant:Nn \dim_gset_min:Nn { c }
%    \end{macrocode}
% \end{macro}
% \end{macro}
% \end{macro}
% \end{macro}
% \end{macro}
%
%    \begin{macrocode}
%</initex|package>
%    \end{macrocode}
%
% \end{implementation}
%
% \PrintIndex
