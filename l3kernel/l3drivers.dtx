% \iffalse meta-comment
%
%% File: l3drivers.dtx Copyright(C) 2011-2014 The LaTeX3 Project
%%
%% It may be distributed and/or modified under the conditions of the
%% LaTeX Project Public License (LPPL), either version 1.3c of this
%% license or (at your option) any later version.  The latest version
%% of this license is in the file
%%
%%    http://www.latex-project.org/lppl.txt
%%
%% This file is part of the "l3kernel bundle" (The Work in LPPL)
%% and all files in that bundle must be distributed together.
%%
%% The released version of this bundle is available from CTAN.
%%
%% -----------------------------------------------------------------------
%%
%% The development version of the bundle can be found at
%%
%%    http://www.latex-project.org/svnroot/experimental/trunk/
%%
%% for those people who are interested.
%%
%%%%%%%%%%%
%% NOTE: %%
%%%%%%%%%%%
%%
%%   Snapshots taken from the repository represent work in progress and may
%%   not work or may contain conflicting material!  We therefore ask
%%   people _not_ to put them into distributions, archives, etc. without
%%   prior consultation with the LaTeX Project Team.
%%
%% -----------------------------------------------------------------------
%%
%
%<*driver>
\documentclass[full]{l3doc}
%</driver>
%<*driver|package>
\GetIdInfo$Id$
  {L3 Experimental drivers}
%</driver|package>
%<*driver>
\begin{document}
  \DocInput{\jobname.dtx}
\end{document}
%</driver>
% \fi
%
% \title{^^A
%   The \textsf{l3drivers} package\\ Drivers^^A
%   \thanks{This file describes v\ExplFileVersion,
%     last revised \ExplFileDate.}^^A
% }
%
% \author{^^A
%  The \LaTeX3 Project\thanks
%    {^^A
%      E-mail:
%        \href{mailto:latex-team@latex-project.org}
%          {latex-team@latex-project.org}^^A
%    }^^A
% }
%
% \date{Released \ExplFileDate}
%
% \maketitle
%
% \begin{documentation}
%
% \TeX{} relies on drivers in order to carry out a number of tasks, such
% as using color, including graphics and setting up hyper-links. The nature
% of the code required depends on the exact driver in use. Currently,
% \LaTeX3 is aware of the following drivers:
% \begin{itemize}
%   \item \texttt{pdfmode}:  The \enquote{driver} for direct PDF output by
%     \emph{both} \pdfTeX{} and \LuaTeX{} (no separate driver is used in this
%     case: the engine deals with PDF creation itself).
%   \item \texttt{dvips}: The \texttt{dvips} program, which works in
%     conjugation with \pdfTeX{} or \LuaTeX{} in DVI mode.
%   \item \texttt{dvipdfmx}: The \texttt{dvipdfmx} program, which works in
%     conjugation with \pdfTeX{} or \LuaTeX{} in DVI mode.
%   \item \texttt{xdvipdfmx}: The driver used by \XeTeX{}.
% \end{itemize}
%
% The code here is all very low-level, and should not in general be used
% outside of the kernel. It is also important to note that many of the
% functions here are closely tied to the immediate level \enquote{up}:
% several variable values must be in the correct locations for the driver
% code to function.
%
% \section{Box clipping}
%
% \begin{function}[added = 2011-11-11]{\__driver_box_use_clip:N}
%   \begin{syntax}
%     \cs{__driver_box_use_clip:N} \meta{box}
%   \end{syntax}
%   Inserts the content of the \meta{box} at the current insertion point
%   such that any material outside of the bounding box will not be displayed
%   by the driver. The material in the \meta{box} is still places in the
%   output stream: the clipping takes place at a driver level.
%   
%   This function should only be used within a surrounding horizontal
%   box construct.
% \end{function}
%
% \section{Box rotation and scaling}
%
% \begin{function}[added = 2011-09-01, updated = 2013-12-27]
%  {\__driver_box_rotate_begin:, \__driver_box_rotate_end:}
%   \begin{syntax}
%     \cs{__driver_box_rotate_begin:}
%     \cs{box_use:N} \cs{l__box_internal_box}
%     \cs{__driver_box_rotate_end:}
%   \end{syntax}
%   Rotates the \meta{box material} anti-clockwise around the current
%   insertion point. The angle of rotation (in degrees counter-clockwise)
%   and the sine and cosine of this angle should be stored in
%   \cs{l__box_angle_fp}, \cs{l__box_sin_fp} and \cs{l__box_cos_fp},
%   respectively. Typically, the box material inserted between the beginning
%   and end markers will be stored in \cs{l__box_internal_box}: this fact is
%   required by some drivers to obtain the correct output.
% \end{function}
%
% \begin{function}[added = 2011-09-02, updated = 2013-12-27]
%   {\__driver_box_scale_begin:, \__driver_box_scale_end:}
%   \begin{syntax}
%     \cs{__driver_box_scale_begin:}
%     \meta{box material}
%     \cs{__driver_box_scale_end:}
%   \end{syntax}
%   Scales the \meta{box material} (which should be either a \cs{box_use:N}
%   or \cs{hbox:n} construct). The \meta{box material} is scaled by the values
%   stored in \cs{l__box_scale_x_fp} and \cs{l__box_scale_y_fp} in the
%   horizontal and vertical directions, respectively. This function is
%   also reused when resizing boxes: at a driver level, only scalings are
%   supported and so the higher-level code must convert the absolute sizes
%   to scale factors.
% \end{function}
%
% \section{Color support}
%
% \begin{function}[added = 2011-09-03, updated = 2012-05-18]
%   {\__driver_color_ensure_current:}
%   \begin{syntax}
%     \cs{__driver_color_ensure_current:}
%   \end{syntax}
%   Ensures that the color used to typeset material is that which was
%   set when the material was placed in a box. This function is therefore
%   required inside any \enquote{color safe} box to ensure that the box may
%   be inserted in a location where the foreground color has been altered,
%   while preserving the color used in the box.
% \end{function}
%
% \end{documentation}
%
% \begin{implementation}
%
% \section{\pkg{l3drivers} Implementation}
%
%    \begin{macrocode}
%<*initex|package>
%<@@=driver>
%    \end{macrocode}
%
%    \begin{macrocode}
%<*package>
\ProvidesExplFile
%<*dvipdfmx>
  {l3dvidpfmx.def}{\ExplFileDate}{\ExplFileVersion}
  {L3 Experimental driver: dvipdfmx}
%</dvipdfmx>
%<*dvips>
  {l3dvips.def}{\ExplFileDate}{\ExplFileVersion}
  {L3 Experimental driver: dvips}
%</dvips>
%<*pdfmode>
  {l3pdfmode.def}{\ExplFileDate}{\ExplFileVersion}
  {L3 Experimental driver: PDF mode}
%</pdfmode>
%<*xdvipdfmx>
  {l3xdvidpfmx.def}{\ExplFileDate}{\ExplFileVersion}
  {L3 Experimental driver: xdvipdfmx}
%</xdvipdfmx>
%</package>
%    \end{macrocode}
%
% \subsection{Settings for direct PDF output}
%
% If the driver loaded is \texttt{pdfmode} then direct PDF output is required.
% (This may of course alter: it might be that the driver is picked based on the
% value of \cs{pdftex_pdfoutput:D}.)
%    \begin{macrocode}
%<*initex>
%<*pdfmode>
\pdftex_pdfoutput:D = 1 \scan_stop:
%</pdfmode>
%</initex>
%    \end{macrocode}
%
% Set up the driver for direct PDF output to set the PDF origin equal to
% \TeX{}'s standard origin. The other settings make use of PDF~$1.5$, which
% is standard in \TeX{} Live 2011 and should be a reasonable baseline for
% the future.
%    \begin{macrocode}
%<*initex>
%<*pdfmode>
\pdftex_pdfhorigin:D          = 1 true in \scan_stop:
\pdftex_pdfvorigin:D          = 1 true in \scan_stop:
\pdftex_pdfdecimaldigits:D    = 3         \scan_stop:
\pdftex_pdfpkresolution:D     = 600       \scan_stop:
\pdftex_pdfminorversion:D     = 5         \scan_stop:
\pdftex_pdfcompresslevel:D    = 9         \scan_stop:
\pdftex_pdfobjcompresslevel:D = 2         \scan_stop:
%</pdfmode>
%</initex>
%    \end{macrocode}
%
% \subsection{Driver utility functions}
%
% \begin{macro}{\@@_state_save:, \@@_state_restore:}
%   All of the drivers have a stack for saving the graphic state. These
%   have slightly different interfaces. For both \texttt{dvips} and
%   \texttt{(x)dvipdfmx} this is done using an appropriate special. Note
%   that here and later, the \texttt{dvipdfmx} documentation does not cover
%   the |literal| key word but that this appears to behave in the same way as
%   \pdfTeX{}'s \tn{pdfliteral} (making life easier all-round).
%    \begin{macrocode}
%<*!pdfmode>
\cs_new_protected_nopar:Npn \@@_state_save:
%<*dvips>
  { \tex_special:D { ps:gsave } }
%</dvips>
%<*dvipdfmx|xdvipdfmx>
  { \tex_special:D { pdf:literal~q } }
%</dvipdfmx|xdvipdfmx>
\cs_new_protected_nopar:Npn \@@_state_restore:
%<*dvips>
  { \tex_special:D { ps:grestore } }
%</dvips>
%<*dvipdfmx|xdvipdfmx>
  { \tex_special:D { pdf:literal~Q } }
%</dvipdfmx|xdvipdfmx>
%</!pdfmode>
%    \end{macrocode}
%   For direct PDF output there is also a need to worry about the version
%   of \pdfTeX{} in use: the \tn{pdfsave} primitive was only introduced
%   in version~1.40.0.
%    \begin{macrocode}
%<*pdfmode>
\cs_if_exist:NTF \pdftex_pdfsave:D
  {
    \cs_new_eq:NN \@@_state_save:    \pdftex_pdfsave:D
    \cs_new_eq:NN \@@_state_restore: \pdftex_pdfrestore:D
  }
  {
    \cs_new_protected_nopar:Npn \@@_state_save:
      { \pdftex_pdfliteral:D { q } }
    \cs_new_protected_nopar:Npn \@@_state_restore:
      { \pdftex_pdfliteral:D { Q } }
  }
%</pdfmode>
%    \end{macrocode}
% \end{macro}
%
% \begin{macro}[int]{\@@_literal:n}
%   The driver code needs to pass on a lot of \enquote{raw} information to
%   the underlying binary. The exact command is driver-dependent but the
%   concept is general enough to use a single function. However, it is
%   important to remember this is a convenient shortcut: the arguments will
%   be driver-specific. Note that these functions set the transformation matrix
%   to the current position: contrast with \cs{\@@_literal_direct:n}.
%    \begin{macrocode}
\cs_new_protected:Npn \@@_literal:n #1
%<*dvipdfmx|xdvipdfmx>
  { \tex_special:D { pdf:literal~ #1 } }
%</dvipdfmx|xdvipdfmx>
%    \end{macrocode}
%   In the case of \texttt{dvips} there is no build-in saving of the current
%   position, and so some additional PostScript is required to set up the
%   transformation matrix and also to restore it afterwards. Notice the use
%   of the stack to save the current position \enquote{up front} and to
%   move back to it at the end of the process.
%    \begin{macrocode}
%<*dvips>
  {
    \tex_special:D
      {
        ps:
          currentpoint~
          currentpoint~translate~
          #1 ~
          neg~exch~neg~exch~translate
      }
  }
%</dvips>
%<*pdfmode>
  { \pdftex_pdfliteral:D {#1} }
%</pdfmode>
%    \end{macrocode}
% \end{macro}
%
% \begin{macro}[int]{\@@_literal_direct:n}
%   Even \enquote{lower level} than \cs{\@@_literal:n}, these commands do
%   not set the transformation matrix but simply dump the driver code directly
%   into the output. In the \texttt{(x)dvipdfmx} case this two-part keyword
%   is documented (\emph{cf.}~|literal| alone).
%    \begin{macrocode}
\cs_new_protected:Npn \@@_literal_direct:n #1
%<*dvipdfmx|xdvipdfmx>
  { \tex_special:D { pdf:literal~direct~ #1 } }
%</dvipdfmx|xdvipdfmx>
%<*dvips>
  { \tex_special:D { ps:: #1 } }
%</dvips>
%<*pdfmode>
  { \pdftex_pdfliteral:D direct {#1} }
%</pdfmode>
%    \end{macrocode}
% \end{macro}
%
% \begin{macro}[int, EXP]{\@@_absolute_lengths:n}
%   The \texttt{dvips} driver scales all absolute dimensions based
%   on the output resolution selected and any \TeX{} magnification. Thus
%   for any operation involving absolute lengths there is a correction to
%   make. This is based on |normalscale| from \texttt{special.pro}.
%    \begin{macrocode}
%<*dvips>
\cs_new:Npn \@@_absolute_lengths:n #1
  {
     /savedmatrix~matrix~currentmatrix~def~
     Resolution~72~div~VResolution~72~div~scale~
     DVImag~dup~scale~
     #1 ~
     savedmatrix~setmatrix
  }
%</dvips>
%    \end{macrocode}
% \end{macro}
%
% \begin{macro}[int]{\@@_matrix:n}
%   Here the appropriate function is set up to insert an affine matrix
%   into the PDF. With a new enough \pdfTeX{} (version~1.40.0 or later)
%   there is a primitive for this, which only needs the rotation/scaling/skew
%   part. With an older \pdfTeX{} or with \texttt{(x)dvipdfmx} the matrix
%   also has to include a translation part: that is always zero and so is
%   built in here.
%    \begin{macrocode}
%<*pdfmode>
\cs_if_exist:NTF \pdftex_pdfsetmatrix:D
  {
    \cs_new_protected:Npn \@@_matrix:n #1
      { \pdftex_pdfsetmatrix:D {#1} }
  }
  {
    \cs_new_protected:Npn \@@_matrix:n #1
      { \@@_literal:n { #1 \c_space_tl 0~0~cm } }
  }
%</pdfmode>
%<*dvipdfmx|xdvipdfmx>
\cs_new_protected:Npn \@@_matrix:n #1
  { \@@_literal:n { #1 \c_space_tl 0~0~cm } }
%</dvipdfmx|xdvipdfmx>
%    \end{macrocode}
% \end{macro}
%
% \subsection{Box clipping}
%
% \begin{macro}{\@@_box_use_clip:N}
%   The overall logic to clipping a box is the same in all cases. The general
%   method is to save the current location, define a clipping path equivalent
%   to the bounding box, then insert the content at the current position
%   and in a zero width box. The \enquote{real} width is then made up using
%   a horizontal skip before tidying up. There are other approaches that
%   can be taken (for example using XForm objects), but the logic here shares
%   as much code as possible and uses the same conversions (and so same
%   rounding errors) in all three cases.
%    \begin{macrocode}
\cs_new_protected:Npn \@@_box_use_clip:N #1
  {
    \@@_state_save:
%<*dvips>
    \@@_literal:n
      {
        \@@_absolute_lengths:n
          {
            0~
            \__dim_strip_bp:n { \box_dp:N #1 } ~
            \__dim_strip_bp:n { \box_wd:N #1 } ~
            \__dim_strip_bp:n { - \box_ht:N #1 - \box_dp:N #1 } ~
            rectclip
          }
      }
%</dvips>
%<*dvipdfmx|pdfmode|xdvipdfmx>
    \@@_literal:n
      {
        0~
        \__dim_strip_bp:n { - \box_dp:N #1 } ~
        \__dim_strip_bp:n { \box_wd:N #1 } ~
        \__dim_strip_bp:n { \box_ht:N #1 + \box_dp:N #1 } ~
        re~W~n
      }
%</dvipdfmx|pdfmode|xdvipdfmx>
%    \end{macrocode}
%   Insert the material in a box of no width, restore the graphic state
%   and then insert the necessary width.
%    \begin{macrocode}
    \hbox_overlap_right:n { \box_use:N #1 }
    \@@_state_restore:
    \skip_horizontal:n { \box_wd:N #1 }
  }
%    \end{macrocode}
% \end{macro}
%
% \subsection{Box rotation and scaling}
%
% \begin{macro}{\@@_box_rotate_begin:, \@@_box_rotate_end:}
%   The driver for \texttt{dvips} works with a simple rotation angle.
%   In PDF mode, an affine matrix is used instead. The transformation for
%   \texttt{(x)dvipdfmx} can be done either way: the affine approach is
%   chosen here as where possible we pick the PDF-style route.
%
%   In both cases, some rounding code is included to limit the floating
%   point values to five decimal places. There is no point using any more
%   as \TeX{}'s dimensions are of that precision, and the extra figures
%   will simply bloat the PDF and make values harder to trace. In the
%   case where the sine and cosine are used, we store the rounded values to
%   avoid rounding twice. There are also a couple of comparisons to ensure
%   that |-0| is not written to the output, as this avoids any issues with
%   problematic display programs.  Note that numbers are compared to~$0$
%   after rounding.
%    \begin{macrocode}
\cs_new_protected_nopar:Npn \@@_box_rotate_begin:
  {
    \@@_state_save:
%<*dvipdfmx|pdfmode|xdvipdfmx>
    \box_set_wd:Nn \l__box_internal_box \c_zero_dim
    \fp_set:Nn \l__box_cos_fp { round ( \l__box_cos_fp , 5 ) }
    \fp_compare:nNnT \l__box_cos_fp = \c_zero_fp
      { \fp_zero:N \l__box_cos_fp }
    \fp_set:Nn \l__box_sin_fp { round ( \l__box_sin_fp , 5 ) }
    \@@_matrix:n
      {
        \fp_use:N \l__box_cos_fp \c_space_tl
        \fp_compare:nNnTF \l__box_sin_fp = \c_zero_fp
          { 0~0 }
          {
            \fp_use:N \l__box_sin_fp
            \c_space_tl
            \fp_eval:n { -\l__box_sin_fp }
          }
        \c_space_tl
        \fp_use:N \l__box_cos_fp
      }
%</dvipdfmx|pdfmode|xdvipdfmx>
%<*dvips>
    \fp_set:Nn \l__box_angle_fp { round ( \l__box_angle_fp , 5 ) }
    \@@_literal:n
      {
        \fp_compare:nNnTF \l__box_angle_fp = \c_zero_fp
          { 0 }
          { \fp_eval:n { - \l__box_angle_fp } }
        \c_space_tl
        rotate
      }
%</dvips>
  }
%    \end{macrocode}
%   The end of a rotation means tidying up the output grouping.
%    \begin{macrocode}
\cs_new_eq:NN \@@_box_rotate_end: \@@_state_restore:
%    \end{macrocode}
% \end{macro}
%
% \begin{macro}{\@@_box_scale_begin:, \@@_box_scale_end:}
%   Scaling is not dissimilar to rotation, but the calculations are somewhat
%   less complex.
%    \begin{macrocode}
\cs_new_protected_nopar:Npn \@@_box_scale_begin:
  {
    \@@_state_save:
    \fp_set:Nn \l__box_scale_x_fp { round ( \l__box_scale_x_fp , 5 ) }
    \fp_set:Nn \l__box_scale_y_fp { round ( \l__box_scale_y_fp , 5 ) }
%<*dvips>
    \@@_literal:n
      {
        \fp_use:N \l__box_scale_x_fp \c_space_tl
        \fp_use:N \l__box_scale_y_fp \c_space_tl
        scale
      }
%</dvips>
%<*dvipdfmx|pdfmode|xdvipdfmx>
    \@@_matrix:n
      {
        \fp_use:N \l__box_scale_x_fp \c_space_tl
        0~0~
        \fp_use:N \l__box_scale_y_fp
      }
%</dvipdfmx|pdfmode|xdvipdfmx>
  }
\cs_new_eq:NN \@@_box_scale_end: \@@_state_restore:
%    \end{macrocode}
% \end{macro}
%
% \subsection{Color support}
%
% \begin{variable}{\l_@@_current_color_tl}
%   The current color is needed by all of the engines, but the way this
%   is stored varies.
%    \begin{macrocode}
\tl_new:N \l_@@_current_color_tl
%<*dvipdfmx|xdvipdfmx>
\tl_set:Nn \l_@@_current_color_tl { gray~0 }
%</dvipdfmx|xdvipdfmx>
%<*dvips>
\tl_set:Nn \l_@@_current_color_tl { Black }
%</dvips>
%<*pdfmode>
\tl_set:Nn \l_@@_current_color_tl { 0~g~0~G }
%</pdfmode>
%    \end{macrocode}
% \end{variable}
%
% \begin{variable}{\l_@@_color_stack_int}
%   \pdfTeX{} (version~1.40.0 or later) and \LuaTeX{} have multiple stacks
%   available, and the color stack therefore needs a number when in PDF mode.
%    \begin{macrocode}
%<*pdfmode>
\int_new:N \l_@@_color_stack_int
%</pdfmode>
%    \end{macrocode}
% \end{variable}
%
% \begin{macro}{\@@_color_ensure_current:}
% \begin{macro}[aux]{\@@_color_reset:}
%   Setting the current color depends on the nature of the color stack
%   available. In all cases there is a need to reset the color after
%   the current group.
%    \begin{macrocode}
%<*dvipdfmx|dvips|xdvipdfmx>
\cs_new_protected_nopar:Npn \@@_color_ensure_current:
  {
    \tex_special:D { color~push~\l_@@_current_color_tl }
    \group_insert_after:N \@@_color_reset:
  }
\cs_new_protected_nopar:Npn \@@_color_reset:
  { \tex_special:D { color~pop } }
%</dvipdfmx|dvips|xdvipdfmx>
%    \end{macrocode}
%   Once again there is a version switch for \pdfTeX{}, as the
%   \tn{pdfcolorstack} primitive was introduced in version~1.40.0.
%    \begin{macrocode}
%<*pdfmode>
\cs_if_exist:NTF \pdftex_pdfcolorstack:D
  {
    \cs_new_protected_nopar:Npn \@@_color_ensure_current:
      {
        \pdftex_pdfcolorstack:D \l_@@_color_stack_int push
          { \l_@@_current_color_tl }
        \group_insert_after:N \@@_color_reset:
      }
    \cs_new_protected_nopar:Npn \@@_color_reset:
      { \pdftex_pdfcolorstack:D \l_@@_color_stack_int pop }
  }
  {
    \cs_new_protected_nopar:Npn \@@_color_ensure_current:
      {
        \@@_literal:n { \l_@@_current_color_tl }
        \group_insert_after:N \@@_color_reset:
      }
    \cs_new_protected_nopar:Npn \@@_color_reset:
      { \@@_literal:n { \l_@@_current_color_tl } }
  }
%</pdfmode>
%    \end{macrocode}
% \end{macro}
% \end{macro}
%
%    \begin{macrocode}
%</initex|package>
%    \end{macrocode}
%
% \end{implementation}
%
% \PrintIndex
