% \iffalse meta-comment
%
%% File: l3fp-basics.dtx
%
% Copyright (C) 2011-2025 The LaTeX Project
%
% It may be distributed and/or modified under the conditions of the
% LaTeX Project Public License (LPPL), either version 1.3c of this
% license or (at your option) any later version.  The latest version
% of this license is in the file
%
%    https://www.latex-project.org/lppl.txt
%
% This file is part of the "l3kernel bundle" (The Work in LPPL)
% and all files in that bundle must be distributed together.
%
% -----------------------------------------------------------------------
%
% The development version of the bundle can be found at
%
%    https://github.com/latex3/latex3
%
% for those people who are interested.
%
%<*driver>
\documentclass[full,kernel]{l3doc}
\begin{document}
  \DocInput{\jobname.dtx}
\end{document}
%</driver>
% \fi
%
% \title{^^A
%   The \pkg{l3fp-basics} module\\
%   Floating point arithmetic^^A
% }
% \author{^^A
%  The \LaTeX{} Project\thanks
%    {^^A
%      E-mail:
%        \href{mailto:latex-team@latex-project.org}
%          {latex-team@latex-project.org}^^A
%    }^^A
% }
% \date{Released 2025-03-26}
%
% \maketitle
%
% \begin{documentation}
%
% \end{documentation}
%
% \begin{implementation}
%
% \section{\pkg{l3fp-basics} implementation}
%
%    \begin{macrocode}
%<*package>
%    \end{macrocode}
%
%    \begin{macrocode}
%<@@=fp>
%    \end{macrocode}
%
% The \pkg{l3fp-basics} module implements addition, subtraction,
% multiplication, and division of two floating points, and the absolute
% value and sign-changing operations on one floating point.
% All operations implemented in this module yield the outcome of
% rounding the infinitely precise result of the operation to the
% nearest floating point.
%
% Some algorithms used below end up being quite similar to some
% described in \enquote{What Every Computer Scientist Should Know About
%   Floating Point Arithmetic}, by David Goldberg, which can be found at
% \texttt{http://cr.yp.to/2005-590/goldberg.pdf}.
%
% \begin{macro}[EXP]
%   {
%     \@@_parse_word_abs:N   ,
%     \@@_parse_word_logb:N  ,
%     \@@_parse_word_sign:N  ,
%     \@@_parse_word_sqrt:N  ,
%   }
%   Unary functions.
%    \begin{macrocode}
\cs_new:Npn \@@_parse_word_abs:N
  { \@@_parse_unary_function:NNN \@@_set_sign_o:w 0 }
\cs_new:Npn \@@_parse_word_logb:N
  { \@@_parse_unary_function:NNN \@@_logb_o:w ? }
\cs_new:Npn \@@_parse_word_sign:N
  { \@@_parse_unary_function:NNN \@@_sign_o:w ? }
\cs_new:Npn \@@_parse_word_sqrt:N
  { \@@_parse_unary_function:NNN \@@_sqrt_o:w ? }
%    \end{macrocode}
% \end{macro}
%
% \subsection{Addition and subtraction}
%
% We define here two functions, \cs{@@_-_o:ww} and \cs{@@_+_o:ww}, which
% perform the subtraction and addition of their two floating point
% operands, and expand the tokens following the result once.
%
% A more obscure function, \cs{@@_add_big_i_o:wNww}, is used in
% \pkg{l3fp-expo}.
%
% The logic goes as follows:
% \begin{itemize}
%   \item \cs{@@_-_o:ww} calls \cs{@@_+_o:ww} to do the work, with the
%     sign of the second operand flipped;
%   \item \cs{@@_+_o:ww} dispatches depending on the type of floating
%     point, calling specialized auxiliaries;
%   \item in all cases except summing two normal floating point numbers,
%     we return one or the other operands depending on the signs, or
%     detect an invalid operation in the case of $\infty - \infty$;
%   \item for normal floating point numbers, compare the signs;
%   \item to add two floating point numbers of the same sign or of
%     opposite signs, shift the significand of the smaller one to match the
%     bigger one, perform the addition or subtraction of significands,
%     check for a carry, round, and pack using the
%     \cs[no-index]{@@_basics_pack_\ldots{}} functions.
% \end{itemize}
% The trickiest part is to round correctly when adding or subtracting
% normal floating point numbers.
%
% \subsubsection{Sign, exponent, and special numbers}
%
% \begin{macro}[EXP]{\@@_-_o:ww}
%   The \cs{@@_+_o:ww} auxiliary has a hook: it takes one argument
%   between the first \cs{s_@@} and \cs{@@_chk:w}, which is applied to
%   the sign of the second operand.  Positioning the hook there means
%   that \cs{@@_+_o:ww} can still perform the sanity check that it was
%   followed by \cs{s_@@}.
%    \begin{macrocode}
\cs_new:cpe { @@_-_o:ww } \s_@@
  {
    \exp_not:c { @@_+_o:ww }
    \exp_not:n { \s_@@ \@@_neg_sign:N }
  }
%    \end{macrocode}
% \end{macro}
%
% \begin{macro}[EXP]{\@@_+_o:ww}
%   This function is either called directly with an empty |#1| to
%   compute an addition, or it is called by \cs{@@_-_o:ww} with
%   \cs{@@_neg_sign:N} as |#1| to compute a subtraction, in which case
%   the second operand's sign should be changed.  If the
%   \meta{types} |#2| and |#4| are the same, dispatch to case |#2| ($0$,
%   $1$, $2$, or $3$), where we call specialized functions: thanks to
%   \cs{int_value:w}, those receive the tweaked \meta{sign_2}
%   (expansion of |#1#5|) as an argument.  If the \meta{types} are
%   distinct, the result is simply the floating point number with the
%   highest \meta{type}.  Since case $3$ (used for two \texttt{nan})
%   also picks the first operand, we can also use it when \meta{type_1}
%   is greater than \meta{type_2}.  Also note that we don't need to
%   worry about \meta{sign_2} in that case since the second operand is
%   discarded.
%    \begin{macrocode}
\cs_new:cpn { @@_+_o:ww }
    \s_@@ #1 \@@_chk:w #2 #3 \@@_sep: \s_@@ \@@_chk:w #4 #5
  {
    \if_case:w
      \if_meaning:w #2 #4
        #2
      \else:
        \if_int_compare:w #2 > #4 \exp_stop_f:
          3
        \else:
          4
        \fi:
      \fi:
      \exp_stop_f:
           \exp_after:wN \@@_add_zeros_o:Nww \int_value:w
    \or:   \exp_after:wN \@@_add_normal_o:Nww \int_value:w
    \or:   \exp_after:wN \@@_add_inf_o:Nww \int_value:w
    \or:   \@@_case_return_i_o:ww
    \else: \exp_after:wN \@@_add_return_ii_o:Nww \int_value:w
    \fi:
    #1 #5
    \s_@@ \@@_chk:w #2 #3 \@@_sep:
    \s_@@ \@@_chk:w #4 #5
  }
%    \end{macrocode}
% \end{macro}
%
% \begin{macro}[EXP]{\@@_add_return_ii_o:Nww}
%   Ignore the first operand, and return the second, but using the sign
%   |#1| rather than |#4|.  As usual, expand after the floating point.
%    \begin{macrocode}
\cs_new:Npn \@@_add_return_ii_o:Nww #1 #2 \@@_sep: \s_@@ \@@_chk:w #3 #4
  { \@@_exp_after_o:w \s_@@ \@@_chk:w #3 #1 }
%    \end{macrocode}
% \end{macro}
%
% \begin{macro}[EXP]{\@@_add_zeros_o:Nww}
%   Adding two zeros yields \cs{c_zero_fp}, except if both zeros were
%   $-0$.
%    \begin{macrocode}
\cs_new:Npn \@@_add_zeros_o:Nww #1 \s_@@ \@@_chk:w 0 #2
  {
    \if_int_compare:w #2 #1 = 20 \exp_stop_f:
      \exp_after:wN \@@_add_return_ii_o:Nww
    \else:
      \@@_case_return_i_o:ww
    \fi:
    #1
    \s_@@ \@@_chk:w 0 #2
  }
%    \end{macrocode}
% \end{macro}
%
% \begin{macro}[EXP]{\@@_add_inf_o:Nww}
%   If both infinities have the same sign, just return that infinity,
%   otherwise, it is an invalid operation.  We find out if that invalid
%   operation is an addition or a subtraction by testing whether the
%   tweaked \meta{sign_2} (|#1|) and the \meta{sign_2} (|#4|) are
%   identical.
%    \begin{macrocode}
\cs_new:Npn \@@_add_inf_o:Nww
    #1 \s_@@ \@@_chk:w 2 #2 #3\@@_sep: \s_@@ \@@_chk:w 2 #4
  {
    \if_meaning:w #1 #2
      \@@_case_return_i_o:ww
    \else:
      \@@_case_use:nw
        {
          \exp_last_unbraced:Nf \@@_invalid_operation_o:Nww
            { \token_if_eq_meaning:NNTF #1 #4 + - }
        }
    \fi:
    \s_@@ \@@_chk:w 2 #2 #3\@@_sep:
    \s_@@ \@@_chk:w 2 #4
  }
%    \end{macrocode}
% \end{macro}
%
% \begin{macro}[EXP]{\@@_add_normal_o:Nww}
%   \begin{quote}
%     \cs{@@_add_normal_o:Nww} \meta{sign_2}
%       \cs{s_@@} \cs{@@_chk:w} |1| \meta{sign_1}
%       \meta{exp_1} \meta{body_1} \cs{@@_sep:}
%       \cs{s_@@} \cs{@@_chk:w} |1| \meta{initial sign_2}
%       \meta{exp_2} \meta{body_2} \cs{@@_sep:}
%   \end{quote}
%   We now have two normal numbers to add, and we have to check signs
%   and exponents more carefully before performing the addition.
%    \begin{macrocode}
\cs_new:Npn \@@_add_normal_o:Nww #1 \s_@@ \@@_chk:w 1 #2
  {
    \if_meaning:w #1#2
      \exp_after:wN \@@_add_npos_o:NnwNnw
    \else:
      \exp_after:wN \@@_sub_npos_o:NnwNnw
    \fi:
    #2
  }
%    \end{macrocode}
% \end{macro}
%
% \subsubsection{Absolute addition}
%
% In this subsection, we perform the addition of two positive normal
% numbers.
%
% \begin{macro}[EXP]{\@@_add_npos_o:NnwNnw}
%   \begin{quote}
%     \cs{@@_add_npos_o:NnwNnw} \meta{sign_1} \meta{exp_1} \meta{body_1}
%     \cs{@@_sep:} \cs{s_@@} \cs{@@_chk:w} |1| \meta{initial sign_2}
%     \meta{exp_2} \meta{body_2} \cs{@@_sep:}
%   \end{quote}
%   Since we are doing an addition, the final sign is \meta{sign_1}.
%   Start an \cs{@@_int_eval:w}, responsible for computing the exponent:
%   the result, and the \meta{final sign} are then given to
%   \cs{@@_sanitize:Nw} which checks for overflow.  The exponent is
%   computed as the largest exponent |#2| or |#5|, incremented if there
%   is a carry.  To add the significands, we decimate the smaller number by
%   the difference between the exponents.  This is done by
%   \cs{@@_add_big_i:wNww} or \cs{@@_add_big_ii:wNww}.  We need to bring
%   the final sign with us in the midst of the calculation to round
%   properly at the end.
%    \begin{macrocode}
\cs_new:Npn \@@_add_npos_o:NnwNnw #1#2#3 \@@_sep: \s_@@ \@@_chk:w 1 #4 #5
  {
    \exp_after:wN \@@_sanitize:Nw
    \exp_after:wN #1
    \int_value:w \@@_int_eval:w
      \if_int_compare:w #2 > #5 \exp_stop_f:
        #2
        \exp_after:wN \@@_add_big_i_o:wNww \int_value:w -
      \else:
        #5
        \exp_after:wN \@@_add_big_ii_o:wNww \int_value:w
      \fi:
      \@@_int_eval:w #5 - #2 \@@_sep: #1 #3\@@_sep:
  }
%    \end{macrocode}
% \end{macro}
%
% \begin{macro}[rEXP]{\@@_add_big_i_o:wNww}
% \begin{macro}[rEXP]{\@@_add_big_ii_o:wNww}
%   \begin{quote}
%     \cs{@@_add_big_i_o:wNww} \meta{shift} \cs{@@_sep:} \meta{final sign}
%       \meta{body_1} \cs{@@_sep:} \meta{body_2} \cs{@@_sep:}
%   \end{quote}
%   Used in \pkg{l3fp-expo}.
%   Shift the significand of the small number, then add with
%   \cs{@@_add_significand_o:NnnwnnnnN}.
%    \begin{macrocode}
\cs_new:Npn \@@_add_big_i_o:wNww #1\@@_sep: #2 #3\@@_sep: #4\@@_sep:
  {
    \@@_decimate:nNnnnn {#1}
      \@@_add_significand_o:NnnwnnnnN
      #4
    #3
    #2
  }
\cs_new:Npn \@@_add_big_ii_o:wNww #1\@@_sep: #2 #3\@@_sep: #4\@@_sep:
  {
    \@@_decimate:nNnnnn {#1}
      \@@_add_significand_o:NnnwnnnnN
      #3
    #4
    #2
  }
%    \end{macrocode}
% \end{macro}
% \end{macro}
%
% \begin{macro}[rEXP]{\@@_add_significand_o:NnnwnnnnN}
% \begin{macro}[rEXP]
%   {\@@_add_significand_pack:NNNNNNN, \@@_add_significand_test_o:N}
%   \begin{quote}\raggedright
%     \cs{@@_add_significand_o:NnnwnnnnN}
%       \meta{rounding digit}
%       \Arg{Y'_1} \Arg{Y'_2} \meta{extra-digits} \cs{@@_sep:}
%       \Arg{X_1} \Arg{X_2} \Arg{X_3} \Arg{X_4}
%       \meta{final sign}
%   \end{quote}
%   To round properly, we must know at which digit the rounding
%   should occur. This requires to know whether the addition
%   produces an overall carry or not. Thus, we do the computation
%   now and check for a carry, then go back and do the rounding.
%   The rounding may cause a carry in very rare cases such as
%   $0.99\cdots 95 \to 1.00\cdots 0$, but this situation always
%   give an exact power of $10$, for which it is easy to correct
%   the result at the end.
%    \begin{macrocode}
\cs_new:Npn \@@_add_significand_o:NnnwnnnnN #1 #2#3 #4\@@_sep: #5#6#7#8
  {
    \exp_after:wN \@@_add_significand_test_o:N
    \int_value:w \@@_int_eval:w 1#5#6 + #2
      \exp_after:wN \@@_add_significand_pack:NNNNNNN
      \int_value:w \@@_int_eval:w 1#7#8 + #3 \@@_sep: #1
  }
\cs_new:Npn \@@_add_significand_pack:NNNNNNN #1 #2#3#4#5#6#7
  {
    \if_meaning:w 2 #1
      + 1
    \fi:
    \@@_sep: #2 #3 #4 #5 #6 #7 \@@_sep:
  }
\cs_new:Npn \@@_add_significand_test_o:N #1
  {
    \if_meaning:w 2 #1
      \exp_after:wN \@@_add_significand_carry_o:wwwNN
    \else:
      \exp_after:wN \@@_add_significand_no_carry_o:wwwNN
    \fi:
  }
%    \end{macrocode}
% \end{macro}
% \end{macro}
%
% \begin{macro}[rEXP]{\@@_add_significand_no_carry_o:wwwNN}
%   \begin{quote}
%     \cs{@@_add_significand_no_carry_o:wwwNN}
%       \meta{8d} \cs{@@_sep:} \meta{6d} \cs{@@_sep:} \meta{2d} \cs{@@_sep:}
%       \meta{rounding digit} \meta{sign}
%   \end{quote}
%   If there's no carry, grab all the digits again and round.  The
%   packing function \cs{@@_basics_pack_high:NNNNNw} takes care of the
%   case where rounding brings a carry.
%    \begin{macrocode}
\cs_new:Npn \@@_add_significand_no_carry_o:wwwNN
    #1\@@_sep: #2\@@_sep: #3#4 \@@_sep: #5#6
  {
    \exp_after:wN \@@_basics_pack_high:NNNNNw
    \int_value:w \@@_int_eval:w 1 #1
      \exp_after:wN \@@_basics_pack_low:NNNNNw
      \int_value:w \@@_int_eval:w 1 #2 #3#4
        + \@@_round:NNN #6 #4 #5
        \exp_after:wN \@@_sep:
  }
%    \end{macrocode}
% \end{macro}
%
% \begin{macro}[rEXP]{\@@_add_significand_carry_o:wwwNN}
%   \begin{quote}
%     \cs{@@_add_significand_carry_o:wwwNN}
%       \meta{8d} \cs{@@_sep:} \meta{6d} \cs{@@_sep:} \meta{2d} \cs{@@_sep:}
%       \meta{rounding digit} \meta{sign}
%   \end{quote}
%   The case where there is a carry is very similar.  Rounding can even
%   raise the first digit from $1$ to $2$, but we don't care.
%    \begin{macrocode}
\cs_new:Npn \@@_add_significand_carry_o:wwwNN
    #1\@@_sep: #2\@@_sep: #3#4\@@_sep: #5#6
  {
    + 1
    \exp_after:wN \@@_basics_pack_weird_high:NNNNNNNNw
    \int_value:w \@@_int_eval:w 1 1 #1
      \exp_after:wN \@@_basics_pack_weird_low:NNNNw
      \int_value:w \@@_int_eval:w 1 #2#3 +
        \exp_after:wN \@@_round:NNN
        \exp_after:wN #6
        \exp_after:wN #3
        \int_value:w \@@_round_digit:Nw #4 #5 \@@_sep:
        \exp_after:wN \@@_sep:
  }
%    \end{macrocode}
% \end{macro}
%
% \subsubsection{Absolute subtraction}
%
% \begin{macro}[EXP]{\@@_sub_npos_o:NnwNnw}
% \begin{macro}[EXP]{\@@_sub_eq_o:Nnwnw, \@@_sub_npos_ii_o:Nnwnw}
%   \begin{quote}
%     \cs{@@_sub_npos_o:NnwNnw}
%       \meta{sign_1} \meta{exp_1} \meta{body_1} \cs{@@_sep:}
%       \cs{s_@@} \cs{@@_chk:w} |1|
%       \meta{initial sign_2} \meta{exp_2} \meta{body_2} \cs{@@_sep:}
%   \end{quote}
%   Rounding properly in some modes requires to know what the sign of
%   the result will be.  Thus, we start by comparing the exponents and
%   significands.  If the numbers coincide, return zero.  If the second
%   number is larger, swap the numbers and call
%   \cs{@@_sub_npos_i_o:Nnwnw} with the opposite of \meta{sign_1}.
%    \begin{macrocode}
\cs_new:Npn \@@_sub_npos_o:NnwNnw
    #1#2#3\@@_sep: \s_@@ \@@_chk:w 1 #4#5#6\@@_sep:
  {
    \if_case:w
        \@@_compare_npos:nwnw {#2} #3\@@_sep: {#5} #6\@@_sep: \exp_stop_f:
      \exp_after:wN \@@_sub_eq_o:Nnwnw
    \or:
      \exp_after:wN \@@_sub_npos_i_o:Nnwnw
    \else:
      \exp_after:wN \@@_sub_npos_ii_o:Nnwnw
    \fi:
    #1 {#2} #3\@@_sep: {#5} #6\@@_sep:
  }
\cs_new:Npn \@@_sub_eq_o:Nnwnw #1#2\@@_sep: #3\@@_sep:
  { \exp_after:wN \c_zero_fp }
\cs_new:Npn \@@_sub_npos_ii_o:Nnwnw #1 #2\@@_sep: #3\@@_sep:
  {
    \exp_after:wN \@@_sub_npos_i_o:Nnwnw
      \int_value:w \@@_neg_sign:N #1
      #3\@@_sep: #2\@@_sep:
  }
%    \end{macrocode}
% \end{macro}
% \end{macro}
%
% \begin{macro}[EXP]{\@@_sub_npos_i_o:Nnwnw}
%   After the computation is done, \cs{@@_sanitize:Nw} checks for
%   overflow/underflow.  It expects the \meta{final sign} and the
%   \meta{exponent} (delimited by \cs{@@_sep:}).  Start an integer expression
%   for the exponent, which starts with the exponent of the largest number,
%   and may be decreased if the two numbers are very close.  If the two
%   numbers have the same exponent, call the \texttt{near} auxiliary.
%   Otherwise, decimate $y$, then call the \texttt{far} auxiliary to
%   evaluate the difference between the two significands.  Note that we
%   decimate by $1$ less than one could expect.
%    \begin{macrocode}
\cs_new:Npn \@@_sub_npos_i_o:Nnwnw #1 #2#3\@@_sep: #4#5\@@_sep:
  {
    \exp_after:wN \@@_sanitize:Nw
    \exp_after:wN #1
    \int_value:w \@@_int_eval:w
      #2
      \if_int_compare:w #2 = #4 \exp_stop_f:
        \exp_after:wN \@@_sub_back_near_o:nnnnnnnnN
      \else:
        \exp_after:wN \@@_decimate:nNnnnn \exp_after:wN
          { \int_value:w \@@_int_eval:w #2 - #4 - 1 \exp_after:wN }
          \exp_after:wN \@@_sub_back_far_o:NnnwnnnnN
      \fi:
        #5
      #3
      #1
  }
%    \end{macrocode}
% \end{macro}
%
% \begin{macro}[rEXP]{\@@_sub_back_near_o:nnnnnnnnN}
% \begin{macro}[rEXP]
%   {\@@_sub_back_near_pack:NNNNNNw, \@@_sub_back_near_after:wNNNNw}
%   \begin{quote}
%     \cs{@@_sub_back_near_o:nnnnnnnnN}
%       \Arg{Y_1} \Arg{Y_2} \Arg{Y_3} \Arg{Y_4}
%       \Arg{X_1} \Arg{X_2} \Arg{X_3} \Arg{X_4}
%       \meta{final sign}
%   \end{quote}
%   In this case, the subtraction is exact, so we discard the
%   \meta{final sign} |#9|.  The very large shifts of $10^{9}$ and
%   $1.1\cdot10^{9}$ are unnecessary here, but allow the auxiliaries to
%   be reused later.  Each integer expression produces a $10$ digit
%   result.  If the resulting $16$ digits start with a $0$, then we need
%   to shift the group, padding with trailing zeros.
%    \begin{macrocode}
\cs_new:Npn \@@_sub_back_near_o:nnnnnnnnN #1#2#3#4 #5#6#7#8 #9
  {
    \exp_after:wN \@@_sub_back_near_after:wNNNNw
    \int_value:w \@@_int_eval:w 10#5#6 - #1#2 - 11
      \exp_after:wN \@@_sub_back_near_pack:NNNNNNw
      \int_value:w \@@_int_eval:w 11#7#8 - #3#4 \exp_after:wN \@@_sep:
  }
\cs_new:Npn \@@_sub_back_near_pack:NNNNNNw #1#2#3#4#5#6#7 \@@_sep:
  { + #1#2 \@@_sep: {#3#4#5#6} {#7} \@@_sep: }
\cs_new:Npn \@@_sub_back_near_after:wNNNNw 10 #1#2#3#4 #5 \@@_sep:
  {
    \if_meaning:w 0 #1
      \exp_after:wN \@@_sub_back_shift:wnnnn
    \fi:
    \@@_sep: {#1#2#3#4} {#5}
  }
%    \end{macrocode}
% \end{macro}
% \end{macro}
%
% \begin{macro}[rEXP]{\@@_sub_back_shift:wnnnn}
% \begin{macro}[rEXP]
%   {
%     \@@_sub_back_shift_ii:ww,
%     \@@_sub_back_shift_iii:NNNNNNNNw,
%     \@@_sub_back_shift_iv:nnnnw
%   }
%   \begin{quote}
%     \cs{@@_sub_back_shift:wnnnn} \cs{@@_sep:}
%       \Arg{Z_1} \Arg{Z_2} \Arg{Z_3} \Arg{Z_4} \cs{@@_sep:}
%   \end{quote}
%   This function is called with $\meta{Z_1}\leq 999$.  Act with
%   \tn{number} to trim leading zeros from \meta{Z_1} \meta{Z_2} (we
%   don't do all four blocks at once, since non-zero blocks would then
%   overflow \TeX{}'s integers).  If the first two blocks are zero, the
%   auxiliary receives an empty |#1| and trims |#2#30| from leading
%   zeros, yielding a total shift between $7$ and~$16$ to the exponent.
%   Otherwise we get the shift from |#1| alone, yielding a result
%   between $1$ and~$6$.  Once the exponent is taken care of, trim
%   leading zeros from |#1#2#3| (when |#1| is empty, the space before
%   |#2#3| is ignored), get four blocks of $4$~digits and finally clean
%   up.  Trailing zeros are added so that digits can be grabbed safely.
%    \begin{macrocode}
\cs_new:Npn \@@_sub_back_shift:wnnnn \@@_sep: #1#2
  {
    \exp_after:wN \@@_sub_back_shift_ii:ww
    \int_value:w #1 #2 0 \@@_sep:
  }
\cs_new:Npn \@@_sub_back_shift_ii:ww #1 0 \@@_sep: #2#3 \@@_sep:
  {
    \if_meaning:w @ #1 @
      - 7
      - \exp_after:wN \use_i:nnn
        \exp_after:wN \@@_sub_back_shift_iii:NNNNNNNNw
        \int_value:w #2#3 0 ~ 123456789\@@_sep:
    \else:
      - \@@_sub_back_shift_iii:NNNNNNNNw #1 123456789\@@_sep:
    \fi:
    \exp_after:wN \@@_pack_twice_four:wNNNNNNNN
    \exp_after:wN \@@_pack_twice_four:wNNNNNNNN
    \exp_after:wN \@@_sub_back_shift_iv:nnnnw
    \exp_after:wN \@@_sep:
    \int_value:w
    #1 ~ #2#3 0 ~ 0000 0000 0000 000 \@@_sep:
  }
\cs_new:Npn \@@_sub_back_shift_iii:NNNNNNNNw #1#2#3#4#5#6#7#8#9\@@_sep: {#8}
\cs_new:Npn \@@_sub_back_shift_iv:nnnnw #1 \@@_sep: #2 \@@_sep:
  { \@@_sep: #1 \@@_sep: }
%    \end{macrocode}
% \end{macro}
% \end{macro}
%
% \begin{macro}[rEXP]{\@@_sub_back_far_o:NnnwnnnnN}
%   \begin{quote}\raggedright
%     \cs{@@_sub_back_far_o:NnnwnnnnN}
%       \meta{rounding} \Arg{Y'_1} \Arg{Y'_2} \meta{extra-digits} \cs{@@_sep:}
%       \Arg{X_1} \Arg{X_2} \Arg{X_3} \Arg{X_4}
%       \meta{final sign}
%   \end{quote}
%   If the difference is greater than $10^{\meta{expo_x}}$, call the
%   \texttt{very_far} auxiliary.  If the result is less than
%   $10^{\meta{expo_x}}$, call the \texttt{not_far} auxiliary.  If it is
%   too close a call to know yet, namely if $1 \meta{Y'_1} \meta{Y'_2} =
%   \meta{X_1} \meta{X_2} \meta{X_3} \meta{X_4} 0$, then call the
%   \texttt{quite_far} auxiliary.  We use the odd combination of space
%   and semi-colon delimiters to allow the \texttt{not_far} auxiliary to
%   grab each piece individually, the \texttt{very_far} auxiliary to use
%   \cs{@@_pack_eight:wNNNNNNNN}, and the \texttt{quite_far} to ignore
%   the significands easily (using the \cs{@@_sep:} delimiter).
%    \begin{macrocode}
\cs_new:Npn \@@_sub_back_far_o:NnnwnnnnN #1 #2#3 #4\@@_sep: #5#6#7#8
  {
    \if_case:w
      \if_int_compare:w 1 #2 = #5#6 \use_i:nnnn #7 \exp_stop_f:
        \if_int_compare:w #3 = \use_none:n #7#8 0 \exp_stop_f:
          0
        \else:
          \if_int_compare:w #3 > \use_none:n #7#8 0 - \fi: 1
        \fi:
      \else:
        \if_int_compare:w 1 #2 > #5#6 \use_i:nnnn #7 - \fi: 1
      \fi:
      \exp_stop_f:
           \exp_after:wN \@@_sub_back_quite_far_o:wwNN
    \or:   \exp_after:wN \@@_sub_back_very_far_o:wwwwNN
    \else: \exp_after:wN \@@_sub_back_not_far_o:wwwwNN
    \fi:
    #2 ~ #3 \@@_sep: #5 #6 ~ #7 #8 \@@_sep: #1
  }
%    \end{macrocode}
% \end{macro}
%
% \begin{macro}[EXP]{\@@_sub_back_quite_far_o:wwNN}
% \begin{macro}[EXP]{\@@_sub_back_quite_far_ii:NN}
%   The easiest case is when $x-y$ is extremely close to a power of
%   $10$, namely the first digit of $x$ is $1$, and all others vanish
%   when subtracting $y$.  Then the \meta{rounding} |#3| and the
%   \meta{final sign} |#4| control whether we get $1$ or $0.9999 9999
%   9999 9999$.  In the usual round-to-nearest mode, we get $1$
%   whenever the \meta{rounding} digit is less than or equal to $5$
%   (remember that the \meta{rounding} digit is only equal to $5$ if
%   there was no further non-zero digit).
%    \begin{macrocode}
\cs_new:Npn \@@_sub_back_quite_far_o:wwNN #1\@@_sep: #2\@@_sep: #3#4
  {
    \exp_after:wN \@@_sub_back_quite_far_ii:NN
    \exp_after:wN #3
    \exp_after:wN #4
  }
\cs_new:Npn \@@_sub_back_quite_far_ii:NN #1#2
  {
    \if_case:w \@@_round_neg:NNN #2 0 #1
      \exp_after:wN \use_i:nn
    \else:
      \exp_after:wN \use_ii:nn
    \fi:
      { \@@_sep: {1000} {0000} {0000} {0000} \@@_sep: }
      { - 1 \@@_sep: {9999} {9999} {9999} {9999} \@@_sep: }
  }
%    \end{macrocode}
% \end{macro}
% \end{macro}
%
% \begin{macro}[rEXP]{\@@_sub_back_not_far_o:wwwwNN}
%   In the present case, $x$ and $y$ have different exponents, but
%   $y$~is large enough that $x-y$ has a smaller exponent than~$x$.
%   Decrement the exponent (with |-1|).  Then proceed in a way
%   similar to the \texttt{near} auxiliaries seen earlier, but
%   multiplying $x$ by~$10$ (|#30| and |#40| below), and with the added
%   quirk that the \meta{rounding} digit has to be taken into account.
%   Namely, we may have to decrease the result by one unit if
%   \cs{@@_round_neg:NNN} returns~$1$.  This function expects the
%   \meta{final sign}~|#6|, the last digit of |1100000000+#40-#2|, and
%   the \meta{rounding} digit.  Instead of redoing the computation for
%   the second argument, we note that \cs{@@_round_neg:NNN} only cares
%   about its parity, which is identical to that of the last digit
%   of~|#2|.
%    \begin{macrocode}
\cs_new:Npn \@@_sub_back_not_far_o:wwwwNN #1 ~ #2\@@_sep: #3 ~ #4\@@_sep: #5#6
  {
    - 1
    \exp_after:wN \@@_sub_back_near_after:wNNNNw
    \int_value:w \@@_int_eval:w 1#30 - #1 - 11
      \exp_after:wN \@@_sub_back_near_pack:NNNNNNw
      \int_value:w \@@_int_eval:w 11 0000 0000 + #40 - #2
        - \exp_after:wN \@@_round_neg:NNN
          \exp_after:wN #6
          \use_none:nnnnnnn #2 #5
        \exp_after:wN \@@_sep:
  }
%    \end{macrocode}
% \end{macro}
%
% \begin{macro}[EXP]{\@@_sub_back_very_far_o:wwwwNN}
% \begin{macro}[EXP]{\@@_sub_back_very_far_ii_o:nnNwwNN}
%   The case where $x-y$ and $x$ have the same exponent is a bit more
%   tricky, mostly because it cannot reuse the same auxiliaries.  Shift
%   the $y$~significand by adding a leading~$0$.  Then the logic is similar
%   to the \texttt{not_far} functions above.  Rounding is a bit more
%   complicated: we have two \meta{rounding} digits |#3| and |#6| (from
%   the decimation, and from the new shift) to take into account, and
%   getting the parity of the main result requires a computation.  The
%   first \cs{int_value:w} triggers the second one because the number
%   is unfinished; we can thus not use $0$ in place of $2$ there.
%    \begin{macrocode}
\cs_new:Npn \@@_sub_back_very_far_o:wwwwNN #1#2#3#4#5#6#7
  {
    \@@_pack_eight:wNNNNNNNN
    \@@_sub_back_very_far_ii_o:nnNwwNN
    { 0 #1#2#3 #4#5#6#7 }
    \@@_sep:
  }
\cs_new:Npn \@@_sub_back_very_far_ii_o:nnNwwNN
    #1#2 \@@_sep: #3 \@@_sep: #4 ~ #5\@@_sep: #6#7
  {
    \exp_after:wN \@@_basics_pack_high:NNNNNw
    \int_value:w \@@_int_eval:w 1#4 - #1 - 1
      \exp_after:wN \@@_basics_pack_low:NNNNNw
      \int_value:w \@@_int_eval:w 2#5 - #2
        - \exp_after:wN \@@_round_neg:NNN
          \exp_after:wN #7
          \int_value:w
            \if_int_odd:w \@@_int_eval:w #5 - #2 \@@_int_eval_end:
              1 \else: 2 \fi:
          \int_value:w \@@_round_digit:Nw #3 #6 \@@_sep:
      \exp_after:wN \@@_sep:
  }
%    \end{macrocode}
% \end{macro}
% \end{macro}
%
% \subsection{Multiplication}
%
% \subsubsection{Signs, and special numbers}
%
% \begin{macro}[EXP]{\@@_*_o:ww}
%   We go through an auxiliary, which is common with \cs{@@_/_o:ww}.
%   The first argument is the operation, used for the invalid operation
%   exception.  The second is inserted in a formula to dispatch cases
%   slightly differently between multiplication and division.  The third
%   is the operation for normal floating points.  The fourth is there
%   for extra cases needed in \cs{@@_/_o:ww}.
%    \begin{macrocode}
\cs_new:cpn { @@_*_o:ww }
  {
    \@@_mul_cases_o:NnNnww
      *
      { - 2 + }
      \@@_mul_npos_o:Nww
      { }
  }
%    \end{macrocode}
% \end{macro}
%
% \begin{macro}[EXP]{\@@_mul_cases_o:nNnnww}
%   Split into $10$ cases ($12$ for division).
%   If both numbers are normal, go to case $0$
%   (same sign) or case $1$ (opposite signs): in both cases, call
%   \cs{@@_mul_npos_o:Nww} to do the work.  If the first operand is
%   \texttt{nan}, go to case $2$, in which the second operand is
%   discarded; if the second operand is \texttt{nan}, go to case $3$, in
%   which the first operand is discarded (note the weird interaction
%   with the final test on signs).  Then we separate the case where the
%   first number is normal and the second is zero: this goes to cases
%   $4$ and $5$ for multiplication, $10$ and $11$ for division.
%   Otherwise, we do a computation which
%   dispatches the products $0\times 0 = 0\times 1 = 1\times 0 = 0$ to
%   case $4$ or $5$ depending on the combined sign, the products
%   $0\times\infty$ and $\infty\times0$ to case $6$ or $7$ (invalid
%   operation), and the products $1\times\infty = \infty\times1 =
%   \infty\times\infty = \infty$ to cases $8$ and $9$.  Note that the
%   code for these two cases (which return $\pm\infty$) is inserted as
%   argument |#4|, because it differs in the case of divisions.
%    \begin{macrocode}
\cs_new:Npn \@@_mul_cases_o:NnNnww
    #1#2#3#4 \s_@@ \@@_chk:w #5#6#7\@@_sep: \s_@@ \@@_chk:w #8#9
  {
    \if_case:w \@@_int_eval:w
                 \if_int_compare:w #5 #8 = 11 ~
                   1
                 \else:
                   \if_meaning:w 3 #8
                     3
                   \else:
                     \if_meaning:w 3 #5
                       2
                     \else:
                       \if_int_compare:w #5 #8 = 10 ~
                         9 #2 - 2
                       \else:
                         (#5 #2 #8) / 2 * 2 + 7
                       \fi:
                     \fi:
                   \fi:
                 \fi:
                 \if_meaning:w #6 #9 - 1 \fi:
               \@@_int_eval_end:
         \@@_case_use:nw { #3 0 }
    \or: \@@_case_use:nw { #3 2 }
    \or: \@@_case_return_i_o:ww
    \or: \@@_case_return_ii_o:ww
    \or: \@@_case_return_o:Nww \c_zero_fp
    \or: \@@_case_return_o:Nww \c_minus_zero_fp
    \or: \@@_case_use:nw { \@@_invalid_operation_o:Nww #1 }
    \or: \@@_case_use:nw { \@@_invalid_operation_o:Nww #1 }
    \or: \@@_case_return_o:Nww \c_inf_fp
    \or: \@@_case_return_o:Nww \c_minus_inf_fp
    #4
    \fi:
    \s_@@ \@@_chk:w #5 #6 #7\@@_sep:
    \s_@@ \@@_chk:w #8 #9
  }
%    \end{macrocode}
% \end{macro}
%
% \subsubsection{Absolute multiplication}
%
% In this subsection, we perform the multiplication
% of two positive normal numbers.
%
% \begin{macro}[EXP]{\@@_mul_npos_o:Nww}
%   \begin{quote}
%     \cs{@@_mul_npos_o:Nww} \meta{final sign}
%     \cs{s_@@} \cs{@@_chk:w} |1| \meta{sign_1} \Arg{exp_1}  \meta{body_1} \cs{@@_sep:}
%     \cs{s_@@} \cs{@@_chk:w} |1| \meta{sign_2} \Arg{exp_2}  \meta{body_2} \cs{@@_sep:}
%   \end{quote}
%   After the computation, \cs{@@_sanitize:Nw} checks for overflow or
%   underflow.  As we did for addition, \cs{@@_int_eval:w} computes the
%   exponent, catching any shift coming from the computation in the
%   significand.  The \meta{final sign} is needed to do the rounding
%   properly in the significand computation.  We setup the post-expansion
%   here, triggered by \cs{@@_mul_significand_o:nnnnNnnnn}.
%
%   This is also used in \pkg{l3fp-convert}.
%    \begin{macrocode}
\cs_new:Npn \@@_mul_npos_o:Nww
    #1 \s_@@ \@@_chk:w #2 #3 #4 #5 \@@_sep:
       \s_@@ \@@_chk:w #6 #7 #8 #9 \@@_sep:
  {
    \exp_after:wN \@@_sanitize:Nw
    \exp_after:wN #1
    \int_value:w \@@_int_eval:w
      #4 + #8
      \@@_mul_significand_o:nnnnNnnnn #5 #1 #9
  }
%    \end{macrocode}
% \end{macro}
%
% \begin{macro}[rEXP]{\@@_mul_significand_o:nnnnNnnnn}
% \begin{macro}[EXP]
%   {\@@_mul_significand_drop:NNNNNw, \@@_mul_significand_keep:NNNNNw}
%   \begin{quote}
%     \cs{@@_mul_significand_o:nnnnNnnnn}
%       \Arg{X_1} \Arg{X_2} \Arg{X_3} \Arg{X_4} \meta{sign}
%       \Arg{Y_1} \Arg{Y_2} \Arg{Y_3} \Arg{Y_4}
%   \end{quote}
%   Note the three semicolons at the end of the definition.  One is for
%   the last \cs{@@_mul_significand_drop:NNNNNw}; one is for
%   \cs{@@_round_digit:Nw} later on; and one, preceded by
%   \cs{exp_after:wN}, which is correctly expanded (within an
%   \cs{@@_int_eval:w}), is used by \cs{@@_basics_pack_low:NNNNNw}.
%
%   The product of two $16$ digit integers has $31$ or $32$ digits,
%   but it is impossible to know which one before computing. The place
%   where we round depends on that number of digits, and may depend
%   on all digits until the last in some rare cases. The approach is
%   thus to compute the $5$ first blocks of $4$ digits (the first one
%   is between $100$ and $9999$ inclusive), and a compact version of
%   the remaining $3$ blocks. Afterwards, the number of digits is
%   known, and we can do the rounding within yet another set of
%   \cs{@@_int_eval:w}.
%    \begin{macrocode}
\cs_new:Npn \@@_mul_significand_o:nnnnNnnnn #1#2#3#4 #5 #6#7#8#9
  {
    \exp_after:wN \@@_mul_significand_test_f:NNN
    \exp_after:wN #5
    \int_value:w \@@_int_eval:w 99990000 + #1*#6 +
      \exp_after:wN \@@_mul_significand_keep:NNNNNw
      \int_value:w \@@_int_eval:w 99990000 + #1*#7 + #2*#6 +
        \exp_after:wN \@@_mul_significand_keep:NNNNNw
        \int_value:w \@@_int_eval:w 99990000 + #1*#8 + #2*#7 + #3*#6 +
          \exp_after:wN \@@_mul_significand_drop:NNNNNw
          \int_value:w \@@_int_eval:w 99990000 + #1*#9 + #2*#8 +
            #3*#7 + #4*#6 +
            \exp_after:wN \@@_mul_significand_drop:NNNNNw
            \int_value:w \@@_int_eval:w 99990000 + #2*#9 + #3*#8 +
              #4*#7 +
              \exp_after:wN \@@_mul_significand_drop:NNNNNw
              \int_value:w \@@_int_eval:w 99990000 + #3*#9 + #4*#8 +
                \exp_after:wN \@@_mul_significand_drop:NNNNNw
                \int_value:w \@@_int_eval:w 100000000 + #4*#9 \@@_sep:
    \@@_sep: \exp_after:wN \@@_sep:
  }
\cs_new:Npn \@@_mul_significand_drop:NNNNNw #1#2#3#4#5 #6\@@_sep:
  { #1#2#3#4#5 \@@_sep: + #6 }
\cs_new:Npn \@@_mul_significand_keep:NNNNNw #1#2#3#4#5 #6\@@_sep:
  { #1#2#3#4#5 \@@_sep: #6 \@@_sep: }
%    \end{macrocode}
% \end{macro}
% \end{macro}
%
% \begin{macro}[rEXP]{\@@_mul_significand_test_f:NNN}
%   \begin{quote}
%     \cs{@@_mul_significand_test_f:NNN} \meta{sign} |1|
%       \meta{digits 1--8} \cs{@@_sep:} \meta{digits 9--12} \cs{@@_sep:} \meta{digits 13--16} \cs{@@_sep:}
%       |+| \meta{digits 17--20} |+| \meta{digits 21--24}
%       |+| \meta{digits 25--28} |+| \meta{digits 29--32} \cs{@@_sep:}
%       \cs{exp_after:wN} \cs{@@_sep:}
%   \end{quote}
%   If the \meta{digit 1} is non-zero, then for rounding we only care
%   about the digits $16$ and $17$, and whether further digits are zero
%   or not (check for exact ties). On the other hand, if \meta{digit 1}
%   is zero, we care about digits $17$ and $18$, and whether further
%   digits are zero.
%    \begin{macrocode}
\cs_new:Npn \@@_mul_significand_test_f:NNN #1 #2 #3
  {
    \if_meaning:w 0 #3
      \exp_after:wN \@@_mul_significand_small_f:NNwwwN
    \else:
      \exp_after:wN \@@_mul_significand_large_f:NwwNNNN
    \fi:
    #1 #3
  }
%    \end{macrocode}
% \end{macro}
%
% \begin{macro}[EXP]{\@@_mul_significand_large_f:NwwNNNN}
%   In this branch, \meta{digit 1} is non-zero. The result is thus
%   \meta{digits 1--16}, plus some rounding which depends on the digits
%   $16$, $17$, and whether all subsequent digits are zero or not.
%   Here, \cs{@@_round_digit:Nw} takes digits $17$ and further (as an
%   integer expression), and replaces it by a \meta{rounding digit},
%   suitable for \cs{@@_round:NNN}.
%    \begin{macrocode}
\cs_new:Npn \@@_mul_significand_large_f:NwwNNNN
    #1 #2\@@_sep: #3\@@_sep: #4#5#6#7\@@_sep: +
  {
    \exp_after:wN \@@_basics_pack_high:NNNNNw
    \int_value:w \@@_int_eval:w 1#2
      \exp_after:wN \@@_basics_pack_low:NNNNNw
      \int_value:w \@@_int_eval:w 1#3#4#5#6#7
        + \exp_after:wN \@@_round:NNN
          \exp_after:wN #1
          \exp_after:wN #7
          \int_value:w \@@_round_digit:Nw
  }
%    \end{macrocode}
% \end{macro}
%
% \begin{macro}[rEXP]{\@@_mul_significand_small_f:NNwwwN}
%   In this branch, \meta{digit 1} is zero. Our result is thus
%   \meta{digits 2--17}, plus some rounding which depends on the digits
%   $17$, $18$, and whether all subsequent digits are zero or not.
%   The $8$ digits |1#3| are followed, after expansion of the
%   \texttt{small_pack} auxiliary, by the next digit, to form a $9$
%   digit number.
%    \begin{macrocode}
\cs_new:Npn \@@_mul_significand_small_f:NNwwwN
    #1 #2#3\@@_sep: #4#5\@@_sep: #6\@@_sep: + #7
  {
    - 1
    \exp_after:wN \@@_basics_pack_high:NNNNNw
    \int_value:w \@@_int_eval:w 1#3#4
      \exp_after:wN \@@_basics_pack_low:NNNNNw
      \int_value:w \@@_int_eval:w 1#5#6#7
        + \exp_after:wN \@@_round:NNN
          \exp_after:wN #1
          \exp_after:wN #7
          \int_value:w \@@_round_digit:Nw
  }
%    \end{macrocode}
% \end{macro}
%
% \subsection{Division}
%
% \subsubsection{Signs, and special numbers}
%
% Time is now ripe to tackle the hardest of the four elementary
% operations: division.
%
% \begin{macro}[EXP]{\@@_/_o:ww}
%   Filtering special floating point is very similar to what we did for
%   multiplications, with a few variations.  Invalid operation
%   exceptions display |/| rather than |*|.  In the formula for
%   dispatch, we replace |- 2 +| by |-|.  The case of normal
%   numbers is treated using \cs{@@_div_npos_o:Nww} rather than
%   \cs{@@_mul_npos_o:Nww}.  There are two additional cases: if the
%   first operand is normal and the second is a zero, then the division
%   by zero exception is raised: cases $10$ and $11$ of the
%   \cs{if_case:w} construction in \cs{@@_mul_cases_o:NnNnww} are
%   provided as the fourth argument here.
%    \begin{macrocode}
\cs_new:cpn { @@_/_o:ww }
  {
    \@@_mul_cases_o:NnNnww
      /
      { - }
      \@@_div_npos_o:Nww
      {
        \or:
          \@@_case_use:nw
            { \@@_division_by_zero_o:NNww \c_inf_fp / }
        \or:
          \@@_case_use:nw
            { \@@_division_by_zero_o:NNww \c_minus_inf_fp / }
      }
  }
%    \end{macrocode}
% \end{macro}
%
% \begin{macro}[EXP]{\@@_div_npos_o:Nww}
%   \begin{quote}
%     \cs{@@_div_npos_o:Nww} \meta{final sign}
%     \cs{s_@@} \cs{@@_chk:w} |1| \meta{sign_A} \Arg{exp A}
%       \Arg{A_1} \Arg{A_2} \Arg{A_3} \Arg{A_4} \cs{@@_sep:}
%     \cs{s_@@} \cs{@@_chk:w} |1| \meta{sign_Z} \Arg{exp Z}
%       \Arg{Z_1} \Arg{Z_2} \Arg{Z_3} \Arg{Z_4} \cs{@@_sep:}
%   \end{quote}
%   We want to compute $A/Z$.  As for multiplication,
%   \cs{@@_sanitize:Nw} checks for overflow or underflow; we provide it
%   with the \meta{final sign}, and an integer expression in which we
%   compute the exponent.  We set up the arguments of
%   \cs{@@_div_significand_i_o:wnnw}, namely an integer \meta{y} obtained
%   by adding $1$ to the first $5$ digits of $Z$ (explanation given soon
%   below), then the four \Arg{A_{i}}, then the four \Arg{Z_{i}}, a
%   semi-colon, and the \meta{final sign}, used for rounding at the end.
%    \begin{macrocode}
\cs_new:Npn \@@_div_npos_o:Nww
    #1 \s_@@ \@@_chk:w 1 #2 #3 #4 \@@_sep:
       \s_@@ \@@_chk:w 1 #5 #6 #7#8#9\@@_sep:
  {
    \exp_after:wN \@@_sanitize:Nw
    \exp_after:wN #1
    \int_value:w \@@_int_eval:w
      #3 - #6
      \exp_after:wN \@@_div_significand_i_o:wnnw
        \int_value:w \@@_int_eval:w #7 \use_i:nnnn #8 + 1 \@@_sep:
        #4
        {#7}{#8}#9 \@@_sep:
        #1
  }
%    \end{macrocode}
% \end{macro}
%
% \subsubsection{Work plan}
%
% In this subsection, we explain how to avoid overflowing \TeX{}'s
% integers when performing the division of two positive normal numbers.
%
% We are given two numbers, $A=0.A_{1}A_{2}A_{3}A_{4}$ and
% $Z=0.Z_{1}Z_{2}Z_{3}Z_{4}$, in blocks of $4$ digits, and we know that
% the first digits of $A_{1}$ and of $Z_{1}$ are non-zero. To compute
% $A/Z$, we proceed as follows.
% \begin{itemize}
%   \item Find an integer $Q_{A} \simeq 10^{4} A / Z$.
%   \item Replace $A$ by $B = 10^{4} A - Q_{A} Z$.
%   \item Find an integer $Q_{B} \simeq 10^{4} B / Z$.
%   \item Replace $B$ by $C = 10^{4} B - Q_{B} Z$.
%   \item Find an integer $Q_{C} \simeq 10^{4} C / Z$.
%   \item Replace $C$ by $D = 10^{4} C - Q_{C} Z$.
%   \item Find an integer $Q_{D} \simeq 10^{4} D / Z$.
%   \item Consider $E = 10^{4} D - Q_{D} Z$, and ensure
%     correct rounding.
% \end{itemize}
% The result is then $Q = 10^{-4} Q_{A} + 10^{-8} Q_{B} + 10^{-12} Q_{C}
% + 10^{-16} Q_{D} + \text{rounding}$.  Since the $Q_{i}$ are integers,
% $B$, $C$, $D$, and~$E$ are all exact multiples of $10^{-16}$, in other
% words, computing with $16$ digits after the decimal separator yields
% exact results.  The problem is the risk of overflow: in general $B$, $C$,
% $D$, and $E$ may be greater than $1$.
%
% Unfortunately, things are not as easy as they seem.  In particular, we
% want all intermediate steps to be positive, since negative results
% would require extra calculations at the end.  This requires that
% $Q_{A} \leq 10^{4} A / Z$ etc.  A reasonable attempt would be
% to define $Q_{A}$ as
% \begin{equation*}
%   \cs{int_eval:n} \left\{
%     \frac{ A_{1} A_{2} }{ Z_{1} + 1 } - 1 \right\}
%   \leq 10^{4} \frac{A}{Z}
% \end{equation*}
% Subtracting $1$ at the end takes care of the fact that \eTeX{}'s
% \cs{@@_int_eval:w} rounds divisions instead of truncating (really,
% $1/2$ would be sufficient, but we work with integers).  We add $1$ to
% $Z_{1}$ because $Z_{1} \leq 10^{4}Z < Z_{1}+1$ and we need $Q_{A}$ to
% be an underestimate.  However, we are now underestimating $Q_{A}$ too
% much: it can be wrong by up to $100$, for instance when $Z = 0.1$ and
% $A \simeq 1$.  Then $B$ could take values up to $10$ (maybe more), and
% a few steps down the line, we would run into arithmetic overflow,
% since \TeX{} can only handle integers less than roughly $2\cdot
% 10^{9}$.
%
% A better formula is to take
% \begin{equation*}
%   Q_{A} = \cs{int_eval:n} \left\{
%     \frac{ 10 \cdot A_{1} A_{2} }
%       { \left\lfloor 10^{-3} \cdot Z_{1} Z_{2} \right\rfloor + 1 }
%     - 1 \right\}.
% \end{equation*}
% This is always less than $10^{9} A / (10^{5} Z)$, as we wanted.  In
% words, we take the $5$ first digits of $Z$ into account, and the $8$
% first digits of $A$, using $0$ as a $9$-th digit rather than the true
% digit for efficiency reasons.  We shall prove that using this formula
% to define all the $Q_{i}$ avoids any overflow.  For convenience, let
% us denote
% \begin{equation*}
%   y = \left\lfloor 10^{-3} \cdot Z_{1} Z_{2} \right\rfloor + 1,
% \end{equation*}
% so that, taking into account the fact that \eTeX{} rounds ties away
% from zero,
% \begin{align*}
%   Q_{A}
%   &= \left\lfloor \frac{A_{1}A_{2}0}{y} - \frac{1}{2} \right\rfloor
%   \\
%   &>\frac{A_{1}A_{2}0}{y} - \frac{3}{2}.
% \end{align*}
% Note that $10^{4}<y\leq 10^{5}$, and $999 \leq Q_{A} \leq 99989$.
% Also note that this formula does not cause an overflow as long as $A <
% (2^{31}-1) / 10^{9} \simeq 2.147\cdots$, since the numerator involves an
% integer slightly smaller than $10^{9} A$.
%
% Let us bound $B$:
% \begin{align*}
%   10^{5} B
%   &=
%   A_{1}A_{2}0 + 10 \cdot 0.A_{3}A_{4}
%   - 10 \cdot Z_{1}.Z_{2}Z_{3}Z_{4} \cdot Q_{A}
%   \\
%   &<
%   A_{1}A_{2}0
%   \cdot \left( 1 - 10 \cdot \frac{Z_{1}.Z_{2}Z_{3}Z_{4}}{y} \right)
%   + \frac{3}{2} \cdot 10 \cdot Z_{1}.Z_{2}Z_{3}Z_{4} + 10
%   \\
%   &\leq
%   \frac{A_{1}A_{2}0 \cdot (y - 10 \cdot Z_{1}.Z_{2}Z_{3}Z_{4})}{y}
%   + \frac{3}{2} y + 10
%   \\
%   &\leq
%   \frac{A_{1}A_{2}0\cdot 1}{y} + \frac{3}{2} y + 10
%   \leq
%   \frac{10^{9} A}{y} + 1.6\cdot y.
% \end{align*}
% At the last step, we hide $10$ into the second term for later
% convenience.  The same reasoning yields
% \begin{align*}
%   10^{5} B &< 10^{9} A/y + 1.6 y, \\
%   10^{5} C &< 10^{9} B/y + 1.6 y, \\
%   10^{5} D &< 10^{9} C/y + 1.6 y, \\
%   10^{5} E &< 10^{9} D/y + 1.6 y. \\
% \end{align*}
% The goal is now to prove that none of $B$, $C$, $D$, and $E$ can go
% beyond $(2^{31}-1) / 10^{9} = 2.147\cdots$.
%
% Combining the various inequalities together with $A<1$, we get
% \begin{align*}
%   10^{5} B &< 10^{9}/y + 1.6 y, \\
%   10^{5} C &< 10^{13}/y^{2} + 1.6 (y + 10^{4}), \\
%   10^{5} D &< 10^{17}/y^{3} + 1.6 (y + 10^{4} + 10^{8}/y), \\
%   10^{5} E &< 10^{21}/y^{4} + 1.6 (y + 10^{4} + 10^{8}/y + 10^{12}/y^{2}). \\
% \end{align*}
% All of those bounds are convex functions of $y$ (since every power of
% $y$ involved is convex, and the coefficients are positive), and thus
% maximal at one of the end-points of the allowed range $10^{4} < y \leq
% 10^{5}$.  Thus,
% \begin{align*}
%   10^{5} B &< \mathrm{max} ( 1.16\cdot 10^{5}, 1.7 \cdot 10^{5}), \\
%   10^{5} C &< \mathrm{max} ( 1.32\cdot 10^{5}, 1.77 \cdot 10^{5}), \\
%   10^{5} D &< \mathrm{max} ( 1.48\cdot 10^{5}, 1.777 \cdot 10^{5}), \\
%   10^{5} E &< \mathrm{max} ( 1.64\cdot 10^{5}, 1.7777 \cdot 10^{5}). \\
% \end{align*}
% All of those bounds are less than $2.147\cdot 10^{5}$, and we are thus
% within \TeX{}'s bounds in all cases!
%
% We later need to have a bound on the $Q_{i}$. Their definitions
% imply that $Q_{A} < 10^{9} A/y - 1/2 < 10^{5} A$ and similarly for the
% other $Q_{i}$.  Thus, all of them are less than $177770$.
%
% The last step is to ensure correct rounding. We have
% \begin{equation*}
%   A/Z = \sum_{i=1}^{4} \left(10^{-4i} Q_{i}\right) + 10^{-16} E/Z
% \end{equation*}
% exactly.  Furthermore, we know that the result is in $[0.1,10)$,
% hence will be rounded to a multiple of $10^{-16}$ or of $10^{-15}$, so
% we only need to know the integer part of $E/Z$, and a
% \enquote{rounding} digit encoding the rest.  Equivalently, we need to
% find the integer part of $2E/Z$, and determine whether it was an
% exact integer or not (this serves to detect ties).  Since
% \begin{equation*}
%   \frac{2E}{Z} = 2\frac{10^{5} E}{10^{5} Z}
%   \leq 2\frac{10^{5} E}{10^{4}} < 36,
% \end{equation*}
% this integer part is between $0$ and $35$ inclusive. We let \eTeX{}
% round
% \begin{equation*}
%   P = \cs{int_eval:n} \left\{
%     \frac{2\cdot E_{1}E_{2}}{Z_{1}Z_{2}} \right\},
% \end{equation*}
% which differs from $2E/Z$ by at most
% \begin{equation*}
%   \frac{1}{2}
%   + 2 \left\lvert \frac{E}{Z} - \frac{E}{10^{-8} Z_{1}Z_{2}}\right\rvert
%   + 2 \left\lvert \frac{10^{8} E - E_{1}E_{2}}{Z_{1}Z_{2}}\right\rvert
%   < 1,
% \end{equation*}
% ($1/2$ comes from \eTeX{}'s rounding) because each absolute value is
% less than $10^{-7}$.  Thus $P$ is either the correct integer part, or
% is off by $1$; furthermore, if $2 E / Z$ is an integer, $P = 2 E / Z$.
% We will check the sign of $2 E - P Z$.  If it is negative, then $E / Z
% \in \big((P - 1) / 2, P / 2\big)$.  If it is zero, then $E / Z = P /
% 2$.  If it is positive, then $E / Z \in \big(P / 2, (P - 1) / 2\big)$.
% In each case, we know how to round to an integer, depending on the
% parity of $P$, and the rounding mode.
%
% \subsubsection{Implementing the significand division}
%
% \begin{macro}[rEXP]{\@@_div_significand_i_o:wnnw}
%   \begin{quote}
%     \cs{@@_div_significand_i_o:wnnw} \meta{y} \cs{@@_sep:}
%       \Arg{A_1} \Arg{A_2} \Arg{A_3} \Arg{A_4}
%       \Arg{Z_1} \Arg{Z_2} \Arg{Z_3} \Arg{Z_4} \cs{@@_sep:} \meta{sign}
%   \end{quote}
%   Compute $10^{6} + Q_{A}$ (a $7$~digit number thanks to the shift),
%   unbrace \meta{A_1} and \meta{A_2}, and prepare the
%   \meta{continuation} arguments for $4$ consecutive calls to
%   \cs{@@_div_significand_calc:wwnnnnnnn}.  Each of these calls needs
%   \meta{y} (|#1|), and it turns out that we need post-expansion there,
%   hence the \cs{int_value:w}.  Here, |#4| is six brace groups, which
%   give the six first |n|-type arguments of the \texttt{calc} function.
%    \begin{macrocode}
\cs_new:Npn \@@_div_significand_i_o:wnnw #1 \@@_sep: #2#3 #4 \@@_sep:
  {
    \exp_after:wN \@@_div_significand_test_o:w
    \int_value:w \@@_int_eval:w
      \exp_after:wN \@@_div_significand_calc:wwnnnnnnn
      \int_value:w \@@_int_eval:w 999999 + #2 #3 0 / #1 \@@_sep:
        #2 #3 \@@_sep:
        #4
        { \exp_after:wN \@@_div_significand_ii:wwn \int_value:w #1 }
        { \exp_after:wN \@@_div_significand_ii:wwn \int_value:w #1 }
        { \exp_after:wN \@@_div_significand_ii:wwn \int_value:w #1 }
        { \exp_after:wN \@@_div_significand_iii:wwnnnnn \int_value:w #1 }
  }
%    \end{macrocode}
% \end{macro}
%
% \begin{macro}[rEXP]{\@@_div_significand_calc:wwnnnnnnn}
% \begin{macro}[rEXP]
%   {
%     \@@_div_significand_calc_i:wwnnnnnnn,
%     \@@_div_significand_calc_ii:wwnnnnnnn,
%   }
%   \begin{quote}
%     \cs{@@_div_significand_calc:wwnnnnnnn} \meta{$10^{6}+{}$Q_{A}} \cs{@@_sep:}
%       \meta{A_1} \meta{A_2} \cs{@@_sep:} \Arg{A_3} \Arg{A_4}
%       \Arg{Z_1} \Arg{Z_2} \Arg{Z_3} \Arg{Z_4}
%       \Arg{continuation}
%   \end{quote}
%   expands to
%   \begin{quote}
%     \meta{$10^{6}+{}$Q_{A}} \meta{continuation} \cs{@@_sep:}
%       \meta{B_1} \meta{B_2} \cs{@@_sep:} \Arg{B_3} \Arg{B_4}
%       \Arg{Z_1} \Arg{Z_2} \Arg{Z_3} \Arg{Z_4}
%   \end{quote}
%   where $B = 10^{4} A - Q_{A} \cdot Z$.  This function is also used to
%   compute $C$, $D$, $E$ (with the input shifted accordingly), and is
%   used in \pkg{l3fp-expo}.
%
%   We know that $0<Q_{A}<1.8\cdot 10^{5}$, so the product of $Q_{A}$
%   with each $Z_{i}$ is within \TeX{}'s bounds.  However, it is a
%   little bit too large for our purposes: we would not be able to use
%   the usual trick of adding a large power of $10$ to ensure that the
%   number of digits is fixed.
%
%   The bound on $Q_{A}$, implies that $10^{6}+Q_{A}$ starts with the
%   digit $1$, followed by $0$ or $1$.  We test, and call different
%   auxiliaries for the two cases.  An earlier implementation did the
%   tests within the computation, but since we added a
%   \meta{continuation}, this is not possible because the macro has $9$
%   parameters.
%
%   The result we want is then (the overall power of $10$ is arbitrary):
%   \begin{align*}
%     &10^{-4} ( \#2 - \#1 \cdot \#5 - 10 \cdot \meta{i} \cdot \#5\#6 )
%     + 10^{-8} ( \#3 - \#1 \cdot \#6 - 10 \cdot \meta{i} \cdot \#7 ) \\
%     &+ 10^{-12}( \#4 - \#1 \cdot \#7 - 10 \cdot \meta{i} \cdot \#8 )
%     + 10^{-16}(     - \#1 \cdot \#8 ),
%   \end{align*}
%   where \meta{i} stands for the $10^{5}$ digit of $Q_{A}$, which is
%   $0$ or~$1$, and $\#1$, $\#2$, \emph{etc.\@} are the parameters of
%   either auxiliary.  The factors of $10$ come from the fact that
%   $Q_{A} = 10\cdot 10^{4} \cdot \meta{i} + \#1$.  As usual, to combine
%   all the terms, we need to choose some shifts which must ensure that
%   the number of digits of the second, third, and fourth terms are each
%   fixed.  Here, the positive contributions are at most $10^{8}$ and
%   the negative contributions can go up to $10^{9}$.  Indeed, for the
%   auxiliary with $\meta{i}=1$, |#1| is at most $80000$, leading to
%   contributions of at worse $-8\cdot 10^{8}4$, while the other
%   negative term is very small $<10^{6}$ (except in the first
%   expression, where we don't care about the number of digits); for the
%   auxiliary with $\meta{i}=0$, |#1| can go up to $99999$, but there is
%   no other negative term.  Hence, a good choice is $2\cdot 10^{9}$,
%   which produces totals in the range $[10^{9}, 2.1\cdot 10^{9}]$.  We
%   are flirting with \TeX{}'s limits once more.
%    \begin{macrocode}
\cs_new:Npn \@@_div_significand_calc:wwnnnnnnn 1#1
  {
    \if_meaning:w 1 #1
      \exp_after:wN \@@_div_significand_calc_i:wwnnnnnnn
    \else:
      \exp_after:wN \@@_div_significand_calc_ii:wwnnnnnnn
    \fi:
  }
\cs_new:Npn \@@_div_significand_calc_i:wwnnnnnnn
  #1\@@_sep: #2\@@_sep:#3#4 #5#6#7#8 #9
  {
    1 1 #1
    #9 \exp_after:wN \@@_sep:
    \int_value:w \@@_int_eval:w \c_@@_Bigg_leading_shift_int
      + #2 - #1 * #5 - #5#60
      \exp_after:wN \@@_pack_Bigg:NNNNNNw
      \int_value:w \@@_int_eval:w \c_@@_Bigg_middle_shift_int
        + #3 - #1 * #6 - #70
        \exp_after:wN \@@_pack_Bigg:NNNNNNw
        \int_value:w \@@_int_eval:w \c_@@_Bigg_middle_shift_int
          + #4 - #1 * #7 - #80
          \exp_after:wN \@@_pack_Bigg:NNNNNNw
          \int_value:w \@@_int_eval:w \c_@@_Bigg_trailing_shift_int
            - #1 * #8 \@@_sep:
    {#5}{#6}{#7}{#8}
  }
\cs_new:Npn \@@_div_significand_calc_ii:wwnnnnnnn
  #1\@@_sep: #2\@@_sep:#3#4 #5#6#7#8 #9
  {
    1 0 #1
    #9 \exp_after:wN \@@_sep:
    \int_value:w \@@_int_eval:w \c_@@_Bigg_leading_shift_int
      + #2 - #1 * #5
      \exp_after:wN \@@_pack_Bigg:NNNNNNw
      \int_value:w \@@_int_eval:w \c_@@_Bigg_middle_shift_int
        + #3 - #1 * #6
        \exp_after:wN \@@_pack_Bigg:NNNNNNw
        \int_value:w \@@_int_eval:w \c_@@_Bigg_middle_shift_int
          + #4 - #1 * #7
          \exp_after:wN \@@_pack_Bigg:NNNNNNw
          \int_value:w \@@_int_eval:w \c_@@_Bigg_trailing_shift_int
            - #1 * #8 \@@_sep:
    {#5}{#6}{#7}{#8}
  }
%    \end{macrocode}
% \end{macro}
% \end{macro}
%
% \begin{macro}[EXP]{\@@_div_significand_ii:wwn}
%   \begin{quote}
%     \cs{@@_div_significand_ii:wwn} \meta{y} \cs{@@_sep:}
%       \meta{B_1} \cs{@@_sep:} \Arg{B_2} \Arg{B_3} \Arg{B_4}
%       \Arg{Z_1} \Arg{Z_2} \Arg{Z_3} \Arg{Z_4}
%       \meta{continuations} \meta{sign}
%   \end{quote}
%   Compute $Q_{B}$ by evaluating $\meta{B_1}\meta{B_2}0 / y - 1$.  The
%   result is output to the left, in an \cs{@@_int_eval:w} which we
%   start now.  Once that is evaluated (and the other $Q_{i}$ also,
%   since later expansions are triggered by this one), a packing
%   auxiliary takes care of placing the digits of $Q_{B}$ in an
%   appropriate way for the final addition to obtain $Q$.  This
%   auxiliary is also used to compute $Q_{C}$ and $Q_{D}$ with the
%   inputs $C$ and $D$ instead of $B$.
%    \begin{macrocode}
\cs_new:Npn \@@_div_significand_ii:wwn #1\@@_sep: #2\@@_sep:#3
  {
    \exp_after:wN \@@_div_significand_pack:NNN
    \int_value:w \@@_int_eval:w
      \exp_after:wN \@@_div_significand_calc:wwnnnnnnn
      \int_value:w \@@_int_eval:w
        999999 + #2 #3 0 / #1 \@@_sep: #2 #3 \@@_sep:
  }
%    \end{macrocode}
% \end{macro}
%
% \begin{macro}[rEXP]{\@@_div_significand_iii:wwnnnnn}
%   \begin{quote}
%     \cs{@@_div_significand_iii:wwnnnnn} \meta{y} \cs{@@_sep:}
%       \meta{E_1} \cs{@@_sep:} \Arg{E_2} \Arg{E_3} \Arg{E_4}
%       \Arg{Z_1} \Arg{Z_2} \Arg{Z_3} \Arg{Z_4} \meta{sign}
%   \end{quote}
%   We compute $P \simeq 2E/Z$ by rounding $2 E_{1} E_{2}/Z_{1}Z_{2}$.
%   Note the first $0$, which multiplies $Q_{D}$ by $10$: we later
%   add (roughly) $5\cdot P$, which amounts to adding $P/2 \simeq E/Z$
%   to $Q_{D}$, the appropriate correction from a hypothetical $Q_{E}$.
%    \begin{macrocode}
\cs_new:Npn \@@_div_significand_iii:wwnnnnn #1\@@_sep: #2\@@_sep:#3#4#5 #6#7
  {
    0
    \exp_after:wN \@@_div_significand_iv:wwnnnnnnn
    \int_value:w \@@_int_eval:w ( 2 * #2 #3) / #6 #7 \@@_sep: % <- P
      #2 \@@_sep: {#3} {#4} {#5}
      {#6} {#7}
  }
%    \end{macrocode}
% \end{macro}
%
% \begin{macro}[rEXP]
%   {
%     \@@_div_significand_iv:wwnnnnnnn,
%     \@@_div_significand_v:NNw,
%     \@@_div_significand_vi:Nw
%   }
%   \begin{quote}
%     \cs{@@_div_significand_iv:wwnnnnnnn} \meta{P} \cs{@@_sep:}
%       \meta{E_1} \cs{@@_sep:} \Arg{E_2} \Arg{E_3} \Arg{E_4}
%       \Arg{Z_1} \Arg{Z_2} \Arg{Z_3} \Arg{Z_4} \meta{sign}
%   \end{quote}
%   This adds to the current expression ($10^{7} + 10\cdot Q_{D}$) a
%   contribution of $5 \cdot P + \operatorname{sign}(T)$ with $T = 2 E -
%   P Z$.  This amounts to adding $P / 2$ to $Q_{D}$, with an extra
%   \meta{rounding} digit.  This \meta{rounding} digit is $0$ or $5$ if
%   $T$ does not contribute, \emph{i.e.,  if $0 = T = 2 E - P Z$, in
%   other words if $10^{16} A / Z$ is an integer or half-integer.
%   Otherwise it is in the appropriate range, $[1,4]$ or $[6,9]$.  This
%   is precise enough for rounding purposes (in any mode).
%
%   It seems an overkill to compute $T$ exactly as I do here, but I see
%   no faster way right now.
%
%   Once more, we need to be careful and show that the calculation
%   $\#1\cdot\#6\#7$ below does not cause an overflow: naively, $P$ can
%   be up to $35$, and $\#6\#7$ up to $10^{8}$, but both cannot happen
%   simultaneously.  To show that things are fine, we split in two
%   (non-disjoint) cases.
%   \begin{itemize}
%   \item For $P < 10$, the product obeys $P\cdot\#6\#7 < 10^{8} \cdot P
%     < 10^{9} $.
%   \item For large $P\geq 3$, the rounding error on $P$, which is at
%     most $1$, is less than a factor of $2$, hence $P\leq 4E/Z$.  Also,
%     $\#6\#7 \leq 10^{8} \cdot Z$, hence $P\cdot \#6\#7 \leq 4E\cdot
%     10^{8} < 10^{9}$.
%   \end{itemize}
%   Both inequalities could be made tighter if needed.
%
%   Note however that $P\cdot \#8\#9$ may overflow, since the two
%   factors are now independent, and the result may reach $3.5\cdot
%   10^{9}$.  Thus we compute the two lower levels separately.  The rest
%   is standard, except that we use |+| as a separator (ending integer
%   expressions explicitly).  $T$ is negative if the first character is
%   |-|, it is positive if the first character is neither |0| nor |-|.
%   It is also positive if the first character is |0| and second
%   argument of \cs{@@_div_significand_vi:Nw}, a sum of several terms, is
%   also zero.  Otherwise, there was an exact agreement: $T = 0$.
%    \begin{macrocode}
\cs_new:Npn \@@_div_significand_iv:wwnnnnnnn
    #1\@@_sep: #2\@@_sep:#3#4#5 #6#7#8#9
  {
    + 5 * #1
    \exp_after:wN \@@_div_significand_vi:Nw
    \int_value:w \@@_int_eval:w -50 + 2*#2#3 - #1*#6#7 +
      \exp_after:wN \@@_div_significand_v:NN
      \int_value:w \@@_int_eval:w 499950 + 2*#4 - #1*#8 +
        \exp_after:wN \@@_div_significand_v:NN
        \int_value:w \@@_int_eval:w 500000 + 2*#5 - #1*#9 \@@_sep:
  }
\cs_new:Npn \@@_div_significand_v:NN #1#2 { #1#2 \@@_int_eval_end: + }
\cs_new:Npn \@@_div_significand_vi:Nw #1#2\@@_sep:
  {
    \if_meaning:w 0 #1
      \if_int_compare:w \@@_int_eval:w #2 > 0 + 1 \fi:
    \else:
      \if_meaning:w - #1 - \else: + \fi: 1
    \fi:
    \@@_sep:
  }
%    \end{macrocode}
% \end{macro}
%
% \begin{macro}[EXP]{\@@_div_significand_pack:NNN}
%   At this stage, we are in the following situation: \TeX{} is in the
%   process of expanding several integer expressions, thus functions at
%   the bottom expand before those above.
%   \begin{quote}
%     \cs{@@_div_significand_test_o:w} $10^{6} + Q_{A}$
%     \cs{@@_div_significand_pack:NNN} $10^{6} + Q_{B}$
%     \cs{@@_div_significand_pack:NNN} $10^{6} + Q_{C}$
%     \cs{@@_div_significand_pack:NNN}
%     $10^{7} + 10\cdot Q_{D} + 5 \cdot P + \varepsilon$ \cs{@@_sep:} \meta{sign}
%   \end{quote}
%   Here, $\varepsilon = \operatorname{sign}(T)$ is $0$ in case $2E=PZ$,
%   $1$ in case $2E>PZ$, which means that $P$ was the correct value, but
%   not with an exact quotient, and $-1$ if $2E<PZ$, i.e., $P$
%   was an overestimate.  The packing function we define now does
%   nothing special: it removes the $10^{6}$ and carries two digits (for
%   the $10^{5}$'s and the $10^{4}$'s).
%    \begin{macrocode}
\cs_new:Npn \@@_div_significand_pack:NNN 1 #1 #2 { + #1 #2 \@@_sep: }
%    \end{macrocode}
% \end{macro}
%
% \begin{macro}[rEXP]{\@@_div_significand_test_o:w}
%   \begin{quote}
%     \cs{@@_div_significand_test_o:w} |1| |0| \meta{5d} \cs{@@_sep:}
%     ~~\meta{4d} \cs{@@_sep:} \meta{4d} \cs{@@_sep:} \meta{5d} \cs{@@_sep:} \meta{sign}
%   \end{quote}
%   The reason we know that the first two digits are |1| and |0| is that
%   the final result is known to be between $0.1$ (inclusive) and $10$,
%   hence $\widetilde{Q_{A}}$ (the tilde denoting the contribution from
%   the other $Q_{i}$) is at most $99999$, and $10^{6}+\widetilde{Q_{A}}
%   = 10\cdots$.
%
%   It is now time to round. This depends on how many digits the final
%   result will have.
%    \begin{macrocode}
\cs_new:Npn \@@_div_significand_test_o:w 10 #1
  {
    \if_meaning:w 0 #1
      \exp_after:wN \@@_div_significand_small_o:wwwNNNNwN
    \else:
      \exp_after:wN \@@_div_significand_large_o:wwwNNNNwN
    \fi:
    #1
  }
%    \end{macrocode}
% \end{macro}
%
% \begin{macro}[EXP]{\@@_div_significand_small_o:wwwNNNNwN}
%   \begin{quote}
%     \cs{@@_div_significand_small_o:wwwNNNNwN} |0| \meta{4d} \cs{@@_sep:}
%     ~~\meta{4d} \cs{@@_sep:} \meta{4d} \cs{@@_sep:} \meta{5d} \cs{@@_sep:} \meta{final sign}
%   \end{quote}
%   Standard use of the functions \cs{@@_basics_pack_low:NNNNNw} and
%   \cs{@@_basics_pack_high:NNNNNw}.  We finally get to use the
%   \meta{final sign} which has been sitting there for a while.
%    \begin{macrocode}
\cs_new:Npn \@@_div_significand_small_o:wwwNNNNwN
    0 #1\@@_sep: #2\@@_sep: #3\@@_sep: #4#5#6#7#8\@@_sep: #9
  {
    \exp_after:wN \@@_basics_pack_high:NNNNNw
    \int_value:w \@@_int_eval:w 1 #1#2
      \exp_after:wN \@@_basics_pack_low:NNNNNw
      \int_value:w \@@_int_eval:w 1 #3#4#5#6#7
        + \@@_round:NNN #9 #7 #8
        \exp_after:wN \@@_sep:
  }
%    \end{macrocode}
% \end{macro}
%
% \begin{macro}[rEXP]{\@@_div_significand_large_o:wwwNNNNwN}
%   \begin{quote}
%     \cs{@@_div_significand_large_o:wwwNNNNwN} \meta{5d} \cs{@@_sep:}
%     ~~\meta{4d} \cs{@@_sep:} \meta{4d} \cs{@@_sep:} \meta{5d} \cs{@@_sep:} \meta{sign}
%   \end{quote}
%   We know that the final result cannot reach $10$, hence |1#1#2|,
%   together with contributions from the level below, cannot reach
%   $2\cdot 10^{9}$.  For rounding, we build the \meta{rounding digit}
%   from the last two of our $18$ digits.
%    \begin{macrocode}
\cs_new:Npn \@@_div_significand_large_o:wwwNNNNwN
    #1\@@_sep: #2\@@_sep: #3\@@_sep: #4#5#6#7#8\@@_sep: #9
  {
    + 1
    \exp_after:wN \@@_basics_pack_weird_high:NNNNNNNNw
    \int_value:w \@@_int_eval:w 1 #1 #2
      \exp_after:wN \@@_basics_pack_weird_low:NNNNw
      \int_value:w \@@_int_eval:w 1 #3 #4 #5 #6 +
        \exp_after:wN \@@_round:NNN
        \exp_after:wN #9
        \exp_after:wN #6
        \int_value:w \@@_round_digit:Nw #7 #8 \@@_sep:
      \exp_after:wN \@@_sep:
  }
%    \end{macrocode}
% \end{macro}
%
% \subsection{Square root}
%
% \begin{macro}[EXP]{\@@_sqrt_o:w}
%   Zeros are unchanged: $\sqrt{-0} = -0$ and $\sqrt{+0} = +0$.
%   Negative numbers (other than $-0$) have no real square root.
%   Positive infinity, and \texttt{nan}, are unchanged.  Finally, for
%   normal positive numbers, there is some work to do.
%    \begin{macrocode}
\cs_new:Npn \@@_sqrt_o:w #1 \s_@@ \@@_chk:w #2#3#4\@@_sep: @
  {
    \if_meaning:w 0 #2 \@@_case_return_same_o:w \fi:
    \if_meaning:w 2 #3
      \@@_case_use:nw { \@@_invalid_operation_o:nw { sqrt } }
    \fi:
    \if_meaning:w 1 #2 \else: \@@_case_return_same_o:w \fi:
    \@@_sqrt_npos_o:w
    \s_@@ \@@_chk:w #2 #3 #4\@@_sep:
  }
%    \end{macrocode}
% \end{macro}
%
% \begin{macro}[EXP]{\@@_sqrt_npos_o:w}
% \begin{macro}[rEXP]
%   {\@@_sqrt_npos_auxi_o:wwnnN, \@@_sqrt_npos_auxii_o:wNNNNNNNN}
%   Prepare \cs{@@_sanitize:Nw} to receive the final sign~|0| (the
%   result is always positive) and the exponent, equal to half of the
%   exponent~|#1| of the argument.  If the exponent~|#1| is even, find a
%   first approximation of the square root of the significand $10^{8}
%   a_1 + a_2 = 10^{8} |#2#3| + |#4#5|$ through Newton's method,
%   starting at $x = 57234133 \simeq 10^{7.75}$.  Otherwise, first shift
%   the significand of the argument by one digit, getting
%   $a_1'\in[10^{6}, 10^{7})$ instead of $[10^{7}, 10^{8})$, then use
%   Newton's method starting at $17782794 \simeq 10^{7.25}$.
%    \begin{macrocode}
\cs_new:Npn \@@_sqrt_npos_o:w \s_@@ \@@_chk:w 1 0 #1#2#3#4#5\@@_sep:
  {
    \exp_after:wN \@@_sanitize:Nw
    \exp_after:wN 0
    \int_value:w \@@_int_eval:w
      \if_int_odd:w #1 \exp_stop_f:
        \exp_after:wN \@@_sqrt_npos_auxi_o:wwnnN
      \fi:
      #1 / 2
      \@@_sqrt_Newton_o:wwn 56234133\@@_sep: 0\@@_sep: {#2#3} {#4#5} 0
  }
\cs_new:Npn \@@_sqrt_npos_auxi_o:wwnnN #1 / 2 #2\@@_sep: 0\@@_sep: #3#4#5
  {
    ( #1 + 1 ) / 2
    \@@_pack_eight:wNNNNNNNN
    \@@_sqrt_npos_auxii_o:wNNNNNNNN
    \@@_sep:
    0 #3 #4
  }
\cs_new:Npn \@@_sqrt_npos_auxii_o:wNNNNNNNN #1\@@_sep: #2#3#4#5#6#7#8#9
  { \@@_sqrt_Newton_o:wwn 17782794\@@_sep: 0\@@_sep: {#1} {#2#3#4#5#6#7#8#9} }
%    \end{macrocode}
% \end{macro}
% \end{macro}
%
% \begin{macro}[rEXP]{\@@_sqrt_Newton_o:wwn}
%   Newton's method maps $x\mapsto\bigl[(x + [10^{8} a_1 / x])/2\bigr]$
%   in each iteration, where $[b/c]$ denotes \eTeX{}'s division.  This
%   division rounds the real number $b/c$ to the closest integer,
%   rounding ties away from zero, hence when $c$~is even,
%   $b/c - 1/2 + 1/c \leq [b/c] \leq b/c + 1/2$
%   and when $c$~is odd,
%   $b/c - 1/2 + 1/(2c) \leq [b/c] \leq b/c + 1/2 - 1/(2c)$.
%   For all~$c$, $b/c - 1/2 + 1/(2c) \leq [b/c] \leq b/c + 1/2$.
%
%   Let us prove that the method converges when implemented with \eTeX{}
%   integer division, for any $10^{6} \leq a_1 < 10^{8}$ and starting
%   value $10^{6} \leq x < 10^{8}$.  Using the inequalities above and
%   the arithmetic--geometric inequality $(x+t)/2 \geq \sqrt{xt}$ for $t
%   = 10^{8} a_1 / x$, we find
%   \[
%   x'
%   = \left[\frac{x + [10^{8} a_1 / x]}{2}\right]
%   \geq \frac{x + 10^{8} a_1 / x - 1/2 + 1/(2x)}{2}
%   \geq \sqrt{10^{8} a_1} - \frac{1}{4} + \frac{1}{4x} \,.
%   \]
%   After any step of iteration, we thus have $\delta = x - \sqrt{10^{8}
%     a_1} \geq -0.25 + 0.25 \cdot 10^{-8}$.  The new difference
%   $\delta' = x' - \sqrt{10^{8} a_1}$ after one step is bounded above
%   as
%   \[
%   x' - \sqrt{10^{8} a_1}
%   \leq \frac{x + 10^{8} a_1 / x + 1/2}{2} + \frac{1}{2}
%   - \sqrt{10^{8} a_1}
%   \leq \frac{\delta}{2} \frac{\delta}{\sqrt{10^{8} a_1} + \delta}
%   + \frac{3}{4} \,.
%   \]
%   For $\delta > 3/2$, this last expression is
%   $\leq\delta/2+3/4<\delta$, hence $\delta$~decreases at each step:
%   since all~$x$ are integers, $\delta$~must reach a value
%   $-1/4<\delta\leq 3/2$.  In this range of values, we get $\delta'
%   \leq \frac{3}{4} \frac{3}{2\sqrt{10^{8} a_1}} + \frac{3}{4} \leq
%   0.75 + 1.125 \cdot 10^{-7}$.  We deduce that the difference $\delta
%   = x - \sqrt{10^{8} a_1}$ eventually reaches a value in the interval
%   $[-0.25 + 0.25\cdot 10^{-8}, 0.75 + 11.25 \cdot 10^{-8}]$, whose
%   width is $1 + 11 \cdot 10^{-8}$.  The corresponding interval for~$x$
%   may contain two integers, hence $x$~might oscillate between those
%   two values.
%
%   However, the fact that $x\mapsto x-1$ and $x-1 \mapsto x$ puts
%   stronger constraints, which are not compatible: the first implies
%   \[
%   x + [10^{8} a_1 / x] \leq 2x - 2
%   \]
%   hence $10^{8} a_1 / x \leq x - 3/2$, while the second implies
%   \[
%   x - 1 + [10^{8} a_1 / (x - 1)] \geq 2x - 1
%   \]
%   hence $10^{8} a_1 / (x - 1) \geq x - 1/2$.  Combining the two
%   inequalities yields $x^2 - 3x/2 \geq 10^{8} a_1 \geq x - 3x/2 +
%   1/2$, which cannot hold.  Therefore, the iteration always converges
%   to a single integer~$x$.  To stop the iteration when two consecutive
%   results are equal, the function \cs{@@_sqrt_Newton_o:wwn} receives
%   the newly computed result as~|#1|, the previous result as~|#2|, and
%   $a_1$ as~|#3|.  Note that \eTeX{} combines the computation of a
%   multiplication and a following division, thus avoiding overflow in
%   |#3 * 100000000 / #1|.  In any case, the result is within $[10^{7},
%   10^{8}]$.
%    \begin{macrocode}
\cs_new:Npn \@@_sqrt_Newton_o:wwn #1\@@_sep: #2\@@_sep: #3
  {
    \if_int_compare:w #1 = #2 \exp_stop_f:
      \exp_after:wN \@@_sqrt_auxi_o:NNNNwnnN
      \int_value:w \@@_int_eval:w 9999 9999 +
        \exp_after:wN \@@_use_none_until_s:w
    \fi:
    \exp_after:wN \@@_sqrt_Newton_o:wwn
    \int_value:w \@@_int_eval:w (#1 + #3 * 1 0000 0000 / #1) / 2 \@@_sep:
    #1\@@_sep: {#3}
  }
%    \end{macrocode}
% \end{macro}
%
% \begin{macro}[rEXP]{\@@_sqrt_auxi_o:NNNNwnnN}
%   This function is followed by $10^{8}+x-1$, which has~$9$ digits
%   starting with~$1$, then \cs{@@_sep:} \Arg{a_1} \Arg{a_2} \meta{a'}.  Here,
%   $x \simeq \sqrt{10^{8} a_1}$ and we want to estimate the square root of
%   $a = 10^{-8} a_1 + 10^{-16} a_2 + 10^{-17} a'$.  We set up an
%   initial underestimate
%   \[
%   y = (x - 1) 10^{-8} + 0.2499998875 \cdot 10^{-8} \lesssim \sqrt{a}\,.
%   \]
%   From the inequalities shown earlier, we know that $y \leq
%   \sqrt{10^{-8} a_1} \leq \sqrt{a}$ and that $\sqrt{10^{-8} a_1} \leq
%   y + 10^{-8} + 11\cdot 10^{-16}$ hence (using $0.1\leq y\leq
%   \sqrt{a}\leq 1$)
%   \[
%   a - y^2 \leq 10^{-8} a_1 + 10^{-8} - y^2
%   \leq (y + 10^{-8} + 11\cdot 10^{-16})^2 - y^2 + 10^{-8}
%   < 3.2 \cdot 10^{-8} \,,
%   \]
%   and $\sqrt{a} - y = (a - y^2)/(\sqrt{a} + y) \leq 16 \cdot 10^{-8}$.
%   Next, \cs{@@_sqrt_auxii_o:NnnnnnnnN} is called several times to
%   get closer and closer underestimates of~$\sqrt{a}$.  By
%   construction, the underestimates~$y$ are always increasing, $a - y^2
%   < 3.2 \cdot 10^{-8}$ for all.  Also, $y<1$.
%    \begin{macrocode}
\cs_new:Npn \@@_sqrt_auxi_o:NNNNwnnN 1 #1#2#3#4#5\@@_sep:
  {
    \@@_sqrt_auxii_o:NnnnnnnnN
      \@@_sqrt_auxiii_o:wnnnnnnnn
      {#1#2#3#4} {#5} {2499} {9988} {7500}
  }
%    \end{macrocode}
% \end{macro}
%
% \begin{macro}[rEXP]{\@@_sqrt_auxii_o:NnnnnnnnN}
%   This receives a continuation function~|#1|, then five blocks of~$4$
%   digits for~$y$, then two $8$-digit blocks and a single digit
%   for~$a$.  A common estimate of $\sqrt{a} - y = (a - y^2) / (\sqrt{a}
%   + y)$ is $(a - y^2)/(2y)$, which leads to alternating overestimates
%   and underestimates.  We tweak this, to only work with underestimates
%   (no need then to worry about signs in the computation).  Each step
%   finds the largest integer $j\leq 6$ such that $10^{4j}(a-y^2) <
%   2\cdot 10^{8}$, then computes the integer (with \eTeX{}'s rounding
%   division)
%   \[
%   10^{4j} z =
%   \Bigl[\bigl(\lfloor 10^{4j}(a-y^2)\rfloor - 257\bigr)
%   \cdot (0.5\cdot 10^{8})
%   \Bigm/ \lfloor 10^{8} y + 1\rfloor\Bigr] \,.
%   \]
%   The choice of~$j$ ensures that $10^{4j} z < 2\cdot 10^{8} \cdot
%   0.5\cdot 10^{8} / 10^{7} = 10^{9}$, thus $10^{9} + 10^{4j} z$ has
%   exactly $10$~digits, does not overflow \TeX{}'s integer range, and
%   starts with~$1$.  Incidentally, since all $a - y^2 \leq 3.2\cdot
%   10^{-8}$, we know that $j\geq 3$.
%
%   Let us show that $z$ is an underestimate of $\sqrt{a} - y$.  On the
%   one hand, $\sqrt{a} - y \leq 16\cdot 10^{-8}$ because this holds for
%   the initial~$y$ and values of~$y$ can only increase.  On the other
%   hand, the choice of~$j$ implies that $\sqrt{a} - y \leq
%   5(\sqrt{a}+y)(\sqrt{a}-y) = 5(a - y^2) < 10^{9-4j}$.  For $j=3$, the
%   first bound is better, while for larger~$j$, the second bound is
%   better.  For all $j\in[3,6]$, we find $\sqrt{a}-y < 16\cdot
%   10^{-2j}$.  From this, we deduce that
%   \[
%   10^{4j} (\sqrt{a}-y)
%   = \frac{10^{4j}\bigl(a-y^2-(\sqrt{a}-y)^2\bigr)}{2y}
%   \geq \frac{\bigl\lfloor 10^{4j}(a-y^2)\bigr\rfloor-257}
%   {2\cdot 10^{-8} \lfloor 10^{8}y+1\rfloor}
%   + \frac{1}{2}
%   \]
%   where we have replaced the bound $10^{4j}(16\cdot 10^{-2j}) = 256$
%   by~$257$ and extracted the corresponding term $1/\bigl(2\cdot
%   10^{-8} \lfloor 10^{8}y+1\rfloor\bigr) \geq 1/2$.  Given that
%   \eTeX{}'s integer division obeys $[b/c] \leq b/c + 1/2$, we deduce
%   that $10^{4j} z \leq 10^{4j} (\sqrt{a}-y)$, hence $y+z\leq\sqrt{a}$
%   is an underestimate of~$\sqrt{a}$, as claimed.  One implementation
%   detail: because the computation involves |-#4*#4| |-| |2*#3*#5| |-|
%   |2*#2*#6| which may be as low as $-5\cdot 10^{8}$, we need to use
%   the \texttt{pack_big} functions, and the \texttt{big} shifts.
%    \begin{macrocode}
\cs_new:Npn \@@_sqrt_auxii_o:NnnnnnnnN #1 #2#3#4#5#6 #7#8#9
  {
    \exp_after:wN #1
    \int_value:w \@@_int_eval:w \c_@@_big_leading_shift_int
      + #7 - #2 * #2
      \exp_after:wN \@@_pack_big:NNNNNNw
      \int_value:w \@@_int_eval:w \c_@@_big_middle_shift_int
        - 2 * #2 * #3
        \exp_after:wN \@@_pack_big:NNNNNNw
        \int_value:w \@@_int_eval:w \c_@@_big_middle_shift_int
          + #8 - #3 * #3 - 2 * #2 * #4
          \exp_after:wN \@@_pack_big:NNNNNNw
          \int_value:w \@@_int_eval:w \c_@@_big_middle_shift_int
            - 2 * #3 * #4 - 2 * #2 * #5
            \exp_after:wN \@@_pack_big:NNNNNNw
            \int_value:w \@@_int_eval:w \c_@@_big_middle_shift_int
              + #9 000 0000 - #4 * #4 - 2 * #3 * #5 - 2 * #2 * #6
              \exp_after:wN \@@_pack_big:NNNNNNw
              \int_value:w \@@_int_eval:w \c_@@_big_middle_shift_int
                - 2 * #4 * #5 - 2 * #3 * #6
                \exp_after:wN \@@_pack_big:NNNNNNw
                \int_value:w \@@_int_eval:w \c_@@_big_middle_shift_int
                  - #5 * #5 - 2 * #4 * #6
                  \exp_after:wN \@@_pack_big:NNNNNNw
                  \int_value:w \@@_int_eval:w
                    \c_@@_big_middle_shift_int
                    - 2 * #5 * #6
                    \exp_after:wN \@@_pack_big:NNNNNNw
                    \int_value:w \@@_int_eval:w
                      \c_@@_big_trailing_shift_int
                      - #6 * #6 \@@_sep:
    % (
    - 257 ) * 5000 0000 / (#2#3 + 1) + 10 0000 0000 \@@_sep:
    {#2}{#3}{#4}{#5}{#6} {#7}{#8}#9
  }
%    \end{macrocode}
% \end{macro}
%
% \begin{macro}[rEXP]
%   {
%     \@@_sqrt_auxiii_o:wnnnnnnnn,
%     \@@_sqrt_auxiv_o:NNNNNw,
%     \@@_sqrt_auxv_o:NNNNNw,
%     \@@_sqrt_auxvi_o:NNNNNw,
%     \@@_sqrt_auxvii_o:NNNNNw
%   }
%   We receive here the difference $a-y^2=d=\sum_i d_i \cdot 10^{-4i}$,
%   as \meta{d_2} \cs{@@_sep:} \Arg{d_3} \ldots{} \Arg{d_{10}}, where each
%   block has~$4$ digits, except \meta{d_2}.  This function finds the largest
%   $j\leq 6$ such that $10^{4j}(a-y^2) < 2\cdot 10^{8}$, then leaves an
%   open parenthesis and the integer
%   $\bigl\lfloor 10^{4j}(a-y^2)\bigr\rfloor$ in an integer
%   expression.  The closing parenthesis is provided by the caller
%   \cs{@@_sqrt_auxii_o:NnnnnnnnN}, which completes the expression
%   \[
%   10^{4j} z =
%   \Bigl[\bigl(\lfloor 10^{4j}(a-y^2)\rfloor - 257\bigr)
%   \cdot (0.5\cdot 10^{8})
%   \Bigm/ \lfloor 10^{8} y + 1\rfloor\Bigr]
%   \]
%   for an estimate of $10^{4j} (\sqrt{a} - y)$.  If $d_2\geq 2$, $j=3$
%   and the \texttt{auxiv} auxiliary receives $10^{12} z$.  If $d_2\leq
%   1$ but $10^{4} d_2 + d_3 \geq 2$, $j=4$ and the \texttt{auxv}
%   auxiliary is called, and receives $10^{16} z$, and so on.  In all
%   those cases, the \texttt{auxviii} auxiliary is set up to add~$z$
%   to~$y$, then go back to the \texttt{auxii} step with continuation
%   \texttt{auxiii} (the function we are currently describing).  The
%   maximum value of $j$ is~$6$, regardless of whether $10^{12} d_2 +
%   10^{8} d_3 + 10^{4} d_4 + d_5 \geq 1$.  In this last case, we detect
%   when $10^{24} z < 10^{7}$, which essentially means $\sqrt{a} - y
%   \lesssim 10^{-17}$: once this threshold is reached, there is enough
%   information to find the correctly rounded~$\sqrt{a}$ with only one
%   more call to \cs{@@_sqrt_auxii_o:NnnnnnnnN}.  Note that the
%   iteration cannot be stuck before reaching $j=6$, because for $j<6$,
%   one has $2\cdot 10^{8}\leq 10^{4(j+1)}(a-y^2)$, hence
%   \[
%   10^{4j} z
%   \geq \frac{(20000-257)(0.5\cdot 10^{8})}{\lfloor 10^{8} y + 1\rfloor}
%   \geq (20000-257)\cdot 0.5 > 0 \,.
%   \]
%    \begin{macrocode}
\cs_new:Npn \@@_sqrt_auxiii_o:wnnnnnnnn
    #1\@@_sep: #2#3#4#5#6#7#8#9
  {
    \if_int_compare:w #1 > \c_one_int
      \exp_after:wN \@@_sqrt_auxiv_o:NNNNNw
      \int_value:w \@@_int_eval:w (#1#2 %)
    \else:
      \if_int_compare:w #1#2 > \c_one_int
        \exp_after:wN \@@_sqrt_auxv_o:NNNNNw
        \int_value:w \@@_int_eval:w (#1#2#3 %)
      \else:
        \if_int_compare:w #1#2#3 > \c_one_int
          \exp_after:wN \@@_sqrt_auxvi_o:NNNNNw
          \int_value:w \@@_int_eval:w (#1#2#3#4 %)
        \else:
          \exp_after:wN \@@_sqrt_auxvii_o:NNNNNw
          \int_value:w \@@_int_eval:w (#1#2#3#4#5 %)
        \fi:
      \fi:
    \fi:
  }
\cs_new:Npn \@@_sqrt_auxiv_o:NNNNNw 1#1#2#3#4#5#6\@@_sep:
  { \@@_sqrt_auxviii_o:nnnnnnn {#1#2#3#4#5#6} {00000000} }
\cs_new:Npn \@@_sqrt_auxv_o:NNNNNw 1#1#2#3#4#5#6\@@_sep:
  { \@@_sqrt_auxviii_o:nnnnnnn {000#1#2#3#4#5} {#60000} }
\cs_new:Npn \@@_sqrt_auxvi_o:NNNNNw 1#1#2#3#4#5#6\@@_sep:
  { \@@_sqrt_auxviii_o:nnnnnnn {0000000#1} {#2#3#4#5#6} }
\cs_new:Npn \@@_sqrt_auxvii_o:NNNNNw 1#1#2#3#4#5#6\@@_sep:
  {
    \if_int_compare:w #1#2 = \c_zero_int
      \exp_after:wN \@@_sqrt_auxx_o:Nnnnnnnn
    \fi:
    \@@_sqrt_auxviii_o:nnnnnnn {00000000} {000#1#2#3#4#5}
  }
%    \end{macrocode}
% \end{macro}
%
% \begin{macro}[rEXP]
%   {\@@_sqrt_auxviii_o:nnnnnnn, \@@_sqrt_auxix_o:wnwnw}
%   Simply add the two $8$-digit blocks of~$z$, aligned to the last four
%   of the five $4$-digit blocks of~$y$, then call the \texttt{auxii}
%   auxiliary to evaluate $y'^{2} = (y+z)^{2}$.
%    \begin{macrocode}
\cs_new:Npn \@@_sqrt_auxviii_o:nnnnnnn #1#2 #3#4#5#6#7
  {
    \exp_after:wN \@@_sqrt_auxix_o:wnwnw
    \int_value:w \@@_int_eval:w #3
      \exp_after:wN \@@_basics_pack_low:NNNNNw
      \int_value:w \@@_int_eval:w #1 + 1#4#5
        \exp_after:wN \@@_basics_pack_low:NNNNNw
        \int_value:w \@@_int_eval:w #2 + 1#6#7 \@@_sep:
  }
\cs_new:Npn \@@_sqrt_auxix_o:wnwnw #1\@@_sep: #2#3\@@_sep: #4#5\@@_sep:
  {
    \@@_sqrt_auxii_o:NnnnnnnnN
      \@@_sqrt_auxiii_o:wnnnnnnnn {#1}{#2}{#3}{#4}{#5}
  }
%    \end{macrocode}
% \end{macro}
%
% \begin{macro}[rEXP]
%   {\@@_sqrt_auxx_o:Nnnnnnnn, \@@_sqrt_auxxi_o:wwnnN}
%   At this stage, $j=6$ and $10^{24} z < 10^{7}$, hence
%   \[
%   10^{7} + 1/2 > 10^{24} z + 1/2 \geq
%   \bigl(10^{24}(a-y^2) - 258\bigr) \cdot (0.5\cdot 10^{8})
%   \Bigm/ (10^{8} y + 1) \,,
%   \]
%   then $10^{24}(a-y^2) - 258 < 2 (10^{7} + 1/2) (y + 10^{-8})$, and
%   \[
%   10^{24}(a-y^2)
%   < (10^{7} + 1290.5) (1 + 10^{-8}/y) (2y)
%   < (10^{7} + 1290.5) (1 + 10^{-7}) (y + \sqrt{a}) \,,
%   \]
%   which finally implies $0\leq\sqrt{a}-y < 0.2\cdot 10^{-16}$.  In
%   particular, $y$~is an underestimate of~$\sqrt{a}$ and $y+0.5\cdot
%   10^{-16}$ is a (strict) overestimate.  There is at exactly one
%   multiple $m$~of $0.5\cdot 10^{-16}$ in the interval $[y, y+0.5\cdot
%   10^{-16})$.  If $m^2>a$, then the square root is inexact and is
%   obtained by rounding $m-\epsilon$ to a multiple of $10^{-16}$ (the
%   precise shift $0<\epsilon<0.5\cdot 10^{-16}$ is irrelevant for
%   rounding).  If $m^2=a$ then the square root is exactly~$m$, and
%   there is no rounding.  If $m^2<a$ then we round $m+\epsilon$.  For
%   now, discard a few irrelevant arguments |#1|, |#2|, |#3|, and find
%   the multiple of $0.5\cdot 10^{-16}$ within $[y, y+0.5\cdot
%   10^{-16})$; rather, only the last $4$~digits |#8| of~$y$ are
%   considered, and we do not perform any carry yet.  The \texttt{auxxi}
%   auxiliary sets up \texttt{auxii} with a continuation function
%   \texttt{auxxii} instead of \texttt{auxiii} as before.  To prevent
%   \texttt{auxii} from giving a negative results $a-m^2$, we compute
%   $a+10^{-16}-m^2$ instead, always positive since $m<\sqrt{a}+0.5\cdot
%   10^{-16}$ and $a\leq 1-10^{-16}$.
%    \begin{macrocode}
\cs_new:Npn \@@_sqrt_auxx_o:Nnnnnnnn #1#2#3 #4#5#6#7#8
  {
    \exp_after:wN \@@_sqrt_auxxi_o:wwnnN
    \int_value:w \@@_int_eval:w
      (#8 + 2499) / 5000 * 5000 \@@_sep:
      {#4} {#5} {#6} {#7} \@@_sep:
  }
\cs_new:Npn \@@_sqrt_auxxi_o:wwnnN #1\@@_sep: #2\@@_sep: #3#4#5
  {
    \@@_sqrt_auxii_o:NnnnnnnnN
      \@@_sqrt_auxxii_o:nnnnnnnnw
      #2 {#1}
      {#3} { #4 + 1 } #5
  }
%    \end{macrocode}
% \end{macro}
%
% \begin{macro}[rEXP]
%   {\@@_sqrt_auxxii_o:nnnnnnnnw, \@@_sqrt_auxxiii_o:w}
%   The difference $0\leq a+10^{-16}-m^2\leq
%   10^{-16}+(\sqrt{a}-m)(\sqrt{a}+m)\leq 2\cdot 10^{-16}$ was just
%   computed: its first $8$~digits vanish, as do the next four,~|#1|,
%   and most of the following four,~|#2|.  The guess~$m$ is an
%   overestimate if $a+10^{-16}-m^2 < 10^{-16}$, that is, |#1#2|
%   vanishes.  Otherwise it is an underestimate, unless
%   $a+10^{-16}-m^2=10^{-16}$ exactly.  For an underestimate, call the
%   \texttt{auxxiv} function with argument~$9998$.  For an exact result
%   call it with~$9999$, and for an overestimate call it with~$10000$.
%    \begin{macrocode}
\cs_new:Npn \@@_sqrt_auxxii_o:nnnnnnnnw 0\@@_sep: #1#2#3#4#5#6#7#8 #9\@@_sep:
  {
    \if_int_compare:w #1#2 > \c_zero_int
      \if_int_compare:w #1#2 = \c_one_int
        \if_int_compare:w #3#4 = \c_zero_int
          \if_int_compare:w #5#6 = \c_zero_int
            \if_int_compare:w #7#8 = \c_zero_int
              \@@_sqrt_auxxiii_o:w
            \fi:
          \fi:
        \fi:
      \fi:
      \exp_after:wN \@@_sqrt_auxxiv_o:wnnnnnnnN
      \int_value:w 9998
    \else:
      \exp_after:wN \@@_sqrt_auxxiv_o:wnnnnnnnN
      \int_value:w 10000
    \fi:
    \@@_sep:
  }
\cs_new:Npn \@@_sqrt_auxxiii_o:w \fi: \fi: \fi: \fi: #1 \fi: \@@_sep:
  {
    \fi: \fi: \fi: \fi: \fi:
    \@@_sqrt_auxxiv_o:wnnnnnnnN 9999 \@@_sep:
  }
%    \end{macrocode}
% \end{macro}
%
% \begin{macro}[rEXP]{\@@_sqrt_auxxiv_o:wnnnnnnnN}
%   This receives $9998$, $9999$ or $10000$ as~|#1| when $m$~is an
%   underestimate, exact, or an overestimate, respectively.  Then
%   comes~$m$ as five blocks of~$4$ digits, but where the last
%   block~|#6| may be $0$, $5000$, or~$10000$.  In the latter case, we
%   need to add a carry, unless $m$~is an overestimate (|#1|~is then
%   $10000$).  Then comes~$a$ as three arguments.  Rounding is done by
%   \cs{@@_round:NNN}, whose first argument is the final sign~$0$
%   (square roots are positive).  We fake its second argument.  It
%   should be the last digit kept, but this is only used when ties are
%   \enquote{rounded to even}, and only when the result is exactly
%   half-way between two representable numbers rational square roots of
%   numbers with $16$~significant digits have: this situation never
%   arises for the square root, as any exact square root of a $16$~digit
%   number has at most $8$~significant digits.  Finally, the last
%   argument is the next digit, possibly shifted by~$1$ when there are
%   further nonzero digits.  This is achieved by \cs{@@_round_digit:Nw},
%   which receives (after removal of the $10000$'s digit) one of $0000$,
%   $0001$, $4999$, $5000$, $5001$, or~$9999$, which it converts to $0$,
%   $1$, $4$, $5$, $6$, and~$9$, respectively.
%    \begin{macrocode}
\cs_new:Npn \@@_sqrt_auxxiv_o:wnnnnnnnN #1\@@_sep: #2#3#4#5#6 #7#8#9
  {
    \exp_after:wN \@@_basics_pack_high:NNNNNw
    \int_value:w \@@_int_eval:w 1 0000 0000 + #2#3
      \exp_after:wN \@@_basics_pack_low:NNNNNw
      \int_value:w \@@_int_eval:w 1 0000 0000
        + #4#5
        \if_int_compare:w #6 > #1 \exp_stop_f: + 1 \fi:
        + \exp_after:wN \@@_round:NNN
          \exp_after:wN 0
          \exp_after:wN 0
          \int_value:w
            \exp_after:wN \use_i:nn
            \exp_after:wN \@@_round_digit:Nw
            \int_value:w \@@_int_eval:w #6 + 19999 - #1 \@@_sep:
    \exp_after:wN \@@_sep:
  }
%    \end{macrocode}
% \end{macro}
%
% \subsection{About the sign and exponent}
%
% \begin{macro}[EXP]{\@@_logb_o:w, \@@_logb_aux_o:w}
%   The exponent of a normal number is its \meta{exponent} minus one.
%    \begin{macrocode}
\cs_new:Npn \@@_logb_o:w ? \s_@@ \@@_chk:w #1#2\@@_sep: @
  {
    \if_case:w #1 \exp_stop_f:
           \@@_case_use:nw
             { \@@_division_by_zero_o:Nnw \c_minus_inf_fp { logb } }
    \or:   \exp_after:wN \@@_logb_aux_o:w
    \or:   \@@_case_return_o:Nw \c_inf_fp
    \else: \@@_case_return_same_o:w
    \fi:
    \s_@@ \@@_chk:w #1 #2\@@_sep:
  }
\cs_new:Npn \@@_logb_aux_o:w \s_@@ \@@_chk:w #1 #2 #3 #4 \@@_sep:
  {
    \exp_after:wN \@@_parse:n \exp_after:wN
      { \int_value:w \int_eval:w #3 - 1 \exp_after:wN }
  }
%    \end{macrocode}
% \end{macro}
%
% \begin{macro}[EXP]{\@@_sign_o:w}
% \begin{macro}[EXP]{\@@_sign_aux_o:w}
%   Find the sign of the floating point: \texttt{nan}, |+0|, |-0|, |+1| or |-1|.
%    \begin{macrocode}
\cs_new:Npn \@@_sign_o:w ? \s_@@ \@@_chk:w #1#2\@@_sep: @
  {
    \if_case:w #1 \exp_stop_f:
           \@@_case_return_same_o:w
    \or:   \exp_after:wN \@@_sign_aux_o:w
    \or:   \exp_after:wN \@@_sign_aux_o:w
    \else: \@@_case_return_same_o:w
    \fi:
    \s_@@ \@@_chk:w #1 #2\@@_sep:
  }
\cs_new:Npn \@@_sign_aux_o:w \s_@@ \@@_chk:w #1 #2 #3 \@@_sep:
  { \exp_after:wN \@@_set_sign_o:w \exp_after:wN #2 \c_one_fp @ }
%    \end{macrocode}
% \end{macro}
% \end{macro}
%
% \begin{macro}[EXP]{\@@_set_sign_o:w}
%   This function is used for the unary minus and for \texttt{abs}.  It
%   leaves the sign of \texttt{nan} invariant, turns negative numbers
%   (sign~$2$) to positive numbers (sign~$0$) and positive numbers
%   (sign~$0$) to positive or negative numbers depending on~|#1|.  It
%   also expands after itself in the input stream, just like
%   \cs{@@_+_o:ww}.
%    \begin{macrocode}
\cs_new:Npn \@@_set_sign_o:w #1 \s_@@ \@@_chk:w #2#3#4\@@_sep: @
  {
    \exp_after:wN \@@_exp_after_o:w
    \exp_after:wN \s_@@
    \exp_after:wN \@@_chk:w
    \exp_after:wN #2
    \int_value:w
      \if_case:w #3 \exp_stop_f: #1 \or: 1 \or: 0 \fi: \exp_stop_f:
    #4\@@_sep:
  }
%    \end{macrocode}
% \end{macro}
%
% \subsection{Operations on tuples}
%
% \begin{macro}[EXP]{\@@_tuple_set_sign_o:w}
% \begin{macro}[EXP]{\@@_tuple_set_sign_aux_o:Nnw, \@@_tuple_set_sign_aux_o:w}
%   Two cases: |abs(|\meta{tuple}|)| for which |#1| is $0$ (invalid for
%   tuples) and |-|\meta{tuple} for which |#1| is $2$.  In that case,
%   map over all items in the tuple an auxiliary that dispatches to the
%   type-appropriate sign-flipping function.
%    \begin{macrocode}
\cs_new:Npn \@@_tuple_set_sign_o:w #1#2 @
  {
    \if_meaning:w 2 #1
      \exp_after:wN \@@_tuple_set_sign_aux_o:Nnw
    \fi:
    \@@_invalid_operation_o:nw { abs }
    #2
  }
\cs_new:Npn \@@_tuple_set_sign_aux_o:Nnw #1#2
  { \@@_tuple_map_o:nw \@@_tuple_set_sign_aux_o:w }
\cs_new:Npn \@@_tuple_set_sign_aux_o:w #1#2 \@@_sep:
  {
    \@@_change_func_type:NNN #1 \@@_set_sign_o:w
      \@@_parse_apply_unary_error:NNw
    2 #1 #2 \@@_sep: @
  }
%    \end{macrocode}
% \end{macro}
% \end{macro}
%
% \begin{macro}[EXP]{\@@_*_tuple_o:ww, \@@_tuple_*_o:ww, \@@_tuple_/_o:ww}
%   For \meta{number}|*|\meta{tuple} and \meta{tuple}|*|\meta{number}
%   and \meta{tuple}|/|\meta{number}, loop through the \meta{tuple} some
%   code that multiplies or divides by the appropriate \meta{number}.
%   Importantly we need to dispatch according to the type, and we make
%   sure to apply the operator in the correct order.
%    \begin{macrocode}
\cs_new:cpn { @@_*_tuple_o:ww } #1 \@@_sep:
  { \@@_tuple_map_o:nw { \@@_binary_type_o:Nww * #1 \@@_sep: } }
\cs_new:cpn { @@_tuple_*_o:ww } #1 \@@_sep: #2 \@@_sep:
  {
    \@@_tuple_map_o:nw { \@@_binary_rev_type_o:Nww * #2 \@@_sep: }
      #1 \@@_sep:
  }
\cs_new:cpn { @@_tuple_/_o:ww } #1 \@@_sep: #2 \@@_sep:
  {
    \@@_tuple_map_o:nw { \@@_binary_rev_type_o:Nww / #2 \@@_sep: }
      #1 \@@_sep:
  }
%    \end{macrocode}
% \end{macro}
%
% \begin{macro}[EXP]{\@@_tuple_+_tuple_o:ww, \@@_tuple_-_tuple_o:ww}
%   Check the two tuples have the same number of items and map through
%   these a helper that dispatches appropriately depending on the types.
%   This means |(1,2)+((1,1),2)| gives |(nan,4)|.
%    \begin{macrocode}
\cs_set_protected:Npn \@@_tmp:w #1
  {
    \cs_new:cpn { @@_tuple_#1_tuple_o:ww }
        \s_@@_tuple \@@_tuple_chk:w ##1 \@@_sep:
        \s_@@_tuple \@@_tuple_chk:w ##2 \@@_sep:
      {
        \int_compare:nNnTF
          { \@@_array_count:n {##1} } = { \@@_array_count:n {##2} }
          { \@@_tuple_mapthread_o:nww { \@@_binary_type_o:Nww #1 } }
          { \@@_invalid_operation_o:nww #1 }
        \s_@@_tuple \@@_tuple_chk:w {##1} \@@_sep:
        \s_@@_tuple \@@_tuple_chk:w {##2} \@@_sep:
      }
  }
\@@_tmp:w +
\@@_tmp:w -
%    \end{macrocode}
% \end{macro}
%
%    \begin{macrocode}
%</package>
%    \end{macrocode}
%
% \end{implementation}
%
% \PrintChanges
%
% \PrintIndex
