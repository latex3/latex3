% \iffalse meta-comment
%
%% File: l3legacy.dtx
%
% Copyright (C) 2019-2025 The LaTeX Project
%
% It may be distributed and/or modified under the conditions of the
% LaTeX Project Public License (LPPL), either version 1.3c of this
% license or (at your option) any later version.  The latest version
% of this license is in the file
%
%    https://www.latex-project.org/lppl.txt
%
% This file is part of the "l3kernel bundle" (The Work in LPPL)
% and all files in that bundle must be distributed together.
%
% -----------------------------------------------------------------------
%
% The development version of the bundle can be found at
%
%    https://github.com/latex3/latex3
%
% for those people who are interested.
%
%<*driver>
\documentclass[full,kernel]{l3doc}
\begin{document}
  \DocInput{\jobname.dtx}
\end{document}
%</driver>
% \fi
%
% \title{^^A
%   The \pkg{l3legacy} module\\ Interfaces to legacy concepts^^A
% }
%
% \author{^^A
%  The \LaTeX{} Project\thanks
%    {^^A
%      E-mail:
%        \href{mailto:latex-team@latex-project.org}
%          {latex-team@latex-project.org}^^A
%    }^^A
% }
%
% \date{Released 2025-06-09}
%
% \maketitle
%
% \begin{documentation}
%
% There are a small number of \TeX{} or \LaTeXe{} concepts which are not used
% in \pkg{expl3} code but which need to be manipulated when working as a \LaTeXe{}
% package. To allow these to be integrated cleanly into \pkg{expl3} code, a set
% of legacy interfaces are provided here.
%
% \begin{function}[EXP,pTF]{\legacy_if:n}
%   \begin{syntax}
%     \cs{legacy_if_p:n} \Arg{name}
%     \cs{legacy_if:nTF} \Arg{name} \Arg{true code} \Arg{false code}
%   \end{syntax}
%   Tests if the \LaTeXe{}/plain \TeX{} conditional (generated by \tn{newif})
%   is \texttt{true} or \texttt{false} and branches accordingly. The
%   \meta{name} of the conditional should \emph{omit} the leading \texttt{if}.
% \end{function}
%
% \begin{function}[added = 2021-05-10]
%   {
%     \legacy_if_set_true:n, \legacy_if_set_false:n,
%     \legacy_if_gset_true:n, \legacy_if_gset_false:n
%   }
%   \begin{syntax}
%     \cs{legacy_if_set_true:n} \Arg{name}
%     \cs{legacy_if_set_false:n} \Arg{name}
%   \end{syntax}
%   Sets the \LaTeXe{}/plain \TeX{} conditional |\if|\meta{name}
%   (generated by \tn{newif}) to be \texttt{true} or \texttt{false}.
% \end{function}
%
% \begin{function}[added = 2021-05-10]{\legacy_if_set:nn, \legacy_if_gset:nn}
%   \begin{syntax}
%     \cs{legacy_if_set:nn} \Arg{name} \Arg{boolexpr}
%   \end{syntax}
%   Sets the \LaTeXe{}/plain \TeX{} conditional |\if|\meta{name}
%   (generated by \tn{newif}) to the result of evaluating the
%   \meta{boolean expression}.
% \end{function}
%
% \end{documentation}
%
% \begin{implementation}
%
% \section{\pkg{l3legacy} implementation}
%
%    \begin{macrocode}
%<*code>
%    \end{macrocode}
%
%    \begin{macrocode}
%<@@=legacy>
%    \end{macrocode}
%
% \begin{macro}[EXP,pTF]{\legacy_if:n}
%   A friendly wrapper. We need to use the \cs{if:w} approach here, rather than
%   testing against \tn{iftrue}/\tn{iffalse} as the latter approach fails for
%   primitive conditionals such as \tn{ifmmode}. The \cs{reverse_if:N} here
%   means that we get a slightly more useful error if the name is undefined.
%    \begin{macrocode}
\prg_new_conditional:Npnn \legacy_if:n #1 { p , T , F , TF }
  {
    \exp_after:wN \reverse_if:N
      \cs:w if#1 \cs_end:
      \prg_return_false:
    \else:
      \prg_return_true:
    \fi: 
  }
%    \end{macrocode}
% \end{macro}
%
% \begin{macro}
%   {
%     \legacy_if_set_true:n, \legacy_if_set_false:n,
%     \legacy_if_gset_true:n, \legacy_if_gset_false:n
%   }
%   A friendly wrapper.
%    \begin{macrocode}
\cs_new_protected:Npn \legacy_if_set_true:n #1
  { \cs_set_eq:cN { if#1 } \if_true: }
\cs_new_protected:Npn \legacy_if_set_false:n #1
  { \cs_set_eq:cN { if#1 } \if_false: }
\cs_new_protected:Npn \legacy_if_gset_true:n #1
  { \cs_gset_eq:cN { if#1 } \if_true: }
\cs_new_protected:Npn \legacy_if_gset_false:n #1
  { \cs_gset_eq:cN { if#1 } \if_false: }
%    \end{macrocode}
% \end{macro}
%
% \begin{macro}{\legacy_if_set:nn, \legacy_if_gset:nn}
%   A more elaborate wrapper.
%    \begin{macrocode}
\cs_new_protected:Npn \legacy_if_set:nn #1#2
  {
    \bool_if:nTF {#2} \legacy_if_set_true:n \legacy_if_set_false:n
    {#1}
  }
\cs_new_protected:Npn \legacy_if_gset:nn #1#2
  {
    \bool_if:nTF {#2} \legacy_if_gset_true:n \legacy_if_gset_false:n
    {#1}
  }
%    \end{macrocode}
% \end{macro}
%
%    \begin{macrocode}
%</code>
%    \end{macrocode}
%
% \end{implementation}
%
% \PrintIndex
