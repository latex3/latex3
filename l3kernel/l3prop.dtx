% \iffalse meta-comment
%
%% File: l3prop.dtx
%
% Copyright (C) 1990-2025 The LaTeX Project
%
% It may be distributed and/or modified under the conditions of the
% LaTeX Project Public License (LPPL), either version 1.3c of this
% license or (at your option) any later version.  The latest version
% of this license is in the file
%
%    https://www.latex-project.org/lppl.txt
%
% This file is part of the "l3kernel bundle" (The Work in LPPL)
% and all files in that bundle must be distributed together.
%
% -----------------------------------------------------------------------
%
% The development version of the bundle can be found at
%
%    https://github.com/latex3/latex3
%
% for those people who are interested.
%
%<*driver>
\documentclass[full,kernel]{l3doc}
\begin{document}
  \DocInput{\jobname.dtx}
\end{document}
%</driver>
% \fi
%
% \title{^^A
%   The \pkg{l3prop} module\\ Property lists^^A
% }
%
% \author{^^A
%  The \LaTeX{} Project\thanks
%    {^^A
%      E-mail:
%        \href{mailto:latex-team@latex-project.org}
%          {latex-team@latex-project.org}^^A
%    }^^A
% }
%
% \date{Released 2025-08-13}
%
% \maketitle
%
% \begin{documentation}
%
% \pkg{expl3} implements a \enquote{property list} data type, which contains
% an unordered list of entries each of which consists of a \meta{key} (string)
% and an associated \meta{value} (token list). The \meta{key} and \meta{value}
% may both be given as any balanced text, and the \meta{key} is processed using
% \cs{tl_to_str:n}, meaning that category codes are ignored. Entries can be
% manipulated individually, as well as collectively by applying a function to
% every key--value pair within the list.
%
% Each entry in a property list must have a unique \meta{key}: if an entry is
% added to a property list which already contains the \meta{key} then the new
% entry overwrites the existing one. The \meta{keys} are compared on a
% string basis, using the same method as \cs{str_if_eq:nnTF}.
%
% Property lists are intended for storing key-based information for use within
% code. They can be converted from and to key--value lists, which are a form of
% \emph{input} parsed by the \pkg{l3keys} module.  If a key--value list contains
% a \meta{key} multiple times, only the last \meta{value} associated to it will
% be kept in the conversion to a property list.
%
% Internally, property lists can use two distinct implementations with different
% data storage, which are decided when declaring the property list variable
% using \cs{prop_new:N} (\enquote{flat} storage) or \cs{prop_new_linked:N}
% (\enquote{linked} storage). After a property list is declared with
% \cs{prop_new:N} or \cs{prop_new_linked:N}, the type of internal data storage
% can be changed by \cs{prop_make_flat:N} or \cs{prop_make_linked:N}, but only
% at the outermost group level. All other \pkg{l3prop} functions transparently
% manipulate either storage method and convert as needed.
% \begin{itemize}
%   \item
%     The (default) \enquote{flat} storage method is suited for a relatively
%     small number of entries, or when the property list is likely to be
%     manipulated (copied, mapped) as a whole rather than entry-wise.  It is
%     significantly faster for \cs{prop_set_eq:NN}, and only slightly faster for
%     \cs{prop_clear:N}, \cs{prop_concat:NNN}, and mapping functions
%     \cs[no-index]{prop_map_\ldots{}}.
%
%   \item
%     The \enquote{linked} storage method is meant for property lists with a
%     large numbers of entries.  It takes up more of \TeX{}'s memory during a run, but is
%     significantly faster (for long lists) when accessing or modifying
%     individual entries using functions such as \cs{prop_if_in:Nn},
%     \cs{prop_item:Nn}, \cs{prop_put:Nnn}, \cs{prop_get:NnN},
%     \cs{prop_pop:NnN}, \cs{prop_remove:Nn}, as it takes a constant
%     time for these operations (rather than the number of items for a
%     \enquote{flat} property list).  A technical drawback is that
%     memory is permanently used\footnote{Until the end of the run, that
%     is.} by \meta{keys} stored in a \enquote{linked} property list,
%     even after they are removed and the property list is deleted.
% \end{itemize}
%
% \section{Creating and initializing property lists}
%
% \begin{function}{\prop_new:N, \prop_new:c}
%   \begin{syntax}
%     \cs{prop_new:N} \meta{property list}
%   \end{syntax}
%   Creates a new \enquote{flat} \meta{property list} or raises an error if the
%   name is already taken. The declaration is global. The \meta{property list}
%   initially contains no entries.  See also \cs{prop_new_linked:N}.
% \end{function}
%
% \begin{function}[added = 2024-02-12]{\prop_new_linked:N, \prop_new_linked:c}
%   \begin{syntax}
%     \cs{prop_new_linked:N} \meta{property list}
%   \end{syntax}
%   Creates a new \enquote{linked} \meta{property list} or raises an error if
%   the name is already taken. The declaration is global. The \meta{property
%   list} initially contains no entries.  The internal data storage differs from
%   that produced by \cs{prop_new:N} and it is optimized for property lists with
%   a large number of entries.
% \end{function}
%
% \begin{function}
%   {\prop_clear:N, \prop_clear:c, \prop_gclear:N, \prop_gclear:c}
%   \begin{syntax}
%     \cs{prop_clear:N} \meta{property list}
%   \end{syntax}
%   Clears all entries from the \meta{property list}.
% \end{function}
%
% \begin{function}
%   {
%     \prop_clear_new:N,  \prop_clear_new:c,
%     \prop_gclear_new:N, \prop_gclear_new:c
%   }
%   \begin{syntax}
%     \cs{prop_clear_new:N} \meta{property list}
%   \end{syntax}
%   Ensures that the \meta{property list} exists globally by applying
%   \cs{prop_new:N} if necessary, then applies
%   \cs[index=prop_clear:N]{prop_(g)clear:N} to leave the list empty.
%   \begin{texnote}
%     If the property list exists and is of \enquote{linked} type, it
%     is cleared but not made into a flat property list.
%   \end{texnote}
% \end{function}
%
% \begin{function}[added = 2024-02-12]
%   {
%     \prop_clear_new_linked:N,  \prop_clear_new_linked:c,
%     \prop_gclear_new_linked:N, \prop_gclear_new_linked:c
%   }
%   \begin{syntax}
%     \cs{prop_clear_new_linked:N} \meta{property list}
%   \end{syntax}
%   Ensures that the \meta{property list} exists globally by applying
%   \cs{prop_new_linked:N} if necessary, then applies
%   \cs[index=prop_clear:N]{prop_(g)clear:N} to leave the list empty.
%   \begin{texnote}
%     If the property list exists and is of \enquote{flat} type, it
%     is cleared but not made into a linked property list.
%   \end{texnote}
% \end{function}
%
% \begin{function}
%   {
%     \prop_set_eq:NN,  \prop_set_eq:cN,  \prop_set_eq:Nc,  \prop_set_eq:cc,
%     \prop_gset_eq:NN, \prop_gset_eq:cN, \prop_gset_eq:Nc, \prop_gset_eq:cc
%   }
%   \begin{syntax}
%     \cs{prop_set_eq:NN} \meta{property list_1} \meta{property list_2}
%   \end{syntax}
%   Sets the content of \meta{property list_1} equal to that of \meta{property
%   list_2}.  This converts as needed between the two storage types.
% \end{function}
%
% \begin{function}[added = 2017-11-28, updated = 2021-11-07]
%   {
%     \prop_set_from_keyval:Nn, \prop_set_from_keyval:cn,
%     \prop_gset_from_keyval:Nn, \prop_gset_from_keyval:cn,
%   }
%   \begin{syntax}
%     \cs{prop_set_from_keyval:Nn} \meta{property list} \\
%     ~~\{ \\
%     ~~~~\meta{key_1} |=| \meta{value_1} |,| \\
%     ~~~~\meta{key_2} |=| \meta{value_2} |,| \ldots{} \\
%     ~~\}
%   \end{syntax}
%   Sets \meta{property list} to contain key--value pairs given in the second
%   argument.  If duplicate keys appear only the last of the values is kept.
%   In contrast to most keyval lists (\emph{e.g.}~those in \pkg{l3keys}), each
%   key here \emph{must} be followed with an \texttt{=} sign even to specify an
%   empty \meta{value}.
%
%   Spaces are trimmed around every \meta{key} and every \meta{value},
%   and if the result of trimming spaces consists of a single brace
%   group then a set of outer braces is removed.  This enables both the
%   \meta{key} and the \meta{value} to contain spaces, commas or equal
%   signs.  The \meta{key} is then processed by \cs{tl_to_str:n}.
%   This function correctly detects the |=| and |,| signs provided they
%   have the standard category code~$12$ or they are active.
% \end{function}
%
% \begin{function}[added = 2017-11-28, updated = 2021-11-07]
%   {\prop_const_from_keyval:Nn, \prop_const_from_keyval:cn}
%   \begin{syntax}
%     \cs{prop_const_from_keyval:Nn} \meta{property list} \\
%     ~~\{ \\
%     ~~~~\meta{key_1} |=| \meta{value_1} |,| \\
%     ~~~~\meta{key_2} |=| \meta{value_2} |,| \ldots{} \\
%     ~~\}
%   \end{syntax}
%   Creates a new constant \enquote{flat} \meta{property list} or raises
%   an error if the
%   name is already taken. The \meta{property list} is set globally to
%   contain key--value pairs given in the second argument, processed in
%   the way described for \cs{prop_set_from_keyval:Nn}.  If duplicate
%   keys appear only the last of the values is kept.
%   This function correctly detects the |=| and |,| signs provided they
%   have the standard category code~$12$ or they are active.
% \end{function}
%
% \begin{function}[added = 2024-02-12]
%   {\prop_const_linked_from_keyval:Nn, \prop_const_linked_from_keyval:cn}
%   \begin{syntax}
%     \cs{prop_const_linked_from_keyval:Nn} \meta{property list} \\
%     ~~\{ \\
%     ~~~~\meta{key_1} |=| \meta{value_1} |,| \\
%     ~~~~\meta{key_2} |=| \meta{value_2} |,| \ldots{} \\
%     ~~\}
%   \end{syntax}
%   Creates a new constant \enquote{linked} \meta{property list} or raises an
%   error if the
%   name is already taken. The \meta{property list} is set globally to
%   contain key--value pairs given in the second argument, processed in
%   the way described for \cs{prop_set_from_keyval:Nn}.  If duplicate
%   keys appear only the last of the values is kept.
%   This function correctly detects the |=| and |,| signs provided they
%   have the standard category code~$12$ or they are active.
% \end{function}
%
% \begin{function}[added = 2024-02-12]{\prop_make_flat:N, \prop_make_flat:c}
%   \begin{syntax}
%     \cs{prop_make_flat:N} \meta{property list}
%   \end{syntax}
%   Changes the internal storage type of the \meta{property list} to be
%   the same \enquote{flat} storage as \cs{prop_new:N}.  The key--value
%   pairs of the \meta{property list} are preserved by the change.  If
%   the property list was already flat then nothing is done.  This
%   function can only be used at the outermost group level.
% \end{function}
%
% \begin{function}[added = 2024-02-12]{\prop_make_linked:N, \prop_make_linked:c}
%   \begin{syntax}
%     \cs{prop_make_linked:N} \meta{property list}
%   \end{syntax}
%   Changes the internal storage type of the \meta{property list} to be
%   the same \enquote{linked} storage as \cs{prop_new_linked:N}.  The
%   key--value pairs of the \meta{property list} are preserved by the
%   change.  If the property list was already linked then nothing is
%   done.  This function can only be used at the outermost group level.
% \end{function}
%
% \section{Adding and updating property list entries}
%
% \begin{function}[updated = 2012-07-09]
%   {
%     \prop_put:Nnn,  \prop_put:NnV,  \prop_put:Nnv,  \prop_put:Nne,
%     \prop_put:NVn,  \prop_put:NVV,  \prop_put:NVv,  \prop_put:NVe,
%     \prop_put:Nvn,  \prop_put:NvV,  \prop_put:Nvv,  \prop_put:Nve, 
%     \prop_put:Nen,  \prop_put:NeV,  \prop_put:Nev,  \prop_put:Nee,
%     \prop_put:Nno,  \prop_put:Non,  \prop_put:Noo,
%     \prop_put:cnn,  \prop_put:cnV,  \prop_put:cnv,  \prop_put:cne,
%     \prop_put:cVn,  \prop_put:cVV,  \prop_put:cVv,  \prop_put:cVe,
%     \prop_put:cvn,  \prop_put:cvV,  \prop_put:cvv,  \prop_put:cve, 
%     \prop_put:cen,  \prop_put:ceV,  \prop_put:cev,  \prop_put:cee,
%     \prop_put:cno,  \prop_put:con,  \prop_put:coo,
%     \prop_gput:Nnn, \prop_gput:NnV, \prop_gput:Nnv, \prop_gput:Nne,
%     \prop_gput:NVn, \prop_gput:NVV, \prop_gput:NVv, \prop_gput:NVe,
%     \prop_gput:Nvn, \prop_gput:NvV, \prop_gput:Nvv, \prop_gput:Nve, 
%     \prop_gput:Nen, \prop_gput:NeV, \prop_gput:Nev, \prop_gput:Nee,
%     \prop_gput:Nno, \prop_gput:Non, \prop_gput:Noo,
%     \prop_gput:cnn, \prop_gput:cnV, \prop_gput:cnv, \prop_gput:cne,
%     \prop_gput:cVn, \prop_gput:cVV, \prop_gput:cVv, \prop_gput:cVe,
%     \prop_gput:cvn, \prop_gput:cvV, \prop_gput:cvv, \prop_gput:cve, 
%     \prop_gput:cen, \prop_gput:ceV, \prop_gput:cev, \prop_gput:cee,
%     \prop_gput:cno, \prop_gput:con, \prop_gput:coo
%   }
%   \begin{syntax}
%     \cs{prop_put:Nnn} \meta{property list} \Arg{key} \Arg{value}
%   \end{syntax}
%   Adds an entry to the \meta{property list} which may be accessed
%   using the \meta{key} and which has \meta{value}. If the \meta{key}
%   is already present in the \meta{property list}, the existing entry
%   is overwritten by the new \meta{value}. Both the \meta{key} and
%   \meta{value} may contain any \meta{balanced text}. The \meta{key} is
%   stored after processing with \cs{tl_to_str:n}, meaning that category
%   codes are ignored.
% \end{function}
%
% \begin{function}[added = 2024-03-30, updated = 2024-05-07]
%   {
%     \prop_put_if_not_in:Nnn, \prop_put_if_not_in:NnV, \prop_put_if_not_in:Nnv, \prop_put_if_not_in:Nne,
%     \prop_put_if_not_in:NVn, \prop_put_if_not_in:NVV, \prop_put_if_not_in:NVv, \prop_put_if_not_in:NVe,
%     \prop_put_if_not_in:Nvn, \prop_put_if_not_in:NvV, \prop_put_if_not_in:Nvv, \prop_put_if_not_in:Nve,
%     \prop_put_if_not_in:Nen, \prop_put_if_not_in:NeV, \prop_put_if_not_in:Nev, \prop_put_if_not_in:Nee,
%     \prop_put_if_not_in:cnn, \prop_put_if_not_in:cnV, \prop_put_if_not_in:cnv, \prop_put_if_not_in:cne,
%     \prop_put_if_not_in:cVn, \prop_put_if_not_in:cVV, \prop_put_if_not_in:cVv, \prop_put_if_not_in:cVe,
%     \prop_put_if_not_in:cvn, \prop_put_if_not_in:cvV, \prop_put_if_not_in:cvv, \prop_put_if_not_in:cve,
%     \prop_put_if_not_in:cen, \prop_put_if_not_in:ceV, \prop_put_if_not_in:cev, \prop_put_if_not_in:cee,
%     \prop_gput_if_not_in:Nnn, \prop_gput_if_not_in:NnV, \prop_gput_if_not_in:Nnv, \prop_gput_if_not_in:Nne,
%     \prop_gput_if_not_in:NVn, \prop_gput_if_not_in:NVV, \prop_gput_if_not_in:NVv, \prop_gput_if_not_in:NVe,
%     \prop_gput_if_not_in:Nvn, \prop_gput_if_not_in:NvV, \prop_gput_if_not_in:Nvv, \prop_gput_if_not_in:Nve,
%     \prop_gput_if_not_in:Nen, \prop_gput_if_not_in:NeV, \prop_gput_if_not_in:Nev, \prop_gput_if_not_in:Nee,
%     \prop_gput_if_not_in:cnn, \prop_gput_if_not_in:cnV, \prop_gput_if_not_in:cnv, \prop_gput_if_not_in:cne,
%     \prop_gput_if_not_in:cVn, \prop_gput_if_not_in:cVV, \prop_gput_if_not_in:cVv, \prop_gput_if_not_in:cVe,
%     \prop_gput_if_not_in:cvn, \prop_gput_if_not_in:cvV, \prop_gput_if_not_in:cvv, \prop_gput_if_not_in:cve,
%     \prop_gput_if_not_in:cen, \prop_gput_if_not_in:ceV, \prop_gput_if_not_in:cev, \prop_gput_if_not_in:cee
%   }
%   \begin{syntax}
%     \cs{prop_put_if_not_in:Nnn} \meta{property list} \Arg{key} \Arg{value}
%   \end{syntax}
%   If the \meta{key} is present in the \meta{property list} then no
%   action is taken. Otherwise, a new entry is added as described for
%   \cs{prop_put:Nnn}.
% \end{function}
%
% \begin{function}[added = 2021-05-16]
%   {
%     \prop_concat:NNN,  \prop_concat:ccc,
%     \prop_gconcat:NNN, \prop_gconcat:ccc
%   }
%   \begin{syntax}
%     \cs{prop_concat:NNN} \meta{property list_1} \meta{property list_2} \meta{property list_3}
%   \end{syntax}
%   Combines the key--value pairs of \meta{property list_2} and
%   \meta{property list_3}, and saves the result in \meta{property list_1}.  If a
%   key appears in both \meta{property list_2} and \meta{property list_3} then the
%   last value, namely the value in \meta{property list_3} is kept.
%   This converts as needed between the two storage types.
% \end{function}
%
% \begin{function}[added = 2021-05-16, updated = 2021-11-07]
%   {
%     \prop_put_from_keyval:Nn, \prop_put_from_keyval:cn,
%     \prop_gput_from_keyval:Nn, \prop_gput_from_keyval:cn,
%   }
%   \begin{syntax}
%     \cs{prop_put_from_keyval:Nn} \meta{property list} \\
%     ~~\{ \\
%     ~~~~\meta{key_1} |=| \meta{value_1} |,| \\
%     ~~~~\meta{key_2} |=| \meta{value_2} |,| \ldots{} \\
%     ~~\}
%   \end{syntax}
%   Updates the \meta{property list} by adding entries for each key--value
%   pair given in the second argument.  The addition is done through
%   \cs{prop_put:Nnn}, hence if the \meta{property list} already contains
%   some of the keys, the corresponding values are discarded and
%   replaced by those given in the key--value list.  If duplicate keys
%   appear in the key--value list then only the last of the values is kept.
%
%   The function is equivalent to storing the key--value pairs in a
%   temporary property list using \cs{prop_set_from_keyval:Nn}, then
%   combining \meta{property list} with the temporary variable using
%   \cs{prop_concat:NNN}.  In particular, the \meta{keys} and
%   \meta{values} are space-trimmed and unbraced as described in
%   \cs{prop_set_from_keyval:Nn}.
%   This function correctly detects the |=| and |,| signs provided they
%   have the standard category code~$12$ or they are active.
% \end{function}
%
% \section{Recovering values from property lists}
%
% \begin{function}[updated = 2011-08-28]
%   {
%     \prop_get:NnN, \prop_get:NVN, \prop_get:NvN, \prop_get:NeN,
%     \prop_get:NoN,
%     \prop_get:cnN, \prop_get:cVN, \prop_get:cvN, \prop_get:ceN,
%     \prop_get:coN,
%     \prop_get:cnc
%   }
%   \begin{syntax}
%     \cs{prop_get:NnN} \meta{property list} \Arg{key} \meta{tl var}
%   \end{syntax}
%   Recovers the \meta{value} stored with \meta{key} from the
%   \meta{property list}, and places this in the \meta{tl
%   var}. If the \meta{key} is not found in the
%   \meta{property list} then the \meta{tl~var} is set
%   to the special marker \cs{q_no_value}. The \meta{tl
%     var} is set within the current \TeX{} group. See also
%   \cs{prop_get:NnNTF}.
% \end{function}
%
% \begin{function}[updated = 2011-08-18]
%   {
%     \prop_pop:NnN, \prop_pop:NVN,
%     \prop_pop:NoN, 
%     \prop_pop:cnN, \prop_pop:cVN,
%     \prop_pop:coN
%   }
%   \begin{syntax}
%     \cs{prop_pop:NnN} \meta{property list} \Arg{key} \meta{tl var}
%   \end{syntax}
%   Recovers the \meta{value} stored with \meta{key} from the
%   \meta{property list}, and places this in the \meta{tl
%   var}. If the \meta{key} is not found in the
%   \meta{property list} then the \meta{tl~var} is set
%   to the special marker \cs{q_no_value}. The \meta{key} and
%   \meta{value} are then deleted from the property list. Both
%   assignments are local.  See also \cs{prop_pop:NnNTF}.
% \end{function}
%
% \begin{function}[updated = 2011-08-18]
%   {
%     \prop_gpop:NnN, \prop_gpop:NVN,
%     \prop_gpop:NoN,
%     \prop_gpop:cnN, \prop_gpop:cVN,
%     \prop_gpop:coN
%   }
%   \begin{syntax}
%     \cs{prop_gpop:NnN} \meta{property list} \Arg{key} \meta{tl var}
%   \end{syntax}
%   Recovers the \meta{value} stored with \meta{key} from the
%   \meta{property list}, and places this in the \meta{tl
%   var}. If the \meta{key} is not found in the
%   \meta{property list} then the \meta{tl~var} is set
%   to the special marker \cs{q_no_value}. The \meta{key} and
%   \meta{value} are then deleted from the property list.
%   The \meta{property list} is modified globally, while the assignment of
%   the \meta{tl~var} is local.  See also \cs{prop_gpop:NnNTF}.
% \end{function}
%
% \begin{function}[EXP, added = 2014-07-17]
%   {
%     \prop_item:Nn, \prop_item:NV, \prop_item:Ne, \prop_item:No,
%     \prop_item:cn, \prop_item:cV, \prop_item:ce, \prop_item:co
%   }
%   \begin{syntax}
%     \cs{prop_item:Nn} \meta{property list} \Arg{key}
%   \end{syntax}
%   Expands to the \meta{value} corresponding to the \meta{key} in
%   the \meta{property list}. If the \meta{key} is missing, this has
%   an empty expansion.
%   \begin{texnote}
%     For \enquote{flat} property lists, this expandable function iterates
%     through every key--value pair and is therefore slower than a
%     non-expandable approach based on \cs{prop_get:NnN}.
%     (For \enquote{linked} property lists both functions are fast.)
%
%     The result is returned within the \tn{unexpanded}
%     primitive (\cs{exp_not:n}), which means that the \meta{value}
%     does not expand further when appearing in an \texttt{e}-type
%     or \texttt{x}-type argument expansion.
%   \end{texnote}
% \end{function}
%
% \begin{function}[EXP]{\prop_count:N, \prop_count:c}
%   \begin{syntax}
%     \cs{prop_count:N} \meta{property list}
%   \end{syntax}
%   Leaves the number of key--value pairs in the \meta{property list} in
%   the input stream as an \meta{integer denotation}.
% \end{function}
%
% \begin{function}[EXP]{\prop_to_keyval:N}
%   \begin{syntax}
%     \cs{prop_to_keyval:N} \meta{property list}
%   \end{syntax}
%   Expands to the \meta{property list} in a key--value notation. Keep in mind
%   that a \meta{property list} is \emph{unordered}, while key--value interfaces
%   are not necessarily, so this cannot be used for arbitrary interfaces.
%   \begin{texnote}
%     The result is returned within the \tn{unexpanded} primitive
%     (\cs{exp_not:n}), which means that the key--value list does not expand
%     further when appearing in an \texttt{e}-type or \texttt{x}-type argument expansion.
%     It also needs exactly two steps of expansion.
%   \end{texnote}
% \end{function}
%
% \section{Modifying property lists}
%
% \begin{function}[added = 2012-05-12]
%   {
%     \prop_remove:Nn,  \prop_remove:NV,  \prop_remove:Ne,
%     \prop_remove:cn,  \prop_remove:cV,  \prop_remove:ce,
%     \prop_gremove:Nn, \prop_gremove:NV, \prop_gremove:Ne,
%     \prop_gremove:cn, \prop_gremove:cV, \prop_gremove:ce,
%   }
%   \begin{syntax}
%     \cs{prop_remove:Nn} \meta{property list} \Arg{key}
%   \end{syntax}
%   Removes the entry listed under \meta{key} from the
%   \meta{property list}.  If the \meta{key} is
%   not found in the \meta{property list} no change occurs,
%   \emph{i.e}~there is no need to test for the existence of a key before
%   deleting it.
% \end{function}
%
% \section{Property list conditionals}
%
% \begin{function}[EXP, pTF, added = 2012-03-03]
%   {\prop_if_exist:N, \prop_if_exist:c}
%   \begin{syntax}
%     \cs{prop_if_exist_p:N} \meta{property list}
%     \cs{prop_if_exist:NTF} \meta{property list} \Arg{true code} \Arg{false code}
%   \end{syntax}
%   Tests whether the \meta{property list} is currently defined.  This does not
%   check that the \meta{property list} really is a property list variable.
% \end{function}
%
% \begin{function}[EXP, pTF]{\prop_if_empty:N, \prop_if_empty:c}
%   \begin{syntax}
%     \cs{prop_if_empty_p:N} \meta{property list}
%     \cs{prop_if_empty:NTF} \meta{property list} \Arg{true code} \Arg{false code}
%   \end{syntax}
%   Tests if the \meta{property list} is empty (containing no entries).
% \end{function}
%
% \begin{function}[EXP, pTF, updated = 2011-09-15]
%   {
%     \prop_if_in:Nn, \prop_if_in:NV, \prop_if_in:Ne, \prop_if_in:No,
%     \prop_if_in:cn, \prop_if_in:cV, \prop_if_in:ce, \prop_if_in:co
%   }
%   \begin{syntax}
%     \cs{prop_if_in_p:Nn} \meta{property list} \Arg{key}
%     \cs{prop_if_in:NnTF} \meta{property list} \Arg{key} \Arg{true code} \Arg{false code}
%   \end{syntax}
%   Tests if the \meta{key} is present in the \meta{property list},
%   making the comparison using the method described by \cs{str_if_eq:nnTF}.
%   \begin{texnote}
%     For \enquote{flat} property lists, this expandable function iterates
%     through every key--value pair and is therefore slower than a
%     non-expandable approach based on \cs{prop_get:NnNTF}.
%     (For \enquote{linked} property lists both functions are fast.)
%   \end{texnote}
% \end{function}
%
% \section{Recovering values from property lists with branching}
%
% The functions in this section combine tests for the presence of a key
% in a property list with  recovery of the associated valued. This makes them
% useful for cases where different code follows depending on the presence
% or absence of a key in a property list. They offer increased readability
% and performance over separate testing and recovery phases.
%
% \begin{function}[TF, updated = 2012-05-19]
%   {
%     \prop_get:NnN, \prop_get:NVN, \prop_get:NvN, \prop_get:NeN,
%     \prop_get:NoN,
%     \prop_get:cnN, \prop_get:cVN, \prop_get:cvN, \prop_get:ceN,
%     \prop_get:coN,
%     \prop_get:cnc
%   }
%   \begin{syntax}
%     \cs{prop_get:NnNTF} \meta{property list} \Arg{key} \meta{tl~var} \\
%     ~~\Arg{true code} \Arg{false code}
%   \end{syntax}
%   If the \meta{key} is not present in the \meta{property list}, leaves
%   the \meta{false code} in the input stream.  The value of the
%   \meta{tl~var} is not defined in this case and should
%   not be relied upon.  If the \meta{key} is present in the
%   \meta{property list}, stores the corresponding \meta{value} in the
%   \meta{tl~var} without removing it from the
%   \meta{property list}, then leaves the \meta{true code} in the input
%   stream.  The \meta{tl~var} is assigned locally.
% \end{function}
%
% \begin{function}[TF, added = 2011-08-18, updated = 2012-05-19]
%   {
%     \prop_pop:NnN, \prop_pop:NVN,
%     \prop_pop:NoN,
%     \prop_pop:cnN, \prop_pop:cVN,
%     \prop_pop:coN
%   }
%   \begin{syntax}
%     \cs{prop_pop:NnNTF} \meta{property list} \Arg{key} \meta{tl~var} \\
%     ~~\Arg{true code} \Arg{false code}
%   \end{syntax}
%   If the \meta{key} is not present in the \meta{property list}, leaves
%   the \meta{false code} in the input stream.  The value of the
%   \meta{tl~var} is not defined in this case and should
%   not be relied upon.  If the \meta{key} is present in
%   the \meta{property list}, pops the corresponding \meta{value}
%   in the \meta{tl~var}, i.e., removes the item from
%   the \meta{property list}.
%   Both the \meta{property list} and the \meta{tl~var}
%   are assigned locally.
% \end{function}
%
% \begin{function}[TF, added = 2011-08-18, updated = 2012-05-19]
%   {
%     \prop_gpop:NnN, \prop_gpop:NVN,
%     \prop_gpop:NoN,
%     \prop_gpop:cnN, \prop_gpop:cVN,
%     \prop_gpop:coN
%   }
%   \begin{syntax}
%     \cs{prop_gpop:NnNTF} \meta{property list} \Arg{key} \meta{tl~var} \\
%     ~~\Arg{true code} \Arg{false code}
%   \end{syntax}
%   If the \meta{key} is not present in the \meta{property list}, leaves
%   the \meta{false code} in the input stream.  The value of the
%   \meta{tl~var} is not defined in this case and should
%   not be relied upon.  If the \meta{key} is present in
%   the \meta{property list}, pops the corresponding \meta{value}
%   in the \meta{tl~var}, i.e., removes the item from
%   the \meta{property list}.
%   The \meta{property list} is modified globally, while the
%   \meta{tl~var} is assigned locally.
% \end{function}
%
% \section{Mapping over property lists}
%
% All mappings are done at the current group level, i.e., any
% local assignments made by the \meta{function} or \meta{code} discussed
% below remain in effect after the loop.
%
% \begin{function}[rEXP, updated = 2013-01-08]
%   {\prop_map_function:NN, \prop_map_function:cN}
%   \begin{syntax}
%     \cs{prop_map_function:NN} \meta{property list} \meta{function}
%   \end{syntax}
%   Applies \meta{function} to every \meta{entry} stored in the
%   \meta{property list}. The \meta{function} receives two arguments for
%   each iteration: the \meta{key} and associated \meta{value}.
%   The order in which \meta{entries} are returned is not defined and
%   should not be relied upon.
%   To pass further arguments to the \meta{function}, see
%   \cs{prop_map_inline:Nn} (non-expandable) or \cs{prop_map_tokens:Nn}.
% \end{function}
%
% \begin{function}[updated = 2013-01-08]
%   {\prop_map_inline:Nn, \prop_map_inline:cn}
%   \begin{syntax}
%     \cs{prop_map_inline:Nn} \meta{property list} \Arg{inline function}
%   \end{syntax}
%   Applies \meta{inline function} to every \meta{entry} stored
%   within the \meta{property list}. The \meta{inline function} should
%   consist of code which receives the \meta{key} as |#1| and the
%   \meta{value} as |#2|.
%   The order in which \meta{entries} are returned is not defined and
%   should not be relied upon.
% \end{function}
%
% \begin{function}[rEXP]
%   {\prop_map_tokens:Nn, \prop_map_tokens:cn}
%   \begin{syntax}
%     \cs{prop_map_tokens:Nn} \meta{property list} \Arg{code}
%   \end{syntax}
%   Analogue of \cs{prop_map_function:NN} which maps several tokens
%   instead of a single function.  The \meta{code} receives each
%   key--value pair in the \meta{property list} as two trailing brace
%   groups. For instance,
%   \begin{verbatim}
%     \prop_map_tokens:Nn \l_my_prop { \str_if_eq:nnT { mykey } }
%   \end{verbatim}
%   expands to the value corresponding to \texttt{mykey}: for each
%   pair in |\l_my_prop| the function \cs{str_if_eq:nnT} receives
%   \texttt{mykey}, the \meta{key} and the \meta{value} as its three
%   arguments.  For that specific task, \cs{prop_item:Nn} is faster.
% \end{function}
%
% \begin{function}[rEXP, updated = 2012-06-29]{\prop_map_break:}
%   \begin{syntax}
%     \cs{prop_map_break:}
%   \end{syntax}
%   Used to terminate a \cs[no-index]{prop_map_\ldots} function before all
%   entries in the \meta{property list} have been processed. This
%   normally takes place within a conditional statement, for example
%   \begin{verbatim}
%     \prop_map_inline:Nn \l_my_prop
%       {
%         \str_if_eq:nnTF { #1 } { bingo }
%           { \prop_map_break: }
%           {
%             % Do something useful
%           }
%       }
%   \end{verbatim}
%   Use outside of a \cs[no-index]{prop_map_\ldots} scenario leads to low
%   level \TeX{} errors.
%   \begin{texnote}
%     When the mapping is broken, additional tokens may be inserted
%     before further items are taken
%     from the input stream. This depends on the design of the mapping
%     function.
%   \end{texnote}
% \end{function}
%
% \begin{function}[rEXP, updated = 2012-06-29]{\prop_map_break:n}
%   \begin{syntax}
%     \cs{prop_map_break:n} \Arg{code}
%   \end{syntax}
%   Used to terminate a \cs[no-index]{prop_map_\ldots} function before all
%   entries in the \meta{property list} have been processed, inserting
%   the \meta{code} after the mapping has ended. This
%   normally takes place within a conditional statement, for example
%   \begin{verbatim}
%     \prop_map_inline:Nn \l_my_prop
%       {
%         \str_if_eq:nnTF { #1 } { bingo }
%           { \prop_map_break:n { <code> } }
%           {
%             % Do something useful
%           }
%       }
%   \end{verbatim}
%   Use outside of a \cs[no-index]{prop_map_\ldots} scenario leads to low
%   level \TeX{} errors.
%   \begin{texnote}
%     When the mapping is broken, additional tokens may be inserted
%     before the \meta{code} is
%     inserted into the input stream.
%     This depends on the design of the mapping function.
%   \end{texnote}
% \end{function}
%
% \section{Viewing property lists}
%
% \begin{function}[updated = 2021-04-29]{\prop_show:N, \prop_show:c}
%   \begin{syntax}
%     \cs{prop_show:N} \meta{property list}
%   \end{syntax}
%   Displays the entries in the \meta{property list} in the terminal,
%   and specifies its storage type.
% \end{function}
%
% \begin{function}[added = 2014-08-12, updated = 2021-04-29]{\prop_log:N, \prop_log:c}
%   \begin{syntax}
%     \cs{prop_log:N} \meta{property list}
%   \end{syntax}
%   Writes the entries in the \meta{property list} in the log file,
%   and specifies its storage type.
% \end{function}
%
% \section{Scratch property lists}
%
% There is no need to include both flat and linked property lists as
% scratch variables.  We arbitrarily pick the older implementation.
%
% \begin{variable}[added = 2012-06-23]{\l_tmpa_prop, \l_tmpb_prop}
%   Scratch \enquote{flat} property lists for local assignment.
%   These are never used by
%   the kernel code, and so are safe for use with any \LaTeX3-defined
%   function. However, they may be overwritten by other non-kernel
%   code and so should only be used for short-term storage.
% \end{variable}
%
% \begin{variable}[added = 2012-06-23]{\g_tmpa_prop, \g_tmpb_prop}
%   Scratch \enquote{flat} property lists for global assignment.
%   These are never used by
%   the kernel code, and so are safe for use with any \LaTeX3-defined
%   function. However, they may be overwritten by other non-kernel
%   code and so should only be used for short-term storage.
% \end{variable}
%
% \section{Constants}
%
% \begin{variable}{\c_empty_prop}
%   A permanently-empty property list used for internal comparisons.
% \end{variable}
%
% \end{documentation}
%
% \begin{implementation}
%
% \section{\pkg{l3prop} implementation}
%
% \TestFiles{m3prop001, m3prop002, m3prop003, m3prop004, m3show001}
%
%    \begin{macrocode}
%<*code>
%    \end{macrocode}
%
%    \begin{macrocode}
%<@@=prop>
%    \end{macrocode}
%
% With the (default) flat data storage, a property list is a macro whose
% top-level expansion is of the form
% \begin{quote}
%   \cs{s_@@} \cs{@@_chk:w}
%   \cs{@@_pair:wn} \meta{key_1} \cs{s_@@} \Arg{value_1} \\
%   \ldots{} \\
%   \cs{@@_pair:wn} \meta{key_n} \cs{s_@@} \Arg{value_n} \\
% \end{quote}
% where \cs{s_@@} is a scan mark (equal to \cs{scan_stop:}), \cs{@@_chk:w}
% produces a suitable error if the property list is used directly in the input
% stream, and \cs{@@_pair:wn} can be used to map through the property list.
%
% With the linked data storage, each property list entry
% \meta{key_i}--\meta{value_i} is stored into a token list
% \cs[no-index]{@@~\meta{prefix}~\meta{key_i}}.  The \meta{prefix} is one or
% more characters (no spaces), constructed automatically only once, when the
% property list is initially declared.  The control sequence name does not
% conform to standard naming for variables because (1) this is an internal
% control sequence, not really a \pkg{expl3} variable; (2) keeping track of the
% scope |l| or~|g| throughout all functions would be a pretty big mess,
% especially if users accidentally mix local and global use (we would have to
% always check for such mistakes, rather than only checking when suitable debug
% options are set); (3) shorter control sequence names use less memory and are
% quicker in case of hash collisions, which may matter since we are using many
% control sequences.
%
% We need to enable mapping through such a property list, but without storing a
% list of all entries anywhere: this is achieved by making each of these token
% lists also store a pointer to the next entry.  To enable efficient deletion,
% the token lists also store a pointer to the previous entry.  This means we
% have a doubly-linked list.  To avoid having to special-case the two ends of
% the doubly-linked list when deleting entries, we include as a zeroth entry in
% the doubly-linked list the property list variable itself, and we include as an
% $(n+1)$-th entry in the doubly-linked list an end-pointer
% \cs{@@~\meta{prefix}} (no trailing space, so it differs from an empty key).
% The space before \meta{prefix} ensures there is no collision with
% other \pkg{l3prop} internal functions, even if we have very many
% linked property lists being defined.
%
% The property list variable itself is a token list of the form
% \begin{quote}
%   \cs{@@_flatten:w} \cs{@@~\meta{prefix}} \cs{s_@@} \Arg{prefix}
%   \cs[no-index]{@@~\meta{prefix}~\meta{key_1}}
% \end{quote}
% Here, \cs{@@_flatten:w} serves as an efficiently recognized marker, and when
% \texttt{f}-expanded it is tasked with fully unpacking the property list into
% the same form as the default data storage so as to ease conversion.  The
% \meta{prefix} is used when looking up an entry.  The token list
% \cs{@@~\meta{prefix}} (see below) contains a pointer to the last key to help
% insert a new entry.  The pointer to \meta{key_1} is needed to start a mapping.
% The token list labeled by \meta{key_i} is of the form
% \begin{quote}
%   \cs{use_none:n} \cs[no-index]{@@~\meta{prefix}~\meta{key$_{i-1}$}}
%   \cs{@@_pair:wn} \meta{key_i} \cs{s_@@} \Arg{value_i}
%   \cs[no-index]{@@~\meta{prefix}~\meta{key$_{i+1}$}}
% \end{quote}
% where the pointer to \meta{key$_{i-1}$} is needed when deleting the
% \meta{key_i}.  Expanding this will run \cs{@@_pair:wn} on the
% \meta{key_i}--\meta{value_i} pair (for speed, \meta{key_i} is kept as explicit
% tokens rather than slowly extracting it from a control sequence name), then
% move on to the next key, thus mapping through the whole list.  The mapping is
% ended upon expanding \cs{@@~\meta{prefix}}, which is the token list
% \begin{quote}
%   \cs{use_none:n} \cs[no-index]{@@~\meta{prefix}~\meta{key_n}}
% \end{quote}
% Let us think about deleting the \meta{key_i}.  We need to update the
% \meta{key$_{i-1}$} and \meta{key$_{i+1}$} to point to each other
% instead of \meta{key_i}.  To edit the corresponding token lists, it is
% important that \cs[no-index]{@@~\meta{prefix}~\meta{key_i}} be at the
% \enquote{same place} in the token lists also in the boundary cases
% $i=1$ or $i=n$, namely as the second token, or as the second argument
% after \cs{s_@@}.
%
% \subsection{Internal auxiliaries}
%
% \begin{macro}{\@@_tmp:w}
%   Scratch macro, defined as needed, for instance to save \cs{@@_pair:wn} when
%   concatenating.
%    \begin{macrocode}
\cs_new_eq:NN \@@_tmp:w ?
%    \end{macrocode}
% \end{macro}
%
% \begin{variable}{\l_@@_tmp_tl}
%   Token list used in various places: for the prefix; when converting from flat
%   to linked props; and to store the new key--value pair inserted by
%   \cs{prop_put:Nnn}.
%    \begin{macrocode}
\tl_new:N \l_@@_tmp_tl
%    \end{macrocode}
% \end{variable}
%
% \begin{variable}{\s_@@_mark,\s_@@_stop}
%   Internal scan marks.
%    \begin{macrocode}
\scan_new:N \s_@@_mark
\scan_new:N \s_@@_stop
%    \end{macrocode}
% \end{variable}
%
% \begin{variable}{\q_@@_recursion_tail,\q_@@_recursion_stop}
%   Internal recursion quarks.
%    \begin{macrocode}
\quark_new:N \q_@@_recursion_tail
\quark_new:N \q_@@_recursion_stop
%    \end{macrocode}
% \end{variable}
%
% \begin{macro}[EXP]{\@@_if_recursion_tail_stop:n}
% \begin{macro}[EXP]{\@@_if_recursion_tail_stop:o}
%   Functions to query recursion quarks.
%    \begin{macrocode}
\__kernel_quark_new_test:N \@@_if_recursion_tail_stop:n
\cs_generate_variant:Nn \@@_if_recursion_tail_stop:n { o }
%    \end{macrocode}
% \end{macro}
% \end{macro}
%
% \subsection{Structure of a property list}
%
% \begin{macro}{\s_@@}
%   A private scan mark is used as a marker after each key, and at the
%   very beginning of the property list.
%    \begin{macrocode}
\scan_new:N \s_@@
%    \end{macrocode}
% \end{macro}
%
% \begin{macro}{\@@_chk:w, \@@_chk_loop:nw, \@@_chk_get:nw}
%   This removes the flat property list from the input stream and complains
%   about a bad use of a property list.  Since property lists do not have an
%   end-marker, we slowly peek ahead in a loop.  Speed does not matter since
%   this is for an error situation.  While \cs{@@_pair:wn} does not keep a fixed
%   definition, it always includes the internal \cs{s_@@} in its argument
%   specification, so that there is no risk of accidentally picking up a public
%   token instead of \cs{@@_pair:wn} when doing a meaning test.  We collect the
%   keys and values to produce a more useful error message.
%    \begin{macrocode}
\cs_new_protected:Npn \@@_chk:w { \@@_chk_loop:nw { } }
\cs_new_protected:Npn \@@_chk_loop:nw #1
  {
    \peek_meaning:NTF \@@_pair:wn
      { \@@_chk_get:nw {#1} }
      { \msg_error:nne { prop } { misused } {#1} }
  }
\cs_new_protected:Npn \@@_chk_get:nw #1 \@@_pair:wn #2 \s_@@ #3
  { \@@_chk_loop:nw { #1 , ~ {#2} = { \tl_to_str:n {#3} } } }
%    \end{macrocode}
% \end{macro}
%
% \begin{macro}{\@@_pair:wn}
%   Used as \cs{@@_pair:wn} \meta{key} \cs{s_@@} \Arg{item} for both storage
%   types, this internal token starts each key--value pair in the property list.
%   This default definition is changed globally by any mapping function, so
%   there is not much point trying to make it an error.  Instead, the error is
%   produced by \cs{@@_chk:w}.
%    \begin{macrocode}
\cs_new:Npn \@@_pair:wn #1 \s_@@ #2 { }
%    \end{macrocode}
% \end{macro}
%
% \begin{macro}{\@@_flatten:w}
%   We implement here the fact that \texttt{f}-expanding a linked property list
%   gives a flat property list.
%   Leaving a linked property list in the input stream will turn it into a flat
%   property list so that the error implemented by \cs{@@_chk:w} will correctly
%   be triggered.
%    \begin{macrocode}
\cs_new_protected:Npn \@@_flatten:w #1 \s_@@ #2#3
  { \use:e { \@@_flatten_aux:N #3 } }
%    \end{macrocode}
% \end{macro}
%
% \begin{macro}[rEXP]
%   {\@@_flatten:N, \@@_flatten_aux:w, \@@_flatten_aux:N, \@@_flatten_loop:w}
%   The main function \cs{@@_flatten:N} receives a linked property list and
%   flattens it.  The auxiliary \cs{@@_flatten_aux:N} receives a pointer to the
%   first key and flattens the linked property list into a flat property list.
%   This is only restricted-expandable as it involves mapping through all of the
%   property list's entries starting from \meta{key_1}.  The looping function
%   \cs{@@_flatten_loop:w} removes \cs{use_none:n} and a backwards pointer~|#2|,
%   leaves the key--value pair for \cs{use:e} to receive, and calls itself
%   again after expanding the next key's token list.  Its argument |#3| is
%   empty, except at the end where it is the \cs{use_none:nnnn} appearing in the
%   definition of~\cs{@@_flatten_aux:N}, which ends the loop.
%    \begin{macrocode}
\cs_new:Npn \@@_flatten:N #1
  { \exp_after:wN \@@_flatten_aux:w #1 }
\cs_new:Npn \@@_flatten_aux:w #1 \s_@@ #2 { \@@_flatten_aux:N }
\cs_new:Npn \@@_flatten_aux:N #1
  {
    \s_@@ \@@_chk:w
    \exp_after:wN \@@_flatten_loop:w #1
    \use_none:nnnn \@@_pair:wn \s_@@ { }
  }
\cs_new:Npn \@@_flatten_loop:w #1#2#3 \@@_pair:wn #4 \s_@@ #5
  {
    #3
    \exp_not:n { \@@_pair:wn #4 \s_@@ {#5} }
    \exp_after:wN \@@_flatten_loop:w
  }
%    \end{macrocode}
% \end{macro}
%
% \begin{variable}{\g_@@_prefix_int, \c_@@_basis_int}
%   Used to assign prefixes for each linked property list.  It is converted to
%   base \cs{c_@@_basis_int}, then each digit is converted to a character,
%   starting at |!| (the character after space).
%    \begin{macrocode}
\int_new:N \g_@@_prefix_int
\int_const:Nn \c_@@_basis_int { \c_max_char_int - `\! }
%    \end{macrocode}
% \end{variable}
%
% \begin{macro}{\@@_next_prefix:, \@@_to_prefix:n}
%   Store in \cs{l_@@_tmp_tl} the conversion of \cs{g_@@_prefix_int} to
%   characters, and increment this integer for use in the next linked property
%   list.  No need to optimize since this is only used when declaring the
%   property list the first time.
%   The aim here is to make this string as short as we can, given the
%   range of distinct characters available.  This speeds up the work of
%   \cs{cs:w} \ldots{} \cs{cs_end:} that looks up keys in the hash table.
%    \begin{macrocode}
\cs_new_protected:Npn \@@_next_prefix:
  {
    \tl_set:Ne \l_@@_tmp_tl
      { \@@_to_prefix:n { \g_@@_prefix_int } }
    \int_gincr:N \g_@@_prefix_int
  }
\cs_new:Npn \@@_to_prefix:n #1
  {
    \int_compare:nNnTF {#1} > \c_@@_basis_int
      {
        \exp_args:Nf \@@_to_prefix:n
          { \int_div_truncate:nn {#1} \c_@@_basis_int }
        \exp_args:Nf \@@_to_prefix:n
          { \int_mod:nn {#1} \c_@@_basis_int }
      }
      { \char_generate:nn { `\! + #1 } { 12 } }
  }
%    \end{macrocode}
% \end{macro}
%
% \begin{macro}{\@@_if_flat:NTF, \@@_if_flat_aux:w}
%   We could either test for the presence of \cs{@@_chk:w} (flat
%   property list) or of \cs{@@_flatten:w} (linked property list).  We
%   make the second choice; this way props that are accidentally
%   \tn{relax} are treated as they were before.  The auxiliary receives
%   \cs{use_i:nn} or \cs{use_ii:nn} as~|#3|.
%   As a transitional fix we avoid erroring in case the prop is undefined (the \cs{exp_after:wN} is omitted in that case, taking the flat branch).
%    \begin{macrocode}
\cs_new:Npn \@@_if_flat:NTF #1
  {
    \prop_if_exist:NT #1
    \exp_after:wN \@@_if_flat_aux:w #1
    \s_@@_mark \use_ii:nn
    \@@_flatten:w \s_@@_mark \use_i:nn \s_@@_stop
  }
\cs_new:Npn \@@_if_flat_aux:w
    #1 \@@_flatten:w #2 \s_@@_mark #3 #4 \s_@@_stop {#3}
%    \end{macrocode}
% \end{macro}
%
% \subsection{Allocation and initialization}
%
% \begin{variable}[tested = m3prop004]{\c_empty_prop}
%   An empty flat prop.
%    \begin{macrocode}
\tl_const:Nn \c_empty_prop { \s_@@ \@@_chk:w }
%    \end{macrocode}
% \end{variable}
%
% \begin{macro}[tested = m3prop001]{\prop_new:N, \prop_new:c}
%   Flat property lists are initialized with the value \cs{c_empty_prop}.
%    \begin{macrocode}
\cs_new_protected:Npn \prop_new:N #1
  {
    \__kernel_chk_if_free_cs:N #1
    \cs_gset_eq:NN #1 \c_empty_prop
  }
\cs_generate_variant:Nn \prop_new:N { c }
%    \end{macrocode}
% \end{macro}
%
% \begin{macro}[tested = m3prop001]{\prop_new_linked:N, \prop_new_linked:c}
% \begin{macro}{\@@_new_linked:N}
%   The auxiliary is used in \cs{prop_make_linked:N}.
%   For linked property lists, get a new prefix in \cs{l_@@_tmp_tl}, then
%   use it to set up the internal structure: the last token in~|#1| is usually a
%   pointer to the first key, which is here the end-pointer.  That end-pointer
%   has a pointer to the previous key (usually the last key), which is the
%   variable~|#1| itself that begins the doubly-linked list.
%    \begin{macrocode}
\cs_new_protected:Npn \prop_new_linked:N #1
  {
    \__kernel_chk_if_free_cs:N #1
    \@@_new_linked:N #1
  }
\cs_new_protected:Npn \@@_new_linked:N #1
  {
    \@@_next_prefix:
    \cs_gset_nopar:Npe #1
      {
        \@@_flatten:w
        \exp_not:c { @@ ~ \l_@@_tmp_tl }
        \s_@@ { \l_@@_tmp_tl }
        \exp_not:c { @@ ~ \l_@@_tmp_tl }
      }
    \cs_gset_nopar:cpe { @@ ~ \l_@@_tmp_tl }
      {
        \exp_not:N \use_none:n
        \exp_not:N #1
      }
  }
\cs_generate_variant:Nn \prop_new_linked:N { c }
%    \end{macrocode}
% \end{macro}
% \end{macro}
%
% \begin{macro}[tested = m3prop001]{\prop_clear:N, \prop_clear:c}
% \begin{macro}[tested = m3prop001]{\prop_gclear:N, \prop_gclear:c}
% \begin{macro}{\@@_clear:NNN, \@@_clear:wNNN, \@@_clear_loop:Nw}
%   Clearing a flat property list is like declaring it anew, simply setting it
%   equal to \cs{c_empty_prop}.  For linked property lists we must clear all of
%   the variables storing individual keys, which requires a loop.  At each step
%   of the loop, \cs{@@_clear_loop:Nw} receives
%   \cs[index=cs_set_eq:NN]{cs_(g)set_eq:NN}, \cs{use_none:n}, the backwards
%   pointer, an empty~|#4| (except at the end of the loop), and the key--value
%   pair |#5=#6| which we disregard.  The looping auxiliary undefines the
%   previous key's token list (this includes the main token list, but that is
%   fine because it is restored at the end) and calls itself after expanding the
%   next key's token list.  The loop ends when |#4| is \cs{use_none:nnnn}.
%   After the loop, \cs{@@_clear:wNNN} correctly sets up the main variable~|#6|
%   and the end-pointer~|#1|.  Importantly, this is done using
%   \cs[index=cs_set_nopar:Npe]{cs_(g)set_nopar:Npe} and \cs{exp_not:n} because
%   the almost-equivalent \cs{tl_set:Nn} would complain in debug mode about the
%   fact that the main variable is undefined at this stage.  Importantly,
%   \cs{@@_clear_entries:NN} is used in the implementation of
%   \cs{prop_set_eq:NN}.
%    \begin{macrocode}
\cs_new_protected:Npn \prop_clear:N
  { \@@_clear:NNN \cs_set_eq:NN \cs_set_nopar:Npe }
\cs_generate_variant:Nn \prop_clear:N  { c }
\cs_new_protected:Npn \prop_gclear:N
  { \@@_clear:NNN \cs_gset_eq:NN \cs_gset_nopar:Npe }
\cs_generate_variant:Nn \prop_gclear:N { c }
\cs_new_protected:Npn \@@_clear:NNN #1#2#3
  {
    \@@_if_flat:NTF #3
      { #1 #3 \c_empty_prop }
      { \exp_after:wN \@@_clear:wNNN #3 #1 #2 #3 }
  }
\cs_new_protected:Npn \@@_clear:wNNN
    \@@_flatten:w #1 \s_@@ #2#3#4#5#6
  {
    \@@_clear_entries:NN #4 #3
    #5 #6 { \exp_not:n { \@@_flatten:w #1 \s_@@ {#2} #1 } }
    #5 #1 { \exp_not:n { \use_none:n #6 } }
  }
\cs_new_protected:Npn \@@_clear_entries:NN #1#2
  {
    \exp_after:wN \@@_clear_loop:Nw \exp_after:wN #1 #2
    \use_none:nnnn \@@_pair:wn \s_@@ { }
  }
\cs_new_protected:Npn \@@_clear_loop:Nw
    #1#2#3#4 \@@_pair:wn #5 \s_@@ #6
  {
    #1 #3 \tex_undefined:D
    #4
    \exp_after:wN \@@_clear_loop:Nw
    \exp_after:wN #1
  }
%    \end{macrocode}
% \end{macro}
% \end{macro}
% \end{macro}
%
% \begin{macro}[tested = m3prop001]
%   {
%     \prop_clear_new:N, \prop_clear_new:c,
%     \prop_gclear_new:N, \prop_gclear_new:c,
%     \prop_clear_new_linked:N, \prop_clear_new_linked:c,
%     \prop_gclear_new_linked:N, \prop_gclear_new_linked:c
%   }
%   A simple variation of the token list functions.
%    \begin{macrocode}
\cs_new_protected:Npn \prop_clear_new:N  #1
  { \prop_if_exist:NTF #1 { \prop_clear:N #1 } { \prop_new:N #1 } }
\cs_generate_variant:Nn \prop_clear_new:N  { c }
\cs_new_protected:Npn \prop_gclear_new:N #1
  { \prop_if_exist:NTF #1 { \prop_gclear:N #1 } { \prop_new:N #1 } }
\cs_generate_variant:Nn \prop_gclear_new:N { c }
\cs_new_protected:Npn \prop_clear_new_linked:N  #1
  { \prop_if_exist:NTF #1 { \prop_clear:N #1 } { \prop_new_linked:N #1 } }
\cs_generate_variant:Nn \prop_clear_new_linked:N  { c }
\cs_new_protected:Npn \prop_gclear_new_linked:N #1
  { \prop_if_exist:NTF #1 { \prop_gclear:N #1 } { \prop_new_linked:N #1 } }
\cs_generate_variant:Nn \prop_gclear_new_linked:N { c }
%    \end{macrocode}
% \end{macro}
%
% \begin{macro}[tested = m3prop001]
%   {\prop_set_eq:NN, \prop_set_eq:cN, \prop_set_eq:Nc, \prop_set_eq:cc}
% \begin{macro}[tested = m3prop001]
%   {\prop_gset_eq:NN, \prop_gset_eq:cN, \prop_gset_eq:Nc, \prop_gset_eq:cc}
% \begin{macro}
%   {
%     \@@_set_eq:NNNN, \@@_set_eq:wNNNN, \@@_set_eq:nNnNN,
%     \@@_set_eq_loop:NNnw, \@@_set_eq_end:w
%   }
%   If both variables are accidentally the same variable (or equal flat property
%   lists, as it turns out) we do nothing, otherwise the following code would
%   lose all entries.  If the target variable~|#3| is a flat prop, either copy
%   directly or flatten before copying.  If it is a linked prop, we must clear
%   it, then go through the entries in~|#4| to add them to~|#3|.
%    \begin{macrocode}
\cs_new_protected:Npn \prop_set_eq:NN
  { \@@_set_eq:NNNN \cs_set_eq:NN \cs_set_nopar:Npe }
\cs_generate_variant:Nn \prop_set_eq:NN { Nc , cN , cc }
\cs_new_protected:Npn \prop_gset_eq:NN
  { \@@_set_eq:NNNN \cs_gset_eq:NN \cs_gset_nopar:Npe }
\cs_generate_variant:Nn \prop_gset_eq:NN { Nc , cN , cc }
\cs_new_protected:Npn \@@_set_eq:NNNN #1#2#3#4
  {
    \cs_if_eq:NNF #3#4
      {
        \@@_if_flat:NTF #3
          {
            \@@_if_flat:NTF #4
              { #1 #3 #4 }
              { #2 #3 { \@@_flatten:N #4 } }
          }
          { \exp_after:wN \@@_set_eq:wNNNN #3 #1#2#3#4 }
      }
  }
\cs_new_protected:Npn \@@_set_eq:wNNNN
    \@@_flatten:w #1 \s_@@ #2#3#4#5#6#7
  {
    \@@_clear_entries:NN #4 #3
    \exp_args:Nf \@@_set_eq:nNnNN {#7} #1 {#2} #5 #6
  }
%    \end{macrocode}
%   We have used that |f|-expanding either type of prop gives a flat prop.  At
%   this stage \cs{@@_set_eq:nNnNN} receives the second variable as a flat prop,
%   the end-pointer, the prefix, the suitable
%   \cs[index=cs_set_nopar:Npe]{cs_(g)set_nopar:Npe} assignment, and the first
%   variable itself.  Remove the leading \cs{s_@@} and \cs{@@_chk:w} with
%   \cs{use_i:nnn}, then start the loop.
%    \begin{macrocode}
\cs_new_protected:Npn \@@_set_eq:nNnNN #1#2#3#4#5
  {
    \use_i:nnn
      {
        \@@_set_eq_loop:NNnw #5 #4 {#3}
        \@@_flatten:w #2 \s_@@ {#3}
      }
      #1
    \use_none:n \@@_pair:wn ? \s_@@
  }
%    \end{macrocode}
%   The looping function receives the current pointer~|#1| (initially the
%   variable itself), the defining function~|#2| and the prefix~|#3|, then a
%   partial definition~|#4| (which in later stages includes the backwards
%   pointer), followed by the current value as \cs{s_@@} |{#5}|.  It seeks the
%   next key~|#7| to construct in \cs{l_@@_tmp_tl} the next pointer
%   \cs[no-index]{@@~\meta{prefix}~\meta{next key}} (the
%   argument~|#6| is empty, except at the end of the loop, where it is
%   \cs{use_none:n} in such a way as to delete the \meta{space} and \meta{next
%   key}).  Then the token list (current pointer)~|#1| is set-up to contain the
%   partial definition and current value, as well as the newly constructed next
%   pointer.  After a line responsible for correctly ending the loop with
%   \cs{@@_set_eq_end:w}, we loop, setting up the next definition, which starts
%   with \cs{use_none:n} and a backwards pointer to~|#1| followed by the
%   \meta{next key}~|#7| and so on.
%    \begin{macrocode}
\cs_new_protected:Npn \@@_set_eq_loop:NNnw
    #1#2#3#4 \s_@@ #5#6 \@@_pair:wn #7 \s_@@
  {
    \tl_set:Ne \l_@@_tmp_tl { \exp_not:c { @@ ~ #3 #6 ~ #7 } }
    #2 #1 { \exp_not:n { #4 \s_@@ {#5} } \exp_not:o \l_@@_tmp_tl }
    \use_none:n #6 \@@_set_eq_end:w
    \exp_after:wN \@@_set_eq_loop:NNnw \l_@@_tmp_tl #2 {#3}
    \use_none:n #1 \@@_pair:wn #7 \s_@@
  }
%    \end{macrocode}
%   The end-code picks up what is needed to correctly assign the last token
%   list (the end pointer), which is simply \cs{use_none:n}
%   \cs[no-index]{@@_\meta{prefix}\meta{space}\meta{key_n}}.
%    \begin{macrocode}
\cs_new_protected:Npn \@@_set_eq_end:w
    \exp_after:wN \@@_set_eq_loop:NNnw #1#2#3
    \use_none:n #4#5 \s_@@
  {
    \exp_after:wN #2 \l_@@_tmp_tl { \exp_not:n { \use_none:n #4 } }
  }
%    \end{macrocode}
% \end{macro}
% \end{macro}
% \end{macro}
%
% \begin{macro}{\prop_make_flat:N, \prop_make_flat:c \@@_make_flat:Nn}
%   The only interesting case is when given a linked prop.  Clear the
%   linked property list using \cs{@@_clear:wNNN} with local assignments
%   (it does not matter since we are at the outermost group level, and
%   \cs{cs_set_eq:NN} is very slightly faster than its global version.
%   Then store the contents (expanded preventively by \cs{exp_args:NNf})
%   with an assignment \cs{cs_set_nopar:Npe} that does not perform
%   \pkg{l3debug} checks.
%    \begin{macrocode}
\cs_new_protected:Npn \prop_make_flat:N #1
  {
    \int_compare:nNnTF { \tex_currentgrouplevel:D } = 0
      {
        \@@_if_flat:NTF #1 { }
          { \exp_args:NNf \@@_make_flat:Nn #1 {#1} }
      }
      {
        \msg_error:nnee { prop } { inner-make }
          { \token_to_str:N \prop_make_flat:N } { \token_to_str:N #1 }
      }
  }
\cs_generate_variant:Nn \prop_make_flat:N { c }
\cs_new_protected:Npn \@@_make_flat:Nn #1#2
  {
    \exp_after:wN \@@_clear:wNNN #1 \cs_set_eq:NN \cs_set_nopar:Npe #1
    \cs_set_nopar:Npe #1 { \exp_not:n {#2} }
  }
%    \end{macrocode}
% \end{macro}
%
% \begin{macro}{\prop_make_linked:N, \prop_make_linked:c, \@@_make_linked:Nn}
%   The only interesting case is when given a flat prop.  We expand the
%   contents for later use.  Then \cs{@@_new_linked:N} disregards that
%   previous value of~|#1| and initializes the linked prop.  We can then
%   use an auxiliary \cs{@@_set_eq:wNNNN} underlying
%   \cs{prop_set_eq:NN}, with the prop contents saved as
%   \cs{l_@@_tmp_tl}.  That step is a bit unsafe, as
%   \cs{l_@@_tmp_tl} (really, a flat prop here) is used within
%   \cs{@@_set_eq:wNNNN} itself, but it is in fact expanded early enough
%   to be ok.
%    \begin{macrocode}
\cs_new_protected:Npn \prop_make_linked:N #1
  {
    \int_compare:nNnTF { \tex_currentgrouplevel:D } = 0
      {
        \@@_if_flat:NTF #1
          { \exp_args:NNo \@@_make_linked:Nn #1 {#1} } { }
      }
      {
        \msg_error:nnee { prop } { inner-make }
          { \token_to_str:N \prop_make_linked:N } { \token_to_str:N #1 }
      }
  }
\cs_generate_variant:Nn \prop_make_linked:N { c }
\cs_new_protected:Npn \@@_make_linked:Nn #1#2
  {
    \@@_new_linked:N #1
    \tl_set:Nn \l_@@_tmp_tl {#2}
    \exp_after:wN \@@_set_eq:wNNNN #1
      \cs_set_eq:NN \cs_set_nopar:Npe #1 \l_@@_tmp_tl
  }
%    \end{macrocode}
% \end{macro}
%
% \begin{variable}[tested = m3prop004]{\l_tmpa_prop, \l_tmpb_prop}
% \begin{variable}[tested = m3prop004]{\g_tmpa_prop, \g_tmpb_prop}
%   We can now initialize the scratch variables.
%    \begin{macrocode}
\prop_new:N \l_tmpa_prop
\prop_new:N \l_tmpb_prop
\prop_new:N \g_tmpa_prop
\prop_new:N \g_tmpb_prop
%    \end{macrocode}
% \end{variable}
% \end{variable}
%
% \begin{macro}
%   {
%     \prop_concat:NNN, \prop_concat:ccc,
%     \prop_gconcat:NNN, \prop_gconcat:ccc,
%     \@@_concat:NNNNN, \@@_concat:nNNN
%   }
%   The basic strategy is to copy the first variable into the target,
%   then loop through the second variable, calling
%   \cs[index=prop_put:Nnn]{prop_(g)put:Nnn} on each item.  To avoid
%   running the \pkg{l3debug} scope checks on each of these steps, we
%   use the auxiliaries that underly \cs{prop_set_eq:NN} and
%   \cs{prop_put:Nnn}, whose syntax is a bit unwieldy.
%   We work directly with the target prop |#3| as a scratch space,
%   because copying over from a temporary variable to |#3| would be slow
%   in the linked case.  If |#5| is |#3| itself we have to be careful
%   not to lose the data, and we even take the opportunity to skip the
%   copying step completely.  To keep the correct version of the
%   duplicate keys we use the code underlying \cs{prop_put_if_not_in:Nnn},
%   which involves passing \cs{use_none:nnn} to the auxiliary instead of
%   nothing.
%   There is no need to check for the case where |#3| is equal to~|#4|
%   because in that case \cs[index=prop_set_eq:NN]{prop_(g)set_eq:NN}
%   |#3| |#4| (or rather the underlying auxiliary) is correctly set up
%   to do no needless work.
%    \begin{macrocode}
\cs_new_protected:Npn \prop_concat:NNN
  { \@@_concat:NNNNN \cs_set_eq:NN \cs_set_nopar:Npe }
\cs_generate_variant:Nn \prop_concat:NNN { ccc }
\cs_new_protected:Npn \prop_gconcat:NNN
  { \@@_concat:NNNNN \cs_gset_eq:NN \cs_gset_nopar:Npe }
\cs_generate_variant:Nn \prop_gconcat:NNN { ccc }
\cs_new_protected:Npn \@@_concat:NNNNN #1#2#3#4#5
  {
    \cs_if_eq:NNTF #3 #5
      { \@@_concat:nNNN \use_none:nnn #2 #3 #4 }
      {
        \@@_set_eq:NNNN #1 #2 #3 #4
        \@@_concat:nNNN { } #2 #3 #5
      }
  }
\cs_new_protected:Npn \@@_concat:nNNN #1#2#3#4
  {
    \cs_gset_eq:NN \@@_tmp:w \@@_pair:wn
    \cs_gset_protected:Npn \@@_pair:wn ##1 \s_@@
      { \@@_put:nNNnn {#1} #2 #3 {##1} }
    \exp_last_unbraced:Nf \use_none:nn #4
    \cs_gset_eq:NN \@@_pair:wn \@@_tmp:w
  }
%    \end{macrocode}
% \end{macro}
%
% \begin{macro}
%   {
%     \prop_put_from_keyval:Nn, \prop_put_from_keyval:cn,
%     \prop_gput_from_keyval:Nn, \prop_gput_from_keyval:cn,
%     \@@_from_keyval:nn, \@@_from_keyval:Nnn,
%     \@@_missing_eq:n
%   }
%   The core is a call to \cs{keyval_parse:nnn}, with an error message
%   \cs{@@_missing_eq:n} for entries without~|=|, and a call to (essentially)
%   \cs[index=prop_put:Nnn]{prop_(g)put:Nnn} for valid key--value pairs.
%   To avoid repeated scope checks (and errors) when \pkg{l3debug} is
%   active, we instead use the auxiliary underlying \cs{prop_put:Nnn}.
%   Because blank keys are valid here, in contrast to \pkg{l3keys}, we set and
%   restore \cs{l__kernel_keyval_allow_blank_keys_bool}.
%   The key--value argument may be quite large so we avoid reading it until it
%   is really necessary.
%    \begin{macrocode}
\cs_new_protected:Npn \prop_put_from_keyval:Nn #1
  { \@@_from_keyval:nn { \@@_put:nNNnn { } \cs_set_nopar:Npe #1 } }
\cs_generate_variant:Nn \prop_put_from_keyval:Nn { c }
\cs_new_protected:Npn \prop_gput_from_keyval:Nn #1
  { \@@_from_keyval:nn { \@@_put:nNNnn { } \cs_gset_nopar:Npe #1 } }
\cs_generate_variant:Nn \prop_gput_from_keyval:Nn { c }
\cs_new_protected:Npn \@@_from_keyval:nn
  {
    \bool_if:NTF \l__kernel_keyval_allow_blank_keys_bool
      { \@@_from_keyval:Nnn \c_true_bool }
      { \@@_from_keyval:Nnn \c_false_bool }
  }
\cs_new_protected:Npn \@@_from_keyval:Nnn #1#2#3
  {
    \bool_set_eq:NN \l__kernel_keyval_allow_blank_keys_bool \c_true_bool
    \keyval_parse:nnn \@@_missing_eq:n {#2} {#3}
    \bool_set_eq:NN \l__kernel_keyval_allow_blank_keys_bool #1
  }
\cs_new_protected:Npn \@@_missing_eq:n
  { \msg_error:nnn { prop } { prop-keyval } }
%    \end{macrocode}
% \end{macro}
%
% \begin{macro}
%   {
%     \prop_set_from_keyval:Nn, \prop_set_from_keyval:cn,
%     \prop_gset_from_keyval:Nn, \prop_gset_from_keyval:cn,
%   }
%   Just empty the prop (with the auxiliary underlying
%   \cs{prop_clear:N} to avoid \pkg{l3debug} problems) and push
%   key--value entries using
%   \cs[index=prop_put_from_keyval:Nn]{prop_(g)put_from_keyval:Nn}.
%    \begin{macrocode}
\cs_new_protected:Npn \prop_set_from_keyval:Nn #1
  {
    \@@_clear:NNN \cs_set_eq:NN \cs_set_nopar:Npe #1
    \prop_put_from_keyval:Nn #1
  }
\cs_generate_variant:Nn \prop_set_from_keyval:Nn { c }
\cs_new_protected:Npn \prop_gset_from_keyval:Nn #1
  {
    \@@_clear:NNN \cs_gset_eq:NN \cs_gset_nopar:Npe #1
    \prop_gput_from_keyval:Nn #1
  }
\cs_generate_variant:Nn \prop_gset_from_keyval:Nn { c }
%    \end{macrocode}
% \end{macro}
%
% \begin{macro}
%   {
%     \prop_const_from_keyval:Nn, \prop_const_from_keyval:cn,
%     \prop_const_linked_from_keyval:Nn, \prop_const_linked_from_keyval:cn
%   }
%   For both flat and linked constant props, we create |#1| then use the
%   same auxiliary as for \cs{prop_gput_from_keyval:Nn}.  It is most
%   natural to use the already packaged \cs{prop_gput:Nnn}, but that
%   would mean doing an assignment on a supposedly constant property
%   list.  To avoid errors when \pkg{l3debug} is activated, we use the
%   auxiliary underlying \cs{prop_gput:Nnn}.
%    \begin{macrocode}
\cs_new_protected:Npn \prop_const_from_keyval:Nn #1
  {
    \prop_new:N #1
    \@@_from_keyval:nn { \@@_put:nNNnn { } \cs_gset_nopar:Npe #1 }
  }
\cs_generate_variant:Nn \prop_const_from_keyval:Nn { c }
\cs_new_protected:Npn \prop_const_linked_from_keyval:Nn #1
  {
    \prop_new_linked:N #1
    \@@_from_keyval:nn { \@@_put:nNNnn { } \cs_gset_nopar:Npe #1 }
  }
\cs_generate_variant:Nn \prop_const_linked_from_keyval:Nn { c }
%    \end{macrocode}
% \end{macro}
%
% \subsection{Accessing data in property lists}
%
% Accessing/deleting/adding entries is mostly done by \cs{@@_split:NnTFn}, which
% must be fast because it is used in many \pkg{l3prop} functions.  Its syntax is
% as follows.
% \begin{quote}
%   \cs{@@_split:NnTFn} \meta{property list} \Arg{key} \\
%   ~~\Arg{true code} \Arg{false code} \Arg{link code}
% \end{quote}
% If the \meta{property list} uses the linked data storage, then it runs
% the \meta{link code}, otherwise it does as follows.
%
% It splits the \meta{property list} at the \meta{key}, giving three
% token lists: the \meta{entries before} the \meta{key}, the
% \meta{value} associated with the \meta{key} and the \meta{entries
% after} the \meta{key}.  Both the \meta{entries before} and the
% \meta{entries after} can be empty or consist of some number of
% consecutive entries \cs{@@_pair:wn} \meta{key_i} \cs{s_@@}
% \Arg{value_i}.  If the \meta{key} is present in the \meta{property
% list} then the \meta{true code} is left in the input stream, with
% |#1|, |#2|, and |#3| replaced by the \meta{entries before},
% \meta{value}, and \meta{entries after}.  If the \meta{key} is not
% present in the \meta{property list} then the \meta{false code} is left
% in the input stream.  Only the \meta{true code} is used in the
% replacement text of a macro defined internally, which requires
% |##|~doubling.
%
% \begin{macro}
%   {
%     \@@_split:NnTFn, \@@_split_aux:nNTFn, \@@_split_test:wn,
%     \@@_split_flat:w, \@@_split_linked:w, \@@_split_wrong:Nw
%   }
%   The aim is to distinguish four cases: a flat prop that contains the
%   given \meta{key}, a flat prop that does not contain it, a linked
%   prop, and an invalid prop.  The last case includes those that are
%   set to \tn{relax} by \texttt{c}-expansion, as well as unrelated
%   token list variables since these unfortunately used to
%   \enquote{work} in earlier implementations.
%   In the first three cases we run the \texttt{T}, \texttt{F}, and
%   \texttt{n} arguments, and in the last case we raise an error, set
%   the variable to a known state (empty prop), and run the \texttt{F}
%   code (some conditionals such as \cs{prop_pop:NnNTF} otherwise blow
%   up pretty badly).
%
%   The first distinction between these cases is done by
%   \cs{@@_split_test:wn}, which looks for the argument after \cs{s_@@}.
%   For a flat prop it will be \cs{@@_chk:w}, which leads to running
%   \cs{@@_split_flat:w}, explained below.  For a linked prop it is the
%   prefix, consisting of characters, so we end up running
%   \cs{@@_split_linked:w}, which cleans up and selects the
%   aforementioned \texttt{n} argument.  For invalid props, or rather,
%   variables that do not contain \cs{s_@@}, the argument includes
%   \cs{fi:}, and we end up calling \cs{@@_split_wrong:Nw}, which calls
%   \cs{prop_show:N} to raise a detailed error stating how the variable
%   is wrong.
%
%   Let us return to \cs{@@_split_flat:w}.  This function is defined
%   dynamically as
%   \begin{quote}
%     \cs{cs_set:Npn} \cs{@@_split_flat:w} \cs{@@_split_linked:w} |#1| \\
%     \quad \cs{@@_pair:wn} \meta{key} \cs{s_@@} |#2| \\
%     \quad |#3| \cs{s_@@_mark} |#4| |#5| \cs{s_@@_stop} \\
%     \quad |{| |#4| \Arg{true code} |}|
%   \end{quote}
%   Its job is to seek the \meta{key} in the property list (known to be
%   flat at this stage) by using an argument |#1| delimited essentially
%   by that key.  If indeed the variable contained the \meta{key}, then
%   |#1|~is the \meta{extract_1} before the key--value pair, |#2|~is the
%   \meta{value} associated with the \meta{key}, |#3|~is the
%   \meta{extract_2} after the key--value pair, |#4|~is \cs{use_i:nnn},
%   and we run \cs{use_i:nnn} \Arg{true code} \Arg{false code} \Arg{link
%   code}, selecting the \meta{true code}.  Otherwise, the whole
%   property list together with \cs{s_@@_mark} \cs{use_i:nnn} is taken
%   in as |#1|, then |#2| is some tokens |?| \cs{fi:}
%   \cs{@@_split_wrong:Nw} \meta{variable} that were only useful in the
%   case of invalid props, |#3|~is empty, and most importantly |#4| is
%   \cs{use_ii:nnn}.  This command selects the \meta{false code}.
%
%   Note that we define \cs{@@_split_flat:w} in all cases even though it
%   is only used in the flat case.  Indeed, to avoid taking in the whole
%   property list (which may be large) as an argument more than strictly
%   necessary, we would have to keep the \meta{true code} positioned
%   before the expansion of the prop variable in order to use it in the
%   definition.  The only way to do that is to store it using an
%   assignment so we might as well just perform the assignment that we
%   can actually use in the flat case.
%    \begin{macrocode}
\cs_new_protected:Npn \@@_split:NnTFn #1#2
  {
    \exp_after:wN \@@_split_aux:nNTFn
    \exp_after:wN { \tl_to_str:n {#2} } #1
  }
\cs_new_protected:Npn \@@_split_aux:nNTFn #1#2#3
  {
    \cs_set:Npn \@@_split_flat:w \@@_split_linked:w ##1
      \@@_pair:wn #1 \s_@@ ##2 ##3 \s_@@_mark ##4 ##5 \s_@@_stop
      { ##4 {#3} }
    \exp_after:wN \@@_split_test:wn #2 \s_@@_mark \use_i:nnn
      \@@_pair:wn #1 \s_@@ { ? \fi: \@@_split_wrong:Nw #2 }
      \s_@@_mark \use_ii:nnn
      \s_@@_stop
  }
\cs_new:Npn \@@_split_flat:w { }
\cs_new_protected:Npn \@@_split_test:wn #1 \s_@@ #2
  {
    \if_meaning:w \@@_chk:w #2 \exp_after:wN \@@_split_flat:w \fi:
    \@@_split_linked:w
  }
\cs_new_protected:Npn \@@_split_linked:w #1 \s_@@_stop #2#3 {#3}
\cs_new_protected:Npn \@@_split_wrong:Nw #1#2 \s_@@_stop #3#4
  {
    \prop_show:N #1
    \cs_gset_eq:NN #1 \c_empty_prop
    #3
  }
%    \end{macrocode}
% \end{macro}
%
% \begin{macro}[tested = m3prop002]
%   {
%     \prop_get:NnN, \prop_get:NVN, \prop_get:NvN, \prop_get:NeN,
%     \prop_get:NoN, \prop_get:NxN,
%     \prop_get:cnN, \prop_get:cVN, \prop_get:cvN, \prop_get:ceN,
%     \prop_get:coN, \prop_get:cxN,
%     \prop_get:cnc
%   }
% \begin{macro}[TF, tested = m3prop004]
%   {
%     \prop_get:NnN, \prop_get:NVN, \prop_get:NvN, \prop_get:NeN,
%     \prop_get:NoN, \prop_get:NxN,
%     \prop_get:cnN, \prop_get:cVN, \prop_get:cvN, \prop_get:ceN,
%     \prop_get:coN, \prop_get:cxN,
%     \prop_get:cnc
%   }
% \begin{macro}{\@@_get:NnnTF}
% \begin{macro}[EXP]{\@@_get_linked:w, \@@_get_linked_aux:w}
%   Here we implement both \cs{prop_get:NnN} and its branching version through
%   \cs{@@_get:NnnTF}.  It receives the prop and key, followed by an assignment
%   used when the value is found, \meta{true code} to run after the assignment,
%   and some fall-back \meta{false code} for absent values.  It relies on
%   \cs{@@_split:NnTFn}.  For a flat prop, the first four arguments of
%   \cs{@@_split:NnTFn} are used, and run either the assignment~|#3{##3}| and
%   \meta{true code}~|#4|, or the \meta{false code}~|#5|.
%    \begin{macrocode}
\cs_new_protected:Npn \prop_get:NnN #1#2#3
  {
    \@@_get:NnnTF #1 {#2}
      { \tl_set:Nn #3 } { } { \tl_set:Nn #3 { \q_no_value } }
  }
\cs_generate_variant:Nn \prop_get:NnN { NV , Nv , Ne , c , cV , cv , ce }
\cs_generate_variant:Nn \prop_get:NnN { No , Nx , co , cx }
\cs_generate_variant:Nn \prop_get:NnN { cnc }
\prg_new_protected_conditional:Npnn \prop_get:NnN #1#2#3 { T , F , TF }
  {
    \@@_get:NnnTF #1 {#2}
      { \tl_set:Nn #3 } \prg_return_true: \prg_return_false:
  }
\prg_generate_conditional_variant:Nnn \prop_get:NnN
  { NV , Nv , Ne , c , cV , cv , ce } { T , F , TF }
\prg_generate_conditional_variant:Nnn \prop_get:NnN
  { No , Nx , co , cx } { T , F , TF }
\prg_generate_conditional_variant:Nnn \prop_get:NnN
  { cnc } { T , F , TF }
\cs_new_protected:Npn \@@_get:NnnTF #1#2#3#4#5
  {
    \@@_split:NnTFn #1 {#2}
      { #3 {##2} #4 }
      {#5}
      { \exp_after:wN \@@_get_linked:w #1 {#2} {#3} {#4} {#5} }
  }
%    \end{macrocode}
%   For a linked prop we must work a bit: \cs{@@_get_linked:w} is followed by
%   the expansion of the prop, then by four brace groups: the key~|#4|, the
%   assignment code~|#5|, \meta{true code}~|#6|, and \meta{false code}~|#7|.  If
%   the key is present, its value is stored in the token list
%   \cs[no-index]{@@_}|#2~#4|.  If that token list exists,
%   \cs{@@_get_linked_aux:w} gets called followed by the expansion of that token
%   list and we grab as~|#2| the value associated to that key, which we feed to
%   the assignment code and follow-up code.  If the key is absent the token list
%   can be \tn{undefined} or \tn{relax}.  In both cases \cs{@@_get_linked_aux:w}
%   finds an empty brace group as~|#2|, \cs{use_none:n} as~|#4| and the
%   \meta{false code} as~|#5|.
%   Note that we made \cs{@@_get_linked:w} and subsequent auxiliaries
%   expandable, because they are also used in \cs{prop_item:Nn}.
%    \begin{macrocode}
\cs_new:Npn \@@_get_linked:w
    \@@_flatten:w #1 \s_@@ #2#3#4#5#6#7
  {
    \if_cs_exist:w @@ ~ #2 ~ \tl_to_str:n {#4} \cs_end:
      \exp_after:wN \exp_after:wN \exp_after:wN \@@_get_linked_aux:w
      \cs:w @@ ~ #2 ~ \tl_to_str:n {#4} \exp_after:wN \cs_end:
    \else:
      \exp_after:wN \@@_get_linked_aux:w
    \fi:
    \s_@@_mark {#5} {#6}
    \s_@@ { } \s_@@_mark \use_none:n {#7}
    \s_@@_stop
  }
\cs_new:Npn \@@_get_linked_aux:w
    #1 \s_@@ #2 #3 \s_@@_mark #4 #5 #6 \s_@@_stop { #4 {#2} #5 }
%    \end{macrocode}
% \end{macro}
% \end{macro}
% \end{macro}
% \end{macro}
%
% \begin{macro}[EXP]
%   {
%     \prop_item:Nn, \prop_item:NV, \prop_item:No, \prop_item:Ne,
%     \prop_item:cn, \prop_item:cV, \prop_item:co, \prop_item:ce
%   }
% \begin{macro}[EXP]{\@@_item:nnn}
%   Getting the value corresponding to a key in a flat property list in an
%   expandable fashion simply uses \cs{prop_map_tokens:Nn} to go through
%   the property list.  The auxiliary \cs{@@_item:nnn} receives the
%   search string~|#1|, the key~|#2| and the value~|#3| and returns as
%   appropriate.
%    \begin{macrocode}
\cs_new:Npn \prop_item:Nn #1#2
  {
    \@@_if_flat:NTF #1
      {
        \exp_args:NNo \prop_map_tokens:Nn #1
          {
            \exp_after:wN \@@_item:nnn
            \exp_after:wN { \tl_to_str:n {#2} }
          }
      }
      { \exp_after:wN \@@_get_linked:w #1 {#2} \exp_not:n { } { } }
  }
\cs_new:Npn \@@_item:nnn #1#2#3
  {
    \str_if_eq:eeT {#1} {#2}
      { \prop_map_break:n { \exp_not:n {#3} } }
  }
\cs_generate_variant:Nn \prop_item:Nn { NV , No , Ne , c , cV , co , ce }
%    \end{macrocode}
% \end{macro}
% \end{macro}
%
% \subsection{Removing data from property lists}
%
% \begin{macro}
%   {
%     \@@_pop:NnNNnTF,
%     \@@_pop_linked:wnNNnTF, \@@_pop_linked:NNNn,
%     \@@_pop_linked:w,
%     \@@_pop_linked_prev:w, \@@_pop_linked_next:w
%   }
%   This auxiliary is used by both the \cs[no-index]{prop_pop} family and
%   the \cs[no-index]{prop_remove} family of functions.
%   It receives a \meta{prop} and a \Arg{key}, three assignment functions
%   (\cs{tl_set:Nn} \cs{cs_set_eq:NN} \cs{cs_set_nopar:Npe} or their global
%   versions), then \Arg{code} \Arg{true code} \Arg{false code}.
%
%   For a flat prop, split it.  If the \meta{key} is there, reconstruct the rest
%   of the prop from the two extracts |##2| |##4| and assign using
%   \cs[index=tl_set:Nn]{tl_(g)set:Nn}, then run \meta{code} \Arg{value} with
%   the \meta{value} found, and run the \meta{true code}.  If the \meta{key} is
%   absent, run the \meta{false code}.
%
%   For a linked prop, the removal is done by \cs{@@_pop_linked:wnNNnTF}, which
%   removes the key--value pair from the doubly-linked list and runs its last
%   three arguments \Arg{code} \Arg{true code} \Arg{false code} depending on
%   whether the key--value is found, in the same way as for flat props.
%    \begin{macrocode}
\cs_new_protected:Npn \@@_pop:NnNNnTF #1#2#3#4#5#6#7
  {
    \@@_split:NnTFn #1 {#2}
      {
        #4 #1 { \exp_not:n { \s_@@ \@@_chk:w ##1 ##3 } }
        #5 {##2}
        #6
      }
      {#7}
      {
        \exp_after:wN \@@_pop_linked:wnNNnTF #1 {#2}
          #3 #4 {#5} {#6} {#7}
      }
  }
%    \end{macrocode}
%   The next auxiliary \cs{@@_pop_linked:wnNNnTF}, together with the |NNNn|
%   auxiliary, checks if the key is present in the \meta{linked prop}, then the
%   corresponding value (if present) is passed as a braced argument to the
%   \meta{code} and the \meta{true code} or \meta{false code} is run as
%   appropriate.  Before that, there are also three assignments: the token lists
%   for the previous key and next key are made to point to each other, cf.\
%   \cs{@@_pop_linked:w}, and the token list for the given key is made
%   undefined.
%    \begin{macrocode}
\cs_new_protected:Npn \@@_pop_linked:wnNNnTF
    \@@_flatten:w #1 \s_@@ #2#3#4#5#6#7
  {
    \if_cs_exist:w @@ ~ #2 ~ \tl_to_str:n {#4} \cs_end:
      \exp_after:wN \@@_pop_linked:NNNn
      \cs:w @@ ~ #2 ~ \tl_to_str:n {#4} \cs_end:
      #5 #6 {#7}
    \else:
      \exp_after:wN \use_iii:nnn
    \fi:
    \use_i:nn
  }
\cs_new_protected:Npn \@@_pop_linked:NNNn #1#2#3#4
  {
    \if_meaning:w \scan_stop: #1
      \exp_after:wN \exp_after:wN \exp_after:wN \use_iii:nnn
    \else:
      \exp_after:wN \@@_pop_linked:w #1 #1 #2 #3 {#4}
    \fi:
  }
\cs_new_protected:Npn \@@_pop_linked:w
    \use_none:n #1#2 \s_@@ #3#4#5#6#7#8
  {
    #6 #5 \tex_undefined:D
    #7 #1
      {
        \exp_after:wN \@@_pop_linked_prev:w #1
        \exp_not:N #4
      }
    #7 #4
      {
        \exp_not:n { \use_none:n #1 }
        \exp_not:f { \exp_after:wN \@@_pop_linked_next:w #4 }
      }
    #8 {#3}
  }
\cs_new:Npn \@@_pop_linked_prev:w #1 \s_@@ #2#3
  { \exp_not:n { #1 \s_@@ {#2} } }
\cs_new:Npn \@@_pop_linked_next:w \use_none:n #1 { \exp_stop_f: }
%    \end{macrocode}
% \end{macro}
%
% \begin{macro}[tested = m3prop002]
%   {
%     \prop_remove:Nn, \prop_remove:NV, \prop_remove:Ne,
%     \prop_remove:cn, \prop_remove:cV, \prop_remove:ce
%   }
% \begin{macro}[tested = m3prop002]
%   {
%     \prop_gremove:Nn, \prop_gremove:NV, \prop_gremove:Ne,
%     \prop_gremove:cn, \prop_gremove:cV, \prop_gremove:ce
%   }
%   Deleting from a property relies on \cs{@@_pop:NnNNnTF}.  The three
%   assignment functions are suitably local or global.  The last three arguments
%   are \cs{use_none:n} and two empty brace groups: if the key is found we get
%   \cs{use_none:n} \Arg{key} \meta{empty}, which expands to nothing, and
%   otherwise we just get \meta{empty}.  The auxiliary takes care of actually
%   removing the entry from the prop.
%    \begin{macrocode}
\cs_new_protected:Npn \prop_remove:Nn #1#2
  {
    \@@_pop:NnNNnTF #1 {#2}
      \cs_set_eq:NN \cs_set_nopar:Npe
      \use_none:n { } { }
  }
\cs_new_protected:Npn \prop_gremove:Nn #1#2
  {
    \@@_pop:NnNNnTF #1 {#2}
      \cs_gset_eq:NN \cs_gset_nopar:Npe
      \use_none:n { } { }
  }
\cs_generate_variant:Nn \prop_remove:Nn  { NV , Ne , c , cV , ce }
\cs_generate_variant:Nn \prop_gremove:Nn { NV , Ne , c , cV , ce }
%    \end{macrocode}
% \end{macro}
% \end{macro}
%
% \begin{macro}[tested = m3prop002]
%   {
%     \prop_pop:NnN, \prop_pop:NVN, \prop_pop:NoN,
%     \prop_pop:cnN, \prop_pop:cVN, \prop_pop:coN,
%     \prop_gpop:NnN, \prop_gpop:NVN, \prop_gpop:NoN,
%     \prop_gpop:cnN, \prop_gpop:cVN, \prop_gpop:coN,
%   }
% \begin{macro}[TF, tested = m3prop004]
%   {
%     \prop_pop:NnN, \prop_pop:NVN, \prop_pop:NoN,
%     \prop_pop:cnN, \prop_pop:cVN, \prop_pop:coN,
%     \prop_gpop:NnN, \prop_gpop:NVN, \prop_gpop:NoN,
%     \prop_gpop:cnN, \prop_gpop:cVN, \prop_gpop:coN,
%   }
%   Popping a value is almost the same, but the value found is kept.
%   For the non-branching version, we additionally set the target token list to
%   \cs{q_no_value}, while for the branching version we must produce
%   \cs{prg_return_true:} or \cs{prg_return_false:}.
%    \begin{macrocode}
\cs_new_protected:Npn \prop_pop:NnN #1#2#3
  {
    \@@_pop:NnNNnTF #1 {#2}
      \cs_set_eq:NN \cs_set_nopar:Npe
      { \tl_set:Nn #3 } { } { \tl_set:Nn #3 { \q_no_value } }
  }
\cs_new_protected:Npn \prop_gpop:NnN #1#2#3
  {
    \@@_pop:NnNNnTF #1 {#2}
      \cs_gset_eq:NN \cs_gset_nopar:Npe
      { \tl_set:Nn #3 } { } { \tl_set:Nn #3 { \q_no_value } }
  }
\cs_generate_variant:Nn \prop_pop:NnN  {     NV , No }
\cs_generate_variant:Nn \prop_pop:NnN  { c , cV , co }
\cs_generate_variant:Nn \prop_gpop:NnN {     NV , No }
\cs_generate_variant:Nn \prop_gpop:NnN { c , cV , co }
\prg_new_protected_conditional:Npnn \prop_pop:NnN #1#2#3 { T , F , TF }
  {
    \@@_pop:NnNNnTF #1 {#2}
      \cs_set_eq:NN \cs_set_nopar:Npe
      { \tl_set:Nn #3 } \prg_return_true: \prg_return_false:
  }
\prg_new_protected_conditional:Npnn \prop_gpop:NnN #1#2#3 { T , F , TF }
  {
    \@@_pop:NnNNnTF #1 {#2}
      \cs_gset_eq:NN \cs_gset_nopar:Npe
      { \tl_set:Nn #3 } \prg_return_true: \prg_return_false:
  }
\prg_generate_conditional_variant:Nnn \prop_pop:NnN
  { NV , No , c , cV , co } { T , F , TF }
\prg_generate_conditional_variant:Nnn \prop_gpop:NnN
  { NV , No , c , cV , co } { T , F , TF }
%    \end{macrocode}
% \end{macro}
% \end{macro}
%
% \subsection{Adding data to property lists}
%
% \begin{macro}[tested = m3prop002]
%   {
%     \prop_put:Nnn,  \prop_put:NnV,  \prop_put:Nnv,  \prop_put:Nne,
%     \prop_put:NVn,  \prop_put:NVV,  \prop_put:NVv,  \prop_put:NVe,
%     \prop_put:Nvn,  \prop_put:NvV,  \prop_put:Nvv,  \prop_put:Nve,
%     \prop_put:Nen,  \prop_put:NeV,  \prop_put:Nev,  \prop_put:Nee,
%     \prop_put:Nno,  \prop_put:Non,  \prop_put:Noo,
%     \prop_put:Nnx,  \prop_put:NVx,  \prop_put:NxV,  \prop_put:Nxx,
%     \prop_put:cnn,  \prop_put:cnV,  \prop_put:cnv,  \prop_put:cne,
%     \prop_put:cVn,  \prop_put:cVV,  \prop_put:cVv,  \prop_put:cVe,
%     \prop_put:cvn,  \prop_put:cvV,  \prop_put:cvv,  \prop_put:cve,
%     \prop_put:cen,  \prop_put:ceV,  \prop_put:cev,  \prop_put:cee,
%     \prop_put:cno,  \prop_put:con,  \prop_put:coo,
%     \prop_put:cnx,  \prop_put:cVx,  \prop_put:cxV,  \prop_put:cxx
%   }
% \begin{macro}[tested = m3prop002]
%   {
%     \prop_gput:Nnn, \prop_gput:NnV, \prop_gput:Nnv, \prop_gput:Nne,
%     \prop_gput:NVn, \prop_gput:NVV, \prop_gput:NVv, \prop_gput:NVe,
%     \prop_gput:Nvn, \prop_gput:NvV, \prop_gput:Nvv, \prop_gput:Nve,
%     \prop_gput:Nen, \prop_gput:NeV, \prop_gput:Nev, \prop_gput:Nee,
%     \prop_gput:Nno, \prop_gput:Non, \prop_gput:Noo,
%     \prop_gput:Nnx, \prop_gput:NVx, \prop_gput:NxV, \prop_gput:Nxx,
%     \prop_gput:cnn, \prop_gput:cnV, \prop_gput:cnv, \prop_gput:cne,
%     \prop_gput:cVn, \prop_gput:cVV, \prop_gput:cVv, \prop_gput:cVe,
%     \prop_gput:cvn, \prop_gput:cvV, \prop_gput:cvv, \prop_gput:cve,
%     \prop_gput:cen, \prop_gput:ceV, \prop_gput:cev, \prop_gput:cee,
%     \prop_gput:cno, \prop_gput:con, \prop_gput:coo,
%     \prop_gput:cnx, \prop_gput:cVx, \prop_gput:cxV, \prop_gput:cxx
%   }
% \begin{macro}[tested = m3prop002]
%   {
%     \prop_put_if_not_in:Nnn, \prop_put_if_not_in:NnV, \prop_put_if_not_in:Nnv, \prop_put_if_not_in:Nne,
%     \prop_put_if_not_in:NVn, \prop_put_if_not_in:NVV, \prop_put_if_not_in:NVv, \prop_put_if_not_in:NVe,
%     \prop_put_if_not_in:Nvn, \prop_put_if_not_in:NvV, \prop_put_if_not_in:Nvv, \prop_put_if_not_in:Nve,
%     \prop_put_if_not_in:Nen, \prop_put_if_not_in:NeV, \prop_put_if_not_in:Nev, \prop_put_if_not_in:Nee,
%     \prop_put_if_not_in:cnn, \prop_put_if_not_in:cnV, \prop_put_if_not_in:cnv, \prop_put_if_not_in:cne,
%     \prop_put_if_not_in:cVn, \prop_put_if_not_in:cVV, \prop_put_if_not_in:cVv, \prop_put_if_not_in:cVe,
%     \prop_put_if_not_in:cvn, \prop_put_if_not_in:cvV, \prop_put_if_not_in:cvv, \prop_put_if_not_in:cve,
%     \prop_put_if_not_in:cen, \prop_put_if_not_in:ceV, \prop_put_if_not_in:cev, \prop_put_if_not_in:cee,
%     \prop_gput_if_not_in:Nnn, \prop_gput_if_not_in:NnV, \prop_gput_if_not_in:Nnv, \prop_gput_if_not_in:Nne,
%     \prop_gput_if_not_in:NVn, \prop_gput_if_not_in:NVV, \prop_gput_if_not_in:NVv, \prop_gput_if_not_in:NVe,
%     \prop_gput_if_not_in:Nvn, \prop_gput_if_not_in:NvV, \prop_gput_if_not_in:Nvv, \prop_gput_if_not_in:Nve,
%     \prop_gput_if_not_in:Nen, \prop_gput_if_not_in:NeV, \prop_gput_if_not_in:Nev, \prop_gput_if_not_in:Nee,
%     \prop_gput_if_not_in:cnn, \prop_gput_if_not_in:cnV, \prop_gput_if_not_in:cnv, \prop_gput_if_not_in:cne,
%     \prop_gput_if_not_in:cVn, \prop_gput_if_not_in:cVV, \prop_gput_if_not_in:cVv, \prop_gput_if_not_in:cVe,
%     \prop_gput_if_not_in:cvn, \prop_gput_if_not_in:cvV, \prop_gput_if_not_in:cvv, \prop_gput_if_not_in:cve,
%     \prop_gput_if_not_in:cen, \prop_gput_if_not_in:ceV, \prop_gput_if_not_in:cev, \prop_gput_if_not_in:cee
%   }
% \begin{macro}[tested = m3prop002]
%   {
%     \@@_put:NNNnn, \@@_put_linked:wnNN,
%     \@@_put_linked:NNNN, \@@_put_linked_old:w,
%     \@@_put_linked_new:w
%   }
%   All of the \cs[no-index]{prop_(g)put(_if_new):Nnn} functions are
%   based on the same auxiliary, which receives \meta{code} and an
%   \enquote{assignment}, followed by \meta{prop} \Arg{key} \Arg{new
%   value}.  The
%   assignment \cs[index=cs_set_nopar:Npe]{cs_(g)set_nopar:Npe} is the
%   primitive assignment without any checking: in the case of linked
%   props it is applied to individual
%   pieces of the linked prop, which are typically not yet defined.
%   Debugging the scope of the variable is done at a higher level by
%   letting \pkg{l3debug} change \cs{prop_put:Nnn} and friends.  This
%   allows other \pkg{l3prop} commands to directly call the underlying
%   auxiliary to skip this checking step and avoid getting multiple
%   error messages for the same error.
%   The \meta{code} (empty for |put| and \cs{use_none:nnn} for
%   |put_if_not_in|) is placed before the assignment in cases where the key
%   is already present, in order to suppress the assignment in the
%   |put_if_not_in| case.
%    \begin{macrocode}
\cs_new_protected:Npn \prop_put:Nnn
  { \@@_put:nNNnn { } \cs_set_nopar:Npe }
\cs_new_protected:Npn \prop_gput:Nnn
  { \@@_put:nNNnn { } \cs_gset_nopar:Npe }
\cs_new_protected:Npn \prop_put_if_not_in:Nnn
  { \@@_put:nNNnn \use_none:nnn \cs_set_nopar:Npe }
\cs_new_protected:Npn \prop_gput_if_not_in:Nnn
  { \@@_put:nNNnn \use_none:nnn \cs_gset_nopar:Npe }
\cs_generate_variant:Nn \prop_put:Nnn
  {
         NnV , Nnv , Nne , NV , NVV , NVv , NVe ,
    Nv , NvV , Nvv , Nve , Ne , NeV , Nev , Nee
  }
\cs_generate_variant:Nn \prop_put:Nnn
  { Nno , No , Noo , Nnx , NVx , NxV , Nxx }
\cs_generate_variant:Nn \prop_put:Nnn
  {
    c  , cnV , cnv , cne , cV , cVV , cVv , cVe ,
    cv , cvV , cvv , cve , ce , ceV , cev , cee
  }
\cs_generate_variant:Nn \prop_put:Nnn
  { cno , co , coo , cnx , cVx , cxV , cxx }
\cs_generate_variant:Nn \prop_gput:Nnn
  {
         NnV , Nnv , Nne , NV , NVV , NVv , NVe ,
    Nv , NvV , Nvv , Nve , Ne , NeV , Nev , Nee
  }
\cs_generate_variant:Nn \prop_gput:Nnn
  { Nno , No , Noo , Nnx , NVx , NxV , Nxx }
\cs_generate_variant:Nn \prop_gput:Nnn
  {
    c  , cnV , cnv , cne , cV , cVV , cVv , cVe ,
    cv , cvV , cvv , cve , ce , ceV , cev , cee
  }
\cs_generate_variant:Nn \prop_gput:Nnn
  { cno , co , coo , cnx , cVx , cxV , cxx }
\cs_generate_variant:Nn \prop_put_if_not_in:Nnn
  {
         NnV , Nnv , Nne , NV , NVV , NVv , NVe ,
    Nv , NvV , Nvv , Nve , Ne , NeV , Nev , Nee ,
    c  , cnV , cnv , cne , cV , cVV , cVv , cVe ,
    cv , cvV , cvv , cve , ce , ceV , cev , cee
  }
\cs_generate_variant:Nn \prop_gput_if_not_in:Nnn
  {
         NnV , Nnv , Nne , NV , NVV , NVv , NVe ,
    Nv , NvV , Nvv , Nve , Ne , NeV , Nev , Nee ,
    c  , cnV , cnv , cne , cV , cVV , cVv , cVe ,
    cv , cvV , cvv , cve , ce , ceV , cev , cee
  }
%    \end{macrocode}
%   Since the true branch of \cs{@@_split:NnTFn} is used as the
%   replacement text of an internal macro, and since the \meta{key} and
%   new \meta{value} may contain arbitrary tokens, it is not safe to
%   include them in the argument of \cs{@@_split:NnTFn}.  We thus start
%   by storing in \cs{l_@@_tmp_tl} tokens which (after
%   \texttt{x}-expansion) encode the key--value pair.  This variable can
%   safely be used in \cs{@@_split:NnTFn}.  For a flat prop, if the
%   \meta{key} was absent, append the new key--value to the list;
%   otherwise concatenate the extracts |##2| and |##4| with the new
%   key--value pair \cs{l_@@_tmp_tl}.  The updated entry is placed
%   at the same spot as the original \meta{key} in the property list,
%   preserving the order of entries.  For a linked prop, call
%   \cs{@@_put_linked:wnNN}, which constructs the control sequence in
%   which we will place the new value.  If it matches \cs{scan_stop:}
%   then the key was not yet there and we add it using
%   \cs{@@_put_linked_new:w}, otherwise it was already there and we use
%   \cs{@@_put_linked_old:w}.
%    \begin{macrocode}
\cs_new_protected:Npn \@@_put:nNNnn #1#2#3#4#5
  {
    \tl_set:Nn \l_@@_tmp_tl
      {
        \exp_not:N \@@_pair:wn \tl_to_str:n {#4}
        \s_@@ { \exp_not:n {#5} }
      }
    \@@_split:NnTFn #3 {#4}
      {
        #1 #2 #3
          {
            \s_@@ \@@_chk:w \exp_not:n {##1}
            \l_@@_tmp_tl \exp_not:n {##3}
          }
      }
      { #2 #3 { \exp_not:o {#3} \l_@@_tmp_tl } }
      { \exp_after:wN \@@_put_linked:wnnN #3 {#4} {#1} #2 }
  }
\cs_new_protected:Npn \@@_put_linked:wnnN
    \@@_flatten:w #1 \s_@@ #2#3#4
  {
    \exp_after:wN \@@_put_linked:NNnN
    \cs:w @@ ~ #2 ~ \tl_to_str:n {#4} \cs_end:
    #1
  }
\cs_new_protected:Npn \@@_put_linked:NNnN #1#2#3#4
  {
    \if_meaning:w \scan_stop: #1
      \exp_after:wN \@@_put_linked_new:w #2 #1 #2 #4
    \else:
      \exp_after:wN \@@_put_linked_old:w #1 { #3 #4 #1 }
    \fi:
  }
%    \end{macrocode}
%   To add a new entry, \cs{@@_put_linked_new:w} receives the expansion
%   of the end-pointer, namely \cs{use_none:n} \meta{last key pointer},
%   followed by the new key pointer~|#2|, the end pointer~|#3|, and an
%   assignment function~|#4|.  Set up the doubly-linked list in the
%   order |#1|, |#2|, |#3|, placing the key--value pair
%   \cs{l_@@_tmp_tl} in~|#2|.  To replace an old entry,
%   \cs{@@_put_linked_old:w} receives the expansion of that entry, and
%   it reassigns it (|#5|) using the assignment~|#6|, by simply
%   replacing the payload |#2| \cs{s_@@} |#3| by \cs{l_@@_tmp_tl}.
%    \begin{macrocode}
\cs_new_protected:Npn \@@_put_linked_new:w
    \use_none:n #1#2#3#4
  {
    #4 #1
      {
        \exp_after:wN \@@_pop_linked_prev:w #1
        \exp_not:N #2
      }
    #4 #2
      {
        \exp_not:n { \use_none:n #1 }
        \l_@@_tmp_tl
        \exp_not:N #3
      }
    #4 #3 { \exp_not:n { \use_none:n #2 } }
  }
\cs_new_protected:Npn \@@_put_linked_old:w
    \use_none:n #1#2 \s_@@ #3#4#5
  {
    #5
      {
        \exp_not:n { \use_none:n #1 }
        \l_@@_tmp_tl
        \exp_not:N #4
      }
  }
%    \end{macrocode}
% \end{macro}
% \end{macro}
% \end{macro}
% \end{macro}
%
% \subsection{Property list conditionals}
%
% \begin{macro}[pTF, tested = m3prop004]{\prop_if_exist:N, \prop_if_exist:c}
%   Copies of the \texttt{cs} functions defined in \pkg{l3basics}.
%    \begin{macrocode}
\prg_new_eq_conditional:NNn \prop_if_exist:N \cs_if_exist:N
  { TF , T , F , p }
\prg_new_eq_conditional:NNn \prop_if_exist:c \cs_if_exist:c
  { TF , T , F , p }
%    \end{macrocode}
% \end{macro}
%
% \begin{macro}[pTF, tested = m3prop003]{\prop_if_empty:N, \prop_if_empty:c}
% \begin{macro}{\@@_if_empty_return:w}
%   A flat property list is empty if it matches \cs{c_empty_prop}.  A linked
%   property list is empty if its second token (the end pointer) and last token
%   (the first key pointer) are equal.  There cannot be false positives because
%   the end pointer takes the form \cs{use_none:n} \meta{pointer} while the
%   other pointers have more elaborate structure.  The subtle code branch here
%   is when a non-empty flat property list is given: then \cs{@@_if_empty:w}
%   reads the whole property list as~|#1| and |#2|, |#3|, |#4| are |2|, |3|,
%   |4|, respectively.
%    \begin{macrocode}
\prg_new_conditional:Npnn \prop_if_empty:N #1 { p , T , F , TF }
  {
    \if_meaning:w #1 \c_empty_prop
      \prg_return_true:
    \else:
      \exp_after:wN \@@_if_empty_return:w #1
        \@@_flatten:w 2 \s_@@ 34 \s_@@_stop
    \fi:
  }
\cs_new:Npn \@@_if_empty_return:w
    #1 \@@_flatten:w #2 \s_@@ #3#4#5 \s_@@_stop
  {
    \if_meaning:w #2 #4
      \prg_return_true:
    \else:
      \prg_return_false:
    \fi:
  }
\prg_generate_conditional_variant:Nnn \prop_if_empty:N
  { c } { p , T , F , TF }
%    \end{macrocode}
% \end{macro}
% \end{macro}
%
% \begin{macro}[pTF, tested = m3prop003]
%   {
%     \prop_if_in:Nn, \prop_if_in:NV, \prop_if_in:Ne, \prop_if_in:No,
%     \prop_if_in:cn, \prop_if_in:cV, \prop_if_in:ce, \prop_if_in:co
%   }
% \begin{macro}[EXP]{\@@_if_in_flat:nnn}
%   For a linked prop, use \cs{@@_get_linked:w} to look up whether the control
%   sequence constructed from the prefix and the sought-after key exists; this
%   auxiliary calls \cs{use_none:n} \Arg{value} \cs{prg_return_true:} if the key
%   is found, and otherwise \cs{prg_return_false:}.  For a flat prop, testing
%   expandably if a key is there requires to go through the key--value pairs one
%   by one.  This is rather slow, and a faster test would be
%   \begin{verbatim}
%     \prg_new_protected_conditional:Npnn \prop_if_in:Nn #1 #2
%       {
%         \@@_split:NnTFn #1 {#2}
%           { \prg_return_true: }
%           { \prg_return_false: }
%           { ... }
%       }
%   \end{verbatim}
%   but \cs{@@_split:NnTFn} is non-expandable.
%   Instead, we use \cs{prop_map_tokens:Nn} to compare the search key to
%   each key in turn using \cs{str_if_eq:ee}, which is expandable.
%    \begin{macrocode}
\prg_new_conditional:Npnn \prop_if_in:Nn #1#2 { p , T , F , TF }
  {
    \@@_if_flat:NTF #1
      {
        \exp_after:wN \prop_map_tokens:Nn \exp_after:wN #1
          {
            \exp_after:wN \@@_if_in_flat:nnn
            \exp_after:wN { \tl_to_str:n {#2} }
          }
        \prg_return_false:
      }
      {
        \exp_after:wN \@@_get_linked:w #1 {#2}
        \use_none:n \prg_return_true: \prg_return_false:
      }
  }
\cs_new:Npn \@@_if_in_flat:nnn #1#2#3
  {
    \str_if_eq:eeT {#1} {#2}
      { \prop_map_break:n { \use_i:nn \prg_return_true: } }
  }
\prg_generate_conditional_variant:Nnn \prop_if_in:Nn
  { NV , Ne , No , c , cV , ce , co } { p , T , F , TF }
%    \end{macrocode}
% \end{macro}
% \end{macro}
%
% \subsection{Mapping over property lists}
%
% \begin{macro}[tested = m3prop003]
%   {
%     \prop_map_function:NN, \prop_map_function:Nc,
%     \prop_map_function:cN, \prop_map_function:cc
%   }
% \begin{macro}{\@@_map_function:Nw}
%   We first |f|-expand to flatten~|#1| in case it was a linked list.
%   The \cs{use_i:nnn} removes the leading \cs{s_@@} \cs{@@_chk:w} of
%   the flattened prop.
%   The even-numbered arguments of \cs{@@_map_function:Nw} are keys,
%   hence have string catcodes, except at the end where they are
%   \cs{fi:} \cs{prop_map_break:}.  The \cs{fi:} ends the \cs{if_false:}
%   |#|\meta{even} \cs{fi:} construction and we jump out of the loop.
%   No need for any quark test.
%    \begin{macrocode}
\cs_new:Npn \prop_map_function:NN #1#2
  {
    \exp_last_unbraced:Nnf
      \use_i:nnn { \@@_map_function:Nw #2 } #1
    \@@_pair:wn \fi: \prop_map_break: \s_@@ { }
    \@@_pair:wn \fi: \prop_map_break: \s_@@ { }
    \@@_pair:wn \fi: \prop_map_break: \s_@@ { }
    \@@_pair:wn \fi: \prop_map_break: \s_@@ { }
    \prg_break_point:Nn \prop_map_break: { }
  }
\cs_new:Npn \@@_map_function:Nw #1
    \@@_pair:wn #2 \s_@@ #3
    \@@_pair:wn #4 \s_@@ #5
    \@@_pair:wn #6 \s_@@ #7
    \@@_pair:wn #8 \s_@@ #9
  {
    \if_false: #2 \fi: #1 {#2} {#3}
    \if_false: #4 \fi: #1 {#4} {#5}
    \if_false: #6 \fi: #1 {#6} {#7}
    \if_false: #8 \fi: #1 {#8} {#9}
    \@@_map_function:Nw #1
  }
\cs_generate_variant:Nn \prop_map_function:NN { Nc , c , cc }
%    \end{macrocode}
% \end{macro}
% \end{macro}
%
% \begin{macro}[tested = m3prop003]{\prop_map_inline:Nn, \prop_map_inline:cn}
%   Mapping in line requires a nesting level counter.  Store the current
%   definition of \cs{@@_pair:wn}, and define it anew.  At the end of
%   the loop, revert to the earlier definition.  Note that besides pairs
%   of the form \cs{@@_pair:wn} \meta{key} \cs{s_@@} \Arg{value}, there
%   are a leading and a trailing tokens, but both are equal to
%   \cs{scan_stop:}, hence have no effect in such inline mapping.
%   Such \cs{scan_stop:} could have affected ligatures if they appeared
%   during the mapping.
%    \begin{macrocode}
\cs_new_protected:Npn \prop_map_inline:Nn #1#2
  {
    \cs_gset_eq:cN
      { @@_map_ \int_use:N \g__kernel_prg_map_int :wn } \@@_pair:wn
    \int_gincr:N \g__kernel_prg_map_int
    \cs_gset_protected:Npn \@@_pair:wn ##1 \s_@@ ##2 {#2}
    \exp_last_unbraced:Nf \use_none:nn #1
    \prg_break_point:Nn \prop_map_break:
      {
        \int_gdecr:N \g__kernel_prg_map_int
        \cs_gset_eq:Nc \@@_pair:wn
          { @@_map_ \int_use:N \g__kernel_prg_map_int :wn }
      }
  }
\cs_generate_variant:Nn \prop_map_inline:Nn { c }
%    \end{macrocode}
% \end{macro}
%
% \begin{macro}[rEXP]{\prop_map_tokens:Nn, \prop_map_tokens:cn}
% \begin{macro}{\@@_map_tokens:nw}
%   The mapping is very similar to \cs{prop_map_function:NN}.
%   The odd construction
%   |\use:n {#1}| allows |#1| to contain any token without interfering
%   with \cs{prop_map_break:}.  The loop stops when the \meta{key}
%   between \cs{@@_pair:wn} and \cs{s_@@} is \cs{fi:}
%   \cs{prop_map_break:} instead of being a string.
%    \begin{macrocode}
\cs_new:Npn \prop_map_tokens:Nn #1#2
  {
    \exp_last_unbraced:Nnf
      \use_i:nnn { \@@_map_tokens:nw {#2} } #1
    \@@_pair:wn \fi: \prop_map_break: \s_@@ { }
    \@@_pair:wn \fi: \prop_map_break: \s_@@ { }
    \@@_pair:wn \fi: \prop_map_break: \s_@@ { }
    \@@_pair:wn \fi: \prop_map_break: \s_@@ { }
    \prg_break_point:Nn \prop_map_break: { }
  }
\cs_new:Npn \@@_map_tokens:nw #1
    \@@_pair:wn #2 \s_@@ #3
    \@@_pair:wn #4 \s_@@ #5
    \@@_pair:wn #6 \s_@@ #7
    \@@_pair:wn #8 \s_@@ #9
  {
    \if_false: #2 \fi: \use:n {#1} {#2} {#3}
    \if_false: #4 \fi: \use:n {#1} {#4} {#5}
    \if_false: #6 \fi: \use:n {#1} {#6} {#7}
    \if_false: #8 \fi: \use:n {#1} {#8} {#9}
    \@@_map_tokens:nw {#1}
  }
\cs_generate_variant:Nn \prop_map_tokens:Nn { c }
%    \end{macrocode}
% \end{macro}
% \end{macro}
%
%
% \begin{macro}[tested = m3prop003]{\prop_map_break:}
% \begin{macro}[tested = m3prop003]{\prop_map_break:n}
%   The break statements are based on the general \cs{prg_map_break:Nn}.
%    \begin{macrocode}
\cs_new:Npn \prop_map_break:
  { \prg_map_break:Nn \prop_map_break: { } }
\cs_new:Npn \prop_map_break:n
  { \prg_map_break:Nn \prop_map_break: }
%    \end{macrocode}
% \end{macro}
% \end{macro}
%
% \subsection{Uses of mapping over property lists}
%
% \begin{macro}[EXP]{\prop_count:N, \prop_count:c}
% \begin{macro}[EXP]{\@@_count:nn}
%   Counting the key--value pairs in a property list is done using the
%   same approach as for other count functions: turn each entry into a
%   \texttt{+1} then use integer evaluation to actually do the
%   mathematics.
%    \begin{macrocode}
\cs_new:Npn \prop_count:N #1
  {
    \int_eval:n
      {
        0
        \prop_map_function:NN #1 \@@_count:nn
      }
  }
\cs_new:Npn \@@_count:nn #1#2 { + 1 }
\cs_generate_variant:Nn \prop_count:N { c }
%    \end{macrocode}
% \end{macro}
% \end{macro}
%
% \begin{macro}[EXP]{\prop_to_keyval:N}
% \begin{macro}[EXP]
%   {\@@_to_keyval_exp_after:wN, \@@_to_keyval:nn, \@@_to_keyval:nnw}
%   Each property name and value pair will be returned in the form
%   \verb*| |\marg{name}\verb*|= |\marg{value}. As one of the main use cases for
%   this macro is to pass the \meta{property list} on to a key--value parser, we
%   have to make sure that the behavior is as good as possible. Using a space
%   before the opening brace we get the correct brace stripping behavior for
%   most of the key--value parsers available in \LaTeX{}.
%   Iterate over the
%   \meta{property list} and remove the leading comma afterwards. Only the value
%   has to be protected in \cs{__kernel_exp_not:w} as the property name is
%   always a string. After the loop the leading comma is removed by
%   \cs{use_none:n} and afterwards \cs{__kernel_exp_not:w} eventually finds the
%   opening brace of its argument.
%    \begin{macrocode}
\cs_new:Npn \prop_to_keyval:N #1
  {
    \__kernel_exp_not:w
      \prop_if_empty:NTF #1
        { {} }
        {
          \exp_after:wN \exp_after:wN \exp_after:wN
          {
            \tex_expanded:D
              {
                \exp_not:N \use_none:n
                \prop_map_function:NN #1 \@@_to_keyval:nn
              }
          }
        }
  }
\cs_new:Npn \@@_to_keyval:nn #1#2
  { , ~ {#1} =~ { \__kernel_exp_not:w {#2} } }
%    \end{macrocode}
% \end{macro}
% \end{macro}
%
% \subsection{Viewing property lists}
%
% \begin{macro}[tested = m3show001]
%   {\prop_show:N, \prop_show:c, \prop_log:N, \prop_log:c}
% \begin{macro}{\@@_show:NN, \@@_show_finally:NNn, \@@_show_prepare:w, \@@_show_loop:NNw, \@@_show_bad_name:NNN, \@@_show_end:NNN, \@@_show_loop_key:wNNN}
% \begin{macro}[rEXP]{\@@_show_flat:w, \@@_show_linked:w}
%   Experience shows one source of problems is very hard to debug: when
%   a data structure such as a |seq| or~|prop| gets corrupted.  In the
%   past, \cs{prop_show:N} would in some cases happily show items of
%   such a |prop|, even though other more demanding \pkg{l3prop}
%   functions would choke.  It is thus best to make \cs{prop_show:N}
%   check very thoroughly the structure and flag issues, even though
%   that is very painful for linked props.  Throughout the code below,
%   we strive to remain as safe as possible, but in the explanations we
%   only state what the arguments are when the prop is correctly formed,
%   rather than saying at every step that various arguments can be
%   arbitrary junk, made safe by using \cs{tl_to_str:n} generously.
%
%   The general \cs{__kernel_chk_tl_type:NnnT} checks that its first
%   argument is a token list, and if it is, then it \texttt{e}-expands
%   its second argument and compares with the contents of its first
%   argument.  Thus, within this \texttt{e}-expansion it is safe to use
%   \cs{@@_if_flat:NTF} to check if the prop is flat or linked.  In the
%   flat case we simply reconstruct the expected structure using
%   \cs{@@_show_flat:w}, which loops through the prop and correctly
%   turns all keys to strings for instance.  In the linked case, we use
%   \cs{@@_show_linked:w}, which ensures the form \cs{@@_flatten:w}
%   \cs[no-index]{@@~\meta{prefix}} \cs{s_@@} \Arg{prefix} \meta{rest},
%   where \meta{prefix} is made into a string and \meta{rest} cannot be
%   a brace group or multiple tokens since \cs{@@_show_linked:w} would
%   in such cases give a different result from the original token list.
%    \begin{macrocode}
\cs_new_protected:Npn \prop_show:N { \@@_show:NN \msg_show:nneeee }
\cs_generate_variant:Nn \prop_show:N { c }
\cs_new_protected:Npn \prop_log:N { \@@_show:NN \msg_log:nneeee }
\cs_generate_variant:Nn \prop_log:N { c }
\cs_new_protected:Npn \@@_show:NN #1#2
  {
    \__kernel_chk_tl_type:NnnT #2 { prop }
      {
        \@@_if_flat:NTF #2
          {
            \s_@@ \@@_chk:w
            \exp_after:wN \@@_show_flat:w #2
            \s_@@ { }
            \@@_pair:wn \q_@@_recursion_tail \s_@@ { }
            \q_@@_recursion_stop
          }
          { \exp_after:wN \@@_show_linked:w #2 \s_@@ ! ? \s_@@_stop }
      }
      {
        \@@_if_flat:NTF #2
          { \@@_show_finally:NNn #1 #2 { flat } }
          {
            \tl_set:Nn \l_@@_tmp_tl { #1 #2 }
            \exp_after:wN \@@_show_prepare:w #2 #2
          }
      }
  }
\cs_new:Npn \@@_show_flat:w #1 \@@_pair:wn #2 \s_@@ #3
  {
    \@@_if_recursion_tail_stop:n {#2}
    \exp_not:N \@@_pair:wn \tl_to_str:n {#2} \s_@@ \exp_not:n { {#3} }
    \@@_show_flat:w
  }
\cs_new:Npn \@@_show_linked:w #1 \s_@@ #2#3#4 \s_@@_stop
  {
    \exp_not:N \@@_flatten:w
    \exp_not:c { @@ ~ \tl_to_str:n {#2} }
    \s_@@ { \tl_to_str:n {#2} }
    \exp_not:n {#3}
  }
%    \end{macrocode}
%   For flat props we are done by using \cs{msg_show:nneeee} or
%   \cs{msg_log:nneeee}.  The auxiliary \cs{@@_show_finally:NNn} is
%   eventually also used in the linked case after some more tests.  To
%   avoid having to bring along the message function and the property
%   list, we store them into \cs{l_@@_tmp_tl}.
%    \begin{macrocode}
\cs_new_protected:Npn \@@_show_finally:NNn #1#2#3
  {
    #1 { prop } { show }
      { \token_to_str:N #2 }
      { \prop_map_function:NN #2 \msg_show_item:nn }
      {#3} { }
  }
%    \end{macrocode}
%   For linked props, we now know they have a reasonable form so that we
%   are calling \cs{@@_show_prepare:w} \cs{@@_flatten:w}
%   \cs[no-index]{@@~\meta{prefix}} \cs{s_@@} \Arg{prefix} \meta{token}
%   \meta{property list}, and the task is to loop through the linked
%   list and check integrity.  We first set things up: the auxiliary
%   \cs{@@_tmp:w} will be in charge of checking that various tokens
%   start with \cs[no-index]{@@~\meta{prefix}} (in the sense of string
%   representations), and calling one of \cs{@@_show_loop_key:wNNN},
%   \cs{@@_show_end:NNN}, \cs{@@_show_bad_name:NNN}.
%    \begin{macrocode}
\cs_new_protected:Npn \@@_show_prepare:w
    \@@_flatten:w #1 \s_@@ #2#3#4
  {
    \use:e
      {
        \cs_set_nopar:Npn \exp_not:N \@@_tmp:w
            ##1 \token_to_str:N #1 ##2 \s_@@_mark ##3 \s_@@_stop
          {
            \exp_not:N \tl_if_empty:nTF {##1}
              {
                \exp_not:N \tl_if_head_is_space:nTF {##2}
                  { \exp_not:N \exp_args:Nf \@@_show_loop_key:wNNN }
                  { \exp_not:N \tl_if_empty:nTF }
                {##2}
              }
              { \exp_not:N \use_ii:nn }
            \@@_show_end:NNN
            \@@_show_bad_name:NNN
          }
      }
    \exp_last_unbraced:NNNo \@@_show_loop:NNw #1 #4 #4
  }
%    \end{macrocode}
%   The loop will consist of calls to \cs{@@_show_loop:NNw}
%   \cs[no-index]{@@~\meta{prefix}} \meta{token} \meta{expansion}, where
%   \meta{token} is one of the items in the list, specifically the key
%   container for \meta{key$_{i-1}$} (starting at $i=1$ with the
%   property list variable itself), and \meta{expansion} stands for the
%   expansion of that token, which has already been checked, and takes
%   the form \meta{junk} \cs{s_@@} \Arg{value}
%   \cs[no-index]{@@~\meta{prefix}~\meta{key_i}}.  Thus, the loop
%   auxiliary receives the prefix command as |#1|, and the $(i-1)$-th
%   and $i$-th key containers as |#2| and~|#5|.  Then \cs{@@_tmp:w}
%   checks that the name of the $i$-th key container is valid.
%    \begin{macrocode}
\cs_new_protected:Npn \@@_show_loop:NNw #1#2 #3 \s_@@ #4#5
  {
    \exp_last_two_unbraced:Noo \@@_tmp:w
      { \token_to_str:N #5 \s_@@_mark }
      { \token_to_str:N #1 \s_@@_mark \s_@@_stop }
      #1 #2 #5
  }
%    \end{macrocode}
%   If the $i$-th key container has the wrong name we get
%   \cs{@@_show_bad_name:NNN} \cs[no-index]{@@~\meta{prefix}}
%   \meta{previous container} \meta{current container with bad name}.
%    \begin{macrocode}
\cs_new_protected:Npn \@@_show_bad_name:NNN #1#2#3
  {
    \msg_error:nneeee { prop } { bad-link }
      { \tl_tail:N \l_@@_tmp_tl }
      { \token_to_str:N #2 }
      { \token_to_str:N #3 }
      { \token_to_str:N #1 }
  }
%    \end{macrocode}
%   If the $i$-th key container has the name
%   \cs[no-index]{@@~\meta{prefix}} (without space), it is the trailing
%   one.  We check that it is the right kind of macro to be a token
%   list, and that it has the right contents \cs{use_none:n}
%   \meta{previous container}.  If so, we are done checking everything,
%   and we display the property list using the message function and
%   property list name stored in \cs{l_@@_tmp_tl}.  Note that we
%   also use this \cs{l_@@_tmp_tl} in the type argument of
%   \cs{__kernel_chk_tl_type:NnnT}, to build up the name
%   \enquote{\meta{property list} prop entry} used in error messages.
%    \begin{macrocode}
\cs_new_protected:Npn \@@_show_end:NNN #1#2#3
  {
    \__kernel_chk_tl_type:NnnT #3
      { \tl_tail:N \l_@@_tmp_tl prop~entry }
      { \exp_not:n { \use_none:n #2 } }
      {
        \exp_after:wN \@@_show_finally:NNn
        \l_@@_tmp_tl { linked }
      }
  }
%    \end{macrocode}
%   If the $i$-th container has a name \cs[no-index]{@@~\meta{prefix}
%   \meta{key}} (with a space before the key), then we have a call to
%   \cs{@@_show_loop_key:wNNN} \Arg{key} \meta{junk_1} \meta{junk_2}
%   \cs[no-index]{@@~\meta{prefix}} \meta{previous container}
%   \meta{current container}. (with an \texttt{f}-expansion to eliminate
%   the space).  The first argument is the \meta{key} without a leading
%   space, thanks to a judicious \texttt{f}-expansion earlier on.  We
%   check that the \meta{current container} is a token list with the
%   expected structure \cs{use_none:n} \meta{previous container}
%   \cs{@@_pair:wn} \meta{string} \cs{s_@@} \Arg{anything} \meta{single
%   token}.  The auxiliary \cs{@@_show_flat:w} is reused to produce the
%   \cs{@@_pair:wn} part, and the last token is produced by
%   \cs{tl_item:Nn} (we don't waste a specialized auxiliary to speed
%   that up).  If the check succeed, move on to the next item.
%    \begin{macrocode}
\cs_new_protected:Npn \@@_show_loop_key:wNNN #1#2#3#4#5#6
  {
    \__kernel_chk_tl_type:NnnT #6
      { \tl_tail:N \l_@@_tmp_tl prop~entry }
      {
        \exp_not:n { \use_none:n #5 }
        \exp_after:wN \@@_show_flat:w #6 \s_@@ { }
        \@@_pair:wn \q_@@_recursion_tail \s_@@ { }
        \q_@@_recursion_stop
        \tl_item:Nn #6 { -1 }
      }
      { \exp_last_unbraced:NNNo \@@_show_loop:NNw #4 #6 #6 }
  }
%    \end{macrocode}
% \end{macro}
% \end{macro}
% \end{macro}
%
%    \begin{macrocode}
%</code>
%    \end{macrocode}
%
% \end{implementation}
%
% \PrintIndex
