% \iffalse meta-comment
%
%% File: l3pdf.dtx
%
% Copyright(C) 2019-2025 The LaTeX Project
%
% It may be distributed and/or modified under the conditions of the
% LaTeX Project Public License (LPPL), either version 1.3c of this
% license or (at your option) any later version.  The latest version
% of this license is in the file
%
%    http://www.latex-project.org/lppl.txt
%
% This file is part of the "l3kernel bundle" (The Work in LPPL)
% and all files in that bundle must be distributed together.
%
% -----------------------------------------------------------------------
%
% The development version of the bundle can be found at
%
%    https://github.com/latex3/latex3
%
% for those people who are interested.
%
%<*driver>
\documentclass[full]{l3doc}
\begin{document}
  \DocInput{\jobname.dtx}
\end{document}
%</driver>
% \fi
%
% \title{^^A
%   The \pkg{l3pdf} module\\ Core PDF support^^A
% }
%
% \author{^^A
%  The \LaTeX{} Project\thanks
%    {^^A
%      E-mail:
%        \href{mailto:latex-team@latex-project.org}
%          {latex-team@latex-project.org}^^A
%    }^^A
% }
%
% \date{Released 2025-05-19}
%
% \maketitle
%
% \begin{documentation}
%
% \section{Objects}
%
% \subsection{Named objects}
%
% An \meta{object} name should fully expand to tokens suitable for use in a
% label-like context.
%
% \begin{function}[added = 2022-08-23]{\pdf_object_new:n}
%   \begin{syntax}
%     \cs{pdf_object_new:n} \Arg{object}
%   \end{syntax}
%   Declares \meta{object} as a PDF object. The object may be referenced
%   from this point on, and written later using \cs{pdf_object_write:nnn}.
% \end{function}
%
% \begin{function}[added = 2022-08-23]
%   {\pdf_object_write:nnn, \pdf_object_write:nne}
%   \begin{syntax}
%     \cs{pdf_object_write:nnn} \Arg{object} \Arg{type} \Arg{content}
%   \end{syntax}
%   Writes the \meta{content} as content of the \meta{object}. Depending on the
%   \meta{type} declared for the object, the format required for
%   \meta{content} will vary:
%   \begin{itemize}
%     \item[\texttt{array}] A space-separated list of values
%     \item[\texttt{dict}] Key--value pairs in the form
%       \texttt{/\meta{key} \meta{value}}
%     \item[\texttt{fstream}] Two brace groups: \meta{file name} and
%       \meta{file content}
%     \item[\texttt{stream}] Two brace groups: \meta{attributes (dictionary)}
%       and \meta{stream contents}
%   \end{itemize}
% \end{function}
%
% \begin{function}[EXP, added = 2021-02-10]{\pdf_object_ref:n}
%   \begin{syntax}
%     \cs{pdf_object_ref:n} \Arg{object}
%   \end{syntax}
%   Inserts the appropriate information to reference the \meta{object}
%   in for example page resource allocation. If the \meta{object} does
%   not exist then the function expands to a reference to object zero;
%   no PDF indirect object ever has this number, so this is a marker for
%   error.
% \end{function}
%
% \begin{function}[EXP, pTF, added = 2020-05-15]{\pdf_object_if_exist:n}
%   \begin{syntax}
%    \cs{pdf_object_if_exist_p:n} \Arg{object}
%    \cs{pdf_object_if_exist:nTF} \Arg{object} \Arg{true code} \Arg{false code}
%   \end{syntax}
%   Tests whether an object with name \Arg{object} has been defined.
% \end{function}
%
% \subsection{Indexed objects}
%
% Objects can also be created using a pair of \meta{class} and \emph{index};
% the \meta{class} argument should expand to character tokens, whilst the
% \meta{index} is an \meta{int expr} and starts at $1$. For large
% families of objects, this approach is more efficient than using
% individual names.
%
% \begin{function}[added = 2024-04-01]{\pdf_object_new_indexed:nn}
%   \begin{syntax}
%     \cs{pdf_object_new_indexed:nn} \Arg{class} \Arg{index}
%   \end{syntax}
%   Declares a PDF object of \meta{class} and \meta{index}. The object may be
%   referenced from this point on, and written later using
%   \cs{pdf_object_write_indexed:nnnn}.
% \end{function}
%
% \begin{function}[added = 2024-04-01]
%   {\pdf_object_write_indexed:nnnn, \pdf_object_write_indexed:nnne}
%   \begin{syntax}
%     \cs{pdf_object_write_indexed:nnnn} \Arg{class} \Arg{index} \Arg{type} \Arg{content}
%   \end{syntax}
%   Writes the \meta{content} as content of the object of \meta{class} and
%   \meta{index}. Depending on the
%   \meta{type} declared for the object, the format required for the
%   \meta{content} will vary
%   \begin{itemize}
%     \item[\texttt{array}] A space-separated list of values
%     \item[\texttt{dict}] Key--value pairs in the form
%       \texttt{/\meta{key} \meta{value}}
%     \item[\texttt{fstream}] Two brace groups: \meta{file name} and
%       \meta{file content}
%     \item[\texttt{stream}] Two brace groups: \meta{attributes (dictionary)}
%       and \meta{stream contents}
%   \end{itemize}
% \end{function}
%
% \begin{function}[EXP, added = 2024-04-01]{\pdf_object_ref_indexed:nn}
%   \begin{syntax}
%     \cs{pdf_object_ref_indexed:nn} \Arg{class} \Arg{index}
%   \end{syntax}
%   Inserts the appropriate information to reference the object of \meta{class}
%   and \meta{index} in for example page resource allocation. If the
%   \meta{class}/\meta{index} combination does
%   not exist then the function expands to a reference to object zero;
%   no PDF indirect object ever has this number, so this is a marker for
%   error.
% \end{function}
%
% \subsection{General functions}
%
% \begin{function}[added = 2021-02-10]
%   {\pdf_object_unnamed_write:nn, \pdf_object_unnamed_write:ne}
%   \begin{syntax}
%     \cs{pdf_object_unnamed_write:nn} \Arg{type} \Arg{content}
%   \end{syntax}
%   Writes the \meta{content} as content of an anonymous object. Depending on the
%   \meta{type}, the format required for \meta{content} will vary:
%   \begin{itemize}
%     \item[\texttt{array}] A space-separated list of values
%     \item[\texttt{dict}] Key--value pairs in the form
%       \texttt{/\meta{key} \meta{value}}
%     \item[\texttt{fstream}] Two brace groups: \meta{attributes (dictionary)}
%       and \meta{file name}
%     \item[\texttt{stream}] Two brace groups: \meta{attributes (dictionary)}
%       and \meta{stream contents}
%   \end{itemize}
% \end{function}
%
% \begin{function}[EXP, added = 2021-02-10]{\pdf_object_ref_last:}
%   \begin{syntax}
%     \cs{pdf_object_ref_last:}
%   \end{syntax}
%   Inserts the appropriate information to reference the last \meta{object}
%   created. This is particularly useful for anonymous objects.
% \end{function}
%
% \begin{function}[EXP, added = 2021-02-10, updated = 2024-04-22]
%   {\pdf_pageobject_ref:n}
%   \begin{syntax}
%     \cs{pdf_pageobject_ref:n} \Arg{abspage}
%   \end{syntax}
%   Inserts the appropriate information to reference the \meta{abspage};
%   the latter is expanded fully before further processing.
% \end{function}
%
% \section{Version}
%
% \begin{function}[pTF, EXP, added = 2021-02-10]{\pdf_version_compare:Nn}
%   \begin{syntax}
%     \cs{pdf_version_compare_p:Nn} \meta{relation} \Arg{version}
%     \cs{pdf_version_compare:NnTF} \meta{relation} \Arg{version} \Arg{true code} \Arg{false code}
%   \end{syntax}
%   Compares the version of the PDF being created with the \meta{version}
%   string specified, using the \meta{relation}. Either the \meta{true code}
%   or \meta{false code} will be left in the output stream.
% \end{function}
%
% \begin{function}[added = 2021-02-10]
%   {\pdf_version_gset:n, \pdf_version_min_gset:n}
%   \begin{syntax}
%     \cs{pdf_version_gset:n} \Arg{version}
%   \end{syntax}
%   Sets the \meta{version} of the PDF being created. The |min| version will
%   not alter the output version unless it is currently lower than the
%   \meta{version} requested.
%
%   This function may only be used up to the point where the PDF file is
%   initialized. With dvips it sets \cs{pdf_version_major:} and \cs{pdf_version_minor:}
%   and allows to compare the values with \cs{pdf_version_compare:Nn}, but the
%   PDF version itself still has to be set with the command line option
%   |-dCompatibilityLevel| of |ps2pdf|.
% \end{function}
%
% \begin{function}[EXP, added = 2021-02-10]
%   {\pdf_version:, \pdf_version_major:, \pdf_version_minor:}
%   \begin{syntax}
%     \cs{pdf_version:}
%   \end{syntax}
%   Expands to the currently-active PDF version.
% \end{function}
%
% \section{Page (media) size}
%
% \begin{function}[added = 2023-01-14]{\pdf_pagesize_gset:nn}
%   \begin{syntax}
%     \cs{pdf_pagesize_gset:nn} \Arg{width} \Arg{height}
%   \end{syntax}
%   Sets the page size (mediabox) of the PDF being created to the
%   \meta{width} and \meta{height}, both of which are \meta{dimexpr}.
%   The page size can only be set at the start of the output with
%   \texttt{dvips};  with other backends, this can be adjusted on a
%   per-page basis.
% \end{function}
%
% \section{Compression}
%
% \begin{function}[added = 2021-02-10]{\pdf_uncompress:}
%   \begin{syntax}
%     \cs{pdf_uncompress:}
%   \end{syntax}
%   Disables any compression of the PDF, where possible.
%
%   This function may only be used up to the point where the PDF file is
%   initialized.
% \end{function}
%
% \section{Destinations}
%
% Destinations are the places a link jumped to.
% Unlike the name may suggest, they don't describe
% an exact location in the PDF. Instead, a destination contains a reference to
% a page along with an instruction how to display this page.
% The normally used \enquote{XYZ \textit{top left zoom}} for example instructs
% the viewer to show the page with the given \textit{zoom} and
% the top left corner at the \textit{top left} coordinates---which then gives
% the impression that there is an anchor at this position.
%
% If an instruction takes a coordinate, it is calculated by the following
% commands relative to the location the command is issued.
% So to get a specific coordinate one has to move the command to the right place.
%
% \begin{function}[added = 2021-01-03]
%   {\pdf_destination:nn}
%   \begin{syntax}
%     \cs{pdf_destination:nn} \Arg{name} \Arg{type or integer}
%   \end{syntax}
%   This creates a destination. \Arg{type or integer} can be one of |fit|, |fith|,
%   |fitv|, |fitb|, |fitbh|, |fitbv|, |fitr|, |xyz|
%   or an integer representing a  scale factor in percent.
%   |fitr| here gives only a lightweight version of |/FitR|:
%   The backend code defines |fitr| so that it will with pdf\LaTeX{} and
%   Lua\LaTeX{} use the coordinates of the surrounding box,
%   with \texttt{dvips} and \texttt{dvipdfmx} it falls back to |fit|.
%   For full control use \cs{pdf_destination:nnnn}.
%
%   The keywords match to the PDF names as described in the following tabular.
%
%   \medskip
%   \noindent\begin{tabular}{ll>{\raggedright\arraybackslash}p{6cm}}
%   \toprule
%   Keyword & PDF & Remarks \\ \midrule
%   |fit|  & |/Fit|
%      & Fits the page to the window\\
%   |fith| & |/FitH|  \textit{top}
%      & Fits the width of the page to the window \\
%   |fitv| & |/FitV|  \textit{left}
%      & Fits the height of the page to the window \\
%   |fitb| & |/FitB|
%      & Fits the page bounding box to the window \\
%   |fitbh|& |/FitBH| \textit{top}
%      & Fits the width of the page bounding box to the window. \\
%   |fitbv|& |/FitBV| \textit{left}
%      & Fits the height of the page bounding box to the window. \\
%   |fitr| & |/FitR| \textit{left bottom right top}
%      & Fits the rectangle specified by the four coordinates to the window
%        (see above for the restrictions)\\
%   |xyz|  & |/XYZ|  \textit{left top} null
%      & Sets a coordinate but doesn't change the zoom.\\
%   \Arg{integer} & |/XYZ|  \textit{left top zoom}
%      & Sets a coordinate and a zoom meaning \Arg{integer}\%.
%    \\\bottomrule
%   \end{tabular}
%
% \end{function}
%
% \begin{function}[added = 2021-01-17]
%   {\pdf_destination:nnnn}
%   \begin{syntax}
%     \cs{pdf_destination:nnnn} \Arg{name} \Arg{width} \Arg{height} \Arg{depth}
%   \end{syntax}
%   This creates a destination with |/FitR| type with the given dimensions relative
%   to the current location. The destination is in a box of size zero, but it doesn't
%   switch to horizontal mode.
% \end{function}
% \end{documentation}
%
% \begin{implementation}
%
% \section{\pkg{l3pdf} implementation}
%
%    \begin{macrocode}
%<@@=pdf>
%    \end{macrocode}
%
%    \begin{macrocode}
%<*code>
%    \end{macrocode}
%
% \begin{variable}{\s_@@_stop}
%   Internal scan marks.
%    \begin{macrocode}
\scan_new:N \s_@@_stop
%    \end{macrocode}
% \end{variable}
%
% \begin{variable}{\g_@@_init_bool}
%   A boolean so we have some chance of avoiding setting things we are not
%   allowed to. As we are potentially early in the format, we have to work
%   a bit harder than ideal.
%    \begin{macrocode}
\bool_new:N \g_@@_init_bool
\bool_lazy_and:nnT
  { \str_if_eq_p:Vn \fmtname { LaTeX2e } }
  { \tl_if_exist_p:N \@expl@finalise@setup@@@@ }
  {
    \tl_gput_right:Nn \@expl@finalise@setup@@@@
      {
        \tl_gput_right:Nn \@kernel@after@begindocument
          { \bool_gset_true:N \g_@@_init_bool }
      }
  }
%    \end{macrocode}
% \end{variable}
%
% \subsection{Compression}
%
% \begin{macro}{\pdf_uncompress:}
%   Simple to do.
%    \begin{macrocode}
\cs_new_protected:Npn \pdf_uncompress:
  {
    \bool_if:NF \g_@@_init_bool
      {
        \@@_backend_compresslevel:n { 0 }
        \@@_backend_compress_objects:n { \c_false_bool }
      }
  }
%    \end{macrocode}
% \end{macro}
%
% \subsection{Objects}
%
% \begin{variable}{\g_@@_backend_object_int}
%   For returning object numbers.
%    \begin{macrocode}
\int_new:N \g_@@_backend_object_int
%    \end{macrocode}
% \end{variable}
%
% \begin{macro}{\pdf_object_new:n}
% \begin{macro}
%   {\pdf_object_write:nnn, \pdf_object_write:nne, \pdf_object_write:nnx}
% \begin{macro}{\pdf_object_ref:n}
% \begin{macro}{\__kernel_pdf_object_id:n}
%   Simple to do: all objects create a constant |int| so it is not a
%   backend-specific name.
%    \begin{macrocode}
\cs_new_protected:Npn \pdf_object_new:n #1
  {
    \@@_backend_object_new:
    \@@_object_record:nN {#1} \g_@@_backend_object_int
  }
\cs_new_protected:Npn \pdf_object_write:nnn #1#2#3
  {
    \exp_args:Ne \@@_backend_object_write:nnn
      { \@@_object_retrieve:n {#1} } {#2} {#3}
    \bool_gset_true:N \g_@@_init_bool
  }
\cs_generate_variant:Nn \pdf_object_write:nnn { nne , nnx }
\cs_new:Npn \pdf_object_ref:n #1
  {
    \exp_args:Ne \@@_backend_object_ref:n
      { \@@_object_retrieve:n {#1} }
  }
\cs_new:Npn \__kernel_pdf_object_id:n #1
  {
    \exp_args:Ne \@@_backend_object_id:n
      { \@@_object_retrieve:n {#1} }
  }
%</code>
%    \end{macrocode}
% \end{macro}
% \end{macro}
% \end{macro}
% \end{macro}
% \begin{macro}{\@@_object_record:nN}
% \begin{macro}[EXP]{\@@_object_retrieve:n}
% \begin{macro}{ltx.pdf.object_id}
%   Object mappings are tracked in Lua for \LuaTeX{} as this makes retrieving
%   them much easier; as a result, there is a split in approaches. In Lua we
%   store values in a table indexed by name. The Lua function here is set up
%   to deal with both named and indexed objects: fits the Lua idiom well.
%    \begin{macrocode}
%<*lua>

local scan_int = token.scan_int
local scan_string = token.scan_string
local cprint = tex.cprint

local @@_objects_named = {}
local @@_objects_indexed = {}

luacmd('@@_object_record:nN', function()
  local name = scan_string()
  local n = scan_int()
  @@_objects_named[name] = n
end,'protected','global')

local function object_id(name,index)
  if index then
    return @@_objects_indexed[name][index] or 0
  else
    return @@_objects_named[name] or 0
  end
end

luacmd('@@_object_retrieve:n', function()
  local name = scan_string()
  return cprint(12,tostring(object_id(name)))
end,'global')

ltx.pdf = ltx.pdf or {}
ltx.pdf.object_id = object_id

%</lua>
%    \end{macrocode}
%  Whereas in \TeX{} we use integer constants.
%    \begin{macrocode}
%<*code>
\sys_if_engine_luatex:F
  {
    \cs_new_protected:Npn \@@_object_record:nN #1#2
      {
        \int_const:cn
          { c_@@_object_ #1 _int } {#2}
      }
    \cs_new:Npn \@@_object_retrieve:n #1
      {
        \int_if_exist:cTF { c_@@_object_ #1 _int }
          {
            \int_use:c
              { c_@@_object_ #1 _int }
          }
          { 0 }
      }
  }
%    \end{macrocode}
% \end{macro}
% \end{macro}
% \end{macro}
% \begin{macro}[pTF,EXP]{\pdf_object_if_exist:n}
%    \begin{macrocode}
\prg_new_conditional:Npnn \pdf_object_if_exist:n #1 { p , T , F , TF }
  {
    \int_compare:nNnTF { \@@_object_retrieve:n {#1} } = 0
      \prg_return_false:
      \prg_return_true:
  }
%    \end{macrocode}
% \end{macro}
%
% \begin{macro}{\pdf_object_new_indexed:nn}
% \begin{macro}{\pdf_object_write_indexed:nnnn, \pdf_object_write_indexed:nnne}
% \begin{macro}[EXP]{\pdf_object_ref_indexed:nn}
% \begin{macro}[EXP]{\__kernel_pdf_object_id_indexed:nn}
%   Again we split between the common code and the macro- or Lua-based
%   implementation. To make life easier for the Lua route, all of the
%   potential expressions are expanded to braced numbers.
%    \begin{macrocode}
\cs_new_protected:Npn \pdf_object_new_indexed:nn #1#2
  {
    \@@_backend_object_new:
    \@@_object_record:neN {#1}
      { \int_eval:n {#2} } \g_@@_backend_object_int
  }
\cs_new_protected:Npn \pdf_object_write_indexed:nnnn #1#2#3#4
  {
    \exp_args:Ne \@@_backend_object_write:nnn
      { \@@_object_retrieve:ne {#1} { \int_eval:n {#2} } } {#3} {#4}
    \bool_gset_true:N \g_@@_init_bool
  }
\cs_generate_variant:Nn \pdf_object_write_indexed:nnnn { nnne }
\cs_new:Npn \pdf_object_ref_indexed:nn #1#2
  {
    \exp_args:Ne \@@_backend_object_ref:n
      { \@@_object_retrieve:ne {#1} { \int_eval:n {#2} } }
  }
\cs_new:Npn \__kernel_pdf_object_id_indexed:nn #1#2
  {
    \exp_args:Ne \@@_backend_object_id:n
      { \@@_object_retrieve:ne {#1} { \int_eval:n {#2} } }
  }
%</code>
%    \end{macrocode}
% \end{macro}
% \end{macro}
% \end{macro}
% \end{macro}
% \begin{macro}{\@@_object_record:nnN, \@@_object_record:neN}
% \begin{macro}[EXP]{\@@_object_retrieve:nn}
% \begin{macro}{\@@_object_record:NnN}
% \begin{macro}[EXP]{\@@_object_retrieve:Nn}
%   Again we split for Lua: the same idea as above but with nested tables.
%   As we've arranged above that the \TeX{} code passes a braced number,
%   we can use |tonumber(scan_string())| rather than |scan_int()|
%   for the index.
%    \begin{macrocode}
%<*lua>

luacmd('@@_object_record:nnN', function()
  local name = scan_string()
  local index = tonumber(scan_string())
  local n = scan_int()
  @@_objects_indexed[name] = @@_objects_indexed[name] or {}
  @@_objects_indexed[name][index] = n
end,'protected','global')

luacmd('@@_object_retrieve:nn', function()
  local name = scan_string()
  local index = tonumber(scan_string())
  return cprint(12,tostring(object_id(name,index)))
end,'global')

%</lua>
%    \end{macrocode}
% The non-Lua approach is to divide the range into blocks, and store in integer
% arrays that can simulate dynamic assignment.
%    \begin{macrocode}
%<*code>
\sys_if_engine_luatex:F
  {
    \cs_new_protected:Npn \@@_object_record:nnN #1#2#3
      {
        \use:e
          {
            \@@_object_record:NnN
               \@@_object_index_split:nn {#1} {#2}
               \exp_not:N #3
          }
      }
    \cs_new_protected:Npn \@@_object_record:NnN #1#2#3
      {
        \intarray_if_exist:NF #1
          { \intarray_new:Nn #1 \c_@@_object_block_size_int }
        \intarray_gset:Nnn #1 {#2} #3
      }
    \cs_new:Npn \@@_object_retrieve:nn #1#2
      {
        \use:e
          {
            \exp_not:N \@@_object_retrieve:Nn
              \@@_object_index_split:nn {#1} {#2}
          }
      }
    \cs_new:Npn \@@_object_retrieve:Nn #1#2
      { \intarray_item:Nn #1 {#2} }
%    \end{macrocode}
%   As we want blocks to start from one, and within the block for the top value
%   to be \enquote{in} the block, we do a little bit of manipulation. By shifting
%   down by one, we keep the values \enquote{in} the block, then we adjust the
%   block/index number to get back on track.
%    \begin{macrocode}
    \cs_new:Npn \@@_object_index_split:nn #1#2
      {
        \exp_not:c
          {
            g_@@_object_ #1 _
            \int_eval:n
              {
                \int_div_truncate:nn { #2 - 1 }
                  \c_@@_object_block_size_int + 1
              }
            _intarray
          }
          {
            \int_eval:n
              { \int_mod:nn { #2 - 1 } \c_@@_object_block_size_int + 1 }
          }
      }
%    \end{macrocode}
% \end{macro}
% \end{macro}
% \end{macro}
% \end{macro}
% \begin{variable}{\c_@@_object_block_size_int}
%   Sets the block size used for managing indexed objects.
%    \begin{macrocode}
    \int_const:Nn \c_@@_object_block_size_int { 10000 }
  }
%    \end{macrocode}
% \end{variable}
% \begin{macro}{\@@_object_record:neN}
% \begin{macro}[EXP]{\@@_object_retrieve:ne}
%   Common variants.
%    \begin{macrocode}
\cs_generate_variant:Nn \@@_object_record:nnN { ne }
\cs_generate_variant:Nn \@@_object_retrieve:nn { ne }
%    \end{macrocode}
% \end{macro}
% \end{macro}
%
% \begin{macro}
%   {
%     \pdf_object_unnamed_write:nn, \pdf_object_unnamed_write:ne,
%     \pdf_object_unnamed_write:nx
%   }
%   No tracking needed here.
%    \begin{macrocode}
\cs_new_protected:Npn \pdf_object_unnamed_write:nn #1#2
  {
    \exp_args:Ne \@@_backend_object_now:nn {#1} {#2}
    \bool_gset_true:N \g_@@_init_bool
  }
\cs_generate_variant:Nn \pdf_object_unnamed_write:nn { ne , nx }
%    \end{macrocode}
% \end{macro}
%
% \begin{macro}{\pdf_object_ref_last:}
%   A one-step wrapper for consistency.
%    \begin{macrocode}
\cs_new:Npn \pdf_object_ref_last: { \@@_backend_object_last: }
%    \end{macrocode}
% \end{macro}
%
% \begin{macro}{\pdf_pageobject_ref:n}
%    \begin{macrocode}
\cs_new:Npn \pdf_pageobject_ref:n #1
  { \exp_args:Ne \@@_backend_pageobject_ref:n {#1} }
%    \end{macrocode}
% \end{macro}
%
% \subsection{Version}
%
% \begin{macro}[pTF,EXP]{\pdf_version_compare:Nn}
% \begin{macro}
%   {
%     @@_version_compare_=:w ,
%     @@_version_compare_<:w ,
%     @@_version_compare_>:w
%   }
%   To compare version, we need to split the given value then deal with both
%   major and minor version
%    \begin{macrocode}
\prg_new_conditional:Npnn \pdf_version_compare:Nn #1#2 { p , T , F , TF }
  { \use:c { @@_version_compare_ #1 :w } #2 . . \s_@@_stop }
\cs_new:cpn { @@_version_compare_=:w } #1 . #2 . #3 \s_@@_stop
  {
    \bool_lazy_and:nnTF
      { \int_compare_p:nNn \@@_backend_version_major: = {#1} }
      { \int_compare_p:nNn \@@_backend_version_minor: = {#2} }
      { \prg_return_true: }
      { \prg_return_false: }
  }
\cs_new:cpn { @@_version_compare_<:w } #1 . #2 . #3 \s_@@_stop
  {
    \bool_lazy_or:nnTF
      { \int_compare_p:nNn \@@_backend_version_major: < {#1} }
      {
        \bool_lazy_and_p:nn
          { \int_compare_p:nNn \@@_backend_version_major: = {#1} }
          { \int_compare_p:nNn \@@_backend_version_minor: < {#2} }
      }
      { \prg_return_true: }
      { \prg_return_false: }
  }
\cs_new:cpn { @@_version_compare_>:w } #1 . #2 . #3 \s_@@_stop
  {
    \bool_lazy_or:nnTF
      { \int_compare_p:nNn \@@_backend_version_major: > {#1} }
      {
        \bool_lazy_and_p:nn
          { \int_compare_p:nNn \@@_backend_version_major: = {#1} }
          { \int_compare_p:nNn \@@_backend_version_minor: > {#2} }
      }
      { \prg_return_true: }
      { \prg_return_false: }
  }
%    \end{macrocode}
% \end{macro}
% \end{macro}
%
% \begin{macro}{\pdf_version_gset:n, \pdf_version_min_gset:n}
% \begin{macro}{\@@_version_gset:w}
%   Split the version and set.
%    \begin{macrocode}
\cs_new_protected:Npn \pdf_version_gset:n #1
  { \@@_version_gset:w  #1 . . \s_@@_stop }
\cs_new_protected:Npn \pdf_version_min_gset:n #1
  {
    \pdf_version_compare:NnT < {#1}
      { \@@_version_gset:w  #1 . . \s_@@_stop }
  }
\cs_new_protected:Npn \@@_version_gset:w  #1 . #2 . #3\s_@@_stop
  {
    \bool_if:NF \g_@@_init_bool
      {
        \@@_backend_version_major_gset:n {#1}
        \@@_backend_version_minor_gset:n {#2}
      }
  }
%    \end{macrocode}
% \end{macro}
% \end{macro}
%
% \begin{macro}[EXP]{\pdf_version:, \pdf_version_major:, \pdf_version_minor:}
%   Wrappers.
%    \begin{macrocode}
\cs_new:Npn \pdf_version:
  { \@@_backend_version_major: . \@@_backend_version_minor: }
\cs_new:Npn \pdf_version_major: { \@@_backend_version_major: }
\cs_new:Npn \pdf_version_minor: { \@@_backend_version_minor: }
%    \end{macrocode}
% \end{macro}
%
% \subsection{Page size}
%
% \begin{macro}{\pdf_pagesize_gset:nn}
%    \begin{macrocode}
\cs_new_protected:Npn \pdf_pagesize_gset:nn #1#2
  { \@@_backend_pagesize_gset:nn {#1} {#2} }
%    \end{macrocode}
% \end{macro}
%
% \subsection{Destinations}
%
% \begin{macro}{\pdf_destination:nn}
%    \begin{macrocode}
\cs_new_protected:Npn \pdf_destination:nn #1#2
  { \@@_backend_destination:nn {#1} {#2} }
%    \end{macrocode}
% \end{macro}
%
% \begin{macro}{\pdf_destination:nnnn}
%    \begin{macrocode}
\cs_new_protected:Npn \pdf_destination:nnnn #1#2#3#4
  {
    \hbox_to_zero:n
      { \@@_backend_destination:nnnn {#1} {#2} {#3} {#4} }
  }
%    \end{macrocode}
% \end{macro}
%
% \subsection{PDF Page size (media box)}
%
%  Everything here is delayed to the start of the document so that the
%  backend will definitely be loaded.
%    \begin{macrocode}
\cs_if_exist:NT \@kernel@before@begindocument
  {
    \tl_gput_right:Nn \@kernel@before@begindocument
      {
        \bool_lazy_all:nT
          {
            { \cs_if_exist_p:N \stockheight }
            { \cs_if_exist_p:N \stockwidth }
            { \cs_if_exist_p:N \IfDocumentMetadataTF }
            { \IfDocumentMetadataTF { \c_true_bool } { \c_false_bool } }
            { \int_compare_p:nNn \tex_mag:D = { 1000 } }
          }
          {
            \bool_lazy_and:nnTF
              { \dim_compare_p:nNn \stockheight > { 0pt } }
              { \dim_compare_p:nNn \stockwidth  > { 0pt } }
              {
                \@@_backend_pagesize_gset:nn
                  \stockwidth \stockheight
              }
              {
                \bool_lazy_or:nnF
                  { \dim_compare_p:nNn \stockheight < { 0pt } }
                  { \dim_compare_p:nNn \stockwidth  < { 0pt } }
                  {
                    \bool_lazy_and:nnT
                      { \dim_compare_p:nNn \paperheight > { 0pt } }
                      { \dim_compare_p:nNn \paperwidth  > { 0pt } }
                      {
                        \@@_backend_pagesize_gset:nn
                          \paperwidth \paperheight
                      }
                  }
              }
          }
      }
  }
%    \end{macrocode}
%
%    \begin{macrocode}
%</code>
%    \end{macrocode}
%
% \end{implementation}
%
% \PrintIndex
