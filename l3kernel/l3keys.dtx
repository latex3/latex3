% \iffalse meta-comment
%
%% File: l3keys.dtx
%
% Copyright (C) 2006-2023 The LaTeX Project
%
% It may be distributed and/or modified under the conditions of the
% LaTeX Project Public License (LPPL), either version 1.3c of this
% license or (at your option) any later version.  The latest version
% of this license is in the file
%
%    https://www.latex-project.org/lppl.txt
%
% This file is part of the "l3kernel bundle" (The Work in LPPL)
% and all files in that bundle must be distributed together.
%
% -----------------------------------------------------------------------
%
% The development version of the bundle can be found at
%
%    https://github.com/latex3/latex3
%
% for those people who are interested.
%
%<*driver>
\documentclass[full,kernel]{l3doc}
\begin{document}
  \DocInput{\jobname.dtx}
\end{document}
%</driver>
% \fi
%
% \title{^^A
%   The \pkg{l3keys} package\\ Key--value interfaces^^A
% }
%
% \author{^^A
%  The \LaTeX{} Project\thanks
%    {^^A
%      E-mail:
%        \href{mailto:latex-team@latex-project.org}
%          {latex-team@latex-project.org}^^A
%    }^^A
% }
%
% \date{Released 2023-06-30}
%
% \maketitle
%
% \begin{documentation}
%
% The key--value method is a popular system for creating large numbers
% of settings for controlling function or package behaviour.  The
% system normally results in input of the form
% \begin{verbatim}
%   \MyModuleSetup{
%     key-one = value one,
%     key-two = value two
%   }
% \end{verbatim}
% or
% \begin{verbatim}
%   \MyModuleMacro[
%     key-one = value one,
%     key-two = value two
%   ]{argument}
% \end{verbatim}
% for the user.
%
% The high level functions here are intended as a method to create
% key--value controls. Keys are themselves created using a key--value
% interface, minimising the number of functions and arguments
% required. Each key is created by setting one or more \emph{properties}
% of the key:
% \begin{verbatim}
%   \keys_define:nn { mymodule }
%     {
%       key-one .code:n   = code including parameter #1,
%       key-two .tl_set:N = \l_mymodule_store_tl
%     }
% \end{verbatim}
% These values can then be set as with other key--value approaches:
% \begin{verbatim}
%   \keys_set:nn { mymodule }
%     {
%       key-one = value one,
%       key-two = value two
%     }
% \end{verbatim}
% As illustrated, keys are created inside a \meta{module}: a set of related
% keys, typically those for a single module/\LaTeXe{} package. See
% Section~\label{sec:l3keys:subdivision} for suggestions on how to divide
% large numbers of keys for a single module.
%
% At a document level, \cs{keys_set:nn} is used within a
% document function, for example
% \begin{verbatim}
%   \DeclareDocumentCommand \MyModuleSetup { m }
%     { \keys_set:nn { mymodule } { #1 }  }
%   \DeclareDocumentCommand \MyModuleMacro { o m }
%     {
%       \group_begin:
%         \keys_set:nn { mymodule } { #1 }
%         % Main code for \MyModuleMacro
%       \group_end:
%     }
% \end{verbatim}
%
% Key names may contain any tokens, as they are handled internally
% using \cs{tl_to_str:n}. As discussed in
% section~\ref{sec:l3keys:subdivision}, it is suggested that the character
% |/| is reserved for sub-division of keys into logical
% groups. Functions and variables are \emph{not} expanded when creating
% key names, and so
% \begin{verbatim}
%   \tl_set:Nn \l_mymodule_tmp_tl { key }
%   \keys_define:nn { mymodule }
%     {
%       \l_mymodule_tmp_tl .code:n = code
%     }
% \end{verbatim}
% creates a key called |\l_mymodule_tmp_tl|, and not one called
% \texttt{key}.
%
% \section{Creating keys}
%
% \begin{function}[updated = 2017-11-14]{\keys_define:nn}
%   \begin{syntax}
%     \cs{keys_define:nn} \Arg{module} \Arg{keyval list}
%   \end{syntax}
%   Parses the \meta{keyval list} and defines the keys listed there for
%   \meta{module}. The \meta{module} name is treated as a string.
%   In practice the
%   \meta{module} should be chosen to be unique to the module in question
%   (unless deliberately adding keys to an existing module).
%
%   The \meta{keyval list} should consist of one or more key names along
%   with an associated key \emph{property}. The properties of a key
%   determine how it acts. The individual properties are described
%   in the following text; a typical use of \cs{keys_define:nn} might
%   read
%   \begin{verbatim}
%     \keys_define:nn { mymodule }
%       {
%         keyname .code:n = Some~code~using~#1,
%         keyname .value_required:n = true
%       }
%   \end{verbatim}
%   where the properties of the key begin from the |.| after the key
%   name.
% \end{function}
%
% The various properties available take either no arguments at
% all, or require one or more arguments. This is indicated in the
% name of the property using an argument specification. In the following
% discussion, each property is illustrated attached to an
% arbitrary \meta{key}, which when used may be supplied with a
% \meta{value}. All key \emph{definitions} are local.
%
% Key properties are applied in the reading order and so the ordering
% is significant. Key properties which define \enquote{actions}, such
% as |.code:n|, |.tl_set:N|, \emph{etc.}, override one another.
% Some other properties are mutually exclusive, notably |.value_required:n|
% and |.value_forbidden:n|, and so they replace one another. However,
% properties covering non-exclusive behaviours may be given in any order. Thus
% for example the following definitions are equivalent.
% \begin{verbatim}
%   \keys_define:nn { mymodule }
%     {
%       keyname .code:n           = Some~code~using~#1,
%       keyname .value_required:n = true
%     }
%   \keys_define:nn { mymodule }
%     {
%       keyname .value_required:n = true,
%       keyname .code:n           = Some~code~using~#1
%     }
% \end{verbatim}
% Note that all key properties define the key within the current \TeX{} group,
% with an exception that the special |.undefine:| property \emph{undefines} the
% key within the current \TeX{} group.
%
% \begin{function}[updated = 2013-07-08]
%   {.bool_set:N, .bool_set:c, .bool_gset:N, .bool_gset:c}
%   \begin{syntax}
%     \meta{key} .bool_set:N = \meta{boolean variable}
%   \end{syntax}
%   Defines \meta{key} to set \meta{boolean variable} to \meta{value} (which
%   must be either \enquote{\texttt{true}} or \enquote{\texttt{false}}).  If the variable
%   does not exist, it will be created globally at the point that
%   the key is set up.
% \end{function}
%
% \begin{function}[added = 2011-08-28, updated = 2013-07-08]
%   {
%     .bool_set_inverse:N, .bool_set_inverse:c,
%     .bool_gset_inverse:N, .bool_gset_inverse:c
%   }
%   \begin{syntax}
%     \meta{key} .bool_set_inverse:N = \meta{boolean variable}
%   \end{syntax}
%   Defines \meta{key} to set \meta{boolean variable} to the logical
%   inverse of \meta{value} (which  must be either \enquote{\texttt{true}} or
%   \enquote{\texttt{false}}).
%   If the \meta{boolean variable} does not exist, it will be created globally
%   at the point that the key is set up.
% \end{function}
%
% \begin{function}{.choice:}
%   \begin{syntax}
%     \meta{key} .choice:
%   \end{syntax}
%   Sets \meta{key} to act as a choice key. Each valid choice
%   for \meta{key} must then be created, as discussed in
%   section~\ref{sec:l3keys:choice}.
% \end{function}
%
% \begin{function}[added = 2011-08-21, updated = 2013-07-10]
%   {.choices:nn, .choices:Vn, .choices:on, .choices:xn}
%   \begin{syntax}
%     \meta{key} .choices:nn = \Arg{choices} \Arg{code}
%   \end{syntax}
%   Sets \meta{key} to act as a choice key, and defines a series \meta{choices}
%   which are implemented using the \meta{code}. Inside \meta{code},
%   \cs{l_keys_choice_tl} will be the name of the choice made, and
%   \cs{l_keys_choice_int} will be the position of the choice in the list
%   of \meta{choices} (indexed from~$1$).
%   Choices are discussed in detail in section~\ref{sec:l3keys:choice}.
% \end{function}
%
% \begin{function}[added = 2011-09-11]
%   {.clist_set:N, .clist_set:c, .clist_gset:N, .clist_gset:c}
%   \begin{syntax}
%     \meta{key} .clist_set:N = \meta{comma list variable}
%   \end{syntax}
%   Defines \meta{key} to set \meta{comma list variable} to \meta{value}.
%   Spaces around commas and empty items will be stripped.
%   If the variable does not exist, it
%   is created globally at the point that the key is set up.
% \end{function}
%
% \begin{function}[updated = 2013-07-10]{.code:n}
%   \begin{syntax}
%     \meta{key} .code:n = \Arg{code}
%   \end{syntax}
%   Stores the \meta{code} for execution when \meta{key} is used.
%   The \meta{code} can include one parameter (|#1|), which will be the
%   \meta{value} given for the \meta{key}.
% \end{function}
%
% \begin{function}[added = 2020-01-11]
%   {
%     .cs_set:Np, .cs_set:cp,
%     .cs_set_protected:Np, .cs_set_protected:cp,
%     .cs_gset:Np, .cs_gset:cp,
%     .cs_gset_protected:Np, .cs_gset_protected:cp,
%   }
%   \begin{syntax}
%     \meta{key} .cs_set:Np = \meta{control sequence} \meta{arg.~spec.}
%   \end{syntax}
%   Defines \meta{key} to set \meta{control sequence} to have \meta{arg.~spec.}
%   and replacement text \meta{value}.
% \end{function}
%
% \begin{function}[updated = 2013-07-09]
%   {.default:n, .default:V, .default:o, .default:x}
%   \begin{syntax}
%     \meta{key} .default:n = \Arg{default}
%   \end{syntax}
%   Creates a \meta{default} value for \meta{key}, which is used if no
%   value is given. This will be used if only the key name is given,
%   but not if a blank \meta{value} is given:
%   \begin{verbatim}
%     \keys_define:nn { mymodule }
%       {
%         key .code:n    = Hello~#1,
%         key .default:n = World
%       }
%     \keys_set:nn { mymodule }
%       {
%         key = Fred, % Prints 'Hello Fred'
%         key,        % Prints 'Hello World'
%         key = ,     % Prints 'Hello '
%       }
%   \end{verbatim}
%   The default does not affect keys where values are required or
%   forbidden. Thus a required value cannot be supplied by a default
%   value, and giving a default value for a key which cannot take a value
%   does not trigger an error.
%
%   When no value is given for a key as part of \cs{keys_set:nn}, the
%   \texttt{.default:n} value provides the value before key properties are
%   considered. The only exception is when the \texttt{.value_required:n}
%   property is active: a required value cannot be supplied by the default,
%   and must be explicitly given as part of \cs{keys_set:nn}.
% \end{function}
%
% \begin{function}[updated = 2020-01-17]
%   {.dim_set:N, .dim_set:c, .dim_gset:N, .dim_gset:c}
%   \begin{syntax}
%     \meta{key} .dim_set:N = \meta{dimension}
%   \end{syntax}
%   Defines \meta{key} to set \meta{dimension} to \meta{value} (which
%   must a dimension expression).  If the variable does not exist, it
%   is created globally at the point that the key is set up. The key will
%   require a value at point-of-use unless a default is set.
% \end{function}
%
% \begin{function}[updated = 2020-01-17]
%   {.fp_set:N, .fp_set:c, .fp_gset:N, .fp_gset:c}
%   \begin{syntax}
%     \meta{key} .fp_set:N = \meta{floating point}
%   \end{syntax}
%   Defines \meta{key} to set \meta{floating point} to \meta{value}
%   (which must a floating point expression).  If the variable does not exist,
%   it is created globally at the point that the key is set up. The key will
%   require a value at point-of-use unless a default is set.
% \end{function}
%
% \begin{function}[added = 2013-07-14]
%   {.groups:n}
%   \begin{syntax}
%     \meta{key} .groups:n = \Arg{groups}
%   \end{syntax}
%   Defines \meta{key} as belonging to the \meta{groups} declared. Groups
%   provide a \enquote{secondary axis} for selectively setting keys, and are
%   described in Section~\ref{sec:l3keys:selective}.
% \end{function}
%
% \begin{function}[added = 2016-11-22]{.inherit:n}
%   \begin{syntax}
%     \meta{key} .inherit:n = \Arg{parents}
%   \end{syntax}
%   Specifies that the \meta{key} path should inherit the keys listed
%   as any of the \meta{parents} (a comma list), which can be a module
%   or a subgroup. For example, after setting
%   \begin{verbatim}
%     \keys_define:nn { foo } { test .code:n = \tl_show:n {#1} }
%     \keys_define:nn { } { bar .inherit:n = foo }
%   \end{verbatim}
%   setting
%   \begin{verbatim}
%     \keys_set:nn { bar } { test = a }
%   \end{verbatim}
%   will be equivalent to
%   \begin{verbatim}
%     \keys_set:nn { foo } { test = a }
%   \end{verbatim}
%   Inheritance applies at point of use, not at definition, thus keys may
%   be added to the \meta{parent} after the use of \texttt{.inherit:n}
%   and will be active.
%   If more than one \meta{parent} is specified, the presence of the
%   \meta{key} will be tested for each in turn, with the first successful
%   hit taking priority.
% \end{function}
%
% \begin{function}[updated = 2013-07-09]
%   {.initial:n, .initial:V, .initial:o, .initial:x}
%   \begin{syntax}
%     \meta{key} .initial:n = \Arg{value}
%   \end{syntax}
%   Initialises the \meta{key} with the \meta{value}, equivalent to
%   \begin{quote}
%     \cs{keys_set:nn} \Arg{module} \{ \meta{key} = \meta{value} \}
%   \end{quote}
% \end{function}
%
% \begin{function}[updated = 2020-01-17]
%   {.int_set:N, .int_set:c, .int_gset:N, .int_gset:c}
%   \begin{syntax}
%     \meta{key} .int_set:N = \meta{integer}
%   \end{syntax}
%   Defines \meta{key} to set \meta{integer} to \meta{value} (which
%   must be an integer expression).  If the variable does not exist, it
%   is created globally at the point that the key is set up. The key will
%   require a value at point-of-use unless a default is set.
% \end{function}
%
% \begin{function}[updated = 2022-01-15]
%   {
%     .legacy_if_set:n, .legacy_if_gset:n,
%     .legacy_if_set_inverse:n, .legacy_if_gset_inverse:n
%   }
%   \begin{syntax}
%     \meta{key} .legacy_if_set:n = \meta{switch}
%   \end{syntax}
%   Defines \meta{key} to set legacy \cs{if} \meta{switch} to \meta{value}
%   (which must be either \enquote{\texttt{true}} or \enquote{\texttt{false}}).
%   The \meta{switch} is the name of the switch \emph{without the leading
%   \cs{if}}.
%
%   The \texttt{inverse} versions will set the \meta{switch} to the logical
%   opposite of the \meta{value}.
% \end{function}
%
% \begin{function}[updated = 2013-07-10]{.meta:n}
%   \begin{syntax}
%     \meta{key} .meta:n = \Arg{keyval list}
%   \end{syntax}
%   Makes \meta{key} a meta-key, which will set \meta{keyval list} in
%   one go.  The \meta{keyval list} can refer as |#1| to the value given
%   at the time the \meta{key} is used (or, if no value is given, the
%   \meta{key}'s default value).
% \end{function}
%
% \begin{function}[added = 2013-07-10]{.meta:nn}
%   \begin{syntax}
%     \meta{key} .meta:nn = \Arg{path} \Arg{keyval list}
%   \end{syntax}
%   Makes \meta{key} a meta-key, which will set \meta{keyval list} in
%   one go using the \meta{path} in place of the current one.  The
%   \meta{keyval list} can refer as |#1| to the value given at the time
%   the \meta{key} is used (or, if no value is given, the \meta{key}'s
%   default value).
% \end{function}
%
% \begin{function}[added = 2011-08-21]{.multichoice:}
%   \begin{syntax}
%     \meta{key} .multichoice:
%   \end{syntax}
%   Sets \meta{key} to act as a multiple choice key. Each valid choice
%   for \meta{key} must then be created, as discussed in
%   section~\ref{sec:l3keys:choice}.
% \end{function}
%
% \begin{function}[added = 2011-08-21, updated = 2013-07-10]
%   {.multichoices:nn, .multichoices:Vn, .multichoices:on, .multichoices:xn}
%   \begin{syntax}
%     \meta{key} .multichoices:nn \Arg{choices} \Arg{code}
%   \end{syntax}
%   Sets \meta{key} to act as a multiple choice key, and defines a series
%   \meta{choices}
%   which are implemented using the \meta{code}. Inside \meta{code},
%   \cs{l_keys_choice_tl} will be the name of the choice made, and
%   \cs{l_keys_choice_int} will be the position of the choice in the list
%   of \meta{choices} (indexed from~$1$).
%   Choices are discussed in detail in section~\ref{sec:l3keys:choice}.
% \end{function}
%
% \begin{function}[added = 2019-05-05, updated = 2020-01-17]
%   {.muskip_set:N, .muskip_set:c, .muskip_gset:N, .muskip_gset:c}
%   \begin{syntax}
%     \meta{key} .muskip_set:N = \meta{muskip}
%   \end{syntax}
%   Defines \meta{key} to set \meta{muskip} to \meta{value} (which
%   must be a muskip expression). If the variable does not exist, it
%   is created globally at the point that the key is set up. The key will
%   require a value at point-of-use unless a default is set.
% \end{function}
%
% \begin{function}[added = 2019-01-31]
%   {.prop_put:N, .prop_put:c, .prop_gput:N, .prop_gput:c}
%   \begin{syntax}
%     \meta{key} .prop_put:N = \meta{property list}
%   \end{syntax}
%   Defines \meta{key} to put the \meta{value} onto the \meta{property list}
%   stored under the \meta{key}.
%   If the variable does not exist, it
%   is created globally at the point that the key is set up.
% \end{function}
%
% \begin{function}[updated = 2020-01-17]
%   {.skip_set:N, .skip_set:c, .skip_gset:N, .skip_gset:c}
%   \begin{syntax}
%     \meta{key} .skip_set:N = \meta{skip}
%   \end{syntax}
%   Defines \meta{key} to set \meta{skip} to \meta{value} (which
%   must be a skip expression). If the variable does not exist, it
%   is created globally at the point that the key is set up. The key will
%   require a value at point-of-use unless a default is set.
% \end{function}
%
% \begin{function}[added = 2021-10-30]
%   {.str_set:N, .str_set:c, .str_gset:N, .str_gset:c}
%   \begin{syntax}
%     \meta{key} .str_set:N = \meta{string variable}
%   \end{syntax}
%   Defines \meta{key} to set \meta{string variable} to \meta{value}.
%   If the variable does not exist, it is created globally
%   at the point that the key is set up.
% \end{function}
%
% \begin{function}[added = 2021-10-30]
%   {.str_set_x:N, .str_set_x:c, .str_gset_x:N, .str_gset_x:c}
%   \begin{syntax}
%     \meta{key} .str_set_x:N = \meta{string variable}
%   \end{syntax}
%   Defines \meta{key} to set \meta{string variable} to \meta{value},
%   which will be subjected to an \texttt{x}-type expansion
%   (\emph{i.e.}~using \cs{str_set:Nx}). If the variable does not exist,
%   it is created globally at the point that the key is set up.
% \end{function}
%
% \begin{function}{.tl_set:N, .tl_set:c, .tl_gset:N, .tl_gset:c}
%   \begin{syntax}
%     \meta{key} .tl_set:N = \meta{token list variable}
%   \end{syntax}
%   Defines \meta{key} to set \meta{token list variable} to \meta{value}.
%   If the variable does not exist, it is created globally
%   at the point that the key is set up.
% \end{function}
%
% \begin{function}{.tl_set_x:N, .tl_set_x:c, .tl_gset_x:N, .tl_gset_x:c}
%   \begin{syntax}
%     \meta{key} .tl_set_x:N = \meta{token list variable}
%   \end{syntax}
%   Defines \meta{key} to set \meta{token list variable} to \meta{value},
%   which will be subjected to an \texttt{x}-type expansion
%   (\emph{i.e.}~using \cs{tl_set:Nx}). If the variable does not exist,
%   it is created globally at the point that the key is set up.
% \end{function}
%
% \begin{function}[added = 2015-07-14]{.undefine:}
%   \begin{syntax}
%     \meta{key} .undefine:
%   \end{syntax}
%   Removes the definition of the \meta{key} within the current \TeX{} group.
% \end{function}
%
% \begin{function}[added = 2015-07-14]{.value_forbidden:n}
%   \begin{syntax}
%     \meta{key} .value_forbidden:n = \texttt{true\string|false}
%   \end{syntax}
%   Specifies that \meta{key} cannot receive a \meta{value} when used.
%   If a \meta{value} is given then an error will be issued. Setting
%   the property \enquote{\texttt{false}} cancels the restriction.
% \end{function}
%
% \begin{function}[added = 2015-07-14]{.value_required:n}
%   \begin{syntax}
%      \meta{key} .value_required:n = \texttt{true\string|false}
%   \end{syntax}
%   Specifies that \meta{key} must receive a \meta{value} when used.
%   If a \meta{value} is not given then an error will be issued. Setting
%   the property \enquote{\texttt{false}} cancels the restriction.
% \end{function}
%
% \section{Sub-dividing keys}
% \label{sec:l3keys:subdivision}
%
% When creating large numbers of keys, it may be desirable to divide
% them into several sub-groups for a given module. This can be achieved
% either by adding a sub-division to the module name:
% \begin{verbatim}
%   \keys_define:nn { mymodule / subgroup }
%     { key .code:n = code }
% \end{verbatim}
% or to the key name:
% \begin{verbatim}
%   \keys_define:nn { mymodule }
%     { subgroup / key .code:n = code }
% \end{verbatim}
% As illustrated, the best choice of token for sub-dividing keys in
% this way is |/|. This is because of the method that is
% used to represent keys internally. Both of the above code fragments
% set the same key, which has full name \texttt{mymodule/subgroup/key}.
%
% As illustrated in the next section, this subdivision is
% particularly relevant to making multiple choices.
%
% \section{Choice and multiple choice keys}
% \label{sec:l3keys:choice}
%
% The \pkg{l3keys} system supports two types of choice key, in which a series
% of pre-defined input values are linked to varying implementations. Choice
% keys are usually created so that the various values are mutually-exclusive:
% only one can apply at any one time. \enquote{Multiple} choice keys are also
% supported: these allow a selection of values to be chosen at the same time.
%
% Mutually-exclusive choices are created by setting the \texttt{.choice:}
% property:
% \begin{verbatim}
%   \keys_define:nn { mymodule }
%     { key .choice: }
% \end{verbatim}
% For keys which are set up as choices, the valid choices are generated
% by creating sub-keys of the choice key. This can be carried out in
% two ways.
%
% In many cases, choices execute similar code which is dependant only
% on the name of the choice or the position of the choice in the
% list of all possibilities. Here, the keys can share the same code, and can
% be rapidly created using the  \texttt{.choices:nn} property.
% \begin{verbatim}
%   \keys_define:nn { mymodule }
%     {
%       key .choices:nn =
%         { choice-a, choice-b, choice-c }
%         {
%           You~gave~choice~'\tl_use:N \l_keys_choice_tl',~
%           which~is~in~position~\int_use:N \l_keys_choice_int \c_space_tl
%           in~the~list.
%         }
%     }
% \end{verbatim}
% The index \cs{l_keys_choice_int} in the list of choices starts at~$1$.
%
% \begin{variable}{\l_keys_choice_int, \l_keys_choice_tl}
%   Inside the code block for a choice generated using \texttt{.choices:nn},
%   the variables \cs{l_keys_choice_tl} and \cs{l_keys_choice_int} are
%   available to indicate the name of the current choice, and its position in
%   the comma list.  The position is indexed from~$1$. Note that, as with
%   standard key code generated using \texttt{.code:n}, the value passed to
%   the key (i.e.~the choice name) is also available as |#1|.
% \end{variable}
%
% On the other hand, it is sometimes useful to create choices which
% use entirely different code from one another. This can be achieved
% by setting the \texttt{.choice:} property of a key, then manually
% defining sub-keys.
% \begin{verbatim}
%   \keys_define:nn { mymodule }
%     {
%       key .choice:,
%       key / choice-a .code:n = code-a,
%       key / choice-b .code:n = code-b,
%       key / choice-c .code:n = code-c,
%     }
% \end{verbatim}
%
% It is possible to mix the two methods, but manually-created choices
% should \emph{not} use \cs{l_keys_choice_tl} or \cs{l_keys_choice_int}.
% These variables do not have defined behaviour when used outside of
% code created using \texttt{.choices:nn}
% (\emph{i.e.}~anything might happen).
%
% It is possible to allow choice keys to take values which have not previously
% been defined by adding code for the special \texttt{unknown} choice. The
% general behavior of the \texttt{unknown} key is described in
% Section~\ref{sec:l3keys:unknown}. A typical example in the case of a choice
% would be to issue a custom error message:
% \begin{verbatim}
%   \keys_define:nn { mymodule }
%     {
%       key .choice:,
%       key / choice-a .code:n = code-a,
%       key / choice-b .code:n = code-b,
%       key / choice-c .code:n = code-c,
%       key / unknown  .code:n =
%         \msg_error:nnxxx { mymodule } { unknown-choice }
%           { key }                              % Name of choice key
%           { choice-a , choice-b ,  choice-c }  % Valid choices
%           { \exp_not:n {#1} }                  % Invalid choice given
%       %
%       %
%     }
% \end{verbatim}
%
% Multiple choices are created in a very similar manner to mutually-exclusive
% choices, using the properties \texttt{.multichoice:} and
% \texttt{.multichoices:nn}. As with mutually exclusive choices, multiple
% choices are define as sub-keys. Thus both
% \begin{verbatim}
%   \keys_define:nn { mymodule }
%     {
%       key .multichoices:nn =
%         { choice-a, choice-b, choice-c }
%         {
%           You~gave~choice~'\tl_use:N \l_keys_choice_tl',~
%           which~is~in~position~
%           \int_use:N \l_keys_choice_int \c_space_tl
%           in~the~list.
%         }
%     }
% \end{verbatim}
% and
% \begin{verbatim}
%   \keys_define:nn { mymodule }
%     {
%       key .multichoice:,
%       key / choice-a .code:n = code-a,
%       key / choice-b .code:n = code-b,
%       key / choice-c .code:n = code-c,
%     }
% \end{verbatim}
% are valid.
%
% When a multiple choice key is set
% \begin{verbatim}
%   \keys_set:nn { mymodule }
%     {
%       key = { a , b , c } % 'key' defined as a multiple choice
%     }
% \end{verbatim}
% each choice is applied in turn, equivalent to a \texttt{clist} mapping or
% to applying each value individually:
% \begin{verbatim}
%   \keys_set:nn { mymodule }
%     {
%       key = a ,
%       key = b ,
%       key = c ,
%     }
% \end{verbatim}
% Thus each separate choice will have passed to it the
% \cs{l_keys_choice_tl} and \cs{l_keys_choice_int} in exactly
% the same way as described for \texttt{.choices:nn}.
%
% \section{Key usage scope}
%
% Some keys will be used as settings which have a strictly limited scope
% of usage. Some will be only available once, others will only be valid
% until typesetting begins. To allow formats to support this in a structured
% way, \pkg{l3keys} allows this information to be specified using the
% \texttt{.usage:n} property.
%
% \begin{function}[added = 2022-01-10]{.usage:n}
%   \begin{syntax}
%     \meta{key} .usage:n = \meta{scope}
%   \end{syntax}
%   Defines the \meta{key} to have usage within the \meta{scope}, which
%   should be one of \texttt{general}, \texttt{preamble} or \texttt{load}.
% \end{function}
%
% \begin{variable}[added = 2022-01-10]
%   {\l_keys_usage_load_prop, \l_keys_usage_preamble_prop}
%   \pkg{l3keys} itself does \emph{not} attempt to redefine keys based on the
%   usage scope. Rather, this information is made available with these
%   two property lists. These hold an entry for each module (prefix); the
%   value of each entry is a comma-separated list of the usage-restricted
%   key(s).
% \end{variable}
%
% \section{Setting keys}
%
% \begin{function}[updated = 2017-11-14]
%   {\keys_set:nn, \keys_set:nV, \keys_set:nv, \keys_set:no, \keys_set:nx}
%   \begin{syntax}
%     \cs{keys_set:nn} \Arg{module} \Arg{keyval list}
%   \end{syntax}
%   Parses the \meta{keyval list}, and sets those keys which are defined
%   for \meta{module}. The behaviour on finding an unknown key can be set
%   by defining a special \texttt{unknown} key: this is illustrated
%   later.
% \end{function}
%
% \begin{variable}[updated = 2020-02-08]
%   {\l_keys_key_str, \l_keys_path_str, \l_keys_value_tl}
%   For each key processed, information of the full \emph{path} of the
%   key, the \emph{name} of the key and the \emph{value} of the key is
%   available within two string and one token list variables.
%   These may be used within the code of the key.
%
%   The \emph{value} is everything after the \texttt{=}, which may be
%   empty if no value was given. This is stored in \cs{l_keys_value_tl}, and
%   is not processed in any way by \cs{keys_set:nn}.
%
%   The \emph{path} of the key is a \enquote{full} description of the key,
%   and is unique for each key. It consists of the module and full key name,
%   thus for example
%   \begin{verbatim}
%     \keys_set:nn { mymodule } { key-a = some-value }
%   \end{verbatim}
%   has path \texttt{mymodule/key-a} while
%   \begin{verbatim}
%     \keys_set:nn { mymodule } { subset  / key-a = some-value }
%   \end{verbatim}
%   has path \texttt{mymodule/subset/key-a}. This information is stored in
%   \cs{l_keys_path_str}.
%
%   The \emph{name} of the key is the part of the path after the last
%   \texttt{/}, and thus is not unique. In the preceding examples, both keys
%   have name \texttt{key-a} despite having different paths.  This information
%   is stored in \cs{l_keys_key_str}.
% \end{variable}
%
% \section{Handling of unknown keys}
% \label{sec:l3keys:unknown}
%
% If a key has not previously been defined (is unknown), \cs{keys_set:nn}
% looks for a special \texttt{unknown} key for the same module, and if this is
% not defined raises an error indicating that the key name was unknown. This
% mechanism can be used for example to issue custom error texts.
% \begin{verbatim}
%   \keys_define:nn { mymodule }
%     {
%       unknown .code:n =
%         You~tried~to~set~key~'\l_keys_key_str'~to~'#1'.
%     }
% \end{verbatim}
%
% \begin{function}[added = 2011-08-23, updated = 2019-01-29]
%   {
%     \keys_set_known:nn, \keys_set_known:nV,
%     \keys_set_known:nv, \keys_set_known:no,
%     \keys_set_known:nnN, \keys_set_known:nVN,
%     \keys_set_known:nvN, \keys_set_known:noN,
%     \keys_set_known:nnnN, \keys_set_known:nVnN,
%     \keys_set_known:nvnN, \keys_set_known:nonN
%   }
%   \begin{syntax}
%     \cs{keys_set_known:nn} \Arg{module} \Arg{keyval list}
%     \cs{keys_set_known:nnN} \Arg{module} \Arg{keyval list} \meta{tl}
%     \cs{keys_set_known:nnnN} \Arg{module} \Arg{keyval list} \Arg{root} \meta{tl}
%   \end{syntax}
%   These functions set keys which are known for the \meta{module}, and
%   simply ignore other keys. The \cs{keys_set_known:nn} function parses the
%   \meta{keyval list}, and sets those keys which are defined for
%   \meta{module}. Any keys which are unknown are not processed further by
%   the parser. In addition, \cs{keys_set_known:nnN} stores the key--value
%   pairs in the \meta{tl} in comma-separated form (\emph{i.e.}~an edited
%   version of the \meta{keyval list}). When a \meta{root} is given
%   (\cs{keys_set_known:nnnN}), the key--value entries are returned
%   relative to this point in the key tree. When it is absent, only the
%   key name and value are provided. The correct list is returned by
%   nested calls.
% \end{function}
%
% \section{Selective key setting}
% \label{sec:l3keys:selective}
%
% In some cases it may be useful to be able to select only some keys for
% setting, even though these keys have the same path. For example, with
% a set of keys defined using
% \begin{verbatim}
%   \keys_define:nn { mymodule }
%     {
%       key-one   .code:n   = { \my_func:n {#1} } ,
%       key-two   .tl_set:N = \l_my_a_tl          ,
%       key-three .tl_set:N = \l_my_b_tl          ,
%       key-four  .fp_set:N = \l_my_a_fp          ,
%     }
% \end{verbatim}
% the use of \cs{keys_set:nn} attempts to set all four keys. However, in
% some contexts it may only be sensible to set some keys, or to control the
% order of setting. To do this, keys may be assigned to \emph{groups}:
% arbitrary sets which are independent of the key tree. Thus modifying the
% example to read
% \begin{verbatim}
%   \keys_define:nn { mymodule }
%     {
%       key-one   .code:n   = { \my_func:n {#1} } ,
%       key-one   .groups:n = { first }           ,
%       key-two   .tl_set:N = \l_my_a_tl          ,
%       key-two   .groups:n = { first }           ,
%       key-three .tl_set:N = \l_my_b_tl          ,
%       key-three .groups:n = { second }          ,
%       key-four  .fp_set:N = \l_my_a_fp          ,
%     }
% \end{verbatim}
% assigns \texttt{key-one} and \texttt{key-two} to group \texttt{first},
% \texttt{key-three} to group \texttt{second}, while \texttt{key-four} is
% not assigned to a group.
%
% Selective key setting may be achieved either by selecting one or more
% groups to be made \enquote{active}, or by marking one or more groups to
% be ignored in key setting.
%
% \begin{function}[added = 2013-07-14, updated = 2019-01-29]
%   {
%     \keys_set_filter:nnn, \keys_set_filter:nnV,
%     \keys_set_filter:nnv, \keys_set_filter:nno,
%     \keys_set_filter:nnnN, \keys_set_filter:nnVN,
%     \keys_set_filter:nnvN, \keys_set_filter:nnoN,
%     \keys_set_filter:nnnnN, \keys_set_filter:nnVnN,
%     \keys_set_filter:nnvnN, \keys_set_filter:nnonN,
%   }
%   \begin{syntax}
%     \cs{keys_set_filter:nnn} \Arg{module} \Arg{groups} \Arg{keyval list}
%     \cs{keys_set_filter:nnnN} \Arg{module} \Arg{groups} \Arg{keyval list} \meta{tl}
%     \cs{keys_set_filter:nnnnN} \Arg{module} \Arg{groups} \Arg{keyval list} \meta{root} \meta{tl}
%   \end{syntax}
%   Activates key filtering in an \enquote{opt-out} sense: keys assigned to any
%   of the \meta{groups} specified are ignored. The \meta{groups} are
%   given as a comma-separated list. Unknown keys are not assigned to any
%   group and are thus always set. The key--value pairs for each
%   key which is filtered out are stored in the \meta{tl} in a
%   comma-separated form (\emph{i.e.}~an edited version of the \meta{keyval
%   list}). The \cs{keys_set_filter:nnn} version skips this stage.
%
%   Use of \cs{keys_set_filter:nnnN} can be nested, with the correct residual
%   \meta{keyval list} returned at each stage. In the version which takes
%   a \meta{root} argument, the key list is returned relative to that point
%   in the key tree. In the cases without a \meta{root} argument, only
%   the key names and values are returned.
% \end{function}
%
% \begin{function}[added = 2013-07-14, updated = 2017-05-27]
%   {
%     \keys_set_groups:nnn, \keys_set_groups:nnV,
%     \keys_set_groups:nnv, \keys_set_groups:nno
%   }
%   \begin{syntax}
%     \cs{keys_set_groups:nnn} \Arg{module} \Arg{groups} \Arg{keyval list}
%   \end{syntax}
%   Activates key filtering in an \enquote{opt-in} sense: only keys assigned to
%   one or more of the \meta{groups} specified are set. The \meta{groups} are
%   given as a comma-separated list. Unknown keys are not assigned to any
%   group and are thus never set.
% \end{function}
%
% \section{Digesting keys}
%
% \begin{function}[added = 2022-03-09]{\keys_precompile:nnN}
%   \begin{syntax}
%     \cs{keys_precompile:nnN} \Arg{module} \Arg{keyval list} \meta{tl}
%   \end{syntax}
%   Parses the \meta{keyval list} as for \cs{keys_set:nn}, placing the
%   resulting code for those which set variables or functions into the
%   \meta{tl}. Thus this function \enquote{precompiles} the keyval list into
%   a set of results which can be applied rapidly.
% \end{function}
%
% \section{Utility functions for keys}
%
% \begin{function}[EXP, pTF, updated = 2022-01-10]
%   {\keys_if_exist:nn, \keys_if_exist:ne}
%   \begin{syntax}
%     \cs{keys_if_exist_p:nn} \Arg{module} \Arg{key} \\
%     \cs{keys_if_exist:nnTF} \Arg{module} \Arg{key} \Arg{true code} \Arg{false code}
%   \end{syntax}
%   Tests if the \meta{key} exists for \meta{module}, \emph{i.e.}~if any code
%   has been defined for \meta{key}.
% \end{function}
%
% \begin{function}[added = 2011-08-21,EXP,pTF, updated = 2017-11-14]
%   {\keys_if_choice_exist:nnn}
%   \begin{syntax}
%     \cs{keys_if_choice_exist_p:nnn} \Arg{module} \Arg{key} \Arg{choice} \\
%     \cs{keys_if_choice_exist:nnnTF} \Arg{module} \Arg{key} \Arg{choice} \Arg{true code} \Arg{false code}
%   \end{syntax}
%   Tests if the \meta{choice} is defined for the \meta{key} within the
%   \meta{module}, \emph{i.e.}~if any code has been defined for
%   \meta{key}/\meta{choice}. The test is \texttt{false} if the \meta{key}
%   itself is not defined.
% \end{function}
%
% \begin{function}[updated = 2015-08-09]{\keys_show:nn}
%   \begin{syntax}
%     \cs{keys_show:nn} \Arg{module} \Arg{key}
%   \end{syntax}
%   Displays in the terminal
%   the information associated to the \meta{key} for a \meta{module},
%   including the function which is used to actually implement it.
% \end{function}
%
% \begin{function}[added = 2014-08-22, updated = 2015-08-09]{\keys_log:nn}
%   \begin{syntax}
%     \cs{keys_log:nn} \Arg{module} \Arg{key}
%   \end{syntax}
%   Writes in the log file the information associated to the \meta{key}
%   for a \meta{module}.  See also \cs{keys_show:nn} which displays the
%   result in the terminal.
% \end{function}
%
% \section{Low-level interface for parsing key--val lists}
%
% To re-cap from earlier, a key--value list is input of the form
% \begin{verbatim}
%   KeyOne = ValueOne ,
%   KeyTwo = ValueTwo ,
%   KeyThree
% \end{verbatim}
% where each key--value pair is separated by a comma from the rest of
% the list, and each key--value pair does not necessarily contain an
% equals sign or a value! Processing this type of input correctly
% requires a number of careful steps, to correctly account for
% braces, spaces and the category codes of separators.
%
% While the functions described earlier are used as a high-level interface
% for processing such input, in special circumstances you may wish to use
% a lower-level approach.
% The low-level parsing system converts a \meta{key--value list}
% into \meta{keys} and associated \meta{values}. After the parsing phase
% is completed, the resulting keys and values (or keys alone) are
% available for further processing. This processing is not carried out by the
% low-level parser itself, and so the parser requires the names of
% two functions along with the key--value list. One function is
% needed to process key--value pairs (it receives two arguments),
% and a second function is required for keys given without any value
% (it is called with a single argument).
%
% The parser does not double |#| tokens or expand any input. Active
% tokens |=| and |,| appearing at the outer level of braces are converted
% to category \enquote{other} (12) so that the parser does not \enquote{miss}
% any due to category code changes. Spaces are removed from the ends
% of the keys and values. Keys and values which are given in braces
% have exactly one set removed (after space trimming), thus
% \begin{verbatim}
%    key = {value here},
% \end{verbatim}
% and
% \begin{verbatim}
%   key = value here,
% \end{verbatim}
% are treated identically.
%
% \begin{function}[rEXP, added=2020-12-19, updated = 2021-05-10]{\keyval_parse:nnn}
%   \begin{syntax}
%     \cs{keyval_parse:nnn} \Arg{code_1} \Arg{code_2} \Arg{key--value list}
%   \end{syntax}
%   Parses the \meta{key--value list} into a series of \meta{keys} and
%   associated \meta{values}, or keys alone (if no \meta{value} was
%   given).  \meta{code_1} receives each \meta{key} (with no \meta{value}) as a
%   trailing brace group, whereas \meta{code_2} is appended by two brace groups,
%   the \meta{key} and \meta{value}.
%   The order of the \meta{keys} in the \meta{key--value list}
%   is preserved. Thus
%   \begin{verbatim}
%     \keyval_parse:nnn
%       { \use_none:nn  { code 1 } }
%       { \use_none:nnn { code 2 } }
%       { key1 = value1 , key2 = value2, key3 = , key4 }
%   \end{verbatim}
%   is converted into an input stream
%   \begin{verbatim}
%     \use_none:nnn { code 2 } { key1 } { value1 }
%     \use_none:nnn { code 2 } { key2 } { value2 }
%     \use_none:nnn { code 2 } { key3 } { }
%     \use_none:nn  { code 1 } { key4 }
%   \end{verbatim}
%   Note that there is a difference between an empty value (an equals
%   sign followed by nothing) and a missing value (no equals sign at
%   all). Spaces are trimmed from the ends of the \meta{key} and \meta{value},
%   then one \emph{outer} set of braces is removed from the \meta{key}
%   and \meta{value} as part of the processing. If you need exactly the output
%   shown above, you'll need to either \texttt{x}-type or \texttt{e}-type expand
%   the function.
%   \begin{texnote}
%     The result of each list element is returned within \cs{exp_not:n}, which
%     means that the converted input stream does not expand further when
%     appearing in an \texttt{x}-type or \texttt{e}-type argument expansion.
%   \end{texnote}
% \end{function}
%
% \begin{function}[rEXP, updated = 2021-05-10]{\keyval_parse:NNn}
%   \begin{syntax}
%     \cs{keyval_parse:NNn} \meta{function_1} \meta{function_2} \Arg{key--value list}
%   \end{syntax}
%   Parses the \meta{key--value list} into a series of \meta{keys} and
%   associated \meta{values}, or keys alone (if no \meta{value} was
%   given).  \meta{function_1} should take one argument, while
%   \meta{function_2} should absorb two arguments. After
%   \cs{keyval_parse:NNn} has parsed the \meta{key--value list},
%   \meta{function_1} is used to process keys given with no value
%   and \meta{function_2} is used to process keys given with a
%   value. The order of the \meta{keys} in the \meta{key--value list}
%   is preserved. Thus
%   \begin{verbatim}
%     \keyval_parse:NNn \function:n \function:nn
%       { key1 = value1 , key2 = value2, key3 = , key4 }
%   \end{verbatim}
%   is converted into an input stream
%   \begin{verbatim}
%     \function:nn { key1 } { value1 }
%     \function:nn { key2 } { value2 }
%     \function:nn { key3 } { }
%     \function:n  { key4 }
%   \end{verbatim}
%   Note that there is a difference between an empty value (an equals
%   sign followed by nothing) and a missing value (no equals sign at
%   all). Spaces are trimmed from the ends of the \meta{key} and \meta{value},
%   then one \emph{outer} set of braces is removed from the \meta{key}
%   and \meta{value} as part of the processing.
%
%   This shares the implementation of \cs{keyval_parse:nnn}, the difference is
%   only semantically.
%   \begin{texnote}
%     The result is returned within \cs{exp_not:n}, which means that the
%     converted input stream does not expand further when appearing in an
%     \texttt{x}-type or \texttt{e}-type argument expansion.
%   \end{texnote}
% \end{function}
%
% \end{documentation}
%
% \begin{implementation}
%
% \section{\pkg{l3keys} implementation}
%
%    \begin{macrocode}
%<*package>
%    \end{macrocode}
%
% \subsection{Low-level interface}
%
% The low-level key parser's implementation is based heavily on \pkg{expkv}.
% Compared to \pkg{keyval} it adds a number of additional \enquote{safety}
% requirements and allows to process the parsed list of key--value pairs in a
% variety of ways.  The net result is that this code needs around one and a half
% the amount of time as \pkg{keyval} to parse the same list of keys. To optimise
% speed as far as reasonably practical, a number of lower-level approaches are
% taken rather than using the higher-level \pkg{expl3} interfaces.
%
%    \begin{macrocode}
%<@@=keyval>
%    \end{macrocode}
%
% \begin{variable}{\s_@@_nil,\s_@@_mark,\s_@@_stop,\s_@@_tail}
%    \begin{macrocode}
\scan_new:N \s_@@_nil
\scan_new:N \s_@@_mark
\scan_new:N \s_@@_stop
\scan_new:N \s_@@_tail
%    \end{macrocode}
% \end{variable}
%
% \begin{variable}{\l__kernel_keyval_allow_blank_keys_bool}
%   The general behavior of the \pkg{l3keys} module is to throw an error on
%   blank key names. However to support the usage of \cs{keyval_parse:nnn} in
%   the \pkg{l3prop} module we allow this error to be switched off temporarily
%   and just ignore blank names.
%    \begin{macrocode}
\bool_new:N \l__kernel_keyval_allow_blank_keys_bool
%    \end{macrocode}
% \end{variable}
%
%   This temporary macro will be used since some of the definitions will need an
%   active comma or equals sign. Inside of this macro |#1| will be the active
%   comma and |#2| will be the active equals sign.
%    \begin{macrocode}
\group_begin:
  \cs_set_protected:Npn \@@_tmp:w #1#2
    {
%    \end{macrocode}
%
% \begin{macro}[EXP]{\keyval_parse:nnn, \keyval_parse:NNn}
%   The main function starts the first of two loops. The outer loop splits the
%   key--value list at active commas, the inner loop will do so at other commas.
%   The use of \cs{s_@@_mark} here prevents loss of braces from the key
%   argument.
%    \begin{macrocode}
      \cs_new:Npn \keyval_parse:nnn ##1 ##2 ##3
        {
          \__kernel_exp_not:w \tex_expanded:D
            {
              {
                \@@_loop_active:nnw {##1} {##2}
                  \s_@@_mark ##3 #1 \s_@@_tail #1
              }
            }
        }
      \cs_new_eq:NN \keyval_parse:NNn \keyval_parse:nnn
%    \end{macrocode}
% \end{macro}
%
% \begin{macro}[EXP]{\@@_loop_active:nnw}
%   First a fast test for the end of the loop is done, it'll gobble everything
%   up to a \cs{s_@@_tail}. The loop ending macro will gobble everything to the
%   last comma in this definition.
%   If the end isn't reached yet, start the second loop splitting at other
%   commas, the next iteration of this first loop will be inserted by the end of
%   \cs{@@_loop_other:nnw}.
%    \begin{macrocode}
      \cs_new:Npn \@@_loop_active:nnw ##1 ##2 ##3 #1
        {
          \@@_if_recursion_tail:w ##3
            \@@_end_loop_active:w \s_@@_tail
          \@@_loop_other:nnw {##1} {##2} ##3 , \s_@@_tail ,
        }
%    \end{macrocode}
% \end{macro}
%
% \begin{macro}[EXP]{\@@_split_other:w, \@@_split_active:w}
%   These two macros allow to split at the first equals sign of category 12 or
%   13. At the same time they also execute branching by inserting the first
%   token following \cs{s_@@_mark} that followed the equals sign. Hence they
%   also test for the presence of such an equals sign simultaneously.
%    \begin{macrocode}
      \cs_new:Npn \@@_split_other:w ##1 = ##2 \s_@@_mark ##3
        { ##3 ##1 \s_@@_stop \s_@@_mark ##2 }
      \cs_new:Npn \@@_split_active:w ##1 #2 ##2 \s_@@_mark ##3
        { ##3 ##1 \s_@@_stop \s_@@_mark ##2 }
%    \end{macrocode}
% \end{macro}
%
% \begin{macro}[EXP]{\@@_loop_other:nnw}
%   The second loop uses the same test for its end as the first loop, next it
%   splits at the first active equals sign using \cs{@@_split_active:w}.  The
%   \cs{s_@@_nil} prevents accidental brace stripping and acts as a delimiter in
%   the next steps. First testing for an active equals sign will reduce the
%   number of necessary expansion steps for the expected average use case of
%   other equals signs and hence perform better on average.
%    \begin{macrocode}
      \cs_new:Npn \@@_loop_other:nnw ##1 ##2 ##3 ,
        {
          \@@_if_recursion_tail:w ##3
            \@@_end_loop_other:w \s_@@_tail
          \@@_split_active:w ##3 \s_@@_nil
            \s_@@_mark \@@_split_active_auxi:w
            #2 \s_@@_mark \@@_clean_up_active:w
          {##1} {##2}
          \s_@@_mark
        }
%    \end{macrocode}
% \end{macro}
%
% \begin{macro}[EXP]{\@@_split_active_auxi:w}
% \begin{macro}[EXP]{\@@_split_active_auxii:w}
% \begin{macro}[EXP]{\@@_split_active_auxiii:w}
% \begin{macro}[EXP]{\@@_split_active_auxiv:w}
% \begin{macro}[EXP]{\@@_split_active_auxv:w}
%   After \cs{@@_split_active:w} the following will only be called if there was
%   at least one active equals sign in the current key--value pair. Therefore
%   this is the execution branch for a key--value pair with an active equals
%   sign. |##1| will be everything up to the first active equals sign. First it
%   tests for other equals signs in the key name, which will eventually throw an
%   error via \cs{@@_misplaced_equal_after_active_error:w}. If none was found we
%   forward the key to \cs{@@_split_active_auxii:w}.
%    \begin{macrocode}
      \cs_new:Npn \@@_split_active_auxi:w ##1 \s_@@_stop
        {
          \@@_split_other:w ##1 \s_@@_nil
            \s_@@_mark \@@_misplaced_equal_after_active_error:w
            = \s_@@_mark \@@_split_active_auxii:w
        }
%    \end{macrocode}
%   \cs{@@_split_active_auxii:w} gets the correct key name with a leading
%   \cs{s_@@_mark} as |##1|. It has to sanitise the remainder of the previous
%   test and trims the key name which will be forwarded to
%   \cs{@@_split_active_auxiii:w}.
%    \begin{macrocode}
      \cs_new:Npn \@@_split_active_auxii:w
          ##1 \s_@@_nil \s_@@_mark \@@_misplaced_equal_after_active_error:w
          \s_@@_stop \s_@@_mark
          ##2 \s_@@_nil #2 \s_@@_mark \@@_clean_up_active:w
        { \@@_trim:nN {##1} \@@_split_active_auxiii:w ##2 \s_@@_nil }
%    \end{macrocode}
%   Next we test for a misplaced active equals sign in the value, if none is
%   found \cs{@@_split_active_auxiv:w} will be called.
%    \begin{macrocode}
      \cs_new:Npn \@@_split_active_auxiii:w ##1 ##2 \s_@@_nil
        {
          \@@_split_active:w ##2 \s_@@_nil
            \s_@@_mark \@@_misplaced_equal_in_split_error:w
            #2 \s_@@_mark \@@_split_active_auxiv:w
            {##1}
        }
%    \end{macrocode}
%   This runs the last test after sanitising the remainder of the previous one.
%   This time test for a misplaced equals sign of category 12 in the value.
%   Finally the last auxiliary macro will be called.
%    \begin{macrocode}
      \cs_new:Npn \@@_split_active_auxiv:w
          ##1 \s_@@_nil \s_@@_mark \@@_misplaced_equal_in_split_error:w
          \s_@@_stop \s_@@_mark
        {
          \@@_split_other:w ##1 \s_@@_nil
            \s_@@_mark \@@_misplaced_equal_in_split_error:w
            = \s_@@_mark \@@_split_active_auxv:w
        }
%    \end{macrocode}
%   This last macro in this execution branch sanitises the last test, trims the
%   value and passes it to \cs{@@_pair:nnnn}.
%    \begin{macrocode}
      \cs_new:Npn \@@_split_active_auxv:w
          ##1 \s_@@_nil \s_@@_mark \@@_misplaced_equal_in_split_error:w
          \s_@@_stop \s_@@_mark
        { \@@_trim:nN { ##1 } \@@_pair:nnnn }
%    \end{macrocode}
% \end{macro}
% \end{macro}
% \end{macro}
% \end{macro}
% \end{macro}
%
% \begin{macro}[EXP]{\@@_clean_up_active:w}
%   The following is the branch taken if the key--value pair doesn't contain an
%   active equals sign. The remainder of that test will be cleaned up by
%   \cs{@@_clean_up_active:w} which will then split at an equals sign of
%   category other.
%    \begin{macrocode}
      \cs_new:Npn \@@_clean_up_active:w
          ##1 \s_@@_nil \s_@@_mark \@@_split_active_auxi:w \s_@@_stop \s_@@_mark
        {
          \@@_split_other:w ##1 \s_@@_nil
            \s_@@_mark \@@_split_other_auxi:w
            = \s_@@_mark \@@_clean_up_other:w
        }
%    \end{macrocode}
% \end{macro}
%
% \begin{macro}[EXP]{\@@_split_other_auxi:w}
% \begin{macro}[EXP]{\@@_split_other_auxii:w}
% \begin{macro}[EXP]{\@@_split_other_auxiii:w}
%   This is executed if the key--value pair doesn't contain an active equals
%   sign but at least one other. |##1| of \cs{@@_split_other_auxi:w} will
%   contain the complete key name, which is trimmed and forwarded to the next
%   auxiliary macro.
%    \begin{macrocode}
      \cs_new:Npn \@@_split_other_auxi:w ##1 \s_@@_stop
        { \@@_trim:nN { ##1 } \@@_split_other_auxii:w }
%    \end{macrocode}
%   We know that the value doesn't contain misplaced active equals signs but we
%   have to test for others. Also we need to sanitise the previous test, which
%   is done here and not earlier to avoid superfluous argument grabbing.
%    \begin{macrocode}
      \cs_new:Npn \@@_split_other_auxii:w
          ##1 ##2 \s_@@_nil = \s_@@_mark \@@_clean_up_other:w
        {
          \@@_split_other:w ##2 \s_@@_nil
            \s_@@_mark \@@_misplaced_equal_in_split_error:w
            = \s_@@_mark \@@_split_other_auxiii:w
            { ##1 }
        }
%    \end{macrocode}
%   \cs{@@_split_other_auxiii:w} sanitises the test for other equals signs,
%   trims the value and forwards it to \cs{@@_pair:nnnn}.
%    \begin{macrocode}
      \cs_new:Npn \@@_split_other_auxiii:w
          ##1 \s_@@_nil \s_@@_mark \@@_misplaced_equal_in_split_error:w
          \s_@@_stop \s_@@_mark
        { \@@_trim:nN { ##1 } \@@_pair:nnnn }
%    \end{macrocode}
% \end{macro}
% \end{macro}
% \end{macro}
%
% \begin{macro}[EXP]{\@@_clean_up_other:w}
%   \cs{@@_clean_up_other:w} is the last branch that might exist. It is called
%   if no equals sign was found, hence the only possibilities left are a blank
%   list element, which is to be skipped, or a lonely key. If it's no empty list
%   element this will trim the key name and forward it to \cs{@@_key:nn}.
%    \begin{macrocode}
      \cs_new:Npn \@@_clean_up_other:w
          ##1 \s_@@_nil \s_@@_mark \@@_split_other_auxi:w \s_@@_stop \s_@@_mark
        {
          \@@_if_blank:w ##1 \s_@@_nil \s_@@_stop \@@_blank_true:w
            \s_@@_mark \s_@@_stop
            \@@_trim:nN { ##1 } \@@_key:nn
        }
%    \end{macrocode}
% \end{macro}
%
% \begin{macro}[EXP]{\@@_misplaced_equal_after_active_error:w}
% \begin{macro}[EXP]{\@@_misplaced_equal_in_split_error:w}
%   All these two macros do is gobble the remainder of the current other loop
%   execution and throw an error. Afterwards they have to insert the next loop
%   iteration.
%    \begin{macrocode}
      \cs_new:Npn \@@_misplaced_equal_after_active_error:w
          \s_@@_mark ##1 \s_@@_stop \s_@@_mark ##2 \s_@@_nil
           = \s_@@_mark \@@_split_active_auxii:w
           \s_@@_mark ##3 \s_@@_nil
           #2 \s_@@_mark \@@_clean_up_active:w
        {
          \msg_expandable_error:nn
            { keyval } { misplaced-equals-sign }
          \@@_loop_other:nnw
        }
      \cs_new:Npn \@@_misplaced_equal_in_split_error:w
          \s_@@_mark ##1 \s_@@_stop \s_@@_mark ##2 \s_@@_nil
          ##3 \s_@@_mark ##4 ##5
        {
          \msg_expandable_error:nn
            { keyval } { misplaced-equals-sign }
          \@@_loop_other:nnw
        }
%    \end{macrocode}
% \end{macro}
% \end{macro}
%
% \begin{macro}[EXP]{\@@_end_loop_other:w, \@@_end_loop_active:w}
%   All that's left for the parsing loops are the macros which end the
%   recursion. Both just gobble the remaining tokens of the respective loop
%   including the next recursion call. \cs{@@_end_loop_other:w} also has to
%   insert the next iteration of the active loop.
%    \begin{macrocode}
      \cs_new:Npn \@@_end_loop_other:w
          \s_@@_tail
          \@@_split_active:w
          \s_@@_mark \s_@@_tail
          \s_@@_nil \s_@@_mark
          \@@_split_active_auxi:w
          #2 \s_@@_mark \@@_clean_up_active:w
        { \@@_loop_active:nnw }
      \cs_new:Npn \@@_end_loop_active:w
          \s_@@_tail
          \@@_loop_other:nnw ##1 \s_@@_mark \s_@@_tail , \s_@@_tail ,
        { }
%    \end{macrocode}
% \end{macro}
%
% The parsing loops are done, so here ends the definition of \cs{@@_tmp:w},
% which will finally set up the macros.
%    \begin{macrocode}
    }
  \char_set_catcode_active:n { `\, }
  \char_set_catcode_active:n { `\= }
  \@@_tmp:w , =
\group_end:
%    \end{macrocode}
%
% \begin{macro}[EXP]{\@@_pair:nnnn, \@@_key:nn}
%   These macros will be called on the parsed keys and values of the key--value
%   list. All arguments are completely trimmed. They test for blank key names
%   and call the functions passed to \cs{keyval_parse:nnn} inside of
%   \cs{exp_not:n} with the correct arguments. Afterwards they insert the next
%   iteration of the other loop.
%    \begin{macrocode}
\group_begin:
  \cs_set_protected:Npn \@@_tmp:w #1#2
    {
      \cs_new:Npn \@@_pair:nnnn ##1 ##2 ##3 ##4
        {
          \@@_if_blank:w \s_@@_mark ##2 \s_@@_nil \s_@@_stop \@@_blank_key_error:w
            \s_@@_mark \s_@@_stop
          #1
          \exp_not:n { ##4 {##2} {##1} }
          #2
          \@@_loop_other:nnw {##3} {##4}
        }
      \cs_new:Npn \@@_key:nn ##1 ##2
        {
          \@@_if_blank:w \s_@@_mark ##1 \s_@@_nil \s_@@_stop \@@_blank_key_error:w
            \s_@@_mark \s_@@_stop
          #1
          \exp_not:n { ##2 {##1} }
          #2
          \@@_loop_other:nnw {##2}
        }
    }
  \@@_tmp:w { } { }
\group_end:
%    \end{macrocode}
% \end{macro}
%
% \begin{macro}[EXP]{\@@_if_empty:w,\@@_if_blank:w,\@@_if_recursion_tail:w}
%   All these tests work by gobbling tokens until a certain combination is met,
%   which makes them pretty fast. The test for a blank argument should be called
%   with an arbitrary token following the argument. Each of these utilize the
%   fact that the argument will contain a leading \cs{s_@@_mark}.
%    \begin{macrocode}
\cs_new:Npn \@@_if_empty:w #1 \s_@@_mark \s_@@_stop { }
\cs_new:Npn \@@_if_blank:w \s_@@_mark #1 { \@@_if_empty:w \s_@@_mark }
\cs_new:Npn \@@_if_recursion_tail:w \s_@@_mark #1 \s_@@_tail { }
%    \end{macrocode}
% \end{macro}
%
% \begin{macro}[EXP]{\@@_blank_true:w,\@@_blank_key_error:w}
%   These macros will be called if the tests above didn't gobble them, they
%   execute the branching.
%    \begin{macrocode}
\cs_new:Npn \@@_blank_true:w \s_@@_mark \s_@@_stop \@@_trim:nN #1 \@@_key:nn
  { \@@_loop_other:nnw }
\cs_new:Npn \@@_blank_key_error:w \s_@@_mark \s_@@_stop #1 \@@_loop_other:nnw
  {
    \bool_if:NTF \l__kernel_keyval_allow_blank_keys_bool
      { #1 }
      { \msg_expandable_error:nn { keyval } { blank-key-name } }
    \@@_loop_other:nnw
  }
%    \end{macrocode}
% \end{macro}
%
% Two messages for the low level parsing system.
%    \begin{macrocode}
\msg_new:nnn { keyval } { misplaced-equals-sign }
  { Misplaced~'='~in~key-value~input~\msg_line_context: }
\msg_new:nnn { keyval } { blank-key-name }
  { Blank~key~name~in~key-value~input~\msg_line_context: }
\prop_gput:Nnn \g_msg_module_name_prop { keyval } { LaTeX }
\prop_gput:Nnn \g_msg_module_type_prop { keyval } { }
%    \end{macrocode}
%
% \begin{macro}[EXP]{\@@_trim:nN}
% \begin{macro}[EXP]
%   {\@@_trim_auxi:w,\@@_trim_auxii:w,\@@_trim_auxiii:w,\@@_trim_auxiv:w}
% And an adapted version of \cs{__tl_trim_spaces:nn} which is a bit faster for
% our use case, as it can strip the braces at the end. This is pretty much the
% same concept, so I won't comment on it here. The speed gain by using this
% instead of \cs{tl_trim_spaces_apply:nN} is about 10\,\% of the total time for
% \cs{keyval_parse:NNn} with one key and one key--value pair, so I think it's
% worth it.
%    \begin{macrocode}
\group_begin:
  \cs_set_protected:Npn \@@_tmp:w #1
    {
      \cs_new:Npn \@@_trim:nN ##1
        {
          \@@_trim_auxi:w
            ##1
            \s_@@_nil
            \s_@@_mark #1 { }
            \s_@@_mark \@@_trim_auxii:w
            \@@_trim_auxiii:w
            #1 \s_@@_nil
            \@@_trim_auxiv:w
        }
      \cs_new:Npn \@@_trim_auxi:w ##1 \s_@@_mark #1 ##2 \s_@@_mark ##3
        {
          ##3
          \@@_trim_auxi:w
          \s_@@_mark
          ##2
          \s_@@_mark #1 {##1}
        }
      \cs_new:Npn \@@_trim_auxii:w \@@_trim_auxi:w \s_@@_mark \s_@@_mark ##1
        {
          \@@_trim_auxiii:w
          ##1
        }
      \cs_new:Npn \@@_trim_auxiii:w ##1 #1 \s_@@_nil ##2
        {
          ##2
          ##1 \s_@@_nil
          \@@_trim_auxiii:w
        }
%    \end{macrocode}
%   This is the one macro which differs from the original definition.
%    \begin{macrocode}
      \cs_new:Npn \@@_trim_auxiv:w
          \s_@@_mark ##1 \s_@@_nil
          \@@_trim_auxiii:w \s_@@_nil \@@_trim_auxiii:w
          ##2
        { ##2 { ##1 } }
    }
  \@@_tmp:w { ~ }
\group_end:
%    \end{macrocode}
% \end{macro}
% \end{macro}
%
% \subsection{Constants and variables}
%
%    \begin{macrocode}
%<@@=keys>
%    \end{macrocode}
%
% \begin{variable}
%   {
%     \c_@@_code_root_str    ,
%     \c_@@_check_root_str   ,
%     \c_@@_default_root_str ,
%     \c_@@_groups_root_str  ,
%     \c_@@_inherit_root_str ,
%     \c_@@_type_root_str
%   }
%   Various storage areas for the different data which make up keys.
%    \begin{macrocode}
\str_const:Nn \c_@@_code_root_str     { key~code~>~ }
\str_const:Nn \c_@@_check_root_str    { key~check~>~ }
\str_const:Nn \c_@@_default_root_str  { key~default~>~ }
\str_const:Nn \c_@@_groups_root_str   { key~groups~>~ }
\str_const:Nn \c_@@_inherit_root_str  { key~inherit~>~ }
\str_const:Nn \c_@@_type_root_str     { key~type~>~ }
%    \end{macrocode}
% \end{variable}
%
% \begin{variable}{\c_@@_props_root_str}
%   The prefix for storing properties.
%    \begin{macrocode}
\str_const:Nn \c_@@_props_root_str { key~prop~>~ }
%    \end{macrocode}
% \end{variable}
%
% \begin{variable}{\l_keys_choice_int, \l_keys_choice_tl}
%   Publicly accessible data on which choice is being used when several
%   are generated as a set.
%    \begin{macrocode}
\int_new:N \l_keys_choice_int
\tl_new:N \l_keys_choice_tl
%    \end{macrocode}
% \end{variable}
%
% \begin{variable}{\l_@@_groups_clist}
%   Used for storing and recovering the list of groups which apply to a key:
%   set as a comma list but at one point we have to use this for a token
%   list recovery.
%    \begin{macrocode}
\clist_new:N \l_@@_groups_clist
%    \end{macrocode}
% \end{variable}
%
% \begin{variable}{\l_keys_key_str, \l_keys_key_tl}
%   The name of a key itself: needed when setting keys. The |tl| version
%   is deprecated but has to be handled manually.
%    \begin{macrocode}
\str_new:N \l_keys_key_str
\tl_new:N \l_keys_key_tl
%    \end{macrocode}
% \end{variable}
%
% \begin{variable}{\l_@@_module_str}
%   The module for an entire set of keys.
%    \begin{macrocode}
\str_new:N \l_@@_module_str
%    \end{macrocode}
% \end{variable}
%
% \begin{variable}{\l_@@_no_value_bool}
%   A marker is needed internally to show if only a key or a key plus a
%   value was seen: this is recorded here.
%    \begin{macrocode}
\bool_new:N \l_@@_no_value_bool
%    \end{macrocode}
% \end{variable}
%
% \begin{variable}{\l_@@_only_known_bool}
%   Used to track if only \enquote{known} keys are being set.
%    \begin{macrocode}
\bool_new:N \l_@@_only_known_bool
%    \end{macrocode}
% \end{variable}
%
% \begin{variable}{\l_keys_path_str, \l_keys_path_tl}
%   The \enquote{path} of the current key is stored here: this is
%   available to the programmer and so is public. The older version is
%   deprecated but has to be handled manually.
%    \begin{macrocode}
\str_new:N \l_keys_path_str
\tl_new:N \l_keys_path_tl
%    \end{macrocode}
% \end{variable}
%
% \begin{variable}{\l_@@_inherit_str}
%    \begin{macrocode}
\str_new:N \l_@@_inherit_str
%    \end{macrocode}
% \end{variable}
%
% \begin{variable}{\l_@@_relative_tl}
%   The relative path for passing keys back to the user. As this can
%   be explicitly no-value, it must be a token list.
%    \begin{macrocode}
\tl_new:N \l_@@_relative_tl
\tl_set:Nn \l_@@_relative_tl { \q_@@_no_value }
%    \end{macrocode}
% \end{variable}
%
% \begin{variable}{\l_@@_property_str}
%   The \enquote{property} begin set for a key at definition time is
%   stored here.
%    \begin{macrocode}
\str_new:N \l_@@_property_str
%    \end{macrocode}
% \end{variable}
%
% \begin{variable}{\l_@@_selective_bool, \l_@@_filtered_bool}
%   Two flags for using key groups: one to indicate that \enquote{selective}
%   setting is active, a second to specify which type (\enquote{opt-in}
%   or \enquote{opt-out}).
%    \begin{macrocode}
\bool_new:N \l_@@_selective_bool
\bool_new:N \l_@@_filtered_bool
%    \end{macrocode}
% \end{variable}
%
% \begin{variable}{\l_@@_selective_seq}
%   The list of key groups being filtered in or out during selective setting.
%    \begin{macrocode}
\seq_new:N \l_@@_selective_seq
%    \end{macrocode}
% \end{variable}
%
% \begin{variable}{\l_@@_unused_clist}
%   Used when setting only some keys to store those left over.
%    \begin{macrocode}
\tl_new:N \l_@@_unused_clist
%    \end{macrocode}
% \end{variable}
%
% \begin{variable}{\l_keys_value_tl}
%   The value given for a key: may be empty if no value was given.
%    \begin{macrocode}
\tl_new:N \l_keys_value_tl
%    \end{macrocode}
% \end{variable}
%
% \begin{variable}{\l_@@_tmp_bool, \l_@@_tmpa_tl, \l_@@_tmpb_tl}
%   Scratch space.
%    \begin{macrocode}
\bool_new:N \l_@@_tmp_bool
\tl_new:N \l_@@_tmpa_tl
\tl_new:N \l_@@_tmpb_tl
%    \end{macrocode}
% \end{variable}
%
% \begin{variable}{\l_@@_precompile_bool, \l_@@_precompile_tl}
%   For digesting keys.
%    \begin{macrocode}
\bool_new:N \l_@@_precompile_bool
\tl_new:N \l_@@_precompile_tl
%    \end{macrocode}
% \end{variable}
%
% \begin{variable}{\l_keys_usage_load_prop, \l_keys_usage_preamble_prop}
%   Global data for document-level information.
%    \begin{macrocode}
\prop_new:N \l_keys_usage_load_prop
\prop_new:N \l_keys_usage_preamble_prop
%    \end{macrocode}
% \end{variable}
%
% \subsubsection{Internal auxiliaries}
%
% \begin{variable}{\s_@@_nil,\s_@@_mark,\s_@@_stop}
%   Internal scan marks.
%    \begin{macrocode}
\scan_new:N \s_@@_nil
\scan_new:N \s_@@_mark
\scan_new:N \s_@@_stop
%    \end{macrocode}
% \end{variable}
%
% \begin{variable}{\q_@@_no_value}
%   Internal quarks.
%    \begin{macrocode}
\quark_new:N \q_@@_no_value
%    \end{macrocode}
% \end{variable}
%
% \begin{macro}[pTF]{\@@_quark_if_no_value:N}
%   Branching quark conditional.
%    \begin{macrocode}
\__kernel_quark_new_conditional:Nn \@@_quark_if_no_value:N { TF }
%    \end{macrocode}
% \end{macro}
%
% \begin{macro}{\@@_precompile:n}
%   An auxiliary to allow cleaner showing of code.
%    \begin{macrocode}
\cs_new_protected:Npn \@@_precompile:n #1
  {
    \bool_if:NTF \l_@@_precompile_bool
      { \tl_put_right:Nn \l_@@_precompile_tl }
      { \use:n }
        {#1}
  }
%    \end{macrocode}
% \end{macro}
%
% \subsection{The key defining mechanism}
%
% \begin{macro}{\keys_define:nn}
% \begin{macro}{\@@_define:nnn, \@@_define:onn}
%   The public function for definitions is just a wrapper for the lower
%   level mechanism, more or less. The outer function is designed to
%   keep a track of the current module, to allow safe nesting. The module is set
%   removing any leading |/| (which is not needed here).
%    \begin{macrocode}
\cs_new_protected:Npn \keys_define:nn
  { \@@_define:onn \l_@@_module_str }
\cs_new_protected:Npn \@@_define:nnn #1#2#3
  {
    \str_set:Nx \l_@@_module_str { \@@_trim_spaces:n {#2} }
    \keyval_parse:NNn \@@_define:n \@@_define:nn {#3}
    \str_set:Nn \l_@@_module_str {#1}
  }
\cs_generate_variant:Nn \@@_define:nnn { o }
%    \end{macrocode}
% \end{macro}
% \end{macro}
%
% \begin{macro}{\@@_define:n}
% \begin{macro}{\@@_define:nn}
% \begin{macro}{\@@_define_aux:nn}
%   The outer functions here record whether a value was given and then
%   converge on a common internal mechanism. There is first a search for
%   a property in the current key name, then a check to make sure it is
%   known before the code hands off to the next step.
%    \begin{macrocode}
\cs_new_protected:Npn \@@_define:n #1
  {
    \bool_set_true:N \l_@@_no_value_bool
    \@@_define_aux:nn {#1} { }
  }
\cs_new_protected:Npn \@@_define:nn #1#2
  {
    \bool_set_false:N \l_@@_no_value_bool
    \@@_define_aux:nn {#1} {#2}
  }
\cs_new_protected:Npn \@@_define_aux:nn #1#2
  {
    \@@_property_find:n {#1}
    \cs_if_exist:cTF { \c_@@_props_root_str \l_@@_property_str }
      { \@@_define_code:n {#2} }
      {
        \str_if_empty:NF \l_@@_property_str
          {
            \msg_error:nnxx { keys } { property-unknown }
              \l_@@_property_str \l_keys_path_str
          }
      }
  }
%    \end{macrocode}
% \end{macro}
% \end{macro}
% \end{macro}
%
% \begin{macro}{\@@_property_find:n}
% \begin{macro}
%   {
%     \@@_property_find_auxi:w   ,
%     \@@_property_find_auxii:w  ,
%     \@@_property_find_auxiii:w ,
%     \@@_property_find_auxiv:w
%   }
%   Searching for a property means finding the last |.| in the input,
%   and storing the text before and after it. Everything is turned into
%   strings, so there is no problem using an \texttt{x}-type expansion. Since
%   \cs{@@_trim_spaces:n} will turn its argument into a string anyway, this
%   function uses \cs{cs_set_nopar:Npx} instead of \cs{tl_set:Nx} to gain some
%   speed.
%    \begin{macrocode}
\cs_new_protected:Npn \@@_property_find:n #1
  {
    \cs_set_nopar:Npx \l_@@_property_str { \@@_trim_spaces:n { #1 } }
    \exp_after:wN \@@_property_find_auxi:w \l_@@_property_str
      \s_@@_nil \@@_property_find_auxii:w
      . \s_@@_nil \@@_property_find_err:w
  }
\cs_new_protected:Npn \@@_property_find_auxi:w #1 . #2 \s_@@_nil #3
  {
    #3 #1 \s_@@_mark #2 \s_@@_nil #3
  }
\cs_new_protected:Npn \@@_property_find_auxii:w
    #1 \s_@@_mark #2 \s_@@_nil \@@_property_find_auxii:w . \s_@@_nil
    \@@_property_find_err:w
  {
    \cs_set_nopar:Npx \l_keys_path_str
      { \str_if_empty:NF \l_@@_module_str { \l_@@_module_str / } #1 }
    \@@_property_find_auxi:w #2 \s_@@_nil \@@_property_find_auxiii:w . \s_@@_nil
      \@@_property_find_auxiv:w
  }
\cs_new_protected:Npn \@@_property_find_auxiii:w #1 \s_@@_mark
  {
    \cs_set_nopar:Npx \l_keys_path_str { \l_keys_path_str . #1 }
    \@@_property_find_auxi:w
  }
\cs_new_protected:Npn \@@_property_find_auxiv:w
    #1 \s_@@_nil \@@_property_find_auxiii:w
    \s_@@_mark \s_@@_nil \@@_property_find_auxiv:w
  {
    \cs_set_nopar:Npx \l_@@_property_str { . #1 }
    \cs_set_nopar:Npx \l_keys_path_str
      { \exp_after:wN \@@_trim_spaces:n \exp_after:wN { \l_keys_path_str } }
    \tl_set_eq:NN \l_keys_path_tl \l_keys_path_str
  }
\cs_new_protected:Npn \@@_property_find_err:w
    #1 \s_@@_nil #2 \@@_property_find_err:w
  {
    \str_clear:N \l_@@_property_str
    \msg_error:nnn { keys } { no-property } {#1}
  }
%    \end{macrocode}
% \end{macro}
% \end{macro}
%
% \begin{macro}{\@@_define_code:n}
% \begin{macro}[EXP]{\@@_define_code:w}
%   Two possible cases. If there is a value for the key, then just use
%   the function. If not, then a check to make sure there is no need for
%   a value with the property. If there should be one then complain,
%   otherwise execute it. There is no need to check for a |:| as if it
%   was missing the earlier tests would have failed.
%    \begin{macrocode}
\cs_new_protected:Npn \@@_define_code:n #1
  {
    \bool_if:NTF \l_@@_no_value_bool
      {
        \exp_after:wN \@@_define_code:w
          \l_@@_property_str \s_@@_stop
          { \use:c { \c_@@_props_root_str \l_@@_property_str } }
          {
            \msg_error:nnxx { keys } { property-requires-value }
              \l_@@_property_str \l_keys_path_str
          }
      }
      { \use:c { \c_@@_props_root_str \l_@@_property_str } {#1} }
  }
\exp_last_unbraced:NNNNo
  \cs_new:Npn \@@_define_code:w #1 \c_colon_str #2 \s_@@_stop
    { \tl_if_empty:nTF {#2} }
%    \end{macrocode}
% \end{macro}
% \end{macro}
%
% \subsection{Turning properties into actions}
%
% \begin{macro}
%   {
%     \@@_bool_set:Nn, \@@_bool_set:cn,
%     \@@_bool_set_inverse:Nn, \@@_bool_set_inverse:cn
%   }
% \begin{macro}{\@@_bool_set:Nnnn}
%   Boolean keys are really just choices, but all done by hand. The
%   second argument here is the scope: either empty or \texttt{ g } for
%   global.
%    \begin{macrocode}
\cs_new_protected:Npn \@@_bool_set:Nn #1#2
  { \@@_bool_set:Nnnn #1 {#2} { true } { false } }
\cs_generate_variant:Nn \@@_bool_set:Nn { c }
\cs_new_protected:Npn \@@_bool_set_inverse:Nn #1#2
  { \@@_bool_set:Nnnn #1 {#2} { false } { true } }
\cs_generate_variant:Nn \@@_bool_set_inverse:Nn { c }
\cs_new_protected:Npn \@@_bool_set:Nnnn #1#2#3#4
  {
    \bool_if_exist:NF #1 { \bool_new:N #1 }
    \@@_choice_make:
    \@@_cmd_set:nx { \l_keys_path_str / true }
      { \exp_not:c { bool_ #2 set_ #3 :N } \exp_not:N #1 }
    \@@_cmd_set:nx { \l_keys_path_str / false }
      { \exp_not:c { bool_ #2 set_ #4 :N } \exp_not:N #1 }
    \@@_cmd_set_direct:nn { \l_keys_path_str / unknown }
      {
        \msg_error:nnx { keys } { boolean-values-only }
          \l_keys_path_str
      }
    \@@_default_set:n { true }
  }
\cs_generate_variant:Nn \@@_bool_set:Nn { c }
%    \end{macrocode}
% \end{macro}
% \end{macro}
%
% \begin{macro}{\@@_choice_make:, \@@_multichoice_make:}
% \begin{macro}{\@@_choice_make:N}
% \begin{macro}{\@@_choice_make_aux:N}
%   To make a choice from a key, two steps: set the code, and set the
%   unknown key. As multichoices and choices are essentially the same bar one
%   function, the code is given together.
%    \begin{macrocode}
\cs_new_protected:Npn \@@_choice_make:
  { \@@_choice_make:N \@@_choice_find:n }
\cs_new_protected:Npn \@@_multichoice_make:
  { \@@_choice_make:N \@@_multichoice_find:n }
\cs_new_protected:Npn \@@_choice_make:N #1
  {
    \cs_if_exist:cTF
      { \c_@@_type_root_str \@@_parent:o \l_keys_path_str }
      {
        \str_if_eq:vnTF
          { \c_@@_type_root_str \@@_parent:o \l_keys_path_str }
          { choice }
          {
            \msg_error:nnxx { keys } { nested-choice-key }
              \l_keys_path_tl { \@@_parent:o \l_keys_path_str }
          }
          { \@@_choice_make_aux:N #1 }
      }
      { \@@_choice_make_aux:N #1 }
  }
\cs_new_protected:Npn \@@_choice_make_aux:N #1
  {
    \cs_set_nopar:cpn { \c_@@_type_root_str \l_keys_path_str }
      { choice }
    \@@_cmd_set_direct:nn \l_keys_path_str { #1 {##1} }
    \@@_cmd_set_direct:nn { \l_keys_path_str / unknown }
      {
        \msg_error:nnxx { keys } { choice-unknown }
          \l_keys_path_str {##1}
      }
  }
%    \end{macrocode}
% \end{macro}
% \end{macro}
% \end{macro}
%
% \begin{macro}{\@@_choices_make:nn, \@@_multichoices_make:nn}
% \begin{macro}{\@@_choices_make:Nnn}
%   Auto-generating choices means setting up the root key as a choice, then
%   defining each choice in turn.
%    \begin{macrocode}
\cs_new_protected:Npn \@@_choices_make:nn
  { \@@_choices_make:Nnn \@@_choice_make: }
\cs_new_protected:Npn \@@_multichoices_make:nn
  { \@@_choices_make:Nnn \@@_multichoice_make: }
\cs_new_protected:Npn \@@_choices_make:Nnn #1#2#3
  {
    #1
    \int_zero:N \l_keys_choice_int
    \clist_map_inline:nn {#2}
      {
        \int_incr:N \l_keys_choice_int
        \@@_cmd_set:nx
          { \l_keys_path_str / \@@_trim_spaces:n {##1} }
          {
            \tl_set:Nn \exp_not:N \l_keys_choice_tl {##1}
            \int_set:Nn \exp_not:N \l_keys_choice_int
              { \int_use:N \l_keys_choice_int }
            \exp_not:n {#3}
          }
      }
  }
%    \end{macrocode}
% \end{macro}
% \end{macro}
%
% \begin{macro}
%   {
%     \@@_cmd_set:nn, \@@_cmd_set:nx, \@@_cmd_set:Vn, \@@_cmd_set:Vo,
%     \@@_cmd_set_direct:nn
%   }
%   Setting the code for a key first logs if appropriate that we are
%   defining a new key, then saves the code.
%    \begin{macrocode}
\cs_new_protected:Npn \@@_cmd_set:nn #1#2
  {  \@@_cmd_set_direct:nn {#1} { \@@_precompile:n {#2} } }
\cs_generate_variant:Nn \@@_cmd_set:nn { nx , Vn , Vo }
\cs_new_protected:Npn \@@_cmd_set_direct:nn #1#2
  { \cs_set_protected:cpn { \c_@@_code_root_str #1 } ##1 {#2} }
%    \end{macrocode}
% \end{macro}
%
% \begin{macro}{\@@_cs_set:NNpn, \@@_cs_set:Ncpn}
%   Creating control sequences is a bit more tricky than other cases as
%   we need to pick up the |p| argument. To make the internals look clearer,
%   the trailing |n| argument here is just for appearance.
%    \begin{macrocode}
\cs_new_protected:Npn \@@_cs_set:NNpn #1#2#3#
  {
    \cs_set_protected:cpx { \c_@@_code_root_str \l_keys_path_str } ##1
      {
        \@@_precompile:n
          { #1 \exp_not:N #2 \exp_not:n {#3} {##1} }
      }
    \use_none:n
  }
\cs_generate_variant:Nn \@@_cs_set:NNpn { Nc }
%    \end{macrocode}
% \end{macro}
%
% \begin{macro}{\@@_default_set:n}
%   Setting a default value is easy. These are stored using \cs{cs_set_nopar:cpx} as this
%   avoids any worries about whether a token list exists.
%    \begin{macrocode}
\cs_new_protected:Npn \@@_default_set:n #1
  {
    \tl_if_empty:nTF {#1}
      {
        \cs_set_eq:cN
          { \c_@@_default_root_str \l_keys_path_str }
          \tex_undefined:D
      }
      {
        \cs_set_nopar:cpx
          { \c_@@_default_root_str \l_keys_path_str }
          { \exp_not:n {#1} }
        \@@_value_requirement:nn { required } { false }
      }
  }
%    \end{macrocode}
% \end{macro}
%
% \begin{macro}{\@@_groups_set:n}
%   Assigning a key to one or more groups uses comma lists. As the list of
%   groups only exists if there is anything to do, the setting is done using
%   a scratch list. For the usual grouping reasons we use the low-level
%   approach to undefining a list.  We also use the low-level approach for
%   the other case to avoid tripping up the |check-declarations| code.
%    \begin{macrocode}
\cs_new_protected:Npn \@@_groups_set:n #1
  {
    \clist_set:Nn \l_@@_groups_clist {#1}
    \clist_if_empty:NTF \l_@@_groups_clist
      {
        \cs_set_eq:cN { \c_@@_groups_root_str \l_keys_path_str }
          \tex_undefined:D
      }
      {
        \cs_set_eq:cN { \c_@@_groups_root_str \l_keys_path_str }
          \l_@@_groups_clist
      }
  }
%    \end{macrocode}
% \end{macro}
%
% \begin{macro}{\@@_inherit:n}
%  Inheritance means ignoring anything already said about the key:
%  zap the lot and set up.
%    \begin{macrocode}
\cs_new_protected:Npn \@@_inherit:n #1
  {
    \@@_undefine:
    \cs_set_nopar:cpn { \c_@@_inherit_root_str \l_keys_path_str } {#1}
  }
%    \end{macrocode}
% \end{macro}
%
% \begin{macro}{\@@_initialise:n}
%   A set up for initialisation: just run the code if it exists.
%   We need to set the key string here, using the deprecated \texttt{tl}
%   as a piece of scratch space.
%    \begin{macrocode}
\cs_new_protected:Npn \@@_initialise:n #1
  {
    \cs_if_exist:cTF
      { \c_@@_inherit_root_str \@@_parent:o \l_keys_path_str }
      { \@@_execute_inherit: }
      {
        \str_clear:N \l_@@_inherit_str
        \cs_if_exist:cT { \c_@@_code_root_str \l_keys_path_str }
          {
            \exp_after:wN \@@_find_key_module:wNN
              \l_keys_path_str \s_@@_stop
                \l_keys_key_tl \l_keys_key_str
            \tl_set_eq:NN \l_keys_key_tl \l_keys_key_str
            \tl_set:Nn \l_keys_value_tl {#1}
            \@@_execute:no \l_keys_path_str \l_keys_value_tl
          }
      }
  }
%    \end{macrocode}
% \end{macro}
%
% \begin{macro}{\@@_legacy_if_set:nn, \@@_legacy_if_inverse:nn}
% \begin{macro}{\@@_legacy_if_inverse:nnnn}
%   Much the same as \pkg{expl3} booleans, except we assume that the switch
%   exists.
%    \begin{macrocode}
\cs_new_protected:Npn \@@_legacy_if_set:nn #1#2
  { \@@_legacy_if_set:nnnn {#1} {#2} { true } { false } }
\cs_new_protected:Npn \@@_legacy_if_set_inverse:nn #1#2
  { \@@_legacy_if_set:nnnn {#1} {#2} { false } { true } }
\cs_new_protected:Npn \@@_legacy_if_set:nnnn #1#2#3#4
  {
    \@@_choice_make:
    \@@_cmd_set:nx { \l_keys_path_str / true }
      { \exp_not:c { legacy_if_#2  set_ #3 :n } { \exp_not:n {#1} } }
    \@@_cmd_set:nx { \l_keys_path_str / false }
      { \exp_not:c { legacy_if_#2  set_ #4 :n } { \exp_not:n {#1} } }
    \@@_cmd_set:nn { \l_keys_path_str / unknown }
      {
        \msg_error:nnx { keys } { boolean-values-only }
          \l_keys_path_str
      }
    \@@_default_set:n { true }
    \cs_if_exist:cF { if#1 }
      {
         \cs:w newif \exp_after:wN \cs_end:
           \cs:w if#1 \cs_end:
      }
  }
%    \end{macrocode}
% \end{macro}
% \end{macro}
%
% \begin{macro}{\@@_meta_make:n}
% \begin{macro}{\@@_meta_make:nn}
%   To create a meta-key, simply set up to pass data through. The internal
%   function is used here as a meta key should respect the prevailing
%   filtering, etc.
%    \begin{macrocode}
\cs_new_protected:Npn \@@_meta_make:n #1
  {
    \exp_args:NVo \@@_cmd_set_direct:nn \l_keys_path_str
      {
        \exp_after:wN \@@_set:nn \exp_after:wN 
          { \l_@@_module_str } {#1}
      }
  }
\cs_new_protected:Npn \@@_meta_make:nn #1#2
  {
    \exp_args:NV \@@_cmd_set_direct:nn
      \l_keys_path_str { \@@_set:nn {#1} {#2} }
  }
%    \end{macrocode}
% \end{macro}
% \end{macro}
%
% \begin{macro}{\@@_prop_put:Nn, \@@_prop_put:cn}
%   Much the same as other variables, but needs a dedicated auxiliary.
%    \begin{macrocode}
\cs_new_protected:Npn \@@_prop_put:Nn #1#2
  {
    \prop_if_exist:NF #1 { \prop_new:N #1 }
    \exp_after:wN \@@_find_key_module:wNN \l_keys_path_str \s_@@_stop
      \l_@@_tmpa_tl \l_@@_tmpb_tl
    \@@_cmd_set:nx \l_keys_path_str
      {
        \exp_not:c { prop_ #2 put:Nnn }
        \exp_not:N #1
        { \l_@@_tmpb_tl }
        \exp_not:n { {##1} }
      }
  }
\cs_generate_variant:Nn \@@_prop_put:Nn { c }
%    \end{macrocode}
% \end{macro}
%
% \begin{macro}{\@@_undefine:}
%   Undefining a key has to be done without \cs{cs_undefine:c} as that
%   function acts globally.
%    \begin{macrocode}
\cs_new_protected:Npn \@@_undefine:
  {
    \clist_map_inline:nn
      { code , default , groups , inherit , type , check }
      {
        \cs_set_eq:cN
          { \tl_use:c { c_@@_ ##1 _root_str } \l_keys_path_str }
          \tex_undefined:D
      }
  }
%    \end{macrocode}
% \end{macro}
%
% \begin{macro}{\@@_value_requirement:nn}
% \begin{macro}{\@@_check_forbidden:, \@@_check_required:}
%   Validating key input is done using a second function which runs before
%   the main key code. Setting that up means setting it equal to a generic
%   stub which does the check. This approach makes the lookup very fast at
%   the cost of one additional csname per key that needs it. The cleanup here
%   has to know the structure of the following code.
%    \begin{macrocode}
\cs_new_protected:Npn \@@_value_requirement:nn #1#2
  {
    \str_case:nnF {#2}
      {
        { true }
          {
            \cs_set_eq:cc
              { \c_@@_check_root_str \l_keys_path_str }
              { @@_check_ #1 : }
          }
        { false }
          {
            \cs_if_eq:ccT
              { \c_@@_check_root_str \l_keys_path_str }
              { @@_check_ #1 : }
              {
                \cs_set_eq:cN
                  { \c_@@_check_root_str \l_keys_path_str }
                  \tex_undefined:D
              }
          }
      }
      {
        \msg_error:nnx { keys }
          { boolean-values-only }
          { .value_ #1 :n }
      }
  }
\cs_new_protected:Npn \@@_check_forbidden:
  {
    \bool_if:NF \l_@@_no_value_bool
      {
        \msg_error:nnxx { keys } { value-forbidden }
          \l_keys_path_str \l_keys_value_tl
        \use_none:nnn
      }
  }
\cs_new_protected:Npn \@@_check_required:
  {
    \bool_if:NT \l_@@_no_value_bool
      {
        \msg_error:nnx { keys } { value-required }
          \l_keys_path_str
        \use_none:nnn
      }
  }
%    \end{macrocode}
% \end{macro}
% \end{macro}
%
% \begin{macro}{\@@_usage:n}
% \begin{macro}{\@@_usage:NN}
% \begin{macro}{\@@_usage:w}
%   Save the relevant data.
%    \begin{macrocode}
\cs_new_protected:Npn \@@_usage:n #1
  {
    \str_case:nnF {#1}
      {
        { general }
          {
            \@@_usage:NN \l_keys_usage_load_prop
              \c_false_bool
            \@@_usage:NN \l_keys_usage_preamble_prop
              \c_false_bool
          }
        { load }
          {
            \@@_usage:NN \l_keys_usage_load_prop
              \c_true_bool
            \@@_usage:NN \l_keys_usage_preamble_prop
              \c_false_bool
          }
        { preamble }
          {
            \@@_usage:NN \l_keys_usage_load_prop
              \c_false_bool
            \@@_usage:NN \l_keys_usage_preamble_prop
              \c_true_bool
          }
      }
      {
        \msg_error:nnnn { keys }
          { choice-unknown }
          { .usage:n }
          {#1}
      }
  }
\cs_new_protected:Npn \@@_usage:NN #1#2
  {
    \prop_get:NVNF #1 \l_@@_module_str \l_@@_tmpa_tl
      { \tl_clear:N \l_@@_tmpa_tl }
    \tl_set:Nx \l_@@_tmpb_tl
      { \exp_after:wN \@@_usage:w \l_keys_path_str \s_@@_stop }
    \bool_if:NTF #2
      { \clist_put_right:NV \l_@@_tmpa_tl \l_@@_tmpb_tl }
      { \clist_remove_all:NV \l_@@_tmpa_tl \l_@@_tmpb_tl }
    \prop_put:NVV #1 \l_@@_module_str
      \l_@@_tmpa_tl
  }
\cs_new:Npn \@@_usage:w #1 / #2 \s_@@_stop {#2}
%    \end{macrocode}
% \end{macro}
% \end{macro}
% \end{macro}
%
% \begin{macro}{\@@_variable_set:NnnN, \@@_variable_set:cnnN}
% \begin{macro}{\@@_variable_set_required:NnnN, \@@_variable_set_required:cnnN}
%   Setting a variable takes the type and scope separately so that
%   it is easy to make a new variable if needed.
%    \begin{macrocode}
\cs_new_protected:Npn \@@_variable_set:NnnN #1#2#3#4
  {
    \use:c { #2_if_exist:NF } #1 { \use:c { #2 _new:N } #1 }
    \@@_cmd_set:nx \l_keys_path_str
      {
        \exp_not:c { #2 _ #3 set:N #4 }
        \exp_not:N #1
        \exp_not:n  { {##1} }
      }
  }
\cs_generate_variant:Nn \@@_variable_set:NnnN { c }
\cs_new_protected:Npn \@@_variable_set_required:NnnN #1#2#3#4
  {
    \@@_variable_set:NnnN #1 {#2} {#3} #4
    \@@_value_requirement:nn { required } { true }
  }
\cs_generate_variant:Nn \@@_variable_set_required:NnnN { c }
%    \end{macrocode}
% \end{macro}
% \end{macro}
%
% \subsection{Creating key properties}
%
% The key property functions are all wrappers for internal functions,
% meaning that things stay readable and can also be altered later on.
%
% Importantly, while key properties have \enquote{normal} argument specs, the
% underlying code always supplies one braced argument to these. As such, argument
% expansion is handled by hand rather than using the standard tools. This shows
% up particularly for the two-argument properties, where things would otherwise
% go badly wrong.
%
% \begin{macro}{.bool_set:N, .bool_set:c}
% \begin{macro}{.bool_gset:N, .bool_gset:c}
%   One function for this.
%    \begin{macrocode}
\cs_new_protected:cpn { \c_@@_props_root_str .bool_set:N } #1
  { \@@_bool_set:Nn #1 { } }
\cs_new_protected:cpn { \c_@@_props_root_str .bool_set:c } #1
  { \@@_bool_set:cn {#1} { } }
\cs_new_protected:cpn { \c_@@_props_root_str .bool_gset:N } #1
  { \@@_bool_set:Nn #1 { g } }
\cs_new_protected:cpn { \c_@@_props_root_str .bool_gset:c } #1
  { \@@_bool_set:cn {#1} { g } }
%    \end{macrocode}
% \end{macro}
% \end{macro}
%
% \begin{macro}{.bool_set_inverse:N, .bool_set_inverse:c}
% \begin{macro}{.bool_gset_inverse:N, .bool_gset_inverse:c}
%   One function for this.
%    \begin{macrocode}
\cs_new_protected:cpn { \c_@@_props_root_str .bool_set_inverse:N } #1
  { \@@_bool_set_inverse:Nn #1 { } }
\cs_new_protected:cpn { \c_@@_props_root_str .bool_set_inverse:c } #1
  { \@@_bool_set_inverse:cn {#1} { } }
\cs_new_protected:cpn { \c_@@_props_root_str .bool_gset_inverse:N } #1
  { \@@_bool_set_inverse:Nn #1 { g } }
\cs_new_protected:cpn { \c_@@_props_root_str .bool_gset_inverse:c } #1
  { \@@_bool_set_inverse:cn {#1} { g } }
%    \end{macrocode}
% \end{macro}
% \end{macro}
%
% \begin{macro}{.choice:}
%   Making a choice is handled internally, as it is also needed by
%   \texttt{.generate_choices:n}.
%    \begin{macrocode}
\cs_new_protected:cpn { \c_@@_props_root_str .choice: }
  { \@@_choice_make: }
%    \end{macrocode}
% \end{macro}
%
% \begin{macro}
%   {.choices:nn, .choices:Vn, .choices:on, .choices:xn}
%   For auto-generation of a series of mutually-exclusive choices.
%   Here, |#1| consists of two separate
%   arguments, hence the slightly odd-looking implementation.
%    \begin{macrocode}
\cs_new_protected:cpn { \c_@@_props_root_str .choices:nn } #1
  { \@@_choices_make:nn #1 }
\cs_new_protected:cpn { \c_@@_props_root_str .choices:Vn } #1
  { \exp_args:NV \@@_choices_make:nn #1 }
\cs_new_protected:cpn { \c_@@_props_root_str .choices:on } #1
  { \exp_args:No \@@_choices_make:nn #1 }
\cs_new_protected:cpn { \c_@@_props_root_str .choices:xn } #1
  { \exp_args:Nx \@@_choices_make:nn #1 }
%    \end{macrocode}
% \end{macro}
%
% \begin{macro}{.code:n}
%   Creating code is simply a case of passing through to the underlying
%   \texttt{set} function.
%    \begin{macrocode}
\cs_new_protected:cpn { \c_@@_props_root_str .code:n } #1
  { \@@_cmd_set:nn \l_keys_path_str {#1} }
%    \end{macrocode}
% \end{macro}
%
% \begin{macro}{.clist_set:N, .clist_set:c}
% \begin{macro}{.clist_gset:N, .clist_gset:c}
%    \begin{macrocode}
\cs_new_protected:cpn { \c_@@_props_root_str .clist_set:N } #1
  { \@@_variable_set:NnnN #1 { clist } { } n }
\cs_new_protected:cpn { \c_@@_props_root_str .clist_set:c } #1
  { \@@_variable_set:cnnN {#1} { clist } { } n }
\cs_new_protected:cpn { \c_@@_props_root_str .clist_gset:N } #1
  { \@@_variable_set:NnnN #1 { clist } { g } n }
\cs_new_protected:cpn { \c_@@_props_root_str .clist_gset:c } #1
  { \@@_variable_set:cnnN {#1} { clist } { g } n }
%    \end{macrocode}
% \end{macro}
% \end{macro}
%
% \begin{macro}
%   {.cs_set:Np, .cs_set:cp, .cs_set_protected:Np, .cs_set_protected:cp}
% \begin{macro}
%   {.cs_gset:Np, .cs_gset:cp, .cs_gset_protected:Np, .cs_gset_protected:cp}
%    \begin{macrocode}
\cs_new_protected:cpn { \c_@@_props_root_str .cs_set:Np } #1
  { \@@_cs_set:NNpn \cs_set:Npn #1 { } }
\cs_new_protected:cpn { \c_@@_props_root_str .cs_set:cp } #1
  { \@@_cs_set:Ncpn \cs_set:Npn #1 { } }
\cs_new_protected:cpn { \c_@@_props_root_str .cs_set_protected:Np } #1
  { \@@_cs_set:NNpn \cs_set_protected:Npn #1 { } }
\cs_new_protected:cpn { \c_@@_props_root_str .cs_set_protected:cp } #1
  { \@@_cs_set:Ncpn \cs_set_protected:Npn #1 { } }
\cs_new_protected:cpn { \c_@@_props_root_str .cs_gset:Np } #1
  { \@@_cs_set:NNpn \cs_gset:Npn #1 { } }
\cs_new_protected:cpn { \c_@@_props_root_str .cs_gset:cp } #1
  { \@@_cs_set:Ncpn \cs_gset:Npn #1 { } }
\cs_new_protected:cpn { \c_@@_props_root_str .cs_gset_protected:Np } #1
  { \@@_cs_set:NNpn \cs_gset_protected:Npn #1 { } }
\cs_new_protected:cpn { \c_@@_props_root_str .cs_gset_protected:cp } #1
  { \@@_cs_set:Ncpn \cs_gset_protected:Npn #1 { } }
%    \end{macrocode}
% \end{macro}
% \end{macro}
%
% \begin{macro}{.default:n, .default:V, .default:o, .default:x}
%   Expansion is left to the internal functions.
%    \begin{macrocode}
\cs_new_protected:cpn { \c_@@_props_root_str .default:n } #1
  { \@@_default_set:n {#1} }
\cs_new_protected:cpn { \c_@@_props_root_str .default:V } #1
  { \exp_args:NV \@@_default_set:n #1 }
\cs_new_protected:cpn { \c_@@_props_root_str .default:o } #1
  { \exp_args:No \@@_default_set:n {#1} }
\cs_new_protected:cpn { \c_@@_props_root_str .default:x } #1
  { \exp_args:Nx \@@_default_set:n {#1} }
%    \end{macrocode}
% \end{macro}
%
% \begin{macro}{.dim_set:N, .dim_set:c}
% \begin{macro}{.dim_gset:N, .dim_gset:c}
% Setting a variable is very easy: just pass the data along.
%    \begin{macrocode}
\cs_new_protected:cpn { \c_@@_props_root_str .dim_set:N } #1
  { \@@_variable_set_required:NnnN #1 { dim } { } n }
\cs_new_protected:cpn { \c_@@_props_root_str .dim_set:c } #1
  { \@@_variable_set_required:cnnN {#1} { dim } { } n }
\cs_new_protected:cpn { \c_@@_props_root_str .dim_gset:N } #1
  { \@@_variable_set_required:NnnN #1 { dim } { g } n }
\cs_new_protected:cpn { \c_@@_props_root_str .dim_gset:c } #1
  { \@@_variable_set_required:cnnN {#1} { dim } { g } n }
%    \end{macrocode}
% \end{macro}
% \end{macro}
%
% \begin{macro}{.fp_set:N, .fp_set:c}
% \begin{macro}{.fp_gset:N, .fp_gset:c}
%   Setting a variable is very easy: just pass the data along.
%    \begin{macrocode}
\cs_new_protected:cpn { \c_@@_props_root_str .fp_set:N } #1
  { \@@_variable_set_required:NnnN #1 { fp } { } n }
\cs_new_protected:cpn { \c_@@_props_root_str .fp_set:c } #1
  { \@@_variable_set_required:cnnN {#1} { fp } { } n }
\cs_new_protected:cpn { \c_@@_props_root_str .fp_gset:N } #1
  { \@@_variable_set_required:NnnN #1 { fp } { g } n }
\cs_new_protected:cpn { \c_@@_props_root_str .fp_gset:c } #1
  { \@@_variable_set_required:cnnN {#1} { fp } { g } n }
%    \end{macrocode}
% \end{macro}
% \end{macro}
%
% \begin{macro}{.groups:n}
%   A single property to create groups of keys.
%    \begin{macrocode}
\cs_new_protected:cpn { \c_@@_props_root_str .groups:n } #1
  { \@@_groups_set:n {#1} }
%    \end{macrocode}
% \end{macro}
%
% \begin{macro}{.inherit:n}
%   Nothing complex: only one variant at the moment!
%    \begin{macrocode}
\cs_new_protected:cpn { \c_@@_props_root_str .inherit:n } #1
  { \@@_inherit:n {#1} }
%    \end{macrocode}
% \end{macro}
%
% \begin{macro}{.initial:n, .initial:V, .initial:o, .initial:x}
%   The standard hand-off approach.
%    \begin{macrocode}
\cs_new_protected:cpn { \c_@@_props_root_str .initial:n } #1
  { \@@_initialise:n {#1} }
\cs_new_protected:cpn { \c_@@_props_root_str .initial:V } #1
  { \exp_args:NV \@@_initialise:n #1 }
\cs_new_protected:cpn { \c_@@_props_root_str .initial:o } #1
  { \exp_args:No \@@_initialise:n {#1} }
\cs_new_protected:cpn { \c_@@_props_root_str .initial:x } #1
  { \exp_args:Nx \@@_initialise:n {#1} }
%    \end{macrocode}
% \end{macro}
%
% \begin{macro}{.int_set:N, .int_set:c}
% \begin{macro}{.int_gset:N, .int_gset:c}
%   Setting a variable is very easy: just pass the data along.
%    \begin{macrocode}
\cs_new_protected:cpn { \c_@@_props_root_str .int_set:N } #1
  { \@@_variable_set_required:NnnN #1 { int } { } n }
\cs_new_protected:cpn { \c_@@_props_root_str .int_set:c } #1
  { \@@_variable_set_required:cnnN {#1} { int } { } n }
\cs_new_protected:cpn { \c_@@_props_root_str .int_gset:N } #1
  { \@@_variable_set_required:NnnN #1 { int } { g } n }
\cs_new_protected:cpn { \c_@@_props_root_str .int_gset:c } #1
  { \@@_variable_set_required:cnnN {#1} { int } { g } n }
%    \end{macrocode}
% \end{macro}
% \end{macro}
%
% \begin{macro}
%   {
%     .legacy_if_set:n, .legacy_if_gset:n,
%     .legacy_if_set_inverse:n, .legacy_if_gset_inverse:n
%   }
%    \begin{macrocode}
\cs_new_protected:cpn { \c_@@_props_root_str .legacy_if_set:n } #1
  { \@@_legacy_if_set:nn {#1} { } }
\cs_new_protected:cpn { \c_@@_props_root_str .legacy_if_gset:n } #1
  { \@@_legacy_if_set:nn {#1} { g } }
\cs_new_protected:cpn { \c_@@_props_root_str .legacy_if_set_inverse:n } #1
  { \@@_legacy_if_set_inverse:nn {#1} { } }
\cs_new_protected:cpn { \c_@@_props_root_str .legacy_if_gset_inverse:n } #1
  { \@@_legacy_if_set_inverse:nn {#1} { g } }
%    \end{macrocode}
% \end{macro}
%
% \begin{macro}{.meta:n}
%   Making a meta is handled internally.
%    \begin{macrocode}
\cs_new_protected:cpn { \c_@@_props_root_str .meta:n } #1
  { \@@_meta_make:n {#1} }
%    \end{macrocode}
% \end{macro}
%
% \begin{macro}{.meta:nn}
%   Meta with path: potentially lots of variants, but for the moment
%   no so many defined.
%    \begin{macrocode}
\cs_new_protected:cpn { \c_@@_props_root_str .meta:nn } #1
  { \@@_meta_make:nn #1 }
%    \end{macrocode}
% \end{macro}
%
% \begin{macro}{.multichoice:}
% \begin{macro}
%   {
%     .multichoices:nn, .multichoices:Vn, .multichoices:on,
%       .multichoices:xn,
%   }
%   The same idea as \texttt{.choice:} and \texttt{.choices:nn}, but
%   where more than one choice is allowed.
%    \begin{macrocode}
\cs_new_protected:cpn { \c_@@_props_root_str .multichoice: }
  { \@@_multichoice_make: }
\cs_new_protected:cpn { \c_@@_props_root_str .multichoices:nn } #1
  { \@@_multichoices_make:nn #1 }
\cs_new_protected:cpn { \c_@@_props_root_str .multichoices:Vn } #1
  { \exp_args:NV \@@_multichoices_make:nn #1 }
\cs_new_protected:cpn { \c_@@_props_root_str .multichoices:on } #1
  { \exp_args:No \@@_multichoices_make:nn #1 }
\cs_new_protected:cpn { \c_@@_props_root_str .multichoices:xn } #1
  { \exp_args:Nx \@@_multichoices_make:nn #1 }
%    \end{macrocode}
% \end{macro}
% \end{macro}
%
% \begin{macro}{.muskip_set:N, .muskip_set:c, .muskip_gset:N, .muskip_gset:c}
%   Setting a variable is very easy: just pass the data along.
%    \begin{macrocode}
\cs_new_protected:cpn { \c_@@_props_root_str .muskip_set:N } #1
  { \@@_variable_set_required:NnnN #1 { muskip } { } n }
\cs_new_protected:cpn { \c_@@_props_root_str .muskip_set:c } #1
  { \@@_variable_set_required:cnnN {#1} { muskip } { } n }
\cs_new_protected:cpn { \c_@@_props_root_str .muskip_gset:N } #1
  { \@@_variable_set_required:NnnN #1 { muskip } { g } n }
\cs_new_protected:cpn { \c_@@_props_root_str .muskip_gset:c } #1
  { \@@_variable_set_required:cnnN {#1} { muskip } { g } n }
%    \end{macrocode}
% \end{macro}
%
% \begin{macro}{.prop_put:N, .prop_put:c, .prop_gput:N, .prop_gput:c}
%   Setting a variable is very easy: just pass the data along.
%    \begin{macrocode}
\cs_new_protected:cpn { \c_@@_props_root_str .prop_put:N } #1
  { \@@_prop_put:Nn #1 { } }
\cs_new_protected:cpn { \c_@@_props_root_str .prop_put:c } #1
  { \@@_prop_put:cn {#1} { } }
\cs_new_protected:cpn { \c_@@_props_root_str .prop_gput:N } #1
  { \@@_prop_put:Nn #1 { g } }
\cs_new_protected:cpn { \c_@@_props_root_str .prop_gput:c } #1
  { \@@_prop_put:cn {#1} { g } }
%    \end{macrocode}
% \end{macro}
%
% \begin{macro}{.skip_set:N, .skip_set:c}
% \begin{macro}{.skip_gset:N, .skip_gset:c}
%   Setting a variable is very easy: just pass the data along.
%    \begin{macrocode}
\cs_new_protected:cpn { \c_@@_props_root_str .skip_set:N } #1
  { \@@_variable_set_required:NnnN #1 { skip } { } n }
\cs_new_protected:cpn { \c_@@_props_root_str .skip_set:c } #1
  { \@@_variable_set_required:cnnN {#1} { skip } { } n }
\cs_new_protected:cpn { \c_@@_props_root_str .skip_gset:N } #1
  { \@@_variable_set_required:NnnN #1 { skip } { g } n }
\cs_new_protected:cpn { \c_@@_props_root_str .skip_gset:c } #1
  { \@@_variable_set_required:cnnN {#1} { skip } { g } n }
%    \end{macrocode}
% \end{macro}
% \end{macro}
%
% \begin{macro}{.str_set:N, .str_set:c}
% \begin{macro}{.str_gset:N, .str_gset:c}
% \begin{macro}{.str_set_x:N, .str_set_x:c}
% \begin{macro}{.str_gset_x:N, .str_gset_x:c}
%   Setting a variable is very easy: just pass the data along.
%    \begin{macrocode}
\cs_new_protected:cpn { \c_@@_props_root_str .str_set:N } #1
  { \@@_variable_set:NnnN #1 { str } { } n }
\cs_new_protected:cpn { \c_@@_props_root_str .str_set:c } #1
  { \@@_variable_set:cnnN {#1} { str } { } n }
\cs_new_protected:cpn { \c_@@_props_root_str .str_set_x:N } #1
  { \@@_variable_set:NnnN #1 { str } { } x }
\cs_new_protected:cpn { \c_@@_props_root_str .str_set_x:c } #1
  { \@@_variable_set:cnnN {#1} { str } { } x }
\cs_new_protected:cpn { \c_@@_props_root_str .str_gset:N } #1
  { \@@_variable_set:NnnN #1 { str } { g } n }
\cs_new_protected:cpn { \c_@@_props_root_str .str_gset:c } #1
  { \@@_variable_set:cnnN {#1} { str } { g } n }
\cs_new_protected:cpn { \c_@@_props_root_str .str_gset_x:N } #1
  { \@@_variable_set:NnnN #1 { str } { g } x }
\cs_new_protected:cpn { \c_@@_props_root_str .str_gset_x:c } #1
  { \@@_variable_set:cnnN {#1} { str } { g } x }
%    \end{macrocode}
% \end{macro}
% \end{macro}
% \end{macro}
% \end{macro}
%
% \begin{macro}{.tl_set:N, .tl_set:c}
% \begin{macro}{.tl_gset:N, .tl_gset:c}
% \begin{macro}{.tl_set_x:N, .tl_set_x:c}
% \begin{macro}{.tl_gset_x:N, .tl_gset_x:c}
%   Setting a variable is very easy: just pass the data along.
%    \begin{macrocode}
\cs_new_protected:cpn { \c_@@_props_root_str .tl_set:N } #1
  { \@@_variable_set:NnnN #1 { tl } { } n }
\cs_new_protected:cpn { \c_@@_props_root_str .tl_set:c } #1
  { \@@_variable_set:cnnN {#1} { tl } { } n }
\cs_new_protected:cpn { \c_@@_props_root_str .tl_set_x:N } #1
  { \@@_variable_set:NnnN #1 { tl } { } x }
\cs_new_protected:cpn { \c_@@_props_root_str .tl_set_x:c } #1
  { \@@_variable_set:cnnN {#1} { tl } { } x }
\cs_new_protected:cpn { \c_@@_props_root_str .tl_gset:N } #1
  { \@@_variable_set:NnnN #1 { tl } { g } n }
\cs_new_protected:cpn { \c_@@_props_root_str .tl_gset:c } #1
  { \@@_variable_set:cnnN {#1} { tl } { g } n }
\cs_new_protected:cpn { \c_@@_props_root_str .tl_gset_x:N } #1
  { \@@_variable_set:NnnN #1 { tl } { g } x }
\cs_new_protected:cpn { \c_@@_props_root_str .tl_gset_x:c } #1
  { \@@_variable_set:cnnN {#1} { tl } { g } x }
%    \end{macrocode}
% \end{macro}
% \end{macro}
% \end{macro}
% \end{macro}
%
% \begin{macro}{.undefine:}
%   Another simple wrapper.
%    \begin{macrocode}
\cs_new_protected:cpn { \c_@@_props_root_str .undefine: }
  { \@@_undefine: }
%    \end{macrocode}
% \end{macro}
%
% \begin{macro}{.usage:n}
%    \begin{macrocode}
\cs_new_protected:cpn { \c_@@_props_root_str .usage:n } #1
  { \@@_usage:n {#1} }
%    \end{macrocode}
% \end{macro}
%
% \begin{macro}{.value_forbidden:n}
% \begin{macro}{.value_required:n}
%   These are very similar, so both call the same function.
%    \begin{macrocode}
\cs_new_protected:cpn { \c_@@_props_root_str .value_forbidden:n } #1
  { \@@_value_requirement:nn { forbidden } {#1} }
\cs_new_protected:cpn { \c_@@_props_root_str .value_required:n } #1
  { \@@_value_requirement:nn { required } {#1} }
%    \end{macrocode}
% \end{macro}
% \end{macro}
%
% \subsection{Setting keys}
%
% \begin{macro}
%   {
%     \keys_set:nn, \keys_set:nV, \keys_set:nv, \keys_set:no, \keys_set:nx,
%     \@@_set:nn
%   }
% \begin{macro}{\@@_set:nnn}
%   A simple wrapper allowing for nesting.
%    \begin{macrocode}
\cs_new_protected:Npn \keys_set:nn #1#2
  {
    \use:x
      {
        \bool_set_false:N \exp_not:N \l_@@_only_known_bool
        \bool_set_false:N \exp_not:N \l_@@_filtered_bool
        \bool_set_false:N \exp_not:N \l_@@_selective_bool
        \tl_set:Nn \exp_not:N \l_@@_relative_tl
          { \exp_not:N \q_@@_no_value }
        \@@_set:nn \exp_not:n { {#1} {#2} }
        \bool_if:NT \l_@@_only_known_bool
          { \bool_set_true:N \exp_not:N \l_@@_only_known_bool }
        \bool_if:NT \l_@@_filtered_bool
          { \bool_set_true:N \exp_not:N \l_@@_filtered_bool }
        \bool_if:NT \l_@@_selective_bool
          { \bool_set_true:N \exp_not:N \l_@@_selective_bool }
        \tl_set:Nn \exp_not:N \l_@@_relative_tl
          { \exp_not:o \l_@@_relative_tl }
      }
  }
\cs_generate_variant:Nn \keys_set:nn { nV , nv , no , nx }
\cs_new_protected:Npn \@@_set:nn #1#2
  { \exp_args:No \@@_set:nnn \l_@@_module_str {#1} {#2} }
\cs_new_protected:Npn \@@_set:nnn #1#2#3
  {
    \str_set:Nx \l_@@_module_str { \@@_trim_spaces:n {#2} }
    \keyval_parse:NNn \@@_set_keyval:n \@@_set_keyval:nn {#3}
    \str_set:Nn \l_@@_module_str {#1}
  }
%    \end{macrocode}
% \end{macro}
% \end{macro}
%
% \begin{macro}
%   {
%     \keys_set_known:nnN, \keys_set_known:nVN,
%     \keys_set_known:nvN, \keys_set_known:noN
%   }
% \begin{macro}
%   {
%     \keys_set_known:nnnN, \keys_set_known:nVnN,
%     \keys_set_known:nvnN, \keys_set_known:nonN
%   }
% \begin{macro}{\@@_set_known:nnnnN}
% \begin{macro}
%   {
%     \keys_set_known:nn, \keys_set_known:nV,
%     \keys_set_known:nv, \keys_set_known:no
%   }
%  \begin{macro}{\@@_set_known:nnn}
%   Setting known keys simply means setting the appropriate flag, then
%   running the standard code. To allow for nested setting, any existing
%   value of \cs{l_@@_unused_clist} is saved on the stack and reset
%   afterwards. Note that for speed/simplicity reasons we use a \texttt{tl}
%   operation to set the \texttt{clist} here!
%    \begin{macrocode}
\cs_new_protected:Npn \keys_set_known:nnN #1#2#3
  {
    \exp_args:No \@@_set_known:nnnnN
      \l_@@_unused_clist \q_@@_no_value {#1} {#2} #3
  }
\cs_generate_variant:Nn \keys_set_known:nnN { nV , nv , no }
\cs_new_protected:Npn \keys_set_known:nnnN #1#2#3#4
  {
    \exp_args:No \@@_set_known:nnnnN
      \l_@@_unused_clist {#3} {#1} {#2} #4
  }
\cs_generate_variant:Nn \keys_set_known:nnnN { nV , nv , no }
\cs_new_protected:Npn \@@_set_known:nnnnN #1#2#3#4#5
  {
    \clist_clear:N \l_@@_unused_clist
    \@@_set_known:nnn {#2} {#3} {#4}
    \__kernel_tl_set:Nx #5 { \exp_not:o \l_@@_unused_clist }
    \tl_set:Nn \l_@@_unused_clist {#1}
  }
\cs_new_protected:Npn \keys_set_known:nn #1#2
  { \@@_set_known:nnn \q_@@_no_value {#1} {#2} }
\cs_generate_variant:Nn \keys_set_known:nn { nV , nv , no }
\cs_new_protected:Npn \@@_set_known:nnn #1#2#3
  {
    \use:x
      {
        \bool_set_true:N \exp_not:N \l_@@_only_known_bool
        \bool_set_false:N \exp_not:N \l_@@_filtered_bool
        \bool_set_false:N \exp_not:N \l_@@_selective_bool
        \tl_set:Nn \exp_not:N \l_@@_relative_tl { \exp_not:n {#1} }
        \@@_set:nn \exp_not:n { {#2} {#3} }
        \bool_if:NF \l_@@_only_known_bool
          { \bool_set_false:N \exp_not:N \l_@@_only_known_bool }
        \bool_if:NT \l_@@_filtered_bool
          { \bool_set_true:N \exp_not:N \l_@@_filtered_bool }
        \bool_if:NT \l_@@_selective_bool
          { \bool_set_true:N \exp_not:N \l_@@_selective_bool }
        \tl_set:Nn \exp_not:N \l_@@_relative_tl
          { \exp_not:o \l_@@_relative_tl }
      }
  }
%    \end{macrocode}
% \end{macro}
% \end{macro}
% \end{macro}
% \end{macro}
% \end{macro}
%
% \begin{macro}
%   {
%     \keys_set_filter:nnnN, \keys_set_filter:nnVN, \keys_set_filter:nnvN,
%       \keys_set_filter:nnoN
%   }
% \begin{macro}
%   {
%     \keys_set_filter:nnnnN, \keys_set_filter:nnVnN,
%     \keys_set_filter:nnvnN, \keys_set_filter:nnonN
%   }
% \begin{macro}{\@@_set_filter:nnnnnN}
% \begin{macro}
%   {
%     \keys_set_filter:nnn, \keys_set_filter:nnV, \keys_set_filter:nnv,
%       \keys_set_filter:nno
%   }
% \begin{macro}{\@@_set_filter:nnnn}
% \begin{macro}
%   {
%     \keys_set_groups:nnn, \keys_set_groups:nnV, \keys_set_groups:nnv,
%       \keys_set_groups:nno
%   }
%  \begin{macro}{\@@_set_selective:nnn}
%  \begin{macro}{\@@_set_selective:nnnn}
%   The idea of setting keys in a selective manner again uses flags
%   wrapped around the basic code. The comments on \cs{keys_set_known:nnN}
%   also apply here. We have a bit more shuffling to do to keep everything
%   nestable.
%    \begin{macrocode}
\cs_new_protected:Npn \keys_set_filter:nnnN #1#2#3#4
  {
    \exp_args:No \@@_set_filter:nnnnnN
      \l_@@_unused_clist
        \q_@@_no_value {#1} {#2} {#3} #4
  }
\cs_generate_variant:Nn \keys_set_filter:nnnN { nnV , nnv , nno }
\cs_new_protected:Npn \keys_set_filter:nnnnN #1#2#3#4#5
  {
    \exp_args:No \@@_set_filter:nnnnnN
      \l_@@_unused_clist {#4} {#1} {#2} {#3} #5
  }
\cs_generate_variant:Nn \keys_set_filter:nnnnN { nnV , nnv , nno }
\cs_new_protected:Npn \@@_set_filter:nnnnnN #1#2#3#4#5#6
  {
    \clist_clear:N \l_@@_unused_clist
    \@@_set_filter:nnnn {#2} {#3} {#4} {#5}
    \__kernel_tl_set:Nx #6 { \exp_not:o \l_@@_unused_clist }
    \tl_set:Nn \l_@@_unused_clist {#1}
  }
\cs_new_protected:Npn \keys_set_filter:nnn #1#2#3
  {\@@_set_filter:nnnn \q_@@_no_value {#1} {#2} {#3} }
\cs_generate_variant:Nn \keys_set_filter:nnn { nnV , nnv , nno }
\cs_new_protected:Npn \@@_set_filter:nnnn #1#2#3#4
  {
    \use:x
      {
        \bool_set_false:N \exp_not:N \l_@@_only_known_bool
        \bool_set_true:N \exp_not:N \l_@@_filtered_bool
        \bool_set_true:N \exp_not:N \l_@@_selective_bool
        \tl_set:Nn \exp_not:N \l_@@_relative_tl { \exp_not:n {#1} }
        \@@_set_selective:nnn \exp_not:n { {#2} {#3} {#4} }
        \bool_if:NT \l_@@_only_known_bool
          { \bool_set_true:N \exp_not:N \l_@@_only_known_bool }
        \bool_if:NF \l_@@_filtered_bool
          { \bool_set_false:N \exp_not:N \l_@@_filtered_bool }
        \bool_if:NF \l_@@_selective_bool
          { \bool_set_false:N \exp_not:N \l_@@_selective_bool }
        \tl_set:Nn \exp_not:N \l_@@_relative_tl
          { \exp_not:o \l_@@_relative_tl }
      }
  }
\cs_new_protected:Npn \keys_set_groups:nnn #1#2#3
  {
    \use:x
      {
        \bool_set_false:N \exp_not:N \l_@@_only_known_bool
        \bool_set_false:N \exp_not:N \l_@@_filtered_bool
        \bool_set_true:N \exp_not:N \l_@@_selective_bool
        \tl_set:Nn \exp_not:N \l_@@_relative_tl
          { \exp_not:N \q_@@_no_value }
        \@@_set_selective:nnn \exp_not:n { {#1} {#2} {#3} }
        \bool_if:NT \l_@@_only_known_bool
          { \bool_set_true:N \exp_not:N \l_@@_only_known_bool }
        \bool_if:NF \l_@@_filtered_bool
          { \bool_set_true:N \exp_not:N \l_@@_filtered_bool }
        \bool_if:NF \l_@@_selective_bool
          { \bool_set_false:N \exp_not:N \l_@@_selective_bool }
        \tl_set:Nn \exp_not:N \l_@@_relative_tl
          { \exp_not:o \l_@@_relative_tl }
      }
  }
\cs_generate_variant:Nn \keys_set_groups:nnn { nnV , nnv , nno }
\cs_new_protected:Npn \@@_set_selective:nnn
  { \exp_args:No \@@_set_selective:nnnn \l_@@_selective_seq }
\cs_new_protected:Npn \@@_set_selective:nnnn #1#2#3#4
  {
    \seq_set_from_clist:Nn \l_@@_selective_seq {#3}
    \@@_set:nn {#2} {#4}
    \tl_set:Nn \l_@@_selective_seq {#1}
  }
%    \end{macrocode}
% \end{macro}
% \end{macro}
% \end{macro}
% \end{macro}
% \end{macro}
% \end{macro}
% \end{macro}
% \end{macro}
%
% \begin{macro}{\keys_precompile:nnN}
%   A simple wrapper.
%    \begin{macrocode}
\cs_new_protected:Npn \keys_precompile:nnN #1#2#3
  {
    \bool_set_true:N \l_@@_precompile_bool
    \tl_clear:N \l_@@_precompile_tl
    \keys_set:nn {#1} {#2}
    \bool_set_false:N \l_@@_precompile_bool
    \tl_set_eq:NN #3 \l_@@_precompile_tl
  }
%    \end{macrocode}
% \end{macro}
%
% \begin{macro}{\@@_set_keyval:n, \@@_set_keyval:nn}
% \begin{macro}{\@@_set_keyval:nnn, \@@_set_keyval:onn}
% \begin{macro}{\@@_find_key_module:wNN}
% \begin{macro}
%   {
%     \@@_find_key_module_auxi:Nw   ,
%     \@@_find_key_module_auxii:Nw  ,
%     \@@_find_key_module_auxiii:Nn ,
%     \@@_find_key_module_auxiv:Nw
%   }
% \begin{macro}{\@@_set_selective:}
%   A shared system once again. First, set the current path and add a
%   default if needed. There are then checks to see if a value is
%   required or forbidden. If everything passes, move on to execute the
%   code.
%    \begin{macrocode}
\cs_new_protected:Npn \@@_set_keyval:n #1
  {
    \bool_set_true:N \l_@@_no_value_bool
    \@@_set_keyval:onn \l_@@_module_str {#1} { }
  }
\cs_new_protected:Npn \@@_set_keyval:nn #1#2
  {
    \bool_set_false:N \l_@@_no_value_bool
    \@@_set_keyval:onn \l_@@_module_str {#1} {#2}
  }
%    \end{macrocode}
%   The key path here can be fully defined, after which there is a search
%   for the key and module names: the user may have passed them with part
%   of what is actually the module (for our purposes) in the key name. As
%   that happens on a per-key basis, we use the stack approach to restore
%   the module name without a group.
%    \begin{macrocode}
\cs_new_protected:Npn \@@_set_keyval:nnn #1#2#3
  {
    \__kernel_tl_set:Nx \l_keys_path_str
      {
        \tl_if_blank:nF {#1}
          { #1 / }
        \@@_trim_spaces:n {#2}
      }
    \str_clear:N \l_@@_module_str
    \str_clear:N \l_@@_inherit_str
    \exp_after:wN \@@_find_key_module:wNN \l_keys_path_str \s_@@_stop
      \l_@@_module_str \l_keys_key_str
    \tl_set_eq:NN \l_keys_key_tl \l_keys_key_str
    \@@_value_or_default:n {#3}
    \bool_if:NTF \l_@@_selective_bool
      \@@_set_selective:
      \@@_execute:
    \str_set:Nn \l_@@_module_str {#1}
  }
\cs_generate_variant:Nn \@@_set_keyval:nnn { o }
%    \end{macrocode}
%   This function uses \cs{cs_set_nopar:Npx} internally for performance reasons,
%   the argument |#1| is already a string in every usage, so turning it into a
%   string again seems unnecessary.
%    \begin{macrocode}
\cs_new_protected:Npn \@@_find_key_module:wNN #1 \s_@@_stop #2 #3
  {
    \@@_find_key_module_auxi:Nw #2 #1 \s_@@_nil \@@_find_key_module_auxii:Nw
      / \s_@@_nil \@@_find_key_module_auxiv:Nw #3
  }
\cs_new_protected:Npn \@@_find_key_module_auxi:Nw #1 #2 / #3 \s_@@_nil #4
  {
    #4 #1 #2 \s_@@_mark #3 \s_@@_nil #4
  }
\cs_new_protected:Npn \@@_find_key_module_auxii:Nw
    #1 #2 \s_@@_mark #3 \s_@@_nil \@@_find_key_module_auxii:Nw
  {
    \cs_set_nopar:Npx #1 { \tl_if_empty:NF #1 { #1 / } #2 }
    \@@_find_key_module_auxi:Nw #1 #3 \s_@@_nil \@@_find_key_module_auxiii:Nw
  }
\cs_new_protected:Npn \@@_find_key_module_auxiii:Nw #1 #2 \s_@@_mark
  {
    \cs_set_nopar:Npx #1 { \tl_if_empty:NF #1 { #1 / } #2 }
    \@@_find_key_module_auxi:Nw #1
  }
\cs_new_protected:Npn \@@_find_key_module_auxiv:Nw
    #1 #2 \s_@@_nil #3 \s_@@_mark
    \s_@@_nil \@@_find_key_module_auxiv:Nw #4
  {
    \cs_set_nopar:Npn #4 { #2 }
  }
%    \end{macrocode}
%  If selective setting is active, there are a number of possible sub-cases
%  to consider. The key name may not be known at all or if it is, it may not
%  have any groups assigned. There is then the question of whether the
%  selection is opt-in or opt-out.
%    \begin{macrocode}
\cs_new_protected:Npn \@@_set_selective:
  {
    \cs_if_exist:cTF { \c_@@_groups_root_str \l_keys_path_str }
      {
        \clist_set_eq:Nc \l_@@_groups_clist
          { \c_@@_groups_root_str \l_keys_path_str }
        \@@_check_groups:
      }
      {
        \bool_if:NTF \l_@@_filtered_bool
          \@@_execute:
          \@@_store_unused:
      }
  }
%    \end{macrocode}
%    In the case where selective setting requires a comparison of the list
%    of groups which apply to a key with the list of those which have been
%    set active. That requires two mappings, and again a different outcome
%    depending on whether opt-in or opt-out is set.
%    We cannot replace the clist mapping by \cs{clist_if_in:NnTF} because
%    catcodes may not be the same; they cannot be normalized easily in the
%    clist because of the remote possibility that some items need braces
%    if they involve commas or leading/trailing spaces.
%    \begin{macrocode}
\cs_new_protected:Npn \@@_check_groups:
  {
    \bool_set_false:N \l_@@_tmp_bool
    \seq_map_inline:Nn \l_@@_selective_seq
      {
        \clist_map_inline:Nn \l_@@_groups_clist
          {
            \str_if_eq:nnT {##1} {####1}
              {
                \bool_set_true:N \l_@@_tmp_bool
                \clist_map_break:n \seq_map_break:
              }
          }
      }
    \bool_if:NTF \l_@@_tmp_bool
      {
        \bool_if:NTF \l_@@_filtered_bool
          \@@_store_unused:
          \@@_execute:
      }
      {
        \bool_if:NTF \l_@@_filtered_bool
          \@@_execute:
          \@@_store_unused:
      }
  }
%    \end{macrocode}
% \end{macro}
% \end{macro}
% \end{macro}
% \end{macro}
% \end{macro}
%
% \begin{macro}{\@@_value_or_default:n}
% \begin{macro}{\@@_default_inherit:}
%   If a value is given, return it as |#1|, otherwise send a default if
%   available.
%    \begin{macrocode}
\cs_new_protected:Npn \@@_value_or_default:n #1
  {
    \bool_if:NTF \l_@@_no_value_bool
      {
        \cs_if_exist:cTF { \c_@@_default_root_str \l_keys_path_str }
          {
            \tl_set_eq:Nc
              \l_keys_value_tl
              { \c_@@_default_root_str \l_keys_path_str }
          }
          {
            \tl_clear:N \l_keys_value_tl
            \cs_if_exist:cT
              { \c_@@_inherit_root_str \@@_parent:o \l_keys_path_str }
              { \@@_default_inherit: }
          }
      }
      { \tl_set:Nn \l_keys_value_tl {#1} }
  }
\cs_new_protected:Npn \@@_default_inherit:
  {
    \clist_map_inline:cn
      { \c_@@_inherit_root_str \@@_parent:o \l_keys_path_str }
      {
        \cs_if_exist:cT
          { \c_@@_default_root_str ##1 / \l_keys_key_str }
          {
            \tl_set_eq:Nc
              \l_keys_value_tl
              { \c_@@_default_root_str ##1 / \l_keys_key_str }
            \clist_map_break:
          }
      }
  }
%    \end{macrocode}
% \end{macro}
% \end{macro}
%
% \begin{macro}{\@@_execute:, \@@_execute_inherit:, \@@_execute_unknown:}
% \begin{macro}[EXP]{\@@_execute:nn, \@@_execute:no}
% \begin{macro}{\@@_store_unused:,\@@_store_unused_aux:}
%   Actually executing a key is done in two parts. First, look for the
%   key itself, then look for the \texttt{unknown} key with the same
%   path. If both of these fail, complain. What exactly happens if a key
%   is unknown depends on whether unknown keys are being skipped or if
%   an error should be raised.
%    \begin{macrocode}
\cs_new_protected:Npn \@@_execute:
  {
    \cs_if_exist:cTF { \c_@@_code_root_str \l_keys_path_str }
      {
        \cs_if_exist_use:c { \c_@@_check_root_str \l_keys_path_str }
        \@@_execute:no \l_keys_path_str \l_keys_value_tl
      }
      {
        \cs_if_exist:cTF
          { \c_@@_inherit_root_str \@@_parent:o \l_keys_path_str }
          { \@@_execute_inherit: }
          { \@@_execute_unknown: }
      }
  }
%    \end{macrocode}
%   To deal with the case where there is no hit, we leave
%   \cs{@@_execute_unknown:} in the input stream and clean it up using the
%    break function: that avoids needing a boolean.
%    \begin{macrocode}
\cs_new_protected:Npn \@@_execute_inherit:
  {
    \clist_map_inline:cn
      { \c_@@_inherit_root_str \@@_parent:o \l_keys_path_str }
      {
        \cs_if_exist:cT
          { \c_@@_code_root_str ##1 / \l_keys_key_str }
          {
            \str_set:Nn \l_@@_inherit_str {##1}
            \cs_if_exist_use:c { \c_@@_check_root_str ##1 / \l_keys_key_str }
            \@@_execute:no { ##1 / \l_keys_key_str } \l_keys_value_tl
            \clist_map_break:n \use_none:n
          }
      }
    \@@_execute_unknown:
  }
\cs_new_protected:Npn \@@_execute_unknown:
  {
    \bool_if:NTF \l_@@_only_known_bool
      { \@@_store_unused: }
      {
        \cs_if_exist:cTF
          { \c_@@_code_root_str \l_@@_module_str / unknown }
          { \@@_execute:no { \l_@@_module_str / unknown } \l_keys_value_tl }
          {
            \msg_error:nnxx { keys } { unknown }
              \l_keys_path_str \l_@@_module_str
          }
      }
  }
%    \end{macrocode}
%   A key's code is in the control sequence with csname
%   \cs{c_@@_code_root_str} |#1|.  We expand it once to get the
%   replacement text (with argument |#2|) and call \cs{use:n}
%   with this replacement as its argument.  This ensures that any
%   undefined control sequence error in the key's code will lead to an
%   error message of the form |<argument>|\ldots{}\meta{control
%   sequence} in which one can read the (undefined) \meta{control
%   sequence} in full, rather than an error message that starts with the
%   potentially very long key name, which would make the (undefined)
%   \meta{control sequence} be truncated or sometimes completely hidden.
%   See \url{https://github.com/latex3/latex2e/issues/351}.
%    \begin{macrocode}
\cs_new:Npn \@@_execute:nn #1#2
  { \@@_execute:no {#1} { \prg_do_nothing: #2 } }
\cs_new:Npn \@@_execute:no #1#2
  {
    \exp_args:NNo \exp_args:No \use:n
      {
        \cs:w \c_@@_code_root_str #1 \exp_after:wN \cs_end:
        \exp_after:wN {#2}
      }
  }
%    \end{macrocode}
%   When there is no relative path, things here are easy: just save the key
%   name and value. When we are working with a relative path, first we
%   need to turn it into a string: that can't happen earlier as we need
%   to store \cs{q_@@_no_value}. Then, use a standard delimited approach to fish
%   out the partial path.
%    \begin{macrocode}
\cs_new_protected:Npn \@@_store_unused:
  {
    \@@_quark_if_no_value:NTF \l_@@_relative_tl
      {
        \clist_put_right:Nx \l_@@_unused_clist
          {
            \l_keys_key_str
            \bool_if:NF \l_@@_no_value_bool
              { = { \exp_not:o \l_keys_value_tl } }
          }
      }
      {
        \tl_if_empty:NTF \l_@@_relative_tl
          {
            \clist_put_right:Nx \l_@@_unused_clist
              {
                \l_keys_path_str
                \bool_if:NF \l_@@_no_value_bool
                  { = { \exp_not:o \l_keys_value_tl } }
              }
          }
          { \@@_store_unused_aux: }
      }
  }
\cs_new_protected:Npn \@@_store_unused_aux:
  {
    \__kernel_tl_set:Nx \l_@@_relative_tl
      { \exp_args:No \@@_trim_spaces:n \l_@@_relative_tl }
    \use:x
      {
        \cs_set_protected:Npn \@@_store_unused:w
          ####1 \l_@@_relative_tl /
          ####2 \l_@@_relative_tl /
          ####3 \s_@@_stop
      }
        {
          \tl_if_blank:nF {##1}
            {
              \msg_error:nnxx { keys } { bad-relative-key-path }
                \l_keys_path_str
                \l_@@_relative_tl
            }
          \clist_put_right:Nx \l_@@_unused_clist
            {
              \exp_not:n {##2}
              \bool_if:NF \l_@@_no_value_bool
                { = { \exp_not:o \l_keys_value_tl } }
            }
        }
    \use:x
      {
        \@@_store_unused:w \l_keys_path_str
          \l_@@_relative_tl / \l_@@_relative_tl /
          \s_@@_stop
      }
  }
\cs_new_protected:Npn \@@_store_unused:w { }
%    \end{macrocode}
% \end{macro}
% \end{macro}
% \end{macro}
%
% \begin{macro}[EXP]{\@@_choice_find:n}
% \begin{macro}[EXP]{\@@_choice_find:nn}
% \begin{macro}[EXP]{\@@_multichoice_find:n}
%   Executing a choice has two parts. First, try the choice given, then
%   if that fails call the unknown key. That always exists, as it is created
%   when a choice is first made. So there is no need for any escape code.
%   For multiple choices, the same code ends up used in a mapping.
%    \begin{macrocode}
\cs_new:Npn \@@_choice_find:n #1
  {
    \str_if_empty:NTF \l_@@_inherit_str
      { \@@_choice_find:nn \l_keys_path_str {#1} }
      {
        \@@_choice_find:nn
          { \l_@@_inherit_str / \l_keys_key_str } {#1}
      }
  }
\cs_new:Npn \@@_choice_find:nn #1#2
  {
    \cs_if_exist:cTF { \c_@@_code_root_str #1 / \@@_trim_spaces:n {#2} }
      { \@@_execute:nn { #1 / \@@_trim_spaces:n {#2} } {#2} }
      { \@@_execute:nn { #1 / unknown } {#2} }
  }
\cs_new:Npn \@@_multichoice_find:n #1
  { \clist_map_function:nN {#1} \@@_choice_find:n }
%    \end{macrocode}
% \end{macro}
% \end{macro}
% \end{macro}
%
% \subsection{Utilities}
%
% \begin{macro}[EXP]{\@@_parent:o}
% \begin{macro}[EXP]
%   {
%     \@@_parent_auxi:w   ,
%     \@@_parent_auxii:w  ,
%     \@@_parent_auxiii:n ,
%     \@@_parent_auxiv:w
%   }
%   Used to strip off the ending part of the key path after the last~|/|.
%    \begin{macrocode}
\cs_new:Npn \@@_parent:o #1
  {
    \exp_after:wN \@@_parent_auxi:w #1 \q_nil \@@_parent_auxii:w
      / \q_nil \@@_parent_auxiv:w
  }
\cs_new:Npn \@@_parent_auxi:w #1 / #2 \q_nil #3
  {
    #3 { #1 } #2 \q_nil #3
  }
\cs_new:Npn \@@_parent_auxii:w #1 #2 \q_nil \@@_parent_auxii:w
  {
    #1 \@@_parent_auxi:w #2 \q_nil \@@_parent_auxiii:n
  }
\cs_new:Npn \@@_parent_auxiii:n #1
  {
    / #1 \@@_parent_auxi:w
  }
\cs_new:Npn \@@_parent_auxiv:w #1 \q_nil \@@_parent_auxiv:w
  {
  }
%    \end{macrocode}
% \end{macro}
% \end{macro}
%
% \begin{macro}[EXP]{\@@_trim_spaces:n}
% \begin{macro}[EXP]
%   {
%     \@@_trim_spaces_auxi:w   ,
%     \@@_trim_spaces_auxii:w  ,
%     \@@_trim_spaces_auxiii:w
%   }
%   Space stripping has to allow for the fact that the key here might have
%   several parts, and spaces need to be stripped from each part. Since the key
%   name is turned into a string groups can't be stripped accidentally and the
%   precautions of \cs{tl_trim_spaces:n} aren't necessary, in this case it is
%   much faster to just directly strip spaces around |/|.
%    \begin{macrocode}
\group_begin:
  \cs_set:Npn \@@_tmp:w #1
    {
      \cs_new:Npn \@@_trim_spaces:n ##1
        {
          \exp_after:wN \@@_trim_spaces_auxi:w \tl_to_str:n { / ##1 } /
            \s_@@_nil  \@@_trim_spaces_auxi:w
            \s_@@_mark \@@_trim_spaces_auxii:w
            #1 / #1
            \s_@@_nil  \@@_trim_spaces_auxii:w
            \s_@@_mark \@@_trim_spaces_auxiii:w
        }
    }
  \@@_tmp:w { ~ }
\group_end:
\cs_new:Npn \@@_trim_spaces_auxi:w #1 ~ / #2 \s_@@_nil #3
  {
    #3 #1 / #2 \s_@@_nil #3
  }
\cs_new:Npn \@@_trim_spaces_auxii:w #1 / ~ #2 \s_@@_mark #3
  {
    #3 #1 / #2 \s_@@_mark #3
  }
\cs_new:Npn \@@_trim_spaces_auxiii:w
    / #1 /
    \s_@@_nil  \@@_trim_spaces_auxi:w
    \s_@@_mark \@@_trim_spaces_auxii:w
    /
    \s_@@_nil  \@@_trim_spaces_auxii:w
    \s_@@_mark \@@_trim_spaces_auxiii:w
  {
    #1
  }
%    \end{macrocode}
% \end{macro}
% \end{macro}
%
% \begin{macro}[EXP,pTF]{\keys_if_exist:nn}
%   A utility for others to see if a key exists.
%    \begin{macrocode}
\prg_new_conditional:Npnn \keys_if_exist:nn #1#2 { p , T , F , TF }
  {
    \cs_if_exist:cTF
      { \c_@@_code_root_str \@@_trim_spaces:n { #1 / #2 } }
      { \prg_return_true: }
      { \prg_return_false: }
  }
\prg_generate_conditional_variant:Nnn \keys_if_exist:nn { ne } { T , F , TF }
%    \end{macrocode}
% \end{macro}
%
% \begin{macro}[EXP,pTF]{\keys_if_choice_exist:nnn}
%   Just an alternative view on \cs{keys_if_exist:nnTF}.
%    \begin{macrocode}
\prg_new_conditional:Npnn \keys_if_choice_exist:nnn #1#2#3
  { p , T , F , TF }
  {
    \cs_if_exist:cTF
      { \c_@@_code_root_str \@@_trim_spaces:n { #1 / #2 / #3 } }
      { \prg_return_true: }
      { \prg_return_false: }
  }
%    \end{macrocode}
% \end{macro}
%
% \begin{macro}{\keys_show:nn, \keys_log:nn}
% \begin{macro}{\@@_show:Nnn}
% \begin{macro}{\@@_show:n}
% \begin{macro}{\@@_show:w}
% \begin{macro}{\@@_show:Nw}
%   To show a key, show its code using a message.
%    \begin{macrocode}
\cs_new_protected:Npn \keys_show:nn
  { \@@_show:Nnn \msg_show:nnxxxx }
\cs_new_protected:Npn \keys_log:nn
  { \@@_show:Nnn \msg_log:nnxxxx }
\cs_new_protected:Npn \@@_show:Nnn #1#2#3
  {
    #1 { keys } { show-key }
      { \@@_trim_spaces:n { #2 / #3 } }
      {
        \keys_if_exist:nnT {#2} {#3}
          {
            \exp_args:Nnf \msg_show_item_unbraced:nn { code }
              {
                \exp_args:Ne \@@_show:n
                  {
                    \exp_args:Nc \cs_replacement_spec:N
                    {
                      \c_@@_code_root_str
                      \@@_trim_spaces:n { #2 / #3 }
                    }
                  }
              }
          }
      }
      { } { }
  }
\cs_new:Npx \@@_show:n #1
  {
    \exp_not:N \@@_show:w
      #1
      \tl_to_str:n { \@@_precompile:n }
      #1
      \tl_to_str:n { \@@_precompile:n }
      \exp_not:N \s_@@_stop
  }
\use:x
  {
    \cs_new:Npn \exp_not:N \@@_show:w
      ##1 \tl_to_str:n { \@@_precompile:n }
      ##2 \tl_to_str:n { \@@_precompile:n }
      ##3 \exp_not:N \s_@@_stop
  }
  {
    \tl_if_blank:nTF {#2}
      {#1}
      { \@@_show:Nw #2 \s_@@_stop }
  }
\use:x
  {
    \cs_new:Npn \exp_not:N \@@_show:Nw ##1##2
      \c_right_brace_str \exp_not:N \s_@@_stop
  }
  {#2}
%    \end{macrocode}
% \end{macro}
% \end{macro}
% \end{macro}
% \end{macro}
% \end{macro}
%
% \subsection{Messages}
%
% For when there is a need to complain.
%    \begin{macrocode}
\msg_new:nnnn { keys } { bad-relative-key-path }
  { The~key~'#1'~is~not~inside~the~'#2'~path. }
  { The~key~'#1'~cannot~be~expressed~relative~to~path~'#2'. }
\msg_new:nnnn { keys } { boolean-values-only }
  { Key~'#1'~accepts~boolean~values~only. }
  { The~key~'#1'~only~accepts~the~values~'true'~and~'false'. }
\msg_new:nnnn { keys } { choice-unknown }
  { Key~'#1'~accepts~only~a~fixed~set~of~choices. }
  {
    The~key~'#1'~only~accepts~predefined~values,~
    and~'#2'~is~not~one~of~these.
  }
\msg_new:nnnn { keys } { unknown }
  { The~key~'#1'~is~unknown~and~is~being~ignored. }
  {
    The~module~'#2'~does~not~have~a~key~called~'#1'.\\
    Check~that~you~have~spelled~the~key~name~correctly.
  }
\msg_new:nnnn { keys } { nested-choice-key }
  { Attempt~to~define~'#1'~as~a~nested~choice~key. }
  {
    The~key~'#1'~cannot~be~defined~as~a~choice~as~the~parent~key~'#2'~is~
    itself~a~choice.
  }
\msg_new:nnnn { keys } { value-forbidden }
  { The~key~'#1'~does~not~take~a~value. }
  {
    The~key~'#1'~should~be~given~without~a~value.\\
    The~value~'#2'~was~present:~the~key~will~be~ignored.
  }
\msg_new:nnnn { keys } { value-required }
  { The~key~'#1'~requires~a~value. }
  {
    The~key~'#1'~must~have~a~value.\\
    No~value~was~present:~the~key~will~be~ignored.
  }
\msg_new:nnn { keys } { show-key }
  {
    The~key~#1~
    \tl_if_empty:nTF {#2}
      { is~undefined. }
      { has~the~properties: #2 . }
  }
\prop_gput:Nnn \g_msg_module_name_prop { keys } { LaTeX }
\prop_gput:Nnn \g_msg_module_type_prop { keys } { }
%    \end{macrocode}
%
%    \begin{macrocode}
%</package>
%    \end{macrocode}
%
%\end{implementation}
%
%\PrintIndex
