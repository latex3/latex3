% \iffalse meta-comment
%
%% File: expl3.dtx
%
% Copyright (C) 1990-2019 The LaTeX3 Project
%
% It may be distributed and/or modified under the conditions of the
% LaTeX Project Public License (LPPL), either version 1.3c of this
% license or (at your option) any later version.  The latest version
% of this license is in the file
%
%    https://www.latex-project.org/lppl.txt
%
% This file is part of the "l3kernel bundle" (The Work in LPPL)
% and all files in that bundle must be distributed together.
%
% -----------------------------------------------------------------------
%
% The development version of the bundle can be found at
%
%    https://github.com/latex3/latex3
%
% for those people who are interested.
%
%<*driver|generic|package|2ekernel>
\def\ExplFileDate{2019-09-28}%
%</driver|generic|package|2ekernel>
%<*driver>
\documentclass[full]{l3doc}
\usepackage{graphicx}
\begin{document}
  \DocInput{\jobname.dtx}
\end{document}
%</driver>
% \fi
%
% \providecommand\acro[1]{\textsc{\MakeLowercase{#1}}}
% \newenvironment{arg-description}{%
%   \begin{itemize}\def\makelabel##1{\hss\llap{\bfseries##1}}}{\end{itemize}}
%
% \title{^^A
%   The \textsf{expl3} package and \LaTeX3 programming^^A
% }
%
% \author{^^A
%  The \LaTeX3 Project\thanks
%    {^^A
%      E-mail:
%        \href{mailto:latex-team@latex-project.org}
%          {latex-team@latex-project.org}^^A
%    }^^A
% }
%
% \date{Released 2019-09-28}
%
% \maketitle
%
% \begin{documentation}
%
% \begin{abstract}
%
% This document gives an introduction to a new set of programming
% conventions that have been designed to meet the requirements of
% implementing large scale \TeX{} macro programming projects such as
% \LaTeX{}. These programming conventions are the base layer of \LaTeX3.
%
% The main features of the system described are:
% \begin{itemize}
%   \item classification of the macros (or, in \LaTeX{} terminology,
%     commands) into \LaTeX{} functions and \LaTeX{} parameters, and also
%     into modules containing related commands;
%   \item  a systematic naming scheme based on these classifications;
%   \item  a simple mechanism for controlling the expansion of a function's
%     arguments.
% \end{itemize}
% This system is being used as the basis for \TeX{} programming within
% the \LaTeX3 project. Note that the language is not intended for either
% document mark-up or style specification. Instead, it is intended that
% such features will be built on top of the conventions described here.
%
% This document is an introduction to the ideas behind the \pkg{expl3}
% programming interface. For the complete documentation of the programming
% layer provided by the \LaTeX3 Project, see the accompanying
% \texttt{interface3} document.
%
% \end{abstract}
%
% \section{Introduction}
%
% The first step to develop a \LaTeX{} kernel beyond \LaTeXe{} is to
% address how the underlying system is programmed.  Rather than the
% current mix of \LaTeX{} and \TeX{} macros, the \LaTeX3 system provides
% its own consistent interface to all of the functions needed to
% control \TeX{}.  A key part of this work is to ensure that everything
% is documented, so that \LaTeX{} programmers and users can work
% efficiently without needing to be familiar with the internal nature
% of the kernel or with plain \TeX{}.
%
% The \pkg{expl3} bundle provides this new programming interface for
% \LaTeX{}. To make programming systematic, \LaTeX3 uses some very
% different conventions to \LaTeXe{} or plain \TeX{}. As a result,
% programmers starting with \LaTeX3 need to become familiar with
% the syntax of the new language.
%
% The next section shows where this language fits into a complete
% \TeX{}-based document processing system.  We then describe the major
% features of the syntactic structure of command names, including the
% argument specification syntax used in function names.
%
% The practical ideas behind this argument syntax will be explained,
% together with the expansion control mechanism and the interface
% used to define variant forms of functions.
%
% As we shall demonstrate, the use of a structured naming scheme and of
% variant forms for functions greatly improves the readability of the
% code and hence also its reliability.  Moreover, experience has shown
% that the longer command names which result from the new syntax do not
% make the process of \emph{writing} code significantly harder.
%
% \section{Languages and interfaces}
%
% It is possible to identify several distinct languages related to the
% various interfaces that are needed in a \TeX{}-based document processing
% system.  This section looks at those we consider most important for
% the \LaTeX3 system.
%
% \begin{description}
%   \item[Document mark-up] This comprises those commands (often called
%     tags) that are to embedded in the document (the |.tex| file).
%
%     It is generally accepted that such mark-up should be essentially
%     \emph{declarative}. It may be traditional \TeX{}-based mark-up such
%      as \LaTeXe{}, as described in~\cite{A-W:LLa94} and~\cite{A-W:GMS94},
%     or a mark-up language defined via \acro{HTML} or \acro{XML}.
%
%     One problem with more traditional \TeX{} coding conventions (as
%     described in~\cite{A-W:K-TB}) is that the names and syntax of \TeX{}'s
%     primitive formatting commands are ingeniously designed to be
%     \enquote{natural} when used directly by the author as document mark-up
%     or in macros.  Ironically, the ubiquity (and widely recognised
%     superiority) of logical mark-up has meant that such explicit
%     formatting commands are almost never needed in documents or in
%     author-defined macros.  Thus they are used almost exclusively by
%     \TeX{} programmers to define higher-level commands, and their
%     idiosyncratic syntax is not at all popular with this community.
%     Moreover, many of them have names that could be very useful as
%     document mark-up tags were they not pre-empted as primitives
%     (\emph{e.g.}~\tn{box} or \tn{special}).
%
%   \item[Designer interface] This relates a (human) typographic
%     designer's specification for a document to a program that
%     \enquote{formats
%     the document}.  It should ideally use a declarative language that
%     facilitates expression of the relationship and spacing rules
%     specified for the layout of the various document elements.
%
%     This language is not embedded in document text and it will be very
%     different in form to the document mark-up language.  For \LaTeX{},
%     this level was almost completely missing  from \LaTeX{}2.09; \LaTeXe{}
%     made some improvements in this area but it is still the case that
%     implementing a design specification in  \LaTeX{} requires far more
%     \enquote{low-level} coding than is acceptable.
%
%   \item[Programmer interface]
%     This language is the implementation language within which the
%     basic typesetting functionality is implemented, building upon the
%     primitives of \TeX{} (or a  successor program).  It may also be used
%     to implement the previous two languages \enquote{within} \TeX{}, as in
%     the
%     current \LaTeX{} system.
%
% \end{description}
%
% The last layer is covered by the conventions described in this
% document, which describes a system aimed at providing a suitable
% basis for coding \LaTeX3. Its main distinguishing features are
% summarised here:
% \begin{itemize}
%   \item A consistent naming scheme for all commands, including \TeX{}
%     primitives.
%   \item The classification of commands as \LaTeX{} functions or \LaTeX{}
%     parameters, and also their division into modules according to their
%     functionality.
%   \item A simple mechanism for controlling argument expansion.
%   \item Provision of a set of core \LaTeX{} functions that is sufficient
%     for handling programming constructs such as queues, sets, stacks,
%     property lists.
%   \item A \TeX{} programming environment in which, for example, all
%     white space is ignored.
% \end{itemize}
%
% \section{The naming scheme}
%
% \LaTeX3 does not use |@| as a \enquote{letter} for defining
% internal macros.  Instead, the symbols |_| and |:|
% are used in internal macro names to provide structure. In
% contrast to the plain \TeX{} format and the \LaTeXe{} kernel, these
% extra letters are used only between parts of a macro name (no
% strange vowel replacement).
%
% While \TeX{} is actually a macro processor, by
% convention for the \pkg{expl3} programming language we distinguish between
% \emph{functions} and \emph{variables}. Functions can have arguments and they
% are either expanded or executed.  Variables can be assigned values and they
% are used in arguments to functions; they are not used directly but are
% manipulated by functions (including getting and setting functions).
% Functions and variables with a related functionality (for example accessing
% counters, or manipulating token lists, \emph{etc.})\ are collected together
% into a
% \emph{module}.
%
% \subsection{Examples}
%
% Before giving the details of the naming scheme, here are a few typical
% examples to indicate the flavour of the scheme; first some variable
% names.
% \begin{quote}
%   \cs{l_tmpa_box} is a local variable (hence the~|l_| prefix)
%     corresponding to a box register.\\
%   \cs{g_tmpa_int} is a global variable (hence the~|g_| prefix)
%     corresponding to an integer register (i.e.~a \TeX{} count
%     register).\\
%   \cs{c_empty_tl} is the constant~(|c_|) token list variable
%     that is always empty.
% \end{quote}
%
% Now here is an example of a typical function name.
%
% \cs{seq_push:Nn} is the function which puts the token list specified
% by its second argument onto the stack specified by its first argument.
% The different natures of the two arguments are indicated by the~|:Nn|
% suffix. The first argument must be a single token which \enquote{names}
% the stack parameter: such single-token arguments are denoted~|N|.
% The second argument is a normal \TeX{} \enquote{undelimited argument},
% which
% may either be a single token or a balanced, brace-delimited token
% list (which we shall here call a \textit{braced token list}): the~|n|
% denotes such a \enquote{normal} argument form. The name of the function
% indicates it belongs to the |seq| module.
%
% \subsection{Formal naming syntax}
%
% We shall now look in more detail at the syntax of these names. A
% function name in \LaTeX3 has a name consisting of three parts:
% \begin{quote}
%   |\|\meta{module}|_|\meta{description}|:|\meta{arg-spec}
% \end{quote}
% while a variable has (up to) four distinct parts to its name:
% \begin{quote}
%   |\|\meta{scope}|_|\meta{module}|_|\meta{description}|_|\meta{type}
% \end{quote}
%
% The syntax of all names contains
% \begin{quote}
%   \meta{module} and \meta{description}
% \end{quote}
% these both give information about the command.
%
% A \emph{module} is a collection of closely related functions and
% variables. Typical module names include~|int| for integer parameters
% and related functions,~|seq| for sequences and~|box| for boxes.
%
% Packages providing new programming functionality will add new modules
% as needed; the programmer can choose any unused name, consisting
% of letters only, for a module. In general, the module name and module
% prefix should be related: for example, the kernel module containing
% \texttt{box} functions is called \texttt{l3box}.  Module names and
% programmers' contact details are listed in \pkg{l3prefixes.csv}.
%
% The \emph{description} gives more detailed information about the
% function or parameter, and provides a unique name for it.  It should
% consist of letters and, possibly,~|_|~characters. In general, the
% description should use |_| to divide up \enquote{words} or other easy to
% follow parts of the name.  For example, the \LaTeX3 kernel provides
% \cs{if_cs_exist:N} which, as might be expected, tests if a command
% name exists.
%
% Where functions for variable manipulation can perform assignments
% either locally or globally, the latter case is indicated by the inclusion of
% a |g| in the second part of the function name. Thus \cs{tl_set:Nn} is a local
% function but \cs{tl_gset:Nn} acts globally. Functions of this type are
% always documented together, and the scope of action may therefore be
% inferred from the presence or absence of a |g|. See the next subsection for
% more detail on variable scope.
%
% \subsubsection{Separating private and public material}
%
% One of the issues with the \TeX{} language is that it doesn't support
% name spaces and encapsulation other than by convention. As a result
% nearly every internal command in the \LaTeXe{} kernel has eventually
% be used by extension packages as an entry point for modifications or
% extensions. The consequences of this is that nowadays it is next to
% impossible to change anything in the \LaTeXe{} kernel (even if it is
% clearly just an internal command) without breaking something.
%
% In \pkg{expl3} we hope to improve this situation drastically by
% clearly separating public interfaces (that extension packages can use
% and rely on) and private functions and variables (that should not
% appear outside of their module).  There is (nearly) no way to enforce
% this without severe computing overhead, so we implement it only
% through a naming convention, and some support mechanisms.  However, we
% think that this naming convention is easy to understand and to follow,
% so that we are confident that this will adopted and provides the
% desired results.
%
% Functions created by a module may either be \enquote{public} (documented
% with a defined interface) or \enquote{private} (to be used only within
% that module, and thus not formally documented). It is important that
% only documented interfaces are used; at the same time, it is necessary to
% show within the name of a function or variable whether it is public
% or private.
%
% To allow clear separation of these two cases, the following convention
% is used. Private functions should be defined with |__| added to the beginning
% of the module name. Thus
% \begin{verbatim}
%   \module_foo:nnn
% \end{verbatim}
% is a public function which should be documented while
% \begin{verbatim}
%   \__module_foo:nnn
% \end{verbatim}
% is private to the module, and should \emph{not} be used outside of that
% module.
%
% In the same way, private variables should use two "__" at the start of the
% module name, such that
% \begin{verbatim}
%   \l_module_foo_tl
% \end{verbatim}
% is a public variable and
% \begin{verbatim}
%   \l__module_foo_tl
% \end{verbatim}
% is private.
%
% \subsubsection{Using \texttt{@@} and \pkg{l3docstrip} to mark private code}
%
% The formal syntax for internal functions allows clear separation of public
% and private code, but includes redundant information (every internal function
% or variable includes \texttt{__\meta{module}}). To aid programmers, the
% \pkg{l3docstrip} program introduces the syntax
% \begin{quote}
%   \ttfamily
%   |%<@@=|\meta{module}|>|
% \end{quote}
% which then allows |@@| (and |_@@| in case of variables) to be used as
% a place holder for \texttt{__\meta{module}} in code. Thus for example
% \begin{verbatim}
%   %<@@=foo>
%   %    \begin{macrocode}
%   \cs_new:Npn \@@_function:n #1
%     ...
%   \tl_new:N \l_@@_my_tl
%   %    \end{macrocode}
% \end{verbatim}
% is converted by \pkg{l3docstrip} to
% \begin{verbatim}
%   \cs_new:Npn \__foo_function:n #1
%     ...
%   \tl_new:N \l__foo_my_tl
% \end{verbatim}
% on extraction. As you can see both |_@@| and |@@| are mapped to
% \texttt{__\meta{module}}, because we think that this helps to
% distinguish variables from functions in the source when the |@@|
% convention is used.
%
% \subsubsection{Variables: scope and type}
%
% The \meta{scope} part of the name describes how the variable can be
% accessed.  Variables are classified as local, global or constant.
% This \emph{scope} type appears as a code at the beginning of the name;
% the codes used are:
% \begin{arg-description}
%   \item[c] constants (global variables whose value should not be
%     changed);
%   \item[g] variables whose value should only be set globally;
%   \item[l] variables whose value should only be set locally.
% \end{arg-description}
%
% Separate functions are provided to assign data to local and global
% variables; for example, \cs{tl_set:Nn} and \cs{tl_gset:Nn} respectively
% set the value of a local or global \enquote{token list} variable.
% Note that it is a poor \TeX{} practice to intermix local and global
% assignments to a variable; otherwise you risk exhausting the save
% stack.\footnote{See \emph{The \TeX{}book}, p.\,301, for further
% information.}
%
% The \meta{type} is in the list of available
% \emph{data-types};\footnote{Of course, if a totally new data type is
% needed then this will not be the case. However, it is hoped that only
% the kernel team will need to create new data types.} these include the
% primitive \TeX{} data-types, such as the various registers, but to
% these are added data-types built within the \LaTeX{} programming
% system.
%
% The data types in \LaTeX3 are:
% \begin{description}
%   \item[bool]   either true or false (the \LaTeX3 implementation does
%                 not use \tn{iftrue} or \tn{iffalse});
%   \item[box]    box register;
%   \item[clist]  comma separated list;
%   \item[coffin] a \enquote{box with handles} --- a higher-level data
%                 type for carrying out |box| alignment operations;
%   \item[dim]    \enquote{rigid} lengths;
%   \item[fp]     floating-point values;
%   \item[ior]    an input stream (for reading from a file);
%   \item[iow]    an output stream (for writing to a file);
%   \item[int]    integer-valued count register;
%   \item[muskip] math mode \enquote{rubber} lengths;
%   \item[prop]   property list;
%   \item[seq]    sequence: a data-type used to implement lists (with
%                 access at both ends) and stacks;
%   \item[skip]   \enquote{rubber} lengths;
%   \item[str]    \TeX{} strings: a special case of |tl| in which all
%                 characters have category \enquote{other} (catcode~$12$),
%                 other than spaces which are category \enquote{space}
%                 (catcode~$10$);
%   \item[tl]     \enquote{token list variables}: placeholders for token lists.
% \end{description}
% When the \meta{type} and \meta{module} are identical (as often happens in
% the more basic modules) the \meta{module} part is often omitted for
% aesthetic reasons.
%
% The name \enquote{token list} may cause confusion, and so some
% background is useful.  \TeX{} works with tokens and lists of tokens,
% rather than characters. It provides two ways to store these token
% lists: within macros and as token registers (|toks|). The
% implementation in \LaTeX3 means that |toks| are not required, and that
% all operations for storing tokens can use the |tl| variable type.
%
% Experienced \TeX{} programmers will notice that some of the variable
% types listed are native \TeX{} registers whilst others are not. In
% general, the underlying \TeX{} implementation for a data structure may
% vary but the \emph{documented interface} will be stable. For example,
% the |prop| data type was originally implemented as a |toks|, but
% is currently built on top of the |tl| data structure.
%
% \subsubsection{Variables: guidance}
%
% Both comma lists and sequences have similar characteristics.
% They both use special delimiters to mark out one entry from the
% next, and are both accessible at both ends. In general, it is
% easier to create comma lists `by hand' as they can be typed
% in directly. User input often takes the form of a comma separated
% list and so there are many cases where this is the obvious
% data type to use. On the other hand, sequences use special internal
% tokens to separate entries. This means that they can be used to
% contain material that comma lists cannot (such as items that may
% themselves contain commas!). In general, comma lists should be
% preferred for creating fixed lists inside programs and for
% handling user input where commas will not occur. On the other
% hand, sequences should be used to store arbitrary lists of
% data.
%
% \pkg{expl3} implements stacks using the sequence data structure.
% Thus creating stacks involves first creating a sequence, and
% then using the sequence functions which work in a stack manner
% (\cs{seq_push:Nn}, \emph{etc}.).
%
% Due to the nature of the underlying \TeX{} implementation, it is
% possible to assign values to token list variables and comma lists
% without first declaring them. However, this is \emph{not supported
% behavior}. The \LaTeX3 coding convention is that all variables must
% be declared before use.
%
% The \pkg{expl3} package can be loaded with the \texttt{check-declarations}
% option to verify that all variables are declared before use. This has
% a performance implication and is therefore intended for testing during
% development and not for use in production documents.
%
% \subsubsection{Functions: argument specifications}
%
% Function names end with an \meta{arg-spec} after a colon.  This
% gives an indication of the types of argument that a function takes,
% and provides a convenient method of naming similar functions that
% differ only in their argument forms (see the next section for
% examples).
%
% The \meta{arg-spec} consists of a (possibly empty) list of letters,
% each denoting one argument of the function. The letter, including
% its case, conveys information about the type of argument required.
%
% All functions have a base form with arguments using one of the
% following argument specifiers:
% \begin{arg-description}
%   \item[n]  Unexpanded token or braced token list.\\
%     This is a standard \TeX{} undelimited macro argument.
%   \item[N]  Single token (unlike~|n|, the argument must \emph{not} be
%     surrounded by braces).\\
%     A typical example of a command taking an~|N|
%     argument is~|\cs_set|, in which the command being defined must be
%     unbraced.
%   \item[p]  Primitive \TeX{} parameter specification.\\
%     This can be something simple like~|#1#2#3|, but may use arbitrary
%     delimited argument syntax such as: |#1,#2\q_stop#3|. This is used
%     when defining functions.
%   \item[T,F]
%     These are special cases of~|n| arguments, used for the
%     true and false code in conditional commands.
% \end{arg-description}
% There are two other specifiers with more general meanings:
% \begin{arg-description}
%   \item[D] This means: \textbf{Do not use}. This special case is used
%     for \TeX{} primitives.  Programmers outside the kernel team should
%     not use these functions!
%   \item[w] This means that the argument syntax is \enquote{weird} in that it
%     does not follow any standard rule.  It is used for functions with
%     arguments that take non standard forms: examples are \TeX{}-level
%     delimited arguments and the boolean tests needed after certain
%     primitive |\if|\ldots{} commands.
% \end{arg-description}
%
% In case of |n| arguments that consist of a single token the
% surrounding braces can be omitted in nearly all
% situations---functions that force the use of braces even for single
% token arguments are explicitly mentioned. However, programmers are
% encouraged to always use braces around \texttt{n} arguments, as this
% makes the relationship between function and argument clearer.
%
% Further argument specifiers are available as part of the expansion
% control system.  These are discussed in the next section.
%
% \section{Expansion control}
%
% Let's take a look at some typical operations one might want to
% perform. Suppose we maintain a stack of open files and we use the
% stack |\g_ior_file_name_seq| to keep track of them (\texttt{ior} is
% the prefix used for the file reading module). The basic operation here
% is to push a name onto this stack which could be done by the operation
% \begin{quote}
%   \cs{seq_gpush:Nn} |\g_ior_file_name_seq {#1}|
% \end{quote}
% where |#1| is the filename. In other words, this operation would
% push the file name as is onto the stack.
%
% However, we might face a situation where the filename is stored in
% a variable of some sort, say |\l_ior_curr_file_tl|. In this case we
% want to retrieve the value of the variable. If we simply use
% \begin{quote}
%   \cs{seq_gpush:Nn} |\g_ior_file_name_seq| |\l_ior_curr_file_tl|
% \end{quote}
% we do not get the value of the variable pushed onto the stack,
% only the variable name itself. Instead a suitable number of
% \cs{exp_after:wN} would be necessary (together with extra braces) to
% change the order of expansion,\footnote{\cs{exp_after:wN} is
% the \LaTeX3 name for the \TeX{} \tn{expandafter} primitive.} \emph{i.e.}
% \begin{quote}
%   \cs{exp_after:wN}                              \\
%   |   |\cs{seq_gpush:Nn}                         \\
%   \cs{exp_after:wN}                              \\
%   |   \g_ior_file_name_seq|                      \\
%   \cs{exp_after:wN}                              \\
%   |   { \l_ior_curr_file_tl }|
% \end{quote}
%
% The above example is probably the simplest case but already shows
% how the code changes to something difficult to understand.
% Furthermore there is an assumption in this: that the storage bin
% reveals its contents after exactly one expansion. Relying on this
% means that you cannot do proper checking plus you have to know
% exactly how a storage bin acts in order to get the correct number
% of expansions.  Therefore \LaTeX3 provides the programmer with a
% general scheme that keeps the code compact and easy to understand.
%
% To denote that some argument to a function needs special treatment one
% just uses different letters in the arg-spec part of the function to
% mark the desired behavior. In the above example one would write
% \begin{quote}
%   \cs{seq_gpush:NV} |\g_ior_file_name_seq \l_ior_curr_file_tl|
% \end{quote}
% to achieve the desired effect. Here the |V| (the second argument)
% is for \enquote{retrieve the value of the variable} before passing it to
% the base function.
%
% The following letters can be used to denote special treatment of
% arguments before passing it to the base function:
% \begin{description}
%   \item[c] Character string used as a command name.\\ The argument (a
%     token or braced token list) is \emph{fully expanded}; the result
%     must be a sequence of characters which is then used to construct a
%     command name (\emph{via}~\tn{csname} \ldots \tn{endcsname}).  This
%     command name is a single token that is passed to the function as
%     the argument. Hence
%     \begin{quote}
%       \cs{seq_gpush:cV} |{ g_file_name_seq }| \cs{l_tmpa_tl}
%     \end{quote}
%     is equivalent to
%     \begin{quote}
%       \cs{seq_gpush:NV} |\g_file_name_seq| \cs{l_tmpa_tl}.
%     \end{quote}
%     Full expansion means that (a) the entire
%     argument must be expandable and (b) any variables are
%     converted to their content. So the preceding examples are also
%     equivalent to
%     \begin{quote}
%       \cs{tl_new:N} |\g_file_seq_name_tl| \\
%       \cs{tl_gset:Nn} |\g_file_seq_name_tl { g_file_name_seq }| \\
%       \cs{seq_gpush:cV} |{| \cs{tl_use:N} |\g_file_seq_name_tl }| \cs{l_tmpa_tl}.
%     \end{quote}
%     (Token list variables are expandable and we could omit the
%     accessor function \cs{tl_use:N}.  Other variable types require the
%     appropriate \cs{\meta{var}_use:N} functions to be used in this
%     context.)
%   \item[V]  Value of a variable.\\
%     This means that the contents of the register in question is used
%     as the argument, be it an integer, a length-type register, a token
%     list variable or similar. The value is passed to the function as a
%     braced token list.  Can be applied to variables which have a
%     \cs{\meta{var}_use:N} function (other than floating points and
%     boxes), and which therefore deliver a single \enquote{value}.
%   \item[v] Value of a register, constructed from a character string
%     used as a command name.\\
%     This is a combination of |c| and |V| which first constructs a
%     control sequence from the argument and then passes the value of
%     the resulting register to the function.  Can be applied to
%     variables which have a \cs{\meta{var}_use:N} function (other than
%     floating points and boxes), and which therefore deliver a single
%     \enquote{value}.
%   \item[x]  Fully-expanded token or braced token list.\\
%     This means that the argument is expanded as in the replacement
%     text of an~\tn{edef}, and the expansion is passed to the function as
%     a braced token list.  Expansion takes place until only unexpandable
%     tokens are left.  |x|-type arguments cannot be nested.
%   \item[e]  Fully-expanded token or braced token list which does
%     not require doubled |#| tokens. This expansions is very similar
%     to |x|-type but may be nested and does not require that |#|
%     tokens are doubled.
%   \item[f] Expanding the first token recursively in a braced token
%     list.\\ Almost the same as the |x| type except here the token list
%     is expanded fully until the first unexpandable token is found and
%     the rest is left unchanged. Note that if this function finds a
%     space at the beginning of the argument it gobbles it and does not
%     expand the next token.
%   \item[o]  One-level-expanded token or braced token list.\\
%     This means that the argument is expanded one level, as by
%     \tn{expandafter}, and the expansion is passed to the function as a
%     braced token list.  Note that if the original argument is a braced
%     token list then only the first token in that list is expanded.
%     In general, using \texttt{V} should be preferred to using
%     \texttt{o} for simple variable retrieval.
% \end{description}
%
% \subsection{Simpler means better}
%
% Anyone who programs in \TeX{} is frustratingly familiar with the
% problem of arranging that arguments to functions are suitably expanded
% before the function is called.  To illustrate how expansion control
% can bring instant relief to this problem we shall consider two
% examples copied from \texttt{latex.ltx}.
%
% \begin{verbatim}
%        \global\expandafter\let
%              \csname\cf@encoding \string#1\expandafter\endcsname
%              \csname ?\string#1\endcsname
% \end{verbatim}
% This first piece of code is in essence simply a global \tn{let} whose
% two arguments firstly have to be constructed before \tn{let} is
% executed. The |#1| is a control sequence name such as
% |\textcurrency|. The token to be defined is obtained by
% concatenating the characters of the current font encoding stored in
% |\cf@encoding|, which has to be fully expanded, and the name of the
% symbol. The second token is the same except it uses the default
% encoding |?|. The result is a mess of interwoven \tn{expandafter}
% and \tn{csname} beloved of all \TeX{} programmers, and the code is
% essentially unreadable.
%
% Using the conventions and functionality outlined here, the task would
% be achieved with code such as this:
% \begin{verbatim}
%   \cs_gset_eq:cc
%     { \cf@encoding \token_to_str:N  #1 } { ? \token_to_str:N #1 }
% \end{verbatim}
% The command \cs{cs_gset_eq:cc} is a global~\tn{let} that generates
% command names out of both of its arguments before making the
% definition. This produces code that is far more readable and more
% likely to be correct first time. (\cs{token_to_str:N} is the \LaTeX3
% name for \tn{string}.)
%
% Here is the second example.
% \begin{verbatim}
%   \expandafter
%     \in@
%   \csname sym#3%
%     \expandafter
%       \endcsname
%     \expandafter
%       {%
%     \group@list}%
% \end{verbatim}
% This piece of code is part of the definition of another function. It
% first produces two things: a token list, by expanding |\group@list| once;
% and a token whose name comes from~`|sym#3|'.  Then the function~\cs{in@}
% is called and this tests if its first argument occurs in the token list
% of its second argument.
%
% Again we can improve enormously on the code.  First we shall rename
% the function~\cs{in@}, which tests if its first argument appears
% within its second argument, according to our conventions.  Such a
% function takes two normal \enquote{\texttt{n}} arguments and operates
% on token lists: it might reasonably be named |\tl_test_in:nn|.  Thus
% the variant function we need would be defined with the appropriate
% argument types and its name would be |\tl_test_in:cV|.  Now this code
% fragment would be simply:
% \begin{verbatim}
%   \tl_test_in:cV { sym #3 } \group@list
% \end{verbatim}
% This code could be improved further by using a sequence |\l_group_seq|
% rather than the bare token list |\group@list|.  Note that, in addition
% to the lack of \tn{expandafter}, the space after the~|}| is
% silently ignored since all white space is ignored in this programming
% environment.
%
% \subsection{New functions from old}
%
% For many common functions the \LaTeX3 kernel provides variants
% with a range of argument forms, and similarly it is expected that
% extension packages providing new functions will make them available in
% all the commonly needed forms.
%
% However, there will be occasions where it is necessary to construct a
% new such variant form; therefore the expansion module provides a
% straightforward mechanism for the creation of functions with any
% required argument type, starting from a function that takes \enquote{normal}
% \TeX{} undelimited arguments.
%
% To illustrate this let us suppose you have a \enquote{base function}
% |\demo_cmd:Nnn| that takes three normal arguments, and that you need
% to construct the variant |\demo_cmd:cnx|, for which the first argument
% is used to construct the \emph{name} of a command, whilst the third
% argument must be fully expanded before being passed to
% |\demo_cmd:Nnn|.
% To produce the variant form from the base form, simply use this:
% \begin{verbatim}
%   \cs_generate_variant:Nn \demo_cmd:Nnn { cnx }
% \end{verbatim}
% This defines the variant form so that you can then write, for example:
% \begin{verbatim}
%   \demo_cmd:cnx { abc } { pq } { \rst \xyz }
% \end{verbatim}
% rather than \ldots\ well, something like this!
% \begin{verbatim}
%   \def \tempa {{pq}}%
%   \edef \tempb {\rst \xyz}%
%   \expandafter
%     \demo@cmd:nnn
%   \csname abc%
%     \expandafter
%       \expandafter
%     \expandafter
%         \endcsname
%     \expandafter
%       \tempa
%     \expandafter
%       {%
%     \tempb
%       }%
% \end{verbatim}
%
% Another example: you may wish to declare a function
% |\demo_cmd_b:xcxcx|, a variant of an existing function
% |\demo_cmd_b:nnnnn|, that fully
% expands arguments 1,~3 and~5, and produces commands to pass as
% arguments 2 and~4 using~\tn{csname}.
% The definition you need is simply
% \begin{verbatim}
%   \cs_generate_variant:Nn \demo_cmd_b:nnnnn { xcxcx }
% \end{verbatim}
%
% This extension mechanism is written so that if the same new form of
% some existing command is implemented by two extension packages then the
% two definitions are identical and thus no conflict occurs.
%
% \section{The distribution}
%
% At present, the \pkg{expl3} modules are designed to be loaded on top
% of \LaTeXe{}. In time, a \LaTeX3 format will be produced based on this
% code. This allows the code to be used in \LaTeXe{} packages \emph{now}
% while a stand-alone \LaTeX3 is developed.
%
% \begin{bfseries}
%   While \pkg{expl3} is still experimental, the bundle is now regarded
%   as broadly stable. The syntax conventions and functions provided
%   are now ready for wider use. There may still be changes to some
%   functions, but these will be minor when compared to the scope of
%   \pkg{expl3}.
% \end{bfseries}
%
% New modules will be added to the distributed version of \pkg{expl3}
% as they reach maturity. At present, the \pkg{expl3} bundle consists
% of a number of modules, most of which are loaded by including the
% line:
% \begin{verbatim}
%   \RequirePackage{expl3}
% \end{verbatim}
% in a \LaTeXe{} package, class or other file. The \pkg{expl3} modules
% regarded as stable, and therefore suitable for basing real code on,
% are as follows:
%
% \begin{description}
%   \providecommand\explpkg[2]{\item[#1]#2}
%   \explpkg{l3basics}{
%     This contains the basic definition modules used
%     by the other packages.
%   }
%   \explpkg{l3box}{
%     Primitives for dealing with boxes.
%   }
%   \explpkg{l3clist}{
%     Methods for manipulating comma-separated token lists.
%   }
%   \explpkg{l3coffins}{
%     Augmented box constructs for alignment operations.
%   }
%   \explpkg{l3expan}{
%     This is the argument expansion module discussed earlier in this
%     document.
%   }
%   \explpkg{l3int}{
%     This implements the integer data-type \texttt{int}.
%   }
%   \explpkg{l3keys}{
%     For processing lists of the form
%     \texttt{\{ key1=val1 , key2=val2 \}}, intended to work
%     as a \LaTeX3 version of \pkg{xkeyval}/\pkg{kvoptions}, although
%     with input syntax more like that of \pkg{pgfkeys}.
%   }
%   \explpkg{l3msg}{
%     Communicating with the user: includes low-level hooks to allow
%     messages to be filtered (higher-level interface for filtering
%     to be written!).
%   }
%   \explpkg{l3names}{
%     This sets up the basic naming scheme and renames all
%     the \TeX{} primitives.
%   }
%   \explpkg{l3prg}{
%     Program control structures such as boolean data type |bool|, generic
%     do-while loops, and conditional flow.
%   }
%   \explpkg{l3prop}{
%     This implements the data-type for \enquote{property lists} that are used, in
%     particular, for storing key/value pairs.
%   }
%   \explpkg{l3quark}{
%     A \enquote{quark} is a command that is defined to expand to
%     itself!  Therefore they must never be expanded as this would generate an
%     infinite recursion; they do however have many uses, \emph{e.g.}~as
%     special markers and delimiters within code.
%   }
%   \explpkg{l3seq}{
%     This implements data-types such as queues and stacks.
%   }
%   \explpkg{l3skip}{
%     Implements the \enquote{rubber length} datatype \texttt{skip}, the
%     \enquote{rigid length} datatype |dim|, and the math mode
%     \enquote{rubber length} datatype \texttt{muskip}.
%   }
%   \explpkg{l3tl}{
%     This implements a basic data-type, called a \textit{token-list
%     variable} (|tl var.|), used for storing named token lists: these are
%     \TeX{} macros with no arguments.
%   }
%   \explpkg{l3token}{
%     Analysing token lists and token streams, including peeking ahead to
%     see what's coming next and inspecting tokens to detect which kind
%     they are.
%   }
% \end{description}
%
% \section{Moving from \LaTeXe{} to \LaTeX3}
%
% To help programmers to use \LaTeX3 code in existing \LaTeXe{} package,
% some short notes on making the change are probably desirable.
% Suggestions for inclusion here are welcome! Some of the following
% is concerned with code, and some with coding style.
%
% \begin{itemize}
%   \item \pkg{expl3} is mainly focused on programming. This means that
%     some areas still require the use of \LaTeXe{} internal macros.
%     For example, you may well need \tn{@ifpackageloaded}, as there
%     is currently no native \LaTeX3 package loading module.
%   \item User level macros should be generated using the mechanism
%     available in the \pkg{xparse} package, which is part of the
%     \texttt{l3package} bundle, available from CTAN or the \LaTeX3 SVN
%     repository.
%   \item At an internal level, most functions should be generated
%     \tn{long} (using \cs{cs_new:Npn}) rather than \enquote{short} (using
%     \cs{cs_new_nopar:Npn}).
%   \item Where possible, declare all variables and functions (using
%     \cs{cs_new:Npn}, \cs{tl_new:N}, etc.) before use.
%   \item Prefer \enquote{higher-level} functions over \enquote{lower-level},
%     where possible. So for example use \cs{cs_if_exist:N(TF)} and not
%     \cs{if_cs_exist:N}.
%   \item Use space to make code readable. In general, we recommend
%     a layout such as:
%     \begin{verbatim}
%       \cs_new:Npn \foo_bar:Nn #1#2
%         {
%           \cs_if_exist:NTF #1
%             { \__foo_bar:n {#2} }
%             { \__foo_bar:nn {#2} { literal } }
%         }
%     \end{verbatim}
%     where spaces are used around |{| and |}| except for isolated
%     |#1|, |#2|, \emph{etc.}
%   \item Put different code items on separate lines: readability is
%     much more useful than compactness.
%   \item Use long, descriptive names for functions and variables,
%     and for auxiliary functions use the parent function name plus
%     |aux|, |auxi|, |auxii| and so on.
%   \item If in doubt, ask the team via the LaTeX-L list: someone will
%     soon get back to you!
% \end{itemize}
%
% \section{Load-time options for \pkg{expl3}}
%
% To support code authors, the \pkg{expl3} package for \LaTeXe{} includes
% a small number of load-time options. These all work in a key--value
% sense, recognising the \texttt{true} and \texttt{false} values. Giving
% the option name alone is equivalent to using the option with the
% \texttt{true} value.
%
% \DescribeOption{undo-recent-deprecations}
% The \texttt{undo-recent-deprecations} option suppresses deprecation
% errors for the first six months after a command is deprecated.  It is
% intended as a last resort measure for users of packages that were not
% updated in time.
%
% \DescribeOption{check-declarations}
% All variables used in \LaTeX3 code should be declared. This is enforced
% by \TeX{} for variable types based on \TeX{} registers, but not for those
% which are constructed using macros as the underlying storage system. The
% \texttt{check-declarations} option enables checking for all variable
% assignments, issuing an error if any variables are assigned without being
% initialised.  See also \cs{debug_on:n} \texttt{\{check-declarations\}}
% in \pkg{l3candidates} for finer control.
%
% \DescribeOption{log-functions}
% The \texttt{log-functions} option is used to enable recording of every new
% function name in the \texttt{.log} file. This is useful for debugging
% purposes, as it means that there is a complete list of all functions
% created by each module loaded (with the exceptions of a very small number
% required by the bootstrap code for \LaTeX3).  See also \cs{debug_on:n}
% \texttt{\{log-functions\}} in \pkg{l3candidates} for finer control.
%
% \DescribeOption{enable-debug}
% To allow more localized checking and logging than provided by
% \texttt{check-declarations} and \texttt{log-functions}, \pkg{expl3}
% provides a few \cs[no-index]{debug_\ldots{}} functions (described
% elsewhere) that turn on the corresponding checks within a group.
% These functions can only be used if \pkg{expl3} is loaded with the
% \texttt{enable-debug} option.
%
% \DescribeOption{backend}
% Selects the backend to be used for color, graphics and related operations that
% are backend-dependent. Options available are
% \begin{itemize}[font = \texttt]
%   \item[dvips] Use the \texttt{dvips} driver.
%   \item[dvipdfmx] Use the \texttt{dvipdfmx} driver.
%   \item[dvisvgm] Use the \texttt{dvisvgm} driver.
%   \item[pdfmode] Use the \texttt{pdfmode} driver (direct PDF output from
%     \pdfTeX{} or \LuaTeX{}).
%   \item[xdvipdfmx] Use the \texttt{xdvipdfmx} driver (\XeTeX{} only).
% \end{itemize}
%
% \section{Using \pkg{expl3} with formats other than \LaTeXe{}}
%
% As well as the \LaTeXe{} package \pkg{expl3}, there is also a
% \enquote{generic} loader for the code, \texttt{expl3.tex}. This may be
% loaded using the plain \TeX{} syntax
% \begin{verbatim}
%   \input expl3-generic %
% \end{verbatim}
% This enables the programming layer to work with the other formats.
% As no options are available loading in this way, the \enquote{native}
% drivers are automatically used. If this \enquote{generic} loader is
% used with \LaTeXe{} the code automatically switches to the appropriate
% package route.
%
% After loading the programming layer using the generic interface, the
% commands \cs{ExplSyntaxOn} and \cs{ExplSyntaxOff} and the code-level
% functions and variables detailed in \pkg{interface3} are available.
% Note that other \LaTeXe{} packages \emph{using} \pkg{expl3} are not
% loadable: package loading is dependent on the \LaTeXe{} package-management
% mechanism.
%
% \section{Engine/primitive requirements}
%
% To use \pkg{expl3} and the higher level packages provided by the
% team, the minimal set of primitive requirements is currently
% \begin{itemize}
%    \item All of those from \TeX90.
%    \item All of those from \eTeX{} \emph{excluding} |\TeXXeTstate|,
%      |\beginL|, |\beginR|, |\endL| and |\endR| (\emph{i.e.}~excluding
%      \TeX{}-\kern0pt-\reflectbox{\TeX{}}).
%    \item Functionality equivalent to the \pdfTeX{} primitive
%      |\pdfstrcmp|.
% \end{itemize}
% Any engine which defines |\pdfoutput| (\emph{i.e.}~allows direct production
% of a PDF file without a DVI intermediate) must also provide |\pdfcolorstack|,
% |\pdfliteral|, |\pdfmatrix|, |\pdfrestore| and |\pdfsave| or equivalent
% functionality. Fully Unicode engines must provide a method for producing
% character tokens in an expandable manner.
%
% Practically, these requirements are met by the engines
% \begin{itemize}
%    \item \pdfTeX{} v1.40 or later.
%    \item \XeTeX{} v0.99992 or later.
%    \item \LuaTeX{} v0.76 or later.
%    \item e-(u)\pTeX{} mid-2012 or later.
% \end{itemize}
%
% Additional modules beyond the core of \pkg{expl3} may require additional
% primitives. In particular, third-party authors may significantly
% extend the primitive coverage requirements.
%
% \section{The \LaTeX3 Project}
%
% Development of \LaTeX3 is carried out by The \LaTeX3 Project:
% \url{https://www.latex-project.org/latex3/}.
%
% \begin{thebibliography}{1}
%
%   \bibitem{A-W:K-TB}
%     Donald E Knuth
%     \newblock \emph{The \TeX{}book}.
%     \newblock Addison-Wesley, Reading, Massachusetts, 1984.
%
%   \bibitem{A-W:GMS94}
%     Goossens, Mittelbach and Samarin.
%     \newblock \emph{ The \LaTeX{} Companion}.
%     \newblock Addison-Wesley, Reading, Massachusetts, 1994.
%
%   \bibitem{A-W:LLa94}
%     Leslie Lamport.
%     \newblock \emph{\LaTeX{}: A Document Preparation System}.
%     \newblock Addison-Wesley, Reading, Massachusetts, second edition, 1994.
%
%   \bibitem{tub:MR97-1}
%     Frank Mittelbach and Chris Rowley.
%     \newblock \enquote{The \LaTeX3 Project}.
%     \newblock \emph{TUGboat},
%     Vol.\,18, No.\,3, pp.\,195--198, 1997.
%
% \end{thebibliography}
%
% \end{documentation}
%
% \begin{implementation}
%
% \section{\pkg{expl3} implementation}
%
% The implementation here covers several things. There are two
% \enquote{loaders} to define: the parts of the code that are specific to
% \LaTeXe{} or to non-\LaTeXe{} formats. These have to cover the same
% concepts as each other but in rather different ways: as a result, much
% of the code is given in separate blocks. There is also a short piece of
% code for the start of the \enquote{payload}: this is to ensure that
% loading is always done in the right way.
%
% \subsection{Loader interlock}
%
% A short piece of set up to check that the loader and \enquote{payload}
% versions match.
%
% \begin{macro}{\ExplLoaderFileDate}
%   As DocStrip is used to generate \cs{ExplFileDate}
%   for all files from the same source, it has to match. Thus the loaders
%   simply save this information with a new name.
%    \begin{macrocode}
%<*loader>
\let\ExplLoaderFileDate\ExplFileDate
%</loader>
%    \end{macrocode}
% \end{macro}
%
% The interlock test itself is simple: \cs{ExplLoaderFileDate} must be
% defined and identical to \cs{ExplFileDate}. As this has to work for
% both \LaTeXe{} and other formats, there is some auto-detection involved.
% (Done this way avoids having two very similar blocks for \LaTeXe{} and
% other formats.)
%    \begin{macrocode}
%<*!loader>
\begingroup
  \def\next{\endgroup}%
  \expandafter\ifx\csname PackageError\endcsname\relax
    \begingroup
      \def\next{\endgroup\endgroup}%
      \def\PackageError#1#2#3%
        {%
          \endgroup
          \errhelp{#3}%
          \errmessage{#1 Error: #2!}%
        }%
  \fi
  \expandafter\ifx\csname ExplLoaderFileDate\endcsname\relax
    \def\next
      {%
        \PackageError{expl3}{No expl3 loader detected}
          {%
            You have attempted to use the expl3 code directly rather than using
            the correct loader. Loading of expl3 will abort.
          }%
        \endgroup
        \endinput
      }
  \else
    \ifx\ExplLoaderFileDate\ExplFileDate
    \else
      \def\next
        {%
          \PackageError{expl3}{Mismatched expl3 files detected}
            {%
              You have attempted to load expl3 with mismatched files:
              probably you have one or more files 'locally installed' which
              are in conflict. Loading of expl3 will abort.
            }%
          \endgroup
          \endinput
        }%
    \fi
\fi
\next
%</!loader>
%    \end{macrocode}
%
% A reload test for the payload, just in case.
%    \begin{macrocode}
%<*!loader>
\begingroup\expandafter\expandafter\expandafter\endgroup
\expandafter\ifx\csname ver@expl3-code.tex\endcsname\relax
  \expandafter\edef\csname ver@expl3-code.tex\endcsname
    {%
      \ExplFileDate\space
      L3 programming layer
    }%
\else
  \expandafter\endinput
\fi
%</!loader>
%    \end{macrocode}
%
% All good: log the version of the code used (for log completeness). As this
% is more-or-less \cs{ProvidesPackage} without a separate file and as this also
% needs to work without \LaTeXe{}, just write the information directly to the
% log.
%    \begin{macrocode}
%<*!loader>
\immediate\write-1 %
  {%
    Package: expl3
      \ExplFileDate\space
      L3 programming layer (code)%
  }%
%</!loader>
%    \end{macrocode}
%
% \subsection{\LaTeXe{} loaders}
%
% Loading with \LaTeXe{} may be as part of the format (pre-loading)
% or as a package. We have to allow for both possible paths, and of
% course the package being loaded on to of the pre-load. That means
% the the code here must be safe against re-loading.
%
%    \begin{macrocode}
%<*package&loader|2ekernel>
%    \end{macrocode}
%
% Identify the package or add to the format message.
%    \begin{macrocode}
%<*2ekernel>
\everyjob\expandafter{\the\everyjob
  \message{^^J\ExplFileDate\space L3 programming layer}%
}
%</2ekernel>
%<*!2ekernel>
\ProvidesPackage{expl3}
  [%
    \ExplFileDate\space
    L3 programming layer (loader)
  ]%
%</!2ekernel>
%    \end{macrocode}
%
% \begin{macro}{\ProvidesExplPackage, \ProvidesExplClass, \ProvidesExplFile}
%   For other packages and classes building on this one it is convenient
%   not to need \cs{ExplSyntaxOn} each time.
%    \begin{macrocode}
\protected\def\ProvidesExplPackage#1#2#3#4%
  {%
    \ProvidesPackage{#1}[#2 \ifx\relax#3\relax\else v#3\space\fi #4]%
    \ExplSyntaxOn
  }%
\protected\def\ProvidesExplClass#1#2#3#4%
  {%
    \ProvidesClass{#1}[#2 \ifx\relax#3\relax\else v#3\space\fi #4]%
    \ExplSyntaxOn
  }%
\protected\def\ProvidesExplFile#1#2#3#4%
  {%
    \ProvidesFile{#1}[#2 \ifx\relax#3\relax\else v#3\space\fi #4]%
    \ExplSyntaxOn
  }%
%    \end{macrocode}
% \end{macro}
%
%  Load the business end: this leaves \cs{expl3} syntax on.
%  The test ensures we only load once without needing to know if
%  there was a preloading step.
%    \begin{macrocode}
\begingroup\expandafter\expandafter\expandafter\endgroup
\expandafter\ifx\csname tex\string _let:D\endcsname\relax
  \expandafter\@firstofone
\else
  \expandafter\@gobble
\fi
  {\input{expl3-code.tex}}%
%    \end{macrocode}
%
% A check that the bootstrap code did not abort loading: if it did,
% bail out silently here.
%    \begin{macrocode}
\begingroup\expandafter\expandafter\expandafter\endgroup
\expandafter\ifx\csname tex\string _let:D\endcsname\relax
  \expandafter\endinput
\fi
%    \end{macrocode}
%
% If \pkg{expl3} was pre-loaded, we now have to deal with the fact that
% the syntax will not be activated for the package mode version:
% simply turn it on. We use \tn{@pushfilenameaux} as a marker: it's defined
% a little later.
%    \begin{macrocode}
%<*!2ekernel>
\ifdefined\@pushfilenameaux
  \ExplSyntaxOn
\fi
%</!2ekernel>
%    \end{macrocode}
%
%    \begin{macrocode}
%<@@=expl>
%    \end{macrocode}
%
% \begin{variable}{\c_@@_def_ext_tl}
%   Needed by \LaTeXe{}, and avoiding a re-load issue.
%    \begin{macrocode}
\cs_if_exist:NF \c_@@_def_ext_tl
  { \tl_const:Nn \c_@@_def_ext_tl { def } }
%    \end{macrocode}
% \end{variable}
%
% \begin{macro}{\__kernel_sys_configuration_load:n}
%   For \LaTeXe{}, option loading uses the built-in functionality: as this
%   may be a reload we have to force the issue.
%    \begin{macrocode}
\cs_gset_protected:Npn \__kernel_sys_configuration_load:n #1
  {
    \cs_undefine:c { ver@ #1 .def }
    \@onefilewithoptions {#1} [ ] [ ]
      \c_@@_def_ext_tl
  }
%    \end{macrocode}
% \end{macro}
%
% \begin{variable}{\l_@@_options_clist}
%    \begin{macrocode}
%<*!2ekernel>
\clist_new:N \l_@@_options_clist
\DeclareOption*
  { \clist_put_right:NV \l_@@_options_clist \CurrentOption }
\ProcessOptions \relax
%</!2ekernel>
%    \end{macrocode}
% \end{variable}
%
%   Pretty standard setting creation.
%    \begin{macrocode}
\keys_define:nn { sys }
  {
    backend .choices:nn =
      { dvipdfmx , dvips , dvisvgm , pdfmode , xdvipdfmx }
      { \sys_load_backend:n {#1} } ,
    check-declarations .code:n =
      {
        \sys_load_debug:
        \debug_on:n { check-declarations }
      } ,
    driver .meta:n = { backend = #1 } ,
    enable-debug .code:n =
      \sys_load_debug: ,
    log-functions .code:n =
      {
        \sys_load_debug:
        \debug_on:n { log-functions }
      } ,
    undo-recent-deprecations .code:n =
      {
        \sys_load_deprecation:
        \bool_gset_true:N \g__kernel_deprecation_undo_recent_bool
      }
  }
%    \end{macrocode}
%
%  A backend has to be in place by the start of the document.
%    \begin{macrocode}
%<*2ekernel>
\AtBeginDocument
  {
    \str_if_exist:NF \c_sys_backend_str
      { \sys_load_backend:n { } }
  }
%</2ekernel>
%<*!2ekernel>
\keys_set:nV { sys } \l_@@_options_clist
\str_if_exist:NF \c_sys_backend_str
  { \sys_load_backend:n { } }
%</!2ekernel>
%    \end{macrocode}
%
% A test for pre-loading: does \tn{@pushfilenameaux} already exist.
% The alrady-loaded mechanism will handle everything now.
%    \begin{macrocode}
%<*package>
\cs_if_exist:NT \@pushfilenameaux
  { \endinput }
%</package>
%    \end{macrocode}
%
% \begin{macro}{\@pushfilename, \@popfilename}
% \begin{macro}{\@@_status_pop:w}
%   The idea here is to use \LaTeXe{}'s \tn{@pushfilename} and
%   \tn{@popfilename} to track the current syntax status. This can be
%   achieved by saving the current status flag at each push to a stack,
%   then recovering it at the pop stage and checking if the code
%   environment should still be active.
%    \begin{macrocode}
\tl_put_left:Nn \@pushfilename
  {
    \exp_args:Nx \__kernel_file_input_push:n
      {
        \tl_to_str:N \@currname
        \tl_to_str:N \@currext
      }
    \tl_put_left:Nx \l_@@_status_stack_tl
      {
        \bool_if:NTF \l__kernel_expl_bool
          { 1 }
          { 0 }
      }
    \ExplSyntaxOff
  }
\tl_put_right:Nn \@pushfilename { \@pushfilenameaux }
%    \end{macrocode}
%   This bit of trickery is needed to grab the name of the file being loaded
%   so we can record it.
%    \begin{macrocode}
\cs_set_protected:Npn \@pushfilenameaux #1#2#3
  {
    \str_gset:Nn \g_file_curr_name_str {#3}
    #1 #2 {#3}
  }
\tl_put_right:Nn \@popfilename
  {
    \__kernel_file_input_pop:
    \tl_if_empty:NTF \l_@@_status_stack_tl
      { \ExplSyntaxOff }
      { \exp_after:wN \@@_status_pop:w \l_@@_status_stack_tl \q_stop }
  }
%    \end{macrocode}
%   The pop auxiliary function removes the first item from the stack,
%   saves the rest of the stack and then does the test. The flag here
%   is not a proper \texttt{bool}, so a low-level test is used.
%    \begin{macrocode}
\cs_new_protected:Npn \@@_status_pop:w #1#2 \q_stop
  {
    \tl_set:Nn \l_@@_status_stack_tl {#2}
    \int_if_odd:nTF {#1}
      { \ExplSyntaxOn }
      { \ExplSyntaxOff }
  }
%    \end{macrocode}
% \end{macro}
% \end{macro}
%
% \begin{variable}{\l_@@_status_stack_tl}
%   As \pkg{expl3} itself cannot be loaded with the code environment
%   already active, at the end of the package \cs{ExplSyntaxOff} can
%   safely be called.
%    \begin{macrocode}
\tl_new:N \l_@@_status_stack_tl
\tl_set:Nn \l_@@_status_stack_tl { 0 }
%    \end{macrocode}
% \end{variable}
%
% For pre-loading, we have to manually disable the syntax.
%    \begin{macrocode}
%<*2ekernel>
\ExplSyntaxOff
%</2ekernel>
%    \end{macrocode}
%
%    \begin{macrocode}
%</package&loader|2ekernel>
%    \end{macrocode}
%
% \subsection{Generic loader}
%
%    \begin{macrocode}
%<*generic>
%    \end{macrocode}
%
% The generic loader starts with a test to ensure that the current format is
% not \LaTeXe{}!
%    \begin{macrocode}
\begingroup
  \def\tempa{LaTeX2e}%
  \def\next{}%
  \ifx\fmtname\tempa
    \def\next
      {%
        \PackageInfo{expl3}{Switching from generic to LaTeX2e loader}%
%    \end{macrocode}
% The \cs{relax} stops \cs{RequirePackage} from scanning for a date
% argument.  Putting \tn{endinput} \emph{after} loading the package is
% crucial, as otherwise \tn{endinput} would close the file
% \file{expl3.sty} at the end of its first line: indeed, as long as
% \file{expl3.sty} is open it is impossible to close the file
% \file{expl3-generic.tex}.
%    \begin{macrocode}
        \RequirePackage{expl3}\relax \endinput
      }%
  \fi
\expandafter\endgroup
\next
%    \end{macrocode}
%
% Reload check and identify the package:
% no \LaTeXe{} mechanism so this is all pretty basic.
%    \begin{macrocode}
\begingroup\expandafter\expandafter\expandafter\endgroup
\expandafter\ifx\csname ver@expl3-generic.tex\endcsname\relax
\else
  \immediate\write-1
    {%
      Package expl3 Info: The package is already loaded.%
    }%
  \expandafter\endinput
\fi
\immediate\write-1
  {%
    Package: expl3
    \ExplFileDate\space
    L3 programming layer (loader)%
  }%
\expandafter\edef\csname ver@expl3-generic.tex\endcsname
  {\ExplFileDate\space L3 programming layer}%
%    \end{macrocode}
%
% \begin{variable}[int]{\l@expl@tidy@tl}
%   Save the category code of |@| and then set it to \enquote{letter}.
%    \begin{macrocode}
\expandafter\edef\csname l@expl@tidy@tl\endcsname
  {%
    \catcode64=\the\catcode64\relax
    \let\expandafter\noexpand\csname l@expl@tidy@tl\endcsname
      \noexpand\undefined
  }%
\catcode64=11 %
%    \end{macrocode}
% \end{variable}
%
% \begin{macro}{\AtBeginDocument}
% \begin{macro}[int]{\expl@AtBeginDocument}
%   There are a few uses of \cs{AtBeginDocument} in the package code: the
%   easiest way around that is to simply do nothing for these. As
%   bundles such as \pkg{miniltx} may have defined \cs{AtBeginDocument}
%   any existing definition is saved for restoration after the  payload.
%    \begin{macrocode}
\let\expl@AtBeginDocument\AtBeginDocument
\def\AtBeginDocument#1{}%
\expandafter\def\expandafter\l@expl@tidy@tl\expandafter
  {%
    \l@expl@tidy@tl
    \let\AtBeginDocument\expl@AtBeginDocument
    \let\expl@AtBeginDocument\undefined
  }%
%    \end{macrocode}
% \end{macro}
% \end{macro}
%
%  Load the business end: this leaves \cs{expl3} syntax on.
%    \begin{macrocode}
\input expl3-code.tex %
%    \end{macrocode}
%
% A check that the bootstrap code did not abort loading: if it did,
% bail out silently here.
%    \begin{macrocode}
\begingroup\expandafter\expandafter\expandafter\endgroup
\expandafter\ifx\csname tex\string _let:D\endcsname\relax
  \expandafter\endinput
\fi
%    \end{macrocode}
%
% \begin{macro}{\__kernel_sys_configuration_load:n}
%   Very basic.
%    \begin{macrocode}
\cs_new_protected:Npn \__kernel_sys_configuration_load:n #1
  {
    \group_begin:
    \cs_set_protected:Npn \ProvidesExplFile
      {
        \char_set_catcode_space:n { `\  }
        \ProvidesExplFileAux
      }
    \cs_set_protected:Npn \ProvidesExplFileAux ##1##2##3##4
      {
        \group_end:
        \iow_log:x { File:~##1~##2~v##3~##4 }
      }
    \tex_input:D #1 .def \scan_stop:
  }
%    \end{macrocode}
% \end{macro}
%
% Load the standard back-end.
%    \begin{macrocode}
\sys_load_backend:n { }
%    \end{macrocode}
%
%  For the generic loader, a few final steps to take. Turn of \cs{expl3}
%  syntax and tidy up the small number of temporary changes.
%    \begin{macrocode}
\ExplSyntaxOff
\l@expl@tidy@tl
%    \end{macrocode}
%
%    \begin{macrocode}
%</generic>
%    \end{macrocode}
%
% \end{implementation}
%
% \PrintIndex
