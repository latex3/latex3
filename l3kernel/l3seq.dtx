% \iffalse meta-comment
%
%% File: l3seq.dtx Copyright (C) 1990-2013 The LaTeX3 Project
%%
%% It may be distributed and/or modified under the conditions of the
%% LaTeX Project Public License (LPPL), either version 1.3c of this
%% license or (at your option) any later version.  The latest version
%% of this license is in the file
%%
%%    http://www.latex-project.org/lppl.txt
%%
%% This file is part of the "l3kernel bundle" (The Work in LPPL)
%% and all files in that bundle must be distributed together.
%%
%% The released version of this bundle is available from CTAN.
%%
%% -----------------------------------------------------------------------
%%
%% The development version of the bundle can be found at
%%
%%    http://www.latex-project.org/svnroot/experimental/trunk/
%%
%% for those people who are interested.
%%
%%%%%%%%%%%
%% NOTE: %%
%%%%%%%%%%%
%%
%%   Snapshots taken from the repository represent work in progress and may
%%   not work or may contain conflicting material!  We therefore ask
%%   people _not_ to put them into distributions, archives, etc. without
%%   prior consultation with the LaTeX3 Project.
%%
%% -----------------------------------------------------------------------
%
%<*driver|package>
\RequirePackage{l3bootstrap}
\GetIdInfo$Id$
  {L3 Sequences and stacks}
%</driver|package>
%<*driver>
\documentclass[full]{l3doc}
\begin{document}
  \DocInput{\jobname.dtx}
\end{document}
%</driver>
% \fi
%
% \title{^^A
%   The \pkg{l3seq} package\\ Sequences and stacks^^A
%   \thanks{This file describes v\ExplFileVersion,
%      last revised \ExplFileDate.}^^A
% }
%
% \author{^^A
%  The \LaTeX3 Project\thanks
%    {^^A
%      E-mail:
%        \href{mailto:latex-team@latex-project.org}
%          {latex-team@latex-project.org}^^A
%    }^^A
% }
%
% \date{Released \ExplFileDate}
%
% \maketitle
%
% \begin{documentation}
%
% \LaTeX3 implements a \enquote{sequence} data type, which contain
% an ordered list of entries which may contain any \meta{balanced text}.
% It is possible to map functions to sequences such that the function
% is applied to every item in the sequence.
%
% Sequences are also used to implement stack functions in \LaTeX3. This
% is achieved using a number of dedicated stack functions.
%
% \section{Creating and initialising sequences}
%
% \begin{function}{\seq_new:N, \seq_new:c}
%   \begin{syntax}
%     \cs{seq_new:N} \meta{sequence}
%   \end{syntax}
%   Creates a new \meta{sequence} or raises an error if the name is
%   already taken. The declaration is global. The \meta{sequence} will
%   initially contain no items.
% \end{function}
%
% \begin{function}{\seq_clear:N, \seq_clear:c, \seq_gclear:N, \seq_gclear:c}
%   \begin{syntax}
%     \cs{seq_clear:N} \meta{sequence}
%   \end{syntax}
%   Clears all items from the \meta{sequence}.
% \end{function}
%
% \begin{function}
%   {\seq_clear_new:N, \seq_clear_new:c, \seq_gclear_new:N, \seq_gclear_new:c}
%   \begin{syntax}
%     \cs{seq_clear_new:N} \meta{sequence}
%   \end{syntax}
%   Ensures that the \meta{sequence} exists globally by applying
%   \cs{seq_new:N} if necessary, then applies \cs{seq_(g)clear:N} to leave
%   the \meta{sequence} empty.
% \end{function}
%
% \begin{function}
%   {
%     \seq_set_eq:NN,  \seq_set_eq:cN,  \seq_set_eq:Nc,  \seq_set_eq:cc,
%     \seq_gset_eq:NN, \seq_gset_eq:cN, \seq_gset_eq:Nc, \seq_gset_eq:cc
%   }
%   \begin{syntax}
%     \cs{seq_set_eq:NN} \meta{sequence_1} \meta{sequence_2}
%   \end{syntax}
%   Sets the content of \meta{sequence_1} equal to that of
%   \meta{sequence_2}.
% \end{function}
%
% \begin{function}[added = 2011-08-15, updated = 2012-07-02]
%   {
%     \seq_set_split:Nnn , \seq_set_split:NnV ,
%     \seq_gset_split:Nnn, \seq_gset_split:NnV
%   }
%   \begin{syntax}
%     \cs{seq_set_split:Nnn} \meta{sequence} \Arg{delimiter} \Arg{token list}
%   \end{syntax}
%   Splits the \meta{token list} into \meta{items} separated
%   by \meta{delimiter}, and assigns the result to the \meta{sequence}.
%   Spaces on both sides of each \meta{item} are ignored,
%   then one set of outer braces is removed (if any);
%   this space trimming behaviour is identical to that of
%   \pkg{l3clist} functions. Empty \meta{items} are preserved by
%   \cs{seq_set_split:Nnn}, and can be removed afterwards using
%   \cs{seq_remove_all:Nn} \meta{sequence} \Arg{}.
%   The \meta{delimiter} may not contain |{|, |}| or |#|
%   (assuming \TeX{}'s normal category code r\'egime).
%   If the \meta{delimiter} is empty, the \meta{token list} is split
%   into \meta{items} as a \meta{token list}.
% \end{function}
%
% \begin{function}
%   {\seq_concat:NNN, \seq_concat:ccc, \seq_gconcat:NNN, \seq_gconcat:ccc}
%   \begin{syntax}
%     \cs{seq_concat:NNN} \meta{sequence_1} \meta{sequence_2} \meta{sequence_3}
%   \end{syntax}
%   Concatenates the content of \meta{sequence_2} and \meta{sequence_3}
%   together and saves the result in \meta{sequence_1}. The items in
%   \meta{sequence_2} will be placed at the left side of the new sequence.
% \end{function}
%
% \begin{function}[EXP, pTF, added=2012-03-03]
%   {\seq_if_exist:N, \seq_if_exist:c}
%   \begin{syntax}
%     \cs{seq_if_exist_p:N} \meta{sequence}
%     \cs{seq_if_exist:NTF} \meta{sequence} \Arg{true code} \Arg{false code}
%   \end{syntax}
%   Tests whether the \meta{sequence} is currently defined.  This does not
%   check that the \meta{sequence} really is a sequence variable.
% \end{function}
%
% \section{Appending data to sequences}
%
% \begin{function}{
%   \seq_put_left:Nn, \seq_put_left:NV, \seq_put_left:Nv,
%   \seq_put_left:No, \seq_put_left:Nx,
%   \seq_put_left:cn, \seq_put_left:cV, \seq_put_left:cv,
%   \seq_put_left:co, \seq_put_left:cx,
%   \seq_gput_left:Nn, \seq_gput_left:NV, \seq_gput_left:Nv,
%   \seq_gput_left:No, \seq_gput_left:Nx,
%   \seq_gput_left:cn, \seq_gput_left:cV, \seq_gput_left:cv,
%   \seq_gput_left:co, \seq_gput_left:cx
% }
%   \begin{syntax}
%     \cs{seq_put_left:Nn} \meta{sequence} \Arg{item}
%   \end{syntax}
%   Appends the \meta{item} to the left of the \meta{sequence}.
% \end{function}
%
% \begin{function}{
%   \seq_put_right:Nn, \seq_put_right:NV, \seq_put_right:Nv,
%   \seq_put_right:No, \seq_put_right:Nx,
%   \seq_put_right:cn, \seq_put_right:cV, \seq_put_right:cv,
%   \seq_put_right:co, \seq_put_right:cx,
%   \seq_gput_right:Nn, \seq_gput_right:NV, \seq_gput_right:Nv,
%   \seq_gput_right:No, \seq_gput_right:Nx,
%   \seq_gput_right:cn, \seq_gput_right:cV, \seq_gput_right:cv,
%   \seq_gput_right:co, \seq_gput_right:cx
% }
%   \begin{syntax}
%     \cs{seq_put_right:Nn} \meta{sequence} \Arg{item}
%   \end{syntax}
%   Appends the \meta{item} to the right of the \meta{sequence}.
% \end{function}
%
% \section{Recovering items from sequences}
%
% Items can be recovered from either the left or the right of sequences.
% For implementation reasons, the actions at the left of the sequence are
% faster than those acting on the right. These functions all assign the
% recovered material locally, \emph{i.e.}~setting the
% \meta{token list variable} used with \cs{tl_set:Nn} and \emph{never}
% \cs{tl_gset:Nn}.
%
% \begin{function}[updated = 2012-05-14]{\seq_get_left:NN, \seq_get_left:cN}
%   \begin{syntax}
%     \cs{seq_get_left:NN} \meta{sequence} \meta{token list variable}
%   \end{syntax}
%   Stores the left-most item from a \meta{sequence} in the
%   \meta{token list variable} without removing it from the
%   \meta{sequence}. The \meta{token list variable} is assigned locally.
%   If \meta{sequence} is empty the \meta{token list variable} will
%   contain the special marker \cs{q_no_value}.
% \end{function}
%
% \begin{function}[updated = 2012-05-19]{\seq_get_right:NN, \seq_get_right:cN}
%   \begin{syntax}
%     \cs{seq_get_right:NN} \meta{sequence} \meta{token list variable}
%   \end{syntax}
%   Stores the right-most item from a \meta{sequence} in the
%   \meta{token list variable} without removing it from the
%   \meta{sequence}. The \meta{token list variable} is assigned locally.
%   If \meta{sequence} is empty the \meta{token list variable} will
%   contain the special marker \cs{q_no_value}.
% \end{function}
%
% \begin{function}[updated = 2012-05-14]{\seq_pop_left:NN, \seq_pop_left:cN}
%   \begin{syntax}
%     \cs{seq_pop_left:NN} \meta{sequence} \meta{token list variable}
%   \end{syntax}
%   Pops the left-most item from a \meta{sequence} into the
%   \meta{token list variable}, \emph{i.e.}~removes the item from the
%   sequence and stores it in the \meta{token list variable}.
%   Both of the variables are assigned locally. If \meta{sequence} is
%   empty the \meta{token list variable} will
%   contain the special marker \cs{q_no_value}.
% \end{function}
%
% \begin{function}[updated = 2012-05-14]{\seq_gpop_left:NN, \seq_gpop_left:cN}
%   \begin{syntax}
%     \cs{seq_gpop_left:NN} \meta{sequence} \meta{token list variable}
%   \end{syntax}
%   Pops the left-most item from a \meta{sequence} into the
%   \meta{token list variable}, \emph{i.e.}~removes the item from the
%   sequence and stores it in the \meta{token list variable}.
%   The \meta{sequence} is modified globally, while the assignment of
%   the \meta{token list variable} is local.
%   If \meta{sequence} is empty the \meta{token list variable} will
%   contain the special marker \cs{q_no_value}.
% \end{function}
%
% \begin{function}[updated = 2012-05-19]{\seq_pop_right:NN, \seq_pop_right:cN}
%   \begin{syntax}
%     \cs{seq_pop_right:NN} \meta{sequence} \meta{token list variable}
%   \end{syntax}
%   Pops the right-most item from a \meta{sequence} into the
%   \meta{token list variable}, \emph{i.e.}~removes the item from the
%   sequence and stores it in in the \meta{token list variable}.
%   Both of the variables are assigned locally. If \meta{sequence} is
%   empty the \meta{token list variable} will
%   contain the special marker \cs{q_no_value}.
% \end{function}
%
% \begin{function}[updated = 2012-05-19]{\seq_gpop_right:NN, \seq_gpop_right:cN}
%   \begin{syntax}
%     \cs{seq_gpop_right:NN} \meta{sequence} \meta{token list variable}
%   \end{syntax}
%   Pops the right-most item from a \meta{sequence} into the
%   \meta{token list variable}, \emph{i.e.}~removes the item from the
%   sequence and stores it in the \meta{token list variable}.
%   The \meta{sequence} is modified globally, while the assignment of
%   the \meta{token list variable} is local.
%   If \meta{sequence} is empty the \meta{token list variable} will
%   contain the special marker \cs{q_no_value}.
% \end{function}
%
% \section{Recovering values from sequences with branching}
%
% The functions in this section combine tests for non-empty sequences
% with recovery of an item from the sequence. They offer increased readability
% and performance over separate testing and recovery phases.
%
% \begin{function}[TF, added = 2012-05-14, updated = 2012-05-19]
%   {\seq_get_left:NN, \seq_get_left:cN}
%   \begin{syntax}
%     \cs{seq_get_left:NNTF} \meta{sequence} \meta{token list variable} \Arg{true code} \Arg{false code}
%   \end{syntax}
%   If the \meta{sequence} is empty, leaves the \meta{false code} in the
%   input stream.  The value of the \meta{token list variable} is
%   not defined in this case and should not be relied upon.  If the
%   \meta{sequence} is non-empty, stores the left-most item from a
%   \meta{sequence}
%   in the \meta{token list variable} without removing it from a
%   \meta{sequence}.
%   The \meta{token list variable} is assigned locally.
% \end{function}
%
% \begin{function}[TF, added = 2012-05-19]
%   {\seq_get_right:NN, \seq_get_right:cN}
%   \begin{syntax}
%     \cs{seq_get_right:NNTF} \meta{sequence} \meta{token list variable} \Arg{true code} \Arg{false code}
%   \end{syntax}
%   If the \meta{sequence} is empty, leaves the \meta{false code} in the
%   input stream.  The value of the \meta{token list variable} is
%   not defined in this case and should not be relied upon.  If the
%   \meta{sequence} is non-empty, stores the right-most item from a
%   \meta{sequence}
%   in the \meta{token list variable} without removing it from a
%   \meta{sequence}.
%   The \meta{token list variable} is assigned locally.
% \end{function}
%
% \begin{function}[TF, added = 2012-05-14, updated = 2012-05-19]
%   {\seq_pop_left:NN, \seq_pop_left:cN}
%   \begin{syntax}
%     \cs{seq_pop_left:NNTF} \meta{sequence} \meta{token list variable} \Arg{true code} \Arg{false code}
%   \end{syntax}
%   If the \meta{sequence} is empty, leaves the \meta{false code} in the
%   input stream.  The value of the \meta{token list variable} is
%   not defined in this case and should not be relied upon.  If the
%   \meta{sequence} is non-empty, pops the left-most item from a
%   \meta{sequence}
%   in the \meta{token list variable}, \emph{i.e.}~removes the item from a
%   \meta{sequence}.
%   Both the \meta{sequence} and the \meta{token list variable} are assigned
%   locally.
% \end{function}
%
% \begin{function}[TF, added = 2012-05-14, updated = 2012-05-19]
%   {\seq_gpop_left:NN, \seq_gpop_left:cN}
%   \begin{syntax}
%     \cs{seq_gpop_left:NNTF} \meta{sequence} \meta{token list variable} \Arg{true code} \Arg{false code}
%   \end{syntax}
%   If the \meta{sequence} is empty, leaves the \meta{false code} in the
%   input stream.  The value of the \meta{token list variable} is
%   not defined in this case and should not be relied upon.  If the
%   \meta{sequence} is non-empty, pops the left-most item from a \meta{sequence}
%   in the \meta{token list variable}, \emph{i.e.}~removes the item from a
%   \meta{sequence}.
%   The \meta{sequence} is modified globally, while the \meta{token list variable}
%   is assigned locally.
% \end{function}
%
% \begin{function}[TF, added = 2012-05-19]
%   {\seq_pop_right:NN, \seq_pop_right:cN}
%   \begin{syntax}
%     \cs{seq_pop_right:NNTF} \meta{sequence} \meta{token list variable} \Arg{true code} \Arg{false code}
%   \end{syntax}
%   If the \meta{sequence} is empty, leaves the \meta{false code} in the
%   input stream.  The value of the \meta{token list variable} is
%   not defined in this case and should not be relied upon.  If the
%   \meta{sequence} is non-empty, pops the right-most item from a \meta{sequence}
%   in the \meta{token list variable}, \emph{i.e.}~removes the item from a
%   \meta{sequence}.
%   Both the \meta{sequence} and the \meta{token list variable} are assigned
%   locally.
% \end{function}
%
% \begin{function}[TF, added = 2012-05-19]
%   {\seq_gpop_right:NN, \seq_gpop_right:cN}
%   \begin{syntax}
%     \cs{seq_gpop_right:NNTF} \meta{sequence} \meta{token list variable} \Arg{true code} \Arg{false code}
%   \end{syntax}
%   If the \meta{sequence} is empty, leaves the \meta{false code} in the
%   input stream.  The value of the \meta{token list variable} is
%   not defined in this case and should not be relied upon.  If the
%   \meta{sequence} is non-empty, pops the right-most item from a \meta{sequence}
%   in the \meta{token list variable}, \emph{i.e.}~removes the item from a
%   \meta{sequence}. The \meta{sequence} is modified globally, while the
%   \meta{token list variable} is assigned locally.
% \end{function}
%
% \section{Modifying sequences}
%
%  While sequences are normally used as ordered lists, it may be
%  necessary to modify the content. The functions here may be used
%  to update sequences, while retaining the order of the unaffected
%  entries.
%
% \begin{function}
%   {
%     \seq_remove_duplicates:N,  \seq_remove_duplicates:c,
%     \seq_gremove_duplicates:N, \seq_gremove_duplicates:c
%   }
%   \begin{syntax}
%     \cs{seq_remove_duplicates:N} \meta{sequence}
%   \end{syntax}
%   Removes duplicate items from the \meta{sequence}, leaving the
%   left most copy of each item in the \meta{sequence}.  The \meta{item}
%   comparison takes place on a token basis, as for \cs{tl_if_eq:nn(TF)}.
%   \begin{texnote}
%     This function iterates through every item in the \meta{sequence} and
%     does a comparison with the \meta{items} already checked. It is therefore
%     relatively slow with large sequences.
%   \end{texnote}
% \end{function}
%
% \begin{function}
%   {
%     \seq_remove_all:Nn , \seq_remove_all:cn,
%     \seq_gremove_all:Nn, \seq_gremove_all:cn
%   }
%   \begin{syntax}
%     \cs{seq_remove_all:Nn} \meta{sequence} \Arg{item}
%   \end{syntax}
%   Removes every occurrence of \meta{item} from the \meta{sequence}.
%   The \meta{item} comparison takes place on a token basis, as for
%   \cs{tl_if_eq:nn(TF)}.
% \end{function}
%
% \section{Sequence conditionals}
%
% \begin{function}[EXP,pTF]{\seq_if_empty:N, \seq_if_empty:c}
%   \begin{syntax}
%     \cs{seq_if_empty_p:N} \meta{sequence}
%     \cs{seq_if_empty:NTF} \meta{sequence} \Arg{true code} \Arg{false code}
%   \end{syntax}
%   Tests if the \meta{sequence} is empty (containing no items).
% \end{function}
%
% \begin{function}[TF]
%   {
%     \seq_if_in:Nn, \seq_if_in:NV, \seq_if_in:Nv, \seq_if_in:No, \seq_if_in:Nx,
%     \seq_if_in:cn, \seq_if_in:cV, \seq_if_in:cv, \seq_if_in:co, \seq_if_in:cx
%   }
%   \begin{syntax}
%     \cs{seq_if_in:NnTF} \meta{sequence} \Arg{item} \Arg{true code} \Arg{false code}
%   \end{syntax}
%   Tests if the \meta{item} is present in the \meta{sequence}.
% \end{function}
%
% \section{Mapping to sequences}
%
% \begin{function}[rEXP, updated = 2012-06-29]
%   {\seq_map_function:NN, \seq_map_function:cN}
%   \begin{syntax}
%     \cs{seq_map_function:NN} \meta{sequence} \meta{function}
%   \end{syntax}
%   Applies \meta{function} to every \meta{item} stored in the
%   \meta{sequence}. The \meta{function} will receive one argument for
%   each iteration. The \meta{items} are returned from left to right.
%   The function \cs{seq_map_inline:Nn} is in general more efficient
%   than \cs{seq_map_function:NN}.
%   One mapping may be nested inside another.
% \end{function}
%
% \begin{function}[updated = 2012-06-29]
%   {\seq_map_inline:Nn, \seq_map_inline:cn}
%   \begin{syntax}
%     \cs{seq_map_inline:Nn} \meta{sequence} \Arg{inline function}
%   \end{syntax}
%   Applies \meta{inline function} to every \meta{item} stored
%   within the \meta{sequence}. The \meta{inline function} should
%   consist of code which will receive the \meta{item} as |#1|.
%   One in line mapping can be nested inside another. The \meta{items}
%   are returned from left to right.
% \end{function}
%
% \begin{function}[updated = 2012-06-29]
%   {
%     \seq_map_variable:NNn, \seq_map_variable:Ncn,
%     \seq_map_variable:cNn, \seq_map_variable:ccn
%   }
%   \begin{syntax}
%     \cs{seq_map_variable:NNn} \meta{sequence} \meta{tl~var.} \Arg{function using tl~var.}
%   \end{syntax}
%   Stores each entry in the \meta{sequence} in turn in the
%   \meta{tl~var.}\ and applies the \meta{function using tl~var.}
%   The \meta{function} will usually consist of code making use of
%   the \meta{tl~var.}, but this is not enforced.  One variable
%   mapping can be nested inside another. The \meta{items}
%   are returned from left to right.
% \end{function}
%
% \begin{function}[rEXP, updated = 2012-06-29]{\seq_map_break:}
%   \begin{syntax}
%     \cs{seq_map_break:}
%   \end{syntax}
%   Used to terminate a \cs{seq_map_\ldots} function before all
%   entries in the \meta{sequence} have been processed. This will
%   normally take place within a conditional statement, for example
%   \begin{verbatim}
%     \seq_map_inline:Nn \l_my_seq
%       {
%         \str_if_eq:nnTF { #1 } { bingo }
%           { \seq_map_break: }
%           {
%             % Do something useful
%           }
%       }
%   \end{verbatim}
%   Use outside of a \cs{seq_map_\ldots} scenario will lead to low
%   level \TeX{} errors.
%   \begin{texnote}
%     When the mapping is broken, additional tokens may be inserted by the
%     internal macro \cs{__prg_break_point:Nn} before further items are taken
%     from the input stream. This will depend on the design of the mapping
%     function.
%   \end{texnote}
% \end{function}
%
% \begin{function}[rEXP, updated = 2012-06-29]{\seq_map_break:n}
%   \begin{syntax}
%     \cs{seq_map_break:n} \Arg{tokens}
%   \end{syntax}
%   Used to terminate a \cs{seq_map_\ldots} function before all
%   entries in the \meta{sequence} have been processed, inserting
%   the \meta{tokens} after the mapping has ended. This will
%   normally take place within a conditional statement, for example
%   \begin{verbatim}
%     \seq_map_inline:Nn \l_my_seq
%       {
%         \str_if_eq:nnTF { #1 } { bingo }
%           { \seq_map_break:n { <tokens> } }
%           {
%             % Do something useful
%           }
%       }
%   \end{verbatim}
%   Use outside of a \cs{seq_map_\ldots} scenario will lead to low
%   level \TeX{} errors.
%   \begin{texnote}
%     When the mapping is broken, additional tokens may be inserted by the
%     internal macro \cs{__prg_break_point:Nn} before the \meta{tokens} are
%     inserted into the input stream.
%     This will depend on the design of the mapping function.
%   \end{texnote}
% \end{function}
%
% \begin{function}[EXP, added = 2012-07-13]{\seq_count:N, \seq_count:c}
%   \begin{syntax}
%     \cs{seq_count:N} \meta{sequence}
%   \end{syntax}
%   Leaves the number of items in the \meta{sequence} in the input
%   stream as an \meta{integer denotation}. The total number of items
%   in a \meta{sequence} will include those which are empty and duplicates,
%   \emph{i.e.}~every item in a \meta{sequence} is unique.
% \end{function}
%
% \section{Sequences as stacks}
%
% Sequences can be used as stacks, where data is pushed to and popped
% from the top of the sequence. (The left of a sequence is the top, for
% performance reasons.) The stack functions for sequences are not
% intended to be mixed with the general ordered data functions detailed
% in the previous section: a sequence should either be used as an
% ordered data type or as a stack, but not in both ways.
%
% \begin{function}[updated = 2012-05-14]{\seq_get:NN, \seq_get:cN}
%   \begin{syntax}
%     \cs{seq_get:NN} \meta{sequence} \meta{token list variable}
%   \end{syntax}
%   Reads the top item from a \meta{sequence} into the
%   \meta{token list variable} without removing it from the
%   \meta{sequence}. The \meta{token list variable} is assigned locally.
%   If \meta{sequence} is empty the \meta{token list variable} will
%   contain the special marker \cs{q_no_value}.
% \end{function}
%
% \begin{function}[updated = 2012-05-14]{\seq_pop:NN, \seq_pop:cN}
%   \begin{syntax}
%     \cs{seq_pop:NN} \meta{sequence} \meta{token list variable}
%   \end{syntax}
%   Pops the top item from a \meta{sequence} into the
%   \meta{token list variable}. Both of the variables are assigned
%   locally. If \meta{sequence} is empty the \meta{token list variable} will
%   contain the special marker \cs{q_no_value}.
% \end{function}
%
% \begin{function}[updated = 2012-05-14]{\seq_gpop:NN, \seq_gpop:cN}
%   \begin{syntax}
%     \cs{seq_gpop:NN} \meta{sequence} \meta{token list variable}
%   \end{syntax}
%   Pops the top item from a \meta{sequence} into the
%   \meta{token list variable}. The \meta{sequence} is modified globally,
%   while the \meta{token list variable} is assigned locally. If
%   \meta{sequence} is empty the \meta{token list variable} will
%   contain the special marker \cs{q_no_value}.
% \end{function}
%
% \begin{function}[TF, added = 2012-05-14, updated = 2012-05-19]{\seq_get:NN, \seq_get:cN}
%   \begin{syntax}
%     \cs{seq_get:NNTF} \meta{sequence} \meta{token list variable} \Arg{true code} \Arg{false code}
%   \end{syntax}
%   If the \meta{sequence} is empty, leaves the \meta{false code} in the
%   input stream.  The value of the \meta{token list variable} is
%   not defined in this case and should not be relied upon.  If the
%   \meta{sequence} is non-empty, stores the top item from a
%   \meta{sequence} in the \meta{token list variable} without removing it from
%   the \meta{sequence}. The \meta{token list variable} is assigned locally.
% \end{function}
%
% \begin{function}[TF, added = 2012-05-14, updated = 2012-05-19]{\seq_pop:NN, \seq_pop:cN}
%   \begin{syntax}
%     \cs{seq_pop:NNTF} \meta{sequence} \meta{token list variable} \Arg{true code} \Arg{false code}
%   \end{syntax}
%   If the \meta{sequence} is empty, leaves the \meta{false code} in the
%   input stream.  The value of the \meta{token list variable} is
%   not defined in this case and should not be relied upon.  If the
%   \meta{sequence} is non-empty, pops the top item from the
%   \meta{sequence} in the \meta{token list variable}, \emph{i.e.}~removes the
%   item from the \meta{sequence}. Both the \meta{sequence} and the
%   \meta{token list variable} are assigned locally.
% \end{function}
%
% \begin{function}[TF, added = 2012-05-14, updated = 2012-05-19]{\seq_gpop:NN, \seq_gpop:cN}
%   \begin{syntax}
%     \cs{seq_gpop:NNTF} \meta{sequence} \meta{token list variable} \Arg{true code} \Arg{false code}
%   \end{syntax}
%   If the \meta{sequence} is empty, leaves the \meta{false code} in the
%   input stream.  The value of the \meta{token list variable} is
%   not defined in this case and should not be relied upon.  If the
%   \meta{sequence} is non-empty, pops the top item from the \meta{sequence}
%   in the \meta{token list variable}, \emph{i.e.}~removes the item from the
%   \meta{sequence}. The \meta{sequence} is modified globally, while the
%   \meta{token list variable} is assigned locally.
% \end{function}
%
% \begin{function}
%   {
%     \seq_push:Nn, \seq_push:NV, \seq_push:Nv, \seq_push:No, \seq_push:Nx,
%     \seq_push:cn, \seq_push:cV, \seq_push:cv, \seq_push:co, \seq_push:cx,
%     \seq_gpush:Nn, \seq_gpush:NV, \seq_gpush:Nv,
%     \seq_gpush:No, \seq_gpush:Nx,
%     \seq_gpush:cn, \seq_gpush:cV, \seq_gpush:cv,
%     \seq_gpush:co, \seq_gpush:cx
%   }
%   \begin{syntax}
%     \cs{seq_push:Nn} \meta{sequence} \Arg{item}
%   \end{syntax}
%   Adds the \Arg{item} to the top of the \meta{sequence}.
% \end{function}
%
% \section{Constant and scratch sequences}
%
% \begin{variable}[added = 2012-07-02]{\c_empty_seq}
%   Constant that is always empty.
% \end{variable}
%
% \begin{variable}[added = 2012-04-26]{\l_tmpa_seq, \l_tmpb_seq}
%   Scratch sequences for local assignment. These are never used by
%   the kernel code, and so are safe for use with any \LaTeX3-defined
%   function. However, they may be overwritten by other non-kernel
%   code and so should only be used for short-term storage.
% \end{variable}
%
% \begin{variable}[added = 2012-04-26]{\g_tmpa_seq, \g_tmpb_seq}
%   Scratch sequences for global assignment. These are never used by
%   the kernel code, and so are safe for use with any \LaTeX3-defined
%   function. However, they may be overwritten by other non-kernel
%   code and so should only be used for short-term storage.
% \end{variable}
%
% \section{Viewing sequences}
%
% \begin{function}[updated = 2012-09-09]{\seq_show:N, \seq_show:c}
%   \begin{syntax}
%     \cs{seq_show:N} \meta{sequence}
%   \end{syntax}
%   Displays the entries in the \meta{sequence} in the terminal.
% \end{function}
%
% \section{Internal sequence functions}
%
% \begin{function}[EXP]{\__seq_item:n}
%   \begin{syntax}
%     \cs{__seq_item:n} \meta{item}
%   \end{syntax}
%   The internal token used to begin each sequence entry. If expanded
%   outside of a mapping or manipulation function, an error will be
%   raised. The definition should always be set globally.
% \end{function}
%
% \begin{function}{\__seq_push_item_def:n, \__seq_push_item_def:x}
%   \begin{syntax}
%     \cs{__seq_push_item_def:n} \Arg{code}
%   \end{syntax}
%   Saves the definition of \cs{__seq_item:n} and redefines it to
%   accept one parameter and expand to \meta{code}. This function
%   should always be balanced by use of \cs{__seq_pop_item_def:}.
% \end{function}
%
% \begin{function}{\__seq_pop_item_def:}
%   \begin{syntax}
%     \cs{__seq_pop_item_def:}
%   \end{syntax}
%   Restores the definition of \cs{__seq_item:n} most recently saved by
%   \cs{__seq_push_item_def:n}. This function should always be used in
%   a balanced pair with \cs{__seq_push_item_def:n}.
% \end{function}
%
% \end{documentation}
%
% \begin{implementation}
%
% \section{\pkg{l3seq} implementation}
%
% \TestFiles{m3seq002,m3seq003}
%
%    \begin{macrocode}
%<*initex|package>
%    \end{macrocode}
%
%    \begin{macrocode}
%<@@=seq>
%    \end{macrocode}
%
%    \begin{macrocode}
%<*package>
\ProvidesExplPackage
  {\ExplFileName}{\ExplFileDate}{\ExplFileVersion}{\ExplFileDescription}
\__expl_package_check:
%</package>
%    \end{macrocode}
%
% A sequence is a control sequence whose top-level expansion is of
% the form \enquote{\cs{@@_item:n} \marg{item_1}
% \ldots \cs{@@_item:n} \marg{item_n}}. An earlier implementation
% used the structure \enquote{\cs{seq_elt:w} \meta{item_1}
% \cs{seq_elt_end:} \ldots \cs{seq_elt:w} \meta{item_n}
% \cs{seq_elt_end:}}. This allows rapid searching using a delimited
% function, but is not suitable for items containing |{|, |}| and |#|
% tokens, and also leads to the loss of surrounding braces
% around items.
%
% \begin{macro}[int]{\@@_item:n}
%   The delimiter is always defined, but when used incorrectly simply
%   removes its argument and hits an undefined control sequence to
%   raise an error.
%    \begin{macrocode}
\cs_new:Npn \@@_item:n
  {
    \__msg_kernel_expandable_error:nn { kernel } { misused-sequence }
    \use_none:n
  }
%    \end{macrocode}
% \end{macro}
%
% \begin{variable}{\l_@@_internal_a_tl, \l_@@_internal_b_tl}
%   Scratch space for various internal uses.
%    \begin{macrocode}
\tl_new:N \l_@@_internal_a_tl
\tl_new:N \l_@@_internal_b_tl
%    \end{macrocode}
% \end{variable}
%
% \begin{variable}{\c_empty_seq}
%   Simply copy the empty token list.
%    \begin{macrocode}
\cs_new_eq:NN \c_empty_seq \c_empty_tl
%    \end{macrocode}
% \end{variable}
%
% \subsection{Allocation and initialisation}
%
% \begin{macro}{\seq_new:N,\seq_new:c}
% \UnitTested
%   Internally, sequences are just token lists.
%    \begin{macrocode}
\cs_new_eq:NN \seq_new:N \tl_new:N
\cs_new_eq:NN \seq_new:c \tl_new:c
%    \end{macrocode}
% \end{macro}
%
% \begin{macro}{\seq_clear:N, \seq_clear:c}
% \UnitTested
% \begin{macro}{\seq_gclear:N, \seq_gclear:c}
% \UnitTested
%   Clearing sequences is just the same as clearing token lists.
%    \begin{macrocode}
\cs_new_eq:NN \seq_clear:N  \tl_clear:N
\cs_new_eq:NN \seq_clear:c  \tl_clear:c
\cs_new_eq:NN \seq_gclear:N \tl_gclear:N
\cs_new_eq:NN \seq_gclear:c \tl_gclear:c
%    \end{macrocode}
% \end{macro}
% \end{macro}
%
% \begin{macro}{\seq_clear_new:N, \seq_clear_new:c}
% \UnitTested
% \begin{macro}{\seq_gclear_new:N, \seq_gclear_new:c}
% \UnitTested
%   Once again a copy from the token list functions.
%    \begin{macrocode}
\cs_new_eq:NN \seq_clear_new:N  \tl_clear_new:N
\cs_new_eq:NN \seq_clear_new:c  \tl_clear_new:c
\cs_new_eq:NN \seq_gclear_new:N \tl_gclear_new:N
\cs_new_eq:NN \seq_gclear_new:c \tl_gclear_new:c
%    \end{macrocode}
% \end{macro}
% \end{macro}
%
% \begin{macro}{\seq_set_eq:NN, \seq_set_eq:cN, \seq_set_eq:Nc, \seq_set_eq:cc}
% \UnitTested
% \begin{macro}
%   {\seq_gset_eq:NN, \seq_gset_eq:cN, \seq_gset_eq:Nc, \seq_gset_eq:cc}
% \UnitTested
%   Once again, these are simple copies from the token list functions.
%    \begin{macrocode}
\cs_new_eq:NN \seq_set_eq:NN  \tl_set_eq:NN
\cs_new_eq:NN \seq_set_eq:Nc  \tl_set_eq:Nc
\cs_new_eq:NN \seq_set_eq:cN  \tl_set_eq:cN
\cs_new_eq:NN \seq_set_eq:cc  \tl_set_eq:cc
\cs_new_eq:NN \seq_gset_eq:NN \tl_gset_eq:NN
\cs_new_eq:NN \seq_gset_eq:Nc \tl_gset_eq:Nc
\cs_new_eq:NN \seq_gset_eq:cN \tl_gset_eq:cN
\cs_new_eq:NN \seq_gset_eq:cc \tl_gset_eq:cc
%    \end{macrocode}
% \end{macro}
% \end{macro}
%
% \begin{macro}
%   {
%     \seq_set_split:Nnn , \seq_set_split:NnV ,
%     \seq_gset_split:Nnn, \seq_gset_split:NnV
%   }
% \begin{macro}[aux]{\@@_set_split:NNnn}
% \begin{macro}[aux]
%   {
%     \@@_set_split_auxi:w, \@@_set_split_auxii:w,
%     \@@_set_split_end:
%   }
%   The goal is to split a given token list at a marker,
%   strip spaces from each item, and remove one set of
%   outer braces if after removing leading and trailing
%   spaces the item is enclosed within braces. After
%   \cs{tl_replace_all:Nnn}, the token list \cs{l_@@_internal_a_tl}
%   is a repetition of the pattern
%   \cs{@@_set_split_auxi:w} \cs{prg_do_nothing:}
%   \meta{item with spaces} \cs{@@_set_split_end:}.
%   Then, \texttt{x}-expansion causes \cs{@@_set_split_auxi:w}
%   to trim spaces, and leaves its result as
%   \cs{@@_set_split_auxii:w} \meta{trimmed item}
%   \cs{@@_set_split_end:}. This is then converted
%   to the \pkg{l3seq} internal structure by another
%   \texttt{x}-expansion. In the first step, we insert
%   \cs{prg_do_nothing:} to avoid losing braces too early:
%   that would cause space trimming to act within those
%   lost braces. The second step is solely there to strip
%   braces which are outermost after space trimming.
%    \begin{macrocode}
\cs_new_protected_nopar:Npn \seq_set_split:Nnn
  { \@@_set_split:NNnn \tl_set:Nx }
\cs_new_protected_nopar:Npn \seq_gset_split:Nnn
  { \@@_set_split:NNnn \tl_gset:Nx }
\cs_new_protected:Npn \@@_set_split:NNnn #1 #2 #3 #4
  {
    \tl_if_empty:nTF {#3}
      { #1 #2 { \tl_map_function:nN {#4} \@@_wrap_item:n } }
      {
        \tl_set:Nn \l_@@_internal_a_tl
          {
            \@@_set_split_auxi:w \prg_do_nothing:
            #4
            \@@_set_split_end:
          }
        \tl_replace_all:Nnn \l_@@_internal_a_tl { #3 }
          {
            \@@_set_split_end:
            \@@_set_split_auxi:w \prg_do_nothing:
          }
        \tl_set:Nx \l_@@_internal_a_tl { \l_@@_internal_a_tl }
        #1 #2 { \l_@@_internal_a_tl }
      }
  }
\cs_new:Npn \@@_set_split_auxi:w #1 \@@_set_split_end:
  {
    \exp_not:N \@@_set_split_auxii:w
    \exp_args:No \tl_trim_spaces:n {#1}
    \exp_not:N \@@_set_split_end:
  }
\cs_new:Npn \@@_set_split_auxii:w #1 \@@_set_split_end:
  { \@@_wrap_item:n {#1} }
\cs_generate_variant:Nn \seq_set_split:Nnn  { NnV }
\cs_generate_variant:Nn \seq_gset_split:Nnn { NnV }
%    \end{macrocode}
% \end{macro}
% \end{macro}
% \end{macro}
%
% \begin{macro}{\seq_concat:NNN, \seq_concat:ccc}
% \UnitTested
% \begin{macro}{\seq_gconcat:NNN, \seq_gconcat:ccc}
% \UnitTested
%   Concatenating sequences is easy.
%    \begin{macrocode}
\cs_new_eq:NN \seq_concat:NNN  \tl_concat:NNN
\cs_new_eq:NN \seq_gconcat:NNN \tl_gconcat:NNN
\cs_new_eq:NN \seq_concat:ccc  \tl_concat:ccc
\cs_new_eq:NN \seq_gconcat:ccc \tl_gconcat:ccc
%    \end{macrocode}
% \end{macro}
% \end{macro}
%
% \begin{macro}[pTF]{\seq_if_exist:N, \seq_if_exist:c}
%   Copies of the \texttt{cs} functions defined in \pkg{l3basics}.
%    \begin{macrocode}
\prg_new_eq_conditional:NNn \seq_if_exist:N \cs_if_exist:N { TF , T , F , p }
\prg_new_eq_conditional:NNn \seq_if_exist:c \cs_if_exist:c { TF , T , F , p }
%    \end{macrocode}
% \end{macro}
%
% \subsection{Appending data to either end}
%
% \begin{macro}{
%   \seq_put_left:Nn, \seq_put_left:NV, \seq_put_left:Nv,
%   \seq_put_left:No, \seq_put_left:Nx,
%   \seq_put_left:cn, \seq_put_left:cV, \seq_put_left:cv,
%   \seq_put_left:co, \seq_put_left:cx
% }
% \UnitTested
% \begin{macro}{
%   \seq_put_right:Nn, \seq_put_right:NV, \seq_put_right:Nv,
%   \seq_put_right:No, \seq_put_right:Nx,
%   \seq_put_right:cn, \seq_put_right:cV, \seq_put_right:cv,
%   \seq_put_right:co, \seq_put_right:cx
% }
% \UnitTested
%   The code here is just a wrapper for adding to token lists.
%    \begin{macrocode}
\cs_new_protected:Npn \seq_put_left:Nn #1#2
  { \tl_put_left:Nn #1 { \@@_item:n {#2} } }
\cs_new_protected:Npn \seq_put_right:Nn #1#2
  { \tl_put_right:Nn #1 { \@@_item:n {#2} } }
\cs_generate_variant:Nn \seq_put_left:Nn  {     NV , Nv , No , Nx }
\cs_generate_variant:Nn \seq_put_left:Nn  { c , cV , cv , co , cx }
\cs_generate_variant:Nn \seq_put_right:Nn {     NV , Nv , No , Nx }
\cs_generate_variant:Nn \seq_put_right:Nn { c , cV , cv , co , cx }
%    \end{macrocode}
% \end{macro}
% \end{macro}
%
% \begin{macro}{
%   \seq_gput_left:Nn, \seq_gput_left:NV, \seq_gput_left:Nv,
%   \seq_gput_left:No, \seq_gput_left:Nx,
%   \seq_gput_left:cn, \seq_gput_left:cV, \seq_gput_left:cv,
%   \seq_gput_left:co, \seq_gput_left:cx
% }
% \begin{macro}{
%   \seq_gput_right:Nn, \seq_gput_right:NV, \seq_gput_right:Nv,
%   \seq_gput_right:No, \seq_gput_right:Nx,
%   \seq_gput_right:cn, \seq_gput_right:cV,\seq_gput_right:cv,
%   \seq_gput_right:co, \seq_gput_right:cx
% }
%   The same for global addition.
%    \begin{macrocode}
\cs_new_protected:Npn \seq_gput_left:Nn #1#2
  { \tl_gput_left:Nn #1 { \@@_item:n {#2} } }
\cs_new_protected:Npn \seq_gput_right:Nn #1#2
  { \tl_gput_right:Nn #1 { \@@_item:n {#2} } }
\cs_generate_variant:Nn \seq_gput_left:Nn  {     NV , Nv , No , Nx }
\cs_generate_variant:Nn \seq_gput_left:Nn  { c , cV , cv , co , cx }
\cs_generate_variant:Nn \seq_gput_right:Nn {     NV , Nv , No , Nx }
\cs_generate_variant:Nn \seq_gput_right:Nn { c , cV , cv , co , cx }
%    \end{macrocode}
% \end{macro}
% \end{macro}
%
% \subsection{Modifying sequences}
%
% \begin{macro}[aux]{\@@_wrap_item:n}
%   This function converts its argument to a proper sequence item
%   in an \texttt{x}-expansion context.
%    \begin{macrocode}
\cs_new:Npn \@@_wrap_item:n #1 { \exp_not:n { \@@_item:n {#1} } }
%    \end{macrocode}
% \end{macro}
%
% \begin{variable}{\l_@@_remove_seq}
%   An internal sequence for the removal routines.
%    \begin{macrocode}
\seq_new:N \l_@@_remove_seq
%    \end{macrocode}
% \end{variable}
%
% \begin{macro}{\seq_remove_duplicates:N, \seq_remove_duplicates:c}
% \UnitTested
% \begin{macro}{\seq_gremove_duplicates:N, \seq_gremove_duplicates:c}
% \UnitTested
% \begin{macro}[aux]{\@@_remove_duplicates:NN}
%   Removing duplicates means making a new list then copying it.
%    \begin{macrocode}
\cs_new_protected:Npn \seq_remove_duplicates:N
  { \@@_remove_duplicates:NN \seq_set_eq:NN }
\cs_new_protected:Npn \seq_gremove_duplicates:N
  { \@@_remove_duplicates:NN \seq_gset_eq:NN }
\cs_new_protected:Npn \@@_remove_duplicates:NN #1#2
  {
    \seq_clear:N \l_@@_remove_seq
    \seq_map_inline:Nn #2
      {
        \seq_if_in:NnF \l_@@_remove_seq {##1}
          { \seq_put_right:Nn \l_@@_remove_seq {##1} }
      }
    #1 #2 \l_@@_remove_seq
  }
\cs_generate_variant:Nn \seq_remove_duplicates:N  { c }
\cs_generate_variant:Nn \seq_gremove_duplicates:N { c }
%    \end{macrocode}
% \end{macro}
% \end{macro}
% \end{macro}
%
% \begin{macro}{\seq_remove_all:Nn, \seq_remove_all:cn}
% \UnitTested
% \begin{macro}{\seq_gremove_all:Nn, \seq_gremove_all:cn}
% \UnitTested
% \begin{macro}[aux]{\@@_remove_all_aux:NNn}
%   The idea of the code here is to avoid a relatively expensive addition of
%   items one at a time to an intermediate sequence.
%   The approach taken is therefore similar to
%   that in \cs{@@_pop_right_aux:NNN}, using a \enquote{flexible}
%   \texttt{x}-type expansion to do most of the work. As \cs{tl_if_eq:nnT}
%   is not expandable, a two-part strategy is needed. First, the
%   \texttt{x}-type expansion uses \cs{str_if_eq:nnT} to find potential
%   matches. If one is found, the expansion is halted and the necessary
%   set up takes place to use the \cs{tl_if_eq:NNT} test. The \texttt{x}-type
%   is started again, including all of the items copied already. This will
%   happen repeatedly until the entire sequence has been scanned. The code
%   is set up to avoid needing and intermediate scratch list: the lead-off
%   \texttt{x}-type expansion (|#1 #2 {#2}|) will ensure that nothing is
%   lost.
%    \begin{macrocode}
\cs_new_protected:Npn \seq_remove_all:Nn
  { \@@_remove_all_aux:NNn \tl_set:Nx }
\cs_new_protected:Npn \seq_gremove_all:Nn
  { \@@_remove_all_aux:NNn \tl_gset:Nx }
\cs_new_protected:Npn \@@_remove_all_aux:NNn #1#2#3
  {
    \@@_push_item_def:n
      {
        \str_if_eq:nnT {##1} {#3}
          {
            \if_false: { \fi: }
            \tl_set:Nn \l_@@_internal_b_tl {##1}
            #1 #2
               { \if_false: } \fi:
                 \exp_not:o {#2}
                 \tl_if_eq:NNT \l_@@_internal_a_tl \l_@@_internal_b_tl
                   { \use_none:nn }
          }
        \@@_wrap_item:n {##1}
      }
    \tl_set:Nn \l_@@_internal_a_tl {#3}
    #1 #2 {#2}
    \@@_pop_item_def:
  }
\cs_generate_variant:Nn \seq_remove_all:Nn  { c }
\cs_generate_variant:Nn \seq_gremove_all:Nn { c }
%    \end{macrocode}
% \end{macro}
% \end{macro}
% \end{macro}
%
% \subsection{Sequence conditionals}
%
% \begin{macro}[pTF]{\seq_if_empty:N, \seq_if_empty:c}
% \UnitTested
%   Simple copies from the token list variable material.
%    \begin{macrocode}
\prg_new_eq_conditional:NNn \seq_if_empty:N \tl_if_empty:N
  { p , T , F , TF }
\prg_new_eq_conditional:NNn \seq_if_empty:c \tl_if_empty:c
  { p , T , F , TF }
%    \end{macrocode}
% \end{macro}
%
% \begin{macro}[TF]
%   {
%     \seq_if_in:Nn, \seq_if_in:NV, \seq_if_in:Nv, \seq_if_in:No, \seq_if_in:Nx,
%     \seq_if_in:cn, \seq_if_in:cV, \seq_if_in:cv, \seq_if_in:co, \seq_if_in:cx
%   }
% \UnitTested
% \begin{macro}[aux]{\@@_if_in:}
%   The approach here is to define \cs{@@_item:n} to compare its
%   argument with the test sequence. If the two items are equal, the
%   mapping is terminated and \cs{group_end:} \cs{prg_return_true:}
%   is inserted after skipping over the rest of the recursion. On the
%   other hand, if there is no match then the loop will break returning
%   \cs{prg_return_false:}.
%   Everything is inside a group so that \cs{@@_item:n} is preserved
%   in nested situations.
%    \begin{macrocode}
\prg_new_protected_conditional:Npnn \seq_if_in:Nn #1#2
  { T , F , TF }
  {
    \group_begin:
      \tl_set:Nn \l_@@_internal_a_tl {#2}
      \cs_set_protected:Npn \@@_item:n ##1
        {
          \tl_set:Nn \l_@@_internal_b_tl {##1}
          \if_meaning:w \l_@@_internal_a_tl \l_@@_internal_b_tl
            \exp_after:wN \@@_if_in:
          \fi:
        }
      #1
    \group_end:
    \prg_return_false:
    \__prg_break_point:
  }
\cs_new_nopar:Npn \@@_if_in:
  { \__prg_break:n { \group_end: \prg_return_true: } }
\cs_generate_variant:Nn \seq_if_in:NnT  {     NV , Nv , No , Nx }
\cs_generate_variant:Nn \seq_if_in:NnT  { c , cV , cv , co , cx }
\cs_generate_variant:Nn \seq_if_in:NnF  {     NV , Nv , No , Nx }
\cs_generate_variant:Nn \seq_if_in:NnF  { c , cV , cv , co , cx }
\cs_generate_variant:Nn \seq_if_in:NnTF {     NV , Nv , No , Nx }
\cs_generate_variant:Nn \seq_if_in:NnTF { c , cV , cv , co , cx }
%    \end{macrocode}
% \end{macro}
% \end{macro}
%
% \subsection{Recovering data from sequences}
%
% \begin{macro}[int]{\@@_pop:NNNN, \@@_pop_TF:NNNN}
%   The two \texttt{pop} functions share their emptiness tests.  We also
%   use a common emptiness test for all branching \texttt{get} and
%   \texttt{pop} functions.
%    \begin{macrocode}
\cs_new_protected:Npn \@@_pop:NNNN #1#2#3#4
  {
    \if_meaning:w #3 \c_empty_seq
      \tl_set:Nn #4 { \q_no_value }
    \else:
      #1#2#3#4
    \fi:
  }
\cs_new_protected:Npn \@@_pop_TF:NNNN #1#2#3#4
  {
    \if_meaning:w #3 \c_empty_seq
      % \tl_set:Nn #4 { \q_no_value }
      \prg_return_false:
    \else:
      #1#2#3#4
      \prg_return_true:
    \fi:
  }
%    \end{macrocode}
% \end{macro}
%
% \begin{macro}{\seq_get_left:NN, \seq_get_left:cN}
% \UnitTested
% \begin{macro}[aux]{\@@_get_left:NnwN}
%   Getting an item from the left of a sequence is pretty easy: just
%   trim off the first item after removing the \cs{@@_item:n} at
%   the start.  We first append a \cs{q_no_value} item to cover the case
%   of an empty sequence
%    \begin{macrocode}
\cs_new_protected:Npn \seq_get_left:NN #1#2
  {
    \tl_set:Nx #2
      {
        \exp_after:wN \@@_get_left:Nnw
        #1 \@@_item:n { \q_no_value } \q_stop
      }
  }
\cs_new:Npn \@@_get_left:Nnw \@@_item:n #1#2 \q_stop
  { \exp_not:n {#1} }
\cs_generate_variant:Nn \seq_get_left:NN { c }
%    \end{macrocode}
% \end{macro}
% \end{macro}
%
% \begin{macro}{\seq_pop_left:NN, \seq_pop_left:cN}
% \UnitTested
% \begin{macro}{\seq_gpop_left:NN, \seq_gpop_left:cN}
% \UnitTested
% \begin{macro}[aux]{\@@_pop_left:NNN, \@@_pop_left:NnwNNN}
%   The approach to popping an item is pretty similar to that to get
%   an item, with the only difference being that the sequence itself has
%   to be redefined. This makes it more sensible to use an auxiliary
%   function for the local and global cases.
%    \begin{macrocode}
\cs_new_protected_nopar:Npn \seq_pop_left:NN
  { \@@_pop:NNNN \@@_pop_left:NNN \tl_set:Nn }
\cs_new_protected_nopar:Npn \seq_gpop_left:NN
  { \@@_pop:NNNN \@@_pop_left:NNN \tl_gset:Nn }
\cs_new_protected:Npn \@@_pop_left:NNN #1#2#3
  { \exp_after:wN \@@_pop_left:NnwNNN #2 \q_stop #1#2#3 }
\cs_new_protected:Npn \@@_pop_left:NnwNNN \@@_item:n #1#2 \q_stop #3#4#5
  {
    #3 #4 {#2}
    \tl_set:Nn #5 {#1}
  }
\cs_generate_variant:Nn \seq_pop_left:NN  { c }
\cs_generate_variant:Nn \seq_gpop_left:NN { c }
%    \end{macrocode}
% \end{macro}
% \end{macro}
% \end{macro}
%
% \begin{macro}{\seq_get_right:NN, \seq_get_right:cN}
% \UnitTested
% \begin{macro}[aux]{\@@_get_right_loop:nn}
%   First prepend \cs{q_no_value}, then take two arguments at a time.
%   Apart from the right-hand end of the sequence, this be a brace group
%   followed by \cs{@@_item:n}.  The \cs{use_none:nn} removes both of
%   those.  At the end of the sequence, the two question marks are taken
%   by \cs{use_none:nn}, and the assignment is placed before the
%   right-most item.  The \tn{afterassignment} primitive places
%   \cs{use_none:n} to get rid of a trailing \cs{@@_get_right_loop:nn}.
%    \begin{macrocode}
\cs_new_protected:Npn \seq_get_right:NN #1#2
  {
    \exp_after:wN \@@_get_right_loop:nn
    \exp_after:wN \q_no_value
    #1
    {
      ??
      \tex_afterassignment:D \use_none:n
      \tl_set:Nn #2
    }
  }
\cs_new_protected:Npn \@@_get_right_loop:nn #1#2
  {
    \use_none:nn #2 {#1}
    \@@_get_right_loop:nn
  }
\cs_generate_variant:Nn \seq_get_right:NN { c }
%    \end{macrocode}
% \end{macro}
% \end{macro}
%
% \begin{macro}{\seq_pop_right:NN, \seq_pop_right:cN}
% \UnitTested
% \begin{macro}{\seq_gpop_right:NN, \seq_gpop_right:cN}
% \UnitTested
% \begin{macro}[aux]{\@@_pop_right_aux:NNN, \@@_pop_right_loop:nn}
%   The approach to popping from the right is a bit more involved, but does
%   use some of the same ideas as getting from the right. What is needed is a
%   \enquote{flexible length} way to set a token list variable. This is
%   supplied by the |{ \if_false: } \fi:| \ldots
%   |\if_false: { \fi: }| construct. Using an \texttt{x}-type
%   expansion and a \enquote{non-expanding} definition for \cs{@@_item:n},
%   the left-most $n - 1$ entries in a sequence of $n$ items will be stored
%   back in the sequence. That needs a loop of unknown length, hence using the
%   strange \cs{if_false:} way of including brackets. When the last item
%   of the sequence is reached, the closing bracket for the assignment is
%   inserted, and |\tl_set:Nn #3| is inserted in front of the final
%   entry.  This therefore does the pop assignment.  The trailing
%   looping macro is removed by placing a \cs{use_none:n} using the
%   \tn{afterassignment} primitive.
%    \begin{macrocode}
\cs_new_protected_nopar:Npn \seq_pop_right:NN
  { \@@_pop:NNNN \@@_pop_right_aux:NNN \tl_set:Nx }
\cs_new_protected_nopar:Npn \seq_gpop_right:NN
  { \@@_pop:NNNN \@@_pop_right_aux:NNN \tl_gset:Nx }
\cs_new_protected:Npn \@@_pop_right_aux:NNN #1#2#3
  {
    \cs_set_eq:NN \seq_tmp:w \@@_item:n
    \cs_set_eq:NN \@@_item:n \scan_stop:
    #1 #2
      { \if_false: } \fi:
        \exp_after:wN \exp_after:wN
        \exp_after:wN \@@_pop_right_loop:nn
        \exp_after:wN \use_none:n
        #2
        {
          \if_false: { \fi: }
          \tex_afterassignment:D \use_none:n
          \tl_set:Nx #3
        }
    \cs_set_eq:NN \@@_item:n \seq_tmp:w
  }
\cs_new:Npn \@@_pop_right_loop:nn #1#2
  {
    #2 { \exp_not:n {#1} }
    \@@_pop_right_loop:nn
  }
\cs_generate_variant:Nn \seq_pop_right:NN  { c }
\cs_generate_variant:Nn \seq_gpop_right:NN { c }
%    \end{macrocode}
% \end{macro}
% \end{macro}
% \end{macro}
%
% \begin{macro}[TF]{\seq_get_left:NN, \seq_get_left:cN}
% \begin{macro}[TF]{\seq_get_right:NN, \seq_get_right:cN}
%   Getting from the left or right with a check on the results.  The
%   first argument to \cs{@@_pop_TF:NNNN} is left unused.
%    \begin{macrocode}
\prg_new_protected_conditional:Npnn \seq_get_left:NN #1#2 { T , F , TF }
  { \@@_pop_TF:NNNN \prg_do_nothing: \seq_get_left:NN #1#2 }
\prg_new_protected_conditional:Npnn \seq_get_right:NN #1#2 { T , F , TF }
  { \@@_pop_TF:NNNN \prg_do_nothing: \seq_get_right:NN #1#2 }
\cs_generate_variant:Nn \seq_get_left:NNT   { c }
\cs_generate_variant:Nn \seq_get_left:NNF   { c }
\cs_generate_variant:Nn \seq_get_left:NNTF  { c }
\cs_generate_variant:Nn \seq_get_right:NNT  { c }
\cs_generate_variant:Nn \seq_get_right:NNF  { c }
\cs_generate_variant:Nn \seq_get_right:NNTF { c }
%    \end{macrocode}
% \end{macro}
% \end{macro}
%
% \begin{macro}[TF]{\seq_pop_left:NN, \seq_pop_left:cN}
% \begin{macro}[TF]{\seq_gpop_left:NN, \seq_gpop_left:cN}
% \begin{macro}[TF]{\seq_pop_right:NN, \seq_pop_right:cN}
% \begin{macro}[TF]{\seq_gpop_right:NN, \seq_gpop_right:cN}
%   More or less the same for popping.
%    \begin{macrocode}
\prg_new_protected_conditional:Npnn \seq_pop_left:NN #1#2 { T , F , TF }
  { \@@_pop_TF:NNNN \@@_pop_left:NNN \tl_set:Nn #1 #2 }
\prg_new_protected_conditional:Npnn \seq_gpop_left:NN #1#2 { T , F , TF }
  { \@@_pop_TF:NNNN \@@_pop_left:NNN \tl_gset:Nn #1 #2 }
\prg_new_protected_conditional:Npnn \seq_pop_right:NN #1#2 { T , F , TF }
  { \@@_pop_TF:NNNN \@@_pop_right_aux:NNN \tl_set:Nx #1 #2 }
\prg_new_protected_conditional:Npnn \seq_gpop_right:NN #1#2 { T , F , TF }
  { \@@_pop_TF:NNNN \@@_pop_right_aux:NNN \tl_gset:Nx #1 #2 }
\cs_generate_variant:Nn \seq_pop_left:NNT    { c }
\cs_generate_variant:Nn \seq_pop_left:NNF    { c }
\cs_generate_variant:Nn \seq_pop_left:NNTF   { c }
\cs_generate_variant:Nn \seq_gpop_left:NNT   { c }
\cs_generate_variant:Nn \seq_gpop_left:NNF   { c }
\cs_generate_variant:Nn \seq_gpop_left:NNTF  { c }
\cs_generate_variant:Nn \seq_pop_right:NNT   { c }
\cs_generate_variant:Nn \seq_pop_right:NNF   { c }
\cs_generate_variant:Nn \seq_pop_right:NNTF  { c }
\cs_generate_variant:Nn \seq_gpop_right:NNT  { c }
\cs_generate_variant:Nn \seq_gpop_right:NNF  { c }
\cs_generate_variant:Nn \seq_gpop_right:NNTF { c }
%    \end{macrocode}
% \end{macro}
% \end{macro}
% \end{macro}
% \end{macro}
%
% \subsection{Mapping to sequences}
%
% \begin{macro}{\seq_map_break:}
% \UnitTested
% \begin{macro}{\seq_map_break:n}
% \UnitTested
%   To break a function, the special token \cs{__prg_break_point:Nn} is
%   used to find the end of the code. Any ending code is then inserted
%   before the return value of \cs{seq_map_break:n} is inserted.
%    \begin{macrocode}
\cs_new_nopar:Npn \seq_map_break:
  { \__prg_map_break:Nn \seq_map_break: { } }
\cs_new_nopar:Npn \seq_map_break:n
  { \__prg_map_break:Nn \seq_map_break: }
%    \end{macrocode}
% \end{macro}
% \end{macro}
%
% \begin{macro}{\seq_map_function:NN, \seq_map_function:cN}
% \UnitTested
% \begin{macro}[aux]{\@@_map_function:NNn}
%   The idea here is to apply the code of |#2| to each item in the
%   sequence without altering the definition of \cs{@@_item:n}. This
%   is done as by noting that every odd token in the sequence must be
%   \cs{@@_item:n}, which can be gobbled by \cs{use_none:n}. At the end of
%   the loop, |#2| is instead |? \seq_map_break:|, which therefore breaks the
%   loop without needing to do a (relatively-expensive) quark test.
%    \begin{macrocode}
\cs_new:Npn \seq_map_function:NN #1#2
  {
    \exp_after:wN \@@_map_function:NNn \exp_after:wN #2 #1
      { ? \seq_map_break: } { }
    \__prg_break_point:Nn \seq_map_break: { }
  }
\cs_new:Npn \@@_map_function:NNn #1#2#3
  {
    \use_none:n #2
    #1 {#3}
    \@@_map_function:NNn #1
  }
\cs_generate_variant:Nn \seq_map_function:NN { c }
%    \end{macrocode}
% \end{macro}
% \end{macro}
%
% \begin{macro}[int]{\@@_push_item_def:n, \@@_push_item_def:x}
% \begin{macro}[aux]{\@@_push_item_def:}
% \begin{macro}[int]{\@@_pop_item_def:}
%   The definition of \cs{@@_item:n} needs to be saved and restored at
%   various points within the mapping and manipulation code. That is handled
%   here: as always, this approach uses global assignments.
%    \begin{macrocode}
\cs_new_protected:Npn \@@_push_item_def:n
  {
    \@@_push_item_def:
    \cs_gset:Npn \@@_item:n ##1
  }
\cs_new_protected:Npn \@@_push_item_def:x
  {
    \@@_push_item_def:
    \cs_gset:Npx \@@_item:n ##1
  }
\cs_new_protected:Npn \@@_push_item_def:
  {
    \int_gincr:N \g__prg_map_int
    \cs_gset_eq:cN { __prg_map_ \int_use:N \g__prg_map_int :w }
      \@@_item:n
  }
\cs_new_protected_nopar:Npn \@@_pop_item_def:
  {
    \cs_gset_eq:Nc \@@_item:n
      { __prg_map_ \int_use:N \g__prg_map_int :w }
    \int_gdecr:N \g__prg_map_int
  }
%    \end{macrocode}
% \end{macro}
% \end{macro}
% \end{macro}
%
% \begin{macro}{\seq_map_inline:Nn, \seq_map_inline:cn}
% \UnitTested
%   The idea here is that \cs{@@_item:n} is already \enquote{applied} to
%   each item in a sequence, and so an in-line mapping is just a case of
%   redefining \cs{@@_item:n}.
%    \begin{macrocode}
\cs_new_protected:Npn \seq_map_inline:Nn #1#2
  {
    \@@_push_item_def:n {#2}
    #1
    \__prg_break_point:Nn \seq_map_break: { \@@_pop_item_def: }
  }
\cs_generate_variant:Nn \seq_map_inline:Nn { c }
%    \end{macrocode}
% \end{macro}
%
% \begin{macro}
%   {
%     \seq_map_variable:NNn,\seq_map_variable:Ncn,
%     \seq_map_variable:cNn,\seq_map_variable:ccn
%   }
% \UnitTested
%   This is just a specialised version of the in-line mapping function,
%   using an \texttt{x}-type expansion for the code set up so that the
%   number of |#| tokens required is as expected.
%    \begin{macrocode}
\cs_new_protected:Npn \seq_map_variable:NNn #1#2#3
  {
    \@@_push_item_def:x
      {
        \tl_set:Nn \exp_not:N #2 {##1}
        \exp_not:n {#3}
      }
    #1
    \__prg_break_point:Nn \seq_map_break: { \@@_pop_item_def: }
  }
\cs_generate_variant:Nn \seq_map_variable:NNn {     Nc }
\cs_generate_variant:Nn \seq_map_variable:NNn { c , cc }
%    \end{macrocode}
% \end{macro}
% 
% \begin{macro}{\seq_count:N, \seq_count:c}
% \begin{macro}[aux]{\@@_count:n}
%   Counting the items in a sequence is done using the same approach as for
%   other count functions: turn each entry into a \texttt{+1} then use
%   integer evaluation to actually do the mathematics.
%    \begin{macrocode}
\cs_new:Npn \seq_count:N #1
  {
    \int_eval:n
      {
        0
        \seq_map_function:NN #1 \@@_count:n
      }
  }
\cs_new:Npn \@@_count:n #1 { + \c_one }
\cs_generate_variant:Nn \seq_count:N { c }
%    \end{macrocode}
% \end{macro}
% \end{macro}
%
% \subsection{Sequence stacks}
%
% The same functions as for sequences, but with the correct naming.
%
% \begin{macro}{
%   \seq_push:Nn, \seq_push:NV, \seq_push:Nv, \seq_push:No, \seq_push:Nx,
%   \seq_push:cn, \seq_push:cV, \seq_push:cV, \seq_push:co, \seq_push:cx
% }
% \UnitTested
% \begin{macro}{
%   \seq_gpush:Nn, \seq_gpush:NV, \seq_gpush:Nv, \seq_gpush:No, \seq_gpush:Nx,
%   \seq_gpush:cn, \seq_gpush:cV, \seq_gpush:cv, \seq_gpush:co, \seq_gpush:cx
% }
% \UnitTested
%   Pushing to a sequence is the same as adding on the left.
%    \begin{macrocode}
\cs_new_eq:NN \seq_push:Nn  \seq_put_left:Nn
\cs_new_eq:NN \seq_push:NV  \seq_put_left:NV
\cs_new_eq:NN \seq_push:Nv  \seq_put_left:Nv
\cs_new_eq:NN \seq_push:No  \seq_put_left:No
\cs_new_eq:NN \seq_push:Nx  \seq_put_left:Nx
\cs_new_eq:NN \seq_push:cn  \seq_put_left:cn
\cs_new_eq:NN \seq_push:cV  \seq_put_left:cV
\cs_new_eq:NN \seq_push:cv  \seq_put_left:cv
\cs_new_eq:NN \seq_push:co  \seq_put_left:co
\cs_new_eq:NN \seq_push:cx  \seq_put_left:cx
\cs_new_eq:NN \seq_gpush:Nn \seq_gput_left:Nn
\cs_new_eq:NN \seq_gpush:NV \seq_gput_left:NV
\cs_new_eq:NN \seq_gpush:Nv \seq_gput_left:Nv
\cs_new_eq:NN \seq_gpush:No \seq_gput_left:No
\cs_new_eq:NN \seq_gpush:Nx \seq_gput_left:Nx
\cs_new_eq:NN \seq_gpush:cn \seq_gput_left:cn
\cs_new_eq:NN \seq_gpush:cV \seq_gput_left:cV
\cs_new_eq:NN \seq_gpush:cv \seq_gput_left:cv
\cs_new_eq:NN \seq_gpush:co \seq_gput_left:co
\cs_new_eq:NN \seq_gpush:cx \seq_gput_left:cx
%    \end{macrocode}
% \end{macro}
% \end{macro}
%
% \begin{macro}{\seq_get:NN, \seq_get:cN}
% \UnitTested
% \begin{macro}{\seq_pop:NN, \seq_pop:cN}
% \UnitTested
% \begin{macro}{\seq_gpop:NN, \seq_gpop:cN}
% \UnitTested
%   In most cases, getting items from the stack does not need to specify
%   that this is from the left. So alias are provided.
%    \begin{macrocode}
\cs_new_eq:NN \seq_get:NN \seq_get_left:NN
\cs_new_eq:NN \seq_get:cN \seq_get_left:cN
\cs_new_eq:NN \seq_pop:NN \seq_pop_left:NN
\cs_new_eq:NN \seq_pop:cN \seq_pop_left:cN
\cs_new_eq:NN \seq_gpop:NN \seq_gpop_left:NN
\cs_new_eq:NN \seq_gpop:cN \seq_gpop_left:cN
%    \end{macrocode}
% \end{macro}
% \end{macro}
% \end{macro}
%
% \begin{macro}[TF]{\seq_get:NN, \seq_get:cN}
% \begin{macro}[TF]{\seq_pop:NN, \seq_pop:cN}
% \begin{macro}[TF]{\seq_gpop:NN, \seq_gpop:cN}
%   More copies.
%    \begin{macrocode}
\prg_new_eq_conditional:NNn \seq_get:NN  \seq_get_left:NN  { T , F , TF }
\prg_new_eq_conditional:NNn \seq_get:cN  \seq_get_left:cN  { T , F , TF }
\prg_new_eq_conditional:NNn \seq_pop:NN  \seq_pop_left:NN  { T , F , TF }
\prg_new_eq_conditional:NNn \seq_pop:cN  \seq_pop_left:cN  { T , F , TF }
\prg_new_eq_conditional:NNn \seq_gpop:NN \seq_gpop_left:NN { T , F , TF }
\prg_new_eq_conditional:NNn \seq_gpop:cN \seq_gpop_left:cN { T , F , TF }
%    \end{macrocode}
% \end{macro}
% \end{macro}
% \end{macro}
%
% \subsection{Viewing sequences}
%
% \begin{macro}{\seq_show:N, \seq_show:c}
% \UnitTested
% Apply the general \cs{__msg_show_variable:Nnn}.
%    \begin{macrocode}
\cs_new_protected:Npn \seq_show:N #1
  {
    \__msg_show_variable:Nnn #1 { seq }
      { \seq_map_function:NN #1 \__msg_show_item:n }
  }
\cs_generate_variant:Nn \seq_show:N { c }
%    \end{macrocode}
% \end{macro}
%
% \subsection{Scratch sequences}
%
% \begin{variable}{\l_tmpa_seq, \l_tmpb_seq, \g_tmpa_seq, \g_tmpb_seq}
%   Temporary comma list variables.
%    \begin{macrocode}
\seq_new:N \l_tmpa_seq
\seq_new:N \l_tmpb_seq
\seq_new:N \g_tmpa_seq
\seq_new:N \g_tmpb_seq
%    \end{macrocode}
% \end{variable}
%
% \subsection{Deprecated interfaces}
%
%   A few functions which are no longer documented: these were moved
%   here on or before 2011-04-20, and will be removed entirely by
%   2011-07-20.
%
% \begin{macro}{\seq_top:NN, \seq_top:cN}
%   These are old stack functions.
%    \begin{macrocode}
%<*deprecated>
\cs_new_eq:NN \seq_top:NN \seq_get_left:NN
\cs_new_eq:NN \seq_top:cN \seq_get_left:cN
%</deprecated>
%    \end{macrocode}
% \end{macro}
%
% \begin{macro}{\seq_display:N, \seq_display:c}
%   An older name for \cs{seq_show:N}.
%    \begin{macrocode}
%<*deprecated>
\cs_new_eq:NN \seq_display:N \seq_show:N
\cs_new_eq:NN \seq_display:c \seq_show:c
%</deprecated>
%    \end{macrocode}
% \end{macro}
%
% Deprecated 2012-05-13 for removal by 2012-11-30.
%
% \begin{macro}{\seq_length:N, \seq_length:c}
%    \begin{macrocode}
%<*deprecated>
\cs_new_eq:NN \seq_length:N \seq_count:N
\cs_new_eq:NN \seq_length:c \seq_count:c
%</deprecated>
%    \end{macrocode}
% \end{macro}
% 
% Deprecated 2012-05-23 for removal by 2012-08-30.
%
% \begin{macro}{\seq_use:N, \seq_use:c}
%   A simple short cut for a mapping.
%    \begin{macrocode}
%<*deprecated>
\cs_new:Npn \seq_use:N #1 { \seq_map_function:NN #1 \use:n }
\cs_generate_variant:Nn \seq_use:N { c }
%</deprecated>
%    \end{macrocode}
% \end{macro}
%
%    \begin{macrocode}
%</initex|package>
%    \end{macrocode}
%
% \end{implementation}
%
% \PrintIndex
