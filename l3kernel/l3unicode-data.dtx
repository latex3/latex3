% \iffalse meta-comment
%
%% File: l3unicode-data.dtx Copyright(C) 2014 The LaTeX3 Project
%%
%% It may be distributed and/or modified under the conditions of the
%% LaTeX Project Public License (LPPL), either version 1.3c of this
%% license or (at your option) any later version.  The latest version
%% of this license is in the file
%%
%%    http://www.latex-project.org/lppl.txt
%%
%% This file is part of the "l3kernel bundle" (The Work in LPPL)
%% and all files in that bundle must be distributed together.
%%
%% The released version of this bundle is available from CTAN.
%%
%% -----------------------------------------------------------------------
%%
%% The development version of the bundle can be found at
%%
%%    http://www.latex-project.org/svnroot/experimental/trunk/
%%
%% for those people who are interested.
%%
%%%%%%%%%%%
%% NOTE: %%
%%%%%%%%%%%
%%
%%   Snapshots taken from the repository represent work in progress and may
%%   not work or may contain conflicting material!  We therefore ask
%%   people _not_ to put them into distributions, archives, etc. without
%%   prior consultation with the LaTeX Project Team.
%%
%% -----------------------------------------------------------------------
%%
%
% Both the driver and the script need \pkg{expl3}: as the script runs with
% plain \TeX{}, set up in generic mode.
%<*driver|script>
\input expl3-generic\relax
\GetIdInfo$Id$
  {L3 Case data script}
%</driver|script>
%
% The same approach as used in \pkg{DocStrip}: if \cs{documentclass}
% is undefined then skip the driver, allowing the file to be used directly.
% This works as the \cs{fi} is only seen if \LaTeX{} is not in use. The odd
% \cs{jobname} business allows the extraction to work with \LaTeX{} provided
% an appropriate \texttt{.ins} file is set up.
%<*gobble>
\ifx\jobname\relax
  \let\documentclass\undefined
\fi
\begingroup\expandafter\expandafter\expandafter\endgroup
\expandafter\ifx\csname documentclass\endcsname\relax
\else
  \csname fi\endcsname
%</gobble>
%
%<*driver>
  \documentclass[full]{l3doc}
  \begin{document}
    \DocInput{\jobname.dtx}
  \end{document}
%<*gobble>
\fi
%</gobble>
%</driver>
% \fi
%
% \title{^^A
%   The \textsf{l3unicode-data} script\\Unicode data script^^A
%   \thanks{This file describes v\ExplFileVersion,
%     last revised \ExplFileDate.}^^A
% }
%
% \author{^^A
%  The \LaTeX3 Project\thanks
%    {^^A
%      E-mail:
%        \href{mailto:latex-team@latex-project.org}
%          {latex-team@latex-project.org}^^A
%    }^^A
% }
%
% \date{Released \ExplFileDate}
%
% \maketitle
%
% \begin{documentation}
%
% The Unicode Consortium provide comprehensive data on the standard
% mapping of characters (or more formally codepoints) when carrying
% out various different case-changing functions:
% \begin{itemize}
%    \item Uppercasing
%    \item Lowercasing
%    \item Titlecasing (used for the first codepoint of a word:
%      may be subtly different to uppercasing)
%    \item Folding (removing case for comparison purposes: close
%      but not identical to lowercasing)
% \end{itemize}
% This data is available in machine readable format, such that many of
% the basics of case changing can be set up on an automated basis.
%
% This file provides a script which will read the raw Unicode files
% and convert the material to a form which can be used by \pkg{expl3}.
% As the conversions here cover the entire UTF-8 range, this cannot
% be carried out by pdf\TeX{}: at present, the script works only
% with Lua\TeX{}.
%
% Note that this file is designed such that running \LaTeX{} will typeset
% the documentation using any engine: the script will be run if the file
% is processed by plain \TeX{}, specifically the |luatex| command.
% This process requires the files |CaseFolding.txt|, |SpecialCasing.txt|
% and |UnicodeData.txt| from the \href{http://www.unicode.org/}^^A
% {Unicode Consortium website}.
%
% The file produced by this script, |l3unicode-data.def|, contains
% appropriate definitions for all of the data structures used by \pkg{expl3}
% for Unicode transformations. It also provides appropriate alternative
% definitions for use with \pdfTeX{}.
%
% \end{documentation}
%
% \begin{implementation}
%
% \section{\pkg{l3unicode-data} Implementation}
%
%    \begin{macrocode}
%<*script>
%    \end{macrocode}
%
% The driver part has loaded \pkg{expl3}: turn on the syntax environment.
%    \begin{macrocode}
\ExplSyntaxOn
%    \end{macrocode}
%
% \subsection{Setup}
%
% \begin{macro}{\str_case_x:nvF}
%   One handy variant.
%    \begin{macrocode}
\cs_generate_variant:Nn \str_case_x:nnF { nv }
%    \end{macrocode}
% \end{macro}
%
% The first step is to generate a series of temporary variables to
% contain the data as it's extracted. This requires a nested loop
% to give a total of $100$ token lists. Two sets are generated for
% use in the upper/lower case part of the script.
%    \begin{macrocode}
\tl_map_inline:nn { 0123456789 }
  {
    \tl_map_inline:nn { 0123456789 }
      {
        \tl_new:c { l__unicode_a_ #1 _ ##1 _tl }
        \tl_new:c { l__unicode_b_ #1 _ ##1 _tl }
      }
  }
%    \end{macrocode}
%
% \begin{variable}{\g__unicode_data_ior}
% \begin{variable}{\g__unicode_result_iow}
%   Streams for reading and writing the data.
%    \begin{macrocode}
\ior_new:N \g__unicode_data_ior
\iow_new:N \g__unicode_result_iow
%    \end{macrocode}
% \end{variable}
% \end{variable}
%
% Open the data file for writing.
%    \begin{macrocode}
\iow_open:Nn \g__unicode_result_iow { l3unicode-data.def }
%    \end{macrocode}
%
% Write an identification line to the file: the file data here can't be set
% automatically and so will need to be edited by hand. As such, the data here
% the standard SVN filler.
%    \begin{macrocode}
\iow_now:Nx \g__unicode_result_iow
  {
    \exp_not:N \ProvidesExplFile
      { l3unicode-data.def } ~ { 0000/00/00 } ~ { -1 } ~ { L3~Unicode~data }
  }
%    \end{macrocode}
%
% \subsection{Verbatim copying}
%
% \begin{macro}[int]{\__unicode_verb:}
% \begin{macro}[aux]{\__unicode_verb_auxi:w, \__unicode_verb_auxii:w}
% \begin{macro}[int]{\__unicode_verb_end:}
%   There are various bits of code which need to be transferred into the data
%   file from the source. This has to take place as part of the general writing
%   process so needs to be done without using DocStrip. That is achieved by
%   having a verbatim-copy mechanism available: this is all set up here.
%   As the line containing the \cs{__unicode_verb:} function will end up with a
%   (category code $12$) space at the start, there is a dedicated function to
%   clear this part up.
%    \begin{macrocode}
\group_begin:
  \char_set_catcode_other:n { `\^^M }%
  \cs_new_protected:Npn \__unicode_verb:%
    {%
      \group_begin:%
        \char_set_catcode_other:n { `\^^M }%
        \tex_endlinechar:D = `\^^M%
        \clist_map_inline:nn%
          { \\ , \{ , \} , \# , \^ , \% , \  }%
          { \char_set_catcode_other:n { `##1 } }%
        \__unicode_verb_auxi:w%
    }%
  \cs_new_protected:Npn \__unicode_verb_auxi:w#1^^M%
    {%
      \exp_after:wN \__unicode_verb_auxii:w \use_none:n #1 ^^M
    }%
  \cs_new_protected:Npn \__unicode_verb_auxii:w#1^^M%
    {%
      \str_if_eq_x:nnTF {#1} { \token_to_str:N \__unicode_verb_end: }%
        { \group_end: }%
        {%
          \iow_now:Nn \g__unicode_result_iow {#1}%
          \__unicode_verb_auxii:w%
        }%
    }%
\group_end:%
%    \end{macrocode}
% \end{macro}
% \end{macro}
% \end{macro}
%
% \subsection{Shared data}
%
% There are some data items which can be stored as numbers rather than as
% literal UTF-8 chars. These could go into the main source files, but as they
% conceptually go with everything else here this makes more sense. They are
% safe for use with \pdfTeX{} so are given first.
%    \begin{macrocode}
\__unicode_verb:
\clist_const:Nn \c__tl_after_final_sigma_clist
  { 0021 , 0022 , 0029 , 002C , 002E , 003A , 003B , 003F , 005D , 007D }
\__unicode_verb_end:
%    \end{macrocode}
%
% \subsection{\pdfTeX{} support}
%
% As \pdfTeX{} does not support UTF-8 input natively, most of the data
% here will not be useful. Rather than use two separate mechanisms for
% each function depending on the engine, the system is designed such that
% \enquote{truncated} data structures are provided for \pdfTeX{}. These
% are coded here for direct transfer to the |.def| file, which can then
% abort loading when \pdfTeX{} is in use.
%
% The idea here is simple: map over all of the letters of the Latin
% alphabet and create appropriate token lists, then add all of the rest
% of the data structures. For case folding, the tokens are all stored as
% strings. For the lower case letters, to ensure there are always three
% digits a bit of maths is used.
% 
% After the mapping, the small number of fixed data structures that are
% used for the special case conversions are created. These are mainly empty,
% but for cases where a match is possible (as the test char is in the \pdfTeX{}
% range), no-op data is included (as the \emph{output} would be out-of-range).
%    \begin{macrocode}
\__unicode_verb:
\pdftex_if_engine:T
  {
    \group_begin:
      \cs_set_protected:Npn \__unicode_tmp:NN #1#2
        {
          \quark_if_recursion_tail_stop:N #1
          \exp_after:wN \__unicode_tmp:NNNNNNN
            \tex_number:D \__int_eval:w `#1 \exp_after:wN \__int_eval_end:
            \tex_number:D \__int_eval:w 100 + `#2 \__int_eval_end:
            #1 #2
          \__unicode_tmp:NN
        }
       \cs_set_protected:Npn \__unicode_tmp:NNNNNNN #1#2#3#4#5#6#7
         {
           \tl_const:cx { c__str_fold_ #1 _ #2 _ tl }
             { \tl_to_str:n { #6#7 } }
           \tl_const:cn { c__tl_lower_ #1 _ #2 _ tl } { #6#7 }
           \tl_const:cn { c__tl_upper_ #4 _ #5 _ tl } { #7#6 }
         }
       \__unicode_tmp:NN
         AaBbCcDdEeFfGgHhIiJjKkLlMmNnOoPpQqRrSsTtUuVvWwXxYyZz
        \q_recursion_tail ? \q_recursion_stop
    \group_end:
    \int_step_inline:nnnn { 0 } { 1 } { 9 }
      {
        \int_step_inline:nnnn { 0 } { 1 } { 9 }
          {
            \tl_if_exist:cF { c__str_fold_ #1 _ ##1 _ tl }
              {
                \tl_const:cn { c__str_fold_ #1 _ ##1 _ tl } { }
              }
            \tl_if_exist:cF { c__tl_lower_ #1 _ ##1 _ tl }
              {
                \tl_const:cn { c__tl_lower_ #1 _ ##1 _ tl } { }
              }
            \tl_if_exist:cF { c__tl_upper_ #1 _ ##1 _ tl }
              {
                \tl_const:cn { c__tl_upper_ #1 _ ##1 _ tl } { }
              }
          }
      }
    \tl_const:Nn \c__tl_title_exceptions_tl { }
    \tl_const:Nn \c__tl_std_sigma_tl   { }
    \tl_const:Nn \c__tl_final_sigma_tl { }
    \tl_const:Nn \c__tl_accents_lt_tl  { }
    \tl_const:Nn \c__tl_dot_above_tl   { }
    \tl_const:Nn \c__tl_dotless_i_tl   { I }
    \tl_const:Nn \c__tl_dotted_I_tl    { i }
    \tex_endinput:D
  }
\__unicode_verb_end:
%    \end{macrocode}
%
% \subsection{Case folding}
%
% \begin{macro}{\__unicode_parse_line:w}
% \begin{macro}[aux]{\__unicode_parse_line_auxi:Nw}
% \begin{macro}[aux]{\__unicode_parse_line_auxii:w}
% \begin{macro}[aux]{\__unicode_parse_line_auxiii:nw}
% \begin{macro}[aux]{\__unicode_parse_line_auxiv:nn}
% \begin{macro}[aux]{\__unicode_parse_line_auxv:wnn}
%   The format of |CaseFolding.txt| allows for both blank lines and
%   C-style comments starting with |#|. Thus the first two steps of
%   the parsing routine are set up to deal with these cases.
%    \begin{macrocode}
\cs_new_protected:Npn \__unicode_parse_line:w #1 \q_stop
  {
    \tl_if_blank:nF {#1}
      { \__unicode_parse_line_auxi:Nw #1 \q_stop }
  }
\cs_new_protected:Npn \__unicode_parse_line_auxi:Nw #1#2 \q_stop
  {
    \str_if_eq_x:nnF { \exp_not:n {#1} } { \cs_to_str:N \# }
      { \__unicode_parse_line_auxii:w #1#2 \q_stop }
  }
%    \end{macrocode}
%   For lines actually containing data, there will be four entries separated by
%   |;| tokens: the hex code for the char itself, which folding regim\'{e}s
%   the line applies to, the hex code(s) for the folded char and a
%   description. Of these, we need all but the last one. In the simple
%   case of core foldings, the mapping is one--one and this information
%   can be passed directly to the next stage. We also handle the full
%   mappings (dropping simple ones plus any Turkic variation): an additional
%   step is needed to parse this case.
%    \begin{macrocode}
\cs_new_protected:Npn \__unicode_parse_line_auxii:w #1 ;~ #2 ; #3 ; #4 \q_stop
  {
    \str_if_eq:nnTF {#2} { C }
      {
        \__unicode_parse_line_auxiv:nn
          {#1} { \luatex_Uchar:D "#3 \c_space_tl }
      }
      {
        \str_if_eq:nnT {#2} { F }
          { \__unicode_parse_line_auxiii:nw {#1} #3 ~ \q_stop }
      }
  }
%    \end{macrocode}
%   Full folding produces two or three Unicode code points from a single
%   input char. To deal with this, we split the relevant part of the input
%   and check how many chars to generate. The entire folding output is
%   braced so that when read back \TeX{} will see this as a group in our
%   replacement code.
%    \begin{macrocode}
\cs_new_protected:Npn \__unicode_parse_line_auxiii:nw #1 ~ #2 ~ #3 ~ #4 \q_stop
  {
    \__unicode_parse_line_auxiv:nn
      {#1}
      {
        {
          \luatex_Uchar:D "#2 \c_space_tl
          \luatex_Uchar:D "#3 \c_space_tl
          \tl_if_empty:nF {#4}
            { \luatex_Uchar:D "#4 \c_space_tl }
        }
      }
  }
%    \end{macrocode}
%   The final stage of extracting the mapping is to split the various cases
%   up such that comparison and replacement does not need to check every
%   character. That is done by taking the charcode modulo $100$: this splits
%   the list of chars into $100$ much shorter lists. With that done, the
%   input and output chars are added to the appropriate token lists.
%    \begin{macrocode}
\cs_new_protected:Npn \__unicode_parse_line_auxiv:nn #1#2
  {
    \exp_last_unbraced:Nf \__unicode_parse_line_auxv:wnn
      { \int_eval:n { 1000000 + "#1 } } \q_stop
      {#1} {#2}
  }
%    \end{macrocode}
%   As the input is read in string mode, there is a need for a rescan
%   here since \tn{Uchar} requires letters for hexadecimal digits
%   beyond~$9$.
%    \begin{macrocode}
\cs_new_protected:Npn \__unicode_parse_line_auxv:wnn
  #1#2#3#4#5#6#7 \q_stop #8#9
  {
     \tl_rescan:nn
      { }
      {
        \tl_put_right:cx { l__unicode_a_ #6 _ #7 _tl }
          {
             \luatex_Uchar:D "#8 \c_space_tl
             #9
          }
      }
  }
%    \end{macrocode}
% \end{macro}
% \end{macro}
% \end{macro}
% \end{macro}
% \end{macro}
% \end{macro}
%
% The main loop can now take place, reading the source data and saving all of
% the information in the token list array.
%    \begin{macrocode}
\ior_open:Nn \g__unicode_data_ior { CaseFolding.txt }
\ior_str_map_inline:Nn \g__unicode_data_ior
  { \__unicode_parse_line:w #1 \q_stop }
\ior_close:N \g__unicode_data_ior
%    \end{macrocode}
%
% \begin{macro}[aux]{\__str_tmp:NNn}
% \begin{macro}[aux, EXP]{\__str_tmp:Nw}
%   To ensure that the output of the case-folding function is a string, all of
%   the stored results need to be detokenized. That is done by including a loop
%   in the |.def| file which will do the necessary change. To set that up, a
%   slightly complicated bit of secondary work: write the functions which do
%   the job into the |.def| file itself, using a group to trap the temporary
%   code. There is also a test in the following so that the result only has
%   braces around items which need it: this is a slight performance tweak when
%   the code actually gets used. Notice that everything in the token list is
%   detokenized except for the |{| and |}| chars needed for grouping: if the
%   search part of the list is not detokenized there are issues with \XeTeX{}
%   and chars beyond $0\mathrm{xFFFF}$ (probably a bug, but can be worked
%   around!).
%    \begin{macrocode}
\__unicode_verb:
\group_begin:
  \cs_set_protected:Npn \__str_tmp:NNn #1#2#3
    {
      \tl_const:cx { c__str_fold_#1_#2_tl }
        { \__str_tmp:Nw #3 \q_recursion_tail { } \q_recursion_stop }
    }
  \cs_set:Npn \__str_tmp:Nw #1#2
    {
      \quark_if_recursion_tail_stop:N #1
      \tl_to_str:N #1
      \tl_if_blank:oT { \use_none:n #2 }
        { \use:n }
        { \tl_to_str:n {#2} }
      \__str_tmp:Nw
    }
\__unicode_verb_end:
%    \end{macrocode}
% \end{macro}
% \end{macro}
%
% The write loop is simple: map over the array and write everything to the
% output. The saved data is also cleared to save a second loop later on when
% dealing with case mappings. The group used for the temporary stuff in the
% |.def| file is also closed at this point.
%    \begin{macrocode}
\tl_map_inline:nn { 0123456789 }
  {
    \tl_map_inline:nn { 0123456789 }
      {
        \iow_now:Nx \g__unicode_result_iow
          {
            \c_space_tl \c_space_tl
            \exp_not:N \__str_tmp:NNn #1 ~ ##1 ~
              { ~ \exp_not:v { l__unicode_a_ #1 _ ##1 _tl } ~ }
          }
        \tl_clear:c { l__unicode_a_ #1 _ ##1 _tl }
      }
  }
\iow_now:Nn \g__unicode_result_iow { \group_end: }
%    \end{macrocode}
%
% \subsection{Upper/lower/title casing}
%
% Unlike the case folding data, case changing data is split into two parts
% which we need to combine into a single data structure. There are therefore
% two parts to this process: first to read the exceptions, then to read the
% main data and combine it.
%
% \begin{macro}^^A
%   {
%     \l__unicode_lower_exceptions_tl,
%     \l__unicode_title_exceptions_tl,
%     \l__unicode_upper_exceptions_tl
%    }
%   There are special cases for lower, title and uppercase changes: these
%   all get read in to appropriate lists. Exceptions could be saved as
%   property lists but that would make life a bit more complex with the
%   titlecase exceptions and wouldn't really gain much (this is after all
%   \enquote{disposable} data).
%    \begin{macrocode}
\tl_new:N \l__unicode_lower_exceptions_tl
\tl_new:N \l__unicode_title_exceptions_tl
\tl_new:N \l__unicode_upper_exceptions_tl
%    \end{macrocode}
% \end{macro}
%
% \begin{macro}[aux]{\__unicode_parse_line_auxii:w}
% \begin{macro}[aux]{\__unicode_parse_line_auxiii:nnn}
% \begin{macro}[aux]{\__unicode_parse_line_auxiv:nwn}
% \begin{macro}[aux]{\__unicode_brace:n}
%   The format of the special cases data is similar to that of the folding
%   data: as such only some of the parsing is altered. This file has four
%   important data fields: the char at it's lower, title and uppercase
%   equivalents. As most of the titlecase exceptions are also uppercase
%   exceptions, a test is made so that we are only storing truly useful
%   exceptions for titlecase.
%    \begin{macrocode}
\cs_set_protected:Npn \__unicode_parse_line_auxii:w
  #1 ;~ #2 ;~ #3 ;~ #4 ; #5 \q_stop
  {
    \__unicode_parse_line_auxiii:nnn {#1} {#2} { lower }
    \str_if_eq:nnF {#3} {#4}
      { \__unicode_parse_line_auxiii:nnn {#1} {#3} { title } }
    \__unicode_parse_line_auxiii:nnn {#1} {#4} { upper }
  }
%    \end{macrocode}
%   Unlike the folding data, the special cases file always has a value for
%   each of the three entries. Some of these have only one hex number in.
%   After a bit of a trick to allow for ease of parsing, we check if there
%   are at least two numbers for the case-changed char. If there are, then
%   save the exception. If not, then the value will also be in the main
%   table and we can ignore it here. There is also a test to see if the
%   current value is a titlecase exception: they don't need extra braces
%   for those.
%    \begin{macrocode}
\cs_new_protected:Npn \__unicode_parse_line_auxiii:nnn #1#2#3
  { \use:n { \__unicode_parse_line_auxiv:nwn {#1} #2 ~ } ~ \q_stop {#3} }
\cs_new_protected:Npn \__unicode_parse_line_auxiv:nwn #1#2 ~ #3 ~ #4 \q_stop #5
  {
    \tl_if_empty:nF {#3}
      {
      \str_if_eq:nnTF {#5} { title }
        { \cs_set_eq:NN \__unicode_brace:n \use:n }
        { \cs_set:Npn \__unicode_brace:n ##1 { { ##1 } } }
        \tl_rescan:nn
          { }
          {
            \tl_put_right:cx { l__unicode_ #5 _exceptions_tl }
              {
                \luatex_Uchar:D "#1 \c_space_tl
                {
                  \__unicode_brace:n
                    {
                      \luatex_Uchar:D "#2 \c_space_tl
                      \luatex_Uchar:D "#3 \c_space_tl
                      \tl_if_empty:nF {#4}
                        { \luatex_Uchar:D "#4 \c_space_tl }
                    }
                }
              }
          }
      }
  }
\cs_new_eq:NN \__unicode_brace:n \use:n
%    \end{macrocode}
% \end{macro}
% \end{macro}
% \end{macro}
% \end{macro}
%
% Parsing set up, read the special cases file. The input contains both
% general special cases and ones dependent on context. We only want to read
% the former, so there is a check for the line that splits the two:
% at that point, simply stop parsing.
%    \begin{macrocode}
\ior_open:Nn \g__unicode_data_ior { SpecialCasing.txt }
\ior_str_map_inline:Nn \g__unicode_data_ior
  {
    \str_if_eq_x:nnTF {#1} { \cs_to_str:N \# \c_space_tl Conditional~Mappings }
      { \ior_map_break: }
      { \__unicode_parse_line:w #1 \q_stop }
  }
\ior_close:N \g__unicode_data_ior
%    \end{macrocode}
%
% \begin{macro}{\__unicode_parse_line:w}
% \begin{macro}[aux]{\__unicode_parse_line_auxi:w}
% \begin{macro}[aux]{\__unicode_parse_line_auxii:nnNn}
% \begin{macro}[aux]{\__unicode_parse_line_auxiii:wnnNn}
% \begin{macro}[aux]{\__unicode_parse_line_auxiv:nnNNNn}
%   Much the same as for the case folding set up: parse the lines of
%   data. Here, the lines are longer but always have one--one mappings.
%    \begin{macrocode}
\cs_set_protected:Npn \__unicode_parse_line:w
  #1 ; #2 ; #3 ; #4 ; #5 ; #6 ; #7 ; #8 ; #9 ;
  {
    \__unicode_parse_line_auxi:w #1 ;
  }
%    \end{macrocode}
%   With some data items removed, at this stage the hexadecimal
%   representation of the char is |#1|, the upper case char is |#5|,
%   the lower case one |#6| and the title case one |#7|. These may or
%   may not be present and the upper and titlecase values may be
%   identical. Where there are values for upper/lowercase, they are
%   saved into the arrays. For titlecase, since the number of exceptions
%   is small: they are added to the existing list of exceptions we've
%   already started.
%    \begin{macrocode}
\cs_new_protected:Npn \__unicode_parse_line_auxi:w
  #1 ; #2 ; #3 ; #4 ; #5 ; #6 ; #7 \q_stop
  {
    \tl_if_empty:nF {#5}
      {
        \__unicode_parse_line_auxii:nnNn {#1} {#5} a { upper }
        \str_if_eq:nnF {#5} {#7}
          {
            \tl_put_right:Nx \l__unicode_title_exceptions_tl
              {
                \luatex_Uchar:D "#1 \c_space_tl
                \luatex_Uchar:D "#7 \c_space_tl
              }
          }
      }
    \tl_if_empty:nF {#6}
      { \__unicode_parse_line_auxii:nnNn {#1} {#6} b { lower } }
  }
%    \end{macrocode}
%   The array structure here is the same as before, except now there
%   are two separate ones to manage.
%    \begin{macrocode}
\cs_new_protected:Npn \__unicode_parse_line_auxii:nnNn #1#2#3
  {
    \exp_last_unbraced:Nf \__unicode_parse_line_auxiii:wnnNn
      { \int_eval:n { 1000000 + "#1 } } \q_stop
      {#1} {#2} #3
  }
\cs_new_protected:Npn \__unicode_parse_line_auxiii:wnnNn
  #1#2#3#4#5#6#7 \q_stop #8#9
  { \__unicode_parse_line_auxiv:nnNNNn {#8} {#9} #6 #7 }
%    \end{macrocode}
%   The final test required here is to look for the special cases and where
%   appropriate use that rather than the one--one mapping value.
%    \begin{macrocode}
\cs_new_protected:Npn \__unicode_parse_line_auxiv:nnNNNn #1#2#3#4#5#6
  {
    \tl_rescan:nn
      { }
      {
        \tl_put_right:cx { l__unicode_ #5 _ #3 _ #4 _tl }
          {
             \luatex_Uchar:D "#1 \c_space_tl
             \str_case_x:nvF
               { \luatex_Uchar:D "#1 \c_space_tl }
               { l__unicode_ #6 _exceptions_tl }
               { \luatex_Uchar:D "#2 \c_space_tl }
          }
     }
  }
%    \end{macrocode}
% \end{macro}
% \end{macro}
% \end{macro}
% \end{macro}
% \end{macro}
%
% Everything is set up and so the read loop can take place: this time
% there are no comment chars to worry about and so normal category
% codes apply.
%    \begin{macrocode}
\ior_open:Nn \g__unicode_data_ior { UnicodeData.txt }
\ior_map_inline:Nn \g__unicode_data_ior
  { \__unicode_parse_line:w #1 \q_stop }
\ior_close:N \g__unicode_data_ior
%    \end{macrocode}
%
% Saving the data uses a single file, with the uppercase array
% followed by the lowercase one and finally the titlecase exceptions.
%    \begin{macrocode}
\tl_map_inline:nn { 0123456789 }
  {
    \tl_map_inline:nn { 0123456789 }
      {
        \iow_now:Nx \g__unicode_result_iow
          {
            \tl_const:cn
              { ~ c__tl_upper_ #1 _ ##1 _tl ~ } ~
              { ~ \exp_not:v { l__unicode_a_ #1 _ ##1 _tl } ~ }
          }
      }
  }
\tl_map_inline:nn { 0123456789 }
  {
    \tl_map_inline:nn { 0123456789 }
      {
        \iow_now:Nx \g__unicode_result_iow
          {
            \tl_const:cn
              { ~ c__tl_lower_ #1 _ ##1 _tl ~ } ~
              { ~ \exp_not:v { l__unicode_b_ #1 _ ##1 _tl } ~ }
          }
      }
  }
\iow_now:Nx \g__unicode_result_iow
  {
    \tl_const:Nn
      \exp_not:N \c__tl_title_exceptions_tl \c_space_tl
      { ~ \exp_not:V \l__unicode_title_exceptions_tl \c_space_tl }
  }
%    \end{macrocode}
%
% Data for the special cases is now needed. This is mainly a series of simple
% token lists with appropriate names and content, but there is also one place
% where a small mapping list is required.
%    \begin{macrocode}
\cs_new_protected:Npn \__unicode_special_case:nn #1#2
  {
    \quark_if_recursion_tail_stop:n {#1}
    \iow_now:Nx \g__unicode_result_iow
      {
        \tl_const:Nn \exp_not:c { c__tl_ #1 _tl } { \luatex_Uchar:D "#2 }
      }
    \__unicode_special_case:nn
  }
\__unicode_special_case:nn
  { std_sigma }     { 03C3 }
  { final_sigma }   { 03C2 }
  { dotless_i }     { 0131 }
  { dot_above }     { 0307 }
  { dotted_I }      { 0130 }
  \q_recursion_tail {      }
  \q_recursion_stop
\iow_now:Nx \g__unicode_result_iow
  {
    \tl_const:Nn \exp_not:N \c__tl_accents_lt_tl
      {
        \luatex_Uchar:D "00CC
          { \luatex_Uchar:D "0069 \luatex_Uchar:D "0307 \luatex_Uchar:D "0300 }
        \luatex_Uchar:D "00CD
          { \luatex_Uchar:D "0069 \luatex_Uchar:D "0307 \luatex_Uchar:D "0301 }
        \luatex_Uchar:D "0128
          { \luatex_Uchar:D "0069 \luatex_Uchar:D "0307 \luatex_Uchar:D "0303 }
      }
  }
%    \end{macrocode}
%
% Job done, end the \TeX{} run.
%    \begin{macrocode}
\iow_close:N \g__unicode_result_iow
\tex_end:D
%    \end{macrocode}
%
%    \begin{macrocode}
%</script>
%    \end{macrocode}
%
% \end{implementation}
%
% \PrintIndex