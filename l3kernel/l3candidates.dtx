% \iffalse meta-comment
%
%% File: l3candidates.dtx
%
% Copyright (C) 2012-2023 The LaTeX Project
%
% It may be distributed and/or modified under the conditions of the
% LaTeX Project Public License (LPPL), either version 1.3c of this
% license or (at your option) any later version.  The latest version
% of this license is in the file
%
%    https://www.latex-project.org/lppl.txt
%
% This file is part of the "l3kernel bundle" (The Work in LPPL)
% and all files in that bundle must be distributed together.
%
% -----------------------------------------------------------------------
%
% The development version of the bundle can be found at
%
%    https://github.com/latex3/latex3
%
% for those people who are interested.
%
%<*driver>
\documentclass[full,kernel]{l3doc}
\begin{document}
  \DocInput{\jobname.dtx}
\end{document}
%</driver>
% \fi
%
% \title{^^A
%   The \textsf{l3candidates} package\\ Experimental additions to
%   \pkg{l3kernel}^^A
% }
%
% \author{^^A
%  The \LaTeX{} Project\thanks
%    {^^A
%      E-mail:
%        \href{mailto:latex-team@latex-project.org}
%          {latex-team@latex-project.org}^^A
%    }^^A
% }
%
% \date{Released 2023-10-23}
%
% \maketitle
%
% \begin{documentation}
%
% \section{Important notice}
%
% This module provides a space in which functions can be added to
% \pkg{l3kernel} (\pkg{expl3}) while still being experimental.
% \begin{quote}
%  \bfseries
% As such, the
% functions here may not remain in their current form, or indeed at all,
% in \pkg{l3kernel} in the future.
% \end{quote}
%  In contrast to the material in
% \pkg{l3experimental}, the functions here are all \emph{small} additions to
% the kernel. We encourage programmers to test them out and report back on
% the \texttt{LaTeX-L} mailing list.
%
% \medskip
%
%   Thus, if you intend to use any of these functions from the candidate module in a public package
%  offered to others for productive use (e.g., being placed on CTAN) please consider the following points carefully:
% \begin{itemize}
% \item Be prepared that your public packages might require updating when such functions
%        are being finalized.
% \item Consider informing us that you use a particular function in your public package, e.g., by
%         discussing this on the \texttt{LaTeX-L}
%    mailing list. This way it becomes easier to coordinate any updates necessary without issues
%    for the users of your package.
% \item Discussing and understanding use cases for a particular addition or concept also helps to
%         ensure that we provide the right interfaces in the final version so please give us feedback
%         if you consider a certain candidate function useful (or not).
% \end{itemize}
% We only add functions in this space if we consider them being serious candidates for a final inclusion
% into the kernel. However, real use sometimes leads to better ideas, so functions from this module are
% \textbf{not necessarily stable} and we may have to adjust them!
%
% \section{Additions to \pkg{l3seq}}
%
% \begin{function}{\seq_set_filter:NNn, \seq_gset_filter:NNn}
%   \begin{syntax}
%     \cs{seq_set_filter:NNn} \meta{sequence_1} \meta{sequence_2} \Arg{inline boolexpr}
%   \end{syntax}
%   Evaluates the \meta{inline boolexpr} for every \meta{item} stored
%   within the \meta{sequence_2}. The \meta{inline boolexpr}
%   receives the \meta{item} as |#1|. The sequence of all \meta{items}
%   for which the \meta{inline boolexpr} evaluated to \texttt{true}
%   is assigned to \meta{sequence_1}.
%   \begin{texnote}
%     Contrarily to other mapping functions, \cs{seq_map_break:} cannot
%     be used in this function, and would lead to low-level \TeX{} errors.
%   \end{texnote}
% \end{function}
%
% \section{Additions to \pkg{l3tl}}
%
% \begin{function}[added = 2018-04-01]{\tl_build_get:NN}
%   \begin{syntax}
%     \cs{tl_build_get:NN} \meta{tl~var_1} \meta{tl~var_2}
%   \end{syntax}
%   Stores the contents of the \meta{tl~var_1} in the \meta{tl~var_2}.
%   The \meta{tl~var_1} must have been set up with \cs{tl_build_begin:N}
%   or \cs{tl_build_gbegin:N}.  The \meta{tl~var_2} is a
%   \enquote{normal} token list variable, assigned locally using
%   \cs{tl_set:Nn}.
% \end{function}
%
% \end{documentation}
%
% \begin{implementation}
%
% \section{\pkg{l3candidates} implementation}
%
%    \begin{macrocode}
%<*package>
%    \end{macrocode}
%
% \subsection{Additions to \pkg{l3seq}}
%
%    \begin{macrocode}
%<@@=seq>
%    \end{macrocode}
%
% \begin{macro}{\seq_set_filter:NNn, \seq_gset_filter:NNn}
% \begin{macro}{\@@_set_filter:NNNn}
%   Similar to \cs{seq_map_inline:Nn}, without a
%   \cs{prg_break_point:} because the user's code
%   is performed within the evaluation of a boolean expression,
%   and skipping out of that would break horribly.
%   The \cs{@@_wrap_item:n} function inserts the relevant
%   \cs{@@_item:n} without expansion in the input stream,
%   hence in the \texttt{e}-expanding assignment.
%    \begin{macrocode}
\cs_new_protected:Npn \seq_set_filter:NNn
  { \@@_set_filter:NNNn \__kernel_tl_set:Ne }
\cs_new_protected:Npn \seq_gset_filter:NNn
  { \@@_set_filter:NNNn \__kernel_tl_gset:Ne }
\cs_new_protected:Npn \@@_set_filter:NNNn #1#2#3#4
  {
    \@@_push_item_def:n { \bool_if:nT {#4} { \@@_wrap_item:n {##1} } }
    #1 #2 { #3 }
    \@@_pop_item_def:
  }
%    \end{macrocode}
% \end{macro}
% \end{macro}
%
% \subsection{Additions to \pkg{l3tl}}
%
% \subsubsection{Building a token list}
%
%    \begin{macrocode}
%<@@=tl>
%    \end{macrocode}
%
% \begin{macro}{\tl_build_get:NN}
%    \begin{macrocode}
\cs_new_protected:Npn \tl_build_get:NN
  { \@@_build_get:NNN \__kernel_tl_set:Ne }
%    \end{macrocode}
% \end{macro}
%
%    \begin{macrocode}
%</package>
%    \end{macrocode}
%
% \end{implementation}
%
% \PrintIndex
