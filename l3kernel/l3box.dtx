% \iffalse meta-comment
%
%% File: l3box.dtx Copyright (C) 2005-2012 The LaTeX3 Project
%%
%% It may be distributed and/or modified under the conditions of the
%% LaTeX Project Public License (LPPL), either version 1.3c of this
%% license or (at your option) any later version.  The latest version
%% of this license is in the file
%%
%%    http://www.latex-project.org/lppl.txt
%%
%% This file is part of the "l3kernel bundle" (The Work in LPPL)
%% and all files in that bundle must be distributed together.
%%
%% The released version of this bundle is available from CTAN.
%%
%% -----------------------------------------------------------------------
%%
%% The development version of the bundle can be found at
%%
%%    http://www.latex-project.org/svnroot/experimental/trunk/
%%
%% for those people who are interested.
%%
%%%%%%%%%%%
%% NOTE: %%
%%%%%%%%%%%
%%
%%   Snapshots taken from the repository represent work in progress and may
%%   not work or may contain conflicting material!  We therefore ask
%%   people _not_ to put them into distributions, archives, etc. without
%%   prior consultation with the LaTeX3 Project.
%%
%% -----------------------------------------------------------------------
%
%<*driver|package>
\RequirePackage{l3names}
\GetIdInfo$Id$
  {L3 Experimental boxes}
%</driver|package>
%<*driver>
\documentclass[full]{l3doc}
\begin{document}
  \DocInput{\jobname.dtx}
\end{document}
%</driver>
% \fi
%
% \title{^^A
%   The \pkg{l3box} package\\ Boxes^^A
%   \thanks{This file describes v\ExplFileVersion,
%      last revised \ExplFileDate.}^^A
% }
%
% \author{^^A
%  The \LaTeX3 Project\thanks
%    {^^A
%      E-mail:
%        \href{mailto:latex-team@latex-project.org}
%          {latex-team@latex-project.org}^^A
%    }^^A
% }
%
% \date{Released \ExplFileDate}
%
% \maketitle
%
% \begin{documentation}
%
% There are three kinds of box operations: horizontal mode denoted
% with prefix |\hbox_|, vertical mode with prefix |\vbox_|, and the
% generic operations working in both modes with prefix |\box_|.
%
% \section{Creating and initialising boxes}
%
% \begin{function}{\box_new:N, \box_new:c}
%   \begin{syntax}
%     \cs{box_new:N} \meta{box}
%   \end{syntax}
%   Creates a new \meta{box} or raises an error if the name is
%   already taken. The declaration is global. The \meta{box} will
%   initially be void.
% \end{function}
%
% \begin{function}{\box_clear:N, \box_clear:c, \box_gclear:N, \box_gclear:c}
%   \begin{syntax}
%     \cs{box_clear:N} \meta{box}
%   \end{syntax}
%   Clears the content of the \meta{box} by setting the box equal to
%   \cs{c_void_box}.
% \end{function}
%
% \begin{function}
%   {\box_clear_new:N, \box_clear_new:c, \box_gclear_new:N, \box_gclear_new:c}
%   \begin{syntax}
%     \cs{box_clear_new:N} \meta{box}
%   \end{syntax}
%   Ensures that the \meta{box} exists globally by applying
%   \cs{box_new:N} if necessary, then applies \cs{box_(g)clear:N} to leave
%   the \meta{box} empty.
% \end{function}
%
% \begin{function}
%   {
%     \box_set_eq:NN,  \box_set_eq:cN,  \box_set_eq:Nc,  \box_set_eq:cc,
%     \box_gset_eq:NN, \box_gset_eq:cN, \box_gset_eq:Nc, \box_gset_eq:cc
%   }
%   \begin{syntax}
%     \cs{box_set_eq:NN} \meta{box_1} \meta{box_2}
%   \end{syntax}
%   Sets the content of \meta{box_1} equal to that of \meta{box_2}.
% \end{function}
%
% \begin{function}
%   {
%     \box_set_eq_clear:NN, \box_set_eq_clear:cN,
%     \box_set_eq_clear:Nc, \box_set_eq_clear:cc
%   }
%   \begin{syntax}
%     \cs{box_set_eq_clear:NN} \meta{box_1} \meta{box_2}
%   \end{syntax}
%   Sets the content of \meta{box_1} within the current \TeX{} group
%   equal to that of \meta{box_2}, then clears \meta{box_2} globally.
% \end{function}
%
% \begin{function}
%   {
%     \box_gset_eq_clear:NN, \box_gset_eq_clear:cN,
%     \box_gset_eq_clear:Nc, \box_gset_eq_clear:cc
%   }
%   \begin{syntax}
%     \cs{box_gset_eq_clear:NN} \meta{box_1} \meta{box_2}
%   \end{syntax}
%   Sets the content of \meta{box_1} equal to that of \meta{box_2}, then
%   clears \meta{box_2}. These assignments are global.
% \end{function}
%
% \begin{function}[EXP, pTF, added=2012-03-03]
%   {\box_if_exist:N, \box_if_exist:c}
%   \begin{syntax}
%     \cs{box_if_exist_p:N} \meta{box}
%     \cs{box_if_exist:NTF} \meta{box} \Arg{true code} \Arg{false code}
%   \end{syntax}
%   Tests whether the \meta{box} is currently defined.  This does not
%   check that the \meta{box} really is a box.
% \end{function}
%
% \section{Using boxes}
%
% \begin{function}{\box_use:N, \box_use:c}
%   \begin{syntax}
%     \cs{box_use:N} \meta{box}
%   \end{syntax}
%   Inserts the current content of the \meta{box} onto the current
%   list for typesetting.
%   \begin{texnote}
%     This is the \TeX{} primitive \tn{copy}.
%   \end{texnote}
% \end{function}
%
% \begin{function}{\box_use_clear:N, \box_use_clear:c}
%   \begin{syntax}
%     \cs{box_use_clear:N} \meta{box}
%   \end{syntax}
%   Inserts the current content of the \meta{box} onto the current
%   list for typesetting, then globally clears the content of the
%   \meta{box}.
%   \begin{texnote}
%     This is the \TeX{} primitive \tn{box}.
%   \end{texnote}
% \end{function}
%
% \begin{function}{\box_move_right:nn, \box_move_left:nn}
%   \begin{syntax}
%     \cs{box_move_right:nn} \Arg{dimexpr} \Arg{box function}
%   \end{syntax}
%   This function operates in vertical mode, and inserts the
%   material specified by the \meta{box function}
%   such that its reference point is displaced horizontally by the given
%   \meta{dimexpr} from the reference point for typesetting, to the right
%   or left as appropriate. The \meta{box function} should be
%   a box operation such as |\box_use:N \<box>| or a \enquote{raw}
%   box specification such as |\vbox:n { xyz }|.
% \end{function}
%
% \begin{function}{\box_move_up:nn, \box_move_down:nn}
%   \begin{syntax}
%     \cs{box_move_up:nn} \Arg{dimexpr} \Arg{box function}
%   \end{syntax}
%   This function operates in horizontal mode, and inserts the
%   material specified by the \meta{box function}
%   such that its reference point is displaced vertical by the given
%   \meta{dimexpr} from the reference point for typesetting, up
%   or down as appropriate. The \meta{box function} should be
%   a box operation such as |\box_use:N \<box>| or a \enquote{raw}
%   box specification such as |\vbox:n { xyz }|.
% \end{function}
%
% \section{Measuring and setting box dimensions}
%
% \begin{function}{\box_dp:N, \box_dp:c}
%   \begin{syntax}
%     \cs{box_dp:N} \meta{box}
%   \end{syntax}
%   Calculates the depth (below the baseline) of the \meta{box}
%   in a form suitable for use in a \meta{dimension expression}.
%   \begin{texnote}
%     This is the \TeX{} primitive \tn{dp}.
%   \end{texnote}
% \end{function}
%
% \begin{function}{\box_ht:N, \box_ht:c}
%   \begin{syntax}
%     \cs{box_ht:N} \meta{box}
%   \end{syntax}
%   Calculates the height (above the baseline) of the \meta{box}
%   in a form suitable for use in a \meta{dimension expression}.
%   \begin{texnote}
%     This is the \TeX{} primitive \tn{ht}.
%   \end{texnote}
% \end{function}
%
% \begin{function}{\box_wd:N, \box_wd:c}
%   \begin{syntax}
%     \cs{box_wd:N} \meta{box}
%   \end{syntax}
%   Calculates the width of the \meta{box} in a form
%   suitable for use in a \meta{dimension expression}.
%   \begin{texnote}
%     This is the \TeX{} primitive \tn{wd}.
%   \end{texnote}
% \end{function}
%
% \begin{function}[updated = 2011-10-22]{\box_set_dp:Nn, \box_set_dp:cn}
%   \begin{syntax}
%     \cs{box_set_dp:Nn} \meta{box} \Arg{dimension expression}
%   \end{syntax}
%   Set the depth (below the baseline) of the \meta{box} to the value of
%   the \Arg{dimension expression}. This is a global assignment.
% \end{function}
%
% \begin{function}[updated = 2011-10-22]{\box_set_ht:Nn, \box_set_ht:cn}
%   \begin{syntax}
%     \cs{box_set_ht:Nn} \meta{box} \Arg{dimension expression}
%   \end{syntax}
%   Set the height (above the baseline) of the \meta{box} to the value of
%   the \Arg{dimension expression}. This is a global assignment.
% \end{function}
%
% \begin{function}[updated = 2011-10-22]{\box_set_wd:Nn, \box_set_wd:cn}
%   \begin{syntax}
%     \cs{box_set_wd:Nn} \meta{box} \Arg{dimension expression}
%   \end{syntax}
%   Set the width of the \meta{box} to the value of the
%   \Arg{dimension expression}. This is a global assignment.
% \end{function}
%
% \section{Box conditionals}
%
% \begin{function}[EXP,pTF]{\box_if_empty:N, \box_if_empty:c}
%   \begin{syntax}
%     \cs{box_if_empty_p:N} \meta{box}
%     \cs{box_if_empty:NTF} \meta{box} \Arg{true code} \Arg{false code}
%   \end{syntax}
%   Tests if \meta{box} is a empty (equal to \cs{c_empty_box}).
% \end{function}
%
% \begin{function}[EXP,pTF]{\box_if_horizontal:N, \box_if_horizontal:c}
%   \begin{syntax}
%     \cs{box_if_horizontal_p:N} \meta{box}
%     \cs{box_if_horizontal:NTF} \meta{box} \Arg{true code} \Arg{false code}
%   \end{syntax}
%   Tests if \meta{box} is a horizontal box.
% \end{function}
%
% \begin{function}[EXP,pTF]{\box_if_vertical:N, \box_if_vertical:c}
%   \begin{syntax}
%     \cs{box_if_vertical_p:N} \meta{box}
%     \cs{box_if_vertical:NTF} \meta{box} \Arg{true code} \Arg{false code}
%   \end{syntax}
%   Tests if \meta{box} is a vertical box.
% \end{function}
%
% \section{The last box inserted}
%
% \begin{function}
%   {
%     \box_set_to_last:N,  \box_set_to_last:c,
%     \box_gset_to_last:N, \box_gset_to_last:c
%   }
%   \begin{syntax}
%     \cs{box_set_to_last:N} \meta{box}
%   \end{syntax}
%   Sets the \meta{box} equal to the last item (box) added to the current
%   partial list, removing the item from the list at the same time. When
%   applied to the main vertical list, the \meta{box} will always be void as
%   it is not possible to recover the last added item.
% \end{function}
%
% \section{Constant boxes}
%
% \begin{variable}{\c_empty_box}
%   This is a permanently empty box, which is neither set as horizontal
%   nor vertical.
% \end{variable}
%
% \section{Scratch boxes}
%
% \begin{variable}{\l_tmpa_box, \l_tmpb_box}
%   Scratch boxes for local assignment. These are never used by
%   the kernel code, and so are safe for use with any \LaTeX3-defined
%   function. However, they may be overwritten by other non-kernel
%   code and so should only be used for short-term storage.
% \end{variable}
%
% \begin{variable}{\g_tmpa_box, \g_tmpb_box}
%   Scratch boxes for global assignment. These are never used by
%   the kernel code, and so are safe for use with any \LaTeX3-defined
%   function. However, they may be overwritten by other non-kernel
%   code and so should only be used for short-term storage.
% \end{variable}
%
% \section{Viewing box contents}
%
% \begin{function}[updated = 2012-05-11]{\box_show:N, \box_show:c}
%   \begin{syntax}
%      \cs{box_show:N} \meta{box}
%   \end{syntax}
%   Shows full details of the content of the \meta{box} in the terminal.
% \end{function}
% 
% \begin{function}[added = 2012-05-11]{\box_show:Nnn, \box_show:cnn}
%   \begin{syntax}
%      \cs{box_show:Nnn} \meta{box} \meta{intexpr_1} \meta{intexpr_2}
%   \end{syntax}
%   Display the contents of \meta{box} in the terminal, showing the first
%   \meta{intexpr_1} items of the box, and descending into \meta{intexpr_2}
%   group levels.
% \end{function}
% 
% \begin{function}[added = 2012-05-11]{\box_log:N, \box_log:c}
%   \begin{syntax}
%      \cs{box_show:N} \meta{box}
%   \end{syntax}
%   Writes full details of the content of the \meta{box} to the log.
% \end{function}
% 
% \begin{function}[added = 2012-05-11]{\box_log:Nnn, \box_log:cnn}
%   \begin{syntax}
%      \cs{box_show:Nnn} \meta{box} \meta{intexpr_1} \meta{intexpr_2}
%   \end{syntax}
%   Writes the contents of \meta{box} to the log, showing the first
%   \meta{intexpr_1} items of the box, and descending into \meta{intexpr_2}
%   group levels.
% \end{function}
%
% \section{Horizontal mode boxes}
%
% \begin{function}{\hbox:n}
%   \begin{syntax}
%     \cs{hbox:n} \Arg{contents}
%   \end{syntax}
%   Typesets the \meta{contents} into a horizontal box of natural
%   width and then includes this box in the current list for typesetting.
%   \begin{texnote}
%     This is the \TeX{} primitive \tn{hbox}.
%   \end{texnote}
% \end{function}
%
% \begin{function}{\hbox_to_wd:nn}
%   \begin{syntax}
%     \cs{hbox_to_wd:nn} \Arg{dimexpr} \Arg{contents}
%   \end{syntax}
%   Typesets the \meta{contents} into a horizontal box of width
%   \meta{dimexpr} and then includes this box in the current list for
%   typesetting.
% \end{function}
%
% \begin{function}{\hbox_to_zero:n}
%   \begin{syntax}
%     \cs{hbox_to_zero:n} \Arg{contents}
%   \end{syntax}
%   Typesets the \meta{contents} into a horizontal box of zero width
%   and then includes this box in the current list for typesetting.
% \end{function}
%
% \begin{function}{\hbox_set:Nn, \hbox_set:cn, \hbox_gset:Nn, \hbox_gset:cn}
%   \begin{syntax}
%     \cs{hbox_set:Nn} \meta{box} \Arg{contents}
%   \end{syntax}
%   Typesets the \meta{contents} at natural width and then stores the
%   result inside the \meta{box}.
% \end{function}
%
% \begin{function}
%   {
%     \hbox_set_to_wd:Nnn,  \hbox_set_to_wd:cnn,
%     \hbox_gset_to_wd:Nnn, \hbox_gset_to_wd:cnn
%   }
%   \begin{syntax}
%     \cs{hbox_set_to_wd:Nnn} \meta{box} \Arg{dimexpr} \Arg{contents}
%   \end{syntax}
%   Typesets the \meta{contents} to the width given by the \meta{dimexpr}
%   and then stores the result inside the \meta{box}.
% \end{function}
%
% \begin{function}{\hbox_overlap_right:n}
%   \begin{syntax}
%     \cs{hbox_overlap_right:n} \Arg{contents}
%   \end{syntax}
%   Typesets the \meta{contents} into a horizontal box of zero width
%   such that material will protrude to the right of the insertion
%   point.
% \end{function}
%
% \begin{function}{\hbox_overlap_left:n}
%   \begin{syntax}
%     \cs{hbox_overlap_left:n} \Arg{contents}
%   \end{syntax}
%   Typesets the \meta{contents} into a horizontal box of zero width
%   such that material will protrude to the left of the insertion
%   point.
% \end{function}
%
% \begin{function}
%   {
%     \hbox_set:Nw, \hbox_set:cw,
%     \hbox_set_end:,
%     \hbox_gset:Nw, \hbox_gset:cw,
%     \hbox_gset_end:
%   }
%   \begin{syntax}
%     \cs{hbox_set:Nw} \meta{box} \meta{contents} \cs{hbox_set_end:}
%   \end{syntax}
%   Typesets the \meta{contents} at natural width and then stores the
%   result inside the \meta{box}. In contrast
%   to \cs{hbox_set:Nn} this function does not absorb the argument
%   when finding the \meta{content}, and so can be used in circumstances
%   where the \meta{content} may not be a simple argument.
% \end{function}
%
% \begin{function}{\hbox_unpack:N, \hbox_unpack:c}
%   \begin{syntax}
%     \cs{hbox_unpack:N} \meta{box}
%   \end{syntax}
%   Unpacks the content of the horizontal \meta{box}, retaining any stretching
%   or shrinking applied when the \meta{box} was set.
%   \begin{texnote}
%     This is the \TeX{} primitive \tn{unhcopy}.
%   \end{texnote}
% \end{function}
%
% \begin{function}{\hbox_unpack_clear:N, \hbox_unpack_clear:c}
%   \begin{syntax}
%     \cs{hbox_unpack_clear:N} \meta{box}
%   \end{syntax}
%   Unpacks the content of the horizontal \meta{box}, retaining any stretching
%   or shrinking applied when the \meta{box} was set. The \meta{box} is
%   then cleared globally.
%   \begin{texnote}
%     This is the \TeX{} primitive \tn{unhbox}.
%   \end{texnote}
% \end{function}
%
% \section{Vertical mode boxes}
%
% Vertical boxes inherit their baseline from their contents. The
% standard case is that the baseline of the box is at the same position
% as that of the last item added to the box. This means that the box
% will have no depth unless the last item added to it had depth. As a
% result most vertical boxes have a large height value and small or
% zero depth. The exception are |_top| boxes, where the reference point
% is that of the first item added. These tend to have a large depth and
% small height, although the latter will typically be non-zero.
%
% \begin{function}[updated = 2011-12-18]{\vbox:n}
%   \begin{syntax}
%     \cs{vbox:n} \Arg{contents}
%   \end{syntax}
%   Typesets the \meta{contents} into a vertical box of natural height
%   and includes this box in the current list for typesetting.
%   \begin{texnote}
%     This is the \TeX{} primitive \tn{vbox}.
%   \end{texnote}
% \end{function}
%
% \begin{function}[updated = 2011-12-18]{\vbox_top:n}
%   \begin{syntax}
%     \cs{vbox_top:n} \Arg{contents}
%   \end{syntax}
%   Typesets the \meta{contents} into a vertical box of natural height
%   and includes this box in the current list for typesetting. The
%   baseline of the box will tbe equal to that of the \emph{first}
%   item added to the box.
%   \begin{texnote}
%     This is the \TeX{} primitive \tn{vtop}.
%   \end{texnote}
% \end{function}
%
% \begin{function}[updated = 2011-12-18]{\vbox_to_ht:nn}
%   \begin{syntax}
%     \cs{vbox_to_ht:nn} \Arg{dimexpr} \Arg{contents}
%   \end{syntax}
%   Typesets the \meta{contents} into a vertical box of height
%   \meta{dimexpr} and then includes this box in the current list for
%   typesetting.
% \end{function}
%
% \begin{function}[updated = 2011-12-18]{\vbox_to_zero:n}
%   \begin{syntax}
%     \cs{vbox_to_zero:n} \Arg{contents}
%   \end{syntax}
%   Typesets the \meta{contents} into a vertical box of zero height
%   and then includes this box in the current list for typesetting.
% \end{function}
%
%  \begin{function}[updated = 2011-12-18]
%    {\vbox_set:Nn, \vbox_set:cn, \vbox_gset:Nn, \vbox_gset:cn}
%   \begin{syntax}
%     \cs{vbox_set:Nn} \meta{box} \Arg{contents}
%   \end{syntax}
%   Typesets the \meta{contents} at natural height and then stores the
%   result inside the \meta{box}.
% \end{function}
%
% \begin{function}[updated = 2011-12-18]
%   {\vbox_set_top:Nn, \vbox_set_top:cn, \vbox_gset_top:Nn, \vbox_gset_top:cn}
%   \begin{syntax}
%     \cs{vbox_set_top:Nn} \meta{box} \Arg{contents}
%   \end{syntax}
%   Typesets the \meta{contents} at natural height and then stores the
%   result inside the \meta{box}. The baseline of the box will tbe equal
%   to that of the \emph{first} item added to the box.
% \end{function}
%
% \begin{function}[updated = 2011-12-18]
%   {
%     \vbox_set_to_ht:Nnn,  \vbox_set_to_ht:cnn,
%     \vbox_gset_to_ht:Nnn, \vbox_gset_to_ht:cnn
%   }
%   \begin{syntax}
%     \cs{vbox_set_to_ht:Nnn} \meta{box} \Arg{dimexpr} \Arg{contents}
%   \end{syntax}
%   Typesets the \meta{contents} to the height given by the
%   \meta{dimexpr} and then stores the result inside the \meta{box}.
% \end{function}
%
% \begin{function}[updated = 2011-12-18]
%  {
%    \vbox_set:Nw, \vbox_set:cw,
%    \vbox_set_end:,
%    \vbox_gset:Nw, \vbox_gset:cw,
%    \vbox_gset_end:
%  }
%    \begin{syntax}
%      \cs{vbox_set:Nw} \meta{box} \meta{contents} \cs{vbox_set_end:}
%    \end{syntax}
%    Typesets the \meta{contents} at natural height and then stores the
%    result inside the \meta{box}. In contrast
%    to \cs{vbox_set:Nn} this function does not absorb the argument
%    when finding the \meta{content}, and so can be used in circumstances
%    where the \meta{content} may not be a simple argument.
%  \end{function}
%
% \begin{function}[updated = 2011-10-22]{\vbox_set_split_to_ht:NNn}
%   \begin{syntax}
%      \cs{vbox_set_split_to_ht:NNn} \meta{box_1} \meta{box_2} \Arg{dimexpr}
%   \end{syntax}
%   Sets \meta{box_1} to contain material to the height given by the
%   \meta{dimexpr} by removing content from the top of \meta{box_2}
%   (which must be a vertical box).
%   \begin{texnote}
%     This is the \TeX{} primitive \tn{vsplit}.
%   \end{texnote}
% \end{function}
%
% \begin{function}{\vbox_unpack:N, \vbox_unpack:c}
%   \begin{syntax}
%     \cs{vbox_unpack:N} \meta{box}
%   \end{syntax}
%   Unpacks the content of the vertical \meta{box}, retaining any stretching
%   or shrinking applied when the \meta{box} was set.
%   \begin{texnote}
%     This is the \TeX{} primitive \tn{unvcopy}.
%   \end{texnote}
% \end{function}
%
% \begin{function}{\vbox_unpack_clear:N, \vbox_unpack_clear:c}
%   \begin{syntax}
%     \cs{vbox_unpack:N} \meta{box}
%   \end{syntax}
%   Unpacks the content of the vertical \meta{box}, retaining any stretching
%   or shrinking applied when the \meta{box} was set. The \meta{box}
%   is then cleared globally.
%   \begin{texnote}
%     This is the \TeX{} primitive \tn{unvbox}.
%   \end{texnote}
% \end{function}
%
% \section{Primitive box conditionals}
%
% \begin{function}[EXP]{\if_hbox:N}
%   \begin{syntax}
%     \cs{if_hbox:N} \meta{box}
%     ~~\meta{true code}
%     \cs{else:}
%     ~~\meta{false code}
%     \cs{fi:}
%   \end{syntax}
%   Tests is \meta{box} is a horizontal box.
%   \begin{texnote}
%     This is the \TeX{} primitive \tn{ifhbox}.
%   \end{texnote}
% \end{function}
%
% \begin{function}[EXP]{\if_vbox:N}
%   \begin{syntax}
%     \cs{if_vbox:N} \meta{box}
%     ~~\meta{true code}
%     \cs{else:}
%     ~~\meta{false code}
%     \cs{fi:}
%   \end{syntax}
%   Tests is \meta{box} is a vertical box.
%   \begin{texnote}
%     This is the \TeX{} primitive \tn{ifvbox}.
%   \end{texnote}
% \end{function}
%
% \begin{function}[EXP]{\if_box_empty:N}
%   \begin{syntax}
%     \cs{if_box_empty:N} \meta{box}
%     ~~\meta{true code}
%     \cs{else:}
%     ~~\meta{false code}
%     \cs{fi:}
%   \end{syntax}
%   Tests is \meta{box} is an empty (void) box.
%   \begin{texnote}
%     This is the \TeX{} primitive \tn{ifvoid}.
%   \end{texnote}
% \end{function}
%
% \end{documentation}
%
% \begin{implementation}
%
% \section{\pkg{l3box} implementation}
%
%    \begin{macrocode}
%<*initex|package>
%    \end{macrocode}
%    
%    \begin{macrocode}
%<@@=box>
%    \end{macrocode}
%
%    \begin{macrocode}
%<*package>
\ProvidesExplPackage
  {\ExplFileName}{\ExplFileDate}{\ExplFileVersion}{\ExplFileDescription}
\__expl_package_check:
%</package>
%    \end{macrocode}
%
%  The code in this module is very straight forward so I'm not going to
%  comment it very extensively.
%
% \subsection{Creating and initialising boxes}
%
% \TestFiles{m3box001.lvt}
%
% \begin{macro}{\box_new:N,\box_new:c}
%   Defining a new \meta{box} register: remember that box $255$ is not
%   generally available.
%    \begin{macrocode}
%<*package>
\cs_new_protected:Npn \box_new:N #1
  {
    \__chk_if_free_cs:N #1
    \newbox #1
  }
%</package>
\cs_generate_variant:Nn \box_new:N { c }
%    \end{macrocode}
%
% \begin{macro}{\box_clear:N, \box_clear:c}
% \begin{macro}{\box_gclear:N, \box_gclear:c}
% \testfile*
%   Clear a \meta{box} register.
%    \begin{macrocode}
\cs_new_protected:Npn \box_clear:N #1
  { \box_set_eq:NN  #1 \c_empty_box }
\cs_new_protected:Npn \box_gclear:N #1
  { \box_gset_eq:NN #1 \c_empty_box }
\cs_generate_variant:Nn \box_clear:N  { c }
\cs_generate_variant:Nn \box_gclear:N { c }
%    \end{macrocode}
% \end{macro}
% \end{macro}
%
% \begin{macro}{\box_clear_new:N, \box_clear_new:c}
% \begin{macro}{\box_gclear_new:N, \box_gclear_new:c}
% \testfile*
%   Clear or new.
%    \begin{macrocode}
\cs_new_protected:Npn \box_clear_new:N #1
  { \box_if_exist:NTF #1 { \box_clear:N #1 } { \box_new:N #1 } }
\cs_new_protected:Npn \box_gclear_new:N #1
  { \box_if_exist:NTF #1 { \box_gclear:N #1 } { \box_new:N #1 } }
\cs_generate_variant:Nn \box_clear_new:N  { c }
\cs_generate_variant:Nn \box_gclear_new:N { c }
%    \end{macrocode}
% \end{macro}
% \end{macro}
%
%  \begin{macro}
%    {\box_set_eq:NN, \box_set_eq:cN, \box_set_eq:Nc, \box_set_eq:cc}
% \testfile*
%  \begin{macro}
%    {\box_gset_eq:NN, \box_gset_eq:cN, \box_gset_eq:Nc, \box_gset_eq:cc}
% \testfile*
%   Assigning the contents of a box to be another box.
%    \begin{macrocode}
\cs_new_protected:Npn \box_set_eq:NN #1#2
  { \tex_setbox:D #1 \tex_copy:D #2 }
\cs_new_protected:Npn \box_gset_eq:NN
  { \tex_global:D \box_set_eq:NN }
\cs_generate_variant:Nn \box_set_eq:NN  { c , Nc , cc }
\cs_generate_variant:Nn \box_gset_eq:NN { c , Nc , cc }
%    \end{macrocode}
%  \end{macro}
%  \end{macro}
%
% \begin{macro}
%   {
%     \box_set_eq_clear:NN, \box_set_eq_clear:cN,
%     \box_set_eq_clear:Nc, \box_set_eq_clear:cc
%   }
% \testfile*
% \begin{macro}
%   {
%     \box_gset_eq_clear:NN, \box_gset_eq_clear:cN,
%     \box_gset_eq_clear:Nc, \box_gset_eq_clear:cc
%   }
% \testfile*
%    Assigning the contents of a box to be another box.
%    This clears the second box globally (that's how \TeX{} does it).
%    \begin{macrocode}
\cs_new_protected:Npn \box_set_eq_clear:NN #1#2
  { \tex_setbox:D #1 \tex_box:D #2 }
\cs_new_protected:Npn \box_gset_eq_clear:NN
  { \tex_global:D  \box_set_eq_clear:NN }
\cs_generate_variant:Nn \box_set_eq_clear:NN  { c , Nc , cc }
\cs_generate_variant:Nn \box_gset_eq_clear:NN { c , Nc , cc }
%    \end{macrocode}
% \end{macro}
% \end{macro}
%
% \begin{macro}[pTF]{\box_if_exist:N, \box_if_exist:c}
%   Copies of the \texttt{cs} functions defined in \pkg{l3basics}.
%    \begin{macrocode}
\cs_new_eq:NN \box_if_exist:NTF \cs_if_exist:NTF
\cs_new_eq:NN \box_if_exist:NT  \cs_if_exist:NT
\cs_new_eq:NN \box_if_exist:NF  \cs_if_exist:NF
\cs_new_eq:NN \box_if_exist_p:N \cs_if_exist_p:N
\cs_new_eq:NN \box_if_exist:cTF \cs_if_exist:cTF
\cs_new_eq:NN \box_if_exist:cT  \cs_if_exist:cT
\cs_new_eq:NN \box_if_exist:cF  \cs_if_exist:cF
\cs_new_eq:NN \box_if_exist_p:c \cs_if_exist_p:c
%    \end{macrocode}
% \end{macro}
%
% \subsection{Measuring and setting box dimensions}
%
% \begin{macro}{\box_ht:N,\box_ht:c}
% \begin{macro}{\box_dp:N,\box_dp:c}
% \begin{macro}{\box_wd:N,\box_wd:c}
% \testfile*
%    Accessing the height, depth, and width of a \meta{box} register.
%    \begin{macrocode}
\cs_new_eq:NN \box_ht:N \tex_ht:D
\cs_new_eq:NN \box_dp:N \tex_dp:D
\cs_new_eq:NN \box_wd:N \tex_wd:D
\cs_generate_variant:Nn \box_ht:N { c }
\cs_generate_variant:Nn \box_dp:N { c }
\cs_generate_variant:Nn \box_wd:N { c }
%    \end{macrocode}
% \end{macro}
% \end{macro}
% \end{macro}
%
% \begin{macro}{\box_set_ht:Nn, \box_set_ht:cn}
% \begin{macro}{\box_set_dp:Nn, \box_set_dp:cn}
% \begin{macro}{\box_set_wd:Nn, \box_set_wd:cn}
%   Measuring is easy: all primitive work. These primitives are not
%   expandable, so the derived functions are not either.
%    \begin{macrocode}
\cs_new_protected:Npn \box_set_dp:Nn #1#2
  { \box_dp:N #1 \__dim_eval:w #2 \__dim_eval_end: }
\cs_new_protected:Npn \box_set_ht:Nn #1#2
  { \box_ht:N #1 \__dim_eval:w #2 \__dim_eval_end: }
\cs_new_protected:Npn \box_set_wd:Nn #1#2
  { \box_wd:N #1 \__dim_eval:w #2 \__dim_eval_end: }
\cs_generate_variant:Nn \box_set_ht:Nn { c }
\cs_generate_variant:Nn \box_set_dp:Nn { c }
\cs_generate_variant:Nn \box_set_wd:Nn { c }
%    \end{macrocode}
% \end{macro}
% \end{macro}
% \end{macro}
%
% \subsection{Using boxes}
%
% \begin{macro}{\box_use_clear:N, \box_use_clear:c}
% \begin{macro}{\box_use:N, \box_use:c}
%   Using a \meta{box}. These are just \TeX{} primitives with meaningful
%   names.
%    \begin{macrocode}
\cs_new_eq:NN \box_use_clear:N \tex_box:D
\cs_new_eq:NN \box_use:N \tex_copy:D
\cs_generate_variant:Nn \box_use_clear:N { c }
\cs_generate_variant:Nn \box_use:N { c }
%    \end{macrocode}
% \end{macro}
% \end{macro}
%
% \begin{macro}{\box_move_left:nn,\box_move_right:nn}
% \begin{macro}{\box_move_up:nn,\box_move_down:nn}
% \testfile*
%   Move box material in different directions.
%    \begin{macrocode}
\cs_new_protected:Npn \box_move_left:nn #1#2
  { \tex_moveleft:D \__dim_eval:w #1 \__dim_eval_end: #2 }
\cs_new_protected:Npn \box_move_right:nn #1#2
  { \tex_moveright:D \__dim_eval:w #1 \__dim_eval_end: #2 }
\cs_new_protected:Npn \box_move_up:nn #1#2
  { \tex_raise:D \__dim_eval:w #1 \__dim_eval_end: #2 }
\cs_new_protected:Npn \box_move_down:nn #1#2
  { \tex_lower:D \__dim_eval:w #1 \__dim_eval_end: #2 }
%    \end{macrocode}
% \end{macro}
% \end{macro}
%
% \subsection{Box conditionals}
%
% \begin{macro}{\if_hbox:N}
% \begin{macro}{\if_vbox:N}
% \begin{macro}{\if_box_empty:N}
%  \testfile*
%    The primitives for testing if a \meta{box} is empty/void or which
%    type of box it is.
%    \begin{macrocode}
\cs_new_eq:NN \if_hbox:N      \tex_ifhbox:D
\cs_new_eq:NN \if_vbox:N      \tex_ifvbox:D
\cs_new_eq:NN \if_box_empty:N \tex_ifvoid:D
%    \end{macrocode}
% \end{macro}
% \end{macro}
% \end{macro}
%
% \begin{macro}[pTF]{\box_if_horizontal:N,\box_if_horizontal:c}
% \testfile*
% \begin{macro}[pTF]{\box_if_vertical:N,\box_if_vertical:c}
% \testfile*
%    \begin{macrocode}
\prg_new_conditional:Npnn \box_if_horizontal:N #1 { p , T , F , TF }
  { \if_hbox:N #1 \prg_return_true: \else: \prg_return_false: \fi: }
\prg_new_conditional:Npnn \box_if_vertical:N #1 { p , T , F , TF }
  { \if_vbox:N #1 \prg_return_true: \else: \prg_return_false: \fi: }
\cs_generate_variant:Nn \box_if_horizontal_p:N { c }
\cs_generate_variant:Nn \box_if_horizontal:NT  { c }
\cs_generate_variant:Nn \box_if_horizontal:NF  { c }
\cs_generate_variant:Nn \box_if_horizontal:NTF { c }
\cs_generate_variant:Nn \box_if_vertical_p:N { c }
\cs_generate_variant:Nn \box_if_vertical:NT  { c }
\cs_generate_variant:Nn \box_if_vertical:NF  { c }
\cs_generate_variant:Nn \box_if_vertical:NTF { c }
%    \end{macrocode}
% \end{macro}
% \end{macro}
%
% \begin{macro}[pTF]{\box_if_empty:N, \box_if_empty:c}
% \testfile*
%   Testing if a \meta{box} is empty/void.
%    \begin{macrocode}
\prg_new_conditional:Npnn \box_if_empty:N #1 { p , T , F , TF }
  { \if_box_empty:N #1 \prg_return_true: \else: \prg_return_false: \fi: }
\cs_generate_variant:Nn \box_if_empty_p:N { c }
\cs_generate_variant:Nn \box_if_empty:NT  { c }
\cs_generate_variant:Nn \box_if_empty:NF  { c }
\cs_generate_variant:Nn \box_if_empty:NTF { c }
%    \end{macrocode}
%  \end{macro}
%  \end{macro}
%
% \subsection{The last box inserted}
%
% \begin{macro}{\box_set_to_last:N, \box_set_to_last:c}
% \begin{macro}{\box_gset_to_last:N, \box_gset_to_last:c}
% \testfile*
%    Set a box to the previous box.
%    \begin{macrocode}
\cs_new_protected:Npn \box_set_to_last:N #1
  { \tex_setbox:D #1 \tex_lastbox:D }
\cs_new_protected:Npn \box_gset_to_last:N
  { \tex_global:D \box_set_to_last:N }
\cs_generate_variant:Nn \box_set_to_last:N  { c }
\cs_generate_variant:Nn \box_gset_to_last:N { c }
%    \end{macrocode}
% \end{macro}
% \end{macro}
%
% \subsection{Constant boxes}
%
%  \begin{variable}{\c_empty_box}
%    \begin{macrocode}
%<*package>
\cs_new_eq:NN \c_empty_box \voidb@x
%</package>
%<*initex>
\box_new:N \c_empty_box
%</initex>
%    \end{macrocode}
% \end{variable}
%
% \subsection{Scratch boxes}
%
%  \begin{variable}{\l_tmpa_box, \l_tmpb_box, \g_tmpa_box, \g_tmpb_box}
%    Reuse the \LaTeXe{} scratch box in package mode.
%    \begin{macrocode}
%<*package>
\cs_new_eq:NN \l_tmpa_box \@tempboxa
%</package>
%<*initex>
\box_new:N \l_tmpa_box
%</initex>
\box_new:N \l_tmpb_box
\box_new:N \g_tmpa_box
\box_new:N \g_tmpb_box
%    \end{macrocode}
% \end{variable}
%
% \subsection{Viewing box contents}
% 
% \TeX{}'s \tn{tex_showbox:D} is not really that helpful in many cases, and
% it is also inconsistent with other \LaTeX3{} \texttt{show} functions as it
% does not actually shows material in the terminal. So we provide a richer
% set of functionality.
%
% \begin{macro}{\box_show:N, \box_show:c}
% \begin{macro}{\box_show:Nnn, \box_show:cnn}
%   Essentially a wrapper around the internal function.
%    \begin{macrocode}
\cs_new_protected:Npn \box_show:N #1
  { \box_show:Nnn #1 \c_max_int \c_max_int }
\cs_generate_variant:Nn \box_show:N { c }
\cs_new_protected_nopar:Npn \box_show:Nnn
  { \@@_show:NNnn \c_one }
\cs_generate_variant:Nn \box_show:Nnn { c }
%    \end{macrocode}
% \end{macro}
% \end{macro}
% 
% \begin{macro}{\box_log:N, \box_log:c}
% \begin{macro}{\box_log:Nnn, \box_log:cnn}
%   Getting \TeX{} to write to the log without interruption the run is done by
%   altering the interaction mode. For that, the \eTeX{} extensions are needed.
%    \begin{macrocode}
\cs_new_protected:Npn \box_log:N #1
  { \box_log:Nnn #1 \c_max_int \c_max_int }
\cs_generate_variant:Nn \box_log:N { c }
\cs_new_protected:Npn \box_log:Nnn #1#2#3
  {
    \use:x
      {
        \etex_interactionmode:D \c_zero
        \@@_show:NNnn \c_zero \exp_not:N #1
          { \int_eval:n {#2} } { \int_eval:n {#3} }
        \etex_interactionmode:D
            = \tex_the:D \etex_interactionmode:D \scan_stop:
      }
  }
\cs_generate_variant:Nn \box_log:Nnn { c }
%    \end{macrocode}
% \end{macro}
% \end{macro}
% 
% \begin{macro}[aux]{\@@_show:NNnn}
%   The internal auxiliary to actually do the output uses a group to deal
%   with breadth and depth values. The \cs{use:n} here gives better output
%   appearance. Setting \tn{tex_tracingonline:D} is used to control what
%   appears in the terminal.
%    \begin{macrocode}
\cs_new_protected:Npn \@@_show:NNnn #1#2#3#4
  {
    \group_begin:
      \int_set:Nn \tex_showboxbreadth:D {#3}
      \int_set:Nn \tex_showboxdepth:D   {#4}
      \int_set_eq:NN \tex_tracingonline:D #1
      \box_if_exist:NTF #2
        { \tex_showbox:D \use:n {#2} }
        {
          \__msg_kernel_error:nnx { kernel } { variable-not-defined }
            { \token_to_str:N #2 }
        }
    \group_end:
  }
%    \end{macrocode}
% \end{macro}
%
%  \subsection{Horizontal mode boxes}
%
% \begin{macro}{\hbox:n}
% \testfile{m3box002.lvt}
%   Put a horizontal box directly into the input stream.
%    \begin{macrocode}
\cs_new_protected:Npn \hbox:n { \tex_hbox:D \scan_stop: }
%    \end{macrocode}
%  \end{macro}
%
% \begin{macro}{\hbox_set:Nn,\hbox_set:cn}
% \begin{macro}{\hbox_gset:Nn,\hbox_gset:cn}
% \testfile*
%    \begin{macrocode}
\cs_new_protected:Npn \hbox_set:Nn #1#2 { \tex_setbox:D #1 \tex_hbox:D {#2} }
\cs_new_protected:Npn \hbox_gset:Nn { \tex_global:D \hbox_set:Nn }
\cs_generate_variant:Nn \hbox_set:Nn { c }
\cs_generate_variant:Nn \hbox_gset:Nn { c }
%    \end{macrocode}
%  \end{macro}
%  \end{macro}
%
% \begin{macro}{\hbox_set_to_wd:Nnn,\hbox_set_to_wd:cnn}
% \begin{macro}{\hbox_gset_to_wd:Nnn,\hbox_gset_to_wd:cnn}
% \testfile*
%   Storing material in a horizontal box with a specified width.
%    \begin{macrocode}
\cs_new_protected:Npn \hbox_set_to_wd:Nnn #1#2#3
  { \tex_setbox:D #1 \tex_hbox:D to \__dim_eval:w #2 \__dim_eval_end: {#3} }
\cs_new_protected:Npn \hbox_gset_to_wd:Nnn
  { \tex_global:D \hbox_set_to_wd:Nnn }
\cs_generate_variant:Nn \hbox_set_to_wd:Nnn { c }
\cs_generate_variant:Nn \hbox_gset_to_wd:Nnn { c }
%    \end{macrocode}
% \end{macro}
% \end{macro}
%
% \begin{macro}{\hbox_set:Nw, \hbox_set:cw}
% \begin{macro}{\hbox_gset:Nw, \hbox_gset:cw}
% \begin{macro}{\hbox_set_end:, \hbox_gset_end:}
% \testfile*
%    Storing material in a horizontal box. This type is useful in
%    environment definitions.
%    \begin{macrocode}
\cs_new_protected:Npn \hbox_set:Nw  #1
  { \tex_setbox:D #1 \tex_hbox:D \c_group_begin_token }
\cs_new_protected:Npn \hbox_gset:Nw
  { \tex_global:D \hbox_set:Nw }
\cs_generate_variant:Nn \hbox_set:Nw  { c }
\cs_generate_variant:Nn \hbox_gset:Nw { c }
\cs_new_eq:NN \hbox_set_end:  \c_group_end_token
\cs_new_eq:NN \hbox_gset_end: \c_group_end_token
%    \end{macrocode}
% \end{macro}
% \end{macro}
% \end{macro}
%
% \begin{macro}{\hbox_set_inline_begin:N, \hbox_set_inline_begin:c}
% \begin{macro}{\hbox_gset_inline_begin:N, \hbox_gset_inline_begin:c}
% \begin{macro}{\hbox_set_inline_end:,\hbox_gset_inline_end:}
% \testfile*
% Renamed September 2011.
%    \begin{macrocode}
\cs_new_eq:NN \hbox_set_inline_begin:N  \hbox_set:Nw
\cs_new_eq:NN \hbox_set_inline_begin:c  \hbox_set:cw
\cs_new_eq:NN \hbox_set_inline_end:     \hbox_set_end:
\cs_new_eq:NN \hbox_gset_inline_begin:N \hbox_gset:Nw
\cs_new_eq:NN \hbox_gset_inline_begin:c \hbox_gset:cw
\cs_new_eq:NN \hbox_gset_inline_end:    \hbox_gset_end:
%    \end{macrocode}
% \end{macro}
% \end{macro}
% \end{macro}
%
%  \begin{macro}{\hbox_to_wd:nn}
%  \begin{macro}{\hbox_to_zero:n}
%  \testfile*
%   Put a horizontal box directly into the input stream.
%    \begin{macrocode}
\cs_new_protected:Npn \hbox_to_wd:nn #1#2
   { \tex_hbox:D to \__dim_eval:w #1 \__dim_eval_end: {#2} }
\cs_new_protected:Npn \hbox_to_zero:n #1 { \tex_hbox:D to \c_zero_skip {#1} }
%    \end{macrocode}
%  \end{macro}
%  \end{macro}
%
% \begin{macro}{\hbox_overlap_left:n, \hbox_overlap_right:n}
%   Put a zero-sized box with the contents pushed against one side (which
%   makes it stick out on the other) directly into the input stream.
%    \begin{macrocode}
\cs_new_protected:Npn \hbox_overlap_left:n  #1
  { \hbox_to_zero:n { \tex_hss:D #1 } }
\cs_new_protected:Npn \hbox_overlap_right:n #1
  { \hbox_to_zero:n { #1 \tex_hss:D } }
%    \end{macrocode}
% \end{macro}
%
% \begin{macro}{\hbox_unpack:N, \hbox_unpack:c}
% \begin{macro}{\hbox_unpack_clear:N, \hbox_unpack_clear:c}
% \testfile*
%   Unpacking a box and if requested also clear it.
%    \begin{macrocode}
\cs_new_eq:NN \hbox_unpack:N \tex_unhcopy:D
\cs_new_eq:NN \hbox_unpack_clear:N \tex_unhbox:D
\cs_generate_variant:Nn \hbox_unpack:N { c }
\cs_generate_variant:Nn \hbox_unpack_clear:N { c }
%    \end{macrocode}
% \end{macro}
% \end{macro}
%
% \subsection{Vertical mode boxes}
%
% \TeX{} ends these boxes directly with the internal \emph{end_graf}
% routine. This means that there is no \cs{par} at the end of vertical
% boxes unless we insert one.
%
% \begin{macro}{\vbox:n}
% \TestFiles{m3box003.lvt}
% \begin{macro}{\vbox_top:n}
% \TestFiles{m3box003.lvt}
%   Put a vertical box directly into the input stream.
%    \begin{macrocode}
\cs_new_protected:Npn \vbox:n #1     { \tex_vbox:D { #1 \par } }
\cs_new_protected:Npn \vbox_top:n #1 { \tex_vtop:D { #1 \par } }
%    \end{macrocode}
% \end{macro}
% \end{macro}
%
% \begin{macro}{\vbox_to_ht:nn,\vbox_to_zero:n}
% \begin{macro}{\vbox_to_ht:nn,\vbox_to_zero:n}
% \testfile*
%   Put a vertical box directly into the input stream.
%    \begin{macrocode}
\cs_new_protected:Npn \vbox_to_ht:nn #1#2
  { \tex_vbox:D to \__dim_eval:w #1 \__dim_eval_end: { #2 \par } }
\cs_new_protected:Npn \vbox_to_zero:n #1
  { \tex_vbox:D to \c_zero_dim { #1 \par } }
%    \end{macrocode}
% \end{macro}
% \end{macro}
%
% \begin{macro}{\vbox_set:Nn, \vbox_set:cn}
% \begin{macro}{\vbox_gset:Nn, \vbox_gset:cn}
% \testfile*
%   Storing material in a vertical box with a natural height.
%    \begin{macrocode}
\cs_new_protected:Npn \vbox_set:Nn #1#2
  { \tex_setbox:D #1 \tex_vbox:D { #2 \par } }
\cs_new_protected:Npn \vbox_gset:Nn  { \tex_global:D \vbox_set:Nn }
\cs_generate_variant:Nn \vbox_set:Nn  { c }
\cs_generate_variant:Nn \vbox_gset:Nn { c }
%    \end{macrocode}
% \end{macro}
% \end{macro}
%
% \begin{macro}{\vbox_set_top:Nn, \vbox_set_top:cn}
% \begin{macro}{\vbox_gset_top:Nn, \vbox_gset_top:cn}
% \testfile*
%   Storing material in a vertical box with a natural height and reference
%   point at the baseline of the first object in the box.
%    \begin{macrocode}
\cs_new_protected:Npn \vbox_set_top:Nn #1#2
  { \tex_setbox:D #1 \tex_vtop:D { #2 \par } }
\cs_new_protected:Npn \vbox_gset_top:Nn
  { \tex_global:D \vbox_set_top:Nn }
\cs_generate_variant:Nn \vbox_set_top:Nn { c }
\cs_generate_variant:Nn \vbox_gset_top:Nn { c }
%    \end{macrocode}
% \end{macro}
% \end{macro}
%
% \begin{macro}{\vbox_set_to_ht:Nnn,\vbox_set_to_ht:cnn}
% \begin{macro}{\vbox_gset_to_ht:Nnn,\vbox_gset_to_ht:cnn}
%  \testfile*
%  Storing material in a vertical box with a specified height.
%    \begin{macrocode}
\cs_new_protected:Npn \vbox_set_to_ht:Nnn #1#2#3
  { \tex_setbox:D #1 \tex_vbox:D to \__dim_eval:w #2 \__dim_eval_end: { #3 \par } }
\cs_new_protected:Npn \vbox_gset_to_ht:Nnn
  { \tex_global:D \vbox_set_to_ht:Nnn }
\cs_generate_variant:Nn \vbox_set_to_ht:Nnn  { c }
\cs_generate_variant:Nn \vbox_gset_to_ht:Nnn { c }
%    \end{macrocode}
% \end{macro}
% \end{macro}
%
% \begin{macro}{\vbox_set:Nw, \vbox_set:cw}
% \begin{macro}{\vbox_gset:Nw, \vbox_gset:cw}
% \begin{macro}{\vbox_set_end:, \vbox_gset_end:}
% \testfile*
%   Storing material in a vertical box. This type is useful in
%   environment definitions.
%    \begin{macrocode}
\cs_new_protected:Npn \vbox_set:Nw #1
  { \tex_setbox:D #1 \tex_vbox:D \c_group_begin_token }
\cs_new_protected:Npn \vbox_gset:Nw
  { \tex_global:D \vbox_set:Nw }
\cs_generate_variant:Nn \vbox_set:Nw  { c }
\cs_generate_variant:Nn \vbox_gset:Nw { c }
\cs_new_protected:Npn \vbox_set_end:
  {
    \par
    \c_group_end_token
  }
\cs_new_eq:NN \vbox_gset_end: \vbox_set_end:
%    \end{macrocode}
% \end{macro}
% \end{macro}
% \end{macro}
%
% \begin{macro}{\vbox_set_inline_begin:N,  \vbox_set_inline_begin:c}
% \begin{macro}{\vbox_gset_inline_begin:N, \vbox_gset_inline_begin:c}
% \begin{macro}{\vbox_set_inline_end:}
% \begin{macro}{\vbox_gset_inline_end:}
% \testfile*
% Renamed September 2011.
%    \begin{macrocode}
\cs_new_eq:NN \vbox_set_inline_begin:N  \vbox_set:Nw
\cs_new_eq:NN \vbox_set_inline_begin:c  \vbox_set:cw
\cs_new_eq:NN \vbox_set_inline_end:  \vbox_set_end:
\cs_new_eq:NN \vbox_gset_inline_begin:N \vbox_gset:Nw
\cs_new_eq:NN \vbox_gset_inline_begin:c \vbox_gset:cw
\cs_new_eq:NN \vbox_gset_inline_end: \vbox_gset_end:
%    \end{macrocode}
% \end{macro}
% \end{macro}
% \end{macro}
% \end{macro}
%
% \begin{macro}{\vbox_unpack:N, \vbox_unpack:c}
% \begin{macro}{\vbox_unpack_clear:N, \vbox_unpack_clear:c}
% \testfile*
%   Unpacking a box and if requested also clear it.
%    \begin{macrocode}
\cs_new_eq:NN \vbox_unpack:N \tex_unvcopy:D
\cs_new_eq:NN \vbox_unpack_clear:N \tex_unvbox:D
\cs_generate_variant:Nn \vbox_unpack:N { c }
\cs_generate_variant:Nn \vbox_unpack_clear:N { c }
%    \end{macrocode}
% \end{macro}
% \end{macro}
%
% \begin{macro}{\vbox_set_split_to_ht:NNn}
% \testfile*
%   Splitting a vertical box in two.
%    \begin{macrocode}
\cs_new_protected:Npn \vbox_set_split_to_ht:NNn #1#2#3
  { \tex_setbox:D #1 \tex_vsplit:D #2 to \__dim_eval:w #3 \__dim_eval_end: }
%    \end{macrocode}
% \end{macro}
%
% \subsection{Deprecated functions}
%
% \begin{variable}{\l_last_box}
%   Deprecated 2011-11-13, for removal by 2012-02-28.
%    \begin{macrocode}
%<*deprecated>
\cs_new_eq:NN \l_last_box \tex_lastbox:D
%</deprecated>
%    \end{macrocode}
% \end{variable}
%
%    \begin{macrocode}
%</initex|package>
%    \end{macrocode}
%
% \end{implementation}
%
% \PrintIndex
