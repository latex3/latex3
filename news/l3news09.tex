% Copyright 2014 The LaTeX3 Project
\documentclass{ltnews}

\publicationmonth{February}
\publicationyear{2014}
\publicationissue{9}

\usepackage{multicol}
\usepackage{xparse}
\newcommand\ctanpkg[1]{\href{http://ctan/tug.org/pkg/#1}{\pkg{#1}}}

\begin{document}
\maketitle

\raisefirstsection
\section{Hiatus?}

Well, it's been a busy couple of years.
Work has slowed down on the LaTeX3 codebase as all active members of the team have been---shall we say---pre-occupied with more pressing concerns in their day to day activities.

Nonetheless, Joseph and Bruno have continue to fine-tune the LaTeX3 kernel and add-on packages.
Browsing through the commit history shows bug fixes and improvements to documentation, test files, and internal code across the entire breadth of the codebase.

Members of the team have presented at two TUG conferences since the last LaTeX3 news. (Has it really been so long?)
In November 2012 (? check), Frank and Will travelled to Boston and discussed respectively: the challenges faced in the past and continuing to the present day due to the limits of the various \TeX\ engines; and, a brief history and recent developments of the \pkg{expl3} code.
In 2013, Joseph and Frank wrote a talk on complex layouts, and the `layers' ideas discussed in LaTeX3; Frank was present in Tokyo in June (? check) to present the work.
Slides of and recordings from these talks are available as always on the \LaTeX3 website.

These conferences are good opportunities to introduce the \pkg{expl3} language to a wider group of people; in many cases, the rationale behind why \pkg{expl3} looks a little strange at first helps to convince the our audience that it's not so weird after all.
In our experience, anyone that's been exposed to some of the more awkward expansion aspects of \TeX\ programming appreciates how \pkg{expl3} makes life much easier for us.

\section{Use in the community}

More and more people are adopting \pkg{expl3} for their own use.
A search on the \TeX\ Stack Exchange website for either `\texttt{expl3}' or `\texttt{latex3}' at time of writing yield around one thousand results each.

In order to help standardise the prefixes used in \pkg{expl3} modules, we have developed a registration procedure for package authors (which amounts to little more than notififying us that their package uses a specifc prefix, which will often be the name of the package itself).

Some thirty package prefixes are registered by fifteen separate individuals (unrelated to the LaTeX3 project---the number grows if you include packages by members of the team).
These packages cover a broad range of \LaTeX\ functionality:
\begin{description}
\item[\ctanpkg{acro}] Interface for creating (classes of) acronyms
\item[\ctanpkg{hobby}]
Hobby's algorithm in PGF/TiKZ for drawing optimally smooth curves.
\item[\ctanpkg{chemmacros}] Typesetting in the field of chemistry.
\item[\ctanpkg{classics}] Traditional-style citations for the classics. 
\item[\ctanpkg{conteq}] Continued (in)equalities in mathematics.
\item[\ctanpkg{endiagram}] Draw­ po­ten­tial en­ergy curve di­a­grams.
\item[\ctanpkg{enotez}] Support for end-notes.
\item[\ctanpkg{exsheets}] Question sheets and exams with  metadata.
%(Note to self: remember to use this for teaching this semester!)
\item[\ctanpkg{lt3graph}] A graph datastructure.
\item[\ctanpkg{newlfm}] The venerable class for memos and letters.
\item[\ctanpkg{fnpct}] Interaction between footnotes and punctuation.
\item[\ctanpkg{GS1}] Barcodes and so forth.
\item[\ctanpkg{hobete}] Beamer theme for the Univ.\ of Hohenheim.
\item[\ctanpkg{kantlipsum}] Generate sentences in Kant's style.
\item[\ctanpkg{lualatex-math}] Extended support for mathematics in \LuaLaTeX.
\item[\ctanpkg{media9}] Multimedia inclusion for Adobe Reader.
\item[\ctanpkg{pkgloader}] Managing the options and loading order of other packages.
\item[\ctanpkg{substances}] Lists of chemicals, etc., in a document.
\item[\ctanpkg{withargs}] Ephemeral macro use.
\item[\ctanpkg{xecjk}] Support for CJK documents in \XeLaTeX.
\item[\ctanpkg{xpatch}, \ctanpkg{regexpatch}] Patch command definitions.
\item[\ctanpkg{xpeek}]  Commands that peek ahead in the input stream.
\item[\ctanpkg{xpinjin}] Automatically add pinyin to Chinese characters
\item[\ctanpkg{zhnumber}] Typeset Chinese representations of numbers
\item[\ctanpkg{zxjatype}] Standards-conforming typesetting of Japanese for \XeLaTeX.
\end{description}
Some of these packages are marked by their authors as experimental, but it is clear that these packages have been developed to solve specific needs for typesetting and document production.

The \pkg{expl3} language has well and truly gained traction after many years of lying dormant.

\section{Current activity}

While changes have tended to be minor in recent times, there are a number of improvements worth discussing explicity.
\begin{enumerate}
\item
Improvements to FPU to cover, e.g., inverse trig and thus make it
  viable for use in most cases needing floating points.
\item
Revision of galley code to allow separation of pargraph shapes from
  margins/cutouts (still needs testing properly).
\item
Work on 'native' drivers which improves robustness (I think ready for
  use by default by \pkg{expl3}) and also feeds back to more 'general' LaTeX
  code.
\end{enumerate}


%\section{Book}
%Breaking news to add is that Enrico Gregorio is working on a 'Programming LaTeX3' book. That should complement/enhance my rather short blog series on the topic, and I hope give us real traction.

\section{Work in progress}

We're still actively discussing a variety of areas to tackle next.
We are aware of various 'odds and ends' in \pkg{expl3} that still need sorting out.
In particular, some experimental functions have been working quite well and it's time to assess moving them into the `stable' modules, in particular the \pkg{l3str} module for dealing with catcode-twelve token lists.

\section{Uppercasing and lowercasing}

The commands \verb"\tl_to_lowercase:n" and \verb"\tl_to_uppercase:n" have long been overdue for good hard look.
From a traditional \TeX\ viewpoint, this command is simply the primitive \verb"\lowercase" and in practice it's well known that there are various limitations and pecularities associated with it.
We know these commands are good for three things:
\begin{enumerate}
\item
Uppercasing text for typesetting purposes such as all-uppercase titles.
\item
Lowercasing text for normalisation in sorting and other applications such as filename comparisons.
\item
Manipulating \verb"\uccode" and the like to achieve special effects such as defining commands that contain characters with different catcodes than usual.
\end{enumerate}
We are working on providing a set of commands to achieve all three of these functions in a more direct and easy-to-use fashion, including support for Unicode in \LuaLaTeX\ and \XeLaTeX.

\section{Space-skipping in \pkg{xparse}}

We are also re-assessing the behaviour of space-skipping in \pkg{xparse}.
Consider the following example:
\begin{multicols}{2}
\begin{verbatim}
\begin{dmath}
[x y z] = [1 2 3]
\end{dmath}

\begin{dmath}[label=foo]
x^2 + y^2 = z^2
\end{dmath}
\end{verbatim}
\end{multicols}
In the first case, we are typsetting some mathematics that contains square backets.
In the second, we are assigning a label to the equation using an optional argument, which also uses  brackets.
The fact that both work correctly is due to behaviour that is specifically programmed into the workings of the \texttt{dmath} environment of \pkg{breqn}: spaces before an optional argument are explicitly forbidden.
At present, this is also how environments (and commands) defined using \pkg{xparse} behave.
But consider a \pkg{pgfplots} environment:
\begin{verbatim}
\begin{pgfplot}
  [
    % plot options
  ]
  \begin{axis}
    [
      % axis options
    ]
    ...
  \end{axis}
\end{pgfplot}
\end{verbatim}
This would seem like quite a natural way to write such environments, but with the current state of \pkg{xparse} this syntax would be incorrect. One would have to write either of these instead:
\begin{multicols}{2}
\begin{verbatim}
\begin{pgfplot}%
  [
   % plot options
  ]
\end{verbatim}
\begin{verbatim}
\begin{pgfplot}[
   % plot options
  ]
\end{verbatim}
\end{multicols}
Is this an acceptable compromise?
We're not entirely sure here---we're in a corner because the humble \texttt{[} has ended up being both part of the syntax and semantics of a \LaTeX\ document.
As a result, we may consider adding a further option to \pkg{xparse} to define the space-skipping behaviour as desired by a package author.
If an environment is to be used for mathematics or in other places that square brackets are common, spaces should not be skipped and this should be denoted in the definition of the command.
Otherwise, the behaviour should default to the current more liberal approach of ignoring spaces before optional arguments.


\section{Until next time\dots}

\dots without such a long time between updates, we hope!

\end{document}



