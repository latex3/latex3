% Copyright 2014 The LaTeX3 Project
\documentclass{ltnews}

\publicationmonth{February}
\publicationyear{2014}
\publicationissue{9}

\usepackage{multicol}
\usepackage{xparse}
\newcommand\ctanpkg[1]{\href{http://ctan/tug.org/pkg/#1}{\pkg{#1}}}

\begin{document}
\maketitle

\raisefirstsection
\section{Hiatus?}

Well, it's been a busy couple of years.
Work has slowed down on the LaTeX3 codebase as all active members of the team have been---shall we say---pre-occupied with more pressing concerns in their day to day activities.

Nonetheless, Joseph and Bruno have continue to fine-tune the LaTeX3 kernel and add-on packages.
Browsing through the commit history shows bug fixes and improvements to documentation, test files, and internal code across the entire breadth of the codebase.

Members of the team have presented at two TUG conferences since the last LaTeX3 news.
In 2012, Frank and Will travelled to Boston and discussed (respectively): the challenges faced in the past and continuing to the present day due to the limits of the various \TeX\ engines; and, a brief history and recent developments of the expl3 code.
In 2013, Joseph and Frank wrote a talk on complex layouts, and the `layers' ideas discussed in LaTeX3; Frank was present in Tokyo to present the work.

These conferences are good opportunities to introduce the expl3 language to a wider group of people; in many cases, the rationale behind why expl3 looks a little strange at first helps to convince the our audience that it's not so weird after all.
In our experience, anyone that's been exposed to some of the more awkward expansion aspects of \TeX\ programming appreciates how expl3 makes life much easier for us.

\section{Use in the community}

At the same time, more and more people appear to be adopting expl3 for their own use.
Some 30 package prefixes are registered by 15 separate individuals (unrelated to the LaTeX3 project).
These packages cover a broad range of \LaTeX\ functionality:
\begin{description}
\item[\ctanpkg{acro}] Interface for creating (classes of) acronyms
\item[\ctanpkg{hobby}]
Hobby's algorithm in PGF/TiKZ for drawing optimally smooth curves.
\item[\ctanpkg{chemmacros}] Typesetting in the field of chemistry.
\item[\ctanpkg{classics}] Traditional-style citations for the classics. 
\item[\ctanpkg{conteq}] Continued (in)equalities in mathematics.
\item[\ctanpkg{endiagram}] Draw­ po­ten­tial en­ergy curve di­a­grams.
\item[\ctanpkg{enotez}] Support for end-notes.
\item[\ctanpkg{exsheets}] Question sheets and exams with solutions and metadata. (Note to self: remember to use this for teaching this semester!)
\item[\ctanpkg{newlfm}] The venerable class for memos and letters (and faxes, if that's applicable in this decade).
\item[\ctanpkg{fnpct}] Interaction between footnotes and punctuation.
\item[\ctanpkg{GS1}] Barcodes and so forth.
\item[\ctanpkg{hobete}] Beamer theme for the University of Hohenheim.
\item[\ctanpkg{kantlipsum}] Generate sentences in Kant's style.
\item[\ctanpkg{lualatex-math}] Advanced support for mathematics in \LuaLaTeX.
\item[\ctanpkg{media9}] Multimedia inclusion in PDF for Adobe Reader.
\item[\ctanpkg{substances}] Lists of chemicals, etc., in a document.
\item[\ctanpkg{xecjk}] Support for CJK documents in \XeLaTeX.
\item[\ctanpkg{xpatch}, \ctanpkg{regexpatch}] Patch command definitions.
\item[\ctanpkg{xpeek}] Define commands that peek ahead in the input stream.
\item[\ctanpkg{xpinjin}] Automatically add pinyin to Chinese characters
\item[\ctanpkg{zhnumber}] Typeset Chinese representations of numbers
\item[\ctanpkg{zxjatype}] Standard-conforming typesetting of Japanese for \XeLaTeX.
\end{description}

The expl3 language has well and truly gained traction after many years of lying dormant.


\section{Current activity}

While changes have tended to be minor in recent times, there are a number of improvements worth discussing explicity.

\begin{enumerate}
\item
Improvements to FPU to cover, e.g., inverse trig and thus make it
  viable for use in most cases needing floating points.
\item
Revision of galley code to allow separation of pargraph shapes from
  margins/cutouts (still needs testing properly).
\item
Work on 'native' drivers which improves robustness (I think ready for
  use by default by expl3) and also feeds back to more 'general' LaTeX
  code.
\end{enumerate}


%\section{Book}
%Breaking news to add is that Enrico Gregorio is working on a 'Programming LaTeX3' book. That should complement/enhance my rather short blog series on the topic, and I hope give us real traction.

\section{Work in progress}

We're still actively discussing a variety of areas to tackle next.
We are aware of various 'odds and ends' in expl3 that still need sorting out.
In particular, some experimental functions have been working quite well and it's time to assess moving them into the `stable' modules.

\subsection{Uppercasing and lowercasing}

The commands \verb"\tl_to_lowercase:n" and \verb"\tl_to_uppercase:n" have long been overdue for good hard look.
From a traditional \TeX\ viewpoint, this command is simply the primitive \verb"\lowercase" and in practice it's well known that there are various limitations and pecularities associated with it.
We know these commands are good for three things:
\begin{enumerate}
\item
Uppercasing text for typesetting purposes such as all-uppercase titles.
\item
Lowercasing text for normalisation in sorting and other applications such as filename comparisons.
\item
Manipulating \verb"\uccode" and the like to achieve special effects such as defining commands that contain characters with different catcodes than usual.
\end{enumerate}
We are working on providing a set of commands to achieve all three of these functions in a more direct and easy-to-use fashion.

\subsection{\pkg{xparse} and space-skipping}

We are also re-assessing the behaviour of space-skipping in \pkg{xparse}.
Consider the following example:
\begin{multicols}{2}
\begin{verbatim}
\begin{dmath}
[x y z] = [1 2 3]
\end{dmath}

\begin{dmath}[label=foo]
x^2 + y^2 = z^2
\end{dmath}
\end{verbatim}
\end{multicols}


\section{How can \LaTeX\ learn from CSS?}



\end{document}



