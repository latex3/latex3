% Copyright 2009 The LaTeX Project
\documentclass{ltnews}
\renewcommand{\LaTeXNews}{\LaTeX3~News}

\usepackage{url}

\publicationmonth{December}
\publicationyear{2009}
\publicationissue{3}

\begin{document}
\maketitle
\raisefirstsection

\section{Happy holidays}

Welcome to the holiday season edition of `news of our activities' for the \LaTeX3 team.

\section{Recent developments}

The last six months has seen two significant releases in the \LaTeX3 code.
In the \textsc{ctan} repository for the \pkg{xpackages},\footnote{\url{http://www.ctan.org/tex-archive/macros/latex/contrib/xpackages/}} you'll find two items of interest:
\begin{itemize}
\item A revised version of \textsf{xparse}; and
\item The new package \textsf{xtemplate}, a re-implementation of \textsf{template} with a new syntax.
\end{itemize}
Special thanks to Joseph Wright who handled the implementations above almost single-handedly (with lots of input and feedback from other members of the team and members of the \textsc{latex-l} mailing list).

These two packages are designed for the \LaTeX\ package author who wishes to define document commands and designer interfaces in a high-level manner.

\paragraph{\textsf{xparse}}
This package allows complex document commands to be constructed with all sorts of optional arguments and flags. Think of how \verb|\newcommand| allows you to create a command with a single optional argument and \textsf{xparse} is a generalisation of that idea.

\paragraph{\textsf{xtemplate}}
This package requires some more words to explain. The idea behind \textsf{xtemplate} is to cleanly separate the logical information in a document and its visual representation. `Templates' are constructed to fulfill individual typesetting requirements for a specific set of arguments; to change the look of a certain part of a document, the template can be swapped out for another without (a) having to change the markup of the source document, or (b) having to edit some internal \LaTeX\ macro.

For example, if you want to change a footnote from looking like {\footnotesize\fbox{\quad$\smash{{}^1}$\,Footnote text}} to {\footnotesize\fbox{1.\enspace Footnote text}}, rather than having to find and re-define \verb|\@makefntext| (an activity which, while \textsf{footmisc} can often help, is never fun) if the footnotes are defined in terms of templates it's just a matter of switching from one template to another.

This is a slight gloss over the complexities of the package itself, which you can read about in the documentation. We've tried to document \textsf{xtemplate} clearly but we'd love feedback on whether the ideas make sense to you.

As an addendum to the introduction of \textsf{xtemplate}, the older \textsf{template} package will be retired in the near future. To our knowledge there is only a single package on \textsc{ctan} that uses \textsf{template}, namely \textsf{xfrac}, and members of the \LaTeX\ team are in the process of switching this package over to \textsf{xtemplate}. If you have any private code that uses \textsf{template}, please let us know!


\section{Upcoming plans}

Having announced the updated \textsf{xparse} and the new \textsf{xtemplate}, the next stage of development will revolve around using these two systems in the other components of the \textsf{xpackages}, feeding back our experience in practise with the original ideas behind the designs and evaluating if the packages are meeting our expectations.

\paragraph{\textsf{xhead}}
The first work along these lines will be a new \textsf{xpackage} called (probably) \textsf{xhead}, for typesetting section headings and other document divisions.
Section headings are one of the more complex areas to work with, so this should stress \textsf{xtemplate} enough to know if its current design is sufficient for most needs.
Nothing has been released yet, but we'll announce further developments on the \textsc{latex-l} mailing list\footnote{\url{http://www.latex-project.org/code.html}} in the mean time.

\paragraph{\textsf{galley}}
We shall also endeavor to perform the same treatment on \textsf{galley} as has been done with \textsf{xparse} and \textsf{xtemplate}. That is, we have an older implementation that needs some work before we're ready to release it to \textsc{ctan}.

The \textsf{galley} package is used to place material into the vertical list while typesetting but before page breaks occur. At such a low-level, it is important to solidify this package before higher level design templates can be written. One issue we currently face is that to achieve best results, it cannot be used in concert with older \LaTeXe\ code. This could limit its usefulness, so we may decide that it's better to scale back the features we're attempting to allow better interoperability for existing packages and documents. More work remains before we can decide between these options.


\end{document}



