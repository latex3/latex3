% \iffalse meta-comment
%
%% File: l3regex.dtx Copyright (C) 2011-2012 The LaTeX3 Project
%%
%% It may be distributed and/or modified under the conditions of the
%% LaTeX Project Public License (LPPL), either version 1.3c of this
%% license or (at your option) any later version.  The latest version
%% of this license is in the file
%%
%%    http://www.latex-project.org/lppl.txt
%%
%% This file is part of the "l3experimental bundle" (The Work in LPPL)
%% and all files in that bundle must be distributed together.
%%
%% The released version of this bundle is available from CTAN.
%%
%% -----------------------------------------------------------------------
%%
%% The development version of the bundle can be found at
%%
%%    http://www.latex-project.org/svnroot/experimental/trunk/
%%
%% for those people who are interested.
%%
%%%%%%%%%%%
%% NOTE: %%
%%%%%%%%%%%
%%
%%   Snapshots taken from the repository represent work in progress and may
%%   not work or may contain conflicting material!  We therefore ask
%%   people _not_ to put them into distributions, archives, etc. without
%%   prior consultation with the LaTeX3 Project.
%%
%% -----------------------------------------------------------------------
%
%<*driver|package>
\RequirePackage{expl3}
\GetIdInfo$Id$
  {L3 Experimental Regular Expressions}
%</driver|package>
%<*driver>
\documentclass[full]{l3doc}
\usepackage{amsmath}
\begin{document}
  \DocInput{\jobname.dtx}
\end{document}
%</driver>
% \fi
%
% \title{^^A
%   The \textsf{l3regex} package: regular expressions in \TeX{}^^A
%   \thanks{This file describes v\ExplFileVersion,
%     last revised \ExplFileDate.}^^A
% }
%
% \author{^^A
%  The \LaTeX3 Project\thanks
%    {^^A
%      E-mail:
%        \href{mailto:latex-team@latex-project.org}
%          {latex-team@latex-project.org}^^A
%    }^^A
% }
%
% \date{Released \ExplFileDate}
%
% \maketitle
%
% \begin{documentation}
% \newenvironment{l3regex-syntax}
%   {\begin{itemize}\def\\{\char`\\}\def\makelabel##1{\hss\llap{\ttfamily##1}}}
%   {\end{itemize}}
%
% \section{\pkg{l3regex} documentation}
%
% The \pkg{l3regex} package provides regular expression testing,
% extraction of submatches, splitting, and replacement, all acting
% on token lists. The syntax of regular expressions is mostly a subset
% of the \textsc{pcre} syntax (and very close to \textsc{posix}),
% with some additions
% due to the fact that \TeX{} manipulates tokens rather than characters.
% For performance reasons, only a limited set of features are implemented.
% Notably, back-references are not supported.
%
% Let us give a few examples. After
% \begin{verbatim}
%   \tl_set:Nn \l_my_tl { That~cat. }
%   \regex_replace_once:nnN { at } { is } \l_my_tl
% \end{verbatim}
% the token list variable \cs{l_my_tl} holds the text
% \enquote{\texttt{This cat.}}, where the first
% occurrence of \enquote{\texttt{at}} was replaced
% by \enquote{\texttt{is}}. A more complicated example is
% a pattern to add a comma at the end of each word:
% \begin{verbatim}
%   \regex_replace_all:nnN { \w+ } { \0 , } \l_my_tl
% \end{verbatim}
% The |\w| sequence represents any \enquote{word} character,
% and |+| indicates that the |\w| sequence should be repeated
% as many times as possible (at least once), hence matching a word in the
% input token list. In the replacement text, |\0| denotes the full match
% (here, a word).
%
% If a regular expression is to be used several times,
% it can be compiled once, and stored in a regex
% variable using \cs{regex_const:Nn}. For example,
% \begin{verbatim}
%    \regex_const:Nn \c_foo_regex { \c{begin} \cB. (\c[^BE].*) \cE. }
% \end{verbatim}
% stores in \cs{c_foo_regex} a regular expression which matches the
% starting marker for an environment: \cs{begin}, followed by a
% begin-group token (|\cB.|), then any number of tokens which are
% neither begin-group nor end-group character tokens (|\c[^BE].*|),
% ending with an end-group token (|\cE.|). As explained in the next
% section, the parentheses \enquote{capture} the result of |\c[^BE].*|,
% giving us access to the name of the environment when doing
% replacements.
%
% \subsection{Syntax of regular expressions}
%
% Most characters match exactly themselves,
% with an arbitrary category code. Some characters are
% special and must be escaped with a backslash (\emph{e.g.}, |\*|
% matches a star character). Some escape sequences of
% the form backslash--letter also have a special meaning
% (for instance |\d| matches any digit). As a rule,
% \begin{itemize}
% \item every alphanumeric character (\texttt{A}--\texttt{Z},
%   \texttt{a}--\texttt{z}, \texttt{0}--\texttt{9}) matches
%   exactly itself, and should not be escaped, because
%   |\A|, |\B|, \ldots{} have special meanings;
% \item non-alphanumeric printable ascii characters can (and should)
%   always be escaped: many of them have special meanings (\emph{e.g.},
%   use |\(|, |\)|, |\?|, |\.|);
% \item spaces should always be escaped (even in character
%   classes);
% \item any other character may be escaped or not, without any
%   effect: both versions will match exactly that character.
% \end{itemize}
% Note that these rules play nicely with the fact that many
% non-alphanumeric characters are difficult to input into \TeX{}
% under normal category codes. For instance, |\\abc\%|
% matches the characters |\abc%| (with arbitrary category codes),
% but does not match the control sequence |\abc| followed by a
% percent character. Matching control sequences can be done
% using the |\c|\Arg{regex} syntax (see below).
%
% Any special character which appears at a place where its special
% behaviour cannot apply matches itself instead (for instance, a
% quantifier appearing at the beginning of a string), after raising a
% warning.
%
% Characters.
% \begin{l3regex-syntax}
%   \item[\\x\{hh\ldots{}\}] Character with hex code \texttt{hh\ldots{}}
%   \item[\\xhh] Character with hex code \texttt{hh}.
%   \item[\\a] Alarm (hex 07).
%   \item[\\e] Escape (hex 1B).
%   \item[\\f] Form-feed (hex 0C).
%   \item[\\n] New line (hex 0A).
%   \item[\\r] Carriage return (hex 0D).
%   \item[\\t] Horizontal tab (hex 09).
% \end{l3regex-syntax}
%
% Character types.
% \begin{l3regex-syntax}
%   \item[.] A single period matches any token.
%   \item[\\d] Any decimal digit.
%   \item[\\h] Any horizontal space character,
%     equivalent to |[\ \^^I]|: space and tab.
%   \item[\\s] Any space character,
%     equivalent to |[\ \^^I\^^J\^^L\^^M]|.
%   \item[\\v] Any vertical space character,
%     equivalent to |[\^^J\^^K\^^L\^^M]|. Note that |\^^K| is a vertical space,
%     but not a space, for compatibility with Perl.
%   \item[\\w] Any word character, \emph{i.e.},
%     alpha-numerics and underscore, equivalent to |[A-Za-z0-9\_]|.
%   \item[\\D] Any token not matched by |\d|.
%   \item[\\H] Any token not matched by |\h|.
%   \item[\\N] Any token other than the |\n| character (hex 0A).
%   \item[\\S] Any token not matched by |\s|.
%   \item[\\V] Any token not matched by |\v|.
%   \item[\\W] Any token not matched by |\w|.
% \end{l3regex-syntax}
% Of those, |.|, |\D|, |\H|, |\N|, |\S|, |\V|, and |\W| will match arbitrary
% control sequences.
%
% Character classes match exactly one token in the subject.
% \begin{l3regex-syntax}
%   \item[{[\ldots{}]}] Positive character class.
%     Matches any of the specified tokens.
%   \item[{[\char`\^\ldots{}]}] Negative character class.
%     Matches any token other than the specified characters.
%   \item[{x-y}] Within a character class, this denotes a range (can be
%     used with escaped characters).
%   \item[{[:\meta{name}:]}] Within a character class (one more set of
%     brackets), this denotes the \textsc{posix} character class
%     \meta{name}, which can be \texttt{alnum}, \texttt{alpha},
%     \texttt{ascii}, \texttt{blank}, \texttt{cntrl}, \texttt{digit},
%     \texttt{graph}, \texttt{lower}, \texttt{print}, \texttt{punct},
%     \texttt{space}, \texttt{upper}, \texttt{word}, or \texttt{xdigit}.
%   \item[{[:\char`\^\meta{name}:]}] Negative \textsc{posix} character class.
% \end{l3regex-syntax}
% For instance, |[a-oq-z\cC.]| matches any lowercase latin letter
% except |p|, as well as control sequences (see below for a description
% of |\c|).
%
% Quantifiers (repetition).
% \begin{l3regex-syntax}
%   \item[?] $0$ or $1$, greedy.
%   \item[??] $0$ or $1$, lazy.
%   \item[*] $0$ or more, greedy.
%   \item[*?] $0$ or more, lazy.
%   \item[+] $1$ or more, greedy.
%   \item[+?] $1$ or more, lazy.
%   \item[\{$n$\}] Exactly $n$.
%   \item[\{$n,$\}] $n$ or more, greedy.
%   \item[\{$n,$\}?] $n$ or more, lazy.
%   \item[\{$n,m$\}] At least $n$, no more than $m$, greedy.
%   \item[\{$n,m$\}?] At least $n$, no more than $m$, lazy.
% \end{l3regex-syntax}
%
% Anchors and simple assertions.
% \begin{l3regex-syntax}
%   \item[\\b] Word boundary: either the previous token is matched by
%     |\w| and the next by |\W|, or the opposite. For this purpose,
%     the ends of the token list are considered as |\W|.
%   \item[\\B] Not a word boundary: between two |\w| tokens
%     or two |\W| tokens (including the boundary).
%   \item[\char`^ \textrm{or} \\A]
%     Start of the subject token list.
%   \item[\char`$\textrm{,} \\Z \textrm{or} \\z]
%     End of the subject token list.
%   \item[\\G] Start of the current match. This is only different from |^|
%     in the case of multiple matches: for instance
%     |\regex_count:nnN { \G a } { aaba } \l_tmpa_int| yields $2$, but
%     replacing |\G| by |^| would result in \cs{l_tmpa_int} holding the
%     value $1$.
% \end{l3regex-syntax}
%
% Alternation and capturing groups.
% \begin{l3regex-syntax}
%   \item[A\char`|B\char`|C] Either one of \texttt{A}, \texttt{B},
%     or \texttt{C}.
%   \item[(\ldots{})] Capturing group.
%   \item[(?:\ldots{})] Non-capturing group.
%   \item[(?\char`|\ldots{})] Non-capturing group which resets
%     the group number for capturing groups in each alternative.
%     The following group will be numbered with the first unused
%     group number.
% \end{l3regex-syntax}
%
% The |\c| escape sequence allows to test the category code of tokens,
% and match control sequences. Each character category is represented
% by a single uppercase letter:
% \begin{itemize}
% \item |C| for control sequences;
% \item |B| for begin-group tokens;
% \item |E| for end-group tokens;
% \item |M| for math shift;
% \item |T| for alignment tab tokens;
% \item |P| for macro parameter tokens;
% \item |U| for superscript tokens (up);
% \item |D| for subscript tokens (down);
% \item |S| for spaces;
% \item |L| for letters;
% \item |O| for others; and
% \item |A| for active characters.
% \end{itemize}
% The |\c| escape sequence is used as follows.
% \begin{l3regex-syntax}
%   \item[\\c\Arg{regex}] A control sequence whose csname matches the
%     \meta{regex}, anchored at the beginning and end, so that |\c{begin}|
%     matches exactly \cs{begin}, and nothing else.
%   \item[\\cX] Applies to the next object, which can be a character,
%     character property, class, or group, and forces this object to
%     only match tokens with category |X| (any of |CBEMTPUDSLOA|. For
%     instance, |\cL[A-Z\d]| matches uppercase letters and digits of
%     category code letter, |\cC.| matches any control sequence, and
%     |\cO(abc)| matches |abc| where each character has category other.
%   \item[{\\c[XYZ]}] Applies to the next object, and forces it to only
%     match tokens with category |X|, |Y|, or |Z| (each being any of
%     |CBEMTPUDSLOA|). For instance, |\c[LSO](..)| matches two tokens of
%     category letter, space, or other.
%   \item[{\\c[\char`\^XYZ]}] Applies to the next object and prevents it
%     from matching any token with category |X|, |Y|, or |Z| (each being
%     any of |CBEMTPUDSLOA|). For instance, |\c[^O]\d| matches digits
%     which have any category different from other.
% \end{l3regex-syntax}
% The category code tests can be used inside classes; for instance,
% |[\cO\d \c[LO][A-F]]| matches what \TeX{} considers as hexadecimal
% digits, namely digits with category other, or uppercase letters from
% |A| to |F| with category either letter or other. Within a group
% affected by a category code test, the outer test can be overridden by
% a nested test: for instance, |\cL(ab\cO\*cd)| matches |ab*cd| where
% all characters are of category letter, except |*| which has category
% other.
%
% The |\u| escape sequence allows to insert the contents of a token list
% directly into a regular expression or a replacement, avoiding the need
% to escape special characters. Namely, |\u|\Arg{tl~var~name} matches
% the exact contents of the token list \meta{tl~var}. Within a |\c{...}|
% control sequence matching, the |\u| escape sequence only expands its
% argument once, in effect performing \cs{tl_to_str:v}. Quantifiers are
% not supported directly: use a group.
%
% The option |(?i)| makes the match case insensitive (identifying
% \texttt{A}--\texttt{Z} with \texttt{a}--\texttt{z}; no Unicode support
% yet). This applies until the end of the group in which it appears, and
% can be reverted using |(?-i)|. For instance, in
% \verb"(?i)(a(?-i)b|c)d", the letters |a| and |d| are affected by the
% |i| option. Characters within ranges and classes are affected
% individually: |(?i)[Y-\\]| is equivalent to |[YZ\[\\yz]|, and
% |(?i)[^aeiou]| matches any character which is not a vowel. Neither
% character properties, nor |\c{...}| nor |\u{...}| are affected by the
% |i| option.
%
% In character classes, only |[|, |^|, |-|, |]|, |\| and spaces are
% special, and should be escaped. Other non-alphanumeric characters can
% still be escaped without harm. Any escape sequence which matches a
% single character (|\d|, |\D|, \emph{etc.}) is supported in character
% classes.  If the first character is |^|, then
% the meaning of the character class is inverted. Ranges of characters
% can be expressed using |-|, for instance, |[\D 0-5]| and |[^6-9]| are
% equivalent.
%
% Capturing groups are a means of extracting information about the
% match. Parenthesized groups are labelled in the order of their
% opening parenthesis, starting at $1$. The contents of those groups
% corresponding to the \enquote{best} match (leftmost longest)
% can be extracted and stored in a sequence of token lists using for
% instance \cs{regex_extract_once:nnNTF}.
%
% The |\K| escape sequence resets the beginning of the match to the
% current position in the token list. This only affects what is reported
% as the full match. For instance,
% \begin{verbatim}
%   \regex_extract_all:nnN { a \K . } { a123aaxyz } \l_foo_seq
% \end{verbatim}
% results in \cs{l_foo_seq} containing the items |{1}| and |{a}|: the
% true matches are |{a1}| and |{aa}|, but they are trimmed by the use of
% |\K|. The |\K| command does not affect capturing groups: for instance,
% \begin{verbatim}
%   \regex_extract_once:nnN { (. \K c)+ \d } { acbc3 } \l_foo_seq
% \end{verbatim}
% results in \cs{l_foo_seq} containing the items |{c3}| and |{bc}|: the
% true match is |{acbc3}|, with first submatch |{bc}|, but |\K| resets
% the beginning of the match to the last position where it appears.
%
% \subsection{Syntax of the replacement text}
%
% Most of the features described in regular expressions do not make sense
% within the replacement text. Escaped characters are supported as inside
% regular expressions. The whole match is accessed as |\0|, and the first
% $9$ submatches are accessed as |\1|, \ldots{}, |\9|. Submatches with
% numbers higher than $9$ are accessed as |\g{|\meta{number}|}| instead.
%
% For instance,
% \begin{verbatim}
%   \tl_set:Nn \l_my_tl { Hello,~world! }
%   \regex_replace_all:nnN { ([er]?l|o) . } { \(\0\-\-\1\) } \l_my_tl
% \end{verbatim}
% results in \cs{l_my_tl} holding |H(ell--el)(o,--o) w(or--o)(ld--l)!|
%
% The characters inserted by the replacement have category code $12$
% (other) by default. The escape sequence |\c| allows to insert characters
% with arbitrary category codes, as well as control sequences.
% \begin{l3regex-syntax}
% \item[\\cXY] Produces the character |Y| (which can be given as
%   an escape sequence such as |\t| for tab) with category code |X|,
%   which must be one of |CBEMTPUDSLOA|.
% \item[\\c\Arg{text}] Produces the control sequence with csname
%   \meta{text}. The \meta{text} may contain references to the submatches
%   |\0|, |\1| \emph{etc.}
% \end{l3regex-syntax}
%
% \subsection{Pre-compiling regular expressions}
%
% If a regular expression is to be used several times,
% it is better to compile it once rather than doing it
% each time the regular expression is used. The compiled
% regular expression is stored in a variable. All
% of the \pkg{l3regex} module's functions can be given their
% regular expression argument either as an explicit string
% or as a compiled regular expression.
%
% \begin{function}{\regex_new:N}
%   \begin{syntax}
%     \cs{regex_new:N} \meta{regex~var}
%   \end{syntax}
%   Creates a new \meta{regex~var} or raises an error if the
%   name is already taken. The declaration is global. The
%   \meta{regex~var} will initially be such that it never matches.
% \end{function}
%
% \begin{function}{\regex_set:Nn, \regex_gset:Nn, \regex_const:Nn}
%   \begin{syntax}
%     \cs{regex_set:Nn} \meta{regex~var} \Arg{regex}
%   \end{syntax}
%   Stores a compiled version of the \meta{regular expression}
%   in the \meta{regex~var}. For instance, this function can be used
%   as
%   \begin{verbatim}
%     \regex_new:N \l_my_regex
%     \regex_set:Nn \l_my_regex { my\ (simple\ )? reg(ex|ular\ expression) }
%   \end{verbatim}
%   The assignment is local for \cs{regex_set:Nn} and global for
%   \cs{regex_gset:Nn}. Use \cs{regex_const:Nn} for compiled expressions
%   which will never change.
% \end{function}
%
% \begin{function}{\regex_show:n, \regex_show:N}
%   \begin{syntax}
%     \cs{regex_show:n} \Arg{regex}
%   \end{syntax}
%   Shows how \pkg{l3regex} interprets the \meta{regex}. For instance,
%   \cs{regex_show:n} \verb+{\A X|Y}+ shows
%   \begin{verbatim}
%     +-branch
%       anchor at start (\A)
%       char code 88
%     +-branch
%       char code 89
%   \end{verbatim}
%   indicating that the anchor |\A| only applies to the first branch:
%   the second branch is not anchored to the beginning of the match.
% \end{function}
%
% \subsection{Matching}
%
% All regular expression functions are available in both |:n| and |:N|
% variants. The former require a \enquote{standard} regular expression,
% while the later require a compiled expression as generated by
% \cs{regex_(g)set:Nn}.
%
% \begin{function}[TF]{\regex_match:nn, \regex_match:Nn}
%   \begin{syntax}
%     \cs{regex_match:nnTF} \Arg{regex} \Arg{token list} \Arg{true code} \Arg{false code}
%   \end{syntax}
%   Tests whether the \meta{regular expression} matches any part
%   of the \meta{token list}. For instance,
%   \begin{verbatim}
%     \regex_match:nnTF { b [cde]* } { abecdcx } { TRUE } { FALSE }
%     \regex_match:nnTF { [b-dq-w] } { example } { TRUE } { FALSE }
%   \end{verbatim}
%   leaves \texttt{TRUE} then \texttt{FALSE} in the input stream.
% \end{function}
%
% \begin{function}{\regex_count:nnN, \regex_count:NnN}
%   \begin{syntax}
%     \cs{regex_count:nnN} \Arg{regex} \Arg{token list} \meta{int var}
%   \end{syntax}
%   Sets \meta{int var} within the current \TeX{} group level
%   equal to the number of times
%   \meta{regular expression} appears in \meta{token list}.
%   The search starts by finding the left-most longest match,
%   respecting greedy and ungreedy operators. Then the search
%   starts again from the character following the last character
%   of the previous match, until reaching the end of the token list.
%   Infinite loops are prevented in the case where the regular expression
%   can match an empty token list: then we count one match between each
%   pair of characters.
%   For instance,
%   \begin{verbatim}
%     \int_new:N \l_foo_int
%     \regex_count:nnN { (b+|c) } { abbababcbb } \l_foo_int
%   \end{verbatim}
%   results in \cs{l_foo_int} taking the value $5$.
% \end{function}
%
% \subsection{Submatch extraction}
%
% \begin{function}[TF]{\regex_extract_once:nnN, \regex_extract_once:NnN}
%   \begin{syntax}
%     \cs{regex_extract_once:nnN} \Arg{regex} \Arg{token list} \meta{seq~var}
%     \cs{regex_extract_once:nnNTF} \Arg{regex} \Arg{token list} \meta{seq~var} \Arg{true code} \Arg{false code}
%   \end{syntax}
%   Finds the first match of the \meta{regular expression}
%   in the \meta{token list}. If it exists, the match is stored
%   as the zeroeth item of the \meta{seq~var}, and further
%   items are the contents of capturing groups, in the order
%   of their opening parenthesis. The \meta{seq~var}
%   is assigned locally. If there is no match,
%   the \meta{seq~var} is cleared.
%   The testing versions insert the \meta{true code} into the input
%   stream if a match was found, and the \meta{false code} otherwise.
%   For instance, assume that you type
%   \begin{verbatim}
%     \regex_extract_once:nnNTF { \A(La)?TeX(!*)\Z } { LaTeX!!! } \l_foo_seq
%       { true } { false }
%   \end{verbatim}
%   Then the regular expression (anchored at the start with |\A| and
%   at the end with |\Z|) will match the whole token list. The first
%   capturing group, |(La)?|, matches |La|, and the second capturing
%   group, |(!*)|, matches |!!!|. Thus, |\l_foo_seq| will contain
%   the items |{LaTeX!!!}|, |{La}|, and |{!!!}|, and the \texttt{true}
%   branch is left in the input stream.
% \end{function}
%
% \begin{function}[TF]{\regex_extract_all:nnN, \regex_extract_all:NnN}
%   \begin{syntax}
%     \cs{regex_extract_all:nnN} \Arg{regex} \Arg{token list} \meta{seq~var}
%     \cs{regex_extract_all:nnNTF} \Arg{regex} \Arg{token list} \meta{seq~var} \Arg{true code} \Arg{false code}
%   \end{syntax}
%   Finds all matches of the \meta{regular expression}
%   in the \meta{token list}, and stores all the submatch information
%   in a single sequence (concatenating the results of
%   multiple \cs{regex_extract_once:nnN} calls).
%   The \meta{seq~var} is assigned locally. If there is no match,
%   the \meta{seq~var} is cleared.
%   The testing versions insert the \meta{true code} into the input
%   stream if a match was found, and the \meta{false code} otherwise.
%   For instance, assume that you type
%   \begin{verbatim}
%     \regex_extract_all:nnNTF { \w+ } { Hello,~world! } \l_foo_seq
%       { true } { false }
%   \end{verbatim}
%   Then the regular expression will match twice, and the resulting
%   sequence contains the two items |{Hello}| and |{world}|,
%   and the \texttt{true} branch is left in the input stream.
% \end{function}
%
% \begin{function}[TF]{\regex_split:nnN, \regex_split:NnN}
%   \begin{syntax}
%     \cs{regex_split:nnN} \Arg{regular expression} \Arg{token list} \meta{seq~var}
%     \cs{regex_split:nnNTF} \Arg{regular expression} \Arg{token list} \meta{seq~var} \Arg{true code} \Arg{false code}
%   \end{syntax}
%   Splits the \meta{token list} into a sequence of parts, delimited by
%   matches of the \meta{regular expression}. If the \meta{regular expression}
%   has capturing groups, then the token lists that they match are stored as
%   items of the sequence as well. The assignment to \meta{seq~var} is local.
%   If no match is found the resulting \meta{seq~var} has the
%   \meta{token list} as its sole item. If the \meta{regular expression}
%   matches the empty token list, then the \meta{token list} is split
%   into single tokens.
%   The testing versions insert the \meta{true code} into the input
%   stream if a match was found, and the \meta{false code} otherwise.
%   For example, after
%   \begin{verbatim}
%     \seq_new:N \l_path_seq
%     \regex_split:nnNTF { / } { the/path/for/this/file.tex } \l_path_seq
%       { true } { false }
%   \end{verbatim}
%   the sequence |\l_path_seq| contains the items |{the}|, |{path}|,
%   |{for}|, |{this}|, and |{file.tex}|, and the \texttt{true} branch
%   is left in the input stream.
% \end{function}
%
% \subsection{Replacement}
%
% \begin{function}[TF]{\regex_replace_once:nnN,\regex_replace_once:NnN}
%   \begin{syntax}
%     \cs{regex_replace_once:nnN} \Arg{regular expression} \Arg{replacement} \meta{tl~var}
%     \cs{regex_replace_once:nnNTF} \Arg{regular expression} \Arg{replacement} \meta{tl~var} \Arg{true code} \Arg{false code}
%   \end{syntax}
%   Searches for the \meta{regular expression} in the \meta{token list}
%   and replaces the first match with the \meta{replacement}. The result
%   is assigned locally to \meta{tl~var}. In the \meta{replacement},
%   |\0| represents the full match, |\1| represent the contents of the
%   first capturing group, |\2| of the second, \emph{etc.}
% \end{function}
%
% \begin{function}[TF]{\regex_replace_all:nnN, \regex_replace_all:NnN}
%   \begin{syntax}
%     \cs{regex_replace_all:nnN} \Arg{regular expression} \Arg{replacement} \meta{tl~var}
%     \cs{regex_replace_all:nnNTF} \Arg{regular expression} \Arg{replacement} \meta{tl~var} \Arg{true code} \Arg{false code}
%   \end{syntax}
%   Replaces all occurrences of the \cs{regular expression} in the
%   \meta{token list} by the \meta{replacement}, where |\0| represents
%   the full match, |\1| represent the contents of the first capturing
%   group, |\2| of the second, \emph{etc.} Every match is treated
%   independently, and matches cannot overlap.  The result is assigned
%   locally to \meta{tl~var}.
% \end{function}
%
% \subsection{Bugs, misfeatures, future work, and other possibilities}
%
% The following need to be done now.
% \begin{itemize}
%   \item Change user function names!
%   \item Clean up the use of messages.
%   \item Rewrite the documentation in a more ordered way, perhaps add a
%     \textsc{bnf}?
% \end{itemize}
%
% Additional error-checking to come.
% \begin{itemize}
%   \item Detect that a trailing |\c|\meta{category} is an invalid regex.
%   \item Currently, |a{\x34}| is recognized as |a{4}|.
%   \item Cleaner error reporting in the replacement phase.
%   \item Add tracing information.
%   \item Detect attempts to use back-references.
%   \item Test for the maximum register \cs{c_max_register_int}.
%   \item Find out whether the fact that |\W| and friends match the
%     end-marker leads to bugs. Possibly update \cs{regex_item_reverse:n}.
%   \item Enforce that |\cC| can only be followed by a match-all dot.
% \end{itemize}
%
% Code improvements to come.
% \begin{itemize}
%   \item Change \tn{skip} to \tn{dimen} for the array of active
%     threads, and shift the array of submatch informations so that it
%     starts at \tn{skip}$0$.
%   \item Optimize |\c{abc}| for matching a specific control sequence.
%   \item Only build \c{...} once.
%   \item Use \tn{skip} for the left and right state stacks when
%     compiling a regex.
%   \item Should \cs{regex_action_free_group:n} only be used for greedy
%     |{n,}| quantifier? (I think not.)
%   \item Quantifiers for |\u| and assertions.
%   \item Improve digit grabbing for the |\g| escape in replacement.
%     Allow arbitrary integer expressions for all those numbers?
%   \item When matching, keep track of an explicit stack of
%     \texttt{current_state} and \texttt{current_submatches}.
%   \item If possible, when a state is reused by the same thread, kill
%     other subthreads.
%   \item Use \tn{dimen} registers rather than \cs{l_regex_balance_tl}
%     to build \cs{regex_replacement_balance_one_match:n}.
%   \item Reduce the number of epsilon-transitions in alternatives.
%   \item Optimize simple strings: use less states (|abcade| should give
%     two states, for |abc| and |ade|). [Does that really make sense?]
%   \item Optimize groups with no alternative.
%   \item Optimize states with a single \cs{regex_action_free:n}.
%   \item Optimize the use of \cs{regex_action_success:} by inserting it
%     in state $2$ directly instead of having an extra transition.
%   \item Optimize the use of \cs{int_step_...} functions.
%   \item Groups don't capture within regexes for csnames; optimize and
%     document.
%   \item Decide and document what |\c{\c{...}}| should do in the
%     replacement text, similar questions for |\u|.
%   \item Better \enquote{show} for anchors, properties, and catcode tests.
%   \item Does |\K| really need a new state for itself?
%   \item When compiling, use a boolean \texttt{in_cs} and less magic
%     numbers.
%   \item Instead of checking whether the character is special or
%     alphanumeric using its character code, check if it is special in
%     regexes with \cs{cs_if_exist} tests.
% \end{itemize}
%
% The following features are likely to be implemented at some point
% in the future.
% \begin{itemize}
%   \item Allow |\cL(abc)| in replacement text.
%   \item General look-ahead/behind assertions.
%   \item Regex matching on external files.
%   \item Conditional subpatterns with look ahead/behind: \enquote{if
%       what follows is [\ldots{}], then [\ldots{}]}.
%   \item |(*..)| and |(?..)| sequences to set some options.
%   \item UTF-8 mode for pdf\TeX{}.
%   \item Newline conventions are not done.
%     In particular, we should have an option for |.| not to match newlines.
%     Also, |\A| should differ from |^|, and |\Z|, |\z| and |$| should
%     differ.
%   \item Unicode properties: |\p{..}| and |\P{..}|;
%     |\X| which should match any \enquote{extended} Unicode sequence.
%     This requires to manipulate a lot of data, probably using tree-boxes.
% \end{itemize}
%
% The following features of \textsc{pcre} or Perl will probably not be
% implemented.
% \begin{itemize}
%   \item |\ddd|, matching the character with octal code \texttt{ddd};
%   \item Callout with |(?C...)|, we cannot run arbitrary user code
%     during the matching, because the regex code uses registers in an
%     unsafe way;
%   \item Conditional subpatterns (other than with a look-ahead or
%     look-behind condition): this is non-regular, isn't it?
%   \item Named subpatterns: \TeX{} programmers have lived so far
%     without any need for named macro parameters.
% \end{itemize}
%
% The following features of \textsc{pcre} or Perl will definitely not be
% implemented.
% \begin{itemize}
%   \item |\cx|, similar to \TeX{}'s own |\^^x|;
%   \item Comments: \TeX{} already has its own system for comments.
%   \item |\Q...\E| escaping: this would require to read the argument
%     verbatim, which is not in the scope of this module.
%   \item Atomic grouping, possessive quantifiers: those tools, mostly
%     meant to fix catastrophic backtracking, are unnecessary in a
%     non-backtracking algorithm, and difficult to implement.
%   \item Subroutine calls: this syntactic sugar is difficult to include
%     in a non-backtracking algorithm, in particular because the
%     corresponding group should be treated as atomic. Also, we cannot
%     afford to run user code within the regular expression matching,
%     because of our \enquote{misuse} of registers.
%   \item Recursion: this is a non-regular feature.
%   \item Back-references: non-regular feature, this requires
%     backtracking, which is prohibitively slow.
%   \item Backtracking control verbs: intrinsically tied to
%     backtracking.
%   \item |\C| single byte in UTF-8 mode: Xe\TeX{} and Lua\TeX{} serve
%     us characters directly, and splitting those into bytes is tricky,
%     encoding dependent, and most likely not useful anyways.
% \end{itemize}
%
% \end{documentation}
%
% \begin{implementation}
%
% \section{\pkg{l3regex} implementation}
%
%<*package>
%    \begin{macrocode}
\ProvidesExplPackage
  {\ExplFileName}{\ExplFileDate}{\ExplFileVersion}{\ExplFileDescription}
\RequirePackage{l3tl-build, l3tl-analysis, l3flag, l3str}
%    \end{macrocode}
%
% \subsection{Plan of attack}
%
% Most regex engines use backtracking. This allows to provide very
% powerful features (back-references come to mind first), but it is
% costly, and raises the problem of catastrophic backtracking. Since
% \TeX{} is not first and foremost a programming language, complicated
% code tends to run slowly, and we must use faster, albeit slightly more
% restrictive, techniques, coming from automata theory.
%
% Given a regular expression of $n$ characters, we do the following:
% \begin{itemize}
%   \item (Compiling.) Analyse the regex, finding invalid input, and
%     convert it to an internal representation.
%   \item (Building.) Convert the compiled regex to a non-deterministic
%     finite automaton (\textsc{nfa}) with roughly $n$ states which
%     accepts precisely token lists matching that regex.
%   \item (Matching.) Loop through the query token list one token (one
%     \enquote{position}) at a time, exploring in parallel every
%     possible path (\enquote{active thread}) through the \textsc{nfa},
%     considering active threads in an order determined by the
%     quantifiers' greediness.
% \end{itemize}
%
% We use the following vocabulary in the code comments (and in variable
% names).
% \begin{itemize}
%   \item \emph{Group}: index of the capturing group, $-1$ for
%     non-capturing groups.
%   \item \emph{Position}: each token in the query is labelled by an
%     integer \meta{position}, with $\texttt{min_pos} - 1 \leq
%     \meta{position} \leq \texttt{max_pos}$. The lowest and highest
%     positions correspond to imaginary begin and end markers (with
%     inacessible category code and character code).
%   \item \emph{Query}: the token list to which we apply the regular
%     expression.
%   \item \emph{State}: each state of the \textsc{nfa} is labelled by an
%     integer \meta{state} with $\texttt{min_state} \leq \meta{state} <
%     \texttt{max_state}$.
%   \item \emph{Active thread}: state of the \textsc{nfa} that is reached
%     when reading the query token list for the matching. Those threads
%     are ordered according to the greediness of quantifiers.
%   \item \emph{Step}: used when matching, starts at $0$, incremented
%     every time a character is read, and is not reset when searching
%     for repeated matches. The integer \cs{l_regex_step_int} is a
%     unique id for all the steps of the matching algorithm.
% \end{itemize}
%
% To acheive a good performance, we abuse \TeX{}'s registers in two
% ways.  We access registers directly by number rather than tying them
% to control sequence using \cs{int_new:N} and other allocation
% functions. And we store integers in \tn{dimen} registers in scaled
% points (\texttt{sp}), using \TeX{}'s implicit conversion from
% dimensions to integers in some contexts. Specifically, the registers
% are used as follows. When compiling, \tn{toks} registers are used
% under the hood by functions from the \pkg{l3tl-build} module.  When
% building,
% \begin{itemize}
%   \item \tn{toks}\meta{state} holds the tests and actions to perform
%     in the \meta{state} of the \textsc{nfa}.
%   \item (Not implemented yet.)
%     \tn{skip}$i$ has the form \meta{group id} \texttt{plus}
%     \meta{left state} \texttt{minus} \meta{right state}.
% \end{itemize}
% When matching,
% \begin{itemize}
%   \item \tn{dimen}\meta{state} is equal to the last \meta{step} in
%     which the \meta{state} was active.
%   \item (Currently, we use \tn{skip} instead of \tn{dimen}.)
%     \tn{dimen}\meta{thread}, with $\texttt{min_active} \leq
%     \meta{thread} < \texttt{max_active}$, is equal to the
%     \meta{state} in which the \meta{thread} currently is. The
%     \meta{threads} or ordered starting from the best to the least
%     preferred.
%   \item \tn{toks}\meta{thread} holds the submatch information for the
%     \meta{thread}, as the contents of a property list.
%   \item \tn{muskip}\meta{position} holds as its main and stretch
%     components the character and category code of the token at this
%     \meta{position} in the query.
%   \item \tn{toks}\meta{position} holds \meta{tokens} which \texttt{o}-
%     and \texttt{x}-expand to the \meta{position}-th token in the query.
%   \item \tn{skip} registers hold the value of end-points of all
%     submatches as would be extracted by the \cs{regex_extract}
%     functions. Since smaller \tn{skip} registers are used, the minimum
%     index is twice \texttt{max_state}, and the used registers go up to
%     \cs{l_regex_submatch_int}. They are organized in blocks of
%     \texttt{capturing_group}, each block corresponding to one match
%     with all its submatches stored in consecutive \tn{skip}s.
% \end{itemize}
% \tn{count} registers are not abused, which means that we can safely
% use named integers in this module. Note that \tn{box} registers are
% not abused either; maybe we could leverage those for some purpose.
%
% The code is structured as follows. Variables are introduced in the
% relevant section. First we present some generic helper functions. Then
% comes the code for compiling a regular expression, and for showing the
% result of the compilation. The building phase converts a compiled
% regex to \textsc{nfa} states, and the automaton is run by the code in
% the following section. The only remaining brick is parsing the
% replacement text and performing the replacement. We are then ready for
% all the user functions. Finally, messages, and a little bit of tracing
% code.
%
% \subsection{Helpers}
%
% \begin{macro}[aux]{\tl_to_str:V}
%   A variant we need for the |\u| escape in the replacement text.
%    \begin{macrocode}
\cs_generate_variant:Nn \tl_to_str:n { V }
%    \end{macrocode}
% \end{macro}
%
% \subsubsection{Constants and variables}
%
% \begin{macro}[aux]{\regex_tmp:w}
%   Temporary function used for various short-term purposes.
%    \begin{macrocode}
\cs_new:Npn \regex_tmp:w { }
%    \end{macrocode}
% \end{macro}
%
% \begin{variable}
%   {
%     \l_regex_internal_a_tl,  \l_regex_internal_b_tl,
%     \l_regex_internal_a_int, \l_regex_internal_b_int,
%     \l_regex_internal_c_int, \l_regex_internal_bool,
%     \l_regex_internal_seq,   \g_regex_internal_tl,
%   }
%   Temporary variables used for various purposes.
%    \begin{macrocode}
\tl_new:N   \l_regex_internal_a_tl
\tl_new:N   \l_regex_internal_b_tl
\int_new:N  \l_regex_internal_a_int
\int_new:N  \l_regex_internal_b_int
\int_new:N  \l_regex_internal_c_int
\bool_new:N \l_regex_internal_bool
\seq_new:N  \l_regex_internal_seq
\tl_new:N   \g_regex_internal_tl
%    \end{macrocode}
% \end{variable}
%
% \begin{variable}{\c_regex_no_match_regex}
%   This regular expression matches nothing, but is still a valid
%   regular expression. We could use a failing assertion, but I went for
%   an empty class. It is used as the initial value for regular
%   expressions declared using \cs{regex_new:N}.
%    \begin{macrocode}
\tl_const:Nn \c_regex_no_match_regex
  {
    \regex_branch:n
      { \regex_class:NnnnN \c_true_bool { } { 1 } { 0 } \c_true_bool }
  }
%    \end{macrocode}
% \end{variable}
%
% \begin{variable}{\l_regex_balance_int}
%   The first thing we do when matching is to go once through the query
%   token list and store the information for each token as \tn{muskip}
%   and \tn{toks} registers. During this phase, \cs{l_regex_balance_int}
%   counts the balance of begin-group and end-group character tokens
%   which appear before a given point in the token list, and we store it
%   as the shrink component of each \tn{muskip} register. This variable
%   is also used to keep track of the balance in the replacement text.
%    \begin{macrocode}
\int_new:N \l_regex_balance_int
%    \end{macrocode}
% \end{variable}
%
% \subsubsection{Testing characters}
%
% \begin{macro}[int]{\regex_break_point:TF}
% \begin{macro}[int]{\regex_break_true:w}
%   When testing whether a character of the query token list matches
%   a given character class in the regular expression, we often
%   have to test it against several ranges of characters, checking
%   if any one of those matches. This is done with a structure like
%   \begin{quote}
%     \meta{test1} \ldots{} \meta{test$\sb{n}$} \\
%     \cs{regex_break_point:TF} \Arg{true code} \Arg{false code}
%   \end{quote}
%   If any of the tests succeeds, it calls \cs{regex_break_true:w},
%   which cleans up and leaves \meta{true code} in the input stream.
%   Otherwise, \cs{regex_break_point:TF} leaves the \meta{false code}
%   in the input stream.
%    \begin{macrocode}
\cs_new_protected:Npn \regex_break_true:w
   #1 \regex_break_point:TF #2 #3 {#2}
\cs_new_protected:Npn \regex_break_point:TF #1 #2 { #2 }
%    \end{macrocode}
% \end{macro}
% \end{macro}
%
% \begin{macro}[int]{\regex_item_reverse:n}
%   This function makes showing regular expressions easier, and lets us
%   define |\D| in terms of |\d| for instance. There is a subtelty: the
%   end of the query is marked by $-2$, and will thus match |\D| and
%   other negated properties; this case is catched by another part of
%   the code.
%    \begin{macrocode}
\cs_new_protected:Npn \regex_item_reverse:n #1
  {
    #1
    \regex_break_point:TF { } \regex_break_true:w
  }
%    \end{macrocode}
% \end{macro}
%
% \begin{macro}[int]
%   {\regex_item_caseful_equal:n, \regex_item_caseful_range:nn}
%   Simple comparisons triggering \cs{regex_break_true:w} when true.
%    \begin{macrocode}
\cs_new_protected:Npn \regex_item_caseful_equal:n #1
  {
    \if_int_compare:w #1 = \l_regex_current_char_int
      \exp_after:wN \regex_break_true:w
    \fi:
  }
\cs_new_protected:Npn \regex_item_caseful_range:nn #1 #2
  {
    \reverse_if:N \if_int_compare:w #1 > \l_regex_current_char_int
      \reverse_if:N \if_int_compare:w #2 < \l_regex_current_char_int
        \exp_after:wN \exp_after:wN \exp_after:wN \regex_break_true:w
      \fi:
    \fi:
  }
%    \end{macrocode}
% \end{macro}
%
% \begin{macro}[int]
%   {\regex_item_caseless_equal:n, \regex_item_caseless_range:nn}
%   For caseless matching, we perform the test both on the
%   \texttt{current_char} and on the \texttt{case_changed_char}. Before
%   doing the second set of tests, we make sure that
%   \texttt{case_changed_char} has been computed.
%    \begin{macrocode}
\cs_new_protected:Npn \regex_item_caseless_equal:n #1
  {
    \if_int_compare:w #1 = \l_regex_current_char_int
      \exp_after:wN \regex_break_true:w
    \fi:
    \if_int_compare:w \l_regex_case_changed_char_int = \c_max_int
      \regex_compute_case_changed_char:
    \fi:
    \if_int_compare:w #1 = \l_regex_case_changed_char_int
      \exp_after:wN \regex_break_true:w
    \fi:
  }
\cs_new_protected:Npn \regex_item_caseless_range:nn #1 #2
  {
    \reverse_if:N \if_int_compare:w #1 > \l_regex_current_char_int
      \reverse_if:N \if_int_compare:w #2 < \l_regex_current_char_int
        \exp_after:wN \exp_after:wN \exp_after:wN \regex_break_true:w
      \fi:
    \fi:
    \if_int_compare:w \l_regex_case_changed_char_int = \c_max_int
      \regex_compute_case_changed_char:
    \fi:
    \reverse_if:N \if_int_compare:w #1 > \l_regex_case_changed_char_int
      \reverse_if:N \if_int_compare:w #2 < \l_regex_case_changed_char_int
        \exp_after:wN \exp_after:wN \exp_after:wN \regex_break_true:w
      \fi:
    \fi:
  }
%    \end{macrocode}
% \end{macro}
%
% \begin{macro}[int]{\regex_compute_case_changed_char:}
%   This function is called when \cs{l_regex_case_changed_char_int} has
%   not yet been computed (or rather, when it is set to the marker value
%   \cs{c_max_int}). If the current character code is in the range
%   $[65,90]$ (upper-case), then add $32$, making it lowercase. If it is
%   in the lower-case letter range $[97,122]$, subtract $32$.
%    \begin{macrocode}
\cs_new_protected_nopar:Npn \regex_compute_case_changed_char:
  {
    \int_set_eq:NN \l_regex_case_changed_char_int \l_regex_current_char_int
    \if_int_compare:w \l_regex_current_char_int < \c_ninety_one
      \if_int_compare:w \l_regex_current_char_int < \c_sixty_five
      \else:
        \int_add:Nn \l_regex_case_changed_char_int { \c_thirty_two }
      \fi:
    \else:
      \if_int_compare:w \l_regex_current_char_int < \c_one_hundred_twenty_three
        \if_int_compare:w \l_regex_current_char_int < \c_ninety_seven
        \else:
          \int_sub:Nn \l_regex_case_changed_char_int { \c_thirty_two }
        \fi:
      \fi:
    \fi:
  }
%    \end{macrocode}
% \end{macro}
%
% \begin{macro}[int, EXP]{\regex_item_equal:n, \regex_item_range:nn}
%   Those must always be defined to expand to a \texttt{caseful}
%   (default) or \texttt{caseless} version, and not be protected: they
%   must expand when compiling, to hard-code which tests are caseless or
%   caseful.
%    \begin{macrocode}
\cs_new_eq:NN \regex_item_equal:n ?
\cs_new_eq:NN \regex_item_range:nn ?
%    \end{macrocode}
% \end{macro}
%
% \begin{macro}[int]{\regex_item_catcode:nT, \regex_item_catcode_reverse:nT}
% \begin{macro}[aux]{\regex_item_catcode_aux:}
%   The argument is a sum of powers of $4$ with exponents given by the
%   allowed category codes (between $0$ and $13$). Dividing by a given
%   power of $4$ gives an odd result if and only if that category code
%   is allowed. If the catcode does not match, then skip the character
%   code tests which follow.
%    \begin{macrocode}
\cs_new_protected:Npn \regex_item_catcode_aux:
  {
    "
    \if_case:w \l_regex_current_catcode_int
         1       \or: 4       \or: 10      \or: 40
    \or: 100     \or:         \or: 1000    \or: 4000
    \or: 10000   \or:         \or: 100000  \or: 400000
    \or: 1000000 \or: 4000000 \else: 1*\c_zero
    \fi:
  }
\cs_new_protected:Npn \regex_item_catcode:nT #1
  {
    \if_int_odd:w \__int_eval:w #1 / \regex_item_catcode_aux: \__int_eval_end:
      \exp_after:wN \use:n
    \else:
      \exp_after:wN \use_none:n
    \fi:
  }
\cs_new_protected:Npn \regex_item_catcode_reverse:nT #1#2
  { \regex_item_catcode:nT {#1} { \regex_item_reverse:n {#2} } }
%    \end{macrocode}
% \end{macro}
% \end{macro}
%
% \begin{macro}[int]{\regex_item_exact:nn, \regex_item_exact_cs:c}
%   This matches an exact \meta{category}-\meta{character code} pair, or
%   an exact control sequence.
%    \begin{macrocode}
\cs_new_protected:Npn \regex_item_exact:nn #1#2
  {
    \if_int_compare:w #1 = \l_regex_current_catcode_int
      \if_int_compare:w #2 = \l_regex_current_char_int
        \exp_after:wN \exp_after:wN \exp_after:wN \regex_break_true:w
      \fi:
    \fi:
  }
\cs_new_protected:Npn \regex_item_exact_cs:c #1
  {
    \int_compare:nNnTF \l_regex_current_catcode_int = \c_zero
      {
        \str_if_eq:xxTF
          {
            \exp_after:wN \exp_after:wN \exp_after:wN \cs_to_str:N
            \tex_the:D \tex_toks:D \l_regex_current_pos_int
          }
          { #1 }
          { \regex_break_true:w } { }
      }
      { }
  }
%    \end{macrocode}
% \end{macro}
%
% \begin{macro}[int]{\regex_item_cs:n}
%   Match a control sequence (the argument is a compiled regex).
%   First test the catcode of the current token to be zero.
%   Then perform the matching test, and break if the csname
%   indeed matches. The three \cs{exp_after:wN} expand the contents
%   of the \tn{toks}\meta{current position} (of the form \cs{exp_not:n}
%   \Arg{control sequence}) to \meta{control sequence}.
%    \begin{macrocode}
\cs_new_protected:Npn \regex_item_cs:n #1
  {
    \int_compare:nNnT \l_regex_current_catcode_int = \c_zero
      {
        \group_begin:
          \regex_single_match:
          \regex_disable_submatches:
          \regex_build_for_cs:n {#1}
          \bool_set_eq:NN \l_regex_saved_success_bool \g_regex_success_bool
          \exp_args:Nx \regex_match:n
            {
              \exp_after:wN \exp_after:wN
              \exp_after:wN \cs_to_str:N
              \tex_the:D \tex_toks:D \l_regex_current_pos_int
            }
          \if_meaning:w \c_true_bool \g_regex_success_bool
            \group_insert_after:N \regex_break_true:w
          \fi:
          \bool_gset_eq:NN \g_regex_success_bool \l_regex_saved_success_bool
        \group_end:
      }
  }
%    \end{macrocode}
% \end{macro}
%
% \subsubsection{Character property tests}
%
% \begin{macro}[aux]
%   {
%     \regex_prop_d:, \regex_prop_h:, \regex_prop_s:,
%     \regex_prop_v:, \regex_prop_w:, \regex_prop_N:
%   }
%   Character property tests for |\d|, |\W|, \emph{etc.} These character
%   properties are not affected by the |(?i)| option. The characters
%   recognized by each one are as follows: |\d=[0-9]|,
%   |\w=[0-9A-Z_a-z]|, \verb*+\s=[\ \^^I\^^J\^^L\^^M]+,
%   \verb*+\h=[\ \^^I]+, |\v=[\^^J-\^^M]|, and the upper case
%   counterparts match anything that the lower case does not match.  The
%   order in which the various tests appear is optimized for usual
%   mostly lower case letter text.
%    \begin{macrocode}
\cs_new_protected_nopar:Npn \regex_prop_d:
  { \regex_item_caseful_range:nn \c_forty_eight { 57 } } % 0--9
\cs_new_protected_nopar:Npn \regex_prop_h:
  {
    \regex_item_caseful_equal:n \c_thirty_two % space
    \regex_item_caseful_equal:n \c_nine       % tab
  }
\cs_new_protected_nopar:Npn \regex_prop_s:
  {
    \regex_item_caseful_equal:n \c_thirty_two % space
    \regex_item_caseful_equal:n \c_nine       % tab
    \regex_item_caseful_equal:n \c_ten        % lf
    \regex_item_caseful_equal:n \c_twelve     % ff
    \regex_item_caseful_equal:n \c_thirteen   % cr
  }
\cs_new_protected_nopar:Npn \regex_prop_v:
  { \regex_item_caseful_range:nn \c_ten \c_thirteen } % lf, vtab, ff, cr
\cs_new_protected_nopar:Npn \regex_prop_w:
  {
    \regex_item_caseful_range:nn \c_ninety_seven { 122 } % a--z
    \regex_item_caseful_range:nn \c_sixty_five   {  90 } % A--Z
    \regex_item_caseful_range:nn \c_forty_eight  {  57 } % 0--9
    \regex_item_caseful_equal:n  { 95 } % _
  }
\cs_new_protected_nopar:Npn \regex_prop_N:
  { \regex_item_reverse:n { \regex_item_caseful_equal:n \c_ten } }
%    \end{macrocode}
% \end{macro}
%
% \begin{macro}[aux]
%   {
%     \regex_posix_alnum:, \regex_posix_alpha:, \regex_posix_ascii:,
%     \regex_posix_blank:, \regex_posix_cntrl:, \regex_posix_digit:,
%     \regex_posix_graph:, \regex_posix_lower:, \regex_posix_print:,
%     \regex_posix_punct:, \regex_posix_space:, \regex_posix_upper:,
%     \regex_posix_word: , \regex_posix_xdigit:
%   }
%   \textsc{posix} properties. No surprise.
%    \begin{macrocode}
\cs_new_protected_nopar:Npn \regex_posix_alnum:
  { \regex_posix_alpha: \regex_posix_digit: }
\cs_new_protected_nopar:Npn \regex_posix_alpha:
  { \regex_posix_lower: \regex_posix_upper: }
\cs_new_protected_nopar:Npn \regex_posix_ascii:
  { \regex_item_caseful_range:nn \c_zero \c_one_hundred_twenty_seven }
\cs_new_eq:NN \regex_posix_blank: \regex_prop_h:
\cs_new_protected_nopar:Npn \regex_posix_cntrl:
  {
    \regex_item_caseful_range:nn \c_zero { 31 }
    \regex_item_caseful_equal:n \c_one_hundred_twenty_seven
  }
\cs_new_eq:NN \regex_posix_digit: \regex_prop_d:
\cs_new_protected_nopar:Npn \regex_posix_graph:
  { \regex_item_caseful_range:nn { 33 } { 126 } }
\cs_new_protected_nopar:Npn \regex_posix_lower:
  { \regex_item_caseful_range:nn \c_ninety_seven { 122 } }
\cs_new_protected_nopar:Npn \regex_posix_print:
  { \regex_item_caseful_range:nn \c_thirty_two { 126 } }
\cs_new_protected_nopar:Npn \regex_posix_punct:
  {
    \regex_item_caseful_range:nn { 33 } { 47 }
    \regex_item_caseful_range:nn { 58 } { 64 }
    \regex_item_caseful_range:nn { 91 } { 96 }
    \regex_item_caseful_range:nn { 123 } { 126 }
  }
\cs_new_protected_nopar:Npn \regex_posix_space:
  {
    \regex_item_caseful_equal:n \c_thirty_two
    \regex_item_caseful_range:nn \c_nine \c_thirteen
  }
\cs_new_protected_nopar:Npn \regex_posix_upper:
  { \regex_item_caseful_range:nn \c_sixty_five { 90 } }
\cs_new_eq:NN \regex_posix_word: \regex_prop_w:
\cs_new_protected_nopar:Npn \regex_posix_xdigit:
  {
    \regex_posix_digit:
    \regex_item_caseful_range:nn \c_sixty_five { 70 }
    \regex_item_caseful_range:nn \c_ninety_seven { 102 }
  }
%    \end{macrocode}
% \end{macro}
%
% \subsubsection{Simple character escape}
%
% Before actually parsing the regular expression or the replacement
% text, we go through them once, converting |\n| to the character $10$,
% \emph{etc.} In this pass, we also convert any special character
% (\texttt{*}, \texttt{?}, \texttt{\{}, etc.) or escaped alphanumeric
% character into a marker indicating that this was a special sequence,
% and replace escaped special characters and non-escaped alphanumeric
% characters by markers indicating that those were \enquote{raw}
% characters. The rest of the code can then avoid caring about escaping
% issues (those can become quite complex to handle in combination with
% ranges in character classes).
%
% Usage: \cs{regex_escape_use:nnnn} \meta{inline~1} \meta{inline~2}
% \meta{inline~3} \Arg{token list} The \meta{token list} is converted to
% a string, then read from left to right, interpreting backslashes as
% escaping the next character.  Unescaped characters are fed to the
% function \meta{inline~1}, and escaped characters are fed to the function
% \meta{inline~2} within an \texttt{x}-expansion context (typically those
% functions perform some tests on their argument to decide how to output
% them).  The escape sequences |\a|, |\e|, |\f|, |\n|, |\r|, |\t| and
% |\x| are recognized, and those are replaced by the corresponding
% character, then fed to \meta{inline~3}. The result is then left in the
% input stream. Spaces are ignored unless escaped.
%
% The conversion is mostly done within an \texttt{x}-expanding
% assignment, except for the |\x| escape sequence, which is not amenable
% to that in general. For this, we use the general framework of
% \cs{tl_set_build:Nw}.
%
% \begin{macro}[int]{\regex_escape_use:nnnn}
%   The result is built in \cs{l_regex_internal_a_tl}, which is then
%   left in the input stream.  Go through |#4| once, applying |#1|,
%   |#2|, or |#3| as relevant to each character (after de-escaping
%   it). Note that we cannot replace \cs{tl_se:Nx} and
%   \cs{tl_build_one:o} by a single call to \cs{tl_build_one:x}, because
%   the \texttt{x}-expanding assignment is interrupted by |\x|.
%    \begin{macrocode}
\cs_new_protected:Npn \regex_escape_use:nnnn #1#2#3#4
  {
%<trace>    \trace_push:nnn { regex } { 1 } { regex_escape_use:nnnn }
    \tl_set_build:Nw \l_regex_internal_a_tl
      \cs_set_nopar:Npn \regex_escape_unescaped:N ##1 { #1 }
      \cs_set_nopar:Npn \regex_escape_escaped:N ##1 { #2 }
      \cs_set_nopar:Npn \regex_escape_raw:N ##1 { #3 }
      \int_set:Nn \tex_escapechar:D { 92 }
      \str_gset_other:Nn \g_regex_internal_tl { #4 }
      \tl_set:Nx \l_regex_internal_b_tl
        {
          \exp_after:wN \regex_escape_loop:N \g_regex_internal_tl
          { break } \prg_break_point:n { }
        }
      \tl_build_one:o \l_regex_internal_b_tl
    \tl_build_end:
%<trace>    \trace_pop:nnn { regex } { 1 } { regex_escape_use:nnnn }
    \l_regex_internal_a_tl
  }
%    \end{macrocode}
% \end{macro}
%
% \begin{macro}[aux]{\regex_escape_loop:N}
% \begin{macro}[aux]+\regex_escape_\:w+
%   \cs{regex_escape_loop:N} reads one character: if it is special
%   (space, backslash, or end-marker), perform the associated action,
%   otherwise it is simply an unescaped character. After a backslash,
%   the same is done, but unknown characters are \enquote{escaped}.
%    \begin{macrocode}
\cs_new:Npn \regex_escape_loop:N #1
  {
    \cs_if_exist_use:cF { regex_escape_\token_to_str:N #1:w }
      { \regex_escape_unescaped:N #1 }
    \regex_escape_loop:N
  }
\cs_new_nopar:cpn { regex_escape_ \c_backslash_str :w }
    \regex_escape_loop:N #1
  {
    \cs_if_exist_use:cF { regex_escape_/\token_to_str:N #1:w }
      { \regex_escape_escaped:N #1 }
    \regex_escape_loop:N
  }
%    \end{macrocode}
% \end{macro}
% \end{macro}
%
% \begin{macro}[aux]
%   {\regex_escape_unescaped:N, \regex_escape_escaped:N, \regex_escape_raw:N}
%   Those functions are never called before being given a new meaning,
%   so their definitions here don't matter.
%    \begin{macrocode}
\cs_new_eq:NN \regex_escape_unescaped:N ?
\cs_new_eq:NN \regex_escape_escaped:N   ?
\cs_new_eq:NN \regex_escape_raw:N       ?
%    \end{macrocode}
% \end{macro}
%
% \begin{macro}[aux]
%   {
%     \regex_escape_break:w, \regex_escape_/break:w,
%     \regex_escape_/a:w, \regex_escape_/e:w, \regex_escape_/f:w,
%     \regex_escape_/n:w, \regex_escape_/r:w, \regex_escape_/t:w
%   }
% \begin{macro}[aux]+\regex_escape_ :w+
%   The loop is ended upon seeing the end-marker
%   \enquote{\texttt{break}}, with an error if the string ended in a
%   backslash.  Spaces are ignored, and |\a|, |\e|, |\f|, |\n|, |\r|,
%   |\t| take their meaning here.
%    \begin{macrocode}
\cs_new_eq:NN \regex_escape_break:w \prg_map_break:
\cs_new_nopar:cpn { regex_escape_/break:w }
  {
    \if_false: { \fi: }
    \__msg_kernel_error:nn { regex } { trailing-backslash }
    \exp_after:wN \use_none:n \exp_after:wN { \if_false: } \fi:
  }
\cs_new_nopar:cpn { regex_escape_~:w } { }
\cs_new_nopar:cpx { regex_escape_/a:w }
  { \exp_not:N \regex_escape_raw:N \iow_char:N \^^G }
\cs_new_nopar:cpx { regex_escape_/t:w }
  { \exp_not:N \regex_escape_raw:N \iow_char:N \^^I }
\cs_new_nopar:cpx { regex_escape_/n:w }
  { \exp_not:N \regex_escape_raw:N \iow_char:N \^^J }
\cs_new_nopar:cpx { regex_escape_/f:w }
  { \exp_not:N \regex_escape_raw:N \iow_char:N \^^L }
\cs_new_nopar:cpx { regex_escape_/r:w }
  { \exp_not:N \regex_escape_raw:N \iow_char:N \^^M }
\cs_new_nopar:cpx { regex_escape_/e:w }
  { \exp_not:N \regex_escape_raw:N \iow_char:N \^^[ }
%    \end{macrocode}
% \end{macro}
% \end{macro}
%
% \begin{macro}[aux]{\regex_escape_/x:w}
% \begin{macro}[aux]{\regex_escape_x_end:w, \regex_escape_x_large:n}
%   When |\x| is encountered, \cs{regex_escape_x_test:N} is responsible
%   for grabbing some hexadecimal digits, and feeding the result to
%   \cs{regex_escape_x_end:w}. If the number is $<256$, then it is
%   turned into a byte and fed to \cs{regex_escape_raw:N}. Otherwise,
%   interrupt the assignment, and either produce an error, or use a
%   standard \tn{lowercase} trick depending on the precise value.
%    \begin{macrocode}
\cs_new:cpn { regex_escape_/x:w } \regex_escape_loop:N
  {
    \exp_after:wN \regex_escape_x_end:w
    \__int_value:w "0 \regex_escape_x_test:N
  }
\cs_new:Npn \regex_escape_x_end:w #1 ;
  {
    \int_compare:nNnTF {#1} < \c_two_hundred_fifty_six
      {
        \exp_last_unbraced:Nf \regex_escape_raw:N
          { \str_output_byte:n {#1} }
      }
      { \regex_escape_x_large:n {#1} }
  }
\group_begin:
  \char_set_catcode_other:n { 0 }
  \cs_new:Npn \regex_escape_x_large:n #1
    {
      \if_false: { \fi: }
      \tl_build_one:o \l_regex_internal_b_tl
      \int_compare:nNnTF {#1} > \c_max_char_int
        {
          \__msg_kernel_error:nnx { regex } { x-overflow } {#1}
          \tl_set:Nx \l_regex_internal_b_tl
            { \if_false: } \fi: \regex_escape_loop:N
        }
        {
          \char_set_lccode:nn { \c_zero } {#1}
          \tl_to_lowercase:n
            {
              \tl_set:Nx \l_regex_internal_b_tl
                { \if_false: } \fi:
                \regex_escape_raw:N ^^@
                \regex_escape_loop:N
            }
        }
    }
\group_end:
%    \end{macrocode}
% \end{macro}
% \end{macro}
%
% \begin{macro}[aux]{\regex_escape_x_test:N}
%   Find out whether the first character is a left brace (allowing any
%   number of hexadecimal digits), or not (allowing up to two
%   hexadecimal digits). We need to check for the end-of-string marker.
%   Eventually, call either \cs{regex_escape_x_loop:N} or
%   \cs{regex_escape_x_ii:N}.
%    \begin{macrocode}
\cs_new:Npn \regex_escape_x_test:N #1
  {
    \str_if_eq:xxTF {#1} { break } { ; }
      {
        \if_charcode:w \c_space_token #1
          \exp_after:wN \regex_escape_x_test:N
        \else:
          \exp_after:wN \regex_escape_x_test_ii:N
          \exp_after:wN #1
        \fi:
      }
  }
\cs_new:Npn \regex_escape_x_test_ii:N #1
  {
    \if_charcode:w \c_lbrace_str #1
      \exp_after:wN \regex_escape_x_loop:N
    \else:
      \str_aux_hexadecimal_use:NTF #1
        { \exp_after:wN \regex_escape_x_ii:N }
        { ; \exp_after:wN \regex_escape_loop:N \exp_after:wN #1 }
    \fi:
  }
%    \end{macrocode}
% \end{macro}
%
% \begin{macro}[aux]{\regex_escape_x_ii:N}
%   This looks for the second digit in the unbraced case.
%    \begin{macrocode}
\cs_new:Npn \regex_escape_x_ii:N #1
  {
    \str_if_eq:xxTF {#1} { break } { ; }
      {
        \str_aux_hexadecimal_use:NTF #1
          { ; \regex_escape_loop:N }
          { ; \regex_escape_loop:N #1 }
      }
  }
%    \end{macrocode}
% \end{macro}
%
% \begin{macro}[aux]{\regex_escape_x_loop:N}
%   Grab hexadecimal digits, skip spaces, and at the end, check that
%   there is a right brace, otherwise raise an error outside the
%   assignment.
%    \begin{macrocode}
\cs_new:Npn \regex_escape_x_loop:N #1
  {
    \str_aux_hexadecimal_use:NTF #1
      { \regex_escape_x_loop:N }
      {
        \token_if_eq_charcode:NNTF \c_space_token #1
          { \regex_escape_x_loop:N }
          {
            ;
            \exp_after:wN \token_if_eq_charcode:NNTF \c_rbrace_str #1
              { \regex_escape_loop:N }
              {
                \if_false: { \fi: }
                \tl_build_one:o \l_regex_internal_b_tl
                \__msg_kernel_error:nn { regex } { x-missing-rbrace } {#1}
                \tl_set:Nx \l_regex_internal_b_tl
                  { \if_false: } \fi: \regex_escape_loop:N #1
              }
          }
      }
  }
%    \end{macrocode}
% \end{macro}
%
% \begin{macro}[EXP, aux]
%   {\regex_char_if_alphanumeric:NTF, \regex_char_if_special:NTF}
%   These two tests are used in the first pass when parsing a regular
%   expression. That pass is responsible for finding escaped and
%   non-escaped characters, and recognizing which ones have special
%   meanings and which should be interpreted as \enquote{raw}
%   characters. Namely,
%   \begin{itemize}
%     \item alphanumerics are \enquote{raw} if they are not escaped, and
%       may have a special meaning when escaped;
%     \item non-alphanumeric printable ascii characters are
%       \enquote{raw} if they are escaped, and may have a special
%       meaning when not escaped;
%     \item characters other than printable ascii are always
%       \enquote{raw}.
%   \end{itemize}
%   The code is ugly, and highly based on magic numbers and the ascii
%   codes of characters. This is mostly unavoidable for performance
%   reasons: testing for instance with \cs{str_if_contains_char:nN}
%   would be much slower. Maybe the tests can be optimized a little
%   bit more.
%   Here, \enquote{alphanumeric} means \texttt{0}--\texttt{9},
%   \texttt{A}--\texttt{Z}, \texttt{a}--\texttt{z};
%   \enquote{special} character means non-alphanumeric
%   but printable ascii, from space (hex \texttt{20}) to
%   \texttt{del} (hex \texttt{7E}).
%    \begin{macrocode}
\prg_new_conditional:Npnn \regex_char_if_special:N #1 { TF }
  {
    \if_int_compare:w `#1 < \c_ninety_one
      \if_int_compare:w `#1 < \c_fifty_eight
        \if_int_compare:w `#1 < \c_forty_eight
          \if_int_compare:w `#1 < \c_thirty_two
            \prg_return_false: \else: \prg_return_true: \fi:
        \else: \prg_return_false: \fi:
      \else:
        \if_int_compare:w `#1 < \c_sixty_five
          \prg_return_true: \else: \prg_return_false: \fi:
      \fi:
    \else:
      \if_int_compare:w `#1 < \c_one_hundred_twenty_three
        \if_int_compare:w `#1 < \c_ninety_seven
          \prg_return_true: \else: \prg_return_false: \fi:
      \else:
        \if_int_compare:w `#1 < \c_one_hundred_twenty_seven
          \prg_return_true: \else: \prg_return_false: \fi:
      \fi:
    \fi:
  }
\prg_new_conditional:Npnn \regex_char_if_alphanumeric:N #1 { TF }
  {
    \if_int_compare:w `#1 < \c_ninety_one
      \if_int_compare:w `#1 < \c_fifty_eight
        \if_int_compare:w `#1 < \c_forty_eight
          \prg_return_false: \else: \prg_return_true: \fi:
      \else:
        \if_int_compare:w `#1 < \c_sixty_five
          \prg_return_false: \else: \prg_return_true: \fi:
      \fi:
    \else:
      \if_int_compare:w `#1 < \c_one_hundred_twenty_three
        \if_int_compare:w `#1 < \c_ninety_seven
          \prg_return_false: \else: \prg_return_true: \fi:
      \else:
        \prg_return_false:
      \fi:
    \fi:
  }
%    \end{macrocode}
% \end{macro}
%
% \subsection{Compiling}
%
% A regular expression starts its life as a string of characters. In
% this section, we convert it to internal instructions, resulting in a
% \enquote{compiled} regular expression. This compiled expression is
% then turned into states of an automaton in the building
% phase. Compiled regular expressions consist of the following:
% \begin{itemize}
%   \item \cs{regex_class:NnnnN} \meta{boolean} \Arg{tests} \Arg{min}
%     \Arg{more} \meta{lazyness}
%   \item \cs{regex_group:nnnN} \Arg{branches} \Arg{min} \Arg{more}
%     \meta{lazyness}, also \cs{regex_group_no_capture:nnnN} and
%     \cs{regex_group_resetting:nnnN} with the same syntax.
%   \item \cs{regex_branch:n} \Arg{contents}
%   \item \cs{regex_command_K:}
%   \item \cs{regex_assertion:Nn} \meta{boolean} \Arg{assertion test},
%     where the \meta{assertion test} is \cs{regex_b_test:} or
%     |{|\cs{regex_anchor:N} \meta{integer}|}|
% \end{itemize}
% Tests can be the following:
% \begin{itemize}
%   \item \cs{regex_item_caseful_equal:n} \Arg{char code}
%   \item \cs{regex_item_caseless_equal:n} \Arg{char code}
%   \item \cs{regex_item_caseful_range:nn} \Arg{min} \Arg{max}
%   \item \cs{regex_item_caseless_range:nn} \Arg{min} \Arg{max}
%   \item \cs{regex_item_catcode:nT} \Arg{catcode bitmap} \Arg{tests}
%   \item \cs{regex_item_catcode_reverse:nT} \Arg{catcode bitmap} \Arg{tests}
%   \item \cs{regex_item_reverse:n} \Arg{tests}
%   \item \cs{regex_item_exact:nn} \Arg{catcode} \Arg{char code}
%   \item \cs{regex_item_exact_cs:c} \Arg{csname}
%   \item \cs{regex_item_cs:n} \Arg{compiled regex}
% \end{itemize}
%
% \subsubsection{Variables used when compiling}
%
% \begin{variable}{\l_regex_group_level_int}
%   We make sure to open the same number of groups as we close.
%    \begin{macrocode}
\int_new:N \l_regex_group_level_int
%    \end{macrocode}
% \end{variable}
%
% \begin{variable}{\l_regex_mode_int}
%   While compiling, ten modes are recognized, labelled $-63$, $-23$,
%   $-6$, $-2$, $0$, $2$, $3$, $6$, $23$, $63$. See
%   section~\ref{sec:regex-modes}.
%    \begin{macrocode}
\int_new:N \l_regex_mode_int
%    \end{macrocode}
% \end{variable}
%
% \begin{variable}{\l_regex_catcodes_int, \l_regex_default_catcodes_int}
% \begin{variable}{\l_regex_catcodes_bool}
%   We wish to allow constructions such as |\c[^BE](..\cL[a-z]..)|,
%   where the outer catcode test applies to the whole group, but is
%   superseeded by the inner catcode test. For this to work, we need to
%   keep track of lists of allowed category codes:
%   \cs{l_regex_catcodes_int} and \cs{l_regex_default_catcodes_int} are
%   bitmaps, sums of $4^c$, for all allowed catcodes $c$. The latter is
%   local to each capturing group, and we reset
%   \cs{l_regex_catcodes_int} to that value after each character or
%   class, changing it only when encountering a |\c| escape. The boolean
%   records whether the list of categories of a catcode test has to be
%   inverted: compare |\c[^BE]| and |\c[BE]|.
%    \begin{macrocode}
\int_new:N \l_regex_catcodes_int
\int_new:N \l_regex_default_catcodes_int
\bool_new:N \l_regex_catcodes_bool
%    \end{macrocode}
% \end{variable}
% \end{variable}
%
% \begin{variable}
%   {
%     \c_regex_catcode_C_int, \c_regex_catcode_B_int, \c_regex_catcode_E_int,
%     \c_regex_catcode_M_int, \c_regex_catcode_T_int, \c_regex_catcode_P_int,
%     \c_regex_catcode_U_int, \c_regex_catcode_D_int, \c_regex_catcode_S_int,
%     \c_regex_catcode_L_int, \c_regex_catcode_O_int, \c_regex_catcode_A_int
%   }
% \begin{variable}{\c_regex_all_catcodes_int}
%   Constants: $4^c$ for each category, and the sum of all powers of $4$.
%    \begin{macrocode}
\int_const:Nn \c_regex_catcode_C_int { "1 }
\int_const:Nn \c_regex_catcode_B_int { "4 }
\int_const:Nn \c_regex_catcode_E_int { "10 }
\int_const:Nn \c_regex_catcode_M_int { "40 }
\int_const:Nn \c_regex_catcode_T_int { "100 }
\int_const:Nn \c_regex_catcode_P_int { "1000 }
\int_const:Nn \c_regex_catcode_U_int { "4000 }
\int_const:Nn \c_regex_catcode_D_int { "10000 }
\int_const:Nn \c_regex_catcode_S_int { "100000 }
\int_const:Nn \c_regex_catcode_L_int { "400000 }
\int_const:Nn \c_regex_catcode_O_int { "1000000 }
\int_const:Nn \c_regex_catcode_A_int { "4000000 }
\int_const:Nn \c_regex_all_catcodes_int { "5515155 }
%    \end{macrocode}
% \end{variable}
% \end{variable}
%
% \begin{variable}{\l_regex_internal_regex}
%   The compilation step stores its result in this variable.
%    \begin{macrocode}
\cs_new_eq:NN \l_regex_internal_regex \c_regex_no_match_regex
%    \end{macrocode}
% \end{variable}
%
% \begin{variable}{\l_regex_show_prefix_seq}
%   This sequence holds the prefix that makes up the line displayed to
%   the user. The various items must be removed from the right, which is
%   tricky with a token list, hence we use a sequence.
%    \begin{macrocode}
\seq_new:N \l_regex_show_prefix_seq
%    \end{macrocode}
% \end{variable}
%
% \begin{variable}{\l_regex_show_lines_int}
%   A hack. To know whether a given class has a single item in it or
%   not, we count the number of lines when showing the class.
%    \begin{macrocode}
\int_new:N \l_regex_show_lines_int
%    \end{macrocode}
% \end{variable}
%
% \subsubsection{Generic helpers used when compiling}
%
% \begin{macro}[int]{\regex_get_digits:NTFw}
% \begin{macro}[aux, rEXP]{\regex_get_digits_loop:w}
%   If followed by some raw digits, collect them one by one in the
%   integer variable |#1|, and take the \texttt{true} branch. Otherwise,
%   take the \texttt{false} branch.
%    \begin{macrocode}
\cs_new_protected:Npn \regex_get_digits:NTFw #1#2#3#4#5
  {
    \regex_if_raw_digit:NNTF #4 #5
      { #1 = #5 \regex_get_digits_loop:nw {#2} }
      { #3 #4 #5 }
  }
\cs_new:Npn \regex_get_digits_loop:nw #1#2#3
  {
    \regex_if_raw_digit:NNTF #2 #3
      { #3 \regex_get_digits_loop:nw {#1} }
      { \scan_stop: #1 #2 #3 }
  }
%    \end{macrocode}
% \end{macro}
% \end{macro}
%
% \begin{macro}[aux, EXP]{\regex_if_raw_digit:NNTF}
%   Test used when grabbing digits for the |{m,n}| quantifier.
%   It only accepts non-escaped digits.
%    \begin{macrocode}
\prg_new_conditional:Npnn \regex_if_raw_digit:NN #1#2 { TF }
  {
    \if_meaning:w \regex_compile_raw:N #1
      \if_int_compare:w \c_one < 1 #2 \exp_stop_f:
        \prg_return_true:
      \else:
        \prg_return_false:
      \fi:
    \else:
      \prg_return_false:
    \fi:
  }
%    \end{macrocode}
% \end{macro}
%
% \subsubsection{Mode}
% \label{sec:regex-modes}
%
% When compiling the \textsc{nfa} corresponding to a given regex string,
% we can be in ten distinct modes, which we label by some magic numbers:
% \begin{itemize}
%   \item[-6] |[\c{...}]| control sequence in a class,
%   \item[-2] |\c{...}| control sequence,
%   \item[0] |...| outer,
%   \item[2] |\c...| catcode test,
%   \item[6] |[\c...]| catcode test in a class,
%   \item[-63] |[\c{[...]}]| class inside mode $-6$,
%   \item[-23] |\c{[...]}| class inside mode $-2$,
%   \item[3] |[...]| class inside mode $-3$,
%   \item[23] |\c[...]| class inside mode $2$,
%   \item[63] |[\c[...]]| class inside mode $6$.
% \end{itemize}
% This list is exhaustive, because |\c| escape sequences cannot be
% nested, and character classes cannot be nested directly. The choice of
% numbers is such as to optimize the most useful tests, and make
% transitions from one mode to another as simple as possible.
% \begin{itemize}
%   \item Even modes mean that we are not directly in a character class.
%     In this case, a left bracket appends $3$ to the mode. In a
%     character class, a right bracket changes the mode as $m\to
%     (m-15)/13$, truncated.
%   \item Grouping, assertion, and anchors are allowed in non-positive
%     even modes ($0$, $-2$, $-6$), and do not change the
%     mode. Otherwise, they trigger an error.
%   \item A left bracket is special in even modes, appending $3$ to the
%     mode; in those modes, quantifiers and the dot are recognized, and
%     the right bracket is normal. In odd modes (within classes), the
%     left bracket is normal, but the right bracket ends the class,
%     changing the mode from $m$ to $(m-15)/13$, truncated; also, ranges
%     are recognized.
%   \item In non-negative modes, left and right braces are normal. In
%     negative modes, however, left braces trigger a warning; right
%     braces end the control sequence, going from $-2$ to $0$ or $-6$ to
%     $3$, with error recovery for odd modes.
%   \item Properties (such as the |\d| character class) can appear in
%     any mode.
% \end{itemize}
%
% \begin{macro}[int, EXP]{\regex_if_in_class:TF}
%   Test whether we are directly in a character class (at the innermost
%   level of nesting). There, many escape sequences are not recognized,
%   and special characters are normal. Also, for every raw character, we
%   must look ahead for a possible raw dash.
%   \begin{macrocode}
\cs_new_nopar:Npn \regex_if_in_class:TF
  {
    \if_int_odd:w \l_regex_mode_int
      \exp_after:wN \use_i:nn
    \else:
      \exp_after:wN \use_ii:nn
    \fi:
  }
%    \end{macrocode}
% \end{macro}
%
% \begin{macro}[int, EXP]{\regex_if_in_cs:TF}
%   Right braces are special only directly inside control sequences (at
%   the inner-most level of nesting, not counting groups).
%    \begin{macrocode}
\cs_new_nopar:Npn \regex_if_in_cs:TF
  {
    \if_int_odd:w \l_regex_mode_int
      \exp_after:wN \use_ii:nn
    \else:
      \if_int_compare:w \l_regex_mode_int < \c_zero
        \exp_after:wN \exp_after:wN \exp_after:wN \use_i:nn
      \else:
        \exp_after:wN \exp_after:wN \exp_after:wN \use_ii:nn
      \fi:
    \fi:
  }
%    \end{macrocode}
% \end{macro}
%
% \begin{macro}[int, EXP]{\regex_if_in_class_or_catcode:TF}
%   Assertions are only allowed in modes $0$, $-2$, and $-6$,
%   \emph{i.e.}, even, non-positive modes.
%    \begin{macrocode}
\cs_new_nopar:Npn \regex_if_in_class_or_catcode:TF
  {
    \if_int_odd:w \l_regex_mode_int
      \exp_after:wN \use_i:nn
    \else:
      \if_int_compare:w \l_regex_mode_int > \c_zero
        \exp_after:wN \exp_after:wN \exp_after:wN \use_i:nn
      \else:
        \exp_after:wN \exp_after:wN \exp_after:wN \use_ii:nn
      \fi:
    \fi:
  }
%    \end{macrocode}
% \end{macro}
%
% \begin{macro}[int, EXP]{\regex_if_within_catcode:TF}
%   This test takes the true branch if we are in a catcode test, either
%   immediately following it (modes $2$ and $6$) or in a class on which
%   it applies (modes $23$ and $63$). This is used to tweak how left
%   brackets behave in modes $2$ and $6$.
%    \begin{macrocode}
\cs_new_nopar:Npn \regex_if_within_catcode:TF
  {
    \if_int_compare:w \l_regex_mode_int > \c_zero
      \exp_after:wN \use_i:nn
    \else:
      \exp_after:wN \use_ii:nn
    \fi:
  }
%    \end{macrocode}
% \end{macro}
%
% \begin{macro}[aux]{\regex_chk_c_allowed:T}
%   The |\c| escape sequence is only allowed in modes $0$ and $3$,
%   \emph{i.e.}, not within any other |\c| escape sequence.
%    \begin{macrocode}
\cs_new_protected:Npn \regex_chk_c_allowed:T
  {
    \if_int_compare:w \l_regex_mode_int = \c_zero
      \exp_after:wN \use:n
    \else:
      \if_int_compare:w \l_regex_mode_int = \c_three
        \exp_after:wN \exp_after:wN \exp_after:wN \use:n
      \else:
        \__msg_kernel_error:nn { regex } { c-bad-mode }
        \exp_after:wN \exp_after:wN \exp_after:wN \use_none:n
      \fi:
    \fi:
  }
%    \end{macrocode}
% \end{macro}
%
% \begin{macro}[aux]{\regex_mode_quit_c:}
%   This function changes the mode as it is needed just after a catcode
%   test.
%    \begin{macrocode}
\cs_new_protected:Npn \regex_mode_quit_c:
  {
    \if_int_compare:w \l_regex_mode_int = \c_two
      \l_regex_mode_int = \c_zero
    \else:
      \if_int_compare:w \l_regex_mode_int = \c_six
        \l_regex_mode_int = \c_three
      \fi:
    \fi:
  }
%    \end{macrocode}
% \end{macro}
%
% \subsubsection{Framework}
%
% \begin{macro}[int]{\regex_compile:w, \regex_compile_end:}
%   Used when compiling a user regex or a regex for the |\c{...}| escape
%   sequence within another regex. Start building a token list within a
%   group (with \texttt{x}-expansion at the outset), and set a few
%   variables (group level, catcodes), then start the first branch. At
%   the end, make sure there are no dangling classes nor groups, close
%   the last branch: we are done building \cs{l_regex_internal_regex}.
%    \begin{macrocode}
\cs_new_protected_nopar:Npn \regex_compile:w
  {
    \tl_set_build_x:Nw \l_regex_internal_regex
      \int_zero:N \l_regex_group_level_int
      \int_set_eq:NN \l_regex_default_catcodes_int \c_regex_all_catcodes_int
      \int_set_eq:NN \l_regex_catcodes_int \l_regex_default_catcodes_int
      \cs_set_nopar:Npn \regex_item_equal:n  { \regex_item_caseful_equal:n }
      \cs_set_nopar:Npn \regex_item_range:nn { \regex_item_caseful_range:nn }
      \tl_build_one:n { \regex_branch:n { \if_false: } \fi: }
  }
\cs_new_protected_nopar:Npn \regex_compile_end:
  {
      \regex_if_in_class:TF
        {
          \__msg_kernel_error:nn { regex } { missing-rbrack }
          \use:c { regex_compile_]: }
          \prg_do_nothing: \prg_do_nothing:
        }
        { }
      \if_int_compare:w \l_regex_group_level_int > \c_zero
        \__msg_kernel_error:nnx { regex } { missing-rparen }
          { \int_use:N \l_regex_group_level_int }
        \prg_replicate:nn
          { \l_regex_group_level_int }
          {
              \tl_build_one:n
                {
                  \if_false: { \fi: }
                  \if_false: { \fi: } { 1 } { 0 } \c_true_bool
                }
            \tl_build_end:
            \tl_build_one:o \l_regex_internal_regex
          }
      \fi:
      \tl_build_one:n { \if_false: { \fi: } }
    \tl_build_end:
  }
%    \end{macrocode}
% \end{macro}
%
% \begin{macro}[int]{\regex_compile:n}
%   The compilation is done between \cs{regex_compile:w} and
%   \cs{regex_compile_end:}, starting in mode~$0$. Then
%   \cs{regex_escape_use:nnnn} distinguishes special characters, escaped
%   alphanumerics, and raw characters, interpreting |\a|, |\x| and other
%   sequences. The $4$ trailing \cs{prg_do_nothing:} are needed because
%   some functions defined later look up to $4$ tokens ahead. Before
%   ending, make sure that any |\c{...}| is properly closed.
%    \begin{macrocode}
\cs_new_protected:Npn \regex_compile:n #1
  {
    \regex_compile:w
      \int_set:Nn \tex_escapechar:D { 92 }
      \int_set_eq:NN \l_regex_mode_int \c_zero
      \regex_escape_use:nnnn
        {
          \regex_char_if_special:NTF ##1
            \regex_compile_special:N \regex_compile_raw:N ##1
        }
        {
          \regex_char_if_alphanumeric:NTF ##1
            \regex_compile_escaped:N \regex_compile_raw:N ##1
        }
        { \regex_compile_raw:N ##1 }
        { #1 }
      \prg_do_nothing: \prg_do_nothing:
      \prg_do_nothing: \prg_do_nothing:
      \int_compare:nNnT \l_regex_mode_int < \c_zero
        {
          \__msg_kernel_error:nn { regex } { c-missing-rbrace }
          \regex_compile_end:
          \regex_compile_one:x
            { \regex_item_cs:n { \exp_not:o \l_regex_internal_regex } }
          \prg_do_nothing: \prg_do_nothing:
          \prg_do_nothing: \prg_do_nothing:
        }
    \regex_compile_end:
  }
%    \end{macrocode}
% \end{macro}
%
% \begin{macro}[int]{\regex_compile_escaped:N, \regex_compile_special:N}
%   If the special character or escaped alphanumeric has a particular
%   meaning in regexes, the corresponding function is used. Otherwise,
%   it is interpreted as a raw character. We distinguish special
%   characters from escaped alphanumeric characters because they behave
%   differently when appearing as an end-point of a range.
%    \begin{macrocode}
\cs_new_protected:Npn \regex_compile_special:N #1
  {
    \cs_if_exist_use:cF { regex_compile_#1: }
      { \regex_compile_raw:N #1 }
  }
\cs_new_protected:Npn \regex_compile_escaped:N #1
  {
    \cs_if_exist_use:cF { regex_compile_/#1: }
      { \regex_compile_raw:N #1 }
  }
%    \end{macrocode}
% \end{macro}
%
% \begin{macro}[int]{\regex_compile_one:x}
%   This is used after finding one \enquote{test}, such as |\d|, or a
%   raw character. If that followed a catcode test (\emph{e.g.}, |\cL|),
%   then restore the mode. If we are not in a class, then the test is
%   \enquote{standalone}, and we need to add \cs{regex_class:NnnnN} and
%   search for quantifiers. In any case, insert the test, possibly
%   together with a catcode test if appropriate.
%    \begin{macrocode}
\cs_new_protected:Npn \regex_compile_one:x #1
  {
    \regex_mode_quit_c:
    \regex_if_in_class:TF { }
      {
        \tl_build_one:n
          { \regex_class:NnnnN \c_true_bool { \if_false: } \fi: }
      }
    \tl_build_one:x
      {
        \if_int_compare:w \l_regex_catcodes_int < \c_regex_all_catcodes_int
          \regex_item_catcode:nT { \int_use:N \l_regex_catcodes_int }
            { \exp_not:N \exp_not:n {#1} }
        \else:
          \exp_not:N \exp_not:n {#1}
        \fi:
      }
    \int_set_eq:NN \l_regex_catcodes_int \l_regex_default_catcodes_int
    \regex_if_in_class:TF { } { \regex_compile_quantifier:w }
  }
%    \end{macrocode}
% \end{macro}
%
% \begin{macro}[int]
%   {\regex_compile_abort_tokens:n, \regex_compile_abort_tokens:x}
%   This function places the collected tokens back in the input stream,
%   each as a raw character. Spaces are not preserved.
%    \begin{macrocode}
\cs_new_protected:Npn \regex_compile_abort_tokens:n #1
  {
    \use:x
      {
        \exp_args:No \tl_map_function:nN { \tl_to_str:n {#1} }
          \regex_compile_raw:N
      }
  }
\cs_generate_variant:Nn \regex_compile_abort_tokens:n { x }
%    \end{macrocode}
% \end{macro}
%
% \subsubsection{Quantifers}
%
% \begin{macro}[int]{\regex_compile_quantifier:w}
%   This looks ahead and finds any quantifier (special character equal
%   to either of \texttt{?+*\{}).
%    \begin{macrocode}
\cs_new_protected:Npn \regex_compile_quantifier:w #1#2
  {
    \token_if_eq_meaning:NNTF #1 \regex_compile_special:N
      {
        \cs_if_exist_use:cF { regex_compile_quantifier_#2:w }
          { \regex_compile_quantifier_none: #1 #2 }
      }
      { \regex_compile_quantifier_none: #1 #2 }
  }
%    \end{macrocode}
% \end{macro}
%
% \begin{macro}[aux]{\regex_compile_quantifier_none:}
% \begin{macro}[aux]{\regex_compile_quantifier_abort:xNN}
%   Those functions are called whenever there is no quantifier, or a
%   braced construction is invalid (equivalent to no quantifier, and
%   whatever characters were grabbed are left raw).
%    \begin{macrocode}
\cs_new_protected:Npn \regex_compile_quantifier_none:
  { \tl_build_one:n { \if_false: { \fi: } { 1 } { 0 } \c_false_bool } }
\cs_new_protected:Npn \regex_compile_quantifier_abort:xNN #1#2#3
  {
    \regex_compile_quantifier_none:
    \__msg_kernel_warning:nnxx { regex } { invalid-quantifier } {#1} {#3}
    \regex_compile_abort_tokens:x {#1}
    #2 #3
  }
%    \end{macrocode}
% \end{macro}
% \end{macro}
%
% \begin{macro}[aux]{\regex_compile_quantifier_lazyness:nnNN}
%   Once the \enquote{main} quantifier (\texttt{?}, \texttt{*},
%   \texttt{+} or a braced construction) is found, we check whether it
%   is lazy (followed by a question mark). We then add to the compiled
%   regex a closing brace (ending \cs{regex_class:NnnnN} and friends),
%   the start-point of the range, its end-point, and a boolean,
%   \texttt{true} for lazy and \texttt{false} for greedy operators.
%    \begin{macrocode}
\cs_new_protected:Npn \regex_compile_quantifier_lazyness:nnNN #1#2#3#4
  {
    \str_if_eq:nnTF { #3 #4 } { \regex_compile_special:N ? }
      { \tl_build_one:n { \if_false: { \fi: } { #1 } { #2 } \c_true_bool } }
      {
        \tl_build_one:n { \if_false: { \fi: } { #1 } { #2 } \c_false_bool }
        #3 #4
      }
  }
%    \end{macrocode}
% \end{macro}
%
% \begin{macro}[aux]
%   {
%     \regex_compile_quantifier_?:w,
%     \regex_compile_quantifier_*:w,
%     \regex_compile_quantifier_+:w
%   }
%   For each \enquote{basic} quantifier, |?|, |*|, |+|, feed the correct
%   arguments to \cs{regex_compile_quantifier_lazyness:nnNN}, $-1$ means
%   that there is no upper bound on the number of repetitions.
%    \begin{macrocode}
\cs_new_protected_nopar:cpn { regex_compile_quantifier_?:w }
  { \regex_compile_quantifier_lazyness:nnNN { 0 } { 1 } }
\cs_new_protected_nopar:cpn { regex_compile_quantifier_*:w }
  { \regex_compile_quantifier_lazyness:nnNN { 0 } { -1 } }
\cs_new_protected_nopar:cpn { regex_compile_quantifier_+:w }
  { \regex_compile_quantifier_lazyness:nnNN { 1 } { -1 } }
%    \end{macrocode}
% \end{macro}
%
% \begin{macro}[aux]+\regex_compile_quantifier_{:w+ ^^A}
% \begin{macro}[aux]
%   {
%     \regex_compile_quantifier_braced_i:w,
%     \regex_compile_quantifier_braced_ii:w,
%     \regex_compile_quantifier_braced_iii:w,
%   }
%   Three possible syntaxes: \texttt{\{\meta{int}\}},
%   \texttt{\{\meta{int},\}}, or \texttt{\{\meta{int},\meta{int}\}}. Any
%   other syntax causes us to abort and put whatever we collected back
%   in the input stream, as \texttt{raw} characters, including the
%   opening brace. Grab a number into \cs{l_regex_internal_a_int}. If
%   the number is followed by a right brace, the range is $[a,a]$. If
%   followed by a comma, grab one more number, and call the \texttt{_ii}
%   or \texttt{_iii} auxiliary. Those auxiliaries check for a closing
%   brace, leading to the range $[a,\infty]$ or $[a,b]$, encoded as
%   $\{a\}\{-1\}$ and $\{a\}\{b-a\}$.
%    \begin{macrocode}
\cs_new_protected:cpn { regex_compile_quantifier_ \c_lbrace_str :w }
  {
    \regex_get_digits:NTFw \l_regex_internal_a_int
      { \regex_compile_quantifier_braced_i:w }
      { \regex_compile_quantifier_abort:xNN { \c_lbrace_str } }
  }
\cs_new_protected:Npn \regex_compile_quantifier_braced_i:w #1#2
  {
    \str_case_x:nnn { #1 #2 }
      {
        { \regex_compile_special:N \c_rbrace_str }
          {
            \exp_args:No \regex_compile_quantifier_lazyness:nnNN
              { \int_use:N \l_regex_internal_a_int } { 0 }
          }
        { \regex_compile_special:N , }
          {
            \regex_get_digits:NTFw \l_regex_internal_b_int
              { \regex_compile_quantifier_braced_iii:w }
              { \regex_compile_quantifier_braced_ii:w }
          }
      }
      {
        \regex_compile_quantifier_abort:xNN
          { \c_lbrace_str \int_use:N \l_regex_internal_a_int }
        #1 #2
      }
  }
\cs_new_protected:Npn \regex_compile_quantifier_braced_ii:w #1#2
  {
    \str_if_eq:xxTF
      { #1 #2 } { \regex_compile_special:N \c_rbrace_str }
      {
        \exp_args:No \regex_compile_quantifier_lazyness:nnNN
          { \int_use:N \l_regex_internal_a_int } { -1 }
      }
      {
        \regex_compile_quantifier_abort:xNN
          { \c_lbrace_str \int_use:N \l_regex_internal_a_int , }
        #1 #2
      }
  }
\cs_new_protected:Npn \regex_compile_quantifier_braced_iii:w #1#2
  {
    \str_if_eq:xxTF
      { #1 #2 } { \regex_compile_special:N \c_rbrace_str }
      {
        \if_int_compare:w \l_regex_internal_a_int > \l_regex_internal_b_int
          \__msg_kernel_error:nnxx { regex } { backwards-quantifier }
            { \int_use:N \l_regex_internal_a_int }
            { \int_use:N \l_regex_internal_b_int }
          \int_zero:N \l_regex_internal_b_int
        \else:
          \int_sub:Nn \l_regex_internal_b_int \l_regex_internal_a_int
        \fi:
        \exp_args:Noo \regex_compile_quantifier_lazyness:nnNN
          { \int_use:N \l_regex_internal_a_int }
          { \int_use:N \l_regex_internal_b_int }
      }
      {
        \regex_compile_quantifier_abort:xNN
          {
            \c_lbrace_str
            \int_use:N \l_regex_internal_a_int ,
            \int_use:N \l_regex_internal_b_int
          }
        #1 #2
      }
  }
%    \end{macrocode}
% \end{macro}
% \end{macro}
%
% \subsubsection{Raw characters}
%
% \begin{macro}[int]{\regex_compile_raw_error:N}
%   Within character classes, and following catcode tests, some escaped
%   alphanumeric sequences such as |\b| do not have any meaning. They
%   are replaced by a raw character, after spitting out an error.
%    \begin{macrocode}
\cs_new_protected:Npn \regex_compile_raw_error:N #1
  {
    \__msg_kernel_error:nnx { regex } { bad-escape } {#1}
    \regex_compile_raw:N #1
  }
%    \end{macrocode}
% \end{macro}
%
% \begin{macro}[int]{\regex_compile_raw:N}
%   If we are in a character class and the next character is an
%   unescaped dash, this denotes a range. Otherwise, the current
%   character |#1| matches itself.
%    \begin{macrocode}
\cs_new_protected:Npn \regex_compile_raw:N #1#2#3
  {
    \regex_if_in_class:TF
      {
        \str_if_eq:nnTF {#2#3} { \regex_compile_special:N - }
          { \regex_compile_range:Nw #1 }
          {
            \regex_compile_one:x
              { \regex_item_equal:n { \__int_value:w `#1 ~ } }
            #2 #3
          }
      }
      {
        \regex_compile_one:x
          { \regex_item_equal:n { \__int_value:w `#1 ~ } }
        #2 #3
      }
  }
%    \end{macrocode}
% \end{macro}
%
% \begin{macro}[aux]{\regex_compile_range:Nw, \regex_if_end_range:NNTF}
%   We have just read a raw character followed by a dash; this should be
%   followed by an end-point for the range. Valid end-points are: any
%   raw character; any special character, except a right bracket. In
%   particular, escaped characters are forbidden.
%    \begin{macrocode}
\prg_new_protected_conditional:Npnn \regex_if_end_range:NN #1#2 { TF }
  {
    \if_meaning:w \regex_compile_raw:N #1
      \prg_return_true:
    \else:
      \if_meaning:w \regex_compile_special:N #1
        \if_charcode:w ] #2
          \prg_return_false:
        \else:
          \prg_return_true:
        \fi:
      \else:
        \prg_return_false:
      \fi:
    \fi:
  }
\cs_new_protected:Npn \regex_compile_range:Nw #1#2#3
  {
    \regex_if_end_range:NNTF #2 #3
      {
        \if_int_compare:w `#1 > `#3 \exp_stop_f:
          \__msg_kernel_error:nnxx { regex } { range-backwards } {#1} {#3}
        \else:
          \tl_build_one:x
            {
              \if_int_compare:w `#1 = `#3 \exp_stop_f:
                \regex_item_equal:n
              \else:
                \regex_item_range:nn { \__int_value:w `#1 ~ }
              \fi:
              { \__int_value:w `#3 ~ }
            }
        \fi:
      }
      {
        \__msg_kernel_warning:nnxx { regex } { range-missing-end }
          {#1} { \c_backslash_str #3 }
        \tl_build_one:x
          {
            \regex_item_equal:n { \__int_value:w `#1 ~ }
            \regex_item_equal:n { \__int_value:w `- ~ }
          }
        #2#3
      }
  }
%    \end{macrocode}
% \end{macro}
%
% \subsubsection{Character properties}
%
% \begin{macro}[aux]{\regex_compile_.:, \regex_prop_.:}
%   In a class, the dot has no special meaning. Outside, insert
%   \cs{regex_prop_.:}, which matches any character or control
%   sequence, and refuses $-2$ (end-marker).
%    \begin{macrocode}
\cs_new_protected_nopar:cpx { regex_compile_.: }
  {
    \exp_not:N \regex_if_in_class:TF
      { \regex_compile_raw:N . }
      { \regex_compile_one:x \exp_not:c { regex_prop_.: } }
  }
\cs_new_protected_nopar:cpn { regex_prop_.: }
  {
    \if_int_compare:w \l_regex_current_char_int > - \c_two
      \exp_after:wN \regex_break_true:w
    \fi:
  }
%    \end{macrocode}
% \end{macro}
%
% \begin{macro}[aux]
%   {
%     \regex_compile_/d:, \regex_compile_/D:,
%     \regex_compile_/h:, \regex_compile_/H:,
%     \regex_compile_/s:, \regex_compile_/S:,
%     \regex_compile_/v:, \regex_compile_/V:,
%     \regex_compile_/w:, \regex_compile_/W:,
%     \regex_compile_/N:,
%   }
%   The constants \cs{regex_prop_d:}, \emph{etc.} hold
%   a list of tests which match the corresponding character
%   class, and jump to the \cs{regex_break_point:TF} marker.
%   As for a normal character, we check for quantifiers.
%    \begin{macrocode}
\cs_set_protected:Npn \regex_tmp:w #1#2
  {
    \cs_new_protected_nopar:cpx { regex_compile_/#1: }
      { \regex_compile_one:x \exp_not:c { regex_prop_#1: } }
    \cs_new_protected_nopar:cpx { regex_compile_/#2: }
      {
        \regex_compile_one:x
          { \regex_item_reverse:n \exp_not:c { regex_prop_#1: } }
      }
  }
\regex_tmp:w d D
\regex_tmp:w h H
\regex_tmp:w s S
\regex_tmp:w v V
\regex_tmp:w w W
\cs_new_protected_nopar:cpn { regex_compile_/N: }
  { \regex_compile_one:x \regex_prop_N: }
%    \end{macrocode}
% \end{macro}
%
% \subsubsection{Anchoring and simple assertions}
%
% \begin{macro}[aux]{\regex_compile_anchor:NF}
% \begin{macro}[aux]+\regex_compile_^:+
% \begin{macro}[aux]{\regex_compile_/A:, \regex_compile_/G:}
% \begin{macro}[aux]+\regex_compile_$:+
% \begin{macro}[aux]{\regex_compile_/Z:, \regex_compile_/z:}
%   In modes where assertions are allowed, anchor to the start of the
%   query, the start of the match, or the end of the query, depending on
%   the integer |#1|. In other modes, |#2| treats the character as raw,
%   with an error for escaped letters (|$| is valid in a class, but |\A|
%   is definitely a mistake on the user's part).
%    \begin{macrocode}
\cs_new_protected:Npn \regex_compile_anchor:NF #1#2
  {
    \regex_if_in_class_or_catcode:TF {#2}
      {
        \tl_build_one:n
          { \regex_assertion:Nn \c_true_bool { \regex_anchor:N #1 } }
      }
  }
\cs_set_protected:Npn \regex_tmp:w #1#2
  {
    \cs_new_protected_nopar:cpn { regex_compile_/#1: }
      { \regex_compile_anchor:NF #2 { \regex_compile_raw_error:N #1 } }
  }
\regex_tmp:w A \l_regex_min_pos_int
\regex_tmp:w G \l_regex_start_pos_int
\regex_tmp:w Z \l_regex_max_pos_int
\regex_tmp:w z \l_regex_max_pos_int
\cs_set_protected:Npn \regex_tmp:w #1#2
  {
    \cs_new_protected_nopar:cpn { regex_compile_#1: }
      { \regex_compile_anchor:NF #2 { \regex_compile_raw:N #1 } }
  }
\exp_args:Nx \regex_tmp:w { \iow_char:N \^ } \l_regex_min_pos_int
\exp_args:Nx \regex_tmp:w { \iow_char:N \$ } \l_regex_max_pos_int
%    \end{macrocode}
% \end{macro}
% \end{macro}
% \end{macro}
% \end{macro}
% \end{macro}
%
% \begin{macro}[aux]{\regex_compile_/b:, \regex_compile_/B:}
%   Contrarily to |^| and |$|, which could be implemented without really
%   knowing what precedes in the token list, this requires more
%   information, namely, the knowledge of the last character code.
%    \begin{macrocode}
\cs_new_protected_nopar:cpn { regex_compile_/b: }
  {
    \regex_if_in_class_or_catcode:TF
      { \regex_compile_raw_error:N b }
      {
        \tl_build_one:n
          { \regex_assertion:Nn \c_true_bool { \regex_b_test: } }
      }
  }
\cs_new_protected_nopar:cpn { regex_compile_/B: }
  {
    \regex_if_in_class_or_catcode:TF
      { \regex_compile_raw_error:N B }
      {
        \tl_build_one:n
          { \regex_assertion:Nn \c_false_bool { \regex_b_test: } }
      }
  }
%    \end{macrocode}
% \end{macro}
%
% \subsubsection{Character classes}
%
% \begin{macro}[aux]{\regex_compile_]:}
%   Outside a class, right brackets have no meaning. In a class, change
%   the mode ($m\to (m-15)/13$, truncated) to reflect the fact that we
%   are leaving the class. Look for quantifiers, unless we are still in
%   a class after leaving one (the case of |[...\cL[...]...]|).
%   quantifiers.
%    \begin{macrocode}
\cs_new_protected:cpn { regex_compile_]: }
  {
    \regex_if_in_class:TF
      {
        \if_int_compare:w \l_regex_mode_int > \c_sixteen
          \tl_build_one:n { \if_false: { \fi: } }
        \fi:
        \tex_advance:D \l_regex_mode_int - \c_fifteen
        \tex_divide:D \l_regex_mode_int \c_thirteen
        \if_int_odd:w \l_regex_mode_int \else:
          \exp_after:wN \regex_compile_quantifier:w
        \fi:
      }
      { \regex_compile_raw:N ] }
  }
%    \end{macrocode}
% \end{macro}
%
% \begin{macro}[aux]{\regex_compile_[:}
%   In a class, left brackets might introduce a \textsc{posix} character
%   class, or mean nothing. Immediately following |\c|\meta{category},
%   we must insert the apropriate catcode test, then parse the class; we
%   pre-expand the catcode as an optimization. Otherwise (modes $0$,
%   $-2$ and $-6$) just parse the class.  The mode is updated later.
%    \begin{macrocode}
\cs_new_protected_nopar:cpn { regex_compile_[: }
  {
    \regex_if_in_class:TF
      { \regex_compile_class_posix_test:w }
      {
        \regex_if_within_catcode:TF
          {
            \exp_after:wN \regex_compile_class_catcode:w
              \int_use:N \l_regex_catcodes_int ;
          }
          { \regex_compile_class_normal:w }
      }
  }
%    \end{macrocode}
% \end{macro}
%
% \begin{macro}[aux]{\regex_compile_class_normal:w}
%   In the \enquote{normal} case, we will insert \cs{regex_class:NnnnN}
%   \meta{boolean} in the compiled code. The \meta{boolean} is true for
%   positive classes, and false for negative classes, characterized by a
%   leading |^|. The auxiliary \cs{regex_compile_class:TFNN} also
%   checks for a leading |]| which has a special meaning.
%    \begin{macrocode}
\cs_new_protected_nopar:Npn \regex_compile_class_normal:w
  {
    \regex_compile_class:TFNN
      { \regex_class:NnnnN \c_true_bool }
      { \regex_class:NnnnN \c_false_bool }
  }
%    \end{macrocode}
% \end{macro}
%
% \begin{macro}[aux]{\regex_compile_class_catcode:w}
%   This function is called for a left bracket in modes $2$ or $6$
%   (catcode test, and catcode test within a class). In mode $2$ the
%   whole construction needs to be put in a class (like single
%   character). Then determine if the class is positive or negative,
%   inserting \cs{regex_item_catcode:nT} or the \texttt{reverse} variant
%   as appropriate, each with the current catcodes bitmap |#1| as an
%   argument, and reset the catcodes.
%    \begin{macrocode}
\cs_new_protected:Npn \regex_compile_class_catcode:w #1;
  {
    \if_int_compare:w \l_regex_mode_int = \c_two
      \tl_build_one:n
        { \regex_class:NnnnN \c_true_bool { \if_false: } \fi: }
    \fi:
    \int_set_eq:NN \l_regex_catcodes_int \l_regex_default_catcodes_int
    \regex_compile_class:TFNN
      { \regex_item_catcode:nT {#1} }
      { \regex_item_catcode_reverse:nT {#1} }
  }
%    \end{macrocode}
% \end{macro}
%
% \begin{macro}[aux]
%   {\regex_compile_class:TFNN, \regex_compile_class_ii:NN}
%   If the first character is |^|, then the class is negative (use
%   |#2|), otherwise it is positive (use |#1|). If the next character
%   is a right bracket, then it should be changed to a raw one.
%    \begin{macrocode}
\cs_new_protected:Npn \regex_compile_class:TFNN #1#2#3#4
  {
    \l_regex_mode_int = \__int_value:w \l_regex_mode_int 3 \exp_stop_f:
    \str_if_eq:nnTF { #3 #4 } { \regex_compile_special:N ^ }
      {
        \tl_build_one:n { #2 { \if_false: } \fi: }
        \regex_compile_class_ii:NN
      }
      {
        \tl_build_one:n { #1 { \if_false: } \fi: }
        \regex_compile_class_ii:NN #3 #4
      }
  }
\cs_new_protected:Npn \regex_compile_class_ii:NN #1#2
  {
    \token_if_eq_charcode:NNTF #2 ]
      { \regex_compile_raw:N #2 }
      { #1 #2 }
  }
%    \end{macrocode}
% \end{macro}
%
% \begin{macro}[aux]
%   {
%     \regex_compile_class_posix_test:w,
%     \regex_compile_class_posix:NNNNw,
%     \regex_compile_class_posix_loop:w,
%     \regex_compile_class_posix_end:w
%   }
%   Here we check for a syntax such as |[:alpha:]|. We also detect |[=|
%   and |[.| which have a meaning in \textsc{posix} regular expressions,
%   but are not implemented in \pkg{l3regex}. In case we see |[:|, grab
%   raw characters until hopefully reaching |:]|. If that's missing, or
%   the \textsc{posix} class is unknown, abort. If all is right, add the
%   test to the current class, with an extra \cs{regex_item_reverse:n}
%   for negative classes.
%    \begin{macrocode}
\cs_new_protected:Npn \regex_compile_class_posix_test:w #1#2
  {
    \token_if_eq_meaning:NNT \regex_compile_special:N #1
      {
        \str_case:nnn { #2 }
          {
            : { \regex_compile_class_posix:NNNNw }
            = { \__msg_kernel_warning:nnx { regex } { posix-unsupported } { = } }
            . { \__msg_kernel_warning:nnx { regex } { posix-unsupported } { . } }
          }
          { }
      }
    \regex_compile_raw:N [ #1 #2
  }
\cs_new_protected:Npn \regex_compile_class_posix:NNNNw #1#2#3#4#5#6
  {
    \str_if_eq:nnTF { #5 #6 } { \regex_compile_special:N ^ }
      {
        \bool_set_false:N \l_regex_internal_bool
        \tl_set:Nx \l_regex_internal_a_tl { \if_false: } \fi:
          \regex_compile_class_posix_loop:w
      }
      {
        \bool_set_true:N \l_regex_internal_bool
        \tl_set:Nx \l_regex_internal_a_tl { \if_false: } \fi:
          \regex_compile_class_posix_loop:w #5 #6
      }
  }
\cs_new:Npn \regex_compile_class_posix_loop:w #1#2
  {
    \token_if_eq_meaning:NNTF \regex_compile_raw:N #1
      { #2 \regex_compile_class_posix_loop:w }
      { \if_false: { \fi: } \regex_compile_class_posix_end:w #1 #2 }
  }
\cs_new_protected:Npn \regex_compile_class_posix_end:w #1#2#3#4
  {
    \str_if_eq:nnTF { #1 #2 #3 #4 }
      { \regex_compile_special:N : \regex_compile_special:N ] }
      {
        \cs_if_exist:cTF { regex_posix_ \l_regex_internal_a_tl : }
          {
            \regex_compile_one:x
              {
                \bool_if:NF \l_regex_internal_bool \regex_item_reverse:n
                \exp_not:c { regex_posix_ \l_regex_internal_a_tl : }
              }
          }
          {
            \__msg_kernel_warning:nnx { regex } { posix-unknown }
              { \l_regex_internal_a_tl }
            \regex_compile_abort_tokens:x
              {
                [: \bool_if:NF \l_regex_internal_bool { ^ }
                \l_regex_internal_a_tl :]
              }
          }
      }
      {
        \__msg_kernel_error:nnxx { regex } { posix-missing-close }
          { [: \l_regex_internal_a_tl } { #2 #4 }
        \regex_compile_abort_tokens:x { [: \l_regex_internal_a_tl }
        #1 #2 #3 #4
      }
  }
%    \end{macrocode}
% \end{macro}
%
% \subsubsection{Groups and alternations}
%
% \begin{macro}[aux]{\regex_compile_group_begin:N, \regex_compile_group_end:}
%   The contents of a regex group are turned into compiled code in
%   \cs{l_regex_internal_regex}, which ends up with items of the form
%   \cs{regex_branch:n} \Arg{concatenation}. This construction is done
%   using \pkg{l3tl-build} within a \TeX{} group, which automatically
%   makes sure that options (case-sensitivity and default catcode) are
%   reset at the end of the group. The argument |#1| is
%   \cs{regex_group:nnnN} or a variant thereof. A small subtelty to
%   support |\cL(abc)| as a shorthand for |(\cLa\cLb\cLc)|: exit any
%   pending catcode test, save the category code at the start of the
%   group as the default catcode for that group, and make sure that the
%   catcode is restored to the default outside the group.
%    \begin{macrocode}
\cs_new_protected:Npn \regex_compile_group_begin:N #1
  {
    \tl_build_one:n { #1 { \if_false: } \fi: }
    \regex_mode_quit_c:
    \tl_set_build:Nw \l_regex_internal_regex
      \int_set_eq:NN \l_regex_default_catcodes_int \l_regex_catcodes_int
      \int_incr:N \l_regex_group_level_int
      \tl_build_one:n { \regex_branch:n { \if_false: } \fi: }
  }
\cs_new_protected:Npn \regex_compile_group_end:
  {
    \if_int_compare:w \l_regex_group_level_int > \c_zero
        \tl_build_one:n { \if_false: { \fi: } }
      \tl_build_end:
      \int_set_eq:NN \l_regex_catcodes_int \l_regex_default_catcodes_int
      \tl_build_one:o \l_regex_internal_regex
      \exp_after:wN \regex_compile_quantifier:w
    \else:
      \__msg_kernel_warning:nn { regex } { extra-rparen }
      \exp_after:wN \regex_compile_raw:N \exp_after:wN )
    \fi:
  }
%    \end{macrocode}
% \end{macro}
%
% \begin{macro}[aux]{\regex_compile_(:}
%   In a class, parentheses are not special. Outside, check for a |?|,
%   denoting special groups, and run the code for the corresponding
%   special group.
%    \begin{macrocode}
\cs_new_protected_nopar:cpn { regex_compile_(: }
  {
    \regex_if_in_class:TF { \regex_compile_raw:N ( }
      { \regex_compile_lparen:w }
  }
\cs_new_protected:Npn \regex_compile_lparen:w #1#2#3#4
  {
    \str_if_eq:nnTF { #1 #2 } { \regex_compile_special:N ? }
      {
        \cs_if_exist_use:cF
          { regex_compile_special_group_\token_to_str:N #4 :w }
          {
            \__msg_kernel_warning:nnx { regex } { special-group-unknown }
              { (? \token_to_str:N #4 }
            \regex_compile_group_begin:N \regex_group:nnnN
              \regex_compile_raw:N ? #3 #4
          }
      }
      {
        \regex_compile_group_begin:N \regex_group:nnnN
          #1 #2 #3 #4
      }
  }
%    \end{macrocode}
% \end{macro}
%
% \begin{macro}[aux]+\regex_compile_|:+
%   In a class, the pipe is not special. Otherwise, end the current
%   branch and open another one.
%    \begin{macrocode}
\cs_new_protected_nopar:cpn { regex_compile_|: }
  {
    \regex_if_in_class:TF { \regex_compile_raw:N | }
      {
        \tl_build_one:n
          { \if_false: { \fi: } \regex_branch:n { \if_false: } \fi: }
      }
  }
%    \end{macrocode}
% \end{macro}
%
% \begin{macro}[aux]{\regex_compile_):}
%   Within a class, parentheses are not special. Outside, close a group.
%    \begin{macrocode}
\cs_new_protected_nopar:cpn { regex_compile_): }
  {
    \regex_if_in_class:TF { \regex_compile_raw:N ) }
      { \regex_compile_group_end: }
  }
%    \end{macrocode}
% \end{macro}
%
% \begin{macro}[aux]{\regex_compile_special_group_::w}
% \begin{macro}[aux]+\regex_compile_special_group_|:w+
%   Non-capturing, and resetting groups are easy to take care of during
%   compilation; for those groups, the harder parts will come when
%   building.
%    \begin{macrocode}
\cs_new_protected_nopar:cpn { regex_compile_special_group_::w }
  { \regex_compile_group_begin:N \regex_group_no_capture:nnnN }
\cs_new_protected_nopar:cpn { regex_compile_special_group_|:w }
  { \regex_compile_group_begin:N \regex_group_resetting:nnnN }
%    \end{macrocode}
% \end{macro}
% \end{macro}
%
% \begin{macro}[aux]
%   {\regex_compile_special_group_i:w, \regex_compile_special_group_-:w}
%   The match can be made case-insensitive by setting the option with
%   \texttt{(?i)}; the original behaviour is restored by \texttt{(?-i)}.
%   This is the only supported option.
%    \begin{macrocode}
\cs_new_protected:Npn \regex_compile_special_group_i:w #1#2
  {
    \str_if_eq:nnTF { #1 #2 } { \regex_compile_special:N ) }
      {
        \cs_set_nopar:Npn \regex_item_equal:n  { \regex_item_caseless_equal:n }
        \cs_set_nopar:Npn \regex_item_range:nn { \regex_item_caseless_range:nn }
      }
      {
        \__msg_kernel_warning:nnx { regex } { unknown-option } { (?i #2 }
        \regex_compile_raw:N (
        \regex_compile_raw:N ?
        \regex_compile_raw:N i
        #1 #2
      }
  }
\cs_new_protected_nopar:cpn { regex_compile_special_group_-:w } #1#2#3#4
  {
    \str_if_eq:nnTF { #1 #2 #3 #4 }
      { \regex_compile_raw:N i \regex_compile_special:N ) }
      {
        \cs_set_nopar:Npn \regex_item_equal:n  { \regex_item_caseful_equal:n }
        \cs_set_nopar:Npn \regex_item_range:nn { \regex_item_caseful_range:nn }
      }
      {
        \__msg_kernel_warning:nnx { regex } { unknown-option } { (?-#2#4 }
        \regex_compile_raw:N (
        \regex_compile_raw:N ?
        \regex_compile_raw:N -
        #1 #2 #3 #4
      }
  }
%    \end{macrocode}
% \end{macro}
%
% \subsubsection{Catcodes and csnames}
%
% \begin{macro}[aux]{\regex_compile_/c:, \regex_compile_c_test:NN}
%   The |\c| escape sequence can be followed by a capital letter
%   representing a character category, by a left bracket which starts a
%   list of categories, or by a brace group holding a regular expression
%   for a control sequence name. Otherwise, raise an error.
%    \begin{macrocode}
\cs_new_protected:cpn { regex_compile_/c: }
  { \regex_chk_c_allowed:T { \regex_compile_c_test:NN } }
\cs_new_protected:Npn \regex_compile_c_test:NN #1#2
  {
    \token_if_eq_meaning:NNTF #1 \regex_compile_raw:N
      {
        \int_if_exist:cTF { c_regex_catcode_#2_int }
          {
            \int_set_eq:Nc \l_regex_catcodes_int { c_regex_catcode_#2_int }
            \l_regex_mode_int
              = \if_case:w \l_regex_mode_int \c_two \else: \c_six \fi:
          }
      }
      { \cs_if_exist_use:cF { regex_compile_c_#2:w } }
          {
            \__msg_kernel_error:nnx { regex } { c-missing-category } {#2}
            #1 #2
          }
  }
%    \end{macrocode}
% \end{macro}
%
% \begin{macro}[aux]
%   {
%     \regex_compile_c_[:w,
%     \regex_compile_c_lbrack_loop:NN,
%     \regex_compile_c_lbrack_add:N,
%     \regex_compile_c_lbrack_end:,
%   }
%   When encountering |\c[|, the task is to collect uppercase letters
%   representing character categories. First check for |^| which negates
%   the list of category codes.
%    \begin{macrocode}
\cs_new_protected:cpn { regex_compile_c_[:w } #1#2
  {
    \l_regex_mode_int
      = \if_case:w \l_regex_mode_int \c_two \else: \c_six \fi:
    \int_zero:N \l_regex_catcodes_int
    \str_if_eq:nnTF { #1 #2 } { \regex_compile_special:N ^ }
      {
        \bool_set_false:N \l_regex_catcodes_bool
        \regex_compile_c_lbrack_loop:NN
      }
      {
        \bool_set_true:N \l_regex_catcodes_bool
        \regex_compile_c_lbrack_loop:NN
        #1 #2
      }
  }
\cs_new_protected:Npn \regex_compile_c_lbrack_loop:NN #1#2
  {
    \token_if_eq_meaning:NNTF #1 \regex_compile_raw:N
      {
        \int_if_exist:cTF { c_regex_catcode_#2_int }
          {
            \exp_args:Nc \regex_compile_c_lbrack_add:N
              { c_regex_catcode_#2_int }
            \regex_compile_c_lbrack_loop:NN
          }
      }
      {
        \token_if_eq_charcode:NNTF #2 ]
          { \regex_compile_c_lbrack_end: }
      }
          {
            \__msg_kernel_error:nnx { regex } { c-missing-rbrack } {#2}
            \regex_compile_c_lbrack_end:
            #1 #2
          }
  }
\cs_new_protected:Npn \regex_compile_c_lbrack_add:N #1
  {
    \if_int_odd:w \__int_eval:w \l_regex_catcodes_int / #1 \__int_eval_end:
    \else:
      \tex_advance:D \l_regex_catcodes_int #1
    \fi:
  }
\cs_new_protected_nopar:Npn \regex_compile_c_lbrack_end:
  {
    \if_meaning:w \c_false_bool \l_regex_catcodes_bool
      \int_set:Nn \l_regex_catcodes_int
        { \c_regex_all_catcodes_int - \l_regex_catcodes_int }
    \fi:
  }
%    \end{macrocode}
% \end{macro}
%
% \begin{macro}+\regex_compile_c_{:+
%   The case of a left brace is easy, based on what we have done so far:
%   in a group, compile the regular expression, after changing the mode
%   to forbid nesting |\c|. Additionally, disable submatch tracking
%   since groups don't escape the scope of |\c{...}|.
%    \begin{macrocode}
\cs_new_protected_nopar:cpn { regex_compile_c_ \c_lbrace_str :w }
  {
    \regex_compile:w
      \regex_disable_submatches:
      \l_regex_mode_int
        = - \if_case:w \l_regex_mode_int \c_two \else: \c_six \fi:
  }
%    \end{macrocode}
% \end{macro}
%
% \begin{macro}+\regex_compile_}:+
%   Non-escaped right braces are only special if they appear when
%   compiling the regular expression for a csname, but not within a
%   class: |\c{[}{]}| matches the control sequences |\}| and
%   |\{|\ldots{} Admittedly, that would be better done as
%   |\c{[{}]}|. So, end compiling the inner regex (this closes any
%   dangling class or group).  Then insert the corresponding test in the
%   outer regex.
%    \begin{macrocode}
\cs_new_protected:cpn { regex_compile_ \c_rbrace_str : }
  {
    \regex_if_in_cs:TF
      {
        \regex_compile_end:
        \regex_compile_one:x
          { \regex_item_cs:n { \exp_not:o \l_regex_internal_regex } }
      }
      { \exp_after:wN \regex_compile_raw:N \c_rbrace_str }
  }
%    \end{macrocode}
% \end{macro}
%
% \subsubsection{Raw token lists with \cs{u}}
%
% \begin{macro}[aux]{\regex_compile_/u:}
% \begin{macro}[aux, EXP]{\regex_compile_u_loop:NN}
%   The |\u| escape is invalid in classes and directly following a
%   catcode test. Otherwise, it must be followed by a left brace. We
%   then collect the characters for the argument of |\u| within an
%   \texttt{x}-expanding assignment. In principle we could just wait to
%   encounter a right brace, but this is unsafe: if the right brace is
%   missing, then we will reach the end-markers of the regex, and
%   continue, leading to obscure fatal errors. Instead, we only allow
%   raw and special characters, and stop when encountering a special
%   right brace, any escaped character, or the end-marker.
%    \begin{macrocode}
\cs_new_protected:cpn { regex_compile_/u: } #1#2
  {
    \regex_if_in_class_or_catcode:TF
      { \regex_compile_raw_error:N u #1 #2 }
      {
        \str_if_eq:xxTF {#1#2} { \regex_compile_special:N \c_lbrace_str }
          {
            \tl_set:Nx \l_regex_internal_a_tl { \if_false: } \fi:
            \regex_compile_u_loop:NN
          }
          {
            \__msg_kernel_error:nn { regex } { u-missing-lbrace }
            \regex_compile_raw:N u #1 #2
          }
      }
  }
\cs_new:Npn \regex_compile_u_loop:NN #1#2
  {
    \token_if_eq_meaning:NNTF #1 \regex_compile_raw:N
      { #2 \regex_compile_u_loop:NN }
      {
        \token_if_eq_meaning:NNTF #1 \regex_compile_special:N
          {
            \exp_after:wN \token_if_eq_charcode:NNTF \c_rbrace_str #2
              { \if_false: { \fi: } \regex_compile_u_end: }
              { #2 \regex_compile_u_loop:NN }
          }
          {
            \if_false: { \fi: }
            \__msg_kernel_error:nnx { regex } { u-missing-rbrace } {#2}
            \regex_compile_u_end:
            #1 #2
          }
      }
  }
%    \end{macrocode}
% \end{macro}
% \end{macro}
%
% \begin{macro}[aux]{\regex_compile_u_end:}
%   Once we have extracted the variable's name, we store the contents of
%   that variable in \cs{l_regex_internal_a_tl}. The behaviour of |\u|
%   then depends on whether we are within a |\c{...}| escape (in this
%   case, the variable is turned to a string), or not.
%    \begin{macrocode}
\cs_new_protected:Npn \regex_compile_u_end:
  {
    \tl_set:Nv \l_regex_internal_a_tl { \l_regex_internal_a_tl }
    \if_int_compare:w \l_regex_mode_int = \c_zero
      \regex_compile_u_not_cs:
    \else:
      \regex_compile_u_in_cs:
    \fi:
  }
%    \end{macrocode}
% \end{macro}
%
% \begin{macro}[aux]{\regex_compile_u_in_cs:}
%   When |\u| appears within a control sequence, we convert the variable
%   to a string with escaped spaces. Then for each character insert a
%   class matching exactly that character, once.
%    \begin{macrocode}
\cs_new_protected:Npn \regex_compile_u_in_cs:
  {
    \exp_args:NNo \str_gset_other:Nn \g_regex_internal_tl
      { \l_regex_internal_a_tl }
    \tl_build_one:x
      {
        \tl_map_function:NN \g_regex_internal_tl
          \regex_compile_u_in_cs_aux:n
      }
  }
\cs_new:Npn \regex_compile_u_in_cs_aux:n #1
  {
    \regex_class:NnnnN \c_true_bool
      { \regex_item_caseful_equal:n { \__int_value:w `#1 } }
      { 1 } { 0 } \c_false_bool
  }
%    \end{macrocode}
% \end{macro}
%
% \begin{macro}[aux]{\regex_compile_u_not_cs:}
%   In mode $0$, the |\u| escape adds one state to the NFA for each
%   token in \cs{l_regex_internal_a_tl}. If a given \meta{token} is a
%   control sequence, then insert a string comparison test, otherwise,
%   \cs{regex_item_exact:nn} which compares catcode and character code.
%    \begin{macrocode}
\cs_new_protected:Npn \regex_compile_u_not_cs:
  {
    \exp_args:No \tl_analysis_map_inline:nn { \l_regex_internal_a_tl }
      {
        \tl_build_one:n
          {
            \regex_class:NnnnN \c_true_bool
              {
                \if_int_compare:w "##2 = \c_zero
                  \regex_item_exact_cs:c { \exp_after:wN \cs_to_str:N ##1 }
                \else:
                  \regex_item_exact:nn { \__int_value:w "##2 } { ##3 }
                \fi:
              }
              { 1 } { 0 } \c_false_bool
          }
      }
  }
%    \end{macrocode}
% \end{macro}
%
% \subsubsection{Other}
%
% \begin{macro}[aux]{\regex_compile_/K:}
%   The |\K| control sequence is currently the only \enquote{command},
%   which performs some action, rather than matching something. It is
%   allowed in the same contexts as |\b|. At the compilation stage, we
%   leave it as a single control sequence, defined later.
%    \begin{macrocode}
\cs_new_protected_nopar:cpn { regex_compile_/K: }
  {
    \int_compare:nNnTF \l_regex_mode_int = \c_zero
      { \tl_build_one:n { \regex_command_K: } }
      { \regex_compile_raw_error:N K }
  }
%    \end{macrocode}
% \end{macro}
%
% \subsubsection{Showing regexes}
%
% \begin{macro}[aux]{\regex_show_aux:Nx}
%   Within a \cs{tl_set_build:Nw} \ldots{} \cs{tl_build_end:} group, we
%   redefine all the function that can appear in a compiled regex, then
%   run the regex. The result is then shown.
%    \begin{macrocode}
\cs_new_protected:Npn \regex_show_aux:Nx #1#2
  {
    \tl_set_build:Nw \l_regex_internal_a_tl
      \cs_set_protected_nopar:Npn \regex_branch:n
        {
          \seq_pop_right:NN \l_regex_show_prefix_seq \l_regex_internal_a_tl
          \regex_show_one:n { +-branch }
          \seq_put_right:No \l_regex_show_prefix_seq \l_regex_internal_a_tl
          \use:n
        }
      \cs_set_protected_nopar:Npn \regex_group:nnnN
        { \regex_show_group_aux:nnnnN { } }
      \cs_set_protected_nopar:Npn \regex_group_no_capture:nnnN
        { \regex_show_group_aux:nnnnN { ~(no~capture) } }
      \cs_set_protected_nopar:Npn \regex_group_resetting:nnnN
        { \regex_show_group_aux:nnnnN { ~(resetting) } }
      \cs_set_eq:NN \regex_class:NnnnN \regex_show_class:NnnnN
      \cs_set_protected_nopar:Npn \regex_command_K:
        { \regex_show_one:n { reset~match~start~(\iow_char:N\\K) } }
      \cs_set_protected:Npn \regex_assertion:Nn ##1##2
        { \regex_show_one:n { \bool_if:NF ##1 { negative~ } assertion:~##2 } }
      \cs_set_nopar:Npn \regex_b_test: { word~boundary }
      \cs_set_eq:NN \regex_anchor:N \regex_show_anchor_to_str:N
      \cs_set_protected:Npn \regex_item_caseful_equal:n ##1
        { \regex_show_one:n { char~code~\int_eval:n{##1} } }
      \cs_set_protected:Npn \regex_item_caseful_range:nn ##1##2
        { \regex_show_one:n { range~[\int_eval:n{##1}, \int_eval:n{##2}] } }
      \cs_set_protected:Npn \regex_item_caseless_equal:n ##1
        { \regex_show_one:n { char~code~\int_eval:n{##1}~(caseless) } }
      \cs_set_protected:Npn \regex_item_caseless_range:nn ##1##2
        {
          \regex_show_one:n
            { Range~[\int_eval:n{##1}, \int_eval:n{##2}]~(caseless) }
        }
      \cs_set_protected:Npn \regex_item_catcode:nT
        { \regex_show_item_catcode:NnT \c_true_bool }
      \cs_set_protected:Npn \regex_item_catcode_reverse:nT
        { \regex_show_item_catcode:NnT \c_false_bool }
      \cs_set_protected:Npn \regex_item_reverse:n
        { \regex_show_scope:nn { Reversed~match } }
      \cs_set_protected:Npn \regex_item_exact:nn ##1##2
        { \regex_show_one:n { char~##2,~catcode~##1 } }
      \cs_set_protected:Npn \regex_item_exact_cs:c ##1
        { \regex_show_one:n { control~sequence~\iow_char:N\\##1 } }
      \cs_set_protected:Npn \regex_item_cs:n
        { \regex_show_scope:nn { control~sequence } }
      \cs_set:cpn { regex_prop_.: } { \regex_show_one:n { any~token } }
      \seq_clear:N \l_regex_show_prefix_seq
      \regex_show_push:n { ~ }
      #1
    \tl_build_end:
    \__msg_aux_show:x { > Compiled~regex~#2: \l_regex_internal_a_tl }
  }
%    \end{macrocode}
% \end{macro}
%
% \begin{macro}[aux]{\regex_show_one:n}
%   Every part of the final message go through this function, which adds
%   one line to the output, with the apropriate prefix.
%    \begin{macrocode}
\cs_new_protected:Npn \regex_show_one:n #1
  {
    \int_incr:N \l_regex_show_lines_int
    \tl_build_one:x
      {
        \iow_newline:
        \seq_map_function:NN \l_regex_show_prefix_seq \use:n
        #1
      }
  }
%    \end{macrocode}
% \end{macro}
%
% \begin{macro}[aux]
%   {\regex_show_push:n, \regex_show_pop:, \regex_show_scope:nn}
%   Enter and exit levels of nesting. The \texttt{scope} function prints
%   its first argument as an \enquote{introduction}, then performs its
%   second argument in a deeper level of nesting.
%    \begin{macrocode}
\cs_new_protected:Npn \regex_show_push:n #1
  { \seq_put_right:Nx \l_regex_show_prefix_seq { #1 ~ } }
\cs_new_protected:Npn \regex_show_pop:
  { \seq_pop_right:NN \l_regex_show_prefix_seq \l_regex_internal_a_tl }
\cs_new_protected:Npn \regex_show_scope:nn #1#2
  {
    \regex_show_one:n {#1}
    \regex_show_push:n { ~ }
    #2
    \regex_show_pop:
  }
%    \end{macrocode}
% \end{macro}
%
% \begin{macro}[aux]{\regex_show_group_aux:nnnnN}
%   We display all groups in the same way, simply adding a message,
%   \texttt{(no capture)} or \texttt{(resetting)}, to special groups.
%   The odd \cs{use_ii:nn} avoids printing a spurious \texttt{+-branch}
%   for the first branch.
%    \begin{macrocode}
\cs_new_protected:Npn \regex_show_group_aux:nnnnN #1#2#3#4#5
  {
    \regex_show_one:n { ,-group~begin #1 }
    \regex_show_push:n { | }
    \use_ii:nn #2
    \regex_show_pop:
    \regex_show_one:n
      { `-group~end \regex_msg_repeated:nnN {#3} {#4} #5 }
  }
%    \end{macrocode}
% \end{macro}
%
% \begin{macro}[aux]{\regex_show_class:NnnnN}
%   I'm entirely unhappy about this function: I couldn't find a way to
%   test if a class is a single test. Instead, collect the
%   representation of the tests in the class. If that had more than one
%   line, write \texttt{Match} or \texttt{Don't match} on its own line,
%   with the repeating information if any. Then the various tests on
%   lines of their own, and finally a line. Otherwise, we need to
%   evaluate the representation of the tests again (since the prefix is
%   incorrect). That's clunky, but not too expensive, since it's only
%   one test.
%    \begin{macrocode}
\cs_set:Npn \regex_show_class:NnnnN #1#2#3#4#5
  {
    \tl_set_build:Nw \l_regex_internal_a_tl
      \int_zero:N \l_regex_show_lines_int
      \regex_show_push:n {~}
      #2
      \exp_last_unbraced:Nf
    \int_case:nnn { \l_regex_show_lines_int }
      {
        {0}
          {
            \tl_build_end:
            \regex_show_one:n { \bool_if:NTF #1 { Fail } { Pass } }
          }
        {1}
          {
            \tl_build_end:
            \bool_if:NTF #1
              {
                #2
                \tl_build_one:n { \regex_msg_repeated:nnN {#3} {#4} #5 }
              }
              {
                \regex_show_one:n
                  { Don't~match~\regex_msg_repeated:nnN {#3} {#4} #5 }
                \tl_build_one:o \l_regex_internal_a_tl
              }
          }
      }
      {
        \tl_build_end:
        \regex_show_one:n
          {
            \bool_if:NTF #1 { M } { Don't~m } atch
            \regex_msg_repeated:nnN {#3} {#4} #5
          }
        \tl_build_one:o \l_regex_internal_a_tl
      }
  }
%    \end{macrocode}
% \end{macro}
%
% \begin{macro}[aux, rEXP]{\regex_show_anchor_to_str:N}
%   The argument is an integer telling us where the anchor is. We
%   convert that to the relevant info.
%    \begin{macrocode}
\cs_new:Npn \regex_show_anchor_to_str:N #1
  {
    anchor~at~
    \str_case:nnn { #1 }
      {
        { \l_regex_min_pos_int   } { start~(\iow_char:N\\A) }
        { \l_regex_start_pos_int } { start~of~match~(\iow_char:N\\G) }
        { \l_regex_max_pos_int   } { end~(\iow_char:N\\Z) }
      }
      { <error:~'#1'~not~recognized> }
  }
%    \end{macrocode}
% \end{macro}
%
% \begin{macro}[aux]{\regex_show_item_catcode:NnT}
%   Produce a sequence of categories which the catcode bitmap |#2|
%   contains, and show it, indenting the tests on which this catcode
%   constraint applies.
%    \begin{macrocode}
\cs_new_protected:Npn \regex_show_item_catcode:NnT #1#2
  {
    \seq_set_split:Nnn \l_regex_internal_seq { } { CBEMTPUDSLOA }
    \seq_set_filter:NNn \l_regex_internal_seq \l_regex_internal_seq
      { \int_if_odd_p:n { #2 / \int_use:c { c_regex_catcode_##1_int } } }
    \regex_show_scope:nn
      {
        categories~
        \seq_map_function:NN \l_regex_internal_seq \use:n
        , ~
        \bool_if:NF #1 { negative~ } class
      }
  }
%    \end{macrocode}
% \end{macro}
%
% \subsection{Building}
%
% \subsubsection{Variables used while building}
%
% \begin{variable}{\l_regex_min_state_int, \l_regex_max_state_int}
%   The last state that was allocated is $\cs{l_regex_max_state_int}-1$,
%   so that \cs{l_regex_max_state_int} always points to a free state.
%   The \texttt{min_state} variable is always $0$, but is included to
%   avoid hard-coding this value.
%    \begin{macrocode}
\int_new:N  \l_regex_min_state_int
\int_new:N  \l_regex_max_state_int
%    \end{macrocode}
% \end{variable}
%
% \begin{variable}{\l_regex_left_state_int, \l_regex_right_state_int}
% \begin{variable}{\l_regex_left_state_seq, \l_regex_right_state_seq}
%   Alternatives are implemented by branching from a \texttt{left} state
%   into the various choices, then merging those into a \texttt{right}
%   state. We store information about those states in two sequences.
%   Those states are also used to implement group quantifiers. Most
%   often, the left and right pointers only differ by~$1$.
%    \begin{macrocode}
\int_new:N  \l_regex_left_state_int
\int_new:N  \l_regex_right_state_int
\seq_new:N  \l_regex_left_state_seq
\seq_new:N  \l_regex_right_state_seq
%    \end{macrocode}
% \end{variable}
% \end{variable}
%
% \begin{variable}{\l_regex_capturing_group_int}
%   \cs{l_regex_capturing_group_int} is the \textsc{id} number that will
%   be assigned to a capturing group if one was opened now. This starts
%   at $0$ for the group enclosing the full regular expression, and
%   groups are counted in the order of their left parenthesis, except
%   when encountering \texttt{resetting} groups.
%    \begin{macrocode}
\int_new:N  \l_regex_capturing_group_int
%    \end{macrocode}
% \end{variable}
%
% \subsubsection{Framework}
%
% This phase is about going from a compiled regex to an \textsc{nfa}.
% Each state of the \textsc{nfa} is stored in a \tn{toks}. The
% operations which can appear in the \tn{toks} are
% \begin{itemize}
%   \item \cs{regex_action_start_wildcard:} inserted at the start
%     of the regular expression to make it unanchored.
%   \item \cs{regex_action_success:} marks the exit state of the
%     \textsc{nfa}.
%   \item \cs{regex_action_cost:n} \Arg{shift} is a transition from the
%     current \meta{state} to $\meta{state}+\meta{shift}$, which
%     consumes the current character: the target state is saved and will
%     be considered again when matching at the next position.
%   \item \cs{regex_action_free:n} \Arg{shift}, and
%     \cs{regex_action_free_group:n} \Arg{shift} are free transitions,
%     which immediately perform the actions for the state
%     $\meta{state}+\meta{shift}$ of the \textsc{nfa}. They differ in
%     how they detect and avoid infinite loops. For now, we just need to
%     know that the \texttt{group} variant must be used for transitions
%     back to the start of a group.
%   \item \cs{regex_action_submatch:n} \Arg{key} where the \meta{key} is
%     a group number followed by |<| or |>| for the beginning or end of
%     group. This causes the current position in the query to be stored
%     as the \meta{key} submatch boundary.
% \end{itemize}
%
% We strive to preserve the following properties while building.
% \begin{itemize}
%   \item The current capturing group is
%     $\text{\texttt{capturing_group}}-1$, and if a group is opened now,
%     it will be labelled \texttt{capturing_group}.
%   \item The last allocated state is $\text{\texttt{max_state}}-1$, so
%     \texttt{max_state} is a free state.
%   \item The \texttt{left_state} points to a state to the left of the
%     current group or of the last class.
%   \item The \texttt{right_state} points to a newly created,
%     empty state, with some transitions leading to it.
%   \item The \texttt{left/right} sequences hold a list of the
%     corresponding end-points of nested groups.
% \end{itemize}
%
% \begin{macro}[int]{\regex_build:n, \regex_build:N}
%   The \texttt{n}-type function first compiles its argument. Reset some
%   variables. Allocate two states, and put a wildcard in state $0$
%   (transitions to state $1$ and $0$ state). Then build the regex
%   within a (capturing) group, which will be numbered $0$ (current
%   value of \texttt{capturing_group}). Finally, if the match reaches the
%   last state, it is successful.
%    \begin{macrocode}
\cs_new_protected:Npn \regex_build:n #1
  {
    \regex_compile:n {#1}
    \regex_build:N \l_regex_internal_regex
  }
\cs_new_protected:Npn \regex_build:N #1
  {
%<trace>    \trace_push:nnn { regex } { 1 } { regex_build }
    \int_set:Nn \tex_escapechar:D { 92 }
    \int_zero:N \l_regex_capturing_group_int
    \int_set_eq:NN \l_regex_max_state_int \l_regex_min_state_int
    \regex_build_new_state:
    \regex_build_new_state:
    \regex_toks_put_right:Nn \l_regex_left_state_int
      { \regex_action_start_wildcard: }
    \regex_group:nnnN {#1} { 1 } { 0 } \c_false_bool
    \regex_toks_put_right:Nn \l_regex_right_state_int
      { \regex_action_success: }
%<trace>    \regex_trace_states:n { 2 }
%<trace>    \trace_pop:nnn { regex } { 1 } { regex_build }
  }
%    \end{macrocode}
% \end{macro}
%
% \begin{macro}[int]{\regex_build_for_cs:n}
%   When using a regex to match a cs, we don't insert a wildcard, we
%   anchor at the end, and since we ignore submatches, there is no need
%   to surround the expression with a group. However, for branches to
%   work properly at the outer level, we need to put the appropriate
%   \texttt{left} and \texttt{right} states in their sequence.
%    \begin{macrocode}
\cs_new_protected:Npn \regex_build_for_cs:n #1
  {
%<trace>    \trace_push:nnn { regex } { 1 } { regex_build_for_cs }
    \int_set_eq:NN \l_regex_max_state_int \l_regex_min_state_int
    \regex_build_new_state:
    \regex_build_new_state:
    \regex_push_lr_states:
    #1
    \regex_pop_lr_states:
    \regex_toks_put_right:Nn \l_regex_right_state_int
      {
        \if_int_compare:w \l_regex_current_pos_int = \l_regex_max_pos_int
          \exp_after:wN \regex_action_success:
        \fi:
      }
%<trace>    \regex_trace_states:n { 2 }
%<trace>    \trace_pop:nnn { regex } { 1 } { regex_build_for_cs }
  }
%    \end{macrocode}
% \end{macro}
%
% \subsubsection{Helpers for building an \textsc{nfa}}
%
% \begin{macro}[int]{\regex_push_lr_states:, \regex_pop_lr_states:}
%   When building the regular expression, we keep track of pointers to
%   the left-end and right-end of each group without help from \TeX{}'s
%   grouping.
%    \begin{macrocode}
\cs_new_protected_nopar:Npn \regex_push_lr_states:
  {
    \seq_push:No \l_regex_left_state_seq
      { \int_use:N \l_regex_left_state_int }
    \seq_push:No \l_regex_right_state_seq
      { \int_use:N \l_regex_right_state_int }
  }
\cs_new_protected_nopar:Npn \regex_pop_lr_states:
  {
    \seq_pop:NN \l_regex_left_state_seq  \l_regex_internal_a_tl
    \int_set:Nn \l_regex_left_state_int  \l_regex_internal_a_tl
    \seq_pop:NN \l_regex_right_state_seq \l_regex_internal_a_tl
    \int_set:Nn \l_regex_right_state_int \l_regex_internal_a_tl
  }
%    \end{macrocode}
% \end{macro}
%
% \begin{macro}[int]{\regex_toks_put_left:Nx}
% \begin{macro}[int]{\regex_toks_put_right:Nx, \regex_toks_put_right:Nn}
%   During the building phase we wish to add \texttt{x}-expanded
%   material to \tn{toks}, either to the left or to the right. The
%   expansion is done \enquote{by hand} for optimization (these
%   operations are used quite a lot). The \texttt{Nn} version of
%   \cs{regex_toks_put_right:Nx} is provided because it is more
%   efficient than \texttt{x}-expanding with \cs{exp_not:n}.
%    \begin{macrocode}
\cs_new_protected:Npn \regex_toks_put_left:Nx #1#2
  {
    \cs_set_nopar:Npx \regex_tmp:w { #2 }
    \tex_toks:D #1 \exp_after:wN \exp_after:wN \exp_after:wN
      { \exp_after:wN \regex_tmp:w \tex_the:D \tex_toks:D #1 }
  }
\cs_new_protected:Npn \regex_toks_put_right:Nx #1#2
  {
    \cs_set_nopar:Npx \regex_tmp:w {#2}
    \tex_toks:D #1 \exp_after:wN
      { \tex_the:D \tex_toks:D \exp_after:wN #1 \regex_tmp:w }
  }
\cs_new_protected:Npn \regex_toks_put_right:Nn #1#2
  { \tex_toks:D #1 \exp_after:wN { \tex_the:D \tex_toks:D #1 #2 } }
%    \end{macrocode}
% \end{macro}
% \end{macro}
%
% \begin{macro}[int]
%   {
%     \regex_build_transition_left:NNN,
%     \regex_build_transition_right:nNn
%   }
%   Add a transition from |#2| to |#3| using the function |#1|. The
%   \texttt{left} function is used for higher priority transitions, and
%   the \texttt{right} function for lower priority transitions (which
%   should be performed later). The signatures differ to reflect the
%   differing usage later on. Both functions could be optimized.
%    \begin{macrocode}
\cs_new_protected:Npn \regex_build_transition_left:NNN #1#2#3
  { \regex_toks_put_left:Nx  #2 { #1 { \int_eval:n { #3 - #2 } } } }
\cs_new_protected:Npn \regex_build_transition_right:nNn #1#2#3
  { \regex_toks_put_right:Nx #2 { #1 { \int_eval:n { #3 - #2 } } } }
%    \end{macrocode}
% \end{macro}
%
% \begin{macro}[int]{\regex_build_new_state:}
%   Add a new empty state to the \textsc{nfa}. Then update the
%   \texttt{left}, \texttt{right}, and \texttt{max} states, so that the
%   \texttt{right} state is the new empty state, and the \texttt{left}
%   state points to the previously \enquote{current} state.
%    \begin{macrocode}
\cs_new_protected_nopar:Npn \regex_build_new_state:
  {
%<*trace>
    \trace:nnx { regex } { 2 }
      {
        regex~new~state~
        L=\int_use:N \l_regex_left_state_int ~ -> ~
        R=\int_use:N \l_regex_right_state_int ~ -> ~
        M=\int_use:N \l_regex_max_state_int ~ -> ~
        \int_eval:n { \l_regex_max_state_int + \c_one }
      }
%</trace>
    \tex_toks:D \l_regex_max_state_int { }
    \int_set_eq:NN \l_regex_left_state_int \l_regex_right_state_int
    \int_set_eq:NN \l_regex_right_state_int \l_regex_max_state_int
    \int_incr:N \l_regex_max_state_int
  }
%    \end{macrocode}
% \end{macro}
%
% \begin{macro}[int]{\regex_build_transitions_lazyness:NNNNN}
%   This function creates a new state, and puts two transitions starting
%   from the old current state. The order of the transitions is
%   controlled by |#1|, true for lazy quantifiers, and false for greedy
%   quantifiers.
%    \begin{macrocode}
\cs_new_protected:Npn \regex_build_transitions_lazyness:NNNNN #1#2#3#4#5
  {
    \regex_build_new_state:
    \regex_toks_put_right:Nx \l_regex_left_state_int
      {
        \if_meaning:w \c_true_bool #1
          #2 { \int_eval:n { #3 - \l_regex_left_state_int } }
          #4 { \int_eval:n { #5 - \l_regex_left_state_int } }
        \else:
          #4 { \int_eval:n { #5 - \l_regex_left_state_int } }
          #2 { \int_eval:n { #3 - \l_regex_left_state_int } }
        \fi:
      }
  }
%    \end{macrocode}
% \end{macro}
%
% \subsubsection{Building classes}
%
% \begin{macro}[int]{\regex_class:NnnnN}
% \begin{macro}[int, rEXP]{\regex_tests_action_cost:n}
%   The arguments are: \meta{boolean} \Arg{tests} \Arg{min} \Arg{more}
%   \meta{lazyness}. First store the tests with a trailing
%   \cs{regex_action_cost:n}, in the true branch of
%   \cs{regex_break_point:TF} for positive classes, or the false branch
%   for negative classes. The integer \meta{more} is $0$ for fixed
%   repetitions, $-1$ for unbounded repetitions, and
%   $\meta{max}-\meta{min}$ for a range of repetitions.
%    \begin{macrocode}
\cs_new_protected:Npn \regex_class:NnnnN #1#2#3#4#5
  {
    \cs_set_nopar:Npx \regex_tests_action_cost:n ##1
      {
        \exp_not:n { \exp_not:n {#2} }
        \bool_if:NTF #1
          { \regex_break_point:TF { \regex_action_cost:n {##1} } { } }
          { \regex_break_point:TF { } { \regex_action_cost:n {##1} } }
      }
    \if_case:w - #4 \exp_stop_f:
           \regex_class_repeat:n   {#3}
    \or:   \regex_class_repeat:nN  {#3}      #5
    \else: \regex_class_repeat:nnN {#3} {#4} #5
    \fi:
  }
\cs_new:Npn \regex_tests_action_cost:n { \regex_action_cost:n }
%    \end{macrocode}
% \end{macro}
% \end{macro}
%
% \begin{macro}[aux]{\regex_class_repeat:n}
%   This is used for a fixed number of repetitions. Build one state for
%   each repetition, with a transition controlled by the tests that we
%   have collected. That works just fine for |#1|${}=0$ repetitions:
%   nothing is built.
%    \begin{macrocode}
\cs_new_protected:Npn \regex_class_repeat:n #1
  {
    \prg_replicate:nn {#1}
      {
        \regex_build_new_state:
        \regex_build_transition_right:nNn \regex_tests_action_cost:n
          \l_regex_left_state_int \l_regex_right_state_int
      }
  }
%    \end{macrocode}
% \end{macro}
%
% \begin{macro}[aux]{\regex_class_repeat:nN}
%   This implements unbounded repetitions of a single class (\emph{e.g.}
%   the |*| and |+| quantifiers). If the minimum number |#1| of
%   repetitions is $0$, then build a transition from the current state
%   to itself governed by the tests, and a free transition to a new
%   state (hence skipping the tests). Otherwise, call
%   \cs{regex_class_repeat:n} for the code to match |#1| repetitions,
%   and add free transitions from the last state to the previous one,
%   and to a new one. In both cases, the order of transitions is
%   controlled by the lazyness boolean |#2|.
%    \begin{macrocode}
\cs_new_protected:Npn \regex_class_repeat:nN #1#2
  {
    \if_int_compare:w #1 = \c_zero
      \regex_build_transitions_lazyness:NNNNN #2
        \regex_action_free:n       \l_regex_right_state_int
        \regex_tests_action_cost:n \l_regex_left_state_int
    \else:
      \regex_class_repeat:n {#1}
      \int_set_eq:NN \l_regex_internal_a_int \l_regex_left_state_int
      \regex_build_transitions_lazyness:NNNNN #2
        \regex_action_free:n \l_regex_right_state_int
        \regex_action_free:n \l_regex_internal_a_int
    \fi:
  }
%    \end{macrocode}
% \end{macro}
%
% \begin{macro}[aux]{\regex_class_repeat:nnN}
%   We want to build the code to match from |#1| to $|#1|+|#2|$
%   repetitions. Match |#1| repetitions (can be $0$). Compute the final
%   state of the next construction as \texttt{a}. Build $|#2|>0$ states,
%   each with a transition to the next state governed by the tests, and
%   a transition to the final state \texttt{a}. The computation of
%   \texttt{a} is safe because states are allocated in order, starting
%   from \texttt{max_state}.
%    \begin{macrocode}
\cs_new_protected:Npn \regex_class_repeat:nnN #1#2#3
  {
    \regex_class_repeat:n {#1}
    \int_set:Nn \l_regex_internal_a_int
      { \l_regex_max_state_int + #2 - \c_one }
    \prg_replicate:nn { #2 }
      {
        \regex_build_transitions_lazyness:NNNNN #3
          \regex_action_free:n       \l_regex_internal_a_int
          \regex_tests_action_cost:n \l_regex_right_state_int
      }
  }
%    \end{macrocode}
% \end{macro}
%
% \subsubsection{Building groups}
%
% \begin{macro}[aux]{\regex_group_aux:nnnnN}
%   Arguments: \Arg{label} \Arg{contents} \Arg{min} \Arg{more}
%   \meta{lazyness}. If \meta{min} is $0$, we need to add a state before
%   building the group, so that the thread which skips the group does
%   not also set the start-point of the submatch. After adding one more
%   state, the \texttt{left_state} is the left end of the group, from
%   which all branches will stem, and the \texttt{right_state} is the
%   right end of the group, and all branches end their course in that
%   state. We store those two integers to be queried for each branch, we
%   build the \textsc{nfa} states for the contents |#2| of the group,
%   and we forget about the two integers. Once this is done, perform the
%   repetition: either exactly |#3| times, or |#3| or more times, or
%   between |#3| and $|#3|+|#4|$ times, with lazyness |#5|. The
%   \meta{label} |#1| is used for submatch tracking. Each of the three
%   auxiliaries expects \texttt{left_state} and \texttt{right_state} to
%   be set properly.
%    \begin{macrocode}
\cs_new_protected:Npn \regex_group_aux:nnnnN #1#2#3#4#5
  {
%<trace>        \trace_push:nnn { regex } { 1 } { regex_group }
      \if_int_compare:w #3 = \c_zero
        \regex_build_new_state:
%<assert>\assert_int:n { \l_regex_max_state_int = \l_regex_right_state_int + 1 }
        \regex_build_transition_right:nNn \regex_action_free_group:n
          \l_regex_left_state_int \l_regex_right_state_int
      \fi:
      \regex_build_new_state:
      \regex_push_lr_states:
      #2
      \regex_pop_lr_states:
      \if_case:w - #4 \exp_stop_f:
             \regex_group_repeat:nn   {#1} {#3}
      \or:   \regex_group_repeat:nnN  {#1} {#3}      #5
      \else: \regex_group_repeat:nnnN {#1} {#3} {#4} #5
      \fi:
%<trace>       \trace_pop:nnn { regex } { 1 } { regex_group }
  }
%    \end{macrocode}
% \end{macro}
%
% \begin{macro}[int]{\regex_group:nnnN, \regex_group_no_capture:nnnN}
%   Hand to \cs{regex_group_aux:nnnnnN} the label of that group
%   (expanded), and the group itself, with some extra commands to
%   perform.
%    \begin{macrocode}
\cs_new_protected:Npn \regex_group:nnnN #1
  {
    \exp_args:No \regex_group_aux:nnnnN
      { \int_use:N \l_regex_capturing_group_int }
      {
        \int_incr:N \l_regex_capturing_group_int
        #1
      }
  }
\cs_new_protected_nopar:Npn \regex_group_no_capture:nnnN
  { \regex_group_aux:nnnnN { -1 } }
%    \end{macrocode}
% \end{macro}
%
% \begin{macro}[int]{\regex_group_resetting:nnnN}
% \begin{macro}[aux]{\regex_group_resetting_loop:nnNn}
%   Again, hand the label $-1$ to \cs{regex_group_aux:nnnnN}, but this
%   time we work a little bit harder to keep track of the maximum group
%   label at the end of any branch, and to reset the group number at
%   each branch. This relies on the fact that a compiled regex always is
%   a sequence of items of the form \cs{regex_branch:n} \Arg{branch}.
%    \begin{macrocode}
\cs_new_protected:Npn \regex_group_resetting:nnnN #1
  {
    \regex_group_aux:nnnnN { -1 }
      {
        \exp_args:Noo \regex_group_resetting_loop:nnNn
          { \int_use:N \l_regex_capturing_group_int }
          { \int_use:N \l_regex_capturing_group_int }
          #1
          { ?? \prg_map_break:n } { }
        \prg_break_point:n { }
      }
  }
\cs_new_protected:Npn \regex_group_resetting_loop:nnNn #1#2#3#4
  {
    \use_none:nn #3 { \int_set:Nn \l_regex_capturing_group_int {#1} }
    \int_set:Nn \l_regex_capturing_group_int {#2}
    #3 {#4}
    \exp_args:Nf \regex_group_resetting_loop:nnNn
      { \int_max:nn {#1} { \l_regex_capturing_group_int } }
      {#2}
  }
%    \end{macrocode}
% \end{macro}
% \end{macro}
%
% \begin{macro}[int]{\regex_branch:n}
%   Add a free transition from the left state of the current group to a
%   brand new state, starting point of this branch. Once the branch is
%   built, add a transition from its last state to the right state of
%   the group. The left and right states of the group are extracted from
%   the relevant sequences.
%    \begin{macrocode}
\cs_new_protected:Npn \regex_branch:n #1
  {
%<trace>    \trace_push:nnn { regex } { 1 } { regex_branch }
    \regex_build_new_state:
    \seq_get:NN \l_regex_left_state_seq \l_regex_internal_a_tl
    \int_set:Nn \l_regex_left_state_int \l_regex_internal_a_tl
    \regex_build_transition_right:nNn \regex_action_free:n
      \l_regex_left_state_int \l_regex_right_state_int
    #1
    \seq_get:NN \l_regex_right_state_seq \l_regex_internal_a_tl
    \regex_build_transition_right:nNn \regex_action_free:n
      \l_regex_right_state_int \l_regex_internal_a_tl
%<trace>    \trace_pop:nnn { regex } { 1 } { regex_branch }
  }
%    \end{macrocode}
% \end{macro}
%
% \begin{macro}[int]{\regex_group_repeat:nn}
%   This function is called to repeat a group a fixed number of times
%   |#2|; if this is $0$ we remove the group altogether (but don't reset
%   the \texttt{capturing_group} label). Otherwise, the auxiliary
%   \cs{regex_group_repeat_aux:n} copies |#2| times the \tn{toks} for
%   the group, and leaves \texttt{internal_a} pointing to the left end
%   of the last repetition. We only record the submatch information at
%   the last repetition. Finally, add a state at the end (the transition
%   to it has been taken care of by the replicating auxiliary.
%    \begin{macrocode}
\cs_new_protected:Npn \regex_group_repeat:nn #1#2
  {
    \if_int_compare:w #2 = \c_zero
      \int_set:Nn \l_regex_max_state_int
        { \l_regex_left_state_int - \c_one }
      \regex_build_new_state:
    \else:
      \regex_group_repeat_aux:n {#2}
      \regex_group_submatches:nNN {#1}
        \l_regex_internal_a_int \l_regex_right_state_int
      \regex_build_new_state:
    \fi:
  }
%    \end{macrocode}
% \end{macro}
%
% \begin{macro}[aux]{\regex_group_submatches:nNN}
%   This inserts in states |#2| and |#3| the code for tracking
%   submatches of the group |#1|, unless inhibited by a label of $-1$.
%    \begin{macrocode}
\cs_new_protected:Npn \regex_group_submatches:nNN #1#2#3
  {
    \if_int_compare:w #1 > \c_minus_one
      \regex_toks_put_left:Nx #2 { \regex_action_submatch:n { #1 < } }
      \regex_toks_put_left:Nx #3 { \regex_action_submatch:n { #1 > } }
    \fi:
  }
%    \end{macrocode}
% \end{macro}
%
% \begin{macro}[aux]{\regex_group_repeat_aux:n}
%   Here we repeat \tn{toks} ranging from \texttt{left_state} to
%   \texttt{max_state}, $|#1|>0$ times. First add a transition so that
%   the copies will \enquote{chain} properly. Compute the shift
%   \texttt{c} between the original copy and the last copy we
%   want. Shift the \texttt{right_state} and \texttt{max_state} to their
%   final values. We then want to perform \texttt{c} copy operations. At
%   the end, \texttt{b} is equal to the \texttt{max_state}, and
%   \texttt{a} points to the left of the last copy of the group.
%    \begin{macrocode}
\cs_new_protected:Npn \regex_group_repeat_aux:n #1
  {
    \regex_build_transition_right:nNn \regex_action_free:n
      \l_regex_right_state_int \l_regex_max_state_int
    \int_set_eq:NN \l_regex_internal_a_int \l_regex_left_state_int
    \int_set_eq:NN \l_regex_internal_b_int \l_regex_max_state_int
    \if_int_compare:w \__int_eval:w #1 > \c_one
      \int_set:Nn \l_regex_internal_c_int
        {
          ( #1 - \c_one )
          * ( \l_regex_internal_b_int - \l_regex_internal_a_int )
        }
      \tex_advance:D \l_regex_right_state_int \l_regex_internal_c_int
      \tex_advance:D \l_regex_max_state_int   \l_regex_internal_c_int
      \prg_replicate:nn \l_regex_internal_c_int
        {
          \tex_toks:D \l_regex_internal_b_int
            = \tex_toks:D \l_regex_internal_a_int
          \tex_advance:D \l_regex_internal_a_int \c_one
          \tex_advance:D \l_regex_internal_b_int \c_one
        }
    \fi:
  }
%    \end{macrocode}
% \end{macro}
%
% \begin{macro}[aux]{\regex_group_repeat:nnN}
%   This function is called to repeat a group at least $n$ times; the
%   case $n=0$ is very different from $n>0$. Assume first that $n=0$.
%   Insert submatch tracking information at the start and end of the
%   group, add a free transition from the right end to the
%   \enquote{true} left state \texttt{a} (remember: in this case we had
%   added an extra state before the left state). This forms the loop,
%   which we break away from by adding a free transition from \texttt{a}
%   to a new state.
%
%   Now consider the case $n>0$. Repeat the group $n$ times, chaining
%   various copies with a free transition. Add submatch tracking only to
%   the last copy, then add a free transition from the right end back to
%   the left end of the last copy, either before or after the transition
%   to move on towards the rest of the \textsc{nfa}. This transition can
%   end up before submatch tracking, but that is irrelevant since it
%   only does so when going again through the group, recording new
%   matches. Finally, add a state; we already have a transition pointing
%   to it from \cs{regex_group_repeat_aux:n}.
%    \begin{macrocode}
\cs_new_protected:Npn \regex_group_repeat:nnN #1#2#3
  {
    \if_int_compare:w #2 = \c_zero
      \regex_group_submatches:nNN {#1}
        \l_regex_left_state_int \l_regex_right_state_int
      \int_set:Nn \l_regex_internal_a_int
        { \l_regex_left_state_int - \c_one }
      \regex_build_transition_right:nNn \regex_action_free:n
        \l_regex_right_state_int \l_regex_internal_a_int
      \regex_build_new_state:
      \if_meaning:w \c_true_bool #3
        \regex_build_transition_left:NNN \regex_action_free:n
          \l_regex_internal_a_int \l_regex_right_state_int
      \else:
        \regex_build_transition_right:nNn \regex_action_free:n
          \l_regex_internal_a_int \l_regex_right_state_int
      \fi:
    \else:
      \regex_group_repeat_aux:n {#2}
      \regex_group_submatches:nNN {#1}
        \l_regex_internal_a_int \l_regex_right_state_int
      \if_meaning:w \c_true_bool #3
        \regex_build_transition_right:nNn \regex_action_free_group:n
          \l_regex_right_state_int \l_regex_internal_a_int
      \else:
        \regex_build_transition_left:NNN \regex_action_free_group:n
          \l_regex_right_state_int \l_regex_internal_a_int
      \fi:
      \regex_build_new_state:
    \fi:
  }
%    \end{macrocode}
% \end{macro}
%
% \begin{macro}[aux]{\regex_group_repeat:nnnN}
%   We wish to repeat the group between |#2| and $|#2|+|#3|$ times, with
%   a lazyness controlled by |#4|. We insert submatch tracking up front:
%   in principle, we could avoid recording submatches for the first |#2|
%   copies of the group, but that forces us to treat specially the case
%   $|#2|=0$. Repeat that group with submatch tracking $|#2|+|#3|$ times
%   (the maximum number of repetitions). Then our goal is to add |#3|
%   transitions from the end of the |#2|-th group, and each subsequent
%   groups, to the end. For a lazy quantifier, we add those transitions
%   to the left states, before submatch tracking. For the greedy case,
%   we add the transitions to the right states, after submatch tracking
%   and the transitions which go on with more repetitions. In the greedy
%   case with $|#2|=0$, the transition which skips over all copies of
%   the group must be added separately, because its starting state does
%   not follow the normal pattern: we had to add it \enquote{by hand}
%   earlier.
%    \begin{macrocode}
\cs_new_protected:Npn \regex_group_repeat:nnnN #1#2#3#4
  {
    \regex_group_submatches:nNN {#1}
      \l_regex_left_state_int \l_regex_right_state_int
    \regex_group_repeat_aux:n { #2 + #3 }
    \if_meaning:w \c_true_bool #4
      \int_set_eq:NN \l_regex_left_state_int \l_regex_max_state_int
      \prg_replicate:nn { #3 }
        {
          \int_sub:Nn \l_regex_left_state_int
            { \l_regex_internal_b_int - \l_regex_internal_a_int }
          \regex_build_transition_left:NNN \regex_action_free:n
            \l_regex_left_state_int \l_regex_max_state_int
        }
    \else:
      \prg_replicate:nn { #3 - \c_one }
        {
          \int_sub:Nn \l_regex_right_state_int
            { \l_regex_internal_b_int - \l_regex_internal_a_int }
          \regex_build_transition_right:nNn \regex_action_free:n
            \l_regex_right_state_int \l_regex_max_state_int
        }
      \if_int_compare:w #2 = \c_zero
        \int_set:Nn \l_regex_right_state_int
          { \l_regex_left_state_int - \c_one }
      \else:
        \int_sub:Nn \l_regex_right_state_int
          { \l_regex_internal_b_int - \l_regex_internal_a_int }
      \fi:
      \regex_build_transition_right:nNn \regex_action_free:n
        \l_regex_right_state_int \l_regex_max_state_int
    \fi:
    \regex_build_new_state:
  }
%    \end{macrocode}
% \end{macro}
%
% \subsubsection{Others}
%
% \begin{macro}[int]{\regex_assertion:Nn, \regex_b_test:, \regex_anchor:N}
%   Usage: \cs{regex_assertion:Nn} \meta{boolean} \Arg{test}, where the
%   \meta{test} is either of the two other functions. Add a free
%   transition to a new state, conditionally to the assertion test. The
%   \cs{regex_b_test:} test is used by the |\b| and |\B| escape: check
%   if the last character was a word character or not, and do the same
%   to the current character. The boundary-markers of the string are
%   non-word characters for this purpose.  Anchors at the start or end
%   of match use \cs{regex_anchor:N}, with a position controlled by the
%   integer |#1|.
%    \begin{macrocode}
\cs_new_protected:Npn \regex_assertion:Nn #1#2
  {
    \regex_build_new_state:
    \regex_toks_put_right:Nx \l_regex_left_state_int
      {
        \exp_not:n {#2}
        \regex_break_point:TF
          \bool_if:NF #1 { { } }
          {
            \regex_action_free:n
              {
                \int_eval:n
                  { \l_regex_right_state_int - \l_regex_left_state_int }
              }
          }
          \bool_if:NT #1 { { } }
      }
  }
\cs_new_protected:Npn \regex_anchor:N #1
  {
    \if_int_compare:w #1 = \l_regex_current_pos_int
      \exp_after:wN \regex_break_true:w
    \fi:
  }
\cs_new_protected_nopar:Npn \regex_b_test:
  {
    \group_begin:
      \int_set_eq:NN \l_regex_current_char_int \l_regex_last_char_int
      \regex_prop_w:
      \regex_break_point:TF
        { \group_end: \regex_item_reverse:n \regex_prop_w: }
        { \group_end: \regex_prop_w: }
  }
%    \end{macrocode}
% \end{macro}
%
% \begin{macro}[int]{\regex_command_K:}
%   Change the starting point of the $0$-th submatch (full match), and
%   transition to a new state, pretending that this is a fresh thread.
%    \begin{macrocode}
\cs_new_protected_nopar:Npn \regex_command_K:
  {
    \regex_build_new_state:
    \regex_toks_put_right:Nx \l_regex_left_state_int
      {
        \regex_action_submatch:n { 0< }
        \bool_set_true:N \l_regex_fresh_thread_bool
        \regex_action_free:n
          { \int_eval:n { \l_regex_right_state_int - \l_regex_left_state_int } }
        \bool_set_false:N \l_regex_fresh_thread_bool
      }
  }
%    \end{macrocode}
% \end{macro}
%
% \subsection{Matching}
%
% We search for matches by running all the execution threads through the
% \textsc{nfa} in parallel, reading one token of the query at each step.
% The \textsc{nfa} contains \enquote{free} transitions to other states,
% and transitions which \enquote{consume} the current token.  For free
% transitions, the instruction at the new state of the \textsc{nfa} is
% performed immediately.  When a transition consumes a character, the
% new state is appended to a list of \enquote{active states}, stored in
% \tn{skip} registers: this thread will be active again when the next
% token is read from the query.  At every step (for each token in the
% query), we unpack that list of active states and the corresponding
% submatch props, and empty those.
%
% If two paths through the \textsc{nfa} \enquote{collide} in the sense
% that they reach the same state after reading a given token, then they
% only differ in how they previously matched, and the future execution
% will be identical for both. (Note that this would be wrong in the
% presence of back-references.) Hence, we only need to keep one of the
% two threads: the thread with the highest priority. Our \textsc{nfa} is
% built in such a way that higher priority actions always come before
% lower priority actions, which makes things work.
%
% The explanation in the previous paragraph may make us think that we
% simply need to keep track of which states were visited at a given
% step: after all, the loop generated when matching |(a?)*| against |a|
% is broken, isn't it? No. The group first matches |a|, as it should,
% then repeats; it attempts to match |a| again but fails; it skips |a|,
% and finds out that this state has already been seen at this position
% in the query: the match stops. The capturing group is (wrongly) |a|.
% What went wrong is that a thread collided with itself, and the later
% version, which has gone through the group one more times with an empty
% match, should have a higher priority than not going through the group.
%
% We solve this by distinguishing \enquote{normal} free transitions
% \cs{regex_action_free:n} from transitions
% \cs{regex_action_free_group:n} which go back to the start of the
% group. The former will keep threads unless they have been visited by a
% \enquote{completed} thread, while the latter kind of transition also
% prevents going back to a state visited by the current thread.
%
% \subsubsection{Variables used when matching}
%
% \begin{variable}
%   {
%     \l_regex_min_pos_int,
%     \l_regex_max_pos_int,
%     \l_regex_current_pos_int,
%     \l_regex_start_pos_int,
%     \l_regex_success_pos_int,
%   }
%   The tokens in the query are indexed from \texttt{min_pos} for the
%   first to $\texttt{max_pos}-1$ for the last, and their information is
%   stored in \tn{muskip} and \tn{toks} registers with those numbers. We
%   don't start from $0$ because the \tn{toks} registers with low
%   numbers are used to hold the states of the \textsc{nfa}. We match
%   without backtracking, keeping all threads in lockstep at the
%   \texttt{current_pos} in the query. The starting point of the current
%   match attempt is \texttt{start_pos}, and \texttt{success_pos},
%   updated whenever a thread succeeds, is used as the next starting
%   position.
%    \begin{macrocode}
\int_new:N \l_regex_min_pos_int
\int_new:N \l_regex_max_pos_int
\int_new:N \l_regex_current_pos_int
\int_new:N \l_regex_start_pos_int
\int_new:N \l_regex_success_pos_int
%    \end{macrocode}
% \end{variable}
%
% \begin{variable}
%   {
%     \l_regex_current_char_int,
%     \l_regex_current_catcode_int,
%     \l_regex_last_char_int,
%     \l_regex_case_changed_char_int
%   }
%   The character and category codes of the token at the current
%   position; the character code of the token at the previous position;
%   and the character code of the result of changing the case of the
%   current token (|A-Z|$\leftrightarrow$|a-z|). This last integer is
%   only computed when necessary, and is otherwise \cs{c_max_int}.  The
%   \texttt{current_char} variable is also used in various other phases
%   to hold a character code.
%    \begin{macrocode}
\int_new:N \l_regex_current_char_int
\int_new:N \l_regex_current_catcode_int
\int_new:N \l_regex_last_char_int
\int_new:N \l_regex_case_changed_char_int
%    \end{macrocode}
% \end{variable}
%
% \begin{variable}{\l_regex_current_state_int}
%   For every character in the token list, each of the active states is
%   considered in turn.  The variable \cs{l_regex_current_state_int}
%   holds the state of the \textsc{nfa} which is currently considered:
%   transitions are then given as shifts relative to the current state.
%    \begin{macrocode}
\int_new:N \l_regex_current_state_int
%    \end{macrocode}
% \end{variable}
%
% \begin{variable}
%   {\l_regex_current_submatches_prop, \l_regex_success_submatches_prop}
%   The submatches for the thread which is currently active are stored
%   in the \texttt{current_submatches} property list variable. This
%   property list is stored by \cs{regex_action_cost:n} into the
%   \tn{toks} register for the target state of the transition, to be
%   retrieved when matching at the next position. When a thread
%   succeeds, this property list is copied to
%   \cs{l_regex_success_submatches_prop}: only the last sucessful thread
%   will remain there.
%    \begin{macrocode}
\prop_new:N \l_regex_current_submatches_prop
\prop_new:N \l_regex_success_submatches_prop
%    \end{macrocode}
% \end{variable}
%
% \begin{variable}{\l_regex_step_int}
%   This integer, always even, is increased every time a character in
%   the query is read, and not reset when doing multiple matches. For
%   each \meta{state} in the \textsc{nfa} we store in
%   \tn{dimen}\meta{state} the last step in which this state was
%   encountered. This lets us break infinite loops by not visiting the
%   same state twice in the same step. In fact, \tn{dimen}\meta{state}
%   is equal \texttt{step} when we have started performing the
%   operations of \tn{toks}\meta{state}, but not finished yet. However,
%   once we finish, we set \tn{dimen}\meta{state} to
%   $\text{\texttt{step}}+1$. This is needed to track submatches
%   properly (see building phase). The \texttt{step} is also used to
%   attach each set of submatch information to a given iteration (and
%   automatically discard it when it corresponds to a past step).
%    \begin{macrocode}
\int_new:N \l_regex_step_int
%    \end{macrocode}
% \end{variable}
%
% \begin{variable}{\l_regex_min_active_int, \l_regex_max_active_int}
%   All the currently active states are kept in order of precedence in
%   the \tn{skip} registers, and the corresponding submatches in the
%   \tn{toks}. For our purposes, those serve as an array, indexed from
%   \texttt{min_active} (inclusive) to \texttt{max_active} (excluded).
%   At the start of every step, the whole array is unpacked, so that the
%   space can immediately be reused, and \texttt{max_active} is reset to
%   \texttt{min_active}, effectively clearing the array.
%    \begin{macrocode}
\int_new:N \l_regex_min_active_int
\int_new:N \l_regex_max_active_int
%    \end{macrocode}
% \end{variable}
%
% \begin{variable}{\l_regex_every_match_tl}
%   Every time a match is found, this token list is used.  For single
%   matching, the token list is empty. For multiple matching, the token
%   list is set to repeat the matching, after performing some operation
%   which depends on the user function. See \cs{regex_single_match:} and
%   \cs{regex_multi_match:n}.
%    \begin{macrocode}
\tl_new:N \l_regex_every_match_tl
%    \end{macrocode}
% \end{variable}
%
% \begin{variable}{\l_regex_fresh_thread_bool, \l_regex_empty_success_bool}
% \begin{macro}[aux]{\regex_if_two_empty_matches:F}
%   When doing multiple matches, we need to avoid infinite loops where
%   each iteration matches the same empty token list. When an empty
%   token list is matched, the next successful match of the same empty
%   token list is suppressed. We detect empty matches by setting
%   \cs{l_regex_fresh_thread_bool} to \texttt{true} for threads which
%   directly come from the start of the regex or from the |\K| command,
%   and testing that boolean whenever a thread succeeds. The function
%   \cs{regex_if_two_empty_matches:F} is redefined at every match
%   attempt, depending on whether the previous match was empty or not:
%   if it was, then the function must cancel a purported success if it
%   is empty and at the same spot as the previous match; otherwise, we
%   definitely don't have two identical empty matches, so the function
%   is \cs{use:n}.
%    \begin{macrocode}
\bool_new:N \l_regex_fresh_thread_bool
\bool_new:N \l_regex_empty_success_bool
\cs_new_eq:NN \regex_if_two_empty_matches:F \use:n
%    \end{macrocode}
% \end{macro}
% \end{variable}
%
% \begin{variable}
%   {
%     \g_regex_success_bool,
%     \l_regex_saved_success_bool,
%     \l_regex_match_success_bool
%   }
%   The boolean \cs{l_regex_match_success_bool} is true if the current
%   match attempt was successful, and \cs{g_regex_success_bool} is true
%   if there was at least one successful match. This is the only global
%   variable in this whole module, but we would need it to be local when
%   matching a control sequence with |\c{...}|. This is done by saving
%   the global variable into \cs{l_regex_saved_success_bool}, which is
%   local, hence not affected by the changes due to inner regex
%   functions.
%    \begin{macrocode}
\bool_new:N \g_regex_success_bool
\bool_new:N \l_regex_saved_success_bool
\bool_new:N \l_regex_match_success_bool
%    \end{macrocode}
% \end{variable}
%
% \subsubsection{Matching: framework}
%
% \begin{macro}[int]{\regex_match:n}
%   First store the query into \tn{toks} and \tn{muskip} registers (see
%   \cs{regex_query_set:nnn}). Then initialize the variables that should
%   be set once for each user function (even for multiple
%   matches). Namely, the overall matching is not yet sucessful; none of
%   the states should be marked as visited (\tn{dimen} registers), and
%   we start at step $0$; we pretend that there was a previous match
%   ending at the start of the query, which was not empty (to avoid
%   smothering an empty match at the start). Once all this is set up, we
%   are ready for the ride. Find the first match.
%    \begin{macrocode}
\cs_new_protected:Npn \regex_match:n #1
  {
%<trace>    \trace_push:nnx { regex } { 1 } { regex_match }
%<trace>    \trace:nnx { regex } { 1 } { analyzing~query~token~list }
    \int_zero:N \l_regex_balance_int
    \int_set:Nn \l_regex_current_pos_int { \c_two * \l_regex_max_state_int }
    \regex_query_set:nnn { } { -1 } { -2 }
    \int_set_eq:NN \l_regex_min_pos_int \l_regex_current_pos_int
    \tl_analysis_map_inline:nn {#1}
      { \regex_query_set:nnn {##1} {"##2} {##3} }
    \int_set_eq:NN \l_regex_max_pos_int \l_regex_current_pos_int
    \regex_query_set:nnn { } { -1 } { -2 }
%<trace>    \trace:nnx { regex } { 1 } { initializing }
    \bool_gset_false:N \g_regex_success_bool
    \int_step_inline:nnnn
      \l_regex_min_state_int \c_one { \l_regex_max_state_int - \c_one }
      { \tex_dimen:D ##1 \c_one sp \scan_stop: }
    \int_set_eq:NN \l_regex_min_active_int \l_regex_max_state_int
    \int_set_eq:NN \l_regex_step_int \c_zero
    \int_set_eq:NN \l_regex_success_pos_int \l_regex_min_pos_int
    \int_set:Nn \l_regex_submatch_int
      { \c_two * \l_regex_max_state_int }
    \bool_set_false:N \l_regex_empty_success_bool
    \regex_match_once:
%<trace>    \trace_pop:nnx { regex } { 1 } { regex_match }
  }
%    \end{macrocode}
% \end{macro}
%
% \begin{macro}[int]{\regex_match_once:}
%   This function finds one match, then does some action defined by the
%   \texttt{every_match} token list, which may recursively call
%   \cs{regex_match_once:}. First initialize some variables: set the
%   conditional which detects identical empty matches; this match
%   attempt starts at the previous \texttt{success_pos}, is not yet
%   sucessful, and has no submatches yet; clear the array of active
%   threads, and put the starting state $0$ in it. We are then almost
%   ready to read our first token in the query, but we actually start
%   one position earlier than the start, and \texttt{get} that token, so
%   that the \texttt{last_char} will be set properly for word
%   boundaries. Then call \cs{regex_match_loop:}, which runs through the
%   query until the end or until a successful match breaks early.
%    \begin{macrocode}
\cs_new_protected_nopar:Npn \regex_match_once:
  {
    \if_meaning:w \c_true_bool \l_regex_empty_success_bool
      \cs_set_nopar:Npn \regex_if_two_empty_matches:F
        { \int_compare:nNnF \l_regex_start_pos_int = \l_regex_current_pos_int }
    \else:
      \cs_set_eq:NN \regex_if_two_empty_matches:F \use:n
    \fi:
    \int_set_eq:NN \l_regex_start_pos_int \l_regex_success_pos_int
    \bool_set_false:N \l_regex_match_success_bool
    \prop_clear:N \l_regex_current_submatches_prop
    \int_set_eq:NN \l_regex_max_active_int \l_regex_min_active_int
    \regex_store_state:n { \l_regex_min_state_int }
    \int_set:Nn \l_regex_current_pos_int
      { \l_regex_start_pos_int - \c_one }
    \regex_query_get:
    \regex_match_loop:
    \l_regex_every_match_tl
  }
%    \end{macrocode}
% \end{macro}
%
% \begin{macro}[int]{\regex_single_match:, \regex_multi_match:n}
%   For a single match, the overall success is determined by whether the
%   only match attempt is a success. When doing multiple matches, the
%   overall matching is successful as soon as any match
%   succeeds. Perform the action |#1|, then find the next match.
%    \begin{macrocode}
\cs_new_protected_nopar:Npn \regex_single_match:
  {
    \tl_set:Nn \l_regex_every_match_tl
      { \bool_gset_eq:NN \g_regex_success_bool \l_regex_match_success_bool }
  }
\cs_new_protected:Npn \regex_multi_match:n #1
  {
    \tl_set:Nn \l_regex_every_match_tl
      {
        \if_meaning:w \c_true_bool \l_regex_match_success_bool
          \bool_gset_true:N \g_regex_success_bool
          #1
          \exp_after:wN \regex_match_once:
        \fi:
      }
  }
%    \end{macrocode}
% \end{macro}
%
% \begin{macro}[aux]{\regex_match_loop:}
% \begin{macro}[aux, rEXP]{\regex_match_one_active:w}
%   At each new position, set some variables and get the new character
%   and category from the query. Then unpack the array of active
%   threads, and clear it by resetting its length
%   (\texttt{max_active}). This results in a sequence of
%   \cs{regex_use_state_and_submatches:nn} \Arg{state} \Arg{prop}, and
%   we consider those states one by one in order. As soon as a thread
%   succeeds, exit the step, and, if there are threads to consider at the
%   next position, and we have not reached the end of the string,
%   repeat the loop. Otherwise, the last thread that succeeded is what
%   \cs{regex_match_once:} matches. We explain the \texttt{fresh_thread}
%   business when describing \cs{regex_action_wildcard:}.
%    \begin{macrocode}
\cs_new_protected_nopar:Npn \regex_match_loop:
  {
    \tex_advance:D \l_regex_step_int \c_two
    \int_incr:N \l_regex_current_pos_int
    \int_set_eq:NN \l_regex_last_char_int \l_regex_current_char_int
    \int_set_eq:NN \l_regex_case_changed_char_int \c_max_int
    \regex_query_get:
    \use:x
      {
        \int_set_eq:NN \l_regex_max_active_int \l_regex_min_active_int
        \exp_after:wN \regex_match_one_active:w
          \int_use:N \l_regex_min_active_int ;
      }
    \prg_break_point:n { \bool_set_false:N \l_regex_fresh_thread_bool }
    \if_int_compare:w \l_regex_max_active_int > \l_regex_min_active_int
      \if_int_compare:w \l_regex_current_pos_int < \l_regex_max_pos_int
        \exp_after:wN \exp_after:wN \exp_after:wN \regex_match_loop:
      \fi:
    \fi:
  }
\cs_new:Npn \regex_match_one_active:w #1;
  {
    \if_int_compare:w #1 < \l_regex_max_active_int
      \regex_use_state_and_submatches:nn
        { \__int_value:w \tex_skip:D #1 }
        { \tex_the:D \tex_toks:D #1 }
      \exp_after:wN \regex_match_one_active:w
        \int_use:N \__int_eval:w #1 + \c_one \exp_after:wN ;
    \fi:
  }
%    \end{macrocode}
% \end{macro}
% \end{macro}
%
% \begin{macro}[aux]{\regex_query_set:nnn}
%   The arguments are: tokens that \texttt{o} and \texttt{x} expand to
%   one token of the query, the catcode, and the character code. Store
%   those, and the current brace balance (used later to check for
%   overall brace balance) in a \tn{muskip} register and a \tn{toks},
%   then update the \texttt{balance}.
%    \begin{macrocode}
\cs_new_protected:Npn \regex_query_set:nnn #1#2#3
  {
    \tex_muskip:D \l_regex_current_pos_int
      = \etex_gluetomu:D
        #3 sp
        plus #2 sp
        minus \l_regex_balance_int sp
      \scan_stop:
    \tex_toks:D \l_regex_current_pos_int {#1}
    \int_incr:N \l_regex_current_pos_int
    \if_case:w #2 \exp_stop_f:
    \or: \int_incr:N \l_regex_balance_int
    \or: \int_decr:N \l_regex_balance_int
    \fi:
  }
%    \end{macrocode}
% \end{macro}
%
% \begin{macro}[aux]{\regex_query_get:}
%   Extract the current character and category codes from the
%   \tn{muskip} register of the current position: those are the main and
%   the stretch components, and we need a conversion to avoid \TeX{}'s
%   \enquote{incompatible glue units} error.
%    \begin{macrocode}
\cs_new_protected_nopar:Npn \regex_query_get:
  {
    \l_regex_current_char_int
      = \etex_mutoglue:D \tex_muskip:D \l_regex_current_pos_int
    \l_regex_current_catcode_int = \etex_gluestretch:D
      \etex_mutoglue:D \tex_muskip:D \l_regex_current_pos_int
  }
%    \end{macrocode}
% \end{macro}
%
% \subsubsection{Using states of the \textsc{nfa}}
%
% \begin{macro}[int]{\regex_use_state:}
%   Use the current \textsc{nfa} instruction. The state is initially
%   marked as belonging to the current \texttt{step}: this allows normal
%   free transition to repeat, but group-repeating transitions
%   won't. Once we are done exploring all the branches it spawned, the
%   state is marked as $\texttt{step}+1$: any thread hitting it at that
%   point will be terminated.
%    \begin{macrocode}
\cs_new_protected_nopar:Npn \regex_use_state:
  {
%<*trace>
    \trace:nnx { regex } { 2 } { state~\int_use:N \l_regex_current_state_int }
%</trace>
    \tex_dimen:D \l_regex_current_state_int
      = \l_regex_step_int sp \scan_stop:
    \tex_the:D \tex_toks:D \l_regex_current_state_int
    \tex_dimen:D \l_regex_current_state_int
      = \__int_eval:w \l_regex_step_int + \c_one \__int_eval_end: sp \scan_stop:
  }
%    \end{macrocode}
% \end{macro}
%
% \begin{macro}[int]{\regex_use_state_and_submatches:nn}
%   This function is called as one item in the array of active threads
%   after that array has been unpacked for a new step. Update the
%   \texttt{current_state} and \texttt{current_submatches} and use the
%   state if it has not yet been encountered at this step.
%    \begin{macrocode}
\cs_new_protected:Npn \regex_use_state_and_submatches:nn #1 #2
  {
    \int_set:Nn \l_regex_current_state_int {#1}
    \if_int_compare:w \tex_dimen:D \l_regex_current_state_int
                      < \l_regex_step_int
      \tl_set:Nn \l_regex_current_submatches_prop {#2}
      \exp_after:wN \regex_use_state:
    \fi:
    \scan_stop:
  }
%    \end{macrocode}
% \end{macro}
%
% \subsubsection{Actions when matching}
%
% \begin{macro}[int]{\regex_action_start_wildcard:}
%   For an unanchored match, state $0$ has a free transition to the next
%   and a costly one to itself, to repeat at the next position. To catch
%   repeated identical empty matches, we need to know if a successful
%   thread corresponds to an empty match. The instruction resetting
%   \cs{l_regex_fresh_thread_bool} may be skipped by a successful
%   thread, hence we had to add it to \cs{regex_match_loop:} too.
%    \begin{macrocode}
\cs_new_protected_nopar:Npn \regex_action_start_wildcard:
  {
    \bool_set_true:N \l_regex_fresh_thread_bool
    \regex_action_free:n {1}
    \bool_set_false:N \l_regex_fresh_thread_bool
    \regex_action_cost:n {0}
  }
%    \end{macrocode}
% \end{macro}
%
% \begin{macro}[int]{\regex_action_free:n, \regex_action_free_group:n}
% \begin{macro}[aux]{\regex_action_free_aux:nn}
%   These functions copy a thread after checking that the \textsc{nfa}
%   state has not already been used at this position. If not, store
%   submatches in the new state, and insert the instructions for that
%   state in the input stream.  Then restore the old value of
%   \cs{l_regex_current_state_int} and of the current submatches.  The
%   two types of free transitions differ by how they test that the state
%   has not been encountered yet: the \texttt{group} version is
%   stricter, and will not use a state if it was used earlier in the
%   current thread, hence forcefully breaking the loop, while the
%   \enquote{normal} version will revisit a state when within the thread
%   itself.
%    \begin{macrocode}
\cs_new_protected_nopar:Npn \regex_action_free:n
  { \regex_action_free_aux:nn { > \l_regex_step_int \else: } }
\cs_new_protected_nopar:Npn \regex_action_free_group:n
  { \regex_action_free_aux:nn { < \l_regex_step_int } }
\cs_new_protected:Npn \regex_action_free_aux:nn #1#2
  {
    \use:x
      {
        \int_add:Nn \l_regex_current_state_int {#2}
        \exp_not:n
          {
            \if_int_compare:w \tex_dimen:D \l_regex_current_state_int #1
              \exp_after:wN \regex_use_state:
            \fi:
          }
        \int_set:Nn \l_regex_current_state_int
          { \int_use:N \l_regex_current_state_int }
        \tl_set:Nn \exp_not:N \l_regex_current_submatches_prop
          { \exp_not:o \l_regex_current_submatches_prop }
      }
  }
%    \end{macrocode}
% \end{macro}
% \end{macro}
%
% \begin{macro}[int]{\regex_action_cost:n}
%   A transition which consumes the current character and shifts the
%   state by |#1|.  The resulting state is stored in the \tn{skip} array
%   for use at the next position, and we also store the current
%   submatches.
%    \begin{macrocode}
\cs_new_protected:Npn \regex_action_cost:n #1
  {
    \exp_args:No \regex_store_state:n
      { \int_use:N \__int_eval:w \l_regex_current_state_int + #1 }
  }
%    \end{macrocode}
% \end{macro}
%
% \begin{macro}[int]{\regex_store_state:n}
% \begin{macro}[aux]{\regex_store_submatches:}
%   Put the given state in the array of \tn{skip} registers (converted
%   to a dimension in scaled points), and increment the length of the
%   array. Then store the current submatch in the This is done by
%   increasing the pointer \cs{l_regex_max_active_int}, and converting
%   the integer to a dimension (suitable for a \tn{skip} assignment) in
%   scaled points.
%    \begin{macrocode}
\cs_new_protected:Npn \regex_store_state:n #1
  {
    \regex_store_submatches:
    \tex_skip:D \l_regex_max_active_int = #1 sp \scan_stop:
    \int_incr:N \l_regex_max_active_int
  }
\cs_new_protected_nopar:Npn \regex_store_submatches:
  {
    \tex_toks:D \l_regex_max_active_int \exp_after:wN
      { \l_regex_current_submatches_prop }
  }
%    \end{macrocode}
% \end{macro}
% \end{macro}
%
% \begin{macro}[int]{\regex_disable_submatches:}
%   Some user functions don't require tracking submatches.
%   We get a performance improvement by simply defining the
%   relevant functions to remove their argument and do nothing
%   with it.
%    \begin{macrocode}
\cs_new_protected_nopar:Npn \regex_disable_submatches:
  {
    \cs_set_protected_nopar:Npn \regex_store_submatches: { }
    \cs_set_protected:Npn \regex_action_submatch:n ##1 { }
  }
%    \end{macrocode}
% \end{macro}
%
% \begin{macro}[int]{\regex_action_submatch:n}
%   Update the current submatches with the information from the current
%   position. Maybe a bottleneck.
%    \begin{macrocode}
\cs_new_protected:Npn \regex_action_submatch:n #1
  {
    \prop_put:Nno \l_regex_current_submatches_prop {#1}
      { \int_use:N \l_regex_current_pos_int }
  }
%    \end{macrocode}
% \end{macro}
%
% \begin{macro}[int]{\regex_action_success:}
%   There is a successful match when an execution path reaches the last
%   state in the \textsc{nfa}, unless this marks a second identical
%   empty match. Then mark that there was a successful match; it is
%   empty if it is \enquote{fresh}; and we store the current position
%   and submatches. The current step is then interrupted with
%   \cs{prg_map_break:}, and only paths with higher precedence are
%   pursued further. The values stored here may be overwritten by a
%   later success of a path with higher precedence.
%    \begin{macrocode}
\cs_new_protected_nopar:Npn \regex_action_success:
  {
    \regex_if_two_empty_matches:F
      {
        \bool_set_true:N \l_regex_match_success_bool
        \bool_set_eq:NN \l_regex_empty_success_bool
          \l_regex_fresh_thread_bool
        \int_set_eq:NN \l_regex_success_pos_int \l_regex_current_pos_int
        \prop_set_eq:NN \l_regex_success_submatches_prop
          \l_regex_current_submatches_prop
        \prg_map_break:
      }
  }
%    \end{macrocode}
% \end{macro}
%
% \subsection{Replacement}
%
% \subsubsection{Variables and helpers used in replacement}
%
% \begin{variable}{\l_regex_replacement_csnames_int}
%   The behaviour of closing braces inside a replacement text depends on
%   whether a sequences |\c{| or |\u{| has been encountered. The number
%       of \enquote{open} such sequences that should be closed by |}| is
%     stored in \cs{l_regex_replacement_csnames_int}, and decreased by
%     $1$ by each |}|.
%    \begin{macrocode}
\int_new:N \l_regex_replacement_csnames_int
%    \end{macrocode}
% \end{variable}
%
% \begin{variable}{\l_regex_balance_tl}
%   This token list holds the replacement text for
%   \cs{regex_replacement_balance_one_match:n} while it is being built
%   incrementally.
%    \begin{macrocode}
\tl_new:N \l_regex_balance_tl
%    \end{macrocode}
% \end{variable}
%
% \begin{macro}[aux, rEXP]{\regex_replacement_balance_one_match:n}
%   This expects as an argument the first index of a range of \tn{skip}
%   registers which hold the submatch information for a given match. It
%   can be used within an integer expression to obtain the brace balance
%   incurred by performing the replacement on that match. This combines
%   the braces lost by removing the match, braces added by all the
%   submatches appearing in the replacement, and braces appearing
%   explicitly in the replacement. Even though it is always redefined
%   before use, we initialize it as for an empty replacement. An
%   important property is that concatenating several calls to that
%   function must result in a valid integer expression (hence a leading
%   |+| in the actual definition).
%    \begin{macrocode}
\cs_new:Npn \regex_replacement_balance_one_match:n #1
  { - \regex_submatch_balance:n {#1} }
%    \end{macrocode}
% \end{macro}
%
% \begin{macro}[aux, rEXP]{\regex_replacement_do_one_match:n}
%   The input is the same as \cs{regex_replacement_balance_one_match:n}.
%   This function is redefined to expand to the part of the token list
%   from the end of the previous match to a given match, followed by the
%   replacement text. Hence concatenating the result of this function
%   with all possible arguments (one call for each match), as well as
%   the range from the end of the last match to the end of the string,
%   will produce the fully replaced token list. The initialization does
%   not matter, but we set it as for an empty replacement.
%    \begin{macrocode}
\cs_new:Npn \regex_replacement_do_one_match:n #1
  {
    \regex_query_range:nn
      { \etex_glueshrink:D \tex_skip:D #1 }
      { \tex_skip:D #1 }
  }
%    \end{macrocode}
% \end{macro}
%
% \begin{macro}[aux]{\regex_replacement_exp_not:N}
%   This function lets us navigate around the fact that the primitive
%   \cs{exp_not:n} requires a braced argument. As far as I can tell, it
%   is only needed if the user tries to include in the replacement text
%   a control sequence set equal to a macro parameter character, such as
%   \cs{c_parameter_token}. Indeed, within an \texttt{x}-expanding
%   assignment, \cs{exp_not:N}~|#| behaves as a single |#|, whereas
%   \cs{exp_not:n}~|{#}| behaves as a doubled |##|.
%    \begin{macrocode}
\cs_new:Npn \regex_replacement_exp_not:N #1 { \exp_not:n {#1} }
%    \end{macrocode}
% \end{macro}
%
% \subsubsection{Query and brace balance}
%
% \begin{macro}[int, rEXP]{\regex_query_range:nn}
% \begin{macro}[aux, rEXP]{\regex_query_range_loop:ww}
%   When it is time to extract submatches from the token list, the
%   various tokens are stored in \tn{toks} registers numbered from
%   \cs{l_regex_min_pos_int} inclusive to \cs{l_regex_max_pos_int}
%   exclusive. The function \cs{regex_query_range:nn} \Arg{min}
%   \Arg{max} unpacks registers from the position \meta{min} to the
%   position $\meta{max}-1$ included. Once this is expanded, a second
%   \texttt{x}-expansion will result in the actual tokens from the
%   query. That second expansion is only done by user functions at the
%   very end of their operation, after checking (and correcting) the
%   brace balance first.
%    \begin{macrocode}
\cs_new:Npn \regex_query_range:nn #1#2
  {
    \exp_after:wN \regex_query_range_loop:ww
    \int_use:N \__int_eval:w #1 \exp_after:wN ;
    \int_use:N \__int_eval:w #2 ;
    \prg_break_point:n { }
  }
\cs_new:Npn \regex_query_range_loop:ww #1 ; #2 ;
  {
    \if_int_compare:w #1 < #2 \exp_stop_f:
    \else:
      \exp_after:wN \prg_map_break:
    \fi:
    \tex_the:D \tex_toks:D #1 \exp_stop_f:
    \exp_after:wN \regex_query_range_loop:ww
      \int_use:N \__int_eval:w #1 + \c_one ; #2 ;
  }
%    \end{macrocode}
% \end{macro}
% \end{macro}
%
% \begin{macro}[int]{\regex_query_submatch:n}
%   When this function is called, \tn{skip}$i$ holds the start and end
%   positions for the $i$-th overall submatch as its main and stretch
%   components. In the case of repeated matches, submatches from all the
%   matches are put one after the other in blocks of
%   \cs{l_regex_capturing_group_int} \tn{skip} registers.
%    \begin{macrocode}
\cs_new:Npn \regex_query_submatch:n #1
  {
    \regex_query_range:nn
      { \tex_skip:D \__int_eval:w #1 }
      { \etex_gluestretch:D \tex_skip:D \__int_eval:w #1 }
  }
%    \end{macrocode}
% \end{macro}
%
% \begin{macro}[rEXP]{\regex_submatch_balance:n}
%   Every user function must result in a balanced token list (unbalanced
%   token lists cannot be stored by TeX). When we unpacked the query, we
%   kept track of the brace balance as the shrink component of
%   \tn{muskip} registers, hence the contribution from a given range is
%   the difference between the shrink components of
%   \tn{muskip}\meta{max~pos} and \tn{muskip}\meta{min~pos}. For the
%   $i$-th submatch, the end-points of the range are the main and
%   stretch components of \tn{skip}$i$. The trailing \cs{scan_stop:} is
%   gobbled by \cs{etex_muexpr:D}, and the whole expression can be cast
%   safely to an integer (no trailing expansion).
%    \begin{macrocode}
\cs_new_protected:Npn \regex_submatch_balance:n #1
  {
    \etex_glueshrink:D \etex_mutoglue:D \etex_muexpr:D
      \tex_muskip:D \etex_gluestretch:D \tex_skip:D #1
      - \tex_muskip:D \tex_skip:D #1
    \scan_stop:
  }
%    \end{macrocode}
% \end{macro}
%
% \subsubsection{Framework}
%
% \begin{macro}[int]{\regex_replacement:n}
% \begin{macro}[aux]{\regex_replacement_aux:n}
%   The replacement text is built incrementally by abusing \tn{toks}
%   within a group (see \pkg{l3tl-build}). We keep track in
%   \cs{l_regex_balance_int} of the balance of explicit begin- and
%   end-group tokens and \cs{l_regex_balance_tl} will consist of some
%   code to compute the brace balance from submatches (see its
%   description). Detect unescaped right braces, and escaped characters,
%   with trailing \cs{prg_do_nothing:} because some of the later
%   function look-ahead. Once the whole replacement text has been
%   parsed, make sure that there is no open csname. Finally, define the
%   \texttt{balance_one_match} and \texttt{do_one_match} functions.
%    \begin{macrocode}
\cs_new_protected:Npn \regex_replacement:n #1
  {
%<trace>    \trace_push:nnn { regex } { 1 } { regex_replacement:n }
    \tl_set_build:Nw \l_regex_internal_a_tl
      \int_zero:N \l_regex_balance_int
      \tl_clear:N \l_regex_balance_tl
      \regex_escape_use:nnnn
        {
          \if_charcode:w \c_rbrace_str ##1
            \regex_replacement_rbrace:N \else: \tl_build_one:n \fi: ##1
        }
        { \regex_replacement_escaped:N ##1 }
        { \tl_build_one:n ##1 }
        {#1}
      \prg_do_nothing: \prg_do_nothing:
      \if_int_compare:w \l_regex_replacement_csnames_int > \c_zero
        \__msg_kernel_error:nnx { regex } { replacement-missing-rbrace }
          { \int_use:N \l_regex_replacement_csnames_int }
        \tl_build_one:x
          { \prg_replicate:nn \l_regex_replacement_csnames_int \cs_end: }
      \fi:
      \cs_gset:Npx \regex_replacement_balance_one_match:n ##1
        {
          + \int_use:N \l_regex_balance_int
          \l_regex_balance_tl
          - \regex_submatch_balance:n {##1}
        }
    \tl_build_end:
    \exp_args:No \regex_replacement_aux:n \l_regex_internal_a_tl
%<trace>    \trace_pop:nnn { regex } { 1 } { regex_replacement:n }
  }
\cs_new_protected:Npn \regex_replacement_aux:n #1
  {
    \cs_set:Npn \regex_replacement_do_one_match:n ##1
      {
        \regex_query_range:nn
          { \etex_glueshrink:D \tex_skip:D ##1 }
          { \tex_skip:D ##1 }
        #1
      }
  }
%    \end{macrocode}
% \end{macro}
% \end{macro}
%
% \begin{macro}[aux]{\regex_replacement_escaped:N}
%   As in parsing a regular expression, we use an auxiliary built from
%   |#1| if defined. Otherwise, check for escaped digits (standing from
%   submatches from $0$ to $9$): anything else is a raw character.
%    \begin{macrocode}
\cs_new_protected:Npn \regex_replacement_escaped:N #1
  {
    \cs_if_exist_use:cF { regex_replacement_#1:w }
      {
        \if_int_compare:w \c_one < 1#1 \exp_stop_f:
          \regex_replacement_put_submatch:n {#1}
        \else:
          \tl_build_one:n #1
        \fi:
      }
  }
%    \end{macrocode}
% \end{macro}
%
% \subsubsection{Submatches}
%
% \begin{macro}[aux]{\regex_replacement_put_submatch:n}
%   Insert a submatch in the replacement text. This is dropped if the
%   submatch number is larger than the number of capturing groups.
%   Unless the submatch appears inside a |\c{...}| or |\u{...}|
%   construction, it must be taken into account in the brace balance.
%   Here, |##1| will receive a pointer to the $0$-th submatch for a
%   given match.  We cannot use \cs{int_eval:n} because it is
%   expandable, and would be expanded too early (short of adding
%   \cs{exp_not:N}, making the code messy again).
%    \begin{macrocode}
\cs_new_protected:Npn \regex_replacement_put_submatch:n #1
  {
    \if_int_compare:w #1 < \l_regex_capturing_group_int
      \tl_build_one:n { \regex_query_submatch:n { #1 + ##1 } }
      \if_int_compare:w \l_regex_replacement_csnames_int = \c_zero
        \tl_put_right:Nn \l_regex_balance_tl
          { + \regex_submatch_balance:n { \__int_eval:w #1+##1 \__int_eval_end: } }
      \fi:
    \fi:
  }
%    \end{macrocode}
% \end{macro}
%
% \begin{macro}[aux]{\regex_replacement_g:w, \regex_replacement_g_digits:NN}
%   An ugly method to grab digits for the |\g| escape sequence. At the
%   end of the run of digits, check that it ends with a right brace.
%    \begin{macrocode}
\cs_new_protected:Npn \regex_replacement_g:w #1#2
  {
    \str_if_eq:xxTF { #1#2 } { \tl_build_one:n \c_lbrace_str }
      {
        \int_zero:N \l_regex_internal_a_int
        \regex_replacement_g_digits:NN
      }
      { \regex_replacement_error:NNN g #1 #2 }
  }
\cs_new_protected:Npn \regex_replacement_g_digits:NN #1#2
  {
    \token_if_eq_meaning:NNTF #1 \tl_build_one:n
      {
        \if_int_compare:w \c_one < 1#2 \exp_stop_f:
          \int_set:Nn \l_regex_internal_a_int
            { \c_ten * \l_regex_internal_a_int + #2 }
          \exp_after:wN \use_i:nnn
          \exp_after:wN \regex_replacement_g_digits:NN
        \else:
          \exp_after:wN \regex_replacement_error:NNN
          \exp_after:wN g
        \fi:
      }
      {
        \if_meaning:w \regex_replacement_rbrace:N #1
          \exp_args:No \regex_replacement_put_submatch:n
            { \int_use:N \l_regex_internal_a_int }
          \exp_after:wN \use_none:nn
        \else:
          \exp_after:wN \regex_replacement_error:NNN
          \exp_after:wN g
        \fi:
      }
    #1 #2
  }
%    \end{macrocode}
% \end{macro}
%
% \subsubsection{Csnames in replacement}
%
% \begin{macro}[aux]{\regex_replacement_c:w}
% \begin{macro}[aux]+\regex_replacement_c_{:w+
%   |\c| can be followed by a left brace, or by a letter for which we
%   have defined a way to produce that category of characters.  The
%   appropriate definitions for catcodes are introduced later.  For
%   control sequences, if we are within a control sequence, convert
%   the token list to a string, otherwise simply prevent expansion,
%   with a weird cross-over between \cs{exp_not:n} and \cs{exp_not:N}
%   (see this helper's description for an explanation).
%    \begin{macrocode}
\cs_new_protected:Npn \regex_replacement_c:w #1#2
  {
    \token_if_eq_meaning:NNTF #1 \tl_build_one:n
      {
        \cs_if_exist_use:cF { regex_replacement_c_#2:w }
          { \regex_replacement_error:NNN c #1#2 }
      }
      { \regex_replacement_error:NNN c #1#2 }
  }
\cs_new_protected_nopar:cpn { regex_replacement_c_ \c_lbrace_str :w }
  {
    \if_case:w \l_regex_replacement_csnames_int
      \tl_build_one:n
        { \exp_not:n { \exp_after:wN \regex_replacement_exp_not:N \cs:w } }
    \else:
      \tl_build_one:n { \exp_not:n { \exp_after:wN \tl_to_str:N \cs:w } }
    \fi:
    \int_incr:N \l_regex_replacement_csnames_int
  }
%    \end{macrocode}
% \end{macro}
% \end{macro}
%
% \begin{macro}[aux]{\regex_replacement_u:w}
%   Check that |\u| is followed by a left brace. If so, start a control
%   sequence with \cs{cs:w}, which is then unpacked either with
%   \cs{exp_not:V} or \cs{tl_to_str:V} depending on the current context.
%    \begin{macrocode}
\cs_new_protected:Npn \regex_replacement_u:w #1#2
  {
    \str_if_eq:xxTF { #1#2 } { \tl_build_one:n \c_lbrace_str }
      {
        \if_case:w \l_regex_replacement_csnames_int
          \tl_build_one:n { \exp_not:n { \exp_after:wN \exp_not:V \cs:w } }
        \else:
          \tl_build_one:n { \exp_not:n { \exp_after:wN \tl_to_str:V \cs:w } }
        \fi:
        \int_incr:N \l_regex_replacement_csnames_int
      }
      { \regex_replacement_error:NNN u #1#2 }
  }
%    \end{macrocode}
% \end{macro}
%
% \begin{macro}[aux]{\regex_replacement_rbrace:N}
%   Within a |\c{...}| or |\u{...}| construction, end the control
%   sequence, and decrease the brace count. Otherwise, this is a raw
%   right brace.
%    \begin{macrocode}
\cs_new_protected:Npn \regex_replacement_rbrace:N #1
  {
    \if_int_compare:w \l_regex_replacement_csnames_int > \c_zero
      \tl_build_one:n \cs_end:
      \int_decr:N \l_regex_replacement_csnames_int
    \else:
      \tl_build_one:n #1
    \fi:
  }
%    \end{macrocode}
% \end{macro}
%
% \subsubsection{Characters in replacement}
%
% We will need to change the category code of the null character many
% times, hence work in a group. The catcode-specific macros below are
% defined in alphabetical order; if you are trying to understand the
% code, start from the end of the alphabet as those categories are
% simpler than active or begin-group.
%    \begin{macrocode}
\group_begin:
%    \end{macrocode}
%
% \begin{macro}[aux]{\regex_replacement_char:nNN}
%   The only way to produce an arbitrary character--catcode pair is to
%   use the \tn{lowercase} or \tn{uppercase} primitives. This is a
%   wrapper for our purposes. The first argument is the null character
%   with various catcodes. The second and third arguments are grabbed
%   from the input stream: |#3| is the character whose character code to
%   reproduce.
%    \begin{macrocode}
  \cs_new_protected:Npn \regex_replacement_char:nNN #1#2#3
    {
      \if_meaning:w \prg_do_nothing: #3
        \__msg_kernel_error:nn { regex } { replacement-catcode-end }
      \else:
        \tex_lccode:D \c_zero = `#3 \scan_stop:
        \tl_to_lowercase:n { \tl_build_one:n {#1} }
      \fi:
    }
%    \end{macrocode}
% \end{macro}
%
% \begin{macro}[aux]{\regex_replacement_c_A:w}
%   For an active character, expansion must be avoided, twice because we
%   later do two \texttt{x}-expansions, to unpack \tn{toks} for the
%   query, and to expand their contents to tokens of the query.
%    \begin{macrocode}
  \char_set_catcode_active:N \^^@
  \cs_new_protected_nopar:Npn \regex_replacement_c_A:w
    { \regex_replacement_char:nNN { \exp_not:n { \exp_not:N ^^@ } } }
%    \end{macrocode}
% \end{macro}
%
% \begin{macro}[aux]{\regex_replacement_c_B:w}
%   An explicit begin-group token increases the balance, unless within a
%   |\c{...}| or |\u{...}| construction. Add the desired begin-group
%   character, using the standard \cs{if_false:} trick. We eventually
%   \texttt{x}-expand twice. The first time must yield a balanced token
%   list, and the second one gives the bare begin-group token. The
%   \cs{exp_after:wN} is not strictly needed, but is more consistent
%   with \pkg{l3tl-analysis}.
%    \begin{macrocode}
  \char_set_catcode_group_begin:N \^^@
  \cs_new_protected_nopar:Npn \regex_replacement_c_B:w
    {
      \if_int_compare:w \l_regex_replacement_csnames_int = \c_zero
        \int_incr:N \l_regex_balance_int
      \fi:
      \regex_replacement_char:nNN
        { \exp_not:n { \exp_after:wN ^^@ \if_false: } \fi: } }
    }
%    \end{macrocode}
% \end{macro}
%
% \begin{macro}[aux]{\regex_replacement_c_C:w}
%   This is not quite catcode-related: when the user requests a
%   character with category \enquote{control sequence}, the
%   one-character control symbol is returned. As for the active
%   character, we prepare for two \texttt{x}-expansions.
%    \begin{macrocode}
  \cs_new_protected:Npn \regex_replacement_c_C:w #1#2
    { \tl_build_one:n { \exp_not:N \exp_not:N \exp_not:c {#2} } }
%    \end{macrocode}
% \end{macro}
%
% \begin{macro}[aux]{\regex_replacement_c_D:w}
%   Subscripts fit the mould: \tn{lowercase} the null byte with the
%   correct category.
%    \begin{macrocode}
  \char_set_catcode_math_subscript:N \^^@
  \cs_new_protected_nopar:Npn \regex_replacement_c_D:w
    { \regex_replacement_char:nNN { ^^@ } }
%    \end{macrocode}
% \end{macro}
%
% \begin{macro}[aux]{\regex_replacement_c_E:w}
%   Similar to the begin-group case, the second \texttt{x}-expansion
%   produces the bare end-group token.
%    \begin{macrocode}
  \char_set_catcode_group_end:N \^^@
  \cs_new_protected_nopar:Npn \regex_replacement_c_E:w
    {
      \if_int_compare:w \l_regex_replacement_csnames_int = \c_zero
        \int_decr:N \l_regex_balance_int
      \fi:
      \regex_replacement_char:nNN
        { \exp_not:n { \if_false: { \fi:  ^^@ } }
    }
%    \end{macrocode}
% \end{macro}
%
% \begin{macro}[aux]{\regex_replacement_c_L:w}
%   Simply \tn{lowercase} a letter null byte to produce an arbitrary letter.
%    \begin{macrocode}
  \char_set_catcode_letter:N \^^@
  \cs_new_protected_nopar:Npn \regex_replacement_c_L:w
    { \regex_replacement_char:nNN { ^^@ } }
%    \end{macrocode}
% \end{macro}
%
% \begin{macro}[aux]{\regex_replacement_c_M:w}
%   No surprise here, we lowercase the null math toggle.
%    \begin{macrocode}
  \char_set_catcode_math_toggle:N \^^@
  \cs_new_protected_nopar:Npn \regex_replacement_c_M:w
    { \regex_replacement_char:nNN { ^^@ } }
%    \end{macrocode}
% \end{macro}
%
% \begin{macro}[aux]{\regex_replacement_c_O:w}
%   Lowercase an other null byte.
%    \begin{macrocode}
  \char_set_catcode_other:N \^^@
  \cs_new_protected_nopar:Npn \regex_replacement_c_O:w
    { \regex_replacement_char:nNN { ^^@ } }
%    \end{macrocode}
% \end{macro}
%
% \begin{macro}[aux]{\regex_replacement_c_P:w}
%   For macro parameters, expansion is a tricky issue. We need to
%   prepare for two \texttt{x}-expansions and passing through various
%   macro definitions. Note that we cannot replace one \cs{exp_not:n} by
%   doubling the macro parameter characters because this would misbehave
%   if a mischievous user asks for |\c{\cP\#}|, since that macro
%   parameter character would be doubled.
%    \begin{macrocode}
  \char_set_catcode_parameter:N \^^@
  \cs_new_protected_nopar:Npn \regex_replacement_c_P:w
    {
      \regex_replacement_char:nNN
        { \exp_not:n { \exp_not:n { ^^@^^@^^@^^@ } } }
    }
%    \end{macrocode}
% \end{macro}
%
% \begin{macro}[aux]{\regex_replacement_c_S:w}
%   Spaces are normalized on input by \TeX{} to have character code
%   $32$. It is in fact impossible to get a token with character code
%   $0$ and category code $10$. Hence we use $32$ instead of $0$ as our
%   base character.
%    \begin{macrocode}
  \cs_new_protected:Npn \regex_replacement_c_S:w #1#2
    {
      \if_meaning:w \prg_do_nothing: #2
        \__msg_kernel_error:nn { regex } { replacement-catcode-end }
      \else:
        \if_int_compare:w `#2 = \c_zero
          \__msg_kernel_error:nn { regex } { replacement-null-space }
        \fi:
        \tex_lccode:D 32 = `#2 \scan_stop:
        \tl_to_lowercase:n { \tl_build_one:n {~} }
      \fi:
    }
%    \end{macrocode}
% \end{macro}
%
% \begin{macro}[aux]{\regex_replacement_c_T:w}
%   No surprise for alignment tabs here. Those are surrounded by the
%   appropriate braces whenever necessary, hence they don't cause
%   trouble in alignment settings.
%    \begin{macrocode}
  \char_set_catcode_alignment:N \^^@
  \cs_new_protected_nopar:Npn \regex_replacement_c_T:w
    { \regex_replacement_char:nNN { ^^@ } }
%    \end{macrocode}
% \end{macro}
%
% \begin{macro}[aux]{\regex_replacement_c_U:w}
%   Simple call to \cs{regex_replacement_char:nNN} which lowercases the
%   math superscript |^^@|.
%    \begin{macrocode}
  \char_set_catcode_math_superscript:N \^^@
  \cs_new_protected_nopar:Npn \regex_replacement_c_U:w
    { \regex_replacement_char:nNN { ^^@ } }
%    \end{macrocode}
% \end{macro}
%
% Restore the catcode of the null byte.
%    \begin{macrocode}
\group_end:
%    \end{macrocode}
%
% \subsubsection{An error}
%
% \begin{macro}[aux]{\regex_replacement_error:NNN}
%   Simple error reporting by calling one of the messages
%   \texttt{replacement-c}, \texttt{replacement-g}, or
%   \texttt{replacement-u}.
%    \begin{macrocode}
\cs_new_protected:Npn \regex_replacement_error:NNN #1#2#3
  {
    \__msg_kernel_error:nnx { regex } { replacement-#1 } {#3}
    #2 #3
  }
%    \end{macrocode}
% \end{macro}
%
% \subsection{User functions}
%
% \begin{macro}{\regex_new:N}
%   Before being assigned a sensible value, a regex variable matches
%   nothing.
%    \begin{macrocode}
\cs_new_protected:Npn \regex_new:N #1
  { \cs_new_eq:NN #1 \c_regex_no_match_regex }
%    \end{macrocode}
% \end{macro}
%
% \begin{macro}{\regex_set:Nn, \regex_gset:Nn, \regex_const:Nn}
%   Compile, then store the result in the user variable with the
%   appropriate assignment function.
%    \begin{macrocode}
\cs_new_protected_nopar:Npn \regex_set:Nn #1#2
  {
    \regex_compile:n {#2}
    \tl_set_eq:NN #1 \l_regex_internal_regex
  }
\cs_new_protected_nopar:Npn \regex_gset:Nn #1#2
  {
    \regex_compile:n {#2}
    \tl_gset_eq:NN #1 \l_regex_internal_regex
  }
\cs_new_protected_nopar:Npn \regex_const:Nn #1#2
  {
    \regex_compile:n {#2}
    \tl_const:Nx #1 { \exp_not:o \l_regex_internal_regex }
  }
%    \end{macrocode}
% \end{macro}
%
% \begin{macro}{\regex_show:N, \regex_show:n}
%   User functions: the \texttt{n} variant requires compilation first.
%   Then show the variable with some appropriate text. The auxiliary
%   \cs{regex_show_aux:Nx} is defined in a different section.
%    \begin{macrocode}
\cs_new_protected:Npn \regex_show:n #1
  {
    \regex_compile:n {#1}
    \regex_show_aux:Nx \l_regex_internal_regex
      { { \tl_to_str:n {#1} } }
  }
\cs_new_protected:Npn \regex_show:N #1
  { \regex_show_aux:Nx #1 { variable~\token_to_str:N #1 } }
%    \end{macrocode}
% \end{macro}
%
% \begin{macro}[TF]{\regex_match:nn, \regex_match:Nn}
%   Those conditionals are based on a common auxiliary defined
%   later. Its first argument builds the \textsc{nfa} corresponding to
%   the regex, and the second argument is the query token list. Once we
%   have performed the match, convert the resulting boolean to
%   \cs{prg_return_true:} or \texttt{false}.
%    \begin{macrocode}
\prg_new_protected_conditional:Npnn \regex_match:nn #1#2 { T , F , TF }
  {
    \regex_match_aux:nn { \regex_build:n {#1} } {#2}
    \regex_aux_return:
  }
\prg_new_protected_conditional:Npnn \regex_match:Nn #1#2 { T , F , TF }
  {
    \regex_match_aux:nn { \regex_build:N #1 } {#2}
    \regex_aux_return:
  }
%    \end{macrocode}
% \end{macro}
%
% \begin{macro}{\regex_count:nnN, \regex_count:NnN}
%   Again, use an auxiliary whose first argument builds the \textsc{nfa}.
%    \begin{macrocode}
\cs_new_protected:Npn \regex_count:nnN #1
  { \regex_count_aux:nnN { \regex_build:n {#1} } }
\cs_new_protected:Npn \regex_count:NnN #1
  { \regex_count_aux:nnN { \regex_build:N #1 } }
%    \end{macrocode}
% \end{macro}
%
% \begin{macro}
%   {
%     \regex_extract_once:nnN, \regex_extract_once:NnN,
%     \regex_extract_all:nnN,  \regex_extract_all:NnN,
%     \regex_replace_once:nnN, \regex_replace_once:NnN,
%     \regex_replace_all:nnN,  \regex_replace_all:NnN,
%     \regex_split:nnN,        \regex_split:NnN
%   }
% \begin{macro}[TF]
%   {
%     \regex_extract_once:nnN, \regex_extract_once:NnN,
%     \regex_extract_all:nnN,  \regex_extract_all:NnN,
%     \regex_replace_once:nnN, \regex_replace_once:NnN,
%     \regex_replace_all:nnN,  \regex_replace_all:NnN,
%     \regex_split:nnN,        \regex_split:NnN
%   }
%   We define here $40$ user functions, following a common pattern in
%   terms of \texttt{_aux:nnN} auxiliaries, defined in the coming
%   subsections.  The auxiliary is handed \cs{regex_build:n} or
%   \cs{regex_build:N} with the appropriate regex argument, then all
%   other necessary arguments (replacement text, token list, \emph{etc.}
%   The conditionals call \cs{regex_aux_return:} to return either
%   \texttt{true} or \texttt{false} once matching has been performed.
%    \begin{macrocode}
\cs_set_protected:Npn \regex_tmp:w #1#2#3
  {
    \cs_new_protected:Npn #2 ##1 { #1 { \regex_build:n {##1} } }
    \cs_new_protected:Npn #3 ##1 { #1 { \regex_build:N  ##1  } }
    \prg_new_protected_conditional:Npnn #2 ##1##2##3 { T , F , TF }
      { #1 { \regex_build:n {##1} } {##2} ##3 \regex_aux_return: }
    \prg_new_protected_conditional:Npnn #3 ##1##2##3 { T , F , TF }
      { #1 { \regex_build:N  ##1  } {##2} ##3 \regex_aux_return: }
  }
\regex_tmp:w \regex_extract_once_aux:nnN
  \regex_extract_once:nnN \regex_extract_once:NnN
\regex_tmp:w \regex_extract_all_aux:nnN
  \regex_extract_all:nnN \regex_extract_all:NnN
\regex_tmp:w \regex_replace_once_aux:nnN
  \regex_replace_once:nnN \regex_replace_once:NnN
\regex_tmp:w \regex_replace_all_aux:nnN
  \regex_replace_all:nnN \regex_replace_all:NnN
\regex_tmp:w \regex_split_aux:nnN \regex_split:nnN \regex_split:NnN
%    \end{macrocode}
% \end{macro}
% \end{macro}
%
% \subsubsection{Variables and helpers for user functions}
%
% \begin{variable}{\l_regex_match_count_int}
%   The number of matches found so far is stored
%   in \cs{l_regex_match_count_int}. This is only used
%   in the \cs{regex_count:nnN} functions.
%    \begin{macrocode}
\int_new:N \l_regex_match_count_int
%    \end{macrocode}
% \end{variable}
%
% \begin{variable}{regex_begin, regex_end}
%   Those flags are raised to indicate extra begin-group
%   or end-group tokens when extracting submatches.
%    \begin{macrocode}
\flag_new:n { regex_begin }
\flag_new:n { regex_end }
%    \end{macrocode}
% \end{variable}
%
% \begin{variable}{\l_regex_submatch_int, \l_regex_zeroth_submatch_int}
%   The end-points of each submatch are stored as main and stretch
%   components of \tn{skip}\meta{submatch}, where \meta{submatch} ranges
%   from \cs{l_regex_max_state_int} (inclusive) to
%   \cs{l_regex_submatch_int} (exclusive). Each successful match comes
%   with a $0$-th submatch (the full match), and one match for each
%   capturing group: submatches corresponding to the last successful
%   match are labelled starting at
%   \texttt{zeroth_submatch}. Additionally, the shrink component of this
%   $0$-th submatch is the position at which that match attempt started:
%   this is used for splitting and replacements.
%    \begin{macrocode}
\int_new:N \l_regex_submatch_int
\int_new:N \l_regex_zeroth_submatch_int
%    \end{macrocode}
% \end{variable}
%
% \begin{macro}[aux]{\regex_aux_return:}
%   This function triggers either \cs{prg_return_false:} or
%   \cs{prg_return_true:} as appropriate to whether a match was found or
%   not. It is used by all user conditionals.
%    \begin{macrocode}
\cs_new_protected_nopar:Npn \regex_aux_return:
  {
    \if_meaning:w \c_true_bool \g_regex_success_bool
      \prg_return_true:
    \else:
      \prg_return_false:
    \fi:
  }
%    \end{macrocode}
% \end{macro}
%
% \subsubsection{Matching}
%
% \begin{macro}[aux]{\regex_match_aux:nn}
%   We don't track submatches, and stop after a single match. Build the
%   \textsc{nfa} with |#1|, and perform the match on the query |#2|.
%    \begin{macrocode}
\cs_new_protected:Npn \regex_match_aux:nn #1#2
  {
    \group_begin:
      \regex_disable_submatches:
      \regex_single_match:
      #1
      \regex_match:n {#2}
    \group_end:
  }
%    \end{macrocode}
% \end{macro}
%
% \begin{macro}[aux]{\regex_count_aux:nnN}
%   Again, we don't care about submatches. Instead of aborting after the
%   first \enquote{longest match} is found, we search for multiple
%   matches, incrementing \cs{l_regex_match_count_int} every time to
%   record the number of matches. Build the \textsc{nfa} and match. At
%   the end, store the result in the user's variable.
%    \begin{macrocode}
\cs_new_protected:Npn \regex_count_aux:nnN #1#2#3
  {
    \group_begin:
      \regex_disable_submatches:
      \int_zero:N \l_regex_match_count_int
      \regex_multi_match:n { \int_incr:N \l_regex_match_count_int }
      #1
      \regex_match:n {#2}
      \exp_args:NNNo
    \group_end:
    \int_set:Nn #3 { \int_use:N \l_regex_match_count_int }
  }
%    \end{macrocode}
% \end{macro}
%
% \subsubsection{Extracting submatches}
%
% \begin{macro}[aux]{\regex_extract_once_aux:nnN, \regex_extract_all_aux:nnN}
%   Match once or multiple times. After each match (or after the only
%   match), extract the submatches using \cs{regex_extract:}. At the
%   end, store the sequence containing all the submatches into the user
%   variable |#3| after closing the group.
%    \begin{macrocode}
\cs_new_protected:Npn \regex_extract_once_aux:nnN #1#2#3
  {
    \group_begin:
      \regex_single_match:
      #1
      \regex_match:n {#2}
      \regex_extract:
    \regex_group_end_extract_seq:N #3
  }
\cs_new_protected:Npn \regex_extract_all_aux:nnN #1#2#3
  {
    \group_begin:
      \regex_multi_match:n { \regex_extract: }
      #1
      \regex_match:n {#2}
    \regex_group_end_extract_seq:N #3
  }
%    \end{macrocode}
% \end{macro}
%
% \begin{macro}[aux]{\regex_split_aux:nnN}
%   Splitting at submatches is a bit more tricky. For each match,
%   extract all submatches, and replace the zeroth submatch by the part
%   of the query between the start of the match attempt and the start of
%   the zeroth submatch. This is inhibited if the delimiter matched an
%   empty token list at the start of this match attempt.  After the last
%   match, store the last part of the token list, which ranges from the
%   start of the match attempt to the end of the query. This step is
%   inhibited if the last match was empty and at the very end: decrement
%   \cs{l_regex_submatch_int}, which controls which \tn{skip} registers
%   will be used.
%    \begin{macrocode}
\cs_new_protected:Npn \regex_split_aux:nnN #1#2#3
  {
    \group_begin:
      \regex_multi_match:n
        {
          \if_int_compare:w \l_regex_start_pos_int < \l_regex_success_pos_int
            \regex_extract:
            \tex_skip:D \l_regex_zeroth_submatch_int
              = \l_regex_start_pos_int sp
                plus \tex_skip:D \l_regex_zeroth_submatch_int \scan_stop:
          \fi:
        }
      #1
      \regex_match:n {#2}
%<assert>\assert_int:n { \l_regex_current_pos_int = \l_regex_max_pos_int }
      \tex_skip:D \l_regex_submatch_int
        = \l_regex_start_pos_int sp plus \l_regex_max_pos_int sp \scan_stop:
      \int_incr:N \l_regex_submatch_int
      \if_meaning:w \c_true_bool \l_regex_empty_success_bool
        \if_int_compare:w \l_regex_start_pos_int = \l_regex_max_pos_int
          \int_decr:N \l_regex_submatch_int
        \fi:
      \fi:
    \regex_group_end_extract_seq:N #3
  }
%    \end{macrocode}
% \end{macro}
%
% \begin{macro}[aux]{\regex_group_end_extract_seq:N}
%   The end-points of submatches are stored as the main and stretch
%   components of \tn{skip} registers from \cs{l_regex_max_state_int} to
%   \cs{l_regex_submatch_int} (exclusive). Extract the relevant ranges
%   into \cs{l_regex_internal_a_tl}. We detect unbalanced results using
%   the two flags \texttt{regex_begin} and \texttt{regex_end}, raised
%   whenever we see too many begin-group or end-group tokens in a
%   submatch. We disable \cs{__seq_item:n} to prevent two
%   \texttt{x}-expansions.
%    \begin{macrocode}
\cs_new_protected:Npn \regex_group_end_extract_seq:N #1
  {
      \cs_set_eq:NN \__seq_item:n \scan_stop:
      \flag_clear:n { regex_begin }
      \flag_clear:n { regex_end }
      \tl_set:Nx \l_regex_internal_a_tl
        {
          \int_step_function:nnnN
            { \c_two * \l_regex_max_state_int }
            \c_one
            { \l_regex_submatch_int - \c_one }
            \regex_extract_seq_aux:n
        }
      \int_compare:nNnF
        { \flag_height:n { regex_begin } + \flag_height:n { regex_end } }
        = \c_zero
        {
          \__msg_kernel_error:nnxxx { regex } { result-unbalanced }
            { splitting~or~extracting~submatches }
            { \flag_height:n { regex_end } }
            { \flag_height:n { regex_begin } }
        }
      \use:x
        {
          \group_end:
          \tl_set:Nn \exp_not:N #1 { \l_regex_internal_a_tl }
        }
  }
%    \end{macrocode}
% \end{macro}
%
% \begin{macro}[aux, EXP]{\regex_extract_seq_aux:n, \regex_extract_seq_aux:ww}
%   The \texttt{:n} auxiliary builds one item of the sequence of
%   submatches. First compute the brace balance of the submatch, then
%   extract the submatch from the query, adding the appropriate braces
%   and raising a flag if the submatch is not balanced.
%    \begin{macrocode}
\cs_new:Npn \regex_extract_seq_aux:n #1
  {
    \__seq_item:n
      {
        \exp_after:wN \regex_extract_seq_aux:ww
        \__int_value:w \regex_submatch_balance:n {#1} ; #1;
      }
  }
\cs_new:Npn \regex_extract_seq_aux:ww #1; #2;
  {
    \if_int_compare:w #1 < \c_zero
      \flag_raise:n { regex_end }
      \prg_replicate:nn {-#1} { \exp_not:n { { \if_false: } \fi: } }
    \fi:
    \regex_query_submatch:n {#2}
    \if_int_compare:w #1 > \c_zero
      \flag_raise:n { regex_begin }
      \prg_replicate:nn {#1} { \exp_not:n { \if_false: { \fi: } } }
    \fi:
  }
%    \end{macrocode}
% \end{macro}
%
% \begin{macro}[aux]
%   {\regex_extract:, \regex_extract_aux_b:wn, \regex_extract_aux_e:wn}
%   Our task here is to extract from the property list
%   \cs{l_regex_success_submatches_prop} the list of end-points of
%   submatches, and store them in \tn{skip} registers, from
%   \cs{l_regex_zeroth_submatch_int} upwards. We begin by emptying those
%   \tn{skip} registers. Then for each \meta{key}--\meta{value} pair in
%   the property list update the appropriate \tn{skip} component. This
%   is somewhat a hack: the \meta{key} is a non-negative integer
%   followed by |<| or |>|, which we use in a comparison to $-1$. At the
%   end, store the information about the position at which the match
%   attempt started, as a shrink component.
%    \begin{macrocode}
\cs_new_protected_nopar:Npn \regex_extract:
  {
    \if_meaning:w \c_true_bool \g_regex_success_bool
      \int_set_eq:NN \l_regex_zeroth_submatch_int \l_regex_submatch_int
      \prg_replicate:nn \l_regex_capturing_group_int
        {
          \tex_skip:D \l_regex_submatch_int \c_zero sp \scan_stop:
          \int_incr:N \l_regex_submatch_int
        }
      \prop_map_inline:Nn \l_regex_success_submatches_prop
        {
          \if_int_compare:w ##1 \c_minus_one
            \exp_after:wN \regex_extract_aux_e:wn \__int_value:w
          \else:
            \exp_after:wN \regex_extract_aux_b:wn \__int_value:w
          \fi:
          \__int_eval:w \l_regex_zeroth_submatch_int + ##1 {##2}
        }
      \tex_skip:D \l_regex_zeroth_submatch_int
        = \tex_the:D \tex_skip:D \l_regex_zeroth_submatch_int
          minus \l_regex_start_pos_int sp \scan_stop:
    \fi:
  }
\cs_new_protected:Npn \regex_extract_aux_b:wn #1 < #2
  {
    \tex_skip:D #1 = #2 sp
      plus \etex_gluestretch:D \tex_skip:D #1 \scan_stop:
  }
\cs_new_protected:Npn \regex_extract_aux_e:wn #1 > #2
  {
    \tex_skip:D #1
      = 1 \tex_skip:D #1 plus #2 sp \scan_stop:
  }
%    \end{macrocode}
% \end{macro}
%
% \subsubsection{Replacement}
%
% \begin{macro}[aux]{\regex_replace_once_aux:nnN}
%   Build the \textsc{nfa} and the replacement functions, then find a
%   single match.  If the match failed, simply exit the
%   group. Otherwise, we do the replacement. Extract submatches. Compute
%   the brace balance corresponding to replacing this match by the
%   replacement (this depends on submatches). Prepare the replaced token
%   list: the replacement function produces the tokens from the start of
%   the query to the start of the match and the replacement text for
%   this match; we need to add the tokens from the end of the match to
%   the end of the query. Finally, store the result in the user's
%   variable after closing the group: this step involves an additional
%   \texttt{x}-expansion, and checks that braces are balanced in the
%   final result.
%    \begin{macrocode}
\cs_new_protected:Npn \regex_replace_once_aux:nnN #1#2#3
  {
    \group_begin:
      \regex_single_match:
      #1
      \regex_replacement:n {#2}
      \exp_args:No \regex_match:n { #3 }
      \if_meaning:w \c_false_bool \g_regex_success_bool
        \group_end:
      \else:
        \regex_extract:
        \int_set:Nn \l_regex_balance_int
          {
            \regex_replacement_balance_one_match:n
              { \l_regex_zeroth_submatch_int }
          }
        \tl_set:Nx \l_regex_internal_a_tl
          {
            \regex_replacement_do_one_match:n { \l_regex_zeroth_submatch_int }
            \regex_query_range:nn
              { \etex_gluestretch:D \tex_skip:D \l_regex_zeroth_submatch_int }
              { \l_regex_max_pos_int }
          }
        \regex_group_end_replace:N #3
      \fi:
  }
%    \end{macrocode}
% \end{macro}
%
% \begin{macro}[aux]{\regex_replace_all_aux:nnN}
%   Match multiple times, and for every match, extract submatches and
%   additionally store the position at which the match attempt started
%   (as the shrink component of a \tn{skip} register). The \tn{skip}
%   registers from \cs{l_regex_max_state_int} to
%   \cs{l_regex_submatch_int} hold information about submatches of every
%   match in order; each match corresponds to
%   \cs{l_regex_capturing_group_int} consecutive \tn{skip} registers.
%   Compute the brace balance corresponding to doing all the
%   replacements: this is the sum of brace balances for replacing each
%   match. Join together the replacement texts for each match (including
%   the part of the query before the match), and the end of the query.
%    \begin{macrocode}
\cs_new_protected:Npn \regex_replace_all_aux:nnN #1#2#3
  {
    \group_begin:
      \regex_multi_match:n { \regex_extract: }
      #1
      \regex_replacement:n {#2}
      \exp_args:No \regex_match:n {#3}
      \int_set:Nn \l_regex_balance_int
        {
          0
          \int_step_function:nnnN
            { \c_two * \l_regex_max_state_int }
            \l_regex_capturing_group_int
            { \l_regex_submatch_int - \c_one }
            \regex_replacement_balance_one_match:n
        }
      \tl_set:Nx \l_regex_internal_a_tl
        {
          \int_step_function:nnnN
            { \c_two * \l_regex_max_state_int }
            \l_regex_capturing_group_int
            { \l_regex_submatch_int - \c_one }
            \regex_replacement_do_one_match:n
          \regex_query_range:nn
            \l_regex_start_pos_int \l_regex_max_pos_int
        }
    \regex_group_end_replace:N #3
  }
%    \end{macrocode}
% \end{macro}
%
% \begin{macro}[aux]{\regex_group_end_replace:N}
%   If the brace balance is not $0$, raise an error. Then set the user's
%   variable |#1| to the \texttt{x}-expansion of
%   \cs{l_regex_internal_a_tl}, adding the appropriate braces to produce
%   a balanced result. And end the group.
%    \begin{macrocode}
\cs_new_protected_nopar:Npn \regex_group_end_replace:N #1
  {
    \if_int_compare:w \l_regex_balance_int = \c_zero
    \else:
      \__msg_kernel_error:nnxxx { regex } { result-unbalanced }
        { replacing }
        { \int_max:nn { - \l_regex_balance_int } { \c_zero } }
        { \int_max:nn { \l_regex_balance_int } { \c_zero } }
    \fi:
    \use:x
      {
        \group_end:
        \tl_set:Nn \exp_not:N #1
          {
            \if_int_compare:w \l_regex_balance_int < \c_zero
              \prg_replicate:nn { - \l_regex_balance_int }
                { { \if_false: } \fi: }
            \fi:
            \l_regex_internal_a_tl
            \if_int_compare:w \l_regex_balance_int > \c_zero
              \prg_replicate:nn { \l_regex_balance_int }
                { \if_false: { \fi: } }
            \fi:
          }
      }
  }
%    \end{macrocode}
% \end{macro}
%
% \subsubsection{Storing and showing compiled patterns}
%
% \subsection{Messages}
%
% Messages for the preparsing phase.
%    \begin{macrocode}
\__msg_kernel_new:nnnn { regex } { trailing-backslash }
  { Trailing~escape~character~\iow_char:N\\. }
  {
    A~regular~expression~or~its~replacement~text~ends~with~
    the~escape~character~\iow_char:N\\.~It~will~be~ignored.
  }
\__msg_kernel_new:nnnn { regex } { x-missing-rbrace }
  { Missing~closing~brace~in~\iow_char:N\\x~hexadecimal~sequence. }
  {
    You~wrote~something~like~
    `\iow_char:N\\x\{\int_to_hexadecimal:n{#1}'.~
    The~closing~brace~is~missing.
  }
\__msg_kernel_new:nnnn { regex } { x-overflow }
  { Character~code~'#1'~too~large~in~\iow_char:N\\x~hexadecimal~sequence. }
  {
    You~wrote~something~like~
    `\iow_char:N\\x\{\int_to_hexadecimal:n{#1}\}'.~
    The~character~code~'#1'~is~larger~than~\int_use:N \c_max_char_int.
  }
%    \end{macrocode}
%
% Invalid quantifier.
%    \begin{macrocode}
\__msg_kernel_new:nnnn { regex } { invalid-quantifier }
  { Braced~quantifier~'#1'~may~not~be~followed~by~'#2'. }
  {
    The~character~'#2'~is~invalid~in~the~braced~quantifier~'#1'.~
    The~only~valid~quantifiers~are~'*',~'?',~'+',~'{<int>}',~
    '{<min>,}'~and~'{<min>,<max>}',~followed~or~not~by~'?'.
  }
%    \end{macrocode}
%
% Messages for missing or extra closing brackets and parentheses, with
% some fancy singular/plural handling for the case of parentheses.
%    \begin{macrocode}
\__msg_kernel_new:nnnn { regex } { missing-rbrack }
  { Missing~right~bracket~inserted~in~regular~expression. }
  {
    LaTeX~was~given~a~regular~expression~where~a~character~class~
    was~started~with~'[',~but~the~matching~']'~is~missing.
  }
\__msg_kernel_new:nnnn { regex } { missing-rparen }
  {
    Missing~right~parenthes\int_compare:nTF{#1=1}{i}{e}s~
    inserted~in~regular~expression.
  }
  {
    LaTeX~was~given~a~regular~expression~with~\int_eval:n{#1}~
    more~left~parenthes\int_compare:nTF{#1=1}{i}{e}s~than~right~
    parenthes\int_compare:nTF{#1=1}{i}{e}s.
  }
\__msg_kernel_new:nnnn { regex } { extra-rparen }
  { Extra~right~parenthesis~ignored~in~regular~expression. }
  {
    LaTeX~came~across~a~closing~parenthesis~when~no~submatch~group~
    was~open.~The~parenthesis~will~be~ignored.
  }
%    \end{macrocode}
%
% Some escaped alphanumerics are not allowed everywhere.
%    \begin{macrocode}
\__msg_kernel_new:nnnn { regex } { bad-escape }
  {
    Invalid~escape~\c_backslash_str #1~
    \regex_if_in_cs:TF { within~a~control~sequence. }
      {
        \regex_if_in_class:TF
          { in~a~character~class. }
          { following~a~category~test. }
      }
  }
  {
    The~escape~sequence~\iow_char:N\\#1~may~not~appear~
    \regex_if_in_cs:TF
      {
        within~a~control~sequence~test~introduced~by~
        \iow_char:N\\c\iow_char:N\{.
      }
      {
        \regex_if_in_class:TF
          { within~a~character~class~ }
          { following~a~category~test~such~as~\iow_char:N\\cL~ }
        because~it~does~not~match~exactly~one~character.
      }
  }
%    \end{macrocode}
%
% Range errors.
%    \begin{macrocode}
\__msg_kernel_new:nnnn { regex } { range-missing-end }
  { Invalid~end-point~for~range~'#1-#2'~in~character~class. }
  {
    The~end-point~'#2'~of~the~range~'#1-#2'~may~not~serve~as~an~
    end-point~for~a~range:~alphanumeric~characters~should~not~be~
    escaped,~and~non-alphanumeric~characters~should~be~escaped.
  }
\__msg_kernel_new:nnnn { regex } { range-backwards }
  { Range~[#1-#2]~out~of~order~in~character~class. }
  {
    In~ranges~of~characters~[x-y]~appearing~in~character~classes,~
    the~first~character~code~must~not~be~larger~than~the~second.~
    Here,~#1~has~character~code~\int_eval:n {`#1},~while~#2~has~
    character~code~\int_eval:n {`#2}.
  }
%    \end{macrocode}
%
% Errors related to |\c| and |\u|.
%    \begin{macrocode}
\__msg_kernel_new:nnnn { regex } { c-bad-mode }
  { Invalid~nested~\iow_char:N\\c~escape~in~regular~expression. }
  {
    The~\iow_char:N\\c~escape~cannot~be~used~within~
    a~control~sequence~test~'\iow_char:N\\c{...}'.~
    To~combine~several~category~tests,~use~'\iow_char:N\\c[...]'.
  }
\__msg_kernel_new:nnnn { regex } { c-missing-rbrace }
  { Missing~right~brace~inserted~for~\iow_char:N\\c~escape. }
  {
    LaTeX~was~given~a~regular~expression~where~a~
    '\iow_char:N\\c\iow_char:N\{...'~construction~was~not~ended~
    with~a~closing~brace~'\iow_char:N\}'.
  }
\__msg_kernel_new:nnnn { regex } { c-missing-rbrack }
  { Missing~right~bracket~inserted~for~\iow_char:N\\c~escape. }
  {
    A~construction~'\iow_char:N\\c[...'~appears~in~a~
    regular~expression,~but~the~closing~']'~is~not~present.
  }
\__msg_kernel_new:nnnn { regex } { c-missing-category }
  { Invalid~character~'#1'~following~\iow_char:N\\c~escape. }
  {
    In~regular~expressions,~the~\iow_char:N\\c~escape~sequence~
    may~only~be~followed~by~a~left~brace,~a~left~bracket,~or~a~
    capital~letter~representing~a~character~category,~namely~
    one~of~ABCDELMOPSTU.
  }
\__msg_kernel_new:nnnn { regex } { u-missing-lbrace }
  { Missing~left~brace~following~\iow_char:N\\u~escape. }
  {
    The~\iow_char:N\\u~escape~sequence~must~be~followed~by~
    a~brace~group~with~the~name~of~the~variable~to~use.
  }
\__msg_kernel_new:nnnn { regex } { u-missing-rbrace }
  { Missing~right~brace~inserted~for~\iow_char:N\\u~escape. }
  {
    LaTeX~
    \str_if_eq_x:nnTF { } {#2}
      { reached~the~end~of~the~string~ }
      { encountered~an~escaped~alphanumeric~character '\iow_char:N\\#2'~ }
    when~parsing~the~argument~of~an~'\iow_char:N\\u\iow_char:N\{...\}'~escape.
  }
%    \end{macrocode}
%
% Errors when encountering the \textsc{posix} syntax |[:...:]|.
%    \begin{macrocode}
\__msg_kernel_new:nnnn { regex } { posix-unsupported }
  { POSIX~collating~element~'[#1 ~ #1]'~not~supported. }
  {
    The~[.foo.]~and~[=bar=]~syntaxes~have~a~special~meaning~in~POSIX~
    regular~expressions.~This~is~not~supported~by~LaTeX.~Maybe~you~
    forgot~to~escape~a~left~bracket~in~a~character~class?
  }
\__msg_kernel_new:nnnn { regex } { posix-unknown }
  { POSIX~class~[:#1:]~unknown. }
  {
    [:#1:]~is~not~among~the~known~POSIX~classes~
    [:alnum:],~[:alpha:],~[:ascii:],~[:blank:],~
    [:cntrl:],~[:digit:],~[:graph:],~[:lower:],~
    [:print:],~[:punct:],~[:space:],~[:upper:],~
    [:word:],~and~[:xdigit:].
  }
\__msg_kernel_new:nnnn { regex } { posix-missing-close }
  { Missing~closing~':]'~for~POSIX~class. }
  { The~POSIX~syntax~'#1'~must~be~followed~by~':]',~not~'#2'. }
%    \end{macrocode}
%
% In various cases, the result of a \pkg{l3regex} operation can leave us
% with an unbalanced token list, which we must re-balance by adding
% begin-group or end-group character tokens.
%    \begin{macrocode}
\__msg_kernel_new:nnnn { regex } { result-unbalanced }
  { Missing~brace~inserted~when~#1. }
  {
    LaTeX~was~asked~to~do~some~regular~expression~operation,~
    and~the~resulting~token~list~would~not~have~the~same~number~
    of~begin-group~and~end-group~tokens.~Braces~were~inserted:~
    #2~left,~#3~right.
  }
%    \end{macrocode}
%
% Error message for unknown options.
%    \begin{macrocode}
\__msg_kernel_new:nnnn { regex } { unknown-option }
  { Unknown~option~`#1'~for~regular~expressions. }
  {
    The~only~available~option~is~'case-insensitive',~toggled~by~
    '(?i)'~and~'(?-i)'.
  }
%    \end{macrocode}
%
% Errors in the replacement text.
%    \begin{macrocode}
\__msg_kernel_new:nnnn { regex } { replacement-c }
  { Misused~\iow_char:N\\c~command~in~a~replacement~text. }
  {
    In~a~replacement~text,~the~\iow_char:N\\c~escape~sequence~
    can~be~followed~by~one~of~the~letters~ABCDELMOPSTU~
    or~a~brace~group,~not~by~'#1'.
  }
\__msg_kernel_new:nnnn { regex } { replacement-u }
  { Misused~\iow_char:N\\u~command~in~a~replacement~text. }
  {
    In~a~replacement~text,~the~\iow_char:N\\u~escape~sequence~
    must~be~~followed~by~a~brace~group~holding~the~name~of~the~
    variable~to~use.
  }
\__msg_kernel_new:nnnn { regex } { replacement-g }
  { Missing~brace~for~the~\iow_char:N\\g~construction~in~a~replacement~text. }
  {
    In~the~replacement~text~for~a~regular~expression~search,~
    submatches~are~represented~either~as~\iow_char:N \\g{dd..d},~
    or~\\d,~where~`d'~are~single~digits.~Here,~a~brace~is~missing.
  }
\__msg_kernel_new:nnnn { regex } { replacement-catcode-end }
  {
    Missing~character~for~the~\iow_char:N\\c<category><character>~
    construction~in~a~replacement~text.
  }
  {
    In~a~replacement~text,~the~\iow_char:N\\c~escape~sequence~
    can~be~followed~by~one~of~the~letters~ABCDELMOPSTU~representing~
    the~character~category.~Then,~a~character~must~follow.~LaTeX~
    reached~the~end~of~the~replacement~when~looking~for~that.
  }
\__msg_kernel_new:nnnn { regex } { replacement-null-space }
  { TeX~cannot~build~a~space~token~with~character~code~0. }
  {
    You~asked~for~a~character~token~with~category~'space',~
    and~character~code~0,~for~instance~through~
    '\iow_char:N\\cS\iow_char:N\\x00'.~
    This~specific~case~is~impossible~and~will~be~replaced~
    by~a~normal~space.
  }
\__msg_kernel_new:nnnn { regex } { replacement-missing-rbrace }
  { Missing~right~brace~inserted~in~replacement~text. }
  {
    There~were~\int_use:N \l_regex_replacement_csnames_int \
    missing~right~braces.
  }
%    \end{macrocode}
%
% \begin{macro}[aux]{\regex_msg_repeated:nnN}
%   This is not technically a message, but seems related enough to go
%   there. The arguments are: |#1| is the minimum number of repetitions;
%   |#2| is the number of allowed extra repetitions ($-1$ for infinite
%   number), and |#3| tells us aboud lazyness.
%    \begin{macrocode}
\cs_new:Npn \regex_msg_repeated:nnN #1#2#3
  {
    \str_if_eq:xxF { #1 #2 } { 1 0 }
      {
        , ~ repeated ~
        \int_case:nnn {#2}
          {
            { -1 } { #1~or~more~times,~\bool_if:NTF #3 { lazy } { greedy } }
            {  0 } { #1~times }
          }
          {
            between~#1~and~\int_eval:n {#1+#2}~times,~
            \bool_if:NTF #3 { lazy } { greedy }
          }
      }
  }
%    \end{macrocode}
% \end{macro}
%
% \subsection{Code for tracing}
%
% The tracing code is still very experimental, and is meant to be used
% with the \pkg{l3trace} package, currently in \texttt{l3trial}.
%
% \begin{macro}[int]{\regex_trace_states:n}
%   This function lists the contents of all states of the \textsc{nfa},
%   stored in \tn{toks} from $0$ to \cs{l_regex_max_state_int}
%   (excluded).
%    \begin{macrocode}
%<*trace>
\cs_new_protected:Npn \regex_trace_states:n #1
  {
    \int_step_inline:nnnn
      \l_regex_min_state_int
      \c_one
      { \l_regex_max_state_int - 1 }
      {
        \trace:nnx { regex } { #1 }
          { \iow_char:N \\toks ##1 = { \tex_the:D \tex_toks:D ##1 } }
      }
  }
%</trace>
%    \end{macrocode}
% \end{macro}
%
%    \begin{macrocode}
%</package>
%    \end{macrocode}
%
% \end{implementation}
%
% \endinput
%^^A NOT IMPLEMENTED
%^^A    \p{xx}     a character with the xx property
%^^A    \P{xx}     a character without the xx property
%^^A    [[:xxx:]]  positive POSIX named set
%^^A    [[:^xxx:]] negative POSIX named set
%^^A    (?=...)    positive look ahead
%^^A    (?!...)    negative look ahead
%^^A    (?<=...)   positive look behind
%^^A    (?<!...)   negative look behind
%^^A    (?<name>...) or (?'name'...) or (?P<name>...)
%^^A               named capturing group
%^^A    \R         a newline sequence
%^^A    \X         an extended Unicode sequence
%^^A    (?C) or (?Cn)   callout with data n
%^^A    (?R)            recurse whole pattern
%^^A    (?[+-]n) or \g<[+-]n> or (?&name) or (?P>name) or \g<name>
%^^A                    call subpattern
%^^A    (?([+-]n)... or (?(<name>)...
%^^A                    reference condition
%^^A    (?(R)... or (?(Rn)... or (?(R&name)...
%^^A                    recursion condition
%^^A    (?(DEFINE)...   define subpattern for reference
%^^A    (?(assert)...   assertion condition
%^^A    (*ACCEPT)       force successful match
%^^A    (*FAIL)         force backtrack; synonym (*F)
%^^A    (*COMMIT)       overall failure, no advance of starting point
%^^A    (*PRUNE)        advance to next starting character
%^^A    (*SKIP)         advance start to current matching position
%^^A    (*THEN)         local failure, backtrack to next alternation
%^^A    (*CR) or (*LF) or (*CRLF) or (*ANYCRLF) or (*ANY)
%^^A                    newline convention
%^^A    (*BSR_ANYCRLF) or (*BSR_UNICODE)
%^^A                    change what \R matches.
%^^A
%^^A    \cx             "control-x", where x is any ASCII character
%^^A    \C              one byte, even in UTF-8 mode (best avoided)
%^^A    +               possessive quantifiers
%^^A    (?>...)         atomic, non-capturing group
%^^A    (?#....)        comment (not nestable)
%^^A    (?JmsUx)        options (duplicate names; multiline; single line;
%^^A                      ungreedy; extended)
%^^A    (*NO_START_OPT) no start-match optimization (PCRE_NO_START_OPTIMIZE)
%^^A    (*UTF8)         set UTF-8 mode (PCRE_UTF8)
%^^A    (*UCP)          set PCRE_UCP (use Unicode properties for \d etc)
%^^A    \n or \gn or \g{[-]n} or \g{name} or (?P=name)
%^^A    or \k<name> or \k'name' or \k{name}
%^^A                    back-references
