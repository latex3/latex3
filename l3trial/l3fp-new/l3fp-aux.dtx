% \iffalse meta-comment
%
%% File: l3fp-aux.dtx Copyright(C) 2011 The LaTeX3 Project
%%
%% It may be distributed and/or modified under the conditions of the
%% LaTeX Project Public License (LPPL), either version 1.3c of this
%% license or (at your option) any later version.  The latest version
%% of this license is in the file
%%
%%    http://www.latex-project.org/lppl.txt
%%
%% This file is part of the "l3trial bundle" (The Work in LPPL)
%% and all files in that bundle must be distributed together.
%%
%% The released version of this bundle is available from CTAN.
%%
%% -----------------------------------------------------------------------
%%
%% The development version of the bundle can be found at
%%
%%    http://www.latex-project.org/svnroot/experimental/trunk/
%%
%% for those people who are interested.
%%
%%%%%%%%%%%
%% NOTE: %%
%%%%%%%%%%%
%%
%%   Snapshots taken from the repository represent work in progress and may
%%   not work or may contain conflicting material!  We therefore ask
%%   people _not_ to put them into distributions, archives, etc. without
%%   prior consultation with the LaTeX Project Team.
%%
%% -----------------------------------------------------------------------
%%
%
%<*driver|package>
\RequirePackage{l3fp-new}
\GetIdInfo$Id$
  {L3 Experimental support for floating points}
%</driver|package>
%<*driver>
\documentclass[full]{l3doc}
\usepackage{amsmath}
\usepackage{l3fp-new}
\begin{document}
  \tableofcontents
  \DocInput{\jobname.dtx}
\end{document}
%</driver>
% \fi
%
% \title{^^A
%   The \textsf{l3fp-aux} package\\ Support for floating points^^A
%   \thanks{This file describes v\ExplFileVersion,
%     last revised \ExplFileDate.}^^A
% }
%
% \author{^^A
%  The \LaTeX3 Project\thanks
%    {^^A
%      E-mail:
%        \href{mailto:latex-team@latex-project.org}
%          {latex-team@latex-project.org}^^A
%    }^^A
% }
%
% \date{Released \ExplFileDate}
%
% \maketitle
%
% \begin{documentation}
%
% \section{Some useful \TeX{} techniques}
%
% This section is out of place here and should be moved to \emph{e.g.},
% \pkg{l3fp-aux}.
%
% Full expansion can be done with \cs{tex_romannumeral:D} |-`0|. It is
% stopped by spaces. We can avoid being stopped by explicit spaces
% (and some braces) if we simply add \cs{use:n} after~|-`0|.
%
% Testing if a character token |#1| is a digit can be done using
% \begin{verbatim}
% \if_num:w \c_nine < 1 \token_to_str:N #1
%   true code
% \else:
%   false code
% \fi:
% \end{verbatim}
% To exclude |0|, replace \cs{c_nine} by \cs{c_ten}. The use of
% \cs{token_to_str:N} ensures that a digit with any catcode is detected.
%
% When a positive integer |#1| is known to be less than $10^8$, the following
% trick will split it into two blocks of $4$ digits, padding with zeros
% on the left.
% \begin{verbatim}
% \cs_new:Npn \aux:NNNNNw #1 #2#3#4#5 #6; { {#2#3#4#5} {#6} }
% \exp_after:wN \aux:NNNNNw \int_use:N \int_eval:w 1 0000 0000 + #1 ;
% \end{verbatim}
% The idea is that adding $10^8$ to the number ensures that it has
% exactly $9$ digits, and can then easily find which digits correspond
% to what position in the number. Of course, this can be modified
% for any number of digits less or equal to~$9$ (we are limited by
% \TeX{}'s integers). This method is very heavily relied upon in
% \texttt{l3fp-basics}.
%
% \end{documentation}
%
% \begin{implementation}
%
% \section{Implementation}
%
% We start by ensuring that the required packages are loaded.
%    \begin{macrocode}
%<*package>
\ProvidesExplPackage
  {\ExplFileName}{\ExplFileDate}{\ExplFileVersion}{\ExplFileDescription}
\package_check_loaded_expl:
%</package>
%<*initex|package>
%    \end{macrocode}
%
% \subsection{Various constants}
%
% \begin{macro}{\c_twenty_six}
%   Integers used in the fp modules.
%    \begin{macrocode}
\int_const:Nn \c_twenty_six {26}
  % \int_const:Nn \c_one_hundred {100}
%    \end{macrocode}
% \end{macro}
%
% \subsection{Constants, and structure of floating points}
%
% \begin{macro}{\s_fp,\fp_use:w}
%   Floating points numbers all start with \cs{s_fp} \cs{fp_use:w},
%   where \cs{s_fp} is equal to \cs{tex_relax:D}, and \cs{fp_use:w}
%   is protected. The rest of the floating point number is made of
%   characters (and \cs{tex_relax:D}, see \cs{fp_info: ...}).
%   This ensures that nothing expands under f-expansion, nor under
%   x-expansion. However, when typeset, \cs{s_fp} does nothing, and
%   \cs{fp_use:w} is expanded. Thus, we define \cs{fp_use:w} such
%   that it typesets the corresponding floating point number.
%^^A \fp_use:w needs thought.
%    \begin{macrocode}
\scan_new:N \s_fp
\cs_new_protected:Npn \fp_use:w { \fp_to_str:w \s_fp \fp_use:w }
%    \end{macrocode}
% \end{macro}
%
% \begin{macro}{\s_fp_mark,\s_fp_stop}
%   Aliases of \cs{tex_relax:D}, used to terminate expressions.
%    \begin{macrocode}
\scan_new:N \s_fp_mark
\scan_new:N \s_fp_stop
%    \end{macrocode}
% \end{macro}
%
% \begin{macro}{\c_fp_min_exponent_int,\c_fp_max_exponent_int}
%   The minimum and maximum exponents. When the result of
%   a floating point operation has an exponent outside the range
%   $[\cs{c_fp_min_exponent_int}, \cs{c_fp_max_exponent_int}]$,
%   it is rounded to a zero or an infinity.
%    \begin{macrocode}
\int_const:Nn \c_fp_min_exponent_int {-1000000}
\int_const:Nn \c_fp_max_exponent_int {1000000}
%    \end{macrocode}
% \end{macro}
%
% \begin{macro}{\c_zero_fp,\c_minus_zero,
%     \c_minus_inf_fp,  \c_inf_fp,
%     \c_empty_qnan_fp, \c_empty_snan_fp}
%   The special floating points. All of them have the form
%   \begin{quote}
%     \cs{s_fp} \cs{fp_use:w} \meta{case} \meta{sign} \cs{fp_info:...} |;|
%   \end{quote}
%   where the dots in \cs{fp_info:...} can be any string,
%   meant to give a diagnosis of where the floating point was created.
%    \begin{macrocode}
\tl_const:Nn \c_zero_fp          { \s_fp \fp_use:w 0 0 \fp_info:  ; }
\tl_const:Nn \c_minus_zero_fp    { \s_fp \fp_use:w 0 2 \fp_info:  ; }
\tl_const:Nn \c_inf_fp           { \s_fp \fp_use:w 2 0 \fp_info:  ; }
\tl_const:Nn \c_minus_inf_fp     { \s_fp \fp_use:w 2 2 \fp_info:  ; }
\tl_const:Nn \c_empty_qnan_fp    { \s_fp \fp_use:w 3 1 \fp_info:  ; }
\tl_const:Nn \c_empty_snan_fp    { \s_fp \fp_use:w 3 1 \fp_info:  ; }
%    \end{macrocode}
%   We don't want \cs{fp_info:...} to expand under x-expansion,
%   so we let them to \cs{tex_relax:D}. This will happen automatically
%   for info messages created with \cs{use:c}.
%    \begin{macrocode}
\cs_new_eq:NN \fp_info:  \tex_relax:D
\cs_new_eq:NN \fp_info:q \tex_relax:D
\cs_new_eq:NN \fp_info:s \tex_relax:D
%    \end{macrocode}
% \end{macro}
%
% \begin{macro}{\fp_aux_zero_fp:N,\fp_aux_inf_fp:N}
%   In case of overflow or underflow, we have to output
%   a zero or infinity with a given sign.
%^^A Add diagnosis information to \fp_info:...
%    \begin{macrocode}
\cs_new:Npn \fp_aux_zero_fp:N #1 { \s_fp \fp_use:w 0 #1 \fp_info: ; }
\cs_new:Npn \fp_aux_inf_fp:N #1  { \s_fp \fp_use:w 2 #1 \fp_info: ; }
\cs_new:Npn \fp_aux_qnan_fp:N #1 { \s_fp \fp_use:w 3 1 #1 ; }
\cs_new:Npn \fp_aux_snan_fp:N #1
  { \fp_aux_error:N #1 \s_fp \fp_use:w 3 1 #1 ; }
%    \end{macrocode}
% \end{macro}
%
% \begin{macro}{\fp_aux_max_fp:N, \fp_aux_min_fp:N,
%     \c_max_fp, \c_minus_max_fp,
%     \c_min_fp, \c_minus_min_fp}
%   In some cases, we need to output the smallest and
%   biggest positive or negative finite numbers.
%    \begin{macrocode}
\cs_new:Npx \fp_aux_max_fp:N #1
  {
    \s_fp \fp_use:w 1 #1
    { \int_use:N \c_fp_max_exponent_int }
    {9999} {9999} {9999} {9999} ;
  }
\cs_new:Npx \fp_aux_min_fp:N #1
  {
    \s_fp \fp_use:w 1 #1
    { \int_use:N \c_fp_min_exponent_int }
    {1000} {0000} {0000} {0000} ;
  }
\tl_const:Nx \c_max_fp       { \fp_aux_max_fp:N 0 }
\tl_const:Nx \c_minus_max_fp { \fp_aux_max_fp:N 2 }
\tl_const:Nx \c_min_fp       { \fp_aux_min_fp:N 0 }
\tl_const:Nx \c_minus_min_fp { \fp_aux_min_fp:N 2 }
%    \end{macrocode}
% \end{macro}
%
% \subsection{Over- and Under-flow}
%
% \begin{macro}{
%     \fp_aux_overflow:w,
%     \fp_aux_underflow:w,
%     \fp_aux_exact_zero:w}
%   \footnote{Bruno: these ones should depend on the rounding mode,
%     and they should throw exceptions.}
%    \begin{macrocode}
\cs_new:Npn \fp_aux_overflow:w \s_fp \fp_use:w 1 #1 #2;
  { \fp_aux_inf_fp:N #1 }
\cs_new:Npn \fp_aux_underflow:w \s_fp \fp_use:w 1 #1 #2;
  { \fp_aux_zero_fp:N #1 }
\cs_new:Npn \fp_aux_exact_zero:w \s_fp \fp_use:w 1 #1 #2;
  { \exp_after:wN \c_zero_fp }
%    \end{macrocode}
% \end{macro}
%
%
% \subsection{Expanding after floating point number}
%
% \begin{macro}{\fp_aux_exp_after_fp:wN}
%   \begin{syntax}
%     \cs{fp_aux_exp_after_fp:wN} \cs{s_fp} \cs{fp_use:w}
%     ~~\meta{case} \meta{sign} \meta{body} |;| \meta{tokens}
%   \end{syntax}
%   Expands \meta{tokens} exactly once (with \cs{exp_after:wN}),
%   and leaves the floating point number unchanged.
%    \begin{macrocode}
\cs_new:Npn \fp_aux_exp_after_fp:wN \s_fp \fp_use:w #1
  {
    \if:w 1 #1
      \exp_after:wN \fp_aux_exp_after_normal:NNnnnnn
    \else:
      \exp_after:wN \fp_aux_exp_after_special:NNN
    \fi:
    #1
  }
%    \end{macrocode}
% \end{macro}
%
% \begin{macro}{\fp_aux_exp_after_special:NNN}
%   \begin{syntax}
%     \cs{fp_aux_exp_after_special:NNN} \meta{case} \meta{sign} \meta{info} |;|
%   \end{syntax}
%   Special floating point numbers are easy to jump over.
%   We don't grab the trailing |;| since we would put it back anyways.
%    \begin{macrocode}
\cs_new:Npn \fp_aux_exp_after_special:NNN #1 #2 #3
  {
    \exp_after:wN \s_fp
    \exp_after:wN \fp_use:w
    \exp_after:wN #1
    \exp_after:wN #2
    \exp_after:wN #3
    \exp_after:wN
  }
%    \end{macrocode}
% \end{macro}
%
% \begin{macro}{\fp_aux_exp_after_normal:NNnnnnn}
%   For normal floating point numbers, life is slightly harder,
%   since we have many tokens to jump over. Here it would be better
%   if the digits were not braced but instead were delimited
%   arguments. However, there's just so much that my brain can
%   cope with, and I'm refusing to recode some parts right now
%   to make this optimization possible.
%    \begin{macrocode}
\cs_new:Npn \fp_aux_exp_after_normal:NNnnnnn 1 #1 #2 #3#4#5#6
  {
    \exp_after:wN \fp_aux_exp_after_normal_ii:Nwwwww
    \exp_after:wN #1
    \int_value:w #2   \exp_after:wN ;
    \int_value:w 1 #3 \exp_after:wN ;
    \int_value:w 1 #4 \exp_after:wN ;
    \int_value:w 1 #5 \exp_after:wN ;
    \int_value:w 1 #6 \exp_after:wN
  }
\cs_new:Npn \fp_aux_exp_after_normal_ii:Nwwwww
    #1 #2; 1 #3 ; 1 #4 ; 1 #5 ; 1 #6 ;
  { \s_fp \fp_use:w 1 #1 {#2} {#3} {#4} {#5} {#6} ; }
%    \end{macrocode}
% \end{macro}
%
%
% \subsection{Decimate (dividing by a power of 10)}
%
% \begin{macro}{\fp_aux_decimate:nNnnnn}
%   \begin{syntax}
%     \cs{fp_aux_decimate:nNnnnn} \Arg{shift} \Arg{f_1}
%     ~~\Arg{X_1} \Arg{X_2} \Arg{X_3} \Arg{X_4}
%   \end{syntax}
%   Each \meta{X$\sb{i}$} consists in $4$ digits exactly,
%   and $1000\leq\meta{X1}<9999$. The first argument determines
%   by how much we shift the digits. \meta{f1} is called as follows:
%   \begin{syntax}
%     \meta{f1} \meta{rounding} \Arg{X'1} \Arg{X'2}
%   \end{syntax}
%   where $0\leq\meta{X$'\sb{i}$}<10^{8}-1$ are $8$ digit numbers,
%   forming the truncation of our number. In other words,
%   \[
%   \left(
%     \sum_{i=1}^{4} \meta{X$\sb{i}$} \cdot 10^{-4i} \cdot 10^{-\meta{shift}}
%     - \meta{X$'$1} \cdot 10^{-8} + \meta{X$'$2} \cdot 10^{-16}
%   \right)
%   \in [0,10^{-16}).
%   \]
%   To round properly later, we need to remember some information
%   about the difference. The \meta{rounding} digit is $0$ if and
%   only if the difference is exactly $0$, and $5$ if and only if
%   the difference is exactly $0.5\cdot 10^{-16}$. Otherwise, it
%   is the (non-$0$, non-$5$) digit closest to $10^{17}$ times the
%   difference.  In particular, if the shift is $17$ or more, all
%   the digits are dropped, \meta{rounding} is $1$ (not $0$), and
%   \meta{X'1} \meta{X'2} are both zero.
%
%   If the shift is $1$, the \meta{rounding} digit is simply the
%   only digit that was pushed out of the brace groups (this is
%   important for subtraction). It would be more natural for the
%   \meta{rounding} digit to be placed after the \meta{X$\sb{i}$},
%   but the choice we make involves less reshuffling.
%
%   Note that this function fails for negative \meta{shift}.
%    \begin{macrocode}
\cs_new:Npn \fp_aux_decimate:nNnnnn #1
  {
    \cs:w
      fp_aux_decimate_
      \if_num:w #1 > \c_sixteen
        tiny
      \else:
        \tex_romannumeral:D #1
      \fi:
      :Nnnnn
    \cs_end:
  }
%    \end{macrocode}
%   Each of the auxiliaries see the function \meta{f1},
%   followed by $4$ blocks of $4$ digits.
% \end{macro}
%
% \begin{macro}{\fp_aux_decimate_:Nnnnn}
% \begin{macro}{\fp_aux_decimate_tiny:Nnnnn}
%   If the \meta{shift} is zero, or too big, life is very easy.
%    \begin{macrocode}
\cs_new:Npn \fp_aux_decimate_:Nnnnn #1 #2#3#4#5
  { #1 0 {#2#3} {#4#5} }
\cs_new:Npn \fp_aux_decimate_tiny:Nnnnn #1 #2#3#4#5
  { #1 1 {00000000} {00000000} }
%    \end{macrocode}
% \end{macro}
% \end{macro}
%
% \begin{macro}{
%     \fp_aux_decimate_i:Nnnnn,    \fp_aux_decimate_ii:Nnnnn,
%     \fp_aux_decimate_iii:Nnnnn,  \fp_aux_decimate_iv:Nnnnn,
%     \fp_aux_decimate_v:Nnnnn,    \fp_aux_decimate_vi:Nnnnn,
%     \fp_aux_decimate_vii:Nnnnn,  \fp_aux_decimate_viii:Nnnnn,
%     \fp_aux_decimate_ix:Nnnnn,   \fp_aux_decimate_x:Nnnnn,
%     \fp_aux_decimate_xi:Nnnnn,   \fp_aux_decimate_xii:Nnnnn,
%     \fp_aux_decimate_xiii:Nnnnn, \fp_aux_decimate_xiv:Nnnnn,
%     \fp_aux_decimate_xv:Nnnnn,   \fp_aux_decimate_xvi:Nnnnn
%   }
%   \begin{syntax}
%     \cs{fp_aux_decimate_i:Nnnnn} \meta{f1}
%     ~~\Arg{X_1} \Arg{X_2} \Arg{X_3} \Arg{X_4}
%   \end{syntax}
%   Shifting happens in two steps: compute the \meta{rounding} digit,
%   and repack digits into two blocks of $8$. The sixteen functions
%   are very similar, and defined through \cs{fp_aux_tmp:w}.
%   The arguments are as follows: |#1| indicates which function is
%   being defined; after one step of expansion, |#2| yields the
%   \enquote{extra digits} which are then converted by
%   \cs{fp_aux_decimate_round:Nw} to the \meta{rounding} digit.
%   This triggers the f-expansion of
%   \cs{fp_aux_decimate_pack:nnnnnnnnnnw},\footnote{No, the argument
%     spec is not a mistake: the function calls an auxiliary to
%     do half of the job.} responsible for building two blocks of
%   $8$ digits, and removing the rest. For this to work, |#3|
%   alternates between braced and unbraced blocks of $4$ digits,
%   in such a way that the $5$ first and $5$ next token groups
%   yield the correct blocks of $8$ digits.
%    \begin{macrocode}
\cs_new:Npn \fp_aux_tmp:w #1 #2 #3
  {
    \cs_new:cpn {fp_aux_decimate_#1:Nnnnn} ##1 ##2##3##4##5
      {
        \exp_after:wN ##1
        \int_value:w
          \exp_after:wN \fp_aux_decimate_round:Nw #2 ;
        \fp_aux_decimate_pack:nnnnnnnnnnw #3 ;
      }
  }
\fp_aux_tmp:w {i}   {\use_none:nnn      #50} {       0 {#2} #3 {#4} #5   }
\fp_aux_tmp:w {ii}  {\use_none:nn       #5 } {       00 {#2} #3 {#4} #5  }
\fp_aux_tmp:w {iii} {\use_none:n        #5 } {       000 {#2} #3 {#4} #5 }
\fp_aux_tmp:w {iv}  {                   #5 } {      {0000} #2 {#3} #4    }
\fp_aux_tmp:w {v}   {\use_none:nnn    #4#5 } {     0 {0000} #2 {#3} #4   }
\fp_aux_tmp:w {vi}  {\use_none:nn     #4#5 } {     00 {0000} #2 {#3} #4  }
\fp_aux_tmp:w {vii} {\use_none:n      #4#5 } {     000 {0000} #2 {#3} #4 }
\fp_aux_tmp:w {viii}{                 #4#5 } {    {0000} 0000 {#2} #3    }
\fp_aux_tmp:w {ix}  {\use_none:nnn  #3#4+#5} {   0 {0000} 0000 {#2} #3   }
\fp_aux_tmp:w {x}   {\use_none:nn   #3#4+#5} {   00 {0000} 0000 {#2} #3  }
\fp_aux_tmp:w {xi}  {\use_none:n    #3#4+#5} {   000 {0000} 0000 {#2} #3 }
\fp_aux_tmp:w {xii} {               #3#4+#5} {  {0000} 0000 {0000} #2    }
\fp_aux_tmp:w {xiii}{\use_none:nnn#2#3+#4#5} { 0 {0000} 0000 {0000} #2   }
\fp_aux_tmp:w {xiv} {\use_none:nn #2#3+#4#5} { 00 {0000} 0000 {0000} #2  }
\fp_aux_tmp:w {xv}  {\use_none:n  #2#3+#4#5} { 000 {0000} 0000 {0000} #2 }
\fp_aux_tmp:w {xvi} {             #2#3+#4#5} {{0000} 0000 {0000} 0000    }
%    \end{macrocode}
% \end{macro}
%
% \begin{macro}{\fp_aux_decimate_round:Nw,
%     \fp_aux_decimate_pack:nnnnnnnnnnw}
%
%   \cs{fp_aux_decimate_round:Nw} will receive the \enquote{extra digits}
%   as its argument, and its expansion is triggered by \cs{int_value:w}.
%   If the first digit is neither $0$ nor $5$, then it is the \meta{rounding}
%   digit. Otherwise, if the remaining digits are not all zero, we need
%   to add $1$ to that leading digit to get the rounding digit. Some caution
%   is required, though, because there may be more than $10$
%   \enquote{extra digits}, and this may overflow \TeX{}'s integers.
%   Instead of feeding the digits directly to \cs{fp_aux_decimate_round:Nw},
%   they come split into several blocks, separated by $+$. Hence the first
%   \cs{int_eval:w} here.
%
%    \begin{macrocode}
\cs_new:Npn \fp_aux_decimate_round:Nw #1 #2;
  {
    \if_int_odd:w 0 \if:w 0 #1 1 \fi:
                    \if:w 5 #1 1 \fi:
              \exp_stop_f:
      \if_num:w \int_eval:w #2 > \c_zero
        \int_eval:w 1 +
      \fi:
    \fi:
    #1
  }
%    \end{macrocode}
%
%   The computation of the \meta{rounding} digit leaves an unfinished
%   \cs{int_value:w}, which expands the following functions. This
%   allows us to repack nicely the digits we keep. Those digits come
%   as an alternation of unbraced and braced blocks of $4$ digits,
%   such that the first $5$ groups of token consist in $4$ single digits,
%   and one brace group (in some order), and the next $5$ have the same
%   structure. This is followed by some digits and a semicolon.
%    \begin{macrocode}
\cs_new:Npn \fp_aux_decimate_pack:nnnnnnnnnnw #1#2#3#4#5
  { \fp_aux_decimate_pack_ii:nnnnnnw { #1#2#3#4#5 } }
\cs_new:Npn \fp_aux_decimate_pack_ii:nnnnnnw #1 #2#3#4#5#6 #7;
  { {#1} {#2#3#4#5#6} }
%    \end{macrocode}
% \end{macro}
%
%
% \subsection{Grabbing digits}
%
% Those functions are used in \pkg{l3fp-convert}.
% ^^A todo: thus move those to l3fp-convert.sty?
%
% \begin{macro}{
%     \fp_aux_grab_digits_ix:N,
%     \fp_aux_grab_digits_viii:N,
%     \fp_aux_grab_digits_vii:N,
%     \fp_aux_grab_digits_vi:N,
%     \fp_aux_grab_digits_v:N,
%     \fp_aux_grab_digits_iv:N,
%     \fp_aux_grab_digits_iii:N,
%     \fp_aux_grab_digits_ii:N,
%     \fp_aux_grab_digits_i:N}
%   \footnote{Bruno: describe why it is done like that.}
%   \footnote{Explain that the following token is fully expanded,
%     even after the last function. Also say that the first token
%     must come fully expanded.}
%   The function \cs{fp_aux_grab_digits_<int>:N} must be triggered by
%   a \cs{int_value:w} expansion. It reads characters which follow,
%   and outputs digits on the left, until meeting a non-digit, or
%   up to a maximum of |<int>| digits. The expansion is
%   \begin{quote}
%     \meta{digits} |,| \meta{filling 0} |;|
%   \end{quote}
%   Here, \meta{filling 0} is a string of zeros such that
%   \meta{digits} \meta{filling 0} has the length given by the
%   index of the function. Thus, \meta{filling 0} is empty if no
%   more digits were to be found, and can have up to $9$ zeros.
%    \begin{macrocode}
\cs_new:Npn \fp_aux_grab_digits_end:w
    #1 \fi: \fp_aux_expand:w { \fi: #1 }
%    \end{macrocode}
%   The other functions are defined through a common \cs{fp_aux_cs:w}.
%    \begin{macrocode}
\cs_set_protected_nopar:Npn \fp_aux_cs:w #1 #2 #3
  {
    \cs_new:cpn { fp_aux_grab_digits_ #1 :N } ##1
      {
        \if_num:w 9 < 1 ##1 \exp_stop_f:
          ##1 \exp_after:wN #2 \tex_romannumeral:D
        \else:
          \fp_aux_grab_digits_end:w #3 ##1
        \fi:
        \fp_aux_expand:w
      }
  }
\fp_aux_cs:w {ix}   \fp_aux_grab_digits_viii:N { , 000000000 ; 9}
\fp_aux_cs:w {viii} \fp_aux_grab_digits_vii:N  { , 00000000 ; 8}
\fp_aux_cs:w {vii}  \fp_aux_grab_digits_vi:N   { , 0000000 ; 7}
\fp_aux_cs:w {vi}   \fp_aux_grab_digits_v:N    { , 000000 ; 6}
\fp_aux_cs:w {v}    \fp_aux_grab_digits_iv:N   { , 00000 ; 5}
\fp_aux_cs:w {iv}   \fp_aux_grab_digits_iii:N  { , 0000 ; 4}
\fp_aux_cs:w {iii}  \fp_aux_grab_digits_ii:N   { , 000 ; 3}
\fp_aux_cs:w {ii}   \fp_aux_grab_digits_i:N    { , 00 ; 2}
\fp_aux_cs:w {i}    \fp_aux_grab_digits_:N     { , 0 ; 1}
\cs_new_nopar:Npn \fp_aux_grab_digits_:N { , ; 0 }
%    \end{macrocode}
% \end{macro}
%
%
% \subsection{Grabbing letters}
% Similar to grabbing digits.
%
% \begin{macro}{\fp_aux_grab_letters:N}
%   \begin{syntax}
%     \cs{cs:w} \meta{str} \cs{fp_aux_grab_letters:N} \meta{tokens}
%   \end{syntax}
%   f-expands and grabs tokens, and outputs letters on the left.
%   Stops at the first non-letter. A letter is here defined as
%   a character token whose uppercase code lies between |`A| and
%   |`Z| inclusive (fastest criterion I found).
%
%   At the end, inserts |; \cs_end:| (the semicolon makes it easier
%   to separate the letters in some cases).
%    \begin{macrocode}
\cs_new:Npn \fp_aux_grab_letters:N #1
  {
    \if_num:w
          \if_catcode:w \tex_relax:D #1
            \c_zero
          \else:
            \int_eval:w \tex_uccode:D `#1 / \c_twenty_six
          \fi:
          = \c_three
      #1
      \exp_after:wN \fp_aux_grab_letters:N
      \tex_romannumeral:D
    \else:
      \fp_aux_swap_fisuf:w ; \cs_end: #1
    \fi:
    \fp_aux_expand:w
  }
%    \end{macrocode}
% \end{macro}
%
% \subsection{Bracing and using arguments}
%
% \begin{macro}{\fp_aux_use_i_until_s:w}
%    \begin{macrocode}
\cs_new:Npn \fp_aux_use_i_until_s:w #1 ; { #1 }
%    \end{macrocode}
% \end{macro}
%
%    \begin{macrocode}
\cs_new:Npn \fp_aux_i_s:N #1 { #1; }
\cs_new:Npn \fp_aux_use_ii_s:NN #1#2 { #2; }
\cs_new:Npn \fp_aux_use_i_until_s:Nw #1 #2; {#1}
%    \end{macrocode}
%
%    \begin{macrocode}
\cs_new:Npn \fp_aux_use_i_ii:nnnn #1#2#3#4 { #1#2 }
%    \end{macrocode}
%
% \subsection{Expansion control}
%
% Two types of expansion are used a lot in the fp modules: simple
% f-expansion, and unpacking f-expansion.
%
% \begin{macro}{\fp_aux_expand:w}
%   \begin{syntax}
%     \cs{tex_romannumeral:D} \cs{fp_aux_expand:w} \meta{tokens}
%   \end{syntax}
%   Ensures that \meta{tokens} are space- and brace-stripped, and expanded.
%    \begin{macrocode}
\cs_new:Npn \fp_aux_expand:w #1 { -`0 #1 }
%    \end{macrocode}
% \end{macro}
%
%
% At some steps in parsing, we wish to have a stronger kind of expansion,
% which ignores spaces, and unpacks registers. It is not possible in
% general to provide that expansion. However, in our limited cases, it is
% possible, because only characters are allowed in floating point numbers
% or expressions. Registers can then be recognized as the only tokens
% other than characters which can remain after f-expansion. Thus, we can
% simply f-expand, then test if what follows is a character. If it is not,
% we tell \TeX{} to unpack what is coming, using the primitive
% \cs{tex_the:D}.
%
% \begin{macro}[EXP]{\fp_aux_unpack:w}
% \begin{macro}[EXP,aux]{\fp_aux_unpack_test:N}
%   \begin{syntax}
%     \cs{tex_romannumeral:D} \cs{fp_aux_unpack:w} \meta{tokens}
%   \end{syntax}
%   Ensures that \meta{tokens} are space- and brace-stripped,
%   f-expanded, and unpacked in the case of non-\cs{tex_relax:D}
%   registers.
%
%   Hitting the two tokens \cs{tex_romannumeral:D} \cs{fp_aux_unpack:w}
%   with one expansion step causes a full \texttt{f}-expansion of what
%   follows, and unpacking of registers.
%
%   \begin{texnote}
%     \cs{fp_aux_unpack:w} takes one argument in order to remove
%     any leading space (and to some extend braces). It is still possible
%     to make the expansion fail by putting things like |{ \macro}| in the
%     input stream (note the space before |\macro|, which stops the expansion
%     of |\tex_romannumeral:D -`0|).
%   \end{texnote}
%
%   It would be possible to recursively f-expand (useful in rare cases
%   for toks, but those are not used in \LaTeX3).
%    \begin{macrocode}
\cs_new:Npn \fp_aux_unpack:w #1
  {
    -`0
    \exp_after:wN \fp_aux_unpack_test:N
    \tex_romannumeral:D -`0 #1
  }
\cs_new:Npn \fp_aux_unpack_test:N #1
  {
    \if_catcode:w \scan_stop: #1
      \reverse_if:N \if_meaning:w \scan_stop: #1
        \tex_the:D
      \fi:
    \fi:
    #1
  }
%    \end{macrocode}
% \end{macro}
% \end{macro}
%
%
% \subsection{Rounding}
%
% \begin{macro}{\l_fp_rounding_mode_int,
%     \c_fp_rounding_mode_nearest_int,
%     \c_fp_rounding_mode_ninf_int,
%     \c_fp_rounding_mode_zero_int,
%     \c_fp_rounding_mode_pinf_int}
%    \begin{macrocode}
\int_new:N    \l_fp_rounding_mode_int
\int_const:Nn \c_fp_rounding_mode_nearest_int {0}
\int_const:Nn \c_fp_rounding_mode_ninf_int    {1}
\int_const:Nn \c_fp_rounding_mode_zero_int    {2}
\int_const:Nn \c_fp_rounding_mode_pinf_int    {3}
%    \end{macrocode}
% \end{macro}
%
% \begin{macro}{\fp_aux_round:NNN,
%     \fp_aux_round_to_nearest:NNN,
%     \fp_aux_round_to_ninf:NNN,
%     \fp_aux_round_to_zero:NNN,
%     \fp_aux_round_to_pinf:NNN}
%   \begin{syntax}
%     \cs{fp_aux_round:NNN} \meta{final sign} \meta{digit1} \meta{digit2}
%   \end{syntax}
%   This expands to \cs{c_zero} if in the current rounding mode a number
%   of sign \meta{final sign} and ending with the \meta{digits} should
%   be rounded towards zero (truncated), and to \cs{c_one} if it should
%   be rounded away from zero.  Typically used witin the scope of an
%   \cs{int_eval:w}, to add $1$ if needed, and thereby round correctly.
%
%   It is very important that \meta{final sign} be the final sign of the
%   result. Otherwise, the result will be incorrect in the case of
%   rounding towards $-\infty$ or towards $+\infty$. Also recall that
%   \meta{final sign} is $0$ for positive, and $2$ for negative.
%
%   We use a trick similar to \cs{prg_return_true:}
%   and \cs{prg_return_false:}, but more dirty. By default,
%   the functions below return \cs{c_zero}, but this is superseeded
%   by \cs{fp_aux_round_return_one:}, which instead returns \cs{c_one},
%   expanding everything and removing \cs{c_zero} in the process.
%   In the case of rounding towards $\pm\infty$, $0$, this is not
%   really useful, but it prepares us for the \enquote{round to nearest,
%     ties to even} mode.
%    \begin{macrocode}
\cs_new:Npn \fp_aux_round_return_one:
  { \exp_after:wN \c_one \tex_romannumeral:D }
\cs_new:Npn \fp_aux_round_to_ninf:NNN #1 #2 #3
  {
    \if:w 2 #1
      \if_num:w #3 > \c_zero
        \fp_aux_round_return_one:
      \fi:
    \fi:
    \c_zero
  }
\cs_new:Npn \fp_aux_round_to_zero:NNN #1 #2 #3 { \c_zero }
\cs_new:Npn \fp_aux_round_to_pinf:NNN #1 #2 #3
  {
    \if:w 0 #1
      \if_num:w #3 > \c_zero
        \fp_aux_round_return_one:
      \fi:
    \fi:
    \c_zero
  }
%    \end{macrocode}
%   The \enquote{round to nearest} mode is the default.
%   If the \meta{digit2} is larger than $5$, then round up.
%   If it is less than $5$, round down. If it is exactly $5$,
%   then round such that \meta{digit1} plus the result is even.
%   In other words, round up if \meta{digit1} is odd.
%    \begin{macrocode}
\cs_new:Npn \fp_aux_round_to_nearest:NNN #1 #2 #3
  {
    \if_num:w #3 > \c_five
      \fp_aux_round_return_one:
    \else:
      \if:w 5 #3
        \if_int_odd:w #2 \exp_stop_f:
          \fp_aux_round_return_one:
        \fi:
      \fi:
    \fi:
    \c_zero
  }
\cs_new_eq:NN \fp_aux_round:NNN \fp_aux_round_to_nearest:NNN
%    \end{macrocode}
% \end{macro}
%
% \begin{macro}{\fp_aux_round_s:NNNw}
%   \begin{syntax}
%     \cs{fp_aux_round_s:NNNw} \meta{final sign}
%     ~~\meta{digit} \meta{more digits} |;|
%   \end{syntax}
%   Expands to \cs{c_zero} or \cs{c_one}, followed by a
%   semicolon. The value \cs{c_zero} is returned if rounding
%   (in the current mode) the number $\meta{final sign}
%   \meta{digit}.\meta{more digits}$ to an integer yields
%   $\meta{final sign} \meta{digit}$, and \cs{c_one} if it
%   yields $\meta{final sign} (\meta{digit} + 1)$. See
%   \cs{fp_aux_round:NNN} for details on the various
%   rounding modes.
%
%   Note that \meta{more digits} must be a digit, followed
%   by a piece that does not overflow a \cs{int_use:N}
%   \cs{int_eval:w} construction. The only relevant
%   information about this piece is whether it is zero or not.
%^^A Optimize for various rounding modes.
%    \begin{macrocode}
\cs_new:Npn \fp_aux_round_s:NNNw #1 #2 #3 #4;
  {
    \exp_after:wN \fp_aux_round:NNN
    \exp_after:wN #1
    \exp_after:wN #2
    \int_use:N \int_eval:w
      \if_int_odd:w 0 \if:w 0 #3 1 \fi:
                      \if:w 5 #3 1 \fi:
                \exp_stop_f:
        \if_num:w \int_eval:w #4 > \c_zero
          1 +
        \fi:
      \fi:
      #3
    ;
  }
%    \end{macrocode}
% \end{macro}
%
% \begin{macro}{\fp_aux_round:NNNN}
%   \begin{syntax}
%     \cs{fp_aux_round:NNNN} \meta{final sign} \meta{digit} \meta{2d}
%   \end{syntax}
%   Identical to \cs{fp_aux_round_s:NNNw} except for trailing semicolon.
%   \footnote{Bruno: I shouldn't have almost identical functions like this.}
%   \begin{macrocode}
\cs_new:Npn \fp_aux_round:NNNN #1 #2 #3 #4
  {
    \exp_after:wN \fp_aux_round:NNN
    \exp_after:wN #1
    \exp_after:wN #2
    \int_use:N \int_eval:w
      \if_int_odd:w 0 \if:w 0 #3 1 \fi:
                      \if:w 5 #3 1 \fi:
                \exp_stop_f:
        \if_num:w #4 > \c_zero
          1 +
        \fi:
      \fi:
      #3
    \int_eval_end:
  }
%    \end{macrocode}
% \end{macro}
%
% \begin{macro}{\fp_aux_round_neg:NNN,
%     \fp_aux_round_to_nearest_neg:NNN,
%     \fp_aux_round_to_ninf_neg:NNN,
%     \fp_aux_round_to_zero_neg:NNN,
%     \fp_aux_round_to_pinf_neg:NNN}
%   \begin{syntax}
%     \cs{fp_aux_round_neg:NNN} \meta{final sign} \meta{digit1} \meta{digit2}
%   \end{syntax}
%   This expands to \cs{c_zero} or \cs{c_one}. Consider a number of
%   the form $ \meta{final sign}.X\ldots X\meta{digit1} $ with exactly
%   $15$ (non-all-zero) digits before \meta{digit1}, and subtract from it
%   $\meta{final sign}.0\ldots0\meta{digit2}$, where there are $16$ zeros.
%   If in the current rounding mode the result should be rounded down,
%   then this function returns \cs{c_one}. Otherwise, \emph{i.e.},
%   if the result is rounded back to the first operand, then this function
%   returns \cs{c_zero}.
%    \begin{macrocode}
\cs_new:Npn \fp_aux_round_to_ninf_neg:NNN #1 #2 #3
  {
    \if:w 0 #1
      \if_num:w #3 > \c_zero
        \fp_aux_round_return_one:
      \fi:
    \fi:
    \c_zero
  }
\cs_new:Npn \fp_aux_round_to_zero_neg:NNN #1 #2 #3
  {
    \if_num:w #3 > \c_zero
      \fp_aux_round_return_one:
    \fi:
    \c_zero
  }
\cs_new:Npn \fp_aux_round_to_pinf_neg:NNN #1 #2 #3
  {
    \if:w 2 #1
      \if_num:w #3 > \c_zero
        \fp_aux_round_return_one:
      \fi:
    \fi:
    \c_zero
  }
%    \end{macrocode}
%   It turns out that this negative \enquote{round to nearest}
%   is identical to the positive one. And this is the default mode.
%    \begin{macrocode}
\cs_new_eq:NN \fp_aux_round_to_nearest_neg:NNN
    \fp_aux_round_to_nearest:NNN
\cs_new_eq:NN \fp_aux_round_neg:NNN
    \fp_aux_round_to_nearest_neg:NNN
%    \end{macrocode}
% \end{macro}
%
% \subsection{Errors in expansion context}
%
% \begin{macro}{\fp_aux_error:N}
%   \begin{syntax}
%     \cs{fp_aux_error:N} \meta{token}
%   \end{syntax}
%   Displays the \meta{token} as an error message, expandably.
%   \footnote{Bruno: we could want to make use this mechanism
%     elsewhere as well (e.g. in V arg expansion).}
%    \begin{macrocode}
\cs_new:Npn \fp_aux_error:N #1
  { \msg_expandable_error:n {#1} }
%    \end{macrocode}
% \end{macro}
%
% \begin{macro}{\fp_aux_error_not_implemented:}
%    \begin{macrocode}
\cs_new:Npn \fp_aux_error_not_implemented:
  { \msg_expandable_error:n {Not~implemented~yet,~sorry!} }
%    \end{macrocode}
% \end{macro}
%
% \begin{macro}{\fp_snan_inf_minus_inf:}
%    \begin{macrocode}
\cs_new:Npx \fp_snan_inf_minus_inf:
  {
    \exp_not:N \exp_after:wN
    \exp_not:N \fp_aux_snan_fp:N
    \exp_not:N \exp_after:wN
    \exp_not:c { fp_info:~ inf - inf ; }
  }
%    \end{macrocode}
% \end{macro}
%
% \subsection{Misc}
%
% \begin{macro}{\fp_aux_swap_fisuf:w}
%    \begin{macrocode}
\cs_new:Npn \fp_aux_swap_fisuf:w #1 \fi: \fp_aux_expand:w { \fi: #1 }
%    \end{macrocode}
% \end{macro}
%
% \begin{macro}{\fp_aux_do_nothing:}
%   Same as \cs{prg_do_nothing:}, but protected. This is inserted
%   at the end of some computations to catch a trailing \cs{exp_after:wN}
%   inserted by the auxiliaries (since those are used for other purposes
%   where the expansion is critical). Making it protected eases the
%   detection of bugs, since it doesn't disappear in x-expansion.
%    \begin{macrocode}
\cs_new_protected_nopar:Npn \fp_aux_do_nothing: { }
%    \end{macrocode}
% \end{macro}
%
% \begin{macro}{\fp_aux_extract_info:N}
%   \begin{syntax}
%     \cs{fp_aux_extract_info:N} \cs{fp_...:~text}
%   \end{syntax}
%   Expands to \verb*|text | (with trailing space).
%   Minimal error-checking.
%    \begin{macrocode}
\cs_new_nopar:Npn \fp_aux_extract_info:N #1
  {
    \exp_after:wN \fp_aux_extract_info_aux:w
    \token_to_str:N #1 ~ \q_stop
  }
\cs_new_nopar:Npn \fp_aux_extract_info_aux:w #1 ~ #2 \q_stop {#2}
%    \end{macrocode}
% \end{macro}
%
% \begin{macro}{\fp_aux_split_relax:N}
%   \begin{syntax}
%     \cs{fp_aux_split_relax:N} \meta{token}
%   \end{syntax}
%   Grabs the pieces of the stringified \meta{token} which
%   lies after the first |s_|.
%   ^^A Bad implementation, and needs better name.
%    \begin{macrocode}
\cs_new:Npx \fp_aux_split_relax:N #1
  {
    \exp_not:n
      {
        \exp_after:wN \fp_aux_split_relax_aux:w
        \token_to_str:N
      }
    #1
    \exp_not:N \q_mark
    \tl_to_str:n {s_}
    \exp_not:N \q_mark
    \exp_not:N \q_stop
  }
\exp_args:Nno \use:nn
  { \cs_new:Npn \fp_aux_split_relax_aux:w #1 }
  { \tl_to_str:n { s_ } #2 \q_mark #3 \q_stop }
  {
    #2
  }
%    \end{macrocode}
% \end{macro}
%
%    \begin{macrocode}
%</initex|package>
%    \end{macrocode}
%
%\end{implementation}
%
%\PrintChanges
%
%\PrintIndex