% \iffalse meta-comment
%
%% File: l3fp-trig.dtx Copyright (C) 2011-2012 The LaTeX3 Project
%%
%% It may be distributed and/or modified under the conditions of the
%% LaTeX Project Public License (LPPL), either version 1.3c of this
%% license or (at your option) any later version.  The latest version
%% of this license is in the file
%%
%%    http://www.latex-project.org/lppl.txt
%%
%% This file is part of the "l3trial bundle" (The Work in LPPL)
%% and all files in that bundle must be distributed together.
%%
%% The released version of this bundle is available from CTAN.
%%
%% -----------------------------------------------------------------------
%%
%% The development version of the bundle can be found at
%%
%%    http://www.latex-project.org/svnroot/experimental/trunk/
%%
%% for those people who are interested.
%%
%%%%%%%%%%%
%% NOTE: %%
%%%%%%%%%%%
%%
%%   Snapshots taken from the repository represent work in progress and may
%%   not work or may contain conflicting material!  We therefore ask
%%   people _not_ to put them into distributions, archives, etc. without
%%   prior consultation with the LaTeX Project Team.
%%
%% -----------------------------------------------------------------------
%%
%
%<*driver>
\RequirePackage{l3names}
\GetIdInfo$Id: l3fp-trig.dtx 3514 2012-03-08 06:14:48Z bruno $
  {L3 Experimental floating-point arithmetic}
\documentclass[full]{l3doc}
\usepackage{l3fp-new}
\begin{document}
  \DocInput{\jobname.dtx}
\end{document}
%</driver>
% \fi
%
% \title{The \textsf{l3fp-trig} package\thanks{This file
%         has version number \ExplFileVersion, last
%         revised \ExplFileDate.}\\
% Floating point trigonometric functions}
% \author{^^A
%  The \LaTeX3 Project\thanks
%    {^^A
%      E-mail:
%        \href{mailto:latex-team@latex-project.org}
%          {latex-team@latex-project.org}^^A
%    }^^A
% }
% \date{Released \ExplFileDate}
%
% \maketitle
%
% \begin{documentation}
%
% \end{documentation}
%
% \begin{implementation}
%
% \section{Implementation}
%
%    \begin{macrocode}
%<*initex|package>
%    \end{macrocode}
%
% \subsection{Some constants}
%
% \subsection{Sine}
%
% \subsubsection{Sign, exponent, and special numbers}
%
% \begin{macro}[int, EXP]
%   {\fp_sin_fp:w, \fp_cos_fp:w, \fp_tan_fp:w, \fp_cot_fp:w}
% \begin{macro}[aux, EXP]{\fp_trig_inf_error:w}
%    \begin{macrocode}
\cs_new:Npn \fp_sin_fp:w \s_fp \fp_use:w #1#2
  {
    \if_case:w #1 \exp_stop_f:
           \fp_aux_case_return_same:w
    \or:
      \exp_after:wN \fp_trig_octant_Taylor:Nwww \exp_after:wN #2
      \int_use:N \int_eval:w \c_zero \exp_after:wN \fp_trig_npos:wNn
    \or:   \fp_aux_case_invalid:nnw {sin(} {)}
    \else: \fp_aux_case_return_same:w
    \fi:
    \s_fp \fp_use:w #1#2
  }
\cs_new:Npn \fp_cos_fp:w \s_fp \fp_use:w #1#2
  {
    \if_case:w #1 \exp_stop_f:
           \fp_aux_case_return_fp:Nw \c_one_fp
    \or:
      \exp_after:wN \fp_trig_octant_Taylor:Nwww \exp_after:wN 0
      \int_use:N \int_eval:w \c_two \exp_after:wN \fp_trig_npos:wNn
    \or:   \fp_aux_case_invalid:nnw {cos(} {)}
    \else: \fp_aux_case_return_same:w
    \fi:
    \s_fp \fp_use:w #1#2
  }
\cs_new:Npn \fp_tan_fp:w \s_fp \fp_use:w #1#2
  {
    \if_case:w #1 \exp_stop_f:
           \fp_aux_case_return_same:w
    \or:
      \exp_after:wN \fp_tan_octant_Taylor:NNwww
      \exp_after:wN 0
      \exp_after:wN #2
      \int_use:N \int_eval:w \c_zero
        \exp_after:wN \fp_trig_npos:wNn
    \or:   \fp_aux_case_invalid:nnw {tan(} {)}
    \else: \fp_aux_case_return_same:w
    \fi:
    \s_fp \fp_use:w #1#2
  }
\cs_new:Npn \fp_cot_fp:w \s_fp \fp_use:w #1#2
  {
    \if_case:w #1 \exp_stop_f:
           \exp_after:wN \fp_cot_zero:w
    \or:
      \exp_after:wN \fp_tan_octant_Taylor:NNwww
      \exp_after:wN 2
      \exp_after:wN #2
      \int_use:N \int_eval:w \c_two
        \exp_after:wN \fp_trig_npos:wNn
    \or:   \fp_aux_case_invalid:nnw {cot(} {)}
    \else: \fp_aux_case_return_same:w
    \fi:
    \s_fp \fp_use:w #1#2
  }
%    \end{macrocode}
% \end{macro}
% \end{macro}
%
% \begin{macro}[aux, EXP]{\fp_cot_zero:w}
%   This function is called to compute $\operatorname{cot}(\pm 0)=\pm
%   \infty$: infinity with the same sign.
%    \begin{macrocode}
\cs_new:Npn \fp_cot_zero:w \s_fp \fp_use:w #1#2#3;
  { \exp_after:wN \fp_aux_inf_fp:N \exp_after:wN #2 }
%    \end{macrocode}
% \end{macro}
%
% \begin{macro}[aux, rEXP]{\fp_trig_npos:wNn}
%   Here |#2| is the exponent.
%    \begin{macrocode}
\cs_new:Npn \fp_trig_npos:wNn \s_fp \fp_use:w 1#1#2
  {
    \if_int_compare:w #2 > \c_zero
      \exp_after:wN \fp_trig_large:w
    \else:
      \exp_after:wN \fp_trig_small:w \int_value:w -
    \fi:
    #2 ;
  }
\cs_new:Npn \fp_trig_small:w #1;
  {
    \exp_after:wN \exp_after:wN \exp_after:wN \fp_trig_small_aux:wwNN
    \prg_replicate:nn {#1} { 0 } ;
  }
\cs_new:Npn \fp_trig_small_aux:wwNN #1; #2#3#4#5;
  {
    \fp_aux_pack_twice_four:wNNNNNNNN
    \fp_aux_pack_twice_four:wNNNNNNNN
    \fp_aux_pack_twice_four:wNNNNNNNN
    .
    ;
    #1#2#3#4#5 0000 0000;
  }
%    \end{macrocode}
% \end{macro}
%
% \begin{macro}[int, EXP]{\fp_trig_large_i:w}
%   We use a value of $2\pi$ rounded up, consistent with the choice of
%   \cs{c_pi_fp}.  This is not quite correct from an accuracy
%   perspective, but has the nice property that $\sin(180\mathrm{deg}) =
%   0$ exactly.
%    \begin{macrocode}
\cs_new:Npn \fp_trig_large:w #1; #2#3;
  { \fp_trig_large_i:www #2; #3 {0000} {0000} ;  #1; }
\cs_new:Npn \fp_trig_large_i:www #1; #2; #3;
  {
    \if_meaning:w 0 #3 \fp_trig_large_break:w \fi:
    \exp_after:wN \fp_trig_large_ii:wnnnnnn
    \int_use:N \int_eval:w (#1-3141) / 6283 ; % always >=0
    {#1} #2;
    \int_use:N \int_eval:w \c_minus_one + #3;
  }
\cs_new:Npn \fp_trig_large_ii:wnnnnnn #1; #2#3#4#5#6#7;
  {
    \exp_after:wN \fp_trig_large_i:www % important: '0' everywhere
    \int_use:N \int_eval:w -5 0000 + #20 - #1*62831
      \exp_after:wN \fp_fixed_mul_pack:NNNNNw
      \int_use:N \int_eval:w 4 9995 0000 + #30 - #1*8530
        \exp_after:wN \fp_fixed_mul_pack:NNNNNw
        \int_use:N \int_eval:w 4 9995 0000 + #40 - #1*7179
          \exp_after:wN \fp_fixed_mul_pack:NNNNNw
          \int_use:N \int_eval:w 4 9995 0000 + #50 - #1*5880
            \exp_after:wN \fp_fixed_mul_pack:NNNNNw
            \int_use:N \int_eval:w 4 9995 0000 + #60 %^^A todo: simplify.
              \exp_after:wN \fp_fixed_mul_pack:NNNNNw
              \int_use:N \int_eval:w 5 0000 0000 + #70
    \exp_after:wN ;
    \exp_after:wN ;
  }
\cs_new:Npn \fp_trig_large_break:w \fi: #1; %#2; #3;
  { \fi: \fp_trig_octant_loop:nw }
\cs_new:Npn \fp_trig_octant_loop:nw #1#2;
  {
    \if_num:w #1 < 15707 \exp_stop_f:
      \fp_trig_octant_break:w
    \fi:
    + \c_two
    \fp_fixed_sub_back:wwN
      {15707} {9632} {6794} {8970} {0000} {0000} ; % (>=) pi/2
      {#1} #2;
    \fp_trig_octant_loop:nw
  }
\cs_new:Npn \fp_trig_octant_break:w #1 \fi: + #2#3 #4; #5#6; #7;
  {
    \fi:
    \if_num:w #5 < 7854 \exp_stop_f:
      \if_num:w #5 < \c_zero
        \exp_after:wN \fp_trig_octant_neg:w
      \fi:
      \exp_after:wN \fp_aux_use_i_until_s:nw
      \exp_after:wN .
    \fi:
    + \c_one
    \fp_fixed_sub:wwN
    {15707} {9632} {6794} {8970} {0000} {0000} ; % pi/2
    {#5} #6 ; . ;
  }
\cs_new:Npn \fp_trig_octant_neg:w #1\fi: #2; #3#4#5#6#7#8; #9
  { %^^A todo: extract a fp_fixed_neg:wN?
    \fi:
    + \c_seven % not - \c_one because the octant must be >0.
    \exp_after:wN \fp_fixed_add_after:NNNNNwN
    \int_use:N \int_eval:w 1 9999 9998 - #30000 - #4
      \exp_after:wN \fp_fixed_add_pack:NNNNNwN
      \int_use:N \int_eval:w 1 9999 9998 - #5#6
        \exp_after:wN \fp_fixed_add_pack:NNNNNwN
        \int_use:N \int_eval:w 2 0000 0000 - #7#8 ; {#9} ;
  }
%    \end{macrocode}
% \end{macro}
%
% \begin{macro}[aux, EXP]{\fp_trig_octant_Taylor:Nwww}
%   Here we receive a \meta{sign} ($0$ or $2$), an \meta{octant} (in the
%   range $[0,9]$), a \meta{fixed point} number, and junk delimited by a
%   semicolon.  The \meta{octant} tells us which Taylor series to use,
%   and gives us a sign, which we combine with the \meta{sign} to
%   produce the sign of the result.
%    \begin{macrocode}
\cs_new:Npn \fp_trig_octant_Taylor:Nwww #1#2.
  {
    \exp_after:wN \fp_trig_octant_Taylor_aux:NNww
    \cs:w
      fp_trig_Taylor_
      \if_int_odd:w \int_eval:w #2 / \c_two B \else: A \fi:
      :wwn
      \exp_after:wN
    \cs_end:
    \int_value:w
      \if_int_odd:w \int_eval:w (#2+\c_two) / \c_four
        #1
      \else:
        \if_meaning:w #1 0 2 \else: 0 \fi:
      \fi:
    \exp_stop_f:
  }
\cs_new:Npn \fp_trig_octant_Taylor_aux:NNww #1#2 #3; #4;
  {
    \fp_fixed_mul:wwn #3; #3;
    #1 #3;
    {
      \exp_after:wN \fp_aux_sanitize:Nw
      \exp_after:wN #2
      \int_use:N \int_eval:w \fp_fixed_to_float:wN
    }
    #2
  }
%    \end{macrocode}
% \end{macro}
%
% \begin{macro}[aux, EXP]{\fp_trig_Taylor_A:wwn, \fp_trig_Taylor_B:wwn}
%   The \texttt{A} Taylor series is used close to the zeros of our
%   functions.  The \texttt{B} Taylor series is used close to the
%   extrema.
%    \begin{macrocode}
\cs_new:Npn \fp_trig_Taylor_A:wwn #1;#2;
  {
    \fp_fixed_continue:wn {0000}{0000}{0000}{0028}{1145}{7254}; % 1/17!
    \fp_fixed_mul_sub_back:wwwn #1; {0000}{0000}{0000}{7647}{1637}{3182};
    \fp_fixed_mul_sub_back:wwwn #1; {0000}{0000}{0160}{5904}{3836}{8216};
    \fp_fixed_mul_sub_back:wwwn #1; {0000}{0002}{5052}{1083}{8544}{1719};
    \fp_fixed_mul_sub_back:wwwn #1; {0000}{0275}{5731}{9223}{9858}{9065};
    \fp_fixed_mul_sub_back:wwwn #1; {0001}{9841}{2698}{4126}{9841}{2698};
    \fp_fixed_mul_sub_back:wwwn #1; {0083}{3333}{3333}{3333}{3333}{3333};
    \fp_fixed_mul_sub_back:wwwn #1; {1666}{6666}{6666}{6666}{6666}{6667};
    \fp_fixed_mul_sub_back:wwwn #1;{10000}{0000}{0000}{0000}{0000}{0000};
    \fp_fixed_mul:wwn #2;
  }
\cs_new:Npn \fp_trig_Taylor_B:wwn #1;#2;
  {
    \fp_fixed_continue:wn {0000}{0000}{0000}{0001}{5619}{2070}; % 1/18!
    \fp_fixed_mul_sub_back:wwwn #1; {0000}{0000}{0000}{0477}{9477}{3324};
    \fp_fixed_mul_sub_back:wwwn #1; {0000}{0000}{0011}{4707}{4559}{7730};
    \fp_fixed_mul_sub_back:wwwn #1; {0000}{0000}{2087}{6756}{9878}{6810};
    \fp_fixed_mul_sub_back:wwwn #1; {0000}{0027}{5573}{1922}{3985}{8907};
    \fp_fixed_mul_sub_back:wwwn #1; {0000}{2480}{1587}{3015}{8730}{1587};
    \fp_fixed_mul_sub_back:wwwn #1; {0013}{8888}{8888}{8888}{8888}{8889};
    \fp_fixed_mul_sub_back:wwwn #1; {0416}{6666}{6666}{6666}{6666}{6667};
    \fp_fixed_mul_sub_back:wwwn #1; {5000}{0000}{0000}{0000}{0000}{0000};
    \fp_fixed_mul_sub_back:wwwn #1;{10000}{0000}{0000}{0000}{0000}{0000};
  }
%    \end{macrocode}
% \end{macro}
%
% \begin{macro}[aux, EXP]{\fp_tan_octant_Taylor:NNwww}
%   Similar to \cs{fp_trig_octant_Taylor:NNwww}, but now we want to
%   compute both the sine and the cosine.
%    \begin{macrocode}
\cs_new:Npn \fp_tan_octant_Taylor:NNwww #1#2#3. #4; #5;
  {
    \exp_after:wN \fp_aux_sanitize:Nw
      \int_value:w
        \if_int_odd:w \int_eval:w (#3 + \c_one) / \c_two \int_eval_end:
          \exp_after:wN \reverse_if:N
        \fi:
        \if_meaning:w #1#2 2 \else: 0 \fi:
      \exp_after:wN \exp_stop_f:
      \int_use:N \int_eval:w
        \fp_fixed_mul:wwn #4; #4;
        \fp_tan_Taylor:wwn #4; {#3}
  }
\cs_new:Npn \fp_tan_Taylor:wwn #1; #2; #3
  {
    \fp_fixed_continue:wn {0000}{0000}{1527}{3493}{0856}{7059};
    \fp_fixed_mul_sub_back:wwwn #1; {0000}{0159}{6080}{0274}{5257}{6472};
    \fp_fixed_mul_sub_back:wwwn #1; {0002}{4571}{2320}{0157}{2558}{8481};
    \fp_fixed_mul_sub_back:wwwn #1; {0115}{5830}{7533}{5397}{3168}{2147};
    \fp_fixed_mul_sub_back:wwwn #1; {1929}{8245}{6140}{3508}{7719}{2982};
    \fp_fixed_mul_sub_back:wwwn #1;{10000}{0000}{0000}{0000}{0000}{0000};
    \fp_fixed_mul:wwn #2;
      {
        \fp_fixed_continue:wn {0000}{0007}{0258}{0681}{9408}{4706};
        \fp_fixed_mul_sub_back:wwwn #1; {0000}{2343}{7175}{1399}{6151}{7670};
        \fp_fixed_mul_sub_back:wwwn #1; {0019}{2638}{4588}{9232}{8861}{3691};
        \fp_fixed_mul_sub_back:wwwn #1; {0536}{6357}{0691}{4344}{6852}{4252};
        \fp_fixed_mul_sub_back:wwwn #1; {5263}{1578}{9473}{6842}{1052}{6315};
        \fp_fixed_mul_sub_back:wwwn #1;{10000}{0000}{0000}{0000}{0000}{0000};
          {
            \if_int_odd:w \int_eval:w #3 / \c_two \int_eval_end:
            \else:
              \exp_after:wN \fp_tan_reverse_args:Nww
            \fi:
            \fp_fixed_div_to_float:ww
          }
      }
  }
\cs_new:Npn \fp_tan_reverse_args:Nww #1#2;#3; { #1 #3; #2; }
%    \end{macrocode}
% \end{macro}
%
%    \begin{macrocode}
%</initex|package>
%    \end{macrocode}
%
% \end{implementation}
%
% \PrintChanges
%
% \PrintIndex