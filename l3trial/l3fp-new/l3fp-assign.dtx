% \iffalse
%%
%% File: l3fp-assign.dtx Copyright (C) 2011 LaTeX3 project
%%
%% It may be distributed and/or modified under the conditions of the
%% LaTeX Project Public License (LPPL), either version 1.3c of this
%% license or (at your option) any later version.  The latest version
%% of this license is in the file
%%
%%    http://www.latex-project.org/lppl.txt
%%
%% This file is part of the ``l3trial bundle'' (The Work in LPPL)
%% and all files in that bundle must be distributed together.
%%
%% The released version of this bundle is available from CTAN.
%%
%% -----------------------------------------------------------------------
%%
%% The development version of the bundle can be found at
%%
%%    http://www.latex-project.org/svnroot/experimental/trunk/
%%
%% for those people who are interested.
%%
%%%%%%%%%%%
%% NOTE: %%
%%%%%%%%%%%
%%
%%   Snapshots taken from the repository represent work in progress and may
%%   not work or may contain conflicting material!  We therefore ask
%%   people _not_ to put them into distributions, archives, etc. without
%%   prior consultation with the LaTeX Project Team.
%%
%% -----------------------------------------------------------------------
%<*driver|package>
\RequirePackage{l3fp-new}
\GetIdInfo$Id$
  {L3 Experimental floating-point assignments}
%</driver|package>
%<*driver>
\documentclass[full]{l3doc}
\usepackage{amsmath}
\usepackage{l3fp-new}
\begin{document}
  \tableofcontents
  \DocInput{\jobname.dtx}
\end{document}
%</driver>
% \fi
%
% \title{The \textsf{l3fp-assign} package\thanks{This file
%         has version number \ExplFileVersion, last
%         revised \ExplFileDate.}\\
% Floating point expressions}
% \author{^^A
%  The \LaTeX3 Project\thanks
%    {^^A
%      E-mail:
%        \href{mailto:latex-team@latex-project.org}
%          {latex-team@latex-project.org}^^A
%    }^^A
% }
% \date{Released \ExplFileDate}
% \maketitle
%
% \begin{documentation}
%
% \end{documentation}
%
% \begin{implementation}
%
% \section{Implementation}
%
%
% We start by ensuring that the required packages are loaded.
%    \begin{macrocode}
%<*package>
\ProvidesExplPackage
  {\ExplFileName}{\ExplFileDate}{\ExplFileVersion}{\ExplFileDescription}
\package_check_loaded_expl:
%</package>
%<*initex|package>
%    \end{macrocode}
%
% \subsection{Assigning values}
%
% \begin{macro}{\fp_new:N}
%    \begin{macrocode}
\cs_new_protected_nopar:Npn \fp_new:N #1
  { \cs_new_eq:NN #1 \c_zero_fp }
\cs_generate_variant:Nn \fp_new:N {c}
%    \end{macrocode}
% \end{macro}
%
% \begin{macro}{\fp_set:Nn,\fp_gset:Nn}
%    \begin{macrocode}
\cs_new_protected_nopar:Npn \fp_set:Nn #1 #2
  { \tl_set:Nx #1 { \fp_parse:n {#2} } }
\cs_new_protected_nopar:Npn \fp_gset:Nn #1 #2
  { \tl_gset:Nx #1 { \fp_parse:n {#2} } }
%    \end{macrocode}
% \end{macro}
%
% \begin{macro}{\fp_const:Nn}
%    \begin{macrocode}
\cs_new_protected_nopar:Npn \fp_const:Nn #1 #2
  { \tl_const:Nx #1 { \fp_parse:n {#2} } }
\cs_generate_variant:Nn \fp_const:Nn {c}
%    \end{macrocode}
% \end{macro}
%
% \begin{macro}{\fp_set_eq:NN, \fp_set_eq:cN,
%     \fp_set_eq:Nc, \fp_set_eq:cc}
% \begin{macro}{\fp_gset_eq:NN, \fp_gset_eq:cN,
%     \fp_gset_eq:Nc, \fp_gset_eq:cc}
%    \begin{macrocode}
\cs_new_protected_nopar:Npn \fp_set_eq:NN #1 #2
  { \cs_set_eq:NN #1 #2 }
\cs_new_protected_nopar:Npn \fp_gset_eq:NN #1 #2
  { \cs_gset_eq:NN #1 #2 }
\cs_generate_variant:Nn \fp_set_eq:NN {c, Nc, cc}
\cs_generate_variant:Nn \fp_gset_eq:NN {c, Nc, cc}
%    \end{macrocode}
% \end{macro}
% \end{macro}
%
% \begin{macro}{\fp_set_from_old:NN}
%    \begin{macrocode}
\cs_new_protected_nopar:Npn \fp_set_from_old:NN #1 #2
  { \tl_set:Nx #1 { \fp_from_old:N #2 } }
%    \end{macrocode}
% \end{macro}
%
% \subsection{Showing values}
%
% \begin{macro}{\fp_show:n}
%   This shows the result of computing its argument.
%   Using two functions, and the odd-looking \cs{use:nn}
%   improves the display slightly.
%    \begin{macrocode}
\cs_new_protected:Npn \fp_show:n #1
  { \exp_args:Nf \fp_show_aux:n { \fp_use:n {#1} } }
\cs_new_protected:Npn \fp_show_aux:n #1 { \use:nn \tl_show:n { {#1} } }
%    \end{macrocode}
% \end{macro}
%
% \subsection{Operations on old floating point numbers}
%
% \begin{macro}{\fp_old_generic:NNn}
%   Just for tests\ldots{}
%    \begin{macrocode}
\cs_new_protected:Npn \fp_old_generic:NNNn #1 #2 #3 #4
  {
    #1 #3
      { \fp_to_old:n { \fp_parse:n { \fp_from_old:N #3 #2 (#4) } } }
  }
%    \end{macrocode}
% \end{macro}
%
% \begin{macro}{\fp_old_add:Nn,\fp_old_sub:Nn}
% \begin{macro}{\fp_old_gadd:Nn,\fp_old_gsub:Nn}
%   Just for tests\ldots{}
%    \begin{macrocode}
\cs_new:Npn \fp_old_add:Nn  { \fp_old_generic:NNNn \tl_set:Nf  + }
\cs_new:Npn \fp_old_sub:Nn  { \fp_old_generic:NNNn \tl_set:Nf  - }
\cs_new:Npn \fp_old_gadd:Nn { \fp_old_generic:NNNn \tl_gset:Nf + }
\cs_new:Npn \fp_old_gsub:Nn { \fp_old_generic:NNNn \tl_gset:Nf - }
%    \end{macrocode}
% \end{macro}
% \end{macro}
%
%    \begin{macrocode}
%</initex|package>
%    \end{macrocode}
%
%\end{implementation}
%
%\PrintChanges
%
%\PrintIndex