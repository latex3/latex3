% \iffalse meta-comment
%
%% File: cs-input.dtx Copyright (C) 2011 The LaTeX3 Project
%%
%% It may be distributed and/or modified under the conditions of the
%% LaTeX Project Public License (LPPL), either version 1.3c of this
%% license or (at your option) any later version.  The latest version
%% of this license is in the file
%%
%%    http://www.latex-project.org/lppl.txt
%%
%% This file is part of the "l3trial bundle" (The Work in LPPL)
%% and all files in that bundle must be distributed together.
%%
%% The released version of this bundle is available from CTAN.
%%
%% -----------------------------------------------------------------------
%%
%% The development version of the bundle can be found at
%%
%%    http://www.latex-project.org/svnroot/experimental/trunk/
%%
%% for those people who are interested.
%%
%%%%%%%%%%%
%% NOTE: %%
%%%%%%%%%%%
%%
%%   Snapshots taken from the repository represent work in progress and may
%%   not work or may contain conflicting material!  We therefore ask
%%   people _not_ to put them into distributions, archives, etc. without
%%   prior consultation with the LaTeX3 Project.
%%
%% -----------------------------------------------------------------------
%
%<*driver|package>
\RequirePackage{expl3,l3str}
\GetIdInfo$Id$
  {L3 Experimental input of any list of tokens}
%</driver|package>
%<*driver>
\documentclass[full]{l3doc}
\usepackage{amsmath}
 %\usepackage{cs-input}
\begin{document}
  \DocInput{\jobname.dtx}
\end{document}
%</driver>
% \fi
%
%
% \title{^^A
%   The \textsf{cs-input} package: inputting arbitrary lists of tokens^^A
%   \thanks{This file describes v\ExplFileVersion,
%     last revised \ExplFileDate.}^^A
% }
%
% \author{^^A
%  The \LaTeX3 Project\thanks
%    {^^A
%      E-mail:
%        \href{mailto:latex-team@latex-project.org}
%          {latex-team@latex-project.org}^^A
%    }^^A
% }
%
% \date{Released \ExplFileDate}
%
% \maketitle
%
% \begin{documentation}
%
% \begin{function}{\cs_input:nn}
%   \begin{syntax}
%     \cs{cs_input:nn} \Arg{code} \Arg{string}
%   \end{syntax}
%   The \meta{string} argument is interpreted according to the rules below,
%   and the result is placed in a brace group following the \meta{code}.
%   For instance, we can define a function which prints to the terminal
%   its argument surrounded by three tilde characters on each side.
%   \begin{verbatim}
%     \cs_input:nn { \cs_set_protected:Npn \mypkg_show:n #1 }
%       { \c0{iow_term:n} \c1\{ \~\~\~ #1 \~\~\~ \c2\} }
%   \end{verbatim}
% \end{function}
%
% Characters in the \meta{string} argument behave mostly as in
% \cs{str_input:Nn}, except for |#| and the |\c| escape sequence.
% First, the following rules apply.
% \begin{itemize}
%   \item Any non-alphanumeric character can be escaped with a backslash.
%   \item Unescaped spaces are ignored.
%   \item Any alphanumeric character stands for itself.
%   \item |\a|, |\e|, |\f|, |\n|, |\r|, and |\t| stand for the ascii
%     characters \texttt{07}, \texttt{1B}, \texttt{0C}, \texttt{0A},
%     \texttt{0D}, and \texttt{09} respectively.
%   \item |\x| followed by up to two hexadecimal digits,
%     or a braced string of digits, is converted to the character
%     with that character code.
% \end{itemize}
% Then category codes can be modified as follows.
% \begin{itemize}
%   \item |#| stands for a macro parameter character (with character
%     code $35$).
%   \item |\c0|\Arg{cs name} gives rise to the control sequence built from
%     the \meta{cs name} (escaped characters can be used within the
%     \meta{cs name}, but the |\c0|\Arg{\ldots{}} cannot be nested).
%   \item |\c|\meta{hex digit}\meta{char} produces the \meta{char} with
%     the category code given by the \meta{hex digit} if this is a valid
%     category code for a token (namely, $5$, $9$, $14$, and $15$ are
%     excluded).
% \end{itemize}
% The hexadecimal digits corresponding to each category code are as follows.
% \begin{itemize}
%   \item |\c0|\Arg{cs name} produces a control sequence.
%   \item |\c1|\meta{char} produces a begin-group token.
%   \item |\c2|\meta{char} produces an end-group token.
%   \item |\c3|\meta{char} produces a math toggle token.
%   \item |\c4|\meta{char} produces an alignment token.
%   \item |\c6|\meta{char} produces a parameter token.
%   \item |\c7|\meta{char} produces a math superscript token.
%   \item |\c8|\meta{char} produces a math subscript token.
%   \item |\ca|\meta{char} or |\cA|\meta{char} produce a blank space token.
%   \item |\cb|\meta{char} or |\cB|\meta{char} produce a letter token.
%   \item |\cc|\meta{char} or |\cC|\meta{char} produce a token
%     with category code \enquote{other} (default).
%   \item |\cd|\meta{char} or |\cD|\meta{char} produce an active token.
% \end{itemize}
%
% \end{documentation}
%
% \begin{implementation}
%
% \section{\pkg{cs-input} implementation}
%
%    \begin{macrocode}
%<*package>
%    \end{macrocode}
%
%    \begin{macrocode}
\ProvidesExplPackage
  {\ExplFileName}{\ExplFileDate}{\ExplFileVersion}{\ExplFileDescription}
%    \end{macrocode}
%
% \begin{variable}{\l_cs_input_balance_int}
%   Since the user inputs begin and end group tokens independently
%   of each other, we need to check that they are as many of each.
%    \begin{macrocode}
\int_new:N \l_cs_input_balance_int
%    \end{macrocode}
% \end{variable}
%
% \begin{macro}{\cs_input:nn}
%   We have a three-pass system.
%   First convert escaped characters such as |\%| to |%|, \emph{etc.}
%   Then change category codes, leaving each begin and end-group character
%   in a construction such as |\if_false: { \fi: }|.
%   In the third pass, expand those constructions and put the user code
%   in front in \cs{g_str_result_tl}. Finally, run the contents of that
%   string.
%    \begin{macrocode}
\cs_new_protected:Npn \cs_input:nn #1#2
  {
    \group_begin:
      \cs_input_i:n {#2}
      \cs_input_ii:
      \cs_input_iii:n {#1}
    \group_end:
    \g_str_result_tl
  }
%    \end{macrocode}
% \end{macro}
%
% \begin{macro}[int]{\cs_input_i:n}
%   In the first pass, everything is delegated to
%   \cs{str_aux_escape:NNNn}. Only |#| and |\c| are
%   treated specially, and we let everything else
%   be converted to a simple character (with category
%   code $12$) by \cs{str_aux_escape:NNNn}).
%    \begin{macrocode}
\cs_new_protected_nopar:Npn \cs_input_i:n
  {
    \str_aux_escape:NNNn
      \cs_input_unescaped:N
      \cs_input_escaped:N
      \use:n
  }
%    \end{macrocode}
% \end{macro}
%
% \begin{macro}[aux]{\cs_input_unescaped:N}
%   Among unescaped characters, only |#| is special.
%    \begin{macrocode}
\cs_new_nopar:Npn \cs_input_unescaped:N #1
  {
    \if_charcode:w ## #1
      \exp_not:N \cs_input_macro_parameter:N
    \else:
      #1
    \fi:
  }
%    \end{macrocode}
% \end{macro}
%
% \begin{macro}[aux]{\cs_input_escaped:N}
%   Among escaped characters, only |\c| is special
%   (of course, |\a|, |\e|, |\f|, |\n|, |\r|, |\t| and |\x|
%   are as well, but that's already treated by \pkg{l3str}).
%    \begin{macrocode}
\cs_new_nopar:Npn \cs_input_escaped:N #1
  {
    \if_charcode:w c #1
      \exp_not:N \cs_input_char:NN
    \else:
      #1
    \fi:
  }
%    \end{macrocode}
% \end{macro}
%
% \begin{macro}[int]{\cs_input_ii:}
%   At the beginning of the second pass, most characters are simply
%   of category code $12$. The only macros that we can encounter
%   are \cs{cs_input_macro_parameter:N} and \cs{cs_input_char:NN}.
%   At this stage,
%   \begin{itemize}
%     \item \cs{cs_input_macro_parameter:N} checks the next token
%       and removes it if it is a second macro parameter;
%     \item \cs{cs_input_char:NN} \meta{catcode} \meta{character}
%       is converted to the appropriate token.
%   \end{itemize}
%   The trailing \cs{prg_do_nothing:} are inserted to ensure that
%   \cs{cs_input_char:NN} can safely grab two arguments, even when
%   the user wrongly ended the input string by |\c|.
%    \begin{macrocode}
\cs_new_protected_nopar:Npn \cs_input_ii:
  {
    \int_zero:N \l_cs_input_balance_int
    \tl_gset:Nx \g_str_result_tl
      { \g_str_result_tl \prg_do_nothing: \prg_do_nothing: }
  }
%    \end{macrocode}
% \end{macro}
%
% \begin{macro}[int]{\cs_input_macro_parameter:N}
%   When a |#| character was found in the input string,
%   it is converted to $4$ macro parameter tokens
%   (hence $8$ |#| characters since this is itself within a definition):
%   those immediately become only $2$ macro parameter tokens
%   in the definition of \cs{g_str_result}, and the third pass
%   cuts down to only one token.
%   If there were two consecutive |#| characters in the input string
%   (most common situation), remove the second one.
%   No error if somehow there was only one |#|.
%    \begin{macrocode}
\cs_new_nopar:Npn \cs_input_macro_parameter:N #1
  {
    ########
    \if_meaning:w \cs_input_macro_parameter:N #1
    \else:
      \exp_after:wN #1
    \fi:
  }
%    \end{macrocode}
% \end{macro}
%
% \begin{macro}[int]{\cs_input_char:NN}
%   Test if the next character is an hexadecimal digit
%   for a valid category code, and if so, check that the
%   following token is a character. Otherwise, raise the
%   appropriate error, and continue expanding, ignoring
%   the |\c| escape sequence.
%    \begin{macrocode}
\cs_new_nopar:Npn \cs_input_char:NN #1#2
  {
    \cs_if_exist:cTF { cs_input_ \token_to_str:N #1 :N }
      {
        \token_if_cs:NTF #2
          {
            \msg_expandable_kernel_error:nnn
              { cs-input } { no-char } { \c #1 }
            #2
          }
          { \use:c { cs_input_ \token_to_str:N #1 :N } #2 }
      }
      {
        \token_if_cs:NTF #1
          {
            \msg_expandable_kernel_error:nnn
              { cs-input } { no-char } { \c }
          }
          {
            \msg_expandable_kernel_error:nnn
              { cs-input } { bad-catcode } {#1}
          }
        #1 #2
      }
  }
%    \end{macrocode}
% \end{macro}
%
% \begin{macro}[int]{\cs_input_interrupt_lowercase:Nn}
% \begin{macro}[int]{\cs_input_resume:}
%
%    \begin{macrocode}
\cs_new_nopar:Npn \cs_input_interrupt_lowercase:Nn #1#2
  {
    \if_false: { \fi: }
    \char_set_lccode:nn { 0 } { `#1 }
    \tl_to_lowercase:n {#2}
  }
\cs_new_protected_nopar:Npn \cs_input_resume:
  { \tl_gput_right:Nx \g_str_result_tl { \if_false: } \fi: }
%    \end{macrocode}
% \end{macro}
% \end{macro}
%
% We are going to change the category code of the |^^@| character
% (character code $0$) for the definitions below. This is done
% within a group, and function definitions are global.
%    \begin{macrocode}
\group_begin:
%    \end{macrocode}
%
% \begin{macro}[aux]
%   {
%     \cs_input_3:N, \cs_input_4:N, \cs_input_7:N,
%     \cs_input_8:N, \cs_input_b:N
%   }
%   For most category codes, check that the following token is a character,
%   and lowercase |^^@| (character code $0$) with the right category code,
%   then resume expansion.
%    \begin{macrocode}
  \char_set_catcode_math_toggle:n { 0 }
  \cs_new_nopar:cpn { cs_input_3:N } #1
    {
      \cs_input_interrupt_lowercase:Nn #1
        { \cs_input_resume: ^^@ }
    }
  \char_set_catcode_alignment:n { 0 }
  \cs_new_nopar:cpn { cs_input_4:N } #1
    {
      \cs_input_interrupt_lowercase:Nn #1
        { \cs_input_resume: ^^@ }
    }
  \char_set_catcode_math_superscript:n { 0 }
  \cs_new_nopar:cpn { cs_input_7:N } #1
    {
      \cs_input_interrupt_lowercase:Nn #1
        { \cs_input_resume: ^^@ }
    }
  \char_set_catcode_math_subscript:n { 0 }
  \cs_new_nopar:cpn { cs_input_8:N } #1
    {
      \cs_input_interrupt_lowercase:Nn #1
        { \cs_input_resume: ^^@ }
    }
  \char_set_catcode_letter:n { 0 }
  \cs_new_nopar:Npn \cs_input_b:N #1
    {
      \cs_input_interrupt_lowercase:Nn #1
        { \cs_input_resume: ^^@ }
    }
%    \end{macrocode}
% \end{macro}
%
% \begin{macro}[aux]{\cs_input_c:N}
%   Category code \enquote{other} is trivial
%   since the character naturally comes with
%   that category code. Only spaces need to
%   be watched.^^A Do they?
%    \begin{macrocode}
  \cs_new_nopar:Npn \cs_input_c:N #1
    {
      \if_charcode:w \c_space_token #1
        \c_catcode_other_space_tl
      \else:
        #1
      \fi:
    }
%    \end{macrocode}
% \end{macro}
%
% \begin{macro}[aux]{\cs_input_6:N}
%   Macro parameter characters must be doubled for the second
%   and third pass, as well as for the definition of \cs{cs_input_6:}
%   itself. That's $8$ characters.
%    \begin{macrocode}
  \char_set_catcode_parameter:n { 0 }
  \cs_new_nopar:cpn { cs_input_6:N } #1
    {
      \cs_input_interrupt_lowercase:Nn #1
        { \cs_input_resume: ^^@^^@^^@^^@^^@^^@^^@^^@ }
    }
%    \end{macrocode}
% \end{macro}
%
% \begin{macro}[aux]{\cs_input_d:N}
%   Active characters must be protected from expansion now,
%   and for the third pass.
%    \begin{macrocode}
  \char_set_catcode_active:n { 0 }
  \cs_new_nopar:Npn \cs_input_d:N #1
    {
      \cs_input_interrupt_lowercase:Nn #1
        {
          \cs_input_resume:
          \exp_not:N \exp_not:N \exp_not:N ^^@
        }
    }
%    \end{macrocode}
% \end{macro}
%
% \begin{macro}[aux]{\cs_input_a:N}
%   The case of category code $10$ is a little bit more tricky.
%   Indeed, it is not possible to produce a |^^@| character with
%   category code $0$, because \TeX{} ignores an \tn{lccode} value
%   of $0$, and it converts any catcode-10 character to an explicit
%   space character token upon input.
%    \begin{macrocode}
  \cs_new_nopar:Npx \cs_input_a:N #1
    {
      \exp_not:n { \if_false: { \fi: } }
      \char_set_lccode:nn { 32 } { `#1 }
      \exp_not:N \int_compare:nNnT { `#1 } = \c_zero
        { \exp_not:N \msg_kernel_error:nn { cs-input } { impossible-char } }
      \tl_to_lowercase:n { \exp_not:n { \cs_input_resume: } ~ }
    }
%    \end{macrocode}
% \end{macro}
%
% \begin{macro}[aux]{\cs_input_1:N, \cs_input_2:N}
%   Begin- and end-group tokens are replaced by an appropriate
%   |\if_false:| \ldots{} |\fi:| constructionto remain balanced
%   at this stage. We also increment or decrement the balance
%   counter: this allows us to avoid problems in the third pass
%   if the user specified an unbalanced token list.
%    \begin{macrocode}
  \char_set_catcode_group_end:n { 0 }
  \cs_new_nopar:cpn { cs_input_2:N } #1
    {
      \cs_input_interrupt_lowercase:Nn #1
        {
          \int_decr:N \l_cs_input_balance_int
          \cs_input_resume:
          \exp_not:N \if_false: { \exp_not:N \fi: ^^@
        }
    }
  \char_set_catcode_group_begin:n { 0 }
  \cs_new_nopar:cpn { cs_input_1:N } #1
    {
      \cs_input_interrupt_lowercase:Nn #1
        {
          \int_incr:N \l_cs_input_balance_int
          \cs_input_resume:
          ^^@ \exp_not:N \if_false: } \exp_not:N \fi:
        }
    }
%    \end{macrocode}
% \end{macro}
%
% \begin{macro}[aux]
%   {
%     \cs_input_A:N, \cs_input_B:N,
%     \cs_input_C:N, \cs_input_D:N
%   }
%   Copies of the above functions for uppercase hexadecimal digits.
%    \begin{macrocode}
  \cs_new_eq:NN \cs_input_A:N \cs_input_a:N
  \cs_new_eq:NN \cs_input_B:N \cs_input_b:N
  \cs_new_eq:NN \cs_input_C:N \cs_input_c:N
  \cs_new_eq:NN \cs_input_D:N \cs_input_d:N
%    \end{macrocode}
% \end{macro}
%
% \begin{macro}[aux]{\cs_input_0:N}
% \begin{macro}[aux]{\cs_input_cs_collect:N}
%   If the argument is not a left brace, then we take it to mean
%   that the user wants a single character control sequence.
%   If the argument is a left brace, collect characters into
%   a \cs{cs:w} \ldots{} \cs{cs_end:} construction, until reaching
%   either a non-character or a right brace.
%    \begin{macrocode}
  \cs_new_nopar:cpn { cs_input_0:N } #1
    {
      \exp_not:N \exp_not:N
      \exp_after:wN \exp_not:N \cs:w
        \int_compare:nNnTF { `#1 } = { 123 } % `\{
          { \cs_input_cs_collect:N }
          { #1 \cs_end: }
    }
  \char_set_catcode:nn { 0 } { 12 }
  \char_set_lccode:nn { 0 } { 125 } % `\}
  \tl_to_lowercase:n
    {
      \cs_new_nopar:Npn \cs_input_cs_collect:N #1
        {
          \if_catcode:w \scan_stop: \exp_not:N #1
            \cs_end:
            \msg_expandable_kernel_error:nn { cs-input } { missing-rbrace }
            \exp_after:wN \use_i:nn
          \else:
            \if_meaning:w ^^@ #1
              \cs_end:
              \exp_after:wN \exp_after:wN
              \exp_after:wN \use_none:nn
            \fi:
          \fi:
          #1
          \cs_input_cs_collect:N
        }
    }
%    \end{macrocode}
% \end{macro}
% \end{macro}
%
% We are now done with the category code and lowercase code assignments,
% hence close the group.
%    \begin{macrocode}
\group_end:
%    \end{macrocode}
%
%
% \begin{macro}[int]{\cs_input_iii:n}
%   The current state is that \cs{g_str_result_tl} contains almost the
%   token list we want, but with begin and end-group tokens replaced
%   by an appropriate |\if_false:| construction which expands to what
%   the user wants. We thus expand \cs{g_str_result_tl} (for this to
%   work, we also had to insert \cs{exp_not:N} for active characters
%   and control sequences, and we doubled macro parameter tokens).
%   If the user has specified a different number of begin and end-group
%   tokens, we raise an error, and correct by adding left or right braces
%   to either side of the result.
%    \begin{macrocode}
\cs_new_protected_nopar:Npn \cs_input_iii:n #1
  {
    \int_compare:nNnF \l_cs_input_balance_int = \c_zero
      {
        \msg_kernel_error:nnx { cs } { unbalanced-input }
          { \int_use:N \l_cs_input_balance_int }
      }
    \tl_gset:Nx \g_str_result_tl
      {
        \exp_not:n {#1}
          {
            \if_int_compare:w \l_cs_input_balance_int < \c_zero
              \prg_replicate:nn { - \l_cs_input_balance_int }
                { { \if_false: } \fi: }
            \fi:
            \g_str_result_tl
            \if_int_compare:w \l_cs_input_balance_int > \c_zero
              \prg_replicate:nn { \l_cs_input_balance_int }
                { \if_false: { \fi: } }
            \fi:
          }
      }
  }
%    \end{macrocode}
% \end{macro}
%
% Messages.
%    \begin{macrocode}
\msg_kernel_new:nnn { cs-input } { no-char }
  { #1~not~followed~by~a~character. }
\msg_kernel_new:nnn { cs-input } { bad-catcode }
  { Invalid~category~code~#1! }
\use:x
  {
    \msg_kernel_new:nnn { cs-input } { missing-rbrace }
      { Missing~\iow_char:N \}~inserted~in~\iow_char:N \\c0{...}. }
  }
\msg_kernel_new:nnn { cs } { unbalanced-input }
  {
    You~asked~for~an~unbalanced~token~list: \\
    \int_compare:nNnTF {#1} > \c_zero
      { #1~more }
      { \int_eval:n {-#1}~less }
    \ begin~group~tokens~than~end~group~tokens.
  }
\msg_kernel_new:nnn { cs } { impossible-char }
  {
    TeX~cannot~handle~the~character~
    \iow_char:N \^^@~(number~0)~with~catcode~10.
  }
%    \end{macrocode}
%
%    \begin{macrocode}
%</package>
%    \end{macrocode}
%
% \end{implementation}
%
% \PrintIndex
