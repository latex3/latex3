% \iffalse
%
%% File l3fp-interchange.dtx (C) Copyright 2017-2018,2020,2021,2023 The LaTeX Project
%
% It may be distributed and/or modified under the conditions of the
% LaTeX Project Public License (LPPL), either version 1.3c of this
% license or (at your option) any later version.  The latest version
% of this license is in the file
%
%    http://www.latex-project.org/lppl.txt
%
% This file is part of the "l3trial bundle" (The Work in LPPL)
% and all files in that bundle must be distributed together.
%
% -----------------------------------------------------------------------
%
% The development version of the bundle can be found at
%
%    https://github.com/latex3/latex3
%
% for those people who are interested.
%
%<*driver|package>
\RequirePackage{expl3}
%</driver|package>
%<*driver>
\documentclass[full]{l3doc}
\usepackage{amsmath}
\begin{document}
  \DocInput{\jobname.dtx}
\end{document}
%</driver>
% \fi
%
% \title{^^A
%   The \pkg{l3fp-interchange} package\\ Floating point interchange^^A
% }
%
% \author{^^A
%  The \LaTeX{} Project\thanks
%    {^^A
%      E-mail:
%        \href{mailto:latex-team@latex-project.org}
%          {latex-team@latex-project.org}^^A
%    }^^A
% }
%
% \date{Released 2018-04-30}
%
% \maketitle
%
% \begin{documentation}
%
% \end{documentation}
%
% \begin{implementation}
%
% \section{\pkg{l3fp-interchange} implementation}
%
%    \begin{macrocode}
%<*package>
%    \end{macrocode}
%
%    \begin{macrocode}
%<@@=fp>
%    \end{macrocode}
%
% This module provides functions to convert from \pkg{l3fp}'s encoding
% of floating-points to the \textsc{ieee} standard's \texttt{decimal64}
% encoding.
%
% The $64$~bits are manipulated here as $8$~bytes, namely character
% tokens with character code in $[0,255]$.
%
% Currently the implementation is incomplete.  There is no effort to be
% efficient: instead, the amount of code is minimized.
%
% There are actually two variants of that encoding, for now we focus on
% the ``binary encoding for the significand''.  Below, we denote by
% $b_0, b_1, \ldots{}, b_{63}$ the $64$ bits.  The (binary-significand)
% \texttt{decimal64} format consists of a sign bit $S=b_0\in\{0,1\}$, an
% exponent, and a significand, but the precise size of the exponent and
% significand differ for different cases.
% \begin{itemize}
% \item If $b_1b_2b_3b_4b_5=11111$ then we get a quiet ($b_6=0$) or
%   signalling ($b_6=1$) \texttt{NaN}, the bits $b_7\cdots b_{13}$ are
%   ignored, and the payload is given by the bits $b_{14}\cdots b_{63}$,
%   an integer in $[0,2^{50})$ where values greater or equal to
%   $10^{15}$ are treated as zero payload;
% \item If $b_1b_2b_3b_4b_5=11110$ then we get $(-1)^S\infty$ and other
%   bits are ignored.
% \item Otherwise, if $b_1b_2=11$ (but $b_3b_4\neq 11$) then the
%   (biased) exponent is $E=b_3\cdots b_{12}$ and the significand is
%   $100b_{13}\cdots b_{63}$, unless this exceeds $10^{16}$ in which
%   case the significand is treated as zero.
% \item Finally, if $b_1b_2\neq 11$ then the (biased) exponent is
%   $E=b_1\cdots b_{10}$ and the significand is
%   $b_{11}b_{12}b_{13}\cdots b_{63}$.
% \end{itemize}
% The encoding is canonical if all bits that are ignored in the
% explanation above are set to zero.  When converting to
% \texttt{decimal64} the encoding must be canonical.  Namely, since all
% our \texttt{NaN} have zero payload and no sign,
% \begin{itemize}
% \item A \texttt{NaN} can be encoded as $011111$ followed by $0$
%   (quiet) or $1$ (signalling) then $57$ zero bits.
% \item An infinity can be encoded as $0$ (positive) or $1$ (negative)
%   followed by $1111$ then $59$ zero bits.
% \item A number whose significand is $c\in[2^{53},10^{16})$ can be
%   encoded as its sign bit, then $1$ and $1$, then the biased exponent
%   $E$ as a $10$-bit integer in $[0,767]$, then $51$-bits denoting
%   $c-2^{53}$ (with leading zeros).
% \item Finally, a number whose significand is $c\in[0,2^{53})$ can be
%   encoded directly as its sign bit, its $10$-bit exponent, and its
%   $53$-bit significand.
% \end{itemize}
%
% \subsection{Inputting the interchange format}
%
% \begin{macro}[EXP]{\@@_from_bytes_invalid:}
%   ^^A todo: invalid operation
%    \begin{macrocode}
\cs_new:Npn \@@_from_bits_invalid: { \c_nan_fp }
%    \end{macrocode}
% \end{macro}
%
% \begin{macro}[EXP]{\fp_from_bytes:n}
%   This reads a \texttt{decimal64} given as $8$ character tokens whose
%   character code is in the range $[0,255]$ (\enquote{bytes}) and
%   converts it to \pkg{l3fp} floating-points.  First test things a bit
%    \begin{macrocode}
\cs_new:Npn \fp_from_bytes:n #1
  { \exp_args:Nf \@@_from_bytes:n { \__kernel_str_to_other:n {#1} } }
\cs_new:Npn \@@_from_bytes:n #1
  {
    \int_compare:nNnTF { \str_count:n {#1} } = { 8 }
      { \@@_from_bytes:Nw #1 \s_@@_stop }
      { \@@_from_bytes_invalid: }
  }
\cs_new:Npn \@@_from_bytes:Nw #1
  {
    \int_compare:nNnTF { `#1 } < { 128 }
      { \exp_args:NNf \@@_from_bytes:Nnw + { \int_eval:n { `#1 } } }
      { \exp_args:NNf \@@_from_bytes:Nnw - { \int_eval:n { `#1 - 128 } } }
  }
\cs_new:Npn \@@_from_bytes:Nnw #1#2
  {
    \int_compare:nNnTF { \int_div_truncate:nn {#2} {8} } = {15}
      { \@@_from_bytes_special:Nnw }
      { \@@_from_bytes_finite:Nnw }
    #1 {#2}
  }
\cs_new:Npn \@@_from_bytes_special:Nnw #1#2#3 \s_@@_stop
  {
    \int_case:nnF { \int_div_truncate:nn {#2} {2} }
      {
        {63} { \c_nan_fp } % ^^A todo signalling
        {62} { \c_nan_fp } % ^^A todo quiet
      }
      { \token_if_eq_charcode:NNTF #1 + { \c_inf_fp } { \c_minus_inf_fp } }
  }
\cs_new:Npn \@@_from_bytes_finite:Nnw #1#2
  {
    \int_compare:nNnTF { \int_div_truncate:nn {#2} {32} } = {3}
      { \@@_from_bytes_finite:NNnNNNNw 1 }
      { \@@_from_bytes_finite:NNnNNNNw 4 }
    #1 {#2}
  }
\cs_new:Npn \@@_from_bytes_finite:NNnNNNNw #1#2#3#4#5#6#7#8 \s_@@_stop
  {
    \@@_parse:n
      {
        #2
        \exp_args:Nf \@@_from_bytes_check:n
          {
            \fp_to_int:n
              {
                \token_if_eq_charcode:NNT #1 1 { 2^53 + }
                \int_mod:nn { `#4 } { 8 * #1 } * 2^48
                + \@@_from_bytes_aux:NNN #5#6#7 * 2^24
                + \@@_from_bytes_aux:NNN #8
              }
          }
        e
        \int_eval:n
          {
            \int_mod:nn {#3} { 32 * #1 } * 32 / #1
             + \int_div_truncate:nn { `#4 } { 8 * #1 }
             - \c_@@_bias_int - 16
          }
      }
  }
\cs_new:Npn \@@_from_bytes_check:n #1
  { \int_compare:nNnTF { \str_count:n {#1} } > {16} { 0 } {#1} }
\cs_new:Npn \@@_from_bytes_aux:NNN #1#2#3
  { \int_eval:n { (`#1 * 256 + `#2) * 256 + `#3 } }
\int_const:Nn \c_@@_bias_int { 398 - 16 }
%    \end{macrocode}
% \end{macro}
%
% \subsection{Outputting the interchange format}
%
% \begin{macro}{\fp_to_bytes:n}
%   ^^A oops, this generates bits instead of bytes!
%    \begin{macrocode}
\cs_new:Npn \fp_to_bytes:n #1
  { \exp_last_unbraced:Nf \@@_to_bytes:w { \@@_parse:n {#1} } }
\group_begin:
  \char_set_catcode_other:N \^^00
  \cs_new:Npn \@@_to_bytes:w \s_@@ \@@_chk:w #1#2#3 ;
    {
      \token_if_eq_charcode:NNTF #1 1
        { \@@_to_bytes_normal:Nnnnnn #2 #3 }
        {
          \str_case:nn { #1#2 }
            {
              { 0 0 } { \token_to_str:N ^^00 }
              { 0 2 } { \token_to_str:N ^^80 }
              { 2 0 } { \token_to_str:N ^^78 }
              { 2 2 } { \token_to_str:N ^^f8 }
              { 3 1 } { \token_to_str:N ^^7c }
            }
          ^^00 ^^00 ^^00 ^^00 ^^00 ^^00 ^^00
        }
    }
\group_end:
\cs_new:Npn \@@_to_bytes_normal:Nnnnnn #1#2#3#4#5#6
  {
    \fp_compare:nTF { #3 #4 #5 #6 < 2^53 }
      {
        \@@_to_bytes_normal:nnnnNNn { #3 #4 } { #5 #6 }
          { } {6} \c_false_bool
      }
      {
        \int_compare:nNnTF { #5 #6 } < { 54740992 }
          {
            \@@_to_bytes_normal:ffffNNn
              { \int_eval:n { #3 #4 - 90071992 - 1 } }
              { \int_eval:n { #5 #6 + 45259008 } }
          }
          {
            \@@_to_bytes_normal:ffffNNn
              { \int_eval:n { #3 #4 - 90071992 } }
              { \int_eval:n { #5 #6 - 54740992 } }
          }
        { } {6} \c_true_bool
      }
    #1 {#2}
  }
\cs_new:Npn \@@_to_bytes_normal:nnnnNNn #1#2#3#4#5#6#7
  {
    \int_compare:nNnTF {#4} = { 0 }
      {
        \exp_args:Nf \@@_to_bytes_end:n
          {
            \int_eval:n
              {
                \token_if_eq_charcode:NNT 2 #6 { 32768 + }
                \bool_if:NT #5 { 24576 + }
                ( #7 + \c_@@_bias_int ) * \bool_if:NTF #5 {8} {32}
                + #2
              }
          }
        #3
      }
      {
        \@@_to_bytes_normal:ffffNNn
          { \int_div_truncate:nn {#1} {256} }
          {
            \int_mod:nn {#1} {256} * 390625
            + \int_div_truncate:nn {#2} {256}
          }
          { \char_generate:nn { \int_mod:nn {#2} {256} } {12} #3 }
          { \int_eval:n { #4 - 1 } }
        #5 #6 {#7}
      }
  }
\cs_generate_variant:Nn \@@_to_bytes_normal:nnnnNNn { ffff }
\cs_new:Npn \@@_to_bytes_end:n #1
  {
    \exp_last_unbraced:Nf \exp_last_unbraced:Nf
      { \char_generate:nn { \int_div_truncate:nn {#1} {256} } {12} }
      { \char_generate:nn { \int_mod:nn {#1} {256} } {12} }
  }
%    \end{macrocode}
% \end{macro}
%
%    \begin{macrocode}
%</package>
%    \end{macrocode}
%
% \end{implementation}
%
% \PrintIndex
