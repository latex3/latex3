% \iffalse
%
%% File l3fp-symbolic.dtx (C) Copyright 2012 The LaTeX3 Project
%%
%% It may be distributed and/or modified under the conditions of the
%% LaTeX Project Public License (LPPL), either version 1.3c of this
%% license or (at your option) any later version.  The latest version
%% of this license is in the file
%%
%%    http://www.latex-project.org/lppl.txt
%%
%% This file is part of the "l3trial bundle" (The Work in LPPL)
%% and all files in that bundle must be distributed together.
%%
%% The released version of this bundle is available from CTAN.
%%
%% -----------------------------------------------------------------------
%%
%% The development version of the bundle can be found at
%%
%%    http://www.latex-project.org/svnroot/experimental/trunk/
%%
%% for those people who are interested.
%%
%%%%%%%%%%%
%% NOTE: %%
%%%%%%%%%%%
%%
%%   Snapshots taken from the repository represent work in progress and may
%%   not work or may contain conflicting material!  We therefore ask
%%   people _not_ to put them into distributions, archives, etc. without
%%   prior consultation with the LaTeX Project Team.
%%
%% -----------------------------------------------------------------------
%%
%
%<*driver|package>
\RequirePackage{expl3}
\GetIdInfo$Id$
  {L3 Experimental manipulations of symbolic expressions}
%</driver|package>
%<*driver>
\documentclass[full]{l3doc}
\usepackage{amsmath}
\begin{document}
  \DocInput{\jobname.dtx}
\end{document}
%</driver>
% \fi
%
% \title{^^A
%   The \pkg{l3fp-symbolic} package\\ Add ons to \pkg{l3kernel}^^A
%   \thanks{This file describes v\ExplFileVersion,
%      last revised \ExplFileDate.}^^A
% }
%
% \author{^^A
%  The \LaTeX3 Project\thanks
%    {^^A
%      E-mail:
%        \href{mailto:latex-team@latex-project.org}
%          {latex-team@latex-project.org}^^A
%    }^^A
% }
%
% \date{Released \ExplFileDate}
%
% \maketitle
%
% \begin{documentation}
%
% \section{\pkg{l3fp-symbolic} documentation}
%
% At the moment, this package redefines a few \pkg{l3fp} internals to
% introduce the ability to use unbound variables in floating point
% expressions.
% \begin{verbatim}
%   \fp_set:Nn \l_test_fp { 1 + A * sin(B) }
%   \fp_show:N \l_test_fp
% \end{verbatim}
% gives |(1+(A*sin(B)))|.  One can assign a value to a variable with
% \cs{fp_var_set:nn}.  For example,
% \begin{verbatim}
%   \fp_var_set:nn { B } { pi / 2 }
%   \fp_show:N \l_test_fp
%   \fp_show:n { \l_test_fp }
% \end{verbatim}
% shows |(1+(A*sin(B)))|, since |\l_test_fp| has not changed, then
% |(1+(A*1))| because \cs{fp_show:n} forces the re-evaluation of
% \cs{l_test_fp}, and \pkg{l3fp} finds out that the sine function can be
% evaluated now that its argument has a definite value.  Note that |A*1|
% is not simplified to |A|: \cs{fp_eval:n} only attempts to evaluate
% computations whose operands are all known.
%
% \end{documentation}
%
% \begin{implementation}
%
% \section{\pkg{l3fp-symbolic} implementation}
%
%    \begin{macrocode}
%<*initex|package>
%    \end{macrocode}
%
%    \begin{macrocode}
%<@@=fp>
%    \end{macrocode}
%
%    \begin{macrocode}
%<*package>
\ProvidesExplPackage
  {\ExplFileName}{\ExplFileDate}{\ExplFileVersion}{\ExplFileDescription}
%</package>
%    \end{macrocode}
%
% \subsection{Building blocks for expressions}
%
% ^^A todo: try to use a variable called "chk".
%
% Every symbolic expression has the form
% \begin{quote}
%   \cs{s_@@_symbolic} \cs{@@_symbolic_chk:w}
%   \cs{@@_symbol_\meta{op}:w} \Arg{arguments} |;|
% \end{quote}
% where the \meta{op} is an operation (binary |+|, |*|, \ldots{}, or
% function such as |sin| or |ln|), and \meta{arguments} is an array of
% floating points or symbolic expressions, with the number of items
% expected for the given operation.  Variables are stored as operations
% with no arguments.
%
% Symbolic expressions are stored in a prefix form.  When encountering a
% symbolic expression in a floating point computation, we attempt to
% evaluate the operands as much as possible, and if that yields floating
% point numbers rather than expressions, we apply the operator which
% follows (if the function is known).
%
% For instance, the expression "a + b * sin(c)" is stored as
% \begin{verbatim}
% \s__fp_symbolic \__fp_symbolic_chk:w \__fp_symbol_+:w
%   {
%     \s__fp_symbolic \__fp_symbolic_chk:w \__fp_symbol_a:w { } ;
%     \s__fp_symbolic \__fp_symbolic_chk:w \__fp_symbol_*:w
%       {
%         \s__fp_symbolic \__fp_symbolic_chk:w \__fp_symbol_b:w { } ;
%         \s__fp_symbolic \__fp_symbolic_chk:w \__fp_symbol_sin:w
%           { \s__fp_symbolic \__fp_symbolic_chk:w \__fp_symbol_c:w { } ; } ;
%       } ;
%   } ;
% \end{verbatim}
%
% \begin{variable}[int]{\s_@@_symbolic}
%   Scan mark indicating the start of a symbolic expression.
%    \begin{macrocode}
\__scan_new:N \s_@@_symbolic
%    \end{macrocode}
% \end{variable}
%
% \begin{macro}[int]{\@@_symbolic_chk:w}
%   Analog of \cs{@@_chk:w} for symbolic expressions.
%    \begin{macrocode}
\cs_new_protected:Npn \@@_symbolic_chk:w #1;
  {
    \__msg_kernel_error:nnx { kernel } { misused-fp }
      { \@@_to_tl:w \s_@@_symbolic \@@_symbolic_chk:w #1; }
  }
%    \end{macrocode}
% \end{macro}
%
% \begin{macro}[aux, EXP]
%   {\@@_if_has_symbolic:nTF, \@@_if_has_symbolic_aux:w}
%   Tests if the string representation of |#1| contains
%   |s__fp_symbolic|.  This test should be precise enough to determine
%   if a given an array contains a symbolic expression or only consists
%   in floating points.  See \cs{@@_exp_after_symbolic_f:nw} for an
%   application.
%    \begin{macrocode}
\cs_set:Npn \@@_tmp:w #1
  {
    \cs_new:Npn \@@_if_has_symbolic:nTF ##1
      {
        \exp_after:wN \@@_if_has_symbolic_aux:w
          \tl_to_str:n {##1} \q_mark \use_i:nn
          #1                 \q_mark \use_ii:nn
        \q_stop
      }
    \cs_new:Npn \@@_if_has_symbolic_aux:w
        ##1 #1 ##2 \q_mark ##3##4 \q_stop { ##3 }
  }
\exp_args:No \@@_tmp:w { \tl_to_str:n { s_@@_symbolic ~ } }
%    \end{macrocode}
% \end{macro}
%
% \begin{macro}[int, EXP]{\@@_exp_after_symbolic_f:nw}
% \begin{macro}[aux, EXP]{\@@_exp_after_symbolic_aux:Nn}
%   This function does two things: trigger an \texttt{f}-expansion of
%   the argument |#1| after the following symbolic expression, and
%   evaluate all pieces of the expression that can be evaluated.
%    \begin{macrocode}
\cs_new:Npn \@@_exp_after_symbolic_f:nw
    #1 \s_@@_symbolic \@@_symbolic_chk:w #2 #3 ;
  {
    \exp_after:wN \@@_exp_after_symbolic_aux:Nn
    \exp_after:wN #2
    \exp_after:wN
      {
        \tex_romannumeral:D -`0
        \@@_exp_after_array_f:w #3 \s_@@_stop
        \exp_after:wN
      }
      \exp_after:wN ;
    \tex_romannumeral:D -`0 #1
  }
\cs_new:Npn \@@_exp_after_symbolic_aux:Nn #1#2
  {
    \@@_if_has_symbolic:nTF {#2}
      { \s_@@_symbolic \@@_symbolic_chk:w #1 {#2} }
      {
        \cs_if_exist:NTF #1
          { #1 #2 @ }
          { \s_@@_symbolic \@@_symbolic_chk:w #1 {#2} }
      }
  }
%    \end{macrocode}
% \end{macro}
% \end{macro}
%
% ^^A todo: test { \fp_gset:Nn \foo { abc } } ...
%
% \subsection{Built-in operators}
%
% \begin{@makeother}{^} %^^A Hack!
% \begin{@makeother}{|} %^^A Hack!
% \begin{@makeother}{&} %^^A Hack!
% \begin{macro}[EXP, aux]
%   {
%     \@@_symbolic_+_symbolic_o:ww,
%     \@@_symbolic_+_o:ww,
%     \@@_+_symbolic_o:ww,
%     \@@_symbol_+:w,
%     \@@_symbolic_-_symbolic_o:ww,
%     \@@_symbolic_-_o:ww,
%     \@@_-_symbolic_o:ww,
%     \@@_symbol_-:w,
%     \@@_symbolic_*_symbolic_o:ww,
%     \@@_symbolic_*_o:ww,
%     \@@_*_symbolic_o:ww,
%     \@@_symbol_*:w,
%     \@@_symbolic_/_symbolic_o:ww,
%     \@@_symbolic_/_o:ww,
%     \@@_/_symbolic_o:ww,
%     \@@_symbol_/:w,
%     \@@_symbolic_^_symbolic_o:ww,
%     \@@_symbolic_^_o:ww,
%     \@@_^_symbolic_o:ww,
%     \@@_symbol_^:w,
%     \@@_symbolic_|_symbolic_o:ww,
%     \@@_symbolic_|_o:ww,
%     \@@_|_symbolic_o:ww,
%     \@@_symbol_|:w,
%     \@@_symbolic_&_symbolic_o:ww,
%     \@@_symbolic_&_o:ww,
%     \@@_&_symbolic_o:ww,
%     \@@_symbol_&:w,
%   }
%    \begin{macrocode}
\cs_set:Npn \@@_tmp:w #1#2
  {
    \cs_new:cpn { @@_symbolic_#2_symbolic_o:ww } ##1; ##2;
      {
        \@@_exp_after_symbolic_f:nw { }
          \s_@@_symbolic \@@_symbolic_chk:w #1 { ##1; ##2; } ;
        \exp_after:wN \exp_stop_f: % ?
      }
    \cs_new_eq:cc
      { @@_symbolic_#2         _o:ww }
      { @@_symbolic_#2_symbolic_o:ww }
    \cs_new_eq:cc
      { @@         _#2_symbolic_o:ww }
      { @@_symbolic_#2_symbolic_o:ww }
    \cs_new_protected:Npx #1 ##1; ##2; @ ;
      {
        \exp_not:c { @@_#2_o:ww } ##1; ##2;
      }
  }
\exp_args:Nc \@@_tmp:w { @@_symbol_+:w } +
\exp_args:Nc \@@_tmp:w { @@_symbol_-:w } -
\exp_args:Nc \@@_tmp:w { @@_symbol_*:w } *
\exp_args:Nc \@@_tmp:w { @@_symbol_/:w } /
\exp_args:Nc \@@_tmp:w { @@_symbol_^:w } ^
\exp_args:Nc \@@_tmp:w { @@_symbol_|:w } |
\exp_args:Nc \@@_tmp:w { @@_symbol_&:w } &
%    \end{macrocode}
% \end{macro}
% \end{@makeother}
% \end{@makeother}
% \end{@makeother}
%
% \subsection{Built-in functions}
%
% \begin{macro}[EXP, aux]
%   {
%     \@@_abs_symbolic_o:w  ,
%     \@@_unary_symbol_abs:w,
%     \@@_cos_symbolic_o:w  ,
%     \@@_unary_symbol_cos:w,
%     \@@_cot_symbolic_o:w  ,
%     \@@_unary_symbol_cot:w,
%     \@@_csc_symbolic_o:w  ,
%     \@@_unary_symbol_csc:w,
%     \@@_exp_symbolic_o:w  ,
%     \@@_unary_symbol_exp:w,
%     \@@_ln_symbolic_o:w   ,
%     \@@_unary_symbol_ln:w ,
%     \@@_sec_symbolic_o:w  ,
%     \@@_unary_symbol_sec:w,
%     \@@_sin_symbolic_o:w  ,
%     \@@_unary_symbol_sin:w,
%     \@@_tan_symbolic_o:w  ,
%     \@@_unary_symbol_tan:w,
%     \@@_!_symbolic_o:w  ,
%     \@@_unary_symbol_!:w,
%     \@@_-_symbolic_o:w  ,
%     \@@_unary_symbol_-:w,
%   }
%    \begin{macrocode}
\cs_set:Npn \@@_tmp:w #1#2
  {
    \cs_new:cpn { @@_#2_symbolic_o:w } ##1;
      {
        \@@_exp_after_symbolic_f:nw { \exp_after:wN \exp_stop_f: } % ?
          \s_@@_symbolic \@@_symbolic_chk:w #1 { ##1; } ;
      }
    \cs_new_protected:Npx #1 ##1 @ ;
      {
        \exp_not:c { @@_#2_o:w } ##1
      }
  }
\tl_map_inline:nn
  { {abs} {cos} {cot} {csc} {exp} {ln} {sec} {sin} {tan} {!} {-} }
  { \exp_args:Nc \@@_tmp:w { @@_unary_symbol_#1:w } {#1} }
%    \end{macrocode}
% \end{macro}
%
% \subsection{Turning unknown fp words into new variables}
%
% \begin{macro}[int, EXP]{\@@_parse_operand_other_word_aux:Nw}
% \begin{macro}[aux, EXP]{\@@_parse_operand_other_word_auxii:NnN}
%    \begin{macrocode}
\cs_gset:Npn \@@_parse_operand_other_word_aux:Nw #1 #2;
  {
    \exp_after:wN \@@_parse_operand_other_word_auxii:NnN
      \cs:w @@_parse_word_#2:N \cs_end: {#2}
      #1
  }
\cs_new:Npn \@@_parse_operand_other_word_auxii:NnN #1#2
  {
    \if_meaning:w \tex_relax:D #1
      \exp_after:wN \@@_exp_after_symbolic_f:nw
      \exp_after:wN \@@_parse_infix:NN
      \exp_after:wN \s_@@_symbolic
      \exp_after:wN \@@_symbolic_chk:w
      \cs:w @@_symbol_ #2 :w \exp_after:wN \cs_end:
      \exp_after:wN {
      \exp_after:wN }
      \exp_after:wN ;
    \else:
      \exp_after:wN #1
    \fi:
  }
%    \end{macrocode}
% \end{macro}
% \end{macro}
%
% \subsection{Giving values to variables}
%
% \begin{macro}{\fp_var_set:nn}
%    \begin{macrocode}
\cs_new:Npn \fp_var_set:nn #1#2
  {
    \cs_set_protected:cpn { @@_symbol_#1:w } @ ;
      {
        \@@_parse:n {#2}
      }
  }
%    \end{macrocode}
% \end{macro}
%
% \subsection{Conversion to token list}
%
% \begin{macro}[EXP, int]{\@@_to_tl:w}
%    \begin{macrocode}
\cs_new_eq:NN \@@_to_tl_aux:w \@@_to_tl:w
\cs_gset:Npn \@@_to_tl:w #1
  { \cs:w @@ \@@_type_from_scan:N #1 _to_tl_aux:w \cs_end: #1 }
\cs_new:Npn \@@_symbolic_to_tl_aux:w
    \s_@@_symbolic \@@_symbolic_chk:w #1 #2;
  {
    \exp_after:wN \@@_symbolic_to_tl_auxii:wNwn
      \token_to_str:N #1 ;
      #2 \s_@@_stop
  }
\use:x
  {
    \cs_new:Npn \exp_not:N \@@_symbolic_to_tl_auxii:wNwn
        ##1 \tl_to_str:n { symbol_ } ##2##3 \tl_to_str:n { :w } ;
  }
      {
        \if_int_compare:w
            \__int_eval:w \tex_uccode:D `#2 / 26 = \c_three
          \exp_after:wN \@@_symbolic_to_tl_aux_letters:nw
        \else:
          \exp_after:wN \@@_symbolic_to_tl_aux_binary:nw
        \fi:
        { #2 #3 }
      }
\cs_generate_variant:Nn \use:nn { nf }
\cs_new:Npn \@@_symbolic_to_tl_aux_letters:nw #1#2 \s_@@_stop
  {
    \@@_expand:n
      { #1 \tl_if_empty:nF {#2} { ( \@@_array_to_clist:n {#2} ) } }
  }
\cs_new:Npn \@@_symbolic_to_tl_aux_binary:nw #1#2; #3 \s_@@_stop
  {
    \@@_expand:n
      { ( \@@_to_tl:w #2; #1 \@@_array_to_clist:n {#3} ) }
  }
%    \end{macrocode}
% \end{macro}
%
% \subsection{Not implemented}
%
% The following functions are not implemented: min, max.
%
%    \begin{macrocode}
%</initex|package>
%    \end{macrocode}
%
% \end{implementation}
%
% \PrintIndex
