% \iffalse
%
%% File l3trace.dtx (C) Copyright 2012 The LaTeX3 Project
%%
%% It may be distributed and/or modified under the conditions of the
%% LaTeX Project Public License (LPPL), either version 1.3c of this
%% license or (at your option) any later version.  The latest version
%% of this license is in the file
%%
%%    http://www.latex-project.org/lppl.txt
%%
%% This file is part of the "l3trial bundle" (The Work in LPPL)
%% and all files in that bundle must be distributed together.
%%
%% The released version of this bundle is available from CTAN.
%%
%% -----------------------------------------------------------------------
%%
%% The development version of the bundle can be found at
%%
%%    http://www.latex-project.org/svnroot/experimental/trunk/
%%
%% for those people who are interested.
%%
%%%%%%%%%%%
%% NOTE: %%
%%%%%%%%%%%
%%
%%   Snapshots taken from the repository represent work in progress and may
%%   not work or may contain conflicting material!  We therefore ask
%%   people _not_ to put them into distributions, archives, etc. without
%%   prior consultation with the LaTeX Project Team.
%%
%% -----------------------------------------------------------------------
%%
%
%<*driver|package>
\RequirePackage{expl3}
\GetIdInfo$Id: l3trace.dtx 3177 2012-01-11 02:11:43Z bruno $
  {L3 Experimental tracing of higher-level modules}
%</driver|package>
%<*driver>
\documentclass[full]{l3doc}
\usepackage{amsmath}
\begin{document}
  \DocInput{\jobname.dtx}
\end{document}
%</driver>
% \fi
%
% \title{^^A
%   The \pkg{l3trace} package\\ Tracing higher-level modules^^A
%   \thanks{This file describes v\ExplFileVersion,
%      last revised \ExplFileDate.}^^A
% }
%
% \author{^^A
%  The \LaTeX3 Project\thanks
%    {^^A
%      E-mail:
%        \href{mailto:latex-team@latex-project.org}
%          {latex-team@latex-project.org}^^A
%    }^^A
% }
%
% \date{Released \ExplFileDate}
%
% \maketitle
%
% \begin{documentation}
%
% \section{\pkg{l3trace} documentation}
%
% \end{documentation}
%
% \begin{implementation}
%
% \section{\pkg{l3trace} implementation}
%
%    \begin{macrocode}
%<*initex|package>
%    \end{macrocode}
%
%    \begin{macrocode}
%<*package>
\ProvidesExplPackage
  {\ExplFileName}{\ExplFileDate}{\ExplFileVersion}{\ExplFileDescription}
%</package>
%    \end{macrocode}
%
% \subsection{Helpers which should be moved}
%
% \begin{macro}[rEXP]{\seq_to_str:N, \seq_to_str:c}
% \begin{macro}[aux, rEXP]{\seq_to_str_aux:n, \seq_to_str_end:w}
%    \begin{macrocode}
\cs_new:Npn \seq_to_str:N #1
  {
    \seq_if_empty:NF #1
      {
        \exp_last_unbraced:Nf \seq_to_str_end:w
          { \seq_map_function:NN #1 \seq_to_str_aux:n }
      }
  }
\cs_new:Npn \seq_to_str_aux:n #1 { , ~ \tl_to_str:n {#1} }
\cs_new:Npn \seq_to_str_end:w , ~ { }
\cs_generate_variant:Nn \seq_to_str:N { c }
%    \end{macrocode}
% \end{macro}
% \end{macro}
%
% \subsection{Variables and keys}
%
% \begin{variable}{\g_trace_nesting_seq}
%   The items of this sequence are module names, one for each
%   \cs{trace_push:nnn} currently active.
%    \begin{macrocode}
\seq_new:N \g_trace_nesting_seq
%    \end{macrocode}
% \end{variable}
%
% \begin{variable}{\g_trace_default_int}
%   This is the level of tracing used when a key is given with no
%   associated value: $1$ within \cs{trace_on:n}, and $0$ within
%   \cs{trace_off:n}.
%    \begin{macrocode}
\int_new:N \g_trace_default_int
%    \end{macrocode}
% \end{variable}
%
% \subsection{Declaring modules to trace}
%
% \begin{macro}{\trace_declare_module:n}
%   This declares a module.
%    \begin{macrocode}
\cs_new_protected:Npn \trace_declare_module:n #1
  {
    \int_new:c  { g_trace_ \tl_to_str:n {#1} _level_int }
    \tl_new:c   { g_trace_ \tl_to_str:n {#1} _symbol_tl }
    \tl_gset:cn { g_trace_ \tl_to_str:n {#1} _symbol_tl } { - }
  }
%    \end{macrocode}
% \end{macro}
%
% \begin{macro}{\trace_chk_module_declared:nT, \trace_chk_module_declared:VT}
%   At several places in the code we make sure that a given module is
%   declared to have tracing information. This catches typos, for
%   instance.
%    \begin{macrocode}
\prg_new_protected_conditional:Npnn \trace_chk_module_declared:n #1 { T }
  {
    \int_if_exist:cTF { g_trace_ \tl_to_str:n {#1} _level_int }
      { \prg_return_true: }
      {
        \__msg_kernel_error:nnx { trace } { module-not-declared }
          { \tl_to_str:n {#1} }
        \prg_return_false:
      }
  }
\cs_generate_variant:Nn \trace_chk_module_declared:nT { V }
\__msg_kernel_new:nnnn { trace } { module-not-declared }
  { The~module~#1~does~not~provide~tracing~information. }
  {
    Either~there~is~no~module~called~#1,~or~it~was~not~declared~
    to~the~trace~module.~Modules~are~declared~using~
    \iow_char:N\\trace_declare_module:n .
  }
%    \end{macrocode}
% \end{macro}
%
% \subsection{Code-level functions}
%
% \begin{macro}{\trace_push:nnx, \trace_push:nnn}
%   The arguments are: the \meta{module} name, the \meta{level} of
%   tracing (positive), and the function to be pushed. We only trace if
%   the \meta{module} is declared and if the \meta{level} is at most the
%   \meta{module}'s current tracing level.
%    \begin{macrocode}
\cs_new_protected:Npn \trace_push:nnx #1#2#3
  {
    \trace_chk_module_declared:nT {#1}
      {
        \int_compare:nNnF {#2} >
          { \int_use:c { g_trace_ \tl_to_str:n {#1} _level_int } }
          {
            \trace_display:x { entering~#3 }
            \seq_gput_right:Nx \g_trace_nesting_seq { \tl_to_str:n {#1} }
          }
      }
  }
\cs_new_protected:Npn \trace_push:nnn #1#2#3
  { \trace_push:nnx {#1} {#2} { \exp_not:n {#3} } }
%    \end{macrocode}
% \end{macro}
%
% \begin{macro}{\trace_pop:nnx, \trace_pop:nnn}
%   Under the same conditions as \cs{trace_push:nnn}, we leave the
%   function. Maybe we shouldn't record this in the case where the
%   \meta{level} is equal to the module's current tracing level:
%   presumably, at this level of tracing, there aren't too many
%   functions nested within that particular function, and indentation
%   (coming from the symbols) should be enough to see that it was
%   closed.
%    \begin{macrocode}
\tl_new:N \l_trace_internal_a_tl
\tl_new:N \l_trace_internal_b_tl
\cs_new_protected:Npn \trace_pop:nnx #1#2#3
  {
    \trace_chk_module_declared:nT {#1}
      {
        \int_compare:nNnF {#2} >
          { \int_use:c { g_trace_ \tl_to_str:n {#1} _level_int } }
          {
            \seq_gpop_right:NN \g_trace_nesting_seq \l_trace_internal_a_tl
            \tl_set:Nx \l_trace_internal_b_tl { \tl_to_str:n {#1} }
            \tl_if_eq:NNF \l_trace_internal_a_tl \l_trace_internal_b_tl
              {
                \__msg_kernel_error:nn { trace } { mismatched-pop }
                  { \tl_to_str:N \l_trace_internal_a_tl }
                  { \tl_to_str:N \l_trace_internal_b_tl }
              }
            \trace_display:x { leaving~#3 }
          }
      }
  }
\cs_new_protected:Npn \trace_pop:nnn #1#2#3
  { \trace_pop:nnx {#1} {#2} { \exp_not:n {#3} } }
 % \__msg_kernel_new:nnn { trace } { mismatched-pop } { }
%    \end{macrocode}
% \end{macro}
%
% \begin{macro}{\trace:nnx, \trace:nnn}
%   Similar to \cs{trace_push:nnx} and \cs{trace_pop:nnx} for the tests,
%   but has no side-effect.
%    \begin{macrocode}
\cs_new_protected:Npn \trace:nnx #1#2#3
  {
    \trace_chk_module_declared:nT {#1}
      {
        \int_compare:nNnF {#2} >
          { \int_use:c { g_trace_ \tl_to_str:n {#1} _level_int } }
          { \trace_display:x { #3 } }
      }
  }
\cs_new_protected:Npn \trace:nnn #1#2#3
  { \trace:nnx {#1} {#2} { \exp_not:n {#3} } }
%    \end{macrocode}
% \end{macro}
%
% \begin{macro}{\trace_display:x}
%    This macro unconditionally displays some tracing information on the
%    screen.
%    In case we want to show the content of some sequence we locally redefine
%    their data structures (may have to be done for other types as well).
%    \begin{macrocode}
\cs_new_nopar:Npn \trace_display:x #1
  {
    \__seq_push_item_def:n { ##1 \space }
    \typeout
      {
        :~
        \seq_map_function:NN \g_trace_nesting_seq \trace_aux_symbol:n
        \c_space_tl
        #1
      }
    \__seq_pop_item_def:
  }
\cs_new:Npn \trace_aux_symbol:n #1
  { \tl_to_str:c { g_trace_ \tl_to_str:n {#1} _symbol_tl } }
%    \end{macrocode}
%  \end{macro}
%
%  \begin{macro}{\traceonly}
%    Tracing only some (at the moment exactly one) function and its subcomponents.
%    \begin{macrocode}
 % \cs_new_nopar:Npn \traceonly #1
 %   {
 %     \notrace
 %     \tl_set:Nn \l_trace_only_tl {#1}
%    \end{macrocode}
%    |\trace_push:n| is redefined to compare the current function on the stack
%    (its argument) with the function to be traced. If the desired function is
%    entered the current |\g_trace_prefix_tl| is stored away in
%    |\g_trace_this_prefix_tl| and tracing is started.\footnote{fix this is
%    rubbish ... the idea should be to allow for recursive entering the same
%    function  ... fix some other time :-) }
%    \begin{macrocode}
     % \cs_gset_nopar:Npn \trace_push:n ##1
     %   {
     %     \tl_set:Nn \l_trace_this_only_tl {##1}
     %     \token_if_eq_meaning:NNT
     %       \l_trace_this_only_tl \l_trace_only_tl
     %       {
     %         \tl_gset_eq:NN \g_trace_this_prefix_tl \g_trace_prefix_tl
     %         \tracefloats
     %         \trace:n {entering~##1}
     %       }
     %     \tl_gset:Nn \g_trace_prefix_tl {-\g_trace_prefix_tl}
     %   }
%    \end{macrocode}
%
%    \begin{macrocode}
   %   \cs_set_nopar:Npn \trace_pop:n ##1
   %     {
   %       \trace:n {leaving~##1}
   %       \tl_gset:Nx \g_trace_prefix_tl { \tl_tail:N \g_trace_prefix_tl }
   %       \token_if_eq_meaning:NNT
   %         \g_trace_this_prefix_tl \g_trace_prefix_tl
   %         {
   %           \tl_gset_eq:NN \g_trace_this_prefix_tl \c_empty_tl
   %           \notrace
   %         }
   %     }
   % }
%    \end{macrocode}
% \end{macro}
%
%
% \subsection{User functions and keys}
%
% \begin{macro}{\trace_on:n, \trace_off:n}
%   The two functions only differ by the default value when a key is
%   given without value.
%    \begin{macrocode}
\cs_new_protected:Npn \trace_on:n #1
  {
    \int_gset_eq:NN \g_trace_default_int \c_one
    \keys_set:nn { trace } { #1 }
  }
\cs_new_protected:Npn \trace_off:n #1
  {
    \int_gset_eq:NN \g_trace_default_int \c_zero
    \keys_set:nn { trace } { #1 }
  }
%    \end{macrocode}
% \end{macro}
%
% \begin{macro}{\trace_symbols:n}
%   This function lets us change what symbol is used for nestings coming
%   from different modules.
%    \begin{macrocode}
\cs_new_protected:Npn \trace_symbols:n #1
  { \keys_set:nn { trace-symbols } {#1} }
%    \end{macrocode}
% \end{macro}
%
% We introduce the keys for the package.
%    \begin{macrocode}
\int_new:N \l_trace_internal_int
\keys_define:nn { trace }
  {
    unknown .code:n =
      {
        \trace_chk_module_declared:VT \l_keys_key_tl
          {
            \int_set:Nn \l_trace_internal_int
              { \tl_if_empty:nTF {#1} { \g_trace_default_int } {#1} }
            \int_gset_eq:cN
              { g_trace_ \l_keys_key_tl _level_int }
              \l_trace_internal_int
            \int_compare:nNnT \l_trace_internal_int > \c_zero
              {
                \group_begin:
                  \cs_set_eq:NN \__chk_if_free_cs:N \use_none:n
                  \cs_set_eq:NN \chk_if_free_msg:nn \use_none:nn
                  \file_if_exist:nTF { \l_keys_key_tl -trace.sty }
                    { \RequirePackage { \l_keys_key_tl -trace } }
                    { \RequirePackage { l3 \l_keys_key_tl -trace } }
                \group_end:
              }
          }
      }
  }
%    \end{macrocode}
%
%    \begin{macrocode}
\keys_define:nn { trace-symbols }
  {
    unknown .default:n = {-} ,
    unknown .code:n =
      {
        \trace_chk_module_declared:VT \l_keys_key_tl
          { \tl_gset:cn { g_trace_ \l_keys_key_tl _symbol_tl } {#1} }
      }
  }
%    \end{macrocode}
%
% \begin{macro}{\traceon, \traceoff, \tracesymbols}
%    \begin{macrocode}
\cs_new_eq:NN \traceon \trace_on:n
\cs_new_eq:NN \traceoff \trace_off:n
\cs_new_eq:NN \tracesymbols \trace_symbols:n
%    \end{macrocode}
% \end{macro}
%
% \subsection{Showing misc stuff}
%
% \begin{macro}{\trace_show_variable:nw}
%    \begin{macrocode}
\cs_new:Npn \trace_show_variable:nw #1
  {
    \iow_newline: > \c_space_tl \c_space_tl \exp_not:n {#1}
    \c_space_tl \c_space_tl => \c_space_tl \c_space_tl
  }
%    \end{macrocode}
% \end{macro}
%
% \begin{macro}{\trace_show_marks:}
%    Show the current meaning of the different marks on the page.
%    \begin{macrocode}
\cs_new_protected_nopar:Npn \trace_show_marks:
  {
    \__msg_term:nn { trace } { show-marks }
    \__msg_show_variable:x
      {
        \trace_show_variable:nw { topmark   } \exp_not:V \tex_topmark:D
        \trace_show_variable:nw { firstmark } \exp_not:V \tex_firstmark:D
        \trace_show_variable:nw { botmark   } \exp_not:V \tex_botmark:D
      }
  }
\__msg_kernel_new:nnn { trace } { show-marks }
  { The~marks~on~this~page~are }
%    \end{macrocode}
% \end{macro}
%
% \begin{macro}{\trace_show_float_lists:}
%   The \texttt{_area} function shows the contents of a single area for
%   debugging purposes.
%    \begin{macrocode}
\cs_new_protected:Npn \trace_show_float_lists:
  {
    \__msg_term:nn { trace } { show-float-lists }
    \__msg_show_variable:x
      {
        \trace_show_float_list_aux:n { free }
        \trace_show_float_list_aux:n { active }
        \trace_show_float_list_aux:n { mvl }
        \trace_show_float_list_aux:n { defer }
        \trace_show_float_list_aux:n { here }
        \clist_map_function:NN \g_xor_areas_used_clist
          \trace_show_float_area_aux:n
        \trace_show_variable:nw { currbox } \tl_to_str:N \g_xor_curr_float_box_tl
        \trace_show_variable:nw { this@float } \tl_to_str:N \g_xor_this_float_box_tl
      }
  }
\cs_new:Npn \trace_show_float_list_aux:n #1
  {
    \trace_show_variable:nw { #1 }
    \exp_args:Nc \seq_to_str:N { g_or_float_#1_seq }
  }
\cs_new:Npn \trace_show_float_area_aux:n #1
  {
    \seq_if_exist:cT { g_xor_area_#1_float_seq }
      {
        \seq_if_empty:cF { g_xor_area_#1_float_seq }
          {
            \trace_show_variable:nw { #1 }
              \seq_to_str:c { g_xor_area_#1_float_seq }
          }
      }
  }
%    \end{macrocode}
% \end{macro}
%
% \begin{macro}{\showmarks}
%    Show the current meaning of the different marks on the page.
%    \begin{macrocode}
\cs_new_eq:NN \showmarks \trace_show_marks:
%    \end{macrocode}
% \end{macro}
%
% \begin{macro}{\showfloatlists}
%    \begin{macrocode}
\cs_new_eq:NN \showfloatlists \trace_show_float_list:
%    \end{macrocode}
% \end{macro}
%
% \subsection{Declare modules}
%
% Eventually, this should be done in each individual module.
%    \begin{macrocode}
\trace_declare_module:n { regex }
%    \end{macrocode}
%
%    \begin{macrocode}
%</initex|package>
%    \end{macrocode}
%
% \end{implementation}
%
% \PrintIndex