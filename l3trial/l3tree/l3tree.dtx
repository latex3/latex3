% \iffalse
%
%% File l3tree.dtx (C) Copyright 2012 The LaTeX3 Project
%%
%% It may be distributed and/or modified under the conditions of the
%% LaTeX Project Public License (LPPL), either version 1.3c of this
%% license or (at your option) any later version.  The latest version
%% of this license is in the file
%%
%%    http://www.latex-project.org/lppl.txt
%%
%% This file is part of the "l3trial bundle" (The Work in LPPL)
%% and all files in that bundle must be distributed together.
%%
%% The released version of this bundle is available from CTAN.
%%
%% -----------------------------------------------------------------------
%%
%% The development version of the bundle can be found at
%%
%%    http://www.latex-project.org/svnroot/experimental/trunk/
%%
%% for those people who are interested.
%%
%%%%%%%%%%%
%% NOTE: %%
%%%%%%%%%%%
%%
%%   Snapshots taken from the repository represent work in progress and may
%%   not work or may contain conflicting material!  We therefore ask
%%   people _not_ to put them into distributions, archives, etc. without
%%   prior consultation with the LaTeX Project Team.
%%
%% -----------------------------------------------------------------------
%%
%
%<*driver|package>
\RequirePackage{expl3}
\GetIdInfo$Id: l3tree.dtx 3177 2012-01-11 02:11:43Z bruno $
  {L3 Experimental tree manipulations}
%</driver|package>
%<*driver>
\documentclass[full]{l3doc}
\usepackage{amsmath}
\begin{document}
  \DocInput{\jobname.dtx}
\end{document}
%</driver>
% \fi
%
% \title{^^A
%   The \pkg{l3tree} package\\ Trees^^A
%   \thanks{This file describes v\ExplFileVersion,
%      last revised \ExplFileDate.}^^A
% }
%
% \author{^^A
%  The \LaTeX3 Project\thanks
%    {^^A
%      E-mail:
%        \href{mailto:latex-team@latex-project.org}
%          {latex-team@latex-project.org}^^A
%    }^^A
% }
%
% \date{Released \ExplFileDate}
%
% \maketitle
%
% \begin{documentation}
%
% \section{\pkg{l3tree} documentation}
%
% \LaTeX3 comes with a tree data-type, whose keys are integers.
% ^^A Currently, we only support binary trees, but that will probably
% ^^A be extended as the need arises.
%
% A tree is a box containing
%
% \begin{function}{\tree_new:N, \tree_new:c}
%   \begin{syntax}
%     \cs{tree_new:N} \meta{tree}
%   \end{syntax}
%   Creates a new tree, with no node.
% \end{function}
%
% \begin{function}{\tree_put:Nn, \tree_put:cn, \tree_gput:Nn, \tree_gput:cn}
%   \begin{syntax}
%     \cs{tree_put:Nn} \meta{tree} \Arg{integer expression}
%   \end{syntax}
%   Evaluates the \meta{integer expression}, and puts the
% \end{function}
%
% \end{documentation}
%
% \begin{implementation}
%
% \section{\pkg{l3tree} implementation}
%
%    \begin{macrocode}
%<*initex|package>
%    \end{macrocode}
%
%    \begin{macrocode}
%<*package>
\ProvidesExplPackage
  {\ExplFileName}{\ExplFileDate}{\ExplFileVersion}{\ExplFileDescription}
%</package>
%    \end{macrocode}
%
% \subsection{Variables}
%
% \begin{variable}{\l_tree_index_int}
%
%    \begin{macrocode}
\int_new:N \l_tree_index_int
%    \end{macrocode}
% \end{variable}
%
% \begin{variable}{\l_tree_item_box}
%
%    \begin{macrocode}
\box_new:N \l_tree_item_box
%    \end{macrocode}
% \end{variable}
%
% \begin{variable}{\l_tree_left_box,\l_tree_right_box}
%   During the splaying step, these two boxes hold the left and right
%   children of the node which we splay.
%    \begin{macrocode}
\box_new:N \l_tree_left_box
\box_new:N \l_tree_right_box
%    \end{macrocode}
% \end{variable}
%
% \begin{variable}{\l_tree_first_box, \l_tree_second_box}
%   Temporary variables holding subtrees of the full tree
%   in various circumstances.
%    \begin{macrocode}
\box_new:N \l_tree_first_box
\box_new:N \l_tree_second_box
%    \end{macrocode}
% \end{variable}
%
% \begin{variable}{\l_tree_parent_box, \l_tree_grandpa_box}
%
%    \begin{macrocode}
\box_new:N \l_tree_parent_box
\box_new:N \l_tree_grandpa_box
%    \end{macrocode}
% \end{variable}
%
% \begin{variable}{\c_one_sp_dim}
%   Smallest dimension that \TeX{} recognizes. When implicitly
%   converting from integers to dimensions, this is the conversion
%   factor.
%    \begin{macrocode}
\dim_new:N \c_one_sp_dim
\dim_set:Nn \c_one_sp_dim { 1 sp }
%    \end{macrocode}
% \end{variable}
%
% \begin{variable}{\l_tree_parent_dim,\l_tree_grandpa_dim}
%
%    \begin{macrocode}
\dim_new:N \l_tree_parent_dim
\dim_new:N \l_tree_grandpa_dim
%    \end{macrocode}
% \end{variable}
%
% \begin{variable}{\g_tree_result_seq}
%   When extracting keys from a tree, the result is temporarily stored
%   globally in \cs{g_tree_result_seq} to exit a group.
%    \begin{macrocode}
\seq_new:N \g_tree_result_seq
%    \end{macrocode}
% \end{variable}
%
% \begin{variable}{\l_tree_internal_box}
%   Scratch box register for temporary use.
%    \begin{macrocode}
\box_new:N \l_tree_internal_box
%    \end{macrocode}
% \end{variable}
%
% \subsection{Generic helpers}
%
% \begin{macro}{\tree_end:}
%   Used to end box assignments started by \cs{tree_edit:Nw},
%   \cs{tree_gedit:Nw}, or \cs{tree_cat:Nw}.
%    \begin{macrocode}
\cs_new_eq:NN \tree_end: \c_group_end_token
%    \end{macrocode}
% \end{macro}
%
%^^A explain that we need protection against a potential \tn{everyhbox}.
%
% \begin{macro}{\tree_edit:Nw, \tree_gedit:Nw}
%   Accessing the contents of a tree requires unpacking its contents,
%   to access them using the \tn{lastbox} primitive. We may not do
%   that in the current list. Instead, this is done within a box
%   assignment to the tree itself: this choice allows us to modify
%   (\enquote{edit}) the tree and not only access the nodes.
%   This editing should be ended by \cs{tree_end:}.
%    \begin{macrocode}
\cs_new_protected:Npn \tree_edit:Nw #1
  {
    \tex_afterassignment:D \use_none_delimit_by_q_stop:w
    \tex_setbox:D #1 \tex_hbox:D \c_group_begin_token
      \q_stop
      \tex_everyhbox:D { }
      \hbox_unpack_clear:N #1
  }
\cs_new_protected:Npn \tree_gedit:Nw #1
  {
    \tex_afterassignment:D \use_none_delimit_by_q_stop:w
    \tex_global:D \tex_setbox:D #1 \tex_hbox:D \c_group_begin_token
      \q_stop
      \tex_everyhbox:D { }
      \hbox_unpack_clear:N #1
  }
%    \end{macrocode}
% \end{macro}
%
% \begin{macro}{\tree_cat:Nw}
%   Similar to \cs{tree_edit:Nw}, but without altering the tree contents.
%    \begin{macrocode}
\cs_new_protected:Npn \tree_cat:Nw #1
  {
    \tex_afterassignment:D \use_none_delimit_by_q_stop:w
    \tex_setbox:D \l_tree_internal_box \tex_hbox:D \c_group_begin_token
      \q_stop
      \tex_everyhbox:D { }
      \hbox_unpack:N #1
  }
%    \end{macrocode}
% \end{macro}
%
% \subsection{Empty trees}
%
% \begin{macro}{\tree_new:N, \tree_new:c}
% \begin{macro}{\tree_clear:N, \tree_clear:c}
% \begin{macro}{\tree_gclear:N, \tree_gclear:c}
%   Trees are simply boxes, so we copy the box functions.
%    \begin{macrocode}
\cs_new_eq:NN \tree_new:N    \box_new:N
\cs_new_eq:NN \tree_new:c    \box_new:c
\cs_new_eq:NN \tree_clear:N  \box_clear:N
\cs_new_eq:NN \tree_clear:c  \box_clear:c
\cs_new_eq:NN \tree_gclear:N \box_gclear:N
\cs_new_eq:NN \tree_gclear:c \box_gclear:c
%    \end{macrocode}
% \end{macro}
% \end{macro}
% \end{macro}
%
% \begin{macro}[pTF, EXP]{\tree_if_empty:N}
%   Test whether a tree is empty, \emph{i.e.}, contains no node at all.
%    \begin{macrocode}
\prg_new_eq_conditional:NNn \tree_if_empty:N
  \box_if_empty:N { p , T , F , TF }
%    \end{macrocode}
% \end{macro}
%
% \begin{macro}[EXP]{\if_tree_empty:N}
%   The primitive conditional is used in the code because it expands
%   in only one step.
%    \begin{macrocode}
\cs_new_eq:NN \if_tree_empty:N \if_box_empty:N
%    \end{macrocode}
% \end{macro}
%
% \subsection{Extracting the keys}
%
% \begin{macro}{\seq_set_from_tree:NN, \seq_gset_from_tree:NN}
% \begin{macro}[aux]{\tree_extract_aux:NNN}
%   The local and global versions only differ by how the sequence
%   is assigned once it has been built in \cs{g_tree_result_seq}.
%   We open the tree with \cs{tree_cat:Nw}, then extract the keys
%   non-expandably into \tn{dimen} registers (used as integers)
%   using \cs{tree_extract_loop:}, and store the values of those
%   \tn{dimen} registers expandably as the sequence
%   \cs{g_tree_result_seq}. After closing the tree, store the
%   result sequence into the user's variable, and clear the
%   internal sequence (a small cost operation to avoid memory leaks).
%    \begin{macrocode}
\cs_new_protected_nopar:Npn \seq_set_from_tree:NN
  { \tree_extract_aux:NNN \seq_set_eq:NN }
\cs_new_protected_nopar:Npn \seq_gset_from_tree:NN
  { \tree_extract_aux:NNN \seq_gset_eq:NN }
\cs_new_protected:Npn \tree_extract_aux:NNN #1#2#3
  {
    \int_zero:N \l_tree_index_int
    \tree_cat:Nw #3
      \tree_extract_loop:
      \tl_gset:Nx \g_tree_result_seq
        {
          \exp_after:wN \tree_extract_loop:w
            \int_use:N \l_tree_index_int ;
          \prg_break_point:n { }
        }
    \tree_end:
    #1 #2 \g_tree_result_seq
    \tl_gclear:N \g_tree_result_seq
  }
%    \end{macrocode}
% \end{macro}
% \end{macro}
%
% \begin{macro}[aux]{\tree_extract_loop:}
%
%    \begin{macrocode}
\cs_new_protected:Npn \tree_extract_loop:
  {
    \tex_setbox:D \l_tree_first_box \tex_lastbox:D
    \if_hbox:N \l_tree_first_box
      \hbox_unpack_clear:N \l_tree_first_box
    \else:
      \if_int_compare:w \tex_ht:D \l_tree_first_box = \c_zero
        \exp_after:wN \exp_after:wN \exp_after:wN \use_none:n
      \else:
        \tex_advance:D \l_tree_index_int \c_one
        \tex_dimen:D \l_tree_index_int \tex_ht:D \l_tree_first_box
      \fi:
    \fi:
    \tree_extract_loop:
  }
%    \end{macrocode}
% \end{macro}
%
% \begin{macro}[aux, EXP]{\tree_extract_loop:w}
%
%    \begin{macrocode}
\cs_new:Npn \tree_extract_loop:w #1 ;
  {
    \if_int_compare:w #1 = \c_zero
      \exp_after:wN \prg_map_break:
    \fi:
    \exp_not:N \seq_item:n { \int_value:w \tex_dimen:D #1 }
    \exp_after:wN \tree_extract_loop:w
      \int_use:N \int_eval:w #1 - \c_one ;
  }
%    \end{macrocode}
% \end{macro}
%
% \subsection{Comparison}
%
% \begin{macro}{\tree_compare:NN}
%
%    \begin{macrocode}
\cs_new_protected:Npn \tree_compare:NN #1 #2
  {
    \if_int_compare:w \tex_ht:D #1 > \tex_ht:D #2 \exp_stop_f:
      \tree_reversed:
    \else:
      \tree_ordered:
    \fi:
  }
%    \end{macrocode}
% \end{macro}
%
% \subsection{Adding nodes to a tree}
%
% \begin{macro}{\tree_put:Nnn, \tree_gput:Nnn}
% \begin{macro}[aux]{\tree_put_aux:NNnn}
%
%    \begin{macrocode}
\cs_new_protected:Npn \tree_put:Nnn
  { \tree_put_aux:NNnn \tree_edit:Nw }
\cs_new_protected:Npn \tree_gput:Nnn
  { \tree_put_aux:NNnn \tree_gedit:Nw }
\cs_new_protected:Npn \tree_put_aux:NNnn #1#2#3#4
  {
    \tex_setbox:D \l_tree_left_box  \tex_box:D \c_empty_box
    \tex_setbox:D \l_tree_right_box \tex_box:D \c_empty_box
    \tex_setbox:D \l_tree_item_box
      \tex_vbox:D to \int_eval:w #3 \int_eval_end: \c_one_sp_dim {#4}
    \exp_after:wN #1 \exp_after:wN #2
      \if_tree_empty:N #2
      \else:
        \tree_put_loop:
        \tree_splay_loop:
      \fi:
      \box_use_clear:N \l_tree_left_box
      \box_use_clear:N \l_tree_item_box
      \box_use_clear:N \l_tree_right_box
    \tree_end:
  }
%    \end{macrocode}
% \end{macro}
% \end{macro}
%
% \begin{macro}[aux]{\tree_put_loop:}
%
%    \begin{macrocode}
\cs_new_protected:Npn \tree_put_loop:
  {
    \tex_setbox:D \l_tree_first_box \tex_lastbox:D
    \tex_setbox:D \l_tree_parent_box
      \if_vbox:N \l_tree_first_box
        \tex_box:D \l_tree_first_box
      \else:
        \tex_lastbox:D
      \fi:
    \tree_compare:NN \l_tree_parent_box \l_tree_item_box
    \if_box_empty:N \l_tree_first_box
      \exp_after:wN \use_none:n
    \else:
      \hbox_unpack_clear:N \l_tree_first_box
    \fi:
    \tree_put_loop:
  }
%    \end{macrocode}
% \end{macro}
%
% \begin{macro}{\tree_ordered:, \tree_reversed:}
%
%    \begin{macrocode}
\cs_new_protected:Npn \tree_ordered:
  {
    \box_use_clear:N \l_tree_parent_box
    \tex_kern:D \c_one_sp_dim
  }
\cs_new_protected:Npn \tree_reversed:
  {
    \tex_setbox:D \l_tree_second_box \tex_lastbox:D
    \box_use_clear:N \l_tree_first_box
    \tex_setbox:D \l_tree_first_box \tex_box:D \l_tree_second_box
    \box_use_clear:N \l_tree_parent_box
    \tex_kern:D - \c_one_sp_dim
  }
%    \end{macrocode}
% \end{macro}
%
% \subsection{Splaying}
%
% \begin{macro}[int]{\tree_splay_loop:}
%
%    \begin{macrocode}
\cs_new_protected:Npn \tree_splay_loop:
  {
    \l_tree_parent_dim \tex_lastkern:D
    \tex_unkern:D
    \tex_setbox:D \l_tree_parent_box \tex_lastbox:D
    \tex_setbox:D \l_tree_first_box  \tex_lastbox:D
    \l_tree_grandpa_dim \tex_lastkern:D
    \tex_unkern:D
    \tex_setbox:D \l_tree_grandpa_box \tex_lastbox:D
    \tex_setbox:D \l_tree_second_box  \tex_lastbox:D
    \cs:w
      tree_splay
      _\int_value:w \l_tree_parent_dim
      _\int_value:w \l_tree_grandpa_dim
      :
    \cs_end:
  }
%    \end{macrocode}
% \end{macro}
%
% \begin{macro}[aux]{\tree_splay_0_0:}
%
%    \begin{macrocode}
\cs_new_protected:cpn { tree_splay_0_0: } { }
%    \end{macrocode}
% \end{macro}
%
% \begin{macro}[aux]{\tree_splay_1_0:}
%
%    \begin{macrocode}
\cs_new_protected:cpn { tree_splay_1_0: }
  {
    \tex_setbox:D \l_tree_left_box \tex_hbox:D
      {
        \box_use_clear:N \l_tree_first_box
        \box_use_clear:N \l_tree_parent_box
        \box_use_clear:N \l_tree_left_box
      }
  }
%    \end{macrocode}
% \end{macro}
%
% \begin{macro}[aux]{\tree_splay_-1_0:}
%
%    \begin{macrocode}
\cs_new_protected:cpn { tree_splay_-1_0: }
  {
    \tex_setbox:D \l_tree_right_box \tex_hbox:D
      {
        \box_use_clear:N \l_tree_right_box
        \box_use_clear:N \l_tree_parent_box
        \box_use_clear:N \l_tree_first_box
      }
  }
%    \end{macrocode}
% \end{macro}
%
% \begin{macro}[aux]{\tree_splay_1_1:}
%
%    \begin{macrocode}
\cs_new_protected:cpn { tree_splay_1_1: }
  {
    \tex_setbox:D \l_tree_left_box \tex_hbox:D
      {
        \tex_hbox:D
          {
            \box_use_clear:N \l_tree_second_box
            \box_use_clear:N \l_tree_grandpa_box
            \box_use_clear:N \l_tree_first_box
          }
        \box_use_clear:N \l_tree_parent_box
        \box_use_clear:N \l_tree_left_box
      }
    \tree_splay_loop:
  }
%    \end{macrocode}
% \end{macro}
%
% \begin{macro}[aux]{\tree_splay_1_-1:}
%
%    \begin{macrocode}
\cs_new_protected:cpn { tree_splay_1_-1: }
  {
    \tex_setbox:D \l_tree_right_box \tex_hbox:D
      {
        \box_use_clear:N \l_tree_right_box
        \box_use_clear:N \l_tree_grandpa_box
        \box_use_clear:N \l_tree_second_box
      }
    \tex_setbox:D \l_tree_left_box \tex_hbox:D
      {
        \box_use_clear:N \l_tree_first_box
        \box_use_clear:N \l_tree_parent_box
        \box_use_clear:N \l_tree_left_box
      }
    \tree_splay_loop:
  }
%    \end{macrocode}
% \end{macro}
%
% \begin{macro}[aux]{\tree_splay_-1_1:}
%
%    \begin{macrocode}
\cs_new_protected:cpn { tree_splay_-1_1: }
  {
    \tex_setbox:D \l_tree_right_box \tex_hbox:D
      {
        \box_use_clear:N \l_tree_right_box
        \box_use_clear:N \l_tree_parent_box
        \box_use_clear:N \l_tree_first_box
      }
    \tex_setbox:D \l_tree_left_box \tex_hbox:D
      {
        \box_use_clear:N \l_tree_second_box
        \box_use_clear:N \l_tree_grandpa_box
        \box_use_clear:N \l_tree_left_box
      }
    \tree_splay_loop:
  }
%    \end{macrocode}
% \end{macro}
%
% \begin{macro}[aux]{\tree_splay_-1_-1:}
%
%    \begin{macrocode}
\cs_new_protected:cpn { tree_splay_-1_-1: }
  {
    \tex_setbox:D \l_tree_right_box \tex_hbox:D
      {
        \box_use_clear:N \l_tree_right_box
        \box_use_clear:N \l_tree_parent_box
        \tex_hbox:D
          {
            \box_use_clear:N \l_tree_first_box
            \box_use_clear:N \l_tree_grandpa_box
            \box_use_clear:N \l_tree_second_box
          }
      }
    \tree_splay_loop:
  }
%    \end{macrocode}
% \end{macro}
%
% \subsection{Sorting}
%
%    \begin{macrocode}
%    \end{macrocode}
%
%    \begin{macrocode}
%</initex|package>
%    \end{macrocode}
%
% \end{implementation}
%
% \PrintIndex